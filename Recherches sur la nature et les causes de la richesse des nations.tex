%%%%%%%%%%%%%%%%%%%%%%%%%%%%%%%%%%%%%%%%%%%%%%%%%%%%%%%%%%%%%%%%%%%%%%%%%%%%%%%%
%                                                                              %
%                                                                              %
%      Recherches sur la nature et les causes de la richesse des nations       %
%                                                                              %
%                                                                              %
%                             écrit par Adam Smith                             %
%                traduit par Germain Garnier et Adolphe Blanqui                %
%                                                                              %
%                           transcrit par Wikisource                           %
%                      mis-en-page par Louis-Kenzo Cahier                      %
%                                                                              %
%                                                                              %
%%%%%%%%%%%%%%%%%%%%%%%%%%%%%%%%%%%%%%%%%%%%%%%%%%%%%%%%%%%%%%%%%%%%%%%%%%%%%%%%

\documentclass{book}

% Paquets
\usepackage{fontspec}    % Contrôle avancé des polices de caractère
\usepackage{polyglossia} % Typographie française
\usepackage{parallel}    % Mise-en-page parallèle de l'original et la traduction
\usepackage{fancyhdr}    % Personnalisation des en-tête et pieds-de-page
\usepackage[explicit]{titlesec} % Personnalisation des titres
\usepackage[usenames,dvipsnames,svgnames]{xcolor} % Couleurs arbitraires
\usepackage{graphicx}    % Opérations géométriques
\usepackage{perpage}     % Notes numérotées par page
\usepackage{xspace}      % Commands without parameters
\usepackage{ifthen}      % Expressions conditionnelles
\usepackage{relsize}     % Taille relative de caractère
\usepackage{enumitem}    % Ajustement du format des listes
\usepackage{setspace}    % Ajustement de l'espace inter-ligne
\usepackage[stable]{footmisc} % Notes dans les titres
\usepackage[paperwidth=148mm,paperheight=210mm,scale=0.666,hmarginratio=1:2,vmarginratio=1:1,marginparsep=5mm,marginparwidth=1cm]{geometry}
%\usepackage{hyperref}    % Table des matières PDF

%%%%%%%%%%%%%%%%%%%%%%%%%%%%%%%%%%%%%%%%%%%%%%%%%%%%%%%%%%%%%%%%%%%%%%%%%%%%%%%%

% 
\setdefaultlanguage{french}
\setotherlanguage{english}



%%%%%%%%%%%%%%%%%%%%%%%%%%%%%%%%%%%%%%%%%%%%%%%%%%%%%%%%%%%%%%%%%%%%%%%%%%%%%%%%

\setmainfont[Numbers=OldStyle,
             Ligatures = {Common,
                          Historical,
                          Rare},
             ItalicFeatures = {Ligatures = {Common,
                                            Historical,
                                            Rare},
                               Contextuals = {WordInitial,
                                              WordFinal,
                                              LineInitial,
                                              LineFinal,
                                              Inner}}]
            {Garamond Premier Pro}

%%%%%%%%%%%%%%%%%%%%%%%%%%%%%%%%%%%%%%%%%%%%%%%%%%%%%%%%%%%%%%%%%%%%%%%%%%%%%%%%

% Rouges
\definecolor{aniline}{RGB}{237,0,0}
\definecolor{rouge}{RGB}{255,0,0}
\definecolor{bordeaux}{RGB}{128,0,0}
\definecolor{coquelicot}{RGB}{198,8,0}
\definecolor{sang}{RGB}{133,6,6}
\definecolor{alizarine}{RGB}{217,1,21}
\definecolor{carmin}{RGB}{150,0,24}
\definecolor{feu}{RGB}{254,28,0}
\definecolor{garance}{RGB}{238,16,16}
\definecolor{bourgogne}{RGB}{144,0,32}
\definecolor{ecrevisse}{RGB}{188,32,1}
\definecolor{ecarlate}{RGB}{255,36,0}
\definecolor{cardinal}{RGB}{184,32,16}
\definecolor{mars}{RGB}{247,35,12}
\definecolor{andrinople}{RGB}{169,17,1}
\definecolor{cinabre}{RGB}{184,32,16}
\definecolor{vermeil}{RGB}{255,9,33}
\definecolor{grenat}{RGB}{110,11,20}
\definecolor{fraise}{RGB}{164,36,36}
\definecolor{groseille}{RGB}{207,10,29}

% Violet
\definecolor{pourpre}{RGB}{158,14,64}
\definecolor{tyr}{RGB}{102,40,60}
\definecolor{lie}{RGB}{172,30,68}
\definecolor{byzantium}{RGB}{112,41,99}
\definecolor{zinzolin}{RGB}{108,2,119}
\definecolor{indigo}{RGB}{47,0,127}
\definecolor{eveque}{RGB}{114,62,100}
\definecolor{glycine}{RGB}{201,160,220}
\definecolor{amethyste}{RGB}{136,77,167}
\definecolor{colombin}{RGB}{106,69,93}
\definecolor{cerise}{RGB}{222,49,99}
\definecolor{prune}{RGB}{129,20,83}
\definecolor{cramoisi}{RGB}{220,20,60}
\definecolor{amaranthe}{RGB}{145,40,59}

% Gris
\definecolor{anthracite}{RGB}{48,48,48}
\definecolor{fer}{RGB}{132,132,132}
\definecolor{souris}{RGB}{158,158,158}
\definecolor{acier}{RGB}{175,175,175}
\definecolor{argent}{RGB}{192,192,192}
\definecolor{perle}{RGB}{206,206,206}
\definecolor{etain}{RGB}{237,237,237}
\definecolor{argile}{RGB}{239,239,239}
\definecolor{nuage}{RGB}{248,248,248}
\definecolor{ardoise}{RGB}{90,94,107}
\definecolor{plomb}{RGB}{121,128,129}

% Jaune
\definecolor{or}{RGB}{255,215,0}
\definecolor{boutondor}{RGB}{252,220,18}
\definecolor{mimosa}{RGB}{254,248,108}
\definecolor{citron}{RGB}{247,255,60}
\definecolor{canari}{RGB}{231,240,13}
\definecolor{ble}{RGB}{232,214,48}
\definecolor{chartreuse}{RGB}{223,255,0}
\definecolor{miel}{RGB}{218,179,10}
\definecolor{moutarde}{RGB}{255,219,88}
\definecolor{ocre}{RGB}{223,175,44}
\definecolor{orpiment}{RGB}{252,210,28}
\definecolor{paille}{RGB}{254,227,71}
\definecolor{ambre}{RGB}{240,195,0}
\definecolor{safran}{RGB}{244,196,48}
\definecolor{topaze}{RGB}{250,234,115}
\definecolor{soufre}{RGB}{255,255,107}
\definecolor{beurre}{RGB}{255,244,141}
\definecolor{blond}{RGB}{226,188,116}
\definecolor{venitien}{RGB}{231,168,84}
\definecolor{poussin}{RGB}{247,227,95}
\definecolor{sable}{RGB}{224,205,169}
\definecolor{champagne}{RGB}{251,242,183}
\definecolor{mais}{RGB}{255,222,117}
\definecolor{naples}{RGB}{250,218,94}
\definecolor{bulle}{RGB}{251,249,233}

%%%%%%%%%%%%%%%%%%%%%%%%%%%%%%%%%%%%%%%%%%%%%%%%%%%%%%%%%%%%%%%%%%%%%%%%%%%%%%%%

% Garamond Premier Pro

\newcommand\eucalyptusdroite{{\fontspec{Garamond Premier Pro} \XeTeXglyph347}\xspace} % Eucalyptus droite
\newcommand\eucalyptusgauche{{\fontspec{Garamond Premier Pro} \XeTeXglyph348}\xspace} % Eucalyptus gauche
\newcommand\eucalyptusbas{{\fontspec{Garamond Premier Pro} \XeTeXglyph349}\xspace}    % Eucalyptus bas
\newcommand\eucalyptushaut{{\fontspec{Garamond Premier Pro} \XeTeXglyph350}\xspace}   % Eucalyptus haut

\newcommand\glandsuspendydroite{{\fontspec{Garamond Premier Pro} \XeTeXglyph351}\xspace} % Gland suspendu droite

\newcommand\trefledroite{{\fontspec{Garamond Premier Pro} \XeTeXglyph353}\xspace} % Trèfle droite

\newcommand\lierredroite{{\fontspec{Garamond Premier Pro} \XeTeXglyph352}\xspace}        % Lierre droite
\newcommand\lierredroitesarment{{\fontspec{Garamond Premier Pro} \XeTeXglyph354}\xspace} % Lierre droite sarment

\newcommand\vignegauche{{\fontspec{Garamond Premier Pro} \XeTeXglyph355}\xspace} % Vigne gauche

\newcommand\hortensiagauche{{\fontspec{Garamond Premier Pro} \XeTeXglyph356}\xspace} % Hortensia gauche
\newcommand\hortensiadroite{{\fontspec{Garamond Premier Pro} \XeTeXglyph357}\xspace} % Hortensia droite

% Hoefler Text

\newcommand\barrehorizontale{{\fontspec{Hoefler Text Ornaments} \XeTeXglyph5}\xspace}         % Barre horizontale
\newcommand\barrehorizontalerenflee{{\fontspec{Hoefler Text Ornaments} \XeTeXglyph6}\xspace}  % Barre horizontale renflée
\newcommand\barrehorizontalecintree{{\fontspec{Hoefler Text Ornaments} \XeTeXglyph11}\xspace} % Barre horizontale cintrée
\newcommand\ferdelancehorizontal{{\fontspec{Hoefler Text Ornaments} \XeTeXglyph12}\xspace}    % Ferre de lance horizontal

\newcommand\croisillonfloral{{\fontspec{Hoefler Text Ornaments} \XeTeXglyph7}\xspace}       % Croisillon floral
\newcommand\croisillonmoustache{{\fontspec{Hoefler Text Ornaments} \XeTeXglyph8}\xspace}    % Croisillon moustache
\newcommand\croisillonlance{{\fontspec{Hoefler Text Ornaments} \XeTeXglyph9}\xspace}        % Croisillon lance
\newcommand\croisillonpetitecroix{{\fontspec{Hoefler Text Ornaments} \XeTeXglyph10}\xspace} % Croisillon petite croix

\newcommand\glanddroiteeclaire{{\fontspec{Hoefler Text Ornaments} \XeTeXglyph29}\xspace} % Gland droite éclairé
\newcommand\glanddroite{{\fontspec{Hoefler Text Ornaments} \XeTeXglyph50}\xspace}        % Gland droite

\newcommand\fleurdelysprofileeclaire{{\fontspec{Hoefler Text Ornaments} \XeTeXglyph30}\xspace} % Fleur-de-lys profil éclairé
\newcommand\fleurdelysprofile{{\fontspec{Hoefler Text Ornaments} \XeTeXglyph51}\xspace}        % Fleur-de-lys profil

\newcommand\glycinedroiteeclairee{{\fontspec{Hoefler Text Ornaments} \XeTeXglyph31}\xspace} % Glycine droite éclairé
\newcommand\glycinedroite{{\fontspec{Hoefler Text Ornaments} \XeTeXglyph52}\xspace}         % Glycine droite

\newcommand\mainbouffongauche{{\fontspec{Hoefler Text Ornaments} \XeTeXglyph32}\xspace} % Main bouffon gauche
\newcommand\mainbouffondroite{{\fontspec{Hoefler Text Ornaments} \XeTeXglyph33}\xspace} % Main bouffon droite

\newcommand\soleil{{\fontspec{Hoefler Text Ornaments} \XeTeXglyph53}\xspace}     % Soleil
\newcommand\lune{{\fontspec{Hoefler Text Ornaments} \XeTeXglyph54}\xspace}       % Lune
\newcommand\couronne{{\fontspec{Hoefler Text Ornaments} \XeTeXglyph55}\xspace}   % Couronne
\newcommand\sablier{{\fontspec{Hoefler Text Ornaments} \XeTeXglyph56}\xspace}    % Sablier
\newcommand\fleurdelys{{\fontspec{Hoefler Text Ornaments} \XeTeXglyph57}\xspace} % Fleur-de-lys
\newcommand\galion{{\fontspec{Hoefler Text Ornaments} \XeTeXglyph58}\xspace}     % Galion
\newcommand\voile{{\fontspec{Hoefler Text Ornaments} \XeTeXglyph59}\xspace}      % Voile
\newcommand\pomme{{\fontspec{Hoefler Text Ornaments} \XeTeXglyph60}\xspace}      % Pomme

\newcommand\voluteeclaireemoustachebasgauche{{\fontspec{Hoefler Text Ornaments} \XeTeXglyph13}\xspace}  % Volute éclairée moustache bas-gauche
\newcommand\voluteeclaireemoustachebasdroite{{\fontspec{Hoefler Text Ornaments} \XeTeXglyph14}\xspace}  % Volute éclairée moustache bas-droite
\newcommand\voluteeclaireemoustachehautgauche{{\fontspec{Hoefler Text Ornaments} \XeTeXglyph15}\xspace} % Volute éclairée moustache haut-gauche
\newcommand\voluteeclaireemoustachehautdroite{{\fontspec{Hoefler Text Ornaments} \XeTeXglyph16}\xspace} % Volute éclairée moustache haut-droite

\newcommand\voluteeclaireecisaillebasgauche{{\fontspec{Hoefler Text Ornaments} \XeTeXglyph17}\xspace}  % Volute éclairée cisaille bas-gauche
\newcommand\voluteeclaireecisaillebasdroite{{\fontspec{Hoefler Text Ornaments} \XeTeXglyph18}\xspace}  % Volute éclairée cisaille bas-droite
\newcommand\voluteeclaireecisaillehautgauche{{\fontspec{Hoefler Text Ornaments} \XeTeXglyph19}\xspace} % Volute éclairée cisaille haut-gauche
\newcommand\voluteeclaireecisaillehautdroite{{\fontspec{Hoefler Text Ornaments} \XeTeXglyph20}\xspace} % Volute éclairée cisaille haut-droite

\newcommand\voluteeclaireerondebasgauche{{\fontspec{Hoefler Text Ornaments} \XeTeXglyph21}\xspace}  % Volute éclairée ronde bas-gauche
\newcommand\voluteeclaireerondebasdroite{{\fontspec{Hoefler Text Ornaments} \XeTeXglyph22}\xspace}  % Volute éclairée ronde bas-droite
\newcommand\voluteeclaireerondehautgauche{{\fontspec{Hoefler Text Ornaments} \XeTeXglyph23}\xspace} % Volute éclairée ronde haut-gauche
\newcommand\voluteeclaireerondehautdroite{{\fontspec{Hoefler Text Ornaments} \XeTeXglyph24}\xspace} % Volute éclairée ronde haut-droite

\newcommand\voluteeclaireecentreebasgauche{{\fontspec{Hoefler Text Ornaments} \XeTeXglyph25}\xspace}  % Volute éclairée centrée bas-gauche
\newcommand\voluteeclaireecentreebasdroite{{\fontspec{Hoefler Text Ornaments} \XeTeXglyph26}\xspace}  % Volute éclairée centrée bas-droite
\newcommand\voluteeclaireecentreehautgauche{{\fontspec{Hoefler Text Ornaments} \XeTeXglyph27}\xspace} % Volute éclairée centrée haut-gauche
\newcommand\voluteeclaireecentreehautdroite{{\fontspec{Hoefler Text Ornaments} \XeTeXglyph28}\xspace} % Volute éclairée centrée haut-droite

\newcommand\volutemoustachebasgauche{{\fontspec{Hoefler Text Ornaments} \XeTeXglyph34}\xspace}  % Volute moustache bas-gauche
\newcommand\volutemoustachebasdroite{{\fontspec{Hoefler Text Ornaments} \XeTeXglyph35}\xspace}  % Volute moustache bas-droite
\newcommand\volutemoustachehautgauche{{\fontspec{Hoefler Text Ornaments} \XeTeXglyph36}\xspace} % Volute moustache haut-gauche
\newcommand\volutemoustachehautdroite{{\fontspec{Hoefler Text Ornaments} \XeTeXglyph37}\xspace} % Volute moustache haut-droite

\newcommand\volutecisaillebasgauche{{\fontspec{Hoefler Text Ornaments} \XeTeXglyph38}\xspace}  % Volute cisaille bas-gauche
\newcommand\volutecisaillebasdroite{{\fontspec{Hoefler Text Ornaments} \XeTeXglyph39}\xspace}  % Volute cisaille bas-droite
\newcommand\volutecisaillehautgauche{{\fontspec{Hoefler Text Ornaments} \XeTeXglyph40}\xspace} % Volute cisaille haut-gauche
\newcommand\volutecisaillehautdroite{{\fontspec{Hoefler Text Ornaments} \XeTeXglyph41}\xspace} % Volute cisaille haut-droite

\newcommand\voluterondebasgauche{{\fontspec{Hoefler Text Ornaments} \XeTeXglyph42}\xspace}  % Volute ronde bas-gauche
\newcommand\voluterondebasdroite{{\fontspec{Hoefler Text Ornaments} \XeTeXglyph43}\xspace}  % Volute ronde bas-droite
\newcommand\voluterondehautgauche{{\fontspec{Hoefler Text Ornaments} \XeTeXglyph44}\xspace} % Volute ronde haut-gauche
\newcommand\voluterondehautdroite{{\fontspec{Hoefler Text Ornaments} \XeTeXglyph45}\xspace} % Volute ronde haut-droite

\newcommand\volutecentreebasgauche{{\fontspec{Hoefler Text Ornaments} \XeTeXglyph46}\xspace}  % Volute centrée bas-gauche
\newcommand\volutecentreebasdroite{{\fontspec{Hoefler Text Ornaments} \XeTeXglyph47}\xspace}  % Volute centrée bas-droite
\newcommand\volutecentreehautgauche{{\fontspec{Hoefler Text Ornaments} \XeTeXglyph48}\xspace} % Volute centrée haut-gauche
\newcommand\volutecentreehautdroite{{\fontspec{Hoefler Text Ornaments} \XeTeXglyph49}\xspace} % Volute centrée haut-droite

%%%%%%%%%%%%%%%%%%%%%%%%%%%%%%%%%%%%%%%%%%%%%%%%%%%%%%%%%%%%%%%%%%%%%%%%%%%%%%%%

\renewcommand{\maketitle}{%
	\thispagestyle{empty}%

	\null % Needed for vfill to work; why?
	
	\begin{center}
		\vskip 15mm

		{
			\addfontfeatures{Contextuals = Swash}
			\itshape
			\LARGE
			Adam Smith
		}

		\vfill

		{
			\addfontfeatures{Kerning = Uppercase,
			                 Style   = Display}

			{
				\color{coquelicot}
				\fontsize{13pt}{13pt}\selectfont
				\addfontfeatures{Letters = SmallCaps}
				{Recherches sur la nature et les causes de la}
			}

			\vskip 4mm

			{
				\color{coquelicot}
				\fontsize{40pt}{40pt}\selectfont
				\addfontfeatures{WordSpace = 1.5,
				                 Letters   = SmallCaps}
				richesse 
				{\fontsize{28.28pt}{40pt}\selectfont des}
				nations
			}
		}

		\vfill
		{
			\fontsize{60pt}{60pt}\selectfont
			\color{argile}
			\soleil
		}
		\vfill

		Traduction de \GG\\
		révisée par \AB

		\vskip 30mm
	\end{center}

	\cleardoublepage
}

%%%%%%%%%%%%%%%%%%%%%%%%%%%%%%%%%%%%%%%%%%%%%%%%%%%%%%%%%%%%%%%%%%%%%%%%%%%%%%%%

\pagestyle{fancy}
\fancyhead{}\fancyfoot{} % Clear header and footer

\renewcommand{\chaptermark}[1]{\markboth{#1}{}}

\fancyheadoffset{1em}
\renewcommand{\headrulewidth}{0.0pt}

\fancyhead[LE,RO]{\addfontfeatures{Numbers=OldStyle} \arabic{page}}
\fancyhead[CE]{\itshape 
               \addfontfeatures{Contextuals = Swash} 
               La Richesse des Nations}
\fancyhead[CO]{\itshape 
               \addfontfeatures{Contextuals = Swash}
               \leftmark}

%%%%%%%%%%%%%%%%%%%%%%%%%%%%%%%%%%%%%%%%%%%%%%%%%%%%%%%%%%%%%%%%%%%%%%%%%%%%%%%%

\makeatletter

\def\cleardoublepage{%
	\clearpage

	\if@twoside
		\ifodd\c@page
			% Verso
			
		\else
			% Recto

			\newgeometry{margin=0cm}

				\thispagestyle{empty}%

				\null
				\begin{center}
					\vfill
					{
						\fontsize{60pt}{60pt}\selectfont%
						\color{bulle}%
						\rotatebox[origin=c]{-90}{\glanddroite}%
					}
					\vfill
				\end{center}

				\newpage

				\restoregeometry
		\fi
	\fi
}

\makeatother

%%%%%%%%%%%%%%%%%%%%%%%%%%%%%%%%%%%%%%%%%%%%%%%%%%%%%%%%%%%%%%%%%%%%%%%%%%%%%%%%

\titleformat{\part}   %
            [display] %
            {} % Part formatting
            {\centering 
             \itshape
             \fontsize{50pt}{0pt}\selectfont
             \partname} % Part label
            {0mm} % Before separator
            {\vfill {\begin{center} \fontsize{12pt}{0pt}\selectfont \color{argile} \barrehorizontale\barrehorizontale\barrehorizontalecintree\voluteeclaireecentreehautgauche\voluteeclaireecentreehautdroite\barrehorizontalecintree\barrehorizontale\barrehorizontale\end{center}}
             \vfill 
			 \centering 
			 \huge
			 \addfontfeature{Letters = SmallCaps, 
                             Weight  = 0.5,
                             Kerning = Uppercase,
			                 Style   = Display} 
			 \thepart #1} % Part title
            [\vfill\vfill\vfill
             \thispagestyle{empty}
             ] % After

\titleformat{\chapter} %
            [display]  %
            {} % Chapter formatting
            {} % Chapter label
            {0mm} % Before separator
            {\centering
             \fontsize{40pt}{10pt}
             \addfontfeature{Letters = SmallCaps,
			                 Kerning = Uppercase}
             #1} % Chapter title
            [] % After

\titleformat{\section} %
            [display]  %
            {\large\addfontfeature{Letters = SmallCaps,
                                   Kerning = Uppercase,
                                   LetterSpace = 2.0}} % Chapter formatting
            {} % Chapter label
            {0mm} % Before separator
            {\centering #1} % Chapter title
            [] % After

%%%%%%%%%%%%%%%%%%%%%%%%%%%%%%%%%%%%%%%%%%%%%%%%%%%%%%%%%%%%%%%%%%%%%%%%%%%%%%%%

\setlist[enumerate]{noitemsep,%
                    leftmargin=*,%
                    itemsep=0.0\baselineskip}
\setlist[itemize]{noitemsep,%
                 leftmargin=*,%
                 itemsep=0.0\baselineskip}

%%%%%%%%%%%%%%%%%%%%%%%%%%%%%%%%%%%%%%%%%%%%%%%%%%%%%%%%%%%%%%%%%%%%%%%%%%%%%%%%

\newcommand{\nom}[1]{{\addfontfeature{Letters=SmallCaps} #1}}

% Noms anglais
\newcommand{\AS}[0]{Adam \nom{Smith}\xspace}
\newcommand{\Smith}[0]{\nom{Smith}\xspace}
\newcommand{\Buchanan}[0]{\nom{Buchanan}\xspace}
\newcommand{\MacCulloch}[0]{\nom{Mac Culloch}\xspace}
\newcommand{\Malthus}[0]{\nom{Malthus}\xspace}
\newcommand{\Ricardo}[0]{\nom{Ricardo}\xspace}
\newcommand{\DavidRicardo}[0]{David \nom{Ricardo}\xspace}
\newcommand{\JeremyBentham}[0]{Jeremy \nom{Bentham}\xspace}
\newcommand{\Bentham}[0]{\nom{Bentham}\xspace}
\newcommand{\Fletcher}[0]{\nom{Fletcher}\xspace}
\newcommand{\Hume}[0]{\nom{Hume}\xspace}
\newcommand{\DavidHume}[0]{David \nom{Hume}\xspace}
\newcommand{\Blair}[0]{\nom{Blair}\xspace}
\newcommand{\Huskisson}[0]{\nom{Huskisson}\xspace}
\newcommand{\Hutcheson}[0]{\nom{Hutcheson}\xspace}
\newcommand{\CharlesTownsend}[0]{Charles \nom{Townsend}\xspace}
\newcommand{\Oswald}[0]{\nom{Oswald}\xspace}
\newcommand{\DugaldStewart}[0]{Dugald \nom{Stewart}\xspace}
\newcommand{\Senior}[0]{\nom{Senior}\xspace}
\newcommand{\Pelham}[0]{\nom{Pelham}\xspace}
\newcommand{\Price}[0]{\nom{Price}\xspace}
\newcommand{\Law}[0]{\nom{Law}\xspace}
\newcommand{\RobertBruce}[0]{Robert \nom{Bruce}\xspace}
\newcommand{\Hobbes}[0]{\nom{Hobbes}\xspace}
\newcommand{\Bazinghen}[0]{\nom{Bazinghen}\xspace}
\newcommand{\Blackstone}[0]{\nom{Blackstone}\xspace}
\newcommand{\Drummond}[0]{\nom{Drummond}\xspace}
\newcommand{\Locke}[0]{\nom{Locke}\xspace}

% Noms français
\newcommand{\AB}[0]{Alphonse \nom{Blanqui}\xspace}
\newcommand{\GG}[0]{Germain \nom{Garnier}\xspace}
\newcommand{\JBS}[0]{Jean-Baptiste \nom{Say}\xspace}
\newcommand{\HoraceSay}[0]{Horace \nom{Say}\xspace}
\newcommand{\Say}[0]{\nom{Say}\xspace}
\newcommand{\Garnier}[0]{\nom{Garnier}\xspace}
\newcommand{\Sismondi}[0]{\nom{Sismondi}\xspace}
\newcommand{\EB}[0]{Eugène \nom{Buret}\xspace}
\newcommand{\Buret}[0]{\nom{Buret}\xspace}
\newcommand{\Turgot}[0]{\nom{Turgot}\xspace}
\newcommand{\Quesnay}[0]{\nom{Quesnay}\xspace}
\newcommand{\Descartes}[0]{\nom{Descartes}\xspace}
\newcommand{\Voltaire}[0]{\nom{Voltaire}\xspace}
\newcommand{\Prevost}[0]{\nom{Prévost}\xspace}
\newcommand{\Cousin}[0]{\nom{Cousin}\xspace}
\newcommand{\Colbert}[0]{\nom{Colbert}\xspace}
\newcommand{\Trudaine}[0]{\nom{Trudaine}\xspace}
\newcommand{\Gournay}[0]{\nom{Gournay}\xspace}
\newcommand{\Malesherbes}[0]{\nom{Malesherbes}\xspace}
\newcommand{\Lavoisier}[0]{\nom{Lavoisier}\xspace}
\newcommand{\Jaucourt}[0]{\nom{Jaucourt}\xspace}
\newcommand{\Raynal}[0]{\nom{Raynal}\xspace}
\newcommand{\Dupont}[0]{\nom{Dupont}\xspace}
\newcommand{\Morellet}[0]{\nom{Morellet}\xspace}
\newcommand{\Letrosne}[0]{\nom{Letrosne}\xspace}
\newcommand{\Leblanc}[0]{\nom{Leblanc}\xspace}
\newcommand{\Lemontey}[0]{\nom{Lemontey}\xspace}
\newcommand{\Jacob}[0]{\nom{Jacob}\xspace}
\newcommand{\Storch}[0]{\nom{Storch}\xspace}
\newcommand{\Necker}[0]{\nom{Necker}\xspace}

% Noms romains francisés
\newcommand{\TiteLive}[0]{Tite-\nom{Live}\xspace}
\newcommand{\ValeriusLaevinus}[0]{\latin{Valerius \nom{Lævinus}}\xspace}
\newcommand{\PaulEmile}[0]{\nom{Paul-Émile}\xspace}

% Noms grecs francisés
\newcommand{\Xenophon}[0]{\nom{Xénophon}\xspace}
\newcommand{\Demosthenes}[0]{\nom{Démosthènes}\xspace}
\newcommand{\Strabon}[0]{\nom{Strabon}\xspace}
\newcommand{\Aristote}[0]{\nom{Aristote}\xspace}
\newcommand{\Phocion}[0]{\nom{Phocion}\xspace}
\newcommand{\Phormion}[0]{\nom{Phormion}\xspace}
\newcommand{\Chrysippe}[0]{\nom{Chrysippe}\xspace}

%%%%%%%%%%%%%%%%%%%%%%%%%%%%%%%%%%%%%%%%%%%%%%%%%%%%%%%%%%%%%%%%%%%%%%%%%%%%%%%%

\MakePerPage{footnote}
\renewcommand\footnoterule{}
\makeatletter
\renewcommand\@makefnmark{\hbox{\@textsuperscript{\normalfont\color{sang}\@thefnmark}}}
\renewcommand\@makefntext[1]{%
  \parindent 1em\noindent
            \hb@xt@1.8em{%
                \hss\normalfont\color{sang}\@thefnmark . \hskip 0.25em}#1}
\makeatother

\newcommand{\notedebasdepage}[2][]{\footnote{#2\ifthenelse{\equal{#1}{}}{}{ — #1}}}
\newcommand{\notedebasdepagesansmarque}[2]{\footnotetext{#2\ifthenelse{\equal{#1}{}}{}{ — #1}}}

%%%%%%%%%%%%%%%%%%%%%%%%%%%%%%%%%%%%%%%%%%%%%%%%%%%%%%%%%%%%%%%%%%%%%%%%%%%%%%%%

% Bricolage des noms de polyglossia pour que les parties soient des livres
\makeatletter
\def\captionsfrench{%
	\def\refname{Références}%
	\def\abstractname{Résumé}%
	\def\bibname{Bibliographie}%
	\def\prefacename{Préface}%
	\def\chaptername{Chapitre}%
	\def\appendixname{Annexe}%
	\def\contentsname{Table des matières}%
	\def\listfigurename{Table des figures}%
	\def\listtablename{Liste des tableaux}%
	\def\indexname{Index}%
	\def\figurename{\textsc{Fig.}}%
	\def\tablename{\textsc{Tab.}}%
	\def\@Fpt{\ifcase\value{part}\or premier\or deuxième\or
	troisième\or quatrième\or cinquième\or sixième\or
	septième\or huitième\or neuvième\or dixième\or onzième\or
	douzième\or treizième\or quatorzième\or quinzième\or
	seizième\or dix-septième\or dix-huitième\or dix-neuvième\or
	vingtième\fi\space}%
	\def\thepart{}%
	\def\partname{Livre \protect\@Fpt}%
	\def\pagename{page}%
	\def\seename{\emph{voir}}%
	\def\alsoname{\emph{voir aussi}}%
	\def\enclname{P.~J. }%
	\def\ccname{Copie à }%
	\def\headtoname{}%
	\def\proofname{Démonstration}%
}
\makeatother

%%%%%%%%%%%%%%%%%%%%%%%%%%%%%%%%%%%%%%%%%%%%%%%%%%%%%%%%%%%%%%%%%%%%%%%%%%%%%%%%

\newcommand{\emphase}[1]{\emph{#1}}
\newcommand{\publication}[1]{\emph{#1}}
\newcommand{\anglais}[1]{\emph{#1}}
\newcommand{\latin}[1]{\emph{#1}}
\newcommand{\grec}[1]{\emph{#1}}

%%%%%%%%%%%%%%%%%%%%%%%%%%%%%%%%%%%%%%%%%%%%%%%%%%%%%%%%%%%%%%%%%%%%%%%%%%%%%%%%

\newcommand{\espacegroupementchiffres}[0]{\hspace{0.10em}}

%%%%%%%%%%%%%%%%%%%%%%%%%%%%%%%%%%%%%%%%%%%%%%%%%%%%%%%%%%%%%%%%%%%%%%%%%%%%%%%%

\newcommand{\franc}[0]{F\xspace}
\newcommand{\livresterling}[0]{£\xspace}

%%%%%%%%%%%%%%%%%%%%%%%%%%%%%%%%%%%%%%%%%%%%%%%%%%%%%%%%%%%%%%%%%%%%%%%%%%%%%%%%

\begin{document}

%%%%%%%%%%%%%%%%%%%%%%%%%%%%%%%%%%%%%%%%%%%%%%%%%%%%%%%%%%%%%%%%%%%%%%%%%%%%%%%%
%                                                                              %
%                                  Couverture                                  %
%                                                                              %
%%%%%%%%%%%%%%%%%%%%%%%%%%%%%%%%%%%%%%%%%%%%%%%%%%%%%%%%%%%%%%%%%%%%%%%%%%%%%%%%

\frontmatter

%%%%%%%%%%%%%%%%%%%%%%%%%%%%%%%%%%%%%%%%%%%%%%%%%%%%%%%%%%%%%%%%%%%%%%%%%%%%%%%%
%                                                                              %
%                                    Titre                                     %
%                                                                              %
%%%%%%%%%%%%%%%%%%%%%%%%%%%%%%%%%%%%%%%%%%%%%%%%%%%%%%%%%%%%%%%%%%%%%%%%%%%%%%%%

\newgeometry{margin=0cm}
\maketitle
\restoregeometry

%%%%%%%%%%%%%%%%%%%%%%%%%%%%%%%%%%%%%%%%%%%%%%%%%%%%%%%%%%%%%%%%%%%%%%%%%%%%%%%%

\mainmatter

%%%%%%%%%%%%%%%%%%%%%%%%%%%%%%%%%%%%%%%%%%%%%%%%%%%%%%%%%%%%%%%%%%%%%%%%%%%%%%%%
%                                                                              %
%                          Préface d’Adolphe Blanqui                           %
%                                                                              %
%%%%%%%%%%%%%%%%%%%%%%%%%%%%%%%%%%%%%%%%%%%%%%%%%%%%%%%%%%%%%%%%%%%%%%%%%%%%%%%%

\chapter*{\textcolor{coquelicot}{{Préface}}\\\vskip 0.2em{\relsize{-2}{\relsize{-4}d’}Adolphe Blanqui}}
\markboth{Préface d’Adolphe Blanqui}{}

Le grand ouvrage d’\AS est resté le livre classique par excellence de l’économie politique. C’est par celui-là qu’il faut commencer l’étude de la science, qui peut-être s’y trouve toute entière encore, malgré les nombreux écrits dont les auteurs se vantent de l’avoir renouvelée de fond en comble. Mais la traduction qu’en a publiée M. le comte \Garnier au commencement de ce siècle, quoique très-supérieure à celles de Blavet et de Roucher, n’était plus à la hauteur des progrès qu’a faits l’art de traduire dans ces derniers temps. Elle n’était même plus au niveau de la science, dont le vocabulaire s’est enrichi et rectifié tout à la fois, depuis que l’enseignement public a permis d’en discuter les termes et d’en fixer la valeur. Notre célèbre économiste \JBS a pris une grande part à cette réforme du langage économique ; M. de \Sismondi y a beaucoup contribué aussi, et le petit livre de \Malthus \emphase{sur les définitions} en économie politique a mis en regard les opinions de tous ces maîtres, y compris les siennes. La langue de la science peut donc être considérée aujourd’hui comme fixée, et ses termes comme suffisamment définis ; mais ils ne l’étaient pas encore lorsque M. le sénateur \Garnier entreprit sa traduction d’\AS.

Il suffit de jeter un regard rapide sur les précédentes éditions pour s’en apercevoir. Le savant traducteur a souvent donné aux mots un sens que la science leur refuse ; quelquefois il a rendu une expression technique par un équivalent vulgaire ; plus souvent il a remplacé par de vagues périphrases des locutions énergiques et précises qui eussent imprimé une allure plus vive à son sujet. Nous avons lieu de penser que cette traduction a dû être faite par des personnes étrangères à la science économique, et revue par l’honorable écrivain qui en a assumé la responsabilité. La gravité de ses ouvrages et leur spécialité ne permettent pas de supposer qu’il eût laissé échapper les nombreuses erreurs que nous avons fait disparaître dans son édition d’\AS, s’il eût traduit lui-même ce beau livre. Toutefois, la traduction que nous donnons après lui n’est autre que la sienne, mais revue et corrigée avec un soin minutieux sur le texte anglais de l’édition \latin{princeps} in-4°, et d’après celles de MM. \Buchanan et \MacCulloch. Cette traduction a même été revue deux fois : la première, par mon malheureux ami, M. \EB, qu’une mort prématurée vient de ravir à la science, et la seconde par moi-même après lui : nous avons apporté un soin extrême à la définition des mots \anglais{stock}, \anglais{currency}, \anglais{circulating medium}, \anglais{legal tender} et une foule d’autres, d’origine anglaise, qui n’avaient pas encore été nettement traduits dans notre langue, du moins avec le sens économique qui s’y rattache. Aussi j’espère que cette nouvelle édition donnera une idée plus exacte de la manière de l’illustre professeur de Glasgow, et qu’elle contribuera à propager de plus en plus en France l’étude du grand ouvrage que nous reproduisons. Plus on approfondit l’économie politique, plus on reconnaît la supériorité du rare génie qui en a jeté les fondements en Europe. Nous avons joint pour la première fois aux \publication{Recherches sur la nature et les causes de la Richesse des nations}, les notes des principaux commentateurs qui en ont développé ou contesté les principes, nommément celles de M. \Buchanan, de M. \MacCulloch, de \Malthus, de \Ricardo, de M. de \Sismondi, de \JeremyBentham. M. \HoraceSay a bien voulu nous communiquer quelques notes inédites que son illustre père avait rédigées sur le livre de \Smith ; enfin nous avons cru devoir ajouter nous-mêmes quelques éclaircissements historiques, quand les commentateurs nous ont manqué, pour lier la chaîne des temps et pour continuer jusqu’à nos jours la partie historique sur laquelle reposent les raisonnements de l’auteur. La nouvelle édition d’\AS est une véritable édition \latin{cum notis variorum} ; non pas que tout ce que les commentateurs ont écrit à propos d’\AS y figure en entier, le commentaire eût été plus long que le livre ; mais rien d’essentiel n’y est omis, et nous avons fait dans M. \MacCulloch même un choix discret et sévère. Les amis de la science nous sauront quelque gré, nous l’espérons du moins, d’avoir reproduit avec plus d’étendue les notes remarquables dont \Buchanan a enrichi son édition de \Smith, devenue si rare en Angleterre, que l’unique exemplaire existant à Paris a coûté 200 francs à la bibliothèque de l’Institut. Cette seule addition au, texte des \publication{Recherches} suffirait pour donner un intérêt particulier à l’édition que nous publions ; mais plusieurs lecteurs attacheront plus de prix encore aux notes historiques, telles que celles qui concernent la banque d’Angleterre et la Compagnie des Indes, dont la situation est exposée depuis 1776 jusqu’à nos jours. 

Au moyen de ces commentaires nombreux et variés, quelquefois plus curieux et plus instructifs que le texte, la lecture d’\AS est devenue indispensable à tous les hommes qui s’occupent en France d’économie sociale, et le nombre s’en accroît tous les jours. Il nous a paru également que ce serait élever au grand économiste un monument digne de lui que d’entourer son ouvrage du cortége des écrivains les plus dignes de figurer à sa suite. Appelé depuis dix années à l’honneur de succéder à \JBS dans la chaire du Conservatoire des arts et métiers, j’ai reconnu par la pratique de l’enseignement et aux difficultés qu’éprouvent les personnes qui commencent l’étude de l’économie politique, combien il serait utile pour elles d’avoir un guide sûr à consulter. La nouvelle édition d’\AS leur sera d’un secours infini. Je n’ai pas cru devoir en détacher les notes de \Garnier ; mais au lieu de les rejeter à la fin des volumes, je les ai fait figurer par longs extraits en regard des passages auxquels elles se rapportent. Rien ne manquera donc à cet ensemble de doctrines, que les progrès de l’art typographique nous ont permis de réunir en deux volumes, et qui seront toujours le point de départ des études économiques en Europe. La traduction de \Garnier était précédée d’une préface dans laquelle l’économiste français a cru devoir envisager à sa manière les théories de \Smith, auxquelles il compare celles des économistes qui l’ont précédé. Quoique cette préface renferme beaucoup de propositions très-susceptibles d’être contestées selon nous, nous l’avons laissée subsister. Les nombreuses notes des divers commentateurs, éparses dans le texte, suffiront pour rétablir les vrais principes.

\vskip 1cm
\hfill \AB \hspace{1cm}
\vskip 1mm
\hfill Paris, 18 novembre 1842 \hspace{1cm}

%%%%%%%%%%%%%%%%%%%%%%%%%%%%%%%%%%%%%%%%%%%%%%%%%%%%%%%%%%%%%%%%%%%%%%%%%%%%%%%%
%                                                                              %
%                Notice sur la vie et les travaux d’Adam Smith                 %
%                                                                              %
%%%%%%%%%%%%%%%%%%%%%%%%%%%%%%%%%%%%%%%%%%%%%%%%%%%%%%%%%%%%%%%%%%%%%%%%%%%%%%%%

\chapter*{\textcolor{coquelicot}{Notice}\\\vskip 0.3em {\relsize{-2} {\relsize{-3}sur} la vie {\relsize{-3}et} les travaux  {\relsize{-3}d’}Adam Smith}}
\markboth{Notice sur la vie et les travaux d’Adam Smith}{}

L’histoire du philosophe célèbre auquel la science de l’économie politique doit ses bases fondamentales est tout entière dans ses ouvrages. Sa vie si simple et si bien remplie n’aurait laissé aucune trace, si la chose eût dépendu de lui-même ; car sa modestie égalait son savoir, nous pouvons dire son génie. On ne connaît presque rien de son enfance, si ce n’est qu’elle fut très-délicate et un moment orageuse. Il fut enlevé à l’âge de trois ans par une bande de chaudronniers ambulants, espèce de bohémiens sur lesquels on ne put le reprendre que dans les bois. C’est un village du comté de Fife, en Écosse, Kirkcaldy, qui a eu l’honneur de donner au monde ce grand économiste : il y naquit le 5 juin 1723, quelques mois après la mort de son père, qui exerçait les fonctions de contrôleur de la douane. Le jeune \AS reçut à l’école de sa ville natale les premiers éléments de son instruction par les soins d’un maître habile, M. David Miller, et il se distingua de bonne heure, comme toutes les natures d’élite, par un grand amour du travail, par des lectures assidues, par la solidité remarquable de sa mémoire. La faiblesse de sa constitution ne lui permettait pas de partager les jeux des enfants de son âge ; aussi vivait-il à l’écart, aimé d’eux néanmoins à cause de la douceur de son caractère, mais pensif et distrait, quelquefois parlant seul et tout haut, ainsi qu’il lui arriva souvent pendant le reste de sa vie. À l’âge de quatorze ans, il quitta l’école de Kirkcaldy pour entrer à l’université de Glasgow, et il y demeura trois années sans que l’on ait jamais su quels furent, dans cette courte période de sa jeunesse, ses travaux de prédilection. C’est seulement à partir de l’année 1740, lors de son entrée au collège de Balliol, à Oxford, que l’on trouve le futur économiste tout entier occupé des mathématiques et de ce que les Anglais appellent \emphase{la philosophie naturelle}, qu’il abandonna bientôt pour se livrer à l’étude des sciences morales et politiques.

Il paraît que sa famille le destinait à la carrière ecclésiastique ; mais soit qu’\AS ne se sentît aucune vocation pour cet état, soit que ses premières lectures philosophiques l’en eussent détourné, il s’adonna avec ardeur à la littérature contemporaine, où régnaient souverainement alors les doctrines de la philosophie railleuse et sceptique dont \Voltaire était l’apôtre en France, et \Hume en Angleterre. \AS fut plus d’une fois réprimandé par l’orthodoxie de ses supérieurs universitaires, pour avoir dérivé vers ces bords dangereux ; mais au bout de sept ans de séjour à Oxford, il était devenu un libre penseur, et sa philosophie s’était affranchie de la routine des écoles, y compris celle du docteur \Hutcheson, célèbre professeur à l’université de Glasgow, qui avait été son premier maître. On croit que c’est de cette époque que datent ses sympathies pour l’historien économiste \Hume, avec lequel il se lia plus tard d’une amitié vive et sincère, qui ne finit qu’avec leur vie. \AS employait ses moments de loisir à l’étude des langues vivantes, principalement de la nôtre, et cette connaissance ne contribua pas peu, par la suite, aux relations qu’il entretint avec les \emphase{économistes} et les \emphase{encyclopédistes} du dix-huitième siècle. Ses biographes n’ont pas assez fait remarquer cette circonstance importante, qui exerça une immense influence sur son génie, et à laquelle nous devons peut-être la tendance philosophique et réformatrice de ses ouvrages. C’est ainsi que peu d’années après, M. \Huskisson, le plus illustre de ses élèves, puisait, dans un premier voyage à Paris, le germe des réformes économiques dont il a eu l’honneur de doter son pays.

Après une résidence de sept ans à Oxford, \AS revint en Écosse auprès de sa mère, et s’établit, en 1748, à Édimbourg, où ses leçons de belles-lettres attirèrent un grand nombre d’auditeurs. On en trouve quelques traces dans la rhétorique de \Blair, qui lui fit plusieurs emprunts sans les avouer, mais qui en a reconnu assez d’autres, pour donner une idée suffisante de la manière simple et sévère de l’économiste écossais. Le succès de ce cours fut tel, qu’\AS ne tarda point à être appelé à Glasgow pour y occuper la chaire de logique, en 1751, et un an après, celle de philosophie morale, illustrée par le professeur \Hutcheson. Son enseignement dura treize ans ; l’empressement des auditeurs fut encore plus considérable qu’à Édimbourg : il en vint de toutes les parties de l’Angleterre et de l’Écosse ; on ne s’entretenait plus que des sujets traités par le nouveau professeur, qui suivit une marche tout à fait différente de celle de ses devanciers, et qui les fit bientôt oublier, si nous en croyons le témoignage des contemporains. Ce n’est pas qu’\AS fût un homme éloquent et capable d’exciter au sein d’un auditoire ces émotions puissantes qui produisent l’enthousiasme : sa diction lente et vulgaire n’avait que le mérite de la clarté. Mais cette clarté était si abondante, les développements que le professeur donnait à ses propositions étaient si riches de faits, si pleins de vues fines et ingénieuses, qu’on se laissait aller au plaisir de l’entendre, comme s’il eût été inspiré. C’est dans la chaire de l’université de Glasgow qu’\AS a jeté les fondements de sa glorieuse renommée ; c’est au service de cette université qu’il a amassé les matériaux de ses deux grands ouvrages : la \publication{Théorie des sentiments moraux} et les \publication{Recherches sur la Richesse des Nations}.

Son cours de philosophie morale, bien que divisé en quatre parties, ne reposait que sur deux bases principales, l’une tout entière de l’ordre métaphysique, et l’autre de l’ordre économique. Aussi sa théologie dégénéra bientôt, si c’est dégénérer, en un cours de morale pratique ; et ses dissertations sur les causes de la prospérité des États se transformèrent sans effort en un traité d’économie politique, qui est devenu le point de départ de tous les autres. Une telle alliance, nouvelle dans les annales de la science des richesses, devait nécessairement assurer à \AS, indépendamment des découvertes opérées par son génie, une supériorité incontestable sur ses prédécesseurs. Ainsi placé aux confins du monde moral et du monde matériel, au point où ces deux grands sujets d’étude se touchent, le philosophe écossais eut de véritables éclairs de révélation, plus brillants toutefois dans les régions de l’industrie que dans les profondeurs de la métaphysique. Toute sa philosophie, développée dans la \publication{Théorie des sentiments moraux}, repose sur l’observation des sentiments qui découlent de la sympathie et de l’antipathie, en vertu desquelles nous compatissons à certaines peines et nous nous associons à certains plaisirs, comme nous éprouvons de la répulsion pour certaines personnes et pour certaines choses. Selon l’auteur, les actions d’autrui seraient toujours le premier objet de nos perceptions morales. Les jugements que nous portons sur la moralité de notre propre conduite ne sont que des applications des jugements portés précédemment sur la conduite de nos semblables. \AS suppose que nous ne pouvons pas nous empêcher de nous mettre à la place d’autrui, pour juger de ce que nous ferions ou de ce que nous faisons nous-mêmes dans des circonstances pareilles. Notre approbation morale est la conséquence de notre sympathie : mais cette sympathie, sur quoi repose-t-elle ? sur la sensibilité, qui est une affaire de tempérament, très-diverse chez les hommes et grandement sujette à l’erreur. Aussi le philosophe écossais est-il obligé de recourir au tribunal de la conscience pour rectifier les écarts ou les lenteurs de l’émotion sympathique, indispensable à consulter, selon lui, dans l’appréciation morale des actions humaines. La raison, cette puissance abstraite et jusqu’à ce jour mal définie, lui semble seule capable de préciser les règles générales qui sont l’expression exacte des décisions de la sympathie. Toutefois, \AS ne saurait admettre que la raison soit la source unique de nos premières notions du juste et de l’injuste. Il se rejette, en désespoir de cause, dans l’utopie d’une bienveillance universelle qui relierait toutes les nations entre elles pour leur bonheur commun, et qui donnerait à la morale une base éternelle et incontestée.

Il faut laisser aux philosophes le soin de prononcer sur ces questions aussi anciennes que le monde, et qui seront encore longtemps débattues. \AS leur a payé tribut, comme tous les grands esprits qui ont régné dans le domaine de la pensée, mais il ne les a point résolues. Il les poursuit une à une dans l’histoire, dans les arts, dans les lettres, avec une sagacité merveilleuse et la loupe à la main. Il les analyse avec patience, les tourne et les retourne en tous sens, et se perd quelquefois avec elles dans un dédale de digressions. On ne peut s’empêcher d’admirer, néanmoins, l’honnêteté de ses maximes, la richesse de ses observations et le choix heureux de ses exemples. Sa \publication{Théorie des sentiments moraux}, incomplète à beaucoup d’égards, comme tous les systèmes philosophiques, produisit une grande sensation lorsqu’elle parut pour la première fois en 1759\footnote{Voici dans quels termes plaisants son ami \Hume lui rendait compte du succès de la \publication{Théorie des sentiments moraux} « Mon cher monsieur \Smith, disposez votre âme à la tranquillité ; montrez-vous philosophe pratique comme vous l’êtes par état ; pensez à la légèreté, à la témérité des jugements ordinaires des hommes, et souvenez-vous que \Phocion soupçonnait toujours qu’il avait dit quelque sottise quand il se voyait accueilli par les applaudissements de la multitude. Supposant donc que, par ces réflexions, vous êtes préparé à tout, j’en viens enfin à vous annoncer que votre livre a éprouvé le plus fâcheux revers, car le public semble disposé à l’applaudir avec excès. Il était attendu par les sots avec impatience, et la tourbe des gens de lettres commence déjà à chanter très-haut ses louanges. Trois évêques passèrent hier à la boutique du libraire pour l’acheter et pour s’informer de l’auteur. \CharlesTownsend, qui passe pour le premier juge d’Angleterre, est si épris de cet ouvrage, qu’il a dit à \Oswald qu’il voudrait confier à l’auteur l’éducation du duc de Buccleugh, et qu’il saurait mettre à ses soins un prix capable de le déterminer. » (\emphase{Lettre du 12 avril 1759.})}. Jusqu’alors \AS ne s’était pas fait connaître comme écrivain, et il n’existait de lui que deux articles insérés dans une revue éphémère qui cessa de paraître après la publication du second numéro. L’un de ces articles, consacré à la critique du grand Dictionnaire de Johnson, avait été remarqué par sa facture pleine de délicatesse et par des nuances très-heureusement saisies. La \emphase{Théorie des sentiments moraux}, bientôt suivie d’une \publication{Dissertation sur l’origine des langues}, plaça le philosophe de Glasgow à un très-haut degré dans l’opinion. On put dès lors juger de ses leçons avec plus de sûreté qu’on ne l’avait encore fait dans les amphithéâtres, et cette épreuve difficile tourna entièrement à son honneur. \AS était revenu depuis près de quatre ans à Glasgow, lorsqu’on lui proposa d’accompagner le jeune duc de Buccleugh dans un voyage sur le continent, vers la fin de 1763. Dans ce premier voyage, il ne fit que traverser la France pour aller résider à Toulouse avec son élève, pendant plus d’une année. \Smith mit à profit cette excursion en observant avec l’exactitude scrupuleuse qui caractérise ses ouvrages, tout ce qui méritait, dans un pays comme le nôtre, l’attention d’un homme tel que lui. On retrouve, dans le cours de son livre, la trace des impressions profondes que ce premier séjour avait laissées dans son esprit. Le profit qu’il retira de sa courte visiter à Genève ne fut pas moins utile à ses études, qui avaient déjà un caractère de solidité pratique, même dans leur première spécialité, exclusivement philosophique et métaphysique.

Mais c’est surtout à l’époque de son second voyage à Paris, en 1765, que les idées de l’illustre Écossais se fixèrent d’une manière définitive sur la science économique, dont il devait être le plus habile réformateur. Une recommandation de son ami \Hume le mit en relations suivies avec les auteurs de l’\publication{Encyclopédie} et avec les principaux chefs de l’école \emphase{physiocrate}. \AS se fut bientôt lié avec eux, nommément avec \Turgot et \Quesnay, et leurs doctes entretiens ne tardèrent point à l’initier aux études qui faisaient l’objet de leurs méditations. \Smith apportait sans doute avec lui des connaissances profondes et des doctrines nouvelles en économie politique ; mais il est impossible de douter que ses rapports avec les encyclopédistes et les économistes français n’aient exercé une influence décisive sur son esprit\footnote{L’abbé Morellet s’exprime ainsi sur \AS, dans ses Mémoires : « J’avais connu \Smith dans un voyage qu’il avait fait en France vers 1762 : il parlait fort mal notre langue ; mais sa \emphase{Théorie des sentiment moraux}, publiée en 1759, m’avait donné une grande idée de sa sagacité et de sa profondeur. Et véritablement, je le regarde encore aujourd’hui comme un des hommes qui a fait les observations et les analyses les plus complètes dans toutes les questions qu’il a traitées. M. \Turgot qui aimait, ainsi que moi, la métaphysique, estimait beaucoup son talent. Nous le vîmes plusieurs fois ; il fut présenté chez Helvétius : nous parlâmes théorie commerciale, banque, crédit public, et de plusieurs points du grand ouvrage qu’il méditait. Il me fit présent d’un joli portefeuille anglais de poche, dont je me suis servi vingt ans. »}. Il a déclaré lui-même que son intention avait été de dédier à \Quesnay son grand ouvrage sur la richesse des nations, si le célèbre docteur ne fût pas mort avant cette publication mémorable. Il est facile, en effet, de reconnaître l’empreinte de l’école \emphase{économiste} dans les œuvres de \Smith, quoique ses doctrines diffèrent en plusieurs points de celles de \Quesnay\footnote{\Smith a déclaré plusieurs fois que « le système d’économie politique de \Quesnay, avec toutes ses imperfections, était l’opinion la plus voisine de la vérité qui eût encore été publiée sur les principes de cette importante science. »}. Mais \Quesnay a eu la priorité d’un système, quel qu’il fût, et nous ne craignons pas de dire que ses erreurs même ont été utiles aux progrès de la science, en appelant sur les questions sociales l’attention et parfois l’enthousiasme de son siècle. \AS a évidemment emprunté à cette école ses arguments les plus éloquents en faveur de la liberté du commerce et de l’industrie ; il n’a inventé contre elle que sa théorie de la puissance du travail, qui a renversé l’hypothèse spécieuse de \Quesnay sur la prédominance de la propriété territoriale.

Cette influence incontestable des encyclopédistes et des économistes français ne se révéla point aux yeux des contemporains d’\AS, à l’apparition de ses \publication{Recherches sur les causes de la Richesse des Nations}. Avant de publier cet immortel ouvrage, l’auteur s’était comme retiré en lui-même, au sein d’une profonde solitude où il vécut dix années en butte aux plaintes et même aux sarcasmes de ses amis. \Hume lui écrivait pendant cette retraite opiniâtre, à la date de 1772 : « Je n’accepterai point l’excuse de votre santé, que je n’envisage que comme un subterfuge inventé par l’indolence et l’amour de la solitude. En vérité, si vous continuez d’écouter tous ces petits maux, vous finirez par rompre entièrement avec la société, au grand détriment des deux parties intéressées. » Déjà en 1769, \Hume avait essayé de vaincre la résistance de \Smith, sans être plus heureux : « Je veux savoir ce que vous avez fait, lui disait-il, et j’ai dessein d’exiger de vous un compte rigoureux de l’emploi de votre temps dans votre retraite. » Pendant ce temps, \AS, inébranlable, vivait modestement à Kirkcaldy auprès de sa mère et de quelques amis d’enfance, et il travaillait sans relâche au monument qui devait immortaliser sa mémoire. Lorsqu’enfin il fit paraître son livre (c’était au commencement de 1776), \Hume, que nous avons plaisir à citer comme l’expression la plus avancée des économistes de l’époque, lui écrivit, sous la date du 1er avril de la même année, ces lignes remarquables : « Courage, mon cher monsieur \Smith : votre ouvrage m’a fait le plus grand plaisir, et en le lisant, je suis sorti d’un état d’anxiété pénible. Cet ouvrage tenait si fort en suspens et vous-même, et vos amis, et le public, que je tremblais de le voir paraître ; mais enfin je suis soulagé. Ce n’est pas qu’en songeant combien cette lecture exige d’attention et combien peu le public est disposé à en accorder, je ne doive encore douter quelque temps du premier souffle de la faveur populaire. Mais on y trouve de la profondeur, de la solidité, des vues fines et ingénieuses, une multitude de faits curieux ; de tels mérites doivent tôt ou tard fixer l’opinion publique. » \Hume terminait cette lettre en annonçant à \Smith qu’il lui contesterait quelques-uns de ses principes ; et certes, au moment où il écrivait, lui seul peut être, en Europe, était en état de lutter contre un si formidable jouteur.

À l’apparition des \publication{Recherches sur les causes de la Richesse des Nations}, la France était sous le charme de l’école \emphase{physiocrate}, et quoique le chef de la secte, \Quesnay, fût déjà mort, ses successeurs, plus clairs et plus complets qu’il ne l’avait été lui-même, propageaient ses doctrines avec une ardeur religieuse. Mercier de La Rivière, le marquis de Mirabeau, Dupont de Nemours, et vingt autres appartenaient à cette église libérale, qui trouva bientôt dans \Turgot un ministre assez puissant pour faire exécuter ses commandements. Aussi le livre d’\AS n’eut-il qu’un retentissement très-borné en France. Tout le monde vivait sous l’empire de la \emphase{Formule universelle}, développée en plusieurs volumes par l’\emphase{Ami des hommes}. Des milliers de livres avaient paru pour attaquer avec une égale ardeur ces dogmes mystérieux du \emphase{produit net}, en vertu desquels l’école \emphase{économiste} classait les producteurs suivant de nouvelles méthodes, et plaçait au premier rang d’entre eux les propriétaires fonciers. \AS renversa d’un trait de plume cet ingénieux échafaudage, en rendant au travail les prérogatives éternelles qui lui appartiennent dans l’intérêt des sociétés. C’est là son plus beau titre de gloire, et quoique les \publication{Traités politiques} de \Hume, qui avaient paru en 1752, aient dû lui suggérer quelques-unes de ses idées sur ces hautes questions, il n’y eut qu’un cri d’admiration, en Angleterre, à l’apparition des \publication{Recherches sur les causes de la Richesse}, comme si nul autre livre important n’eût été publié avant celui-là\footnote{La première édition des \publication{Recherches} a paru en 1776, en deux volumes in-4°. L’auteur a fait quelques transpositions et quelques changements dans la seconde, qui est devenue le point de départ de toutes les autres, sauf quelques corrections de peu d’importance à la quatrième édition, publiée en 1784.}.

Deux ans après cette publication, \AS fut nommé commissaire des douanes en Écosse, par l’influence du duc de Buccleugh son ancien élève ; mais cette position qui assurait le repos de ses vieux jours, a été fatale à la science, en condamnant le philosophe de Glasgow à des travaux d’un ordre inférieur, qui ont absorbé le reste de sa vie. En effet, depuis le moment de son installation à Édimbourg en qualité de commissaire des douanes, en 1778, jusqu’en 1790, époque de sa mort, l’illustre économiste se borna au rôle d’éditeur de ses ouvrages. L’université de Glasgow, justement fière des succès du professeur qui lui avait appartenu, lui décerna, en 1787, le titre de recteur, flatteuse distinction à laquelle il se montra très-sensible\footnote{« Aucune place, dit-il, ne pouvait me causer une satisfaction plus réelle. Nul homme ne peut avoir plus d’obligations à une société que je n’en ai à l’université de Glasgow. C’est elle qui m’a élevé et m’a envoyé à Oxford. Peu après mon retour en Écosse, elle m’élut au nombre de ses membres. Lorsque je repasse cette période de treize années, pendant lesquelles j’ai été membre de cette société, je l’envisage comme la plus heureuse époque de ma vie ; et maintenant, après vingt-trois ans l’absence, me voir rappelé au souvenir de mes amis d’une manière si agréable, c’est un sentiment qui pénètre mon cœur d’une joie pure et que je ne saurais exprimer. »}. Trois années auparavant, \AS avait perdu sa mère et une parente à laquelle il paraissait attaché par les liens les plus tendres. Cette fâcheuse circonstance aggrava chez lui les infirmités de l’âge qui s’étaient fait sentir de bonne heure, malgré la régularité de ses habitudes, et sa mort arriva comme s’était écoulée sa vie, sans altérer en rien la sérénité de son âme. Chacun sait que, sentant sa fin approcher, il fit brûler par ses amis une foule de manuscrits qu’il jugeait indignes de lui survivre, et ses volontés à ce sujet furent religieusement exécutées. Il existe une lettre écrite par lui à \DavidHume, en avril 1773, dans laquelle se manifestait déjà la ferme résolution de se montrer sévère au point d’envelopper dans une même réprobation tous ses travaux inédits, à l’exception d’une \publication{Histoire des systèmes astronomiques} jusqu’au temps de \Descartes.

Ainsi mourut cet illustre fondateur de l’économie politique, après une carrière paisible et honorée, mais dépourvue de l’éclat qui devait bientôt s’attacher à son nom. On n’apprit qu’après sa mort une foule de bonnes actions qu’il avait cachées et de services généreux qu’il avait rendus. Sa vie avait été si simple et si retirée, qu’on en connaît à peine les principaux événements ; on sait seulement qu’il était d’un commerce agréable, d’un caractère timide et distrait, et d’une indépendance philosophique à la hauteur de son génie. À l’université d’Oxford, il fut un étudiant sceptique et hardi ; dans sa chaire, à Glasgow, il se montra professeur consciencieux, original, clair et profond tout à la fois. Quoiqu’il improvisât ses leçons avec lenteur et sans élégance, on l’écoutait avec avidité ; on discutait avec chaleur les sujets qu’il avait traités et sur lesquels il savait répandre un intérêt inexprimable. Son style reproduit assez fidèlement ce que ses contemporains ont dit de ses leçons. Il est toujours grave, simple et lucide, mais souvent assez lourd, prolixe et traînant. \AS ne s’est servi de la langue que comme d’un instrument. Préoccupé du fond plutôt que de la forme, il semble dédaigner de descendre aux artifices de langage, trop souvent nécessaires pour fixer l’attention d’un nombreux auditoire et celle des lecteurs. Cependant, le feu sacré de l’éloquence brille par moments dans plusieurs de ses pages, lorsque, entraîné par l’importance du sujet et quelquefois ébloui par les vives clartés de son génie, il promène un regard ferme et tranquille sur les phénomènes économiques de l’existence des sociétés. Sa véritable gloire est d’en avoir découvert un grand nombre, et d’avoir analysé les plus essentiels d’une manière admirable. On soupçonnait à peine, avant lui, les lois qui président au développement social des peuples ; on n’avait qu’une connaissance imparfaite et empirique des éléments de leur prospérité. La richesse s’ignorait elle-même, comme la pauvreté. Les bons gouvernements agissaient au hasard, guidés seulement par l’honnêteté de leurs intentions, qui ne les empêchait pas toujours de faire fausse route. La science des finances et celle du commerce, les procédés économiques de l’industrie, les bases fondamentales du développement agricole, n’étaient qu’ébauches avant lui. \AS a expliqué le premier comment la vie circulait dans ces grands corps, qu’on appelle des nations ; il a exposé les causes de leur élévation et de leur décadence avec une supériorité inconnue aux plus habiles historiens.

Sa véritable renommée repose tout entière sur le traité d’économie politique qu’il a modestement intitulé \publication{Recherches sur la nature et les causes de la Richesse des Nations}. Il est très-probable, malgré le soin extrême qu’il mit à la rédaction de cet ouvrage, qu’\AS n’en soupçonna jamais toute la portée. Il affectionnait de préférence ses œuvres philosophiques, et il était loin de prévoir qu’un jour ses travaux économiques deviendraient le point de départ d’une ère nouvelle dans le gouvernement des sociétés. Comme il avait publié avant sa mort cinq éditions de sa \publication{Théorie des sentiments moraux}\footnote{Voyez, pour de plus amples détails, le \publication{Précis sur la vie et les écrits d’\AS}, par M. \DugaldStewart, traduction de \Prevost de Genève. Cette édition comprend les \publication{Essais philosophiques}, dont M. \Cousin a fait l’exposition et la critique dans ses leçons à la Faculté des lettres de Paris. M. \MacCulloch a publié une notice biographique sur \AS, en tête de l’édition qu’il a donnée de ses \publication{Recherches}.}, et seulement quatre éditions des \publication{Recherches}, il dut croire que ses contemporains faisaient plus de cas de sa philosophie que de son économie politique. Et pourtant, quelle différence dans la destinée de ces deux livres ! Personne ne songe plus à l’un, et la politique de l’avenir repose sur l’autre. La seule réhabilitation du travail suffirait à la gloire de \Smith ; mais il en a signalé les avantages et analysé les procédés avec une telle supériorité de vues, que ses théories peuvent être considérées comme de magnifiques découvertes. C’est lui qui a le premier démontré la nécessité d’une alliance perpétuelle entre le capital et le travail, trop souvent divisés. Les économistes les plus hardis de l’époque actuelle n’ont rien écrit de plus énergique que le tableau qu’il a tracé des coalitions d’ouvriers et de maîtres, ni rien de plus éloquent que ses irrésistibles manifestes en faveur de la liberté de l’industrie. C’est à lui que nous devons toutes les libertés dont on abuse tant aujourd’hui, et que des novateurs rétrogrades voudraient proscrire, pour s’épargner l’embarras de les organiser. Nul n’a porté des regards plus sûrs et plus profonds sur les éléments du crédit. Son chapitre des Banques est resté un modèle inimitable de clarté, de logique et de prudence. \AS a tracé d’une main ferme la limite qu’elles ne devaient pas franchir ; et quiconque, depuis, peuple ou roi, a osé s’écarter de ses sages prescriptions, et \emphase{se suspendre aux ailes d’Icare}, pour nous servir de l’expression de l’auteur, est tombé dans l’abîme.

On éprouve, en étudiant ce bel ouvrage, un sentiment particulier de satisfaction qui est dû à la rectitude des idées et à l’enchaînement rigoureux des déductions. Une fois le sujet \emphase{lancé}, si j’ose dire, \AS ne lui laisse ni paix ni trêve : il l’examine sous toutes ses faces, l’appuie de mille exemples, le vivifie de mille comparaisons, et l’inonde, en un mot, de lumière. Ses voyages lui fournissent des observations de tout genre qu’il distribue avec un art merveilleux, et sa philosophie l’aide à en tirer le meilleur parti. Son indépendance ne recule devant aucune conséquence, dès qu’il s’agit des intérêts de la vérité. Quelquefois même, ces intérêts lui inspirent des accents d’indignation qui feraient honneur aux écrivains les plus avancés de nos jours. L’esprit de monopole, source de tant de guerres et de crimes, lui était surtout antipathique. « Le commerce, dit-il, qui, pour les nations comme pour les individus, devrait être un lien d’union et d’amitié, est devenu la source la plus féconde des animosités et de la discorde. L’ambition capricieuse des rois et des ministres n’a pas été plus fatale au repos de l’Europe, que l’impertinente jalousie des commerçants et des manufacturiers. La violence et l’injustice de ceux qui gouvernent le monde sont un mal qui date de loin, et contre lequel la nature des affaires humaines laisse peu espérer de remède assuré. Mais la basse rapacité, le génie monopoleur des négociants et des manufacturiers, qui ne sont ni ne doivent être les maîtres du monde, sont des vices, incorrigibles peut-être, mais qu’on peut très-aisément empêcher de troubler le repos de tout autre que de ceux qui s’y livrent. »

L’expérience n’a infirmé jusqu’à ce jour qu’une seule des doctrines d’\AS, je veux parler de celle qui attribue à la liberté absolue de l’industrie le soin de suffire à toutes les nécessités sociales, et la possibilité de réaliser toutes les sortes de progrès. Ce grand économiste avait dit quelque part : « Pour élever un État du dernier degré de barbarie au plus haut degré d’opulence, il ne faut que trois choses : la paix, des taxes modérées et une administration tolérable de la justice. \emphase{Tout le reste est amené par le cours naturel des choses}. » Nous avons vu, depuis, le cours naturel des choses produire des effets désastreux et créer l’anarchie dans la production, la guerre pour les débouchés, et la piraterie dans la concurrence. La division du travail et le perfectionnement des machines, qui devaient réaliser pour la grande famille ouvrière du genre humain la conquête de quelques loisirs au profit de sa dignité, n’ont engendré, sur plusieurs points, que l’abrutissement et la misère ! Quand \Smith écrivait, la liberté n’était pas encore venue avec ses embarras et ses abus. Le professeur de Glasgow n’en prévoyait que les douceurs. Il croyait le printemps perpétuel sur cette terre inconnue qu’il allait découvrir. C’est à ses successeurs que devaient échoir les rigueurs de l’hiver, et \Smith aurait sans doute écrit comme M. de \Sismondi, s’il eût été témoin du triste état de l’Irlande et des districts manufacturiers de l’Angleterre au temps où nous vivons. Nous avons appris en Europe, par une dure expérience, que les gouvernements étaient bons à quelque chose, et que la liberté mal cultivée donnait, comme tous les arbres sauvages, des fruits souvent très-amers. L’horizon industriel était bien étroit, quand \AS pouvait le percer d’outre en outre, en allant de Glasgow à Toulouse. Les États-Unis n’avaient alors que quinze cent mille habitants, au lieu de vingt millions, et la Compagnie des Indes ne menaçait pas, comme aujourd’hui, les remparts de Pékin. \Smith se plaignait beaucoup des douanes de provinces et des petites entraves de son temps. Qu’aurait-il dit en présence du blocus continental ?

Tous les éléments de la richesse, sauf la terre, ont donc éprouvé de grandes modifications depuis la publication du livre de \Smith. L’Europe d’aujourd’hui n’a presque plus rien de commun avec l’Europe de son temps. En 1776, l’industrie du coton, la filature mécanique, la machine à vapeur, les chemins de fer, n’existaient réellement point. Nous avons porté, en France, le dernier coup à tous les préjugés de caste et à la propriété féodale. L’Amérique du Sud est émancipée, convulsivement sans doute et stérilement jusqu’à présent ; mais le voile qui couvrait ce vaste continent est tout entier levé. Nos bateaux à vapeur ont repris la vieille route de l’Inde abandonnée depuis la grande querelle des Vénitiens et des Portugais. Que dis-je ? Venise elle-même n’est plus, la Grèce est affranchie, l’Égypte se réveille ; tout est changé depuis l’œuvre de \Smith, et néanmoins cette œuvre demeure immortelle. Elle peut se résumer en deux mots : la paix et le travail. C’est par ce double chemin que l’humanité a pris son essor que rien n’arrêtera désormais. La gloire de \Smith est de l’avoir tracé, d’en avoir démontré la supériorité sur tous les autres. C’est sur la nature aujourd’hui, grâce à lui, que les grands peuples aiment à faire des conquêtes. C’est l’esprit de son livre qui a prévalu aux États-Unis et qui a couvert ce pays de villes, de canaux et de défrichements. C’est l’oubli de ses préceptes qui l’infeste à présent de banqueroutes et de sinistres. Sur quelque point du globe que l’on tourne les yeux, la fortune sourit aux nations qui se montrent fidèles à la sagesse économique ; la misère désole les contrées où cette sagesse est méconnue. \AS a eu l’honneur insigne d’être le plus habile interprète de cette sagesse collective, œuvre du temps et du génie, qu’on appelle la science économique. Quelques progrès que la science fasse à l’avenir, le philosophe de Glasgow en sera toujours considéré comme le fondateur, et son livre sera toujours lu avec fruit, même quand il en aura paru de meilleurs.

Nous ne relèverons point ici les défauts très-connus du sien : il manque de méthode, d’ordre et de composition. La lecture en est difficile et fatigante au premier abord ; mais il ne faut pas s’arrêter aux aspérités qu’on y trouve, et bientôt la solidité de l’édifice, ses vastes dépendances, ses admirables compartiments apparaîtront à la vue du lecteur. \Smith se répète quelquefois ; plus souvent il s’oublie et paraît s’égarer ; mais le fil qui le guide ne se brise jamais : vous le voyez toujours arriver à son but, même après les plus longs détours qui devaient l’en éloigner. Quiconque est assez curieux pour le suivre, ne fût-ce qu’un moment, se sent entraîné dans sa course opiniâtre et sévère, comme celle des cylindres de nos industries, où tout le corps doit passer pour peu qu’on y engage le petit doigt. On ne quitte point cet auteur sans être plus instruit. Son génie projette des lueurs si vives sur tous les sujets, que même lorsqu’il se trompe, il aide le lecteur à reconnaître ses erreurs et lui apprend à s’en défendre. Chez lui, jamais rien de hasardé, d’aventureux et de conjectural : il ne parle que des choses qu’il a approfondies, des villes qu’il a vues, des faits qu’il a vérifiés. Sa probité se fût révoltée à l’idée des extravagances de toute sorte qui devaient agiter le terrain de l’économie politique, et des promesses décevantes qu’on fait ou qu’on accueille de nos jours en son nom. L’expérience lui avait appris que l’humanité marche d’un pas plus lent que la vie de l’homme, et qu’il faut plusieurs relais de générations pour arriver à certains résultats qu’on ne saurait atteindre en quelques années. \AS était surtout un homme de bon sens, d’un jugement exquis, d’une raison inébranlable. On dirait qu’il a vécu exempt de passions, sauf celle des livres, en voyant avec quelle haute impartialité il a envisagé toute chose et poursuivi, au travers des devoirs de sa position, le cours de ses longues et sérieuses études.

Tous ceux de ses contemporains qui ont vécu dans son intimité nous le représentent comme doué d’une humeur douce, spirituel et gai dans la conversation, mais souvent embarrassé de sa contenance, surtout en présence des étrangers. On cite une foule d’anecdotes plaisantes à propos des distractions auxquelles il était sujet ; mais personne n’a jamais eu à se plaindre de son caractère, et il est demeuré constamment fidèle à ses amis, malgré les vicissitudes de la vie littéraire au dix-huitième siècle. Quelques-uns de ses biographes ont assuré qu’il avait entretenu avec Turgot une correspondance dont il n’est resté aucune trace. Ce qui est certain, c’est que pendant plusieurs années il ne cessa de suivre avec sollicitude la marche de l’école \emphase{économiste} française, et qu’il sembla recevoir de Paris une partie des inspirations dont il se nourrissait à Kirkcaldy, pendant la rédaction de son grand ouvrage. Nous pouvons donc revendiquer pour notre pays l’honneur d’avoir fourni quelques matériaux au monument élevé par \Smith. \Smith est de la famille des encyclopédistes et des physiocrates. Sa philosophie est de l’école dont son ami \Hume représentait les principes en Angleterre ; mais son économie politique lui appartient plus exclusivement. Elle est aujourd’hui traduite dans toutes les langues et enseignée dans toutes les chaires. Elle est devenue le guide le plus indispensable des historiens et des hommes d’État, et c’est là qu’il faut étudier la physionomie de ce penseur original et profond, dont il ne nous reste pas même un portrait\footnote{Il n’existe de lui qu’un médaillon de profil, par Tassie, et une silhouette en pied, dessinée par Kay, en 1790, l’année de sa mort.}.

%%%%%%%%%%%%%%%%%%%%%%%%%%%%%%%%%%%%%%%%%%%%%%%%%%%%%%%%%%%%%%%%%%%%%%%%%%%%%%%%
%                                                                              %
%                          Préface de Germain Garnier                          %
%                                                                              %
%%%%%%%%%%%%%%%%%%%%%%%%%%%%%%%%%%%%%%%%%%%%%%%%%%%%%%%%%%%%%%%%%%%%%%%%%%%%%%%%

\chapter*{\textcolor{coquelicot}{Préface}\\\vskip -2mm{\relsize{-2}{\relsize{-4}de}\\\vskip -2.5mm Germain Garnier}}
\markboth{Préface de Germain Garnier}{}

\begin{center}
	{
		\itshape
		contenant
	}
\end{center}

1° Un précis des divers systèmes d’économie politique qui ont été suivis par les gouvernements — 2° Un exposé sommaire de la doctrine de \Smith, comparée avec celle des économistes français — 3° Une méthode pour faciliter l’étude de l’ouvrage de \Smith.

\vskip 1em

\section{Précis des divers systèmes d’économie politique qui ont été suivis par les Gouvernements}

L’observation des lois d’après lesquelles les richesses d’une nation se distribuent naturellement entre les différents ordres de la société, et la recherche des causes qui tendent à multiplier ces richesses, forment la partie la plus difficile, la plus compliquée et la plus controversée de la science connue sous le nom d’\emphase{économie politique}.

Cette branche importante de la science sociale n’avait point occupé les anciens philosophes, et elle ne pouvait pas en effet s’offrir à leurs méditations sous le même aspect où elle fut considérée par les peuples modernes, le seul qui puisse en faire le sujet d’une étude philosophique.

D’après la constitution politique des sociétés chez les peuples de antiquité, la terre productive, le capital employé à son exploitation, l’ouvrier chargé de la culture, étaient tous la propriété de la même personne. Le citoyen propriétaire du fonds l’était aussi nécessairement des bestiaux, des engrais et des instruments de culture. Les travaux du labour et de la récolte étaient exécutés par ses esclaves, et la régie ou inspection du domaine était confiée à l’un de ses principaux esclaves ou à quelqu’un de ses affranchis\footnote{La plupart des affranchis restaient dans la maison de leur maître, où ils étaient nourris et entretenus, et où ils se rendaient utiles : ils recevaient des gratifications méritées par leurs services. Si l’affranchi eût été obligé de quitter la maison où il avait été élevé, la liberté aurait été pour lui, le plus souvent, un présent funeste.}. Les vêtements, les meubles d’usage étaient fabriqués par des esclaves, et le commerce étranger fournissait les articles de luxe. On achetait à haut prix les ouvrages des artistes, mais on ne connaissait pas l’industrie manufacturière. Il n’y avait guère d’entreprises particulières que pour l’exploitation des mines ou pour la fabrication des armes et de ce qui se consommait à la guerre. Les citoyens qui possédaient ces sortes d’établissements les faisaient diriger par quelque esclave de confiance, et s’ils avaient quelquefois besoin d’emprunter pour soutenir ou étendre leur entreprise, ce qu’ils empruntaient n’était pas un capital pécuniaire, mais un capital en esclaves, dont ils payaient le loyer à raison de tant par jour et par tête, ainsi que nous l’apprenons de \Xenophon, dans son \publication{Traité sur l’amélioration des finances d’Athènes}, dans lequel on trouve des informations sur le prix et les clauses en usage dans ces sortes de marchés. Le seul commerce de quelque importance était celui qui se faisait avec l’étranger ; les particuliers qui voulaient s’y intéresser prêtaient leur argent au négociant voyageur à des conditions réglées par la coutume du lieu, et qui variaient selon le plus ou le moins de risques du voyage. \Demosthenes (\latin{in} \publication{Phormionem}) donne un exposé très-clair des formes usitées dans les contrats ou \emphase{prêts à l’aventure}.

Les magistratures, les premiers emplois civils et militaires étant exercés gratuitement, les dépenses de l’État étaient peu considérables et ne donnaient lieu qu’à de faibles tributs. Dans les crises inattendues, et lorsque l’État avait à pourvoir à de grands besoins, le zèle et le dévouement des principaux citoyens offraient à la patrie des ressources toujours suffisantes. Lorsque, en l’an 347 de Rome, le sénat décréta qu’il serait donné une solde à l’infanterie, les patriciens s’empressèrent de faire don à la république d’une partie des richesses qu’ils possédaient, afin de la mettre en état de faire face à cette nouvelle dépense. « C’était, dit \TiteLive, un beau spectacle que cette file de chariots chargés de cuivre brut (\emphase{œs grave}) qui se dirigeaient vers le trésor public. » (Liv. IV, § 60.)

Pendant la seconde guerre punique, lorsqu’il fallut faire de nouvelles levées de troupes, équiper des flottes et faire tête à un ennemi formidable qui pressait Rome de toutes parts, les consuls proposèrent, comme il avait été déjà pratiqué plusieurs fois, d’obliger les citoyens, chacun selon ses facultés, à fournir la solde et les vivres pour trente jours à un certain nombre de rameurs. Cette proposition, dit \TiteLive, dès qu’elle fut connue du peuple, excita de violents murmures. « Nous sommes, disaient les mécontents, épuisés par les impôts ; les esclaves qui devaient cultiver nos terres sont aux armées ou sur la flotte, et nos champs restent en friche. Que les consuls vendent donc et nos biens et nos personnes, car aucune autorité ne saurait nous faire donner ce que nous n’avons pas. » C’est dans cette conjoncture si pressante et si critique que le consul \ValeriusLaevinus invita les sénateurs à donner les premiers l’exemple par une contribution volontaire de tout ce qu’ils possédaient en matière d’or et d’argent et en monnaie de cuivre, sans se réserver autre chose que leur anneau d’or, celui de leurs femmes, la bulle de leurs fils et la quantité de monnaie indispensable pour la dépense de leur maison. Chacun répond à ce noble appel par un assentiment général et par des acclamations unanimes ; la séance est levée spontanément, et les sénateurs se disputent l’honneur d’être les premiers inscrits sur les rôles avec un tel empressement, que les triumvirs et les greffiers ne peuvent suffire à recevoir et à enregistrer les soumissions. (Liv. XXVI, § 35 et 36.)

Il en était de même à Athènes. \Demosthenes, en plaidant contre \Phormion, rappelle diverses occasions dans lesquelles les frères \Chrysippe, qu’il défend, avaient fait à l’État des dons patriotiques, soit en argent, soit en denrées. Le butin que \PaulEmile versa au trésor de la république romaine, après la défaite de Persée, parut suffisant pour satisfaire à l’avenir aux dépenses du Gouvernement, et dès lors tous les tributs furent abolis.

Les dettes publiques, les emprunts de l’État, les moyens de crédit et toutes ces créations de propriétés imaginaires dont la jouissance repose sur les impôts que nos arrière-neveux voudront bien payer un jour, sont des fictions qui étaient totalement inconnues aux anciens, même dans ces temps dégénérés où la subtilité du sophisme prit la place de cette saine et franche philosophie qui réglait leur conduite publique et privée. Ces peuples n’auraient jamais pu comprendre comment un gouvernement peut se constituer débiteur à perpétuité envers ses sujets, et comment ceux-ci comptent pour unique gage de leur créance les tributs qu’ils fourniront eux-mêmes à l’avenir. Cette invention, dont il est fort douteux que les peuples et les gouvernements aient à se féliciter, appartient entièrement à notre moderne Europe.

Ce n’est pas qu’il soit sans exemple que, dans des besoins urgents, les chefs du Gouvernement se soient momentanément aidés de la bourse de quelques riches citoyens, lorsque le trésor public manquait absolument de fonds ; mais ces emprunts, contractés personnellement par les magistrats et sur leur foi, étaient remboursables à une échéance déterminée dont le terme était très-court. La plus entière confiance d’une part, la plus religieuse fidélité de l’autre, présidaient à ces contrats, et aucune considération n’aurait pu en modifier ni même en différer arbitrairement l’exécution.

Deux ans après la contribution volontaire provoquée par le consul Lævinus, les besoins toujours croissants de la guerre la plus redoutable que la république ait eu à soutenir, mirent les consuls dans la nécessité d’emprunter de quelques citoyens une somme d’argent qui fut stipulée payable en trois termes égaux de deux en deux ans. Il fut satisfait avec ponctualité au payement des deux premiers termes, au milieu même des embarras et des charges de la guerre, et lorsque les armées victorieuses d’Annibal et de ses puissants alliés semblaient devoir apporter à Rome, d’un moment à l’autre, la destruction ou la servitude. Au commencement de l’an 550, le troisième et dernier terme de cet emprunt était échu. Les particuliers qui avaient fait ces avances aux consuls se présentent au sénat et réclament leur payement. Le sénat, qui ne pouvait méconnaître la justice de cette réclamation, mais qui se trouvait dans l’impuissance absolue d’y satisfaire, à cause de l’extrême pénurie du trésor, ayant su que ces créanciers ne seraient pas éloignés de s’accommoder de quelques terres qui faisaient partie du domaine public et qui étaient aliénables, leur fait proposer la cession d’une partie de ces terres, jusques à concurrence des sommes dues, d’après une estimation équitablement faite, et avec la clause expresse que celui d’entre ces créanciers qui préférerait son payement en argent serait admis à rétrocéder à la république le lot de terre à lui adjugé, pour en toucher l’équivalent dès que le trésor se trouvera en état de l’acquitter.

Cette proposition, très-agréable aux créanciers, est acceptée avec empressement, et \TiteLive, en rapportant ce fait, ajoute que c’est de là que le champ ainsi concédé pour l’acquit de ce dernier tiers (\latin{trientis tubula}) a conservé le nom de \latin{trienlius tabuluis ager}. (Liv. XXXI, § 13.)

On voit donc que, d’après la manière dont les peuples anciens avaient formé leur organisation : sociale, cette séparation d’intérêts qui existe chez nous entre le propriétaire foncier et le cultivateur, son fermier, toute distinction entre le produit \emphase{brut} et le produit \emphase{net} des terres, les conventions entre le maître et l’ouvrier, le contrat et les statuts d’apprentissage, les recherches sur le taux moyen des salaires et du profit des capitaux, et sur les causes qui peuvent les élever ou les abaisser, l’influence de la cherté ou du bas prix des subsistances sur le prix ou l’abondance des produits manufacturés, le change, ses variations et arbitrages, les principes de l’impôt et de sa répartition sur les différentes sources de revenu, la dette publique, les rentes, annuités et autres effets qui la représentent, les fonds à faire pour son service et son amortissement, les combinaisons et les ressources du crédit, et généralement tous les éléments dont se compose notre science de l’\emphase{économie politique} pour ce qui concerne l’accroissement de la richesse nationale et sa distribution entre les différentes classes de la société, étaient des choses totalement ignorées des philosophes anciens, non pas pour avoir échappé à leur sagacité, mais bien par une suite nécessaire de la constitution politique, et parce que les faits qui sont la matière des observations d’une telle science ne pouvaient pas se présenter à leur esprit.

La monnaie était à peu près la seule institution qui leur fût commune avec les modernes, et elle fut établie chez eux sur un système infiniment plus simple et plus raisonnable que chez nous, et la manière dont leurs philosophes ont parlé de la nature et des propriétés de cet instrument des échanges, suffit pour prouver que si les études et les méditations de ceux-ci eussent pu se diriger vers les objets qui ont occupé nos écrivains en économie politique, nous n’aurions pas, sur ce point, plus de titres à la supériorité que sur tout autre. Quel auteur moderne a donné de la monnaie une définition plus juste que celle contenue dans cette phrase d’\Aristote : C’est une marchandise intermédiaire destinée à faciliter l’échange entre deux autres marchandises ? » Les avantages d’un commerce extérieur qui se solde avec l’argent pouvaient-ils être mieux compris et mieux rendus que dans ce passage de \Xenophon dans son \publication{Traité sur les finances d’Athènes} ? « Dans la plupart des autres villes, dit-il, un marchand est obligé de prendre des marchandises en retour de celles qu’il y apporte, parce que la monnaie dont on y fait usage n’a pas grand crédit au dehors. Chez nous, au contraire, le commerçant étranger a l’avantage de trouver une multitude d’objets qui sont partout en demande, et, de plus, s’il ne veut pas encombrer son vaisseau de marchandises, il se fait solder en argent comptant, qui, de tous les articles commerçables, est le plus sûr et le plus commode, attendu qu’il est reçu en tout pays, et qu’en outre il rapporte toujours quelque profit à son maître, quand celui-ci juge à propos de s’en défaire. »

Lorsque l’empire romain fut démembré et que ses provinces furent envahies par les peuples du Nord, on ne reconnut plus dans le pays conquis de propriété privée. Le souverain du peuple conquérant était alors réputé seul propriétaire du territoire sur lequel il régnait ; il en conférait les domaines à titre de bénéfice, soit ecclésiastique, soit militaire. Ce ne fut que lorsque les seigneurs titulaires usurpèrent la propriété de leurs bénéfices, en les convertissant en hérédité masculine et de primogéniture, et lorsqu’ils établirent le régime féodal, qu’il commença à exister de nouveau dans ces pays des propriétés particulières, mais d’une nature inconnue aux ouvriers. Pendant cette longue période de troubles et d’anarchie, qui forme l’âge de la féodalité, durant laquelle il n’y eut d’autre garantie pour la sûreté des personnes et des propriétés que la voie des armes, où les routes et les marchés n’étaient sous la protection d’aucune force publique, où les marchands qui se rendaient aux foires étaient pillés, ou tout au moins rançonnés sur chaque domaine qu’ils avaient à traverser, le gouvernement royal n’était occupé qu’à se défendre contre les grands vassaux ligués contre son autorité, et qui lui disputaient tour à tour quelque portion de ses États. Ce ne fut guère qu’au seizième siècle, lorsque enfin, par la force des choses et la réunion d’intérêts entre le monarque et ses peuples, qui, comme lui, ne voulaient connaître qu’un seul maître, il s’établit dans les différentes contrées de l’Europe une forme de gouvernement plus centralisée et plus régulière, que le prince songea à se créer une source constante de revenu public, en fournissant à ses sujets tous les moyens d’accroître leur fortune particulière.

Le moyen qui sembla généralement le plus court et le plus sûr pour enrichir les particuliers, celui vers lequel se tournèrent d’abord tous les regards, ce fut le commerce étranger. C’était une opinion universellement reçue, et qui remontait même jusques aux âges de l’antiquité, que le commerce au loin était la source de richesses la plus abondante. Tous les auteurs anciens se réunissent pour témoigner que les peuples qui s’étaient livrés au commerce étaient bientôt devenus riches et puissants. Les villes de Tyr, de Sidon et de Carthage, les cités de l’Asie Mineure, les colonies grecques de l’Italie méridionale avaient dirigé de ce côté tous leurs efforts, et toujours avec succès. La politique des gouvernements de l’antiquité s’était constamment attachée à protéger les entreprises de ce genre, et à s’assurer, autant que possible, les monopoles dont ils avaient pu se prévaloir. Nous voyons dans \Strabon (liv. III), qu’un marchand phénicien se rendant aux îles Cassitérides pour y chercher du plomb et de l’étain, par une navigation qui n’était connue que des gens de sa nation, s’aperçut qu’il était suivi par un navire romain qui voulait acquérir la connaissance de cette route. Le Phénicien aima mieux se jeter sur des récifs qui brisèrent son vaisseau, pour faire périr après lui celui qui suivait sa trace, et ayant eu l’adresse de sauver sa personne, il fut largement indemnisé de sa perte par ses compatriotes, et en reçut même une glorieuse récompense.

Dans les temps modernes, les Vénitiens, les Génois, les Pisans, en suivant la même carrière, s’étaient élevés à un haut degré de puissance et de prospérité. Enfin, les Portugais qui venaient de découvrir ou de retrouver le passage aux Indes par le cap de Bonne-Espérance, étonnaient l’Europe de leurs succès, et ne purent manquer d’exciter vivement l’émulation de tous les peuples qui s’étaient déjà pourvus de quelques moyens de navigation.

Dès lors toutes les grandes nations de l’Europe, l’Angleterre, la Hollande, l’Espagne et la France regardèrent la mer qui baignait, leurs côtes comme la route infaillible qui devait les conduire à la prééminence sur tous les autres peuples. Ainsi prit naissance ce \emphase{système commercial} qui domine encore dans la politique de tous les gouvernements modernes. Croire qu’il fut le fruit de profondes méditations, de calculs habilement combinés, ce serait complètement méconnaître la manière dont se règlent les affaires publiques et dont l’administration se conduit dans sa marche. Subjuguée par les habitudes et par l’impulsion reçue, entraînée malgré elle par les agents subalternes qui la délivrent d’une grande partie de ses soins, redoutant par-dessus tout les innovations dont elle est hors d’état de bien juger les effets, considérant les vieilles routines comme consacrées par l’expérience, tant que les dommages qui en résultent ne sont pas encore d’une évidence trop frappante, elle s’abandonne par instinct à la route frayée, comme étant celle qui lui paraît la moins pénible et la moins périlleuse. Les peuples, comme les individus, sont disposés à marcher, par imitation, à la suite les uns des autres, et ceux qui les guident obéissent à ce mouvement général, loin de faire effort pour le contrarier. Par leur position et leurs rapports, ces hommes sont enclins à dédaigner la théorie et à se défier des études spéculatives, et ceux qui les entourent leur persuadent aisément que toute la science consiste essentiellement dans la pratique des affaires. On édifia donc en conséquence de ce système ; on créa successivement des compagnies privilégiées pour le commerce des Indes, pour celui du Levant, pour celui de l’Afrique, de la mer du Sud, etc. Des hommes aventureux se précipitèrent dans toutes ces entreprises, qui dévorèrent d’immenses capitaux en pure perte pour les entrepreneurs et pour le pays. Quand on s’avisa de réfléchir et de rechercher comment et par quels moyens le commerce étranger pouvait enrichir la nation qui s’y livrait, on s’arrêta à ces idées saillantes qui se présentent dès la superficie, et par là frappent tous les esprits vulgaires, et qui, pour cette raison, obtiennent toujours un grand crédit parmi la multitude.

L’économie politique est de toutes les sciences celle qui donne le plus de prise aux préjugés populaires et celle qui les trouve le plus fortement enracinés. Le désir d’améliorer sa condition, ce principe qui agit universellement et sans relâche sur tous les membres du corps social, tourne continuellement les pensées de chaque individu vers les moyens d’accroître sa fortune privée ; et si cet individu vient par la suite à élever ses pensées jusques à l’administration de la fortune publique, il sera naturellement porté à raisonner par analogie et à appliquer à l’intérêt général de son pays ces mêmes maximes que la réflexion et sa propre expérience lui auront fait reconnaître pour les meilleurs guides dans la conduite de ses affaires personnelles. Ainsi, de ce que l’argent constitue véritablement une partie essentielle du fonds productif de la fortune d’un particulier, et de ce que cette fortune se grossit évidemment à mesure que cet article vient à augmenter dans ses mains, s’est formée cette fausse opinion si généralement répandue, que l’argent est une des parties constituantes de la richesse nationale, et qu’un pays s’enrichit à proportion de ce qu’il en peut recueillir des autres pays avec lesquels il entretient des relations de commerce.

Des marchands habitués à se retirer chaque soir dans leur comptoir et à y calculer avec empressement la quantité d’argent comptant ou de bonnes créances que leur a produits la vente journalière de leurs marchandises, n’évaluent leurs profits que sur ce résultat, en quoi ils raisonnent juste. Bien certains que cette méthode ne les a jamais trompés, ils ont dû penser que les affaires de leur nation ne pouvaient pas suivre une autre marche, et ils se sont affermis dans leur idée avec cette imperturbable confiance qu’inspire une longue expérience dont on s’est parfaitement bien trouvé pour son propre compte, et qui ne s’est jamais démentie. De là cette opinion exagérée sur les avantages et les profits du commerce étranger et sur l’augmentation de la masse de numéraire dans le pays, et sur le danger de le laisser s’écouler au dehors ; de là ces calculs absurdes qui ont fait de ce qu’on appelle la \emphase{balance du commerce} le thermomètre de la prospérité publique ; de là tous ces systèmes prohibitifs et réglementaires, ces monopoles oppressifs imaginés pour grossir de plus en plus l’un des côtés de cette balance ; de là enfin, ce qui est bien plus déplorable, ces guerres sanglantes et destructives qui ont embrasé les deux hémisphères depuis l’époque où la route des Indes et celle du Nouveau-Monde sont devenues familières aux nations européennes.

Quand on observe que, depuis plus de deux siècles, tant de flots de sang versé dans les différentes parties du globe n’ont eu pour principal motif que le maintien de quelques monopoles contraires même aux véritables intérêts de la nation armée pour les défendre, on sent toute l’importance du service qu’a voulu rendre à l’humanité l’illustre auteur de \publication{la Richesse des nations}, quand il a écrit pour combattre victorieusement des préjugés aussi puissants et aussi funestes. C’était au milieu du peuple le plus profondément imbu de ces idées mercantiles, le plus fortement subjugué par leur police réglementaire, que \Smith sapait d’une main si ferme les fondements de ce système absurde et tyrannique ; c’était au moment même où l’Angleterre alarmée ne voyait qu’avec effroi la possibilité d’une séparation avec ses colonies américaines ; c’était alors que le philosophe écossais se riait de ces vaines terreurs, prédisait hautement le succès de la cause des colons et leur prochaine indépendance ; c’était alors qu’il annonçait avec confiance ce que les événements postérieurs ont pleinement confirmé, les conséquences heureuses qu’auraient pour la prospérité de l’Angleterre, comme pour celle de la colonie, cette séparation et cette indépendance tant redoutées. (Liv. IV, chap. vii.)

Un autre système qui se rattache au \emphase{système commercial}, mais qui se soutient par des moyens différents, c’est le \emphase{système manufacturier}, qui se propose de favoriser et d’encourager les manufactures du pays par toutes les mesures de contrainte qui sont au pouvoir du gouvernement, afin de faire acquérir aux produits de ces manufactures un degré de perfection ou de bas prix qui leur assure constamment la préférence dans tous les marchés étrangers, sans toutefois prétendre diminuer chez ces nations étrangères le moyen de payer ces produits avec des équivalents, ce qui eût fait manquer le but principal qu’on avait en vue.

Ce système, dont la seule énonciation montre l’absurdité, fut adopté et suivi en Angleterre avec une grande persévérance sous le règne d’Élisabeth. Les vues de la législation furent dirigées sans relâche vers cet objet. On créa des corporations et jurandes, dans lesquelles l’ouvrier n’était admis qu’après un temps prescrit d’apprentissage sous un maître privilégié, et en présentant un échantillon de son travail qui pût attester son habileté. Les agrégés aux maîtrises avaient exclusivement le droit d’exercer leur genre d’industrie, et ils étaient autorisés à faire punir quiconque se permettrait de travailler, sans leur aveu, dans le métier qui leur était réservé. Les produits des manufactures étrangères furent sévèrement prohibés, mais on laissa entrer les matières premières propres à employer les manufactures nationales ; même quand on craignit que ces matières premières ne fussent pas en assez grande abondance pour tenir en activité tous les ouvriers, il fut accordé une prime pour l’importation de ces articles. Par le même motif, les matières premières produites dans l’intérieur y furent retenues par des prohibitions de sortie et des mesures encore plus violentes. La tentative d’exporter une brebis fut un crime capital, et le simple transport des laines dans le voisinage des côtes fut soumis à la plus active surveillance. Ainsi les droits les plus respectables, ceux pour la garantie desquels l’état social est principalement institué, le droit de disposer de ses bras, de son industrie, de sa propriété, à son plus grand avantage, et comme on l’entend, tout fut sacrifié à la classe des manufacturiers incorporés, et on ne balança pas même à leur subordonner les intérêts de l’agriculture. C’était à ces manufacturiers qu’il fallait vendre, c’était d’eux qu’il fallait acheter. On ne se contenta pas encore de leur assurer la pratique de leurs compatriotes vivants, un acte du Parlement de 1678 prescrivit que les morts fussent ensevelis dans une étoffe de laine.

Le résultat de ce système fut sans doute une très-grande accumulation de richesses matérielles ; mais quels devaient être à la longue les effets d’une telle politique sur la population, la force et la puissance réelle de la nation manufacturière, relativement à celles des pays avec lesquels elle entretenait des relations de commerce ?

L’industrie anglaise, forcée, par les bornes étroites de son territoire, d’économiser le nombre des bras qu’elle salarie, a tourné tous ses efforts vers la recherche des moyens propres à rendre le travail manufacturier plus productif. Une division du travail très-habilement distribuée et un grand nombre de machines ingénieuses ont donné au travail de cette nation une supériorité marquée sur celui des autres peuples, en sorte que dans les échanges qu’elle lit avec l’étranger, il fut ordinaire que le produit d’une journée de son travail se trouvât être l’équivalent du produit de deux ou trois journées d’un autre. On sent combien, dans de telles opérations, elle dut gagner sur la valeur qu’elle recevait en échange, sans que le peuple avec lequel elle traitait éprouvât pour cela aucune perte, puisque la chose que celui-ci recevait de l’Angleterre valait en réalité pour lui le nombre de journées qu’elle lui eût coûté à faire, s’il l’eût fabriquée lui-même. Mais pour que les Anglais obtiennent ces grands bénéfices, il faut qu’ils échangent du produit manufacturé contre du produit brut ; aussi repoussent-ils le plus qu’ils peuvent tout produit manufacturé par des mains étrangères, et ne demandent-ils aux autres que des produits bruts. Or, ce dernier genre de produit ne peut se multiplier dans un pays qu’avec l’aide d’une nombreuse population, et encourager dans ces pays, par des demandes, la multiplication des produits bruts, c’est nécessairement y encourager de la manière la plus efficace la culture et la population. Donc les transactions commerciales que fait l’Angleterre avec les autres nations tendent à encourager chez celles-ci la multiplication des hommes et des subsistances, tandis que ces mêmes transactions produisent un effet tout contraire dans son intérieur, en excitant de plus en plus les fabricants à manufacturer le plus de produits bruts possible avec le plus petit nombre possible de bras. Cette direction forcée de l’industrie humaine nuit à la population d’un pays sous un double rapport ; d’abord en dégradant les facultés intellectuelles de l’ouvrier qui se trouve réduit au mouvement uniforme et continu d’une simple machine, et secondement en diminuant de plus en plus le nombre des ouvriers entretenus par l’industrie nationale. De ces deux effets nuisibles, le premier a été remarqué plusieurs fois ; l’autre, qui l’a moins été, paraîtra encore plus sensible par un exemple.

Je suppose donc qu’un fabricant de couteaux de Sheffield ait quarante ouvriers, qui, au moyen d’une habile distribution des tâches et le secours des machines, viennent à bout de faite par jour dix douzaines de couteaux qui se vendent en France une guinée ou 25 fr. la douzaine. En travaillant vingt-cinq jours par mois, le produit de cette fabrique sera de 72\espacegroupementchiffres000 \franc par an, dont un tiers ira au salaire des ouvriers, et les deux autres tiers, après avoir remplacé au fabricant ses matières premières, lui donneront le surplus pour intérêts et profits des capitaux fixe et circulant mis dans son entreprise. Que le produit de cette manufacture soit échangé en France contre une valeur égale en blés, formant environ la quantité de 9000 quintaux (lesquels n’entreront vraisemblablement jamais en Angleterre et seront un objet de spéculation pour quelque autre négociant anglais auquel le fabricant de Sheffield les cédera, et serviront à la consommation de quelque pays étranger), le résultat final de cette opération, quelque profitable qu’elle soit pour le fabricant anglais, sera extrêmement peu avantageux à l’accroissement de la puissance de sa nation. Cet emploi de l’industrie et du capital anglais aura fait subsister quarante ouvriers et l’entrepreneur de l’ouvrage ; mais l’emploi correspondant d’industrie et de capital qui aura travaillé en France à fournir un équivalent, y aura fait subsister au moins 3000 personnes : car, pour pouvoir disposer de 9000 quintaux de blé, il a fallu nécessairement en faire produire à la terre environ trois fois autant.

Par la nature même de l’industrie anglaise et par la direction forcée que lui impriment les circonstances les plus impérieuses, les capitaux productifs doivent se porter naturellement par préférence vers le commerce étranger, le moins avantageux de tous pour le pays, et principalement au commerce du produit manufacturé contre le produit brut, celui de tous les commerces étrangers le plus nuisible à la population et à la puissance réelle du peuple qui s’y livre, puisque, en dernière analyse, ce commerce n’est qu’une lutte dans laquelle celui-ci s’efforce d’obtenir la plus grande quantité de produits avec le moindre emploi possible d’hommes et de terre. Par l’extension que provoque ce genre de combat, dans la population et la culture des autres peuples, un tel commerce doit naturellement aller toujours en croissant ; aussi, chaque année, l’Angleterre a-t-elle importé une plus grande somme de produits bruts qu’elle a renvoyés manufacturés, ce qui a grossi annuellement la quantité numérique de ses exportations et de ses importations, à l’inexprimable contentement de ses spéculateurs politiques.

En définitive donc, l’Angleterre travaille constamment à multiplier chez ses rivaux les hommes et les produits bruts, les deux principaux éléments de richesse et de puissance qui ont toujours fini par assurer la domination au peuple qui les a possédés au plus haut degré, et qui, dans tous les temps, ont décidé en dernier ressort du destin des nations.

Chez une nation, au contraire, qui est foncièrement riche, mais qui se trouve épuisée par de longues guerres ou par des dissensions intestines, l’industrie nationale est comme ces substances chimiques qui ont été privées du principe avec lequel elles ont le plus d’affinité ; plus elles ont été dépouillées, plus elles le saisissent avec avidité et s’en emparent rapidement tout autour d’elles, jusqu’à ce qu’elles en soient saturées et qu’elles arrivent au degré de combinaison déterminé par la nature. C’est avec cette activité dévorante que l’industrie française, aussitôt qu’elle s’est retrouvée dans une atmosphère de calme et de sécurité, a repris tout ce qu’une longue suite de troubles civils lui avait enlevé ; tandis que celle de ses voisins, surchargée de capitaux au delà de ce qu’elle en peut absorber dans le cours naturel des choses, est au point où commencent à se faire sentir le ralentissement et le déclin.

Les nations qui ont eu le plus de relations commerciales avec l’Angleterre ont ressenti chez elles tous les effets salutaires attachés à un commerce dans lequel sont toujours demandés et payés des produits bruts. La Russie, en travaillant pour fournir à l’Angleterre des cargaisons de chanvre, de suif, de cire, de goudron, de peaux, de bois de construction, etc., a travaillé en même temps pour accroître sa culture et sa population : aussi, depuis environ cinquante ans que ce commerce a été entretenu sans interruption, la population de l’empire russe a quadruplé de ce qu’elle était auparavant.

\Colbert\footnote{L’habileté de \Colbert est encore aujourd’hui un problème, parce qu’elle est exaltée et ravalée par les écrivains de nos jours, selon l’opinion qu’ils ont à soutenir. Pour bien juger des talents et de la capacité d’un ministre, il faut consulter les actes de son administration. Le marché que fit \Colbert, en 1674, pour faire fabriquer des pièces d’argent de 4 sous à un titre d’un 5e de fin au-dessous des écus d’argent et des quarts d’écu dont elle s’étaient des coupures, est une opération qui décèle la plus profonde ignorance des premiers principes du régime des monnaies, l’une des parties les plus importantes du ministère des finances. On peut voir dans le \publication{Traité historique des monnaies}, par \Leblanc (pag. 308 et suiv., édition d’Amsterdam), tous les détails de cette affaire ; les réclamations bien motivées qui furent présentées au ministre, et dont il ne tint aucun compte ; les abus qu’entraîna cette mauvaise mesure, sur laquelle on ne revint que lorsqu’elle eut causé d’énormes pertes au public et au trésor, au profit de quelques traitants.}, doué d’une infatigable activité, et impatient d’ajouter un nouvel éclat à la gloire et à la puissance de son maître, se laissa séduire par les illusions du système manufacturier. Ce ministre employa tous les moyens qui étaient à sa disposition pour attirer vers le travail des manufactures une partie du capital français, qui se trouva ainsi détournée de la pente naturelle qui la portait vers le progrès de la culture et l’amélioration des terres. Il voulut, à force de règlements et de mesures de gouvernement, hâter une maturité dont l’époque n’était pas encore venue, et qu’il eût été plus sage d’attendre. Il renchérit même sur le système anglais, qui commençait déjà à se relâcher sur quelques points, car les nouveaux métiers qui s’étaient introduits postérieurement aux statuts d’Élisabeth ne furent point assujettis aux entraves de la maîtrise. Ce fut \Colbert qui imagina de prescrire aux manufacturiers, par des ordonnances, jusques aux procédés de la fabrication, la largeur des étoffes, le nombre de fils à observer dans la chaîne et dans la trame, et jusques aux plus petits détails de leur ouvrage. L’activité actuelle de nos manufactures, dans tous les genres, démontre assez que l’industrie française pouvait se passer de ces encouragements extraordinaires, et l’admirable perfection qu’acquièrent tous les jours les produits de nos fabriques témoigne toute l’inutilité de cette police réglementaire et minutieuse, qui avait la prétention d’enseigner à nos fabricants comment ils devaient travailler pour plaire aux consommateurs et s’assurer un débit avantageux.

Pour les hommes chargés de la direction des affaires publiques, la tâche la plus difficile, comme l’a observé \Smith, c’est de s’abstenir de ce qui ne les concerne pas, de laisser le travail et l’industrie suivre en liberté leur pente naturelle, et de se borner à les couvrir de cette protection impartiale qui est la seule faveur qu’ils attendent du Gouvernement.

Après avoir suivi pendant longtemps le \emphase{système commercial} et le \emphase{système manufacturier}, la législation anglaise s’attacha au \emphase{système agricole}. Comme tous ces différents systèmes consistent à sacrifier une portion de la liberté et de la fortune des sujets, pour favoriser une classe particulière d’agents de l’industrie, ils s’excluent nécessairement l’un l’autre. On ne peut pas attirer par force, vers un genre de travail, plus de capital qu’il ne s’y en fût porté dans l’état naturel des choses, sans arracher cette portion de capital à l’espèce d’industrie qui l’aurait appelé, car toutes ces mesures systématiques n’augmentent pas la masse du capital national, et c’est même parce qu’elles en supposent l’insuffisance, qu’elles s’efforcent de changer sa direction et de la déterminer dans d’autres proportions que celles qui eussent eu lieu si le gouvernement eût laissé faire. Ainsi, adopter le \emphase{système agricole}, c’était vouloir attirer à la culture et à l’amélioration des terres une portion du capital employé dans les entreprises de commerce et de manufactures ; c’était, jusques à un certain point, abandonner les deux autres systèmes.

Vers la fin du dix-septième siècle, le parlement d’Angleterre jugea à propos d’encourager la culture et l’amélioration des terres, au moyen de gratifications accordées à l’exportation des grains indigènes dans les pays étrangers.

Cette mesure ne fut pas sans effet, puisque \Smith nous dit que, d’après les registres des douanes, la quantité de grains de toute espèce, exportés pendant les dix années qui s’écoulèrent de 1741 à 1750, a monté à plus de huit millions de anglais, et que la somme des gratifications payées pour cet objet a donné lieu à une dépense de 1500 mille livres sterling. Il ajoute qu’en 1749, M. \Pelham, alors premier ministre, déclara à la Chambre des communes qu’il avait été dépensé, dans les trois années précédentes, une somme exorbitante en gratifications pour exportations de grains, et qu’enfin l’année suivante (1750), la somme payée pour cet article excéda 334\espacegroupementchiffres000 \livresterling, c’est-à-dire, plus du double de l’année moyenne de cette période de dix années.

Ce dernier système du moins n’était pas de nature à entraîner des conséquences aussi graves que les deux autres, et tout le dommage qu’il pouvait occasionner se bornait à une dépense inutile pour le trésor et à un renchérissement artificiel du blé, qui fit payer au peuple sa subsistance un peu plus cher qu’il ne devait la payer si la gratification n’était pas établie. L’exportation des grains, provoquée par cette mesure, fit naître dans les marchés intérieurs une rareté qui ne se fût pas fait sentir si les quantités exportées se fussent montrées dans ces marchés.

Quant au but que se propose ce système, d’encourager la culture et d’augmenter la masse totale de la production, rien n’est certainement plus illusoire. Toute terre qui, à la récolte, donnera, année commune, une plus grande quantité de grains que celle qui a été avancée pour la semer et pour entretenir les ouvriers de la culture, sera nécessairement exploitée avec profit, et elle ne restera pas inculte, sans qu’il soit besoin d’exciter le propriétaire ou le fermier par l’appât de gagner une prime en argent ; et toute terre qui, étant cultivée, ne rendra pas, année commune, plus que la quantité dépensée pour la semer et pour nourrir les ouvriers, ne pourra être exploitée qu’avec perte, tant pour le particulier qui la possède ou qui la cultive, que pour le pays dont elle fait partie ; et la gratification qui aurait l’effet de faire mettre en culture une terre aussi ingrate, ne ferait qu’ajouter une perte de plus à une entreprise déjà ruineuse par elle-même. La hausse du prix du blé en argent ne peut pas faire mettre en culture un pouce de terre de plus, quand cette hausse n’est pas l’effet d’un déficit réel. On ne produit du blé qu’avec du blé ; et si, par des moyens factices, vous parvenez à faire renchérir le prix du blé de la récolte, vous aurez fait également renchérir le prix du blé employé à la semence et à la nourriture des ouvriers de la culture. De quelque manière que l’on s’y prenne, ce sera toujours la qualité de la terre et son degré de fertilité qui décideront s’il peut y avoir profit ou non à la mettre en culture. « La nature, dit \Smith, a imprimé au blé sa valeur. Il n’y a pas de monopole pour la vente au-dedans, pas de gratification pour l’exportation qui aient la puissance de hausser cette valeur, comme la concurrence la plus libre ne saurait la faire baisser. »

Enfin, ce système agricole a été totalement abandonné, et l’Angleterre, qui avait payé des gratifications pour l’exportation de ses propres blés, s’est vue obligée, sur la fin du siècle dernier, de donner de très-fortes primes pour encourager importation des blés étrangers.

Il faudrait avoir une foi bien implicite dans la sagesse de ceux qui dirigent l’administration de la fortune publique pour croire que, dans cette variation continuelle de systèmes de conduite, ils ont été guidés par un jugement bien solide et convenablement éclairé par la maturité de la réflexion et par les leçons de expérience.

Après ce coup d’œil rapide sur les divers systèmes qui ont été suivis par les hommes investis de la haute et importante fonction de diriger la marche du gouvernement, voyons maintenant quel a été le fruit des méditations des philosophes qui se sont occupés de la théorie de l’économie politique, en ce qui touche la formation et la distribution des richesses. Ce rapprochement mettra le lecteur en état de juger jusqu’à quel point les premiers ont été bien fondés dans le dédain qu’ils ont presque toujours affecté pour l’instruction qu’ils auraient pu puiser dans les écrits des autres.

\section{Exposé sommaire de la doctrine de \Smith, comparée avec celle des économistes français}

Dès le seizième et le dix-septième siècle il parut, tant en France qu’en Angleterre, divers écrits sur les finances, sur l’impôt, sur l’importance relative de l’agriculture et du commerce, et sur plusieurs autres objets d’administration publique. Les désordres que jeta la banque de Law dans la fortune de l’État et dans la plupart des fortunes privées, tournèrent l'attention des esprits spéculatifs vers des matières dans lesquelles tant de personnes se trouvaient intéressées. On écrivit sur la circulation, sur le crédit, sur l'industrie, la population, les effets du luxe, etc. Les connaissances étaient déjà assez avancées pour que, dans cette dernière question, si délicate et si controversée, on ait su faire le départ de ce qui appartenait à l’économie politique d’avec ce qui était purement du ressort de la morale.

Ce fut vers cette époque qu’une réunion de philosophes français, qui ont été désignés depuis sous le nom d’\emphase{économistes}, s’appliqua à la recherche des principes de la formation des richesses et de leur distribution naturelle entre les différentes classes de la société ; et ces hommes, distingués pour la plupart par de rares talents et de vastes connaissances, furent les premiers qui formèrent un corps complet de doctrine sur cette branche de l’économie politique. 

Ces philosophes observèrent :

1° Que toutes les richesses provenaient d’une source unique, qui était la terre, puisque c’était elle qui fournissait à tous les travailleurs leur subsistance et les matériaux ou produits bruts sur lesquels ils exerçaient leur industrie.

2° Que le travail appliqué à la culture de la terre produisait non-seulement de quoi s’alimenter lui-même pendant toute la durée de l’ouvrage, mais encore un surplus de produit après le remplacement de toute la dépense, surplus qui ajoute conséquemment à la masse des richesses déjà existantes, et qui formait un \emphase{produit net} nécessairement dévolu au propriétaire de la terre, et constituait entre ses mains un revenu pleinement disponible. Qu’au contraire, le travail appliqué aux productions détachées de la terre, ce qui comprend le travail des manufactures et du commerce, ne pouvait rien ajouter aux choses sur lesquelles il s’exerçait ; que la valeur additionnelle qu’elles recevaient de ce travail n’était que l’équivalent du salaire plus ou moins élevé des ouvriers ou de l’entrepreneur qui avait dirigé le travail, et par conséquent de ce que ces ouvriers on ce directeur d’ouvrage avaient consommé ou avaient eu le droit de consommer pendant la durée de cet ouvrage, en sorte qu’après le travail achevé, la somme totale des richesses existantes dans la société était précisément la même qu’auparavant, à moins que les ouvriers ou le directeur de l’ouvrage n’aient mis en réserve une partie de ce qu’ils avaient le droit de consommer. Les économistes en tirèrent la conséquence que le travail appliqué à la terre était le seul travail productif de richesses, et que le travail des artisans, manufacturiers ou commerçants, tout utile, tout indispensable même qu’il était, devait néanmoins être considéré, relativement à l’autre, comme un travail \emphase{stérile}, puisqu’il ne servait nullement à accroître la somme des richesses, et que cette classe d’ouvriers, ne produisant aucune richesse nouvelle. ne pouvait concourir à l’augmentation de la masse des richesses que par ses privations et ses économies.

D’après ces principes, ils établirent que les propriétaires de la terre recueillaient, en première ligne, la totalité des richesses produites ; que les non-propriétaires ne pouvaient consommer que ce qu’ils recevaient directement ou indirectement des propriétaires ; que, par conséquent, les non-propriétaires, quelles que fussent l’utilité et l’éminence de leurs services, n’étaient que les salariés des propriétaires. et que la circulation des richesses dans la société portait sur une suite continuelle d’échanges entre ces deux classes, l’une donnant son travail, son industrie et ses services, et l’autre distribuant son revenu disponible, en salaires et en récompenses.

Enfin, ils en déduisirent que l’impôt étant une portion de la richesse disponible appliquée aux services publics, de quelque manière qu’il fût perçu, était toujours, en définitive, supporté par les propriétaires fonciers, comme étant les distributeurs en chef de toutes les richesses disponibles ; qu’ainsi cette charge les frappait seuls, soit en prenant directement dans leurs mains une portion du revenu applicable à leurs jouissances, soit indirectement, en renchérissant le prix des services et en grevant les propriétaires d’un surcroît de dépense dans les salaires et récompenses qu’ils avaient à payer ; qu’ainsi tout impôt qui n’était pas immédiatement prélevé sur le produit net de la terre retombait finalement sur le propriétaire foncier, avec encore une surcharge de frais qui était en pure perte pour l’État.

L’intérêt général de toutes les classes de la société était de multiplier autant que possible les produits agricoles ; les propriétaires y trouvaient une augmentation de leur revenu disponible, les cultivateurs une source plus abondante de profits ; les artisans, manufacturiers et commerçants trouvaient aussi dans la masse toujours croissante des subsistances qui leur étaient destinées et des matières premières sur lesquelles s’exerçait leur travail, des moyens de travailler davantage et de vivre avec plus d’aisance, le bon marché des vivres et des produits bruts provoquant généralement une plus grande consommation de tous les articles de manufacture et de commerce.

D’après cette doctrine, l’industrie manufacturière et commerçante, dégagée de tout impôt, délivrée de toute espèce de contrainte, encouragée par le bon marché et l’abondance toujours croissante des subsistances et des matières premières, prendrait nécessairement un tel essor, qu’elle ne pourrait redouter dans l’intérieur aucune concurrence, et que même elle pourrait offrir à l’extérieur ses produits à si bon marché, qu’on obtiendrait aux meilleures conditions possibles les marchandises étrangères que le pays désirerait de consommer.

C’est ainsi que les économistes décrivaient le cours naturel que devait suivre le progrès de la richesse publique, en la laissant se développer en pleine liberté. Lorsqu’ils représentaient la propriété foncière comme source de toutes les richesses, ils ne sollicitaient pour elle ni faveur, ni privilège quelconque ; au contraire, l’impôt, réduit à un seul mode d’assiette, était une charge réservée uniquement aux propriétaires fonciers. Les seuls privilèges qui restaient à ceux-ci étaient ceux qu’ils tenaient de la nature même et des principes de l’ordre social. On ne demandait au gouvernement que de ne pas contrarier le cours des choses et de ne pas mettre d’obstacles au progrès naturel vers lequel elles doivent marcher d’elles-mêmes. Tous les règlements que l’on sollicitait en faveur de industrie et du commerce étaient contenus dans ces quatre mots : \emphase{laissez faire, laissez passer}.

Cette doctrine, si simple dans son exposition, si généreuse dans ses résultats, eut de zélés partisans et d’ardents adversaires. Elle fut embrassée par les hommes d’État les plus éclairés de cette époque, \Turgot, \Trudaine, \Gournay, \Malesherbes, \Lavoisier, de \Jaucourt, Condorcet, \Raynal, \Dupont, \Morellet, \Letrosne, etc. Parmi ses détracteurs, qui furent très-nombreux, il serait difficile de citer un nom de quelque poids. \Smith, qui a été le plus redoutable adversaire de la doctrine des économistes, puisqu’il l’a anéantie, n’en parle qu’avec les plus grands égards, comme d’un système aussi noble que savant et ingénieux, rempli de vues saines et droites, et celui de tous qui s’est le plus approché des vrais principes de la matière.

Mais cette doctrine de pleine liberté jeta l’alarme parmi les traitants, les fermiers des revenus publics et leurs innombrables préposés. Elle blessa l’orgueil des ministres et les prétentions non moins exigeantes des premiers commis, dont elle semblait réduire à rien le profond savoir et la longue pratique des affaires. Les négociants et les gros manufacturiers s’indignaient de ce que leurs professions étaient flétries du nom de \emphase{stériles}. Les propriétaires fonciers eux-mêmes jetèrent de hauts cris contre l’impôt unique dont on proposait de les charger. Il n’y eut donc pas un intérêt, pas un préjugé, pas une passion qui ne se soulevât contre les économistes, et, à défaut de raisonnements, on attaqua les doctrines avec des pamphlets et des satires.

Les idées spéculatives présentées par ces philosophes, quoique difficiles à contester au fond, s’accordent toutefois si peu avec le train des affaires humaines, elles offrent une application si éloignée aux intérêts de la société, tels que le monde les comprend, qu’elles ne furent généralement accueillies par l’opinion publique que comme d’ingénieuses rêveries. D’ailleurs, les maîtres de cette école crurent devoir forger une langue technique pour exprimer avec précision des maximes qui étaient neuves, ce qui ouvrit à la critique une large voie pour jeter sur leurs leçons les traits du ridicule, arme bien plus meurtrière que tout le feu de la dispute.

Peu d’années après la publication de la doctrine économique, un homme doué du génie le plus vaste et le plus pénétrant, qui avait médité et approfondi les vérités découvertes par les économistes, conçut le projet d’en faire une application utile et sensible pour tout le monde, en les rattachant à l’intérêt national de son pays, et en leur donnant tout le développement dont elles sont susceptibles sous ce nouveau rapport.

Les économistes français avaient recherché l’origine et la marche de la formation des richesses parmi les hommes en général. \AS s’attacha particulièrement à la recherche des causes du progrès des richesses parmi les nations. En paraissant ainsi circonscrire son sujet, il lui fit prendre une dimension beaucoup plus étendue. Il vit que les nations s’enrichissaient non-seulement par la multiplication des richesses produites sur leur propre sol, mais plus rapidement encore par les échanges les plus avantageux possible avec les autres nations. C’est ce que l’auteur annonce des la première phrase de son livre, en disant que le \emphase{travail annuel} d’une nation est la source primitive d’où elle tire toutes ses richesses, et que ces richesses sont toujours ou le produit immédiat de son travail, ou achetées des autres nations avec ce travail.

Cette considération le porta à examiner la nature du travail, sa puissance et ses effets ; il rechercha les causes qui ajoutent à ses produits. Il montra dans le travail la mesure universelle et invariable des valeurs ; il fit voir que toute chose vénale avait son prix naturel, vers lequel elle gravitait sans cesse au milieu des fluctuations continuelles du prix courant, occasionnées par des circonstances accidentelles étrangères à la valeur intrinsèque de la chose. Ensuite, pour expliquer les causes de ces fluctuations passagères, il analysa avec une merveilleuse sagacité les éléments divers dont se compose le prix de toute chose échangeable, en indiquant les variations dont chacun de ces éléments était susceptible. Toutes ces importantes vérités, appuyées sur des exemples familiers, étaient d’une application facile et prochaine aux circonstances actuelles de nos sociétés, et les intérêts de toutes les classes purent y puiser d’utiles leçons.

Les services que ce grand homme rendit à son pays et à tous les peuples civilisés sont inappréciables, mais on ne peut se dissimuler que sa route lui fut indiquée par les économistes français. Ceux-ci avaient habilement creusé un terrain que personne n’avait su défricher avant eux ; \AS est le premier qui ait su lui faire porter des fruits.

En effet, si l’on médite avec attention la doctrine des économistes, on reconnaîtra que le côté faible de ce système, c’est d’avoir trop peu apprécié toute l’influence du travail des arts et manufactures sur la multiplication des richesses. D’après ce qu’ils enseignent, une nation ne pourrait parvenir à un haut degré de prospérité et d’opulence que par une route longue et difficile, qui suppose une persévérance dont les affaires humaines et surtout l’administration publique sont peu susceptibles. L’impôt unique sur les terres, l’affranchissement complet de l’industrie et du commerce de toute entrave, de toute charge étrangère, l’abondance et le bon prix des vivres et des matières premières résultant de ce nouvel ordre de choses, étaient les conditions préalables pour amener une nation à cet état d’aisance et d’activité qui devait lui assurer les moyens de braver la concurrence de toutes les nations étrangères.
Le mot \emphase{richesse} n’est point entendu par \Smith dans le même sens que par l’école de \Quesnay. Celle-ci semble l’appliquer exclusivement aux produits que la terre multiplie et qu’en suite le travail humain modifie, prépare et dispose pour la consommation. Dans la définition de \Smith, les richesses sont toutes les choses propres à satisfaire les besoins ou à procurer à l’homme des commodités et des jouissances. Cette dernière définition s’accorde mieux avec le train habituel de la vie. Nos richesses sont tout ce qui sert à nous nourrir, nous vêtir, nous loger d’une manière plus ou moins agréable et commode, ce qui suppose des produits que l’art a façonnés pour ces différents usages.

En considérant les richesses sous cet aspect, qu’importe que le travail, appliqué à la culture de la terre, produise au delà de ses propres frais des êtres nouveaux qui n’eussent pas existé sans lui, et qu’il ait ce genre d’avantage sur le travail des manufactures et du commerce ? S’ensuit-il pour cela que cette première espèce de travail sera, dans tous les temps, plus profitable que l’autre à la société ? Ce qui constitue véritablement une richesse et ce qui en détermine la valeur, c’est le besoin du consommateur. Il n’existe point de richesse proprement dite ni de valeur absolue. Ces deux mots \emphase{richesse} et \emphase{valeur} ne sont que des mots corrélatifs de ceux-ci : \emphase{consommation} et \emphase{demande}, quoique cette relation, comme nous l’avons observé plus haut, soit sujette à éprouver des variations accidentelles et momentanées, soit d’un côté, soit de l’autre. Même ce qui est propre à nourrir l’homme n’est point une richesse dans un pays inhabité et inaccessible, et à quelque degré que la civilisation soit parvenue, le principe reste le même. Si la masse des richesses vient à excéder la somme des besoins, dès lors une partie de cette richesse cessera d’être richesse et rentrera dans la classe des êtres sans valeur. Vainement donc l’agriculture multipliera ses produits ; au moment où ils dépasseront les besoins de la consommation actuelle, une partie de ces produits sera sans valeur, et l’intérêt privé, ce premier régulateur de la direction du travail et de l’industrie, se voyant trompé dans ses spéculations, ne manquera pas de tourner d’un autre côté son activité et ses efforts.

Distinguer le travail des ouvriers de l’agriculture d’avec celui des autres ouvriers, est une abstraction presque toujours oiseuse. Toute richesse, dans le sens dans lequel nous la concevons, est nécessairement le résultat de ces deux genres de travail, et la consommation ne peut pas plus se passer de l’un que de l’autre. Sans leur concours simultané il ne peut y avoir de chose consommable, et par conséquent point de richesse. Comment pourrait-on donc comparer leurs produits respectifs, puisque, en séparant ces deux espèces de travail, on ne peut plus concevoir de véritable produit, de produit consommable et ayant une valeur réelle ? La valeur du blé sur pied résulte de l’industrie du moissonneur qui le recueillera, du batteur qui le séparera de la paille, du meunier et du boulanger qui le convertiront successivement en farine et en pain, tout comme elle résulte du travail du laboureur et du semeur. Sans le travail du tisserand, le lin n’aurait pas plus le droit d’être compté au nombre des richesses, que l’ortie ou tout autre végétal inutile. À quoi pourrait-il donc servir de rechercher lequel de ces deux genres de travail contribue le plus à l’avancement de la richesse nationale ? N’est-ce pas comme si l’on disputait pour savoir lequel, du pied droit ou du pied gauche, est plus utile dans l’action de marcher ? 

Les ouvriers des manufactures n’ajoutent à la chose sur laquelle ils exercent leur industrie qu’une valeur précisément égale à ce qu’ils ont consommé ou pu consommer pendant la durée de l’ouvrage. Cette observation est juste ; mais que peut-on en conclure ? Qu’il s’est opéré une sorte d’échange au moyen duquel les aliments consommés par les ouvriers se trouvent représentés par l’augmentation de valeur résultant de la main-d’œuvre, en sorte que la laine, par exemple, convertie en drap ou en tricot, a gagné justement en valeur, dans cette mutation de forme, ce qu’a dépensé ou pu dépenser l’ouvrier employé à ce travail ; mais s’il est démontré que, sans cet échange, la laine fût restée sans valeur, et que, d’un autre côté, les vivres et autres objets fournis à l’ouvrier pour salaires fussent demeurés sans consommateur, il s’ensuit que cet échange a produit le même effet que s’il eût créé ces deux valeurs, et qu’il a été pour la société une opération infiniment plus avantageuse que si pareille quantité de travail eût été employée à multiplier des produits bruts déjà surabondants. Le premier travail a été vraiment productif ; l’autre aurait été, dans la réalité, stérile, puisqu’il n’en serait pas résulté de valeur.

La terre, ont dit les économistes, est la source de toutes les richesses ; mais pour que cette proposition ne conduise pas à de fausses conséquences, il est nécessaire de l’expliquer. C’est dans le sein de la terre que commencent toutes les richesses ; c’est le travail qui les achève et qui complète leur valeur en les rendant consommables. La terre ne fournit jamais que la matière avec laquelle se forment les richesses, et celles-ci n’existeraient pas sans la main industrieuse qui modifie, divise, assemble, combine les diverses productions de la terre pour les rendre propres à nos usages. Dans le commerce, il est vrai, ces productions encore brutes sont évaluées comme véritables richesses ; mais il ne faut pas perdre de vue qu’elles doivent cet avantage à la certitude qu’a toujours le possesseur d’en faire, à sa volonté, des choses consommables, en les soumettant aux divers degrés de main-d’œuvre qui leur sont nécessaires. Elles n’ont donc qu’une valeur virtuelle, comme celle d’un billet de banque, qui passe comme argent comptant parce que le porteur est assuré de le convertir en espèces réelles quand il lui plaira. La terre recèle des mines d’or et d’argent bien connues, qui ne sont pas exploitées, parce que le produit n’en couvrirait pas la dépense. Ces métaux sont, au fond, de la même nature que ceux dont nos monnaies sont fabriquées ; cependant, comme il n’y a nulle probabilité qu’ils soient jamais extraits de la mine qui les contient, ils n’ont aucune espèce de valeur, et il serait absurde de les compter au nombre de nos richesses. L’oiseau sauvage devient une richesse au moment où l’adresse du chasseur l’a fait tomber en son pouvoir ; celui qui échappe n’en est pas une.

Il est encore de toute évidence que quiconque n’est pas propriétaire foncier ne peut subsister que de salaires reçus directement ou indirectement de la main des propriétaires ; il n’y a que les voleurs qui fassent exception ; et les services les plus honorables, comme les plus vils, sont là cet égard dans la même catégorie. Il est encore certain que les circonstances qui ont déterminé le taux de ces divers salaires étant supposées toujours les mêmes, c’est-à-dire, les offres et les demandes de services restant entre elles dans la même proportion, après l’impôt comme auparavant, dans ce cas les salaires seront nécessairement aussi payés sur le même taux, et par conséquent l’impôt, de quelque manière qu’il soit établi, retombera toujours, en dernière analyse, exclusivement sur la classe qui fournit les salaires, et, après l’impôt, cette classe aura à subir, par suite de l’impôt, ou une augmentation dans ses dépenses, ou un retranchement dans les jouissances qu’elle pouvait se procurer auparavant. Cette charge sera d’autant plus forte, que la perception de l’impôt s’écartera davantage de la ligne directe, parce qu’il y aura à supporter, en sus de l’impôt, les frais et indemnités de tous les intermédiaires qui en auront fait l’avance, et la dépense du plus grand nombre d’agents employés à cette perception. La théorie conduit donc nécessairement à conclure que l’impôt directement perçu sur le revenu net du propriétaire foncier est l’impôt le plus conforme à la raison et à la justice, le moins onéreux aux contribuables et le plus profitable au trésor.

Mais si cette théorie fait abstraction d’une foule de circonstances morales qui ont une grande influence sur la facilité de la perception et même sur les conséquences de l’impôt, et si les inconvénients qui résultent de cette influence l’emportent de beaucoup sur l’avantage unique d’une charge moins forte, alors la théorie ne se composant pas de tous les éléments qui entrent dans la pratique, se trouve nécessairement démentie par celle-ci. Or, c’est précisément ce qui se rencontre dans la question où il s’agit de comparer les avantages et les inconvénients des deux modes de perception de l’impôt.

L’habitude qu’ont les hommes de voir dans l’argent la représentation de toutes les choses qui servent au soutien ou à l’agrément de la vie, leur fait naturellement contracter une extrême répugnance à se défaire de l’argent qu’ils possèdent, à moins qu’il ne s’agisse de pourvoir à un besoin ou de se procurer une jouissance. On dépense avec plaisir, mais il faut un effort pour payer une dette ; et celle qui coûte le plus à acquitter, parce que la valeur reçue en échange est moins aperçue et moins sensible pour tout le monde, c’est l’impôt. En attachant l’impôt à la chose consommable, en le confondant dans le prix de celle-ci, en faisant que le payement de la dette et la jouissance soient un seul et même acte, on fait en quelque sorte participer l’impôt à l’attrait que porte avec soi la consommation, et l’on fait naître dans l’esprit du consommateur le désir d’acquitter l’impôt. C’est au milieu de la profusion des repas que se payent, les taxes sur le vin, la bière, le sucre, le sel et les articles de ce genre, et le trésor public trouve une source de gain dans les provocations à la dépense qui sont excitées par l’abandon et la gaieté des fêtes.
Un autre avantage de même nature en faveur de l’impôt indirect ou de consommation, c’est son extrême divisibilité et la facilité donnée au contribuable de s’acquitter jour par jour et même d’une minute à l’autre. L’artisan qui soupe d’une partie du salaire de sa journée, satisfait quelquefois en un quart d’heure à quatre ou cinq payements divers de l’impôt.

Dans la perception directe, l’impôt se montre sans nul déguisement ; il vient sans être attendu, à cause de l’imprévoyance si ordinaire au commun des hommes, et il apporte toujours avec lui de la gêne et du découragement.

Toutes ces considérations sont négligées par les partisans de la perception directe ; et cependant, quiconque a réfléchi sur l’art de gouverner les hommes, peut juger de ce qu’elles ont d’importance.

Mais ce n’est peut-être pas tout encore. L’impôt indirect, en ajoutant successivement un surcroît de prix aux articles de consommation générale et journalière, au moment où tous les membres de la société ont contracté l’habitude de ces consommations, rend ces divers articles un peu plus coûteux à acquérir, c’est-à-dire qu’il donne lieu à ce qu’il faille, pour se les procurer, un surcroît proportionné de travail et d’industrie. Or, si cet impôt est mesuré de manière à ne pas aller jusqu’à décourager la consommation, ne semble-t-il pas, dans ce cas, agir comme un stimulant universel sur la partie active et industrieuse de la société, qui l’excite à un redoublement d’efforts, pour n’être pas obligée de renoncer à des jouissances que l’habitude lui a rendues presque nécessaires, et qui, en conséquence, donne un plus grand développement aux facultés productives du travail et aux ressources de l’industrie ? Ne doit-il pas en résulter qu’après l’impôt, il y à la même somme de travail et d’industrie qu’auparavant, pour fournir aux besoins et aux jouissances habituelles des hommes qui composent la classe laborieuse, plus la somme de travail et d’industrie qui a dû pourvoir au surcroît de prix destiné à l’impôt ? Or, cet, impôt, ou ce surcroît de produit qui se paye, étant dépensé par le gouvernement qui le recueille, sert à alimenter une nouvelle classe de consommateurs qui forment des demandes que l’impôt les met à portée de payer.

Si ces conjectures étaient fondées, il s’ensuivrait que, bien loin d’avoir une influence nuisible sur la richesse et la population du pays, l’impôt de consommation sagement combiné tendrait à accroître et à fortifier ces deux grands fondements de la prospérité et de la puissance nationale. Il y tendrait précisément par la raison qu’il porte immédiatement sur la généralité du peuple, et qu’il agit sur la classe ouvrière et industrieuse qui est la plus active du corps social, tandis que l’impôt direct ou foncier n’agit que sur la classe oisive des propriétaires.

Ces observations semblent donner l’explication du phénomène le plus surprenant de l’économie politique, savoir, l’accroissement rapide et prodigieux de la richesse chez les nations les plus chargées d’impôts sur les articles de la consommation générale. Elles mériteraient peut-être d’être développées avec plus d’étendue que n’en comportent les bornes d’une préface ; mais on en a dit assez pour faire pressentir que ce n’est pas en soumettant la théorie de l’impôt à un calcul rigoureux, et pour ainsi dire mathématique, que l’on peut apprécier ses véritables effets sur la prospérité publique.

Ainsi, de toutes les vérités qui ont été découvertes et publiées par les économistes, les unes sont d’une faible utilité dans la pratique ; les autres se trouvent contredites dans leur application par des circonstances accessoires que la théorie n’avait pas fait entrer dans ses calculs.

Ce qui établit la différence capitale entre la doctrine de \Smith et celle des économistes, c’est le point duquel elles partent l’une et l’autre pour tirer leurs conséquences. Les derniers remontent à la terre, comme source primitive des richesses ; l’autre s’appuie sur le travail, comme l’agent universel dont elles sont toutes produites.

Dès le premier coup d’œil on reconnaît combien l’école du professeur d’Édimbourg doit l’emporter sur celle des philosophes français, sous le rapport de l’utilité publique et de l’application de ses préceptes. Le travail étant une puissance dont l’homme est la machine, l’accroissement de cette puissance ne doit guère rencontrer d’autres limites que celles presque indéfinies de l’intelligence et de l’industrie humaine, et elle est, comme ces facultés, susceptible d’être dirigée par des conseils et perfectionnée par le secours de la méditation. La terre, tout au contraire, abstraction faite de l’influence qu’a le travail sur l’espèce et la quantité des productions qu’elle peut rendre, est entièrement hors du pouvoir des hommes, sous tous les autres rapports qui pourraient la rendre plus ou moins avantageuse pour la nation qui la possède, son étendue, sa situation et ses propriétés physiques.

Ainsi la science de l’économie politique, considérée sous le point de vue qu’ont adopté les économistes, rentre dans la classe des \emphase{sciences naturelles}, qui sont purement spéculatives, et qui ne peuvent se proposer autre chose que la connaissance et l’exposition des lois qui régissent l’objet dont la science s’occupe ; au lieu que, vue sous l’aspect sous lequel \Smith nous la représente, cette science se trouve réunie aux autres \emphase{sciences morales} qui tendent à améliorer leur objet et à le porter au plus haut degré de perfection dont il est susceptible. La doctrine de \Smith peut être réduite à un petit nombre de principes extrêmement simples, et peu de mots suffisent pour en faire l’exposition.

La puissance avec laquelle une nation produit ou acquiert toutes ses richesses, c’est le \emphase{travail}.

Les produits de cette puissance seront d’autant plus grands, qu’elle recevra plus d’accroissement.

Or, elle peut s’accroître de deux manières, en \emphase{énergie} et en \emphase{étendue}.

Le travail gagne en \emphase{énergie}, quand la même quantité de travail fournit de plus grands produits. La division des parties d’un même ouvrage en autant de tâches séparées, exécutées par des mains différentes, l’invention de machines et de procédés propres à abréger et à faciliter la main-d’œuvre, sont les deux moyens principaux par lesquels le travail acquiert de l’énergie, et qui perfectionnent ses facultés productives.

Le travail gagne en \emphase{étendue}, quand le nombre des travailleurs augmente dans sa proportion avec celui des consommateurs. Cette augmentation résulte de l’accumulation progressive des capitaux et aussi du genre d’emploi vers lequel ils sont dirigés ; certains emplois, à égalité de capital, tenant en activité une plus grande quantité de travail national que d’autres emplois.

Pour que le travail puisse accroître dans l’une et l’autre de ces dimensions et arrive progressivement au \emphase{maximum} d’énergie et d’étendue qu’il peut atteindre dans une nation, vu la situation, la nature et la qualité du territoire qu’elle possède, qu’ont à faire les administrateurs qui la gouvernent ?

La division des parties du même corps d’ouvrage ou article de marchandise, l’invention et le perfectionnement des machines et procédés d’industrie, ces deux grands moyens d’augmenter l’énergie du travail, avancent en raison de l’étendue du marché, c’est-à-dire à proportion du nombre des échanges qui peuvent se faire, de la facilité et de la promptitude avec laquelle ils s’opèrent. Que le gouvernement mette donc tous ses soins à agrandir pour ses sujets le marché ouvert aux produits de leur travail. Des routes sûres et commodes, tant par terre que par eau, la plus grande liberté de communication entre tous acheteurs et tous vendeurs, tant au dedans qu’au dehors du pays, un bon système de monnaies, la garantie de l’exécution fidèle des contrats et promesses, sont des mesures indispensables, mais toujours efficaces pour parvenir à ce but. Plus le gouvernement approchera du mieux sur chacun de ces trois points, plus il sera certain de donner au marché national tout l’agrandissement dont ce marché est susceptible. Le premier des trois est sans contredit le plus essentiel, puisqu’il ne peut être suppléé par aucun autre expédient, et qu’à son défaut les autres seront sans effet.

L’accumulation graduelle des capitaux est une suite nécessaire de l’augmentation des facultés productives du travail, et elle contribue encore, comme cause, à une augmentation ultérieure de ces facultés ; mais à mesure que cette accumulation vient à grossir, elle ajoute encore, sous un autre rapport, à la puissance du travail, en lui donnant, plus d’étendue, parce qu’elle multiplie considérablement le nombre des travailleurs et augmente la somme du travail national, et cette augmentation dans le nombre de bras employés parmi les nationaux dépendra de la nature d’emploi à laquelle le capital sera destiné.

Sous ce second rapport de l’augmentation des produits par une plus grande étendue de travail, la tâche du gouvernement est encore plus facile. Ici il n’a point à agir, il lui suffit de ne pas nuire. On ne lui demande que de protéger la liberté naturelle de l’industrie, de lui laisser ouverts tous les canaux dans lesquels elle sera entraînée par son impulsion spontanée et par la suggestion de l’intérêt privé, de l’abandonner à son propre penchant et de ne pas avoir la prétention de connaître mieux qu’elle dans quel sens elle doit diriger ses efforts, attendu que l’infaillible intérêt qui lui sert de guide, le sentiment de ce qui lui est le plus convenable ou le plus avantageux, la conduiront d’une manière plus sûre que ne pourront jamais le faire les soins et l’autorité de l’administration publique.

On voit que ces deux écoles, les seules que la philosophie moderne ait fondées sur cette branche de l’économie politique qui concerne la formation et la distribution des richesses, quoique différentes dans les principes sur lesquels elles établissent leur doctrine, s’accordent néanmoins sur ce point : c’est que le désir inné dans chaque individu d’améliorer sa condition est le premier mobile de l’accroissement progressif de la richesse nationale ; qu’ainsi ce désir, tant qu’il n’agit point d’une manière contraire aux droits d’autrui, doit jouir de la liberté la plus illimitée et de la protection la plus impartiale ; qu’il n’existe point dans une nation, sous le rapport de sa richesse, d’autre intérêt général que la réunion de tous les intérêts privés concourant librement au même but ; que c’est une erreur que de supposer dans ce cas un intérêt national en opposition avec les intérêts privés, et que de croire qu’on puisse jamais servir la chose publique en sacrifiant certains intérêts privés à d’autres intérêts privés. Elles s’accordent aussi à soutenir que la force publique n’a été confiée au gouvernement que pour lui donner les moyens de protéger également et sans aucune partialité le libre exercice du travail et de l’industrie, tant qu’il ne porte aucun dommage aux droits d’autrui ; que le gouvernement fait un abus injuste et déraisonnable du pouvoir qui lui a été remis, lorsqu’il ne couvre pas de la même protection tous les droits et tous les intérêts : injuste, lorsqu’il restreint et gêne la circulation du travail et de l’industrie par des créations de corporations, jurandes, maîtrises et autres institutions de ce genre qui tendent à déshériter la plus grande partie de la classe indigente du seul patrimoine qu’elle ait reçu de la nature ; lorsqu’il arme et solde des troupes de commis sur les frontières pour empêcher ses sujets de consommer ce qui leur paraît plus agréable, plus commode et moins coûteux, et pour empêcher les producteurs de disposer de leur légitime propriété au prix le plus avantageux : déraisonnable, lorsqu’il diminue le nombre des travailleurs et resserre le développement de l’industrie ; lorsqu’en écartant par force la concurrence des vendeurs et des acheteurs, il met obstacle à des échanges plus avantageux et détruit le plus puissant stimulant qui puisse aiguillonner l’activité et l’industrie de ses propres sujets, les piquer d’une utile émulation et les mettre dans la nécessité d’obtenir une juste préférence sur tous autres par la bonne qualité et par le bon prix ; enfin lorsqu’il veut, par des faveurs et des récompenses, attirer le travail et les capitaux dans des voies que leur propre intérêt les détourne de suivre, parce qu’ils reconnaissent que leurs produits y seraient inférieurs.

Rien, sans doute, ne serait plus facile que de concilier ces deux écoles et de les ramener aux mêmes principes ; mais la tâche véritablement difficile, celle qu’il ne faut pas se flatter de pouvoir remplir de longtemps, ce serait de déterminer l’administration publique à se désister des mesures fausses et illusoires auxquelles elle attache tant d’importance, et pour la conservation desquelles tant d’intérêts particuliers seront toujours prêts à se coaliser.

On a été dans ces derniers temps jusqu’à contester à \Smith qu’il ait été le créateur de la doctrine qui fait dériver la richesse du travail. Longtemps avant \Smith, sans doute, il avait été reconnu que l’homme doit au travail ses moyens de subsistance, et que les nations comme les individus ne peuvent s’enrichir qu’à force de travail et d’industrie. C’est une vérité aussi ancienne que le monde. On la trouve consignée dans les premières pages d’un livre que sa haute antiquité et la simplicité sublime avec laquelle il est écrit, suffiraient seules pour recommander à la vénération des hommes. (\publication{Genèse}, chap. 3, v. 17 et 19.) Tous ceux qui ont été appelés à instruire ou à gouverner des sociétés humaines, et qui ont cherché à les rendre heureuses et puissantes, se sont appliqués à leur présenter tous les moyens possibles d’encouragement au travail, et l’on ne connaît guère de législateurs ou de moralistes qui n’aient prescrit aux hommes de travailler pour rendre leur vie plus heureuse dans le présent et mieux assurée pour l’avenir. Mais que pourrait-on inférer de ces observations pour affaiblir la gloire que s’est acquise \AS par son immortel ouvrage sur la richesse des nations ?

Il y a autour de nous une multitude de faits qui se présentent si fréquemment qu’ils frappent tous les yeux ; mais entre les faits les plus communs il existe souvent d’importantes relations qui demeurent inaperçues jusqu’au.moment où un habile et profond observateur parvient à les découvrir et à les révéler à son siècle. Il y a des vérités morales tellement évidentes, qu’il n’est guère d’esprits auxquels elles échappent ; mais quelquefois les vérités les plus vulgaires sont fécondes en conséquences de la plus grande utilité, et ces conséquences restent cachées jusqu’à ce que la méditation de l’homme de génie, s’attachant à ce principe, s’obstine à le creuser et finit par mettre au jour les trésors qui y étaient renfermés. Dès ce moment la science est créée et commence à répandre ses premiers bienfaits. Longtemps avant qu’il existât des mathématiques, il est probable que les hommes pratiquaient des méthodes informes pour combiner les quantités et mesurer les surfaces. Avant que les premiers éléments de l’astronomie fussent connus, les pâtres avaient remarqué que le soleil décrivait chaque jour sur leur horizon une portion de cercle qui s’élevait et qui s’abaissait aux différentes saisons de l’année. Les études et les aperçus d’un grand homme fécondent ces germes perdus et en font éclore la science. Cet homme donne une grande impulsion à toute la société dont il est membre. La découverte qu’il a faite devient la propriété de tous ; transmise par la voie seule de l’enseignement à la génération suivante, cette génération a déjà fait un pas immense et se trouve tout à coup portée à un degré d’élévation dont ses ancêtres n’avaient aucune idée.

\AS s’est emparé d’une vérité triviale, et sous ses mains cette vérité est devenue une mine inépuisable. Qui, avant lui, avait imaginé de considérer philosophiquement la nature et les effets du travail ? Qui avait observé comment son énergie grandit et se développe, quand son action, distribuée sur les diverses parties d’un même ouvrage, s’exerce séparément sur chacune d’elles ? Qui avait trouvé les rapports naturels qui subsistent entre le travail et les valeurs qu’il a produites, de manière à ce que le premier puisse servir de mesure aux secondes ? Qui avait remonté jusqu’au principe qui donne l’impulsion au travail et indique les causes qui en tiennent à la fois une plus grande masse en activité avec l’emploi d’une quantité égale de ce mobile ?

Toute cette théorie sur le travail, sur les effets de sa division, ainsi que de tous les autres moyens qui l’abrègent et le facilitent, sur l’étendue dont ces moyens de perfectionnement sont susceptibles et sur ce qui la limite, sur l’action du capital considéré comme moteur du travail, et sur les divers emplois qui répètent cette action plus ou moins souvent dans un temps donné ; enfin sur la puissance qu’acquièrent les facultés productives du travail par le progrès successif des arts de la civilisation, et sur l’effet que cet accroissement de puissance a sur la valeur réelle et intrinsèque de la subsistance qui alimente l’ouvrier, et par conséquent sur la richesse et le bien-être de la classe qui dispose de la subsistance et qui commande en première ligne tout le travail de la société, compose une science entièrement neuve, dont \Smith doit être incontestablement proclamé le créateur.

Cette vaste et profonde théorie n’est encore qu’une partie des services que cet illustre écrivain a rendus à l’économie politique. Ce n’est pas avec moins de justesse et de sagacité qu’il a analysé les échanges et distingué les parties constituantes du prix des marchandises, en considérant chaque prix en particulier, ainsi que celles du revenu total de la société et la distribution de chacune d’elles entre les différentes classes du peuple ; qu’il a posé les principes d’après lesquels on doit reconnaître le prix \emphase{réel} des choses fondé sur la nature même de l’homme et des besoins que ces choses sont destinées à satisfaire, afin de ne pas le confondre avec le prix \emphase{nominal} ou pécuniaire, qui s’éloigne souvent du prix réel à cause des variations de valeur auxquelles l’argent est sujet ; qu’il a également établi les conditions qui forment le prix \emphase{naturel} et ordinaire des choses, prix qui suppose que les quantités produites sont avec les quantités demandées dans cet état d’équilibre auquel elles tendent sans cesse ; de manière à ce que ce prix naturel puisse être distingué du prix courant ou \emphase{prix du marché}, qui s’écarte souvent du prix \emphase{naturel}, lorsque l’équilibre entre les quantités offertes et les quantités demandées se trouve momentanément et accidentellement dérangé par des circonstances extraordinaires, soit aux dépens de la production quand elle est surabondante et dépasse la somme des besoins de la consommation, soit aux dépens de celle-ci quand les quantités produites sont insuffisantes et ne peuvent satisfaire à la quantité des demandes.

Toute cette partie de la doctrine de \Smith est également neuve et lui appartient entièrement ; elle est toujours une déduction du même principe. Quand une marchandise est venue au marché où elle doit être vendue, elle y représente la quantité de travail qui a été employée à la produire, à la fabriquer et à la transporter ; et, dans l’état ordinaire des choses, il faut qu’elle obtienne en échange la même quantité de travail, sans quoi elle cesserait d’être produite ou de reparaître à ce marché. Le blé, qui est la subsistance de l’ouvrier, a aussi sa valeur naturelle réglée par le travail, mais d’après un autre principe. La valeur réelle d’une mesure de blé, année moyenne, c’est la quantité de travail qu’elle peut alimenter et entretenir dans l’état actuel de la société. C’est le prix qu’elle ne peut manquer de trouver, car il y a toujours du travail qui s’offre pour échange de la subsistance. Ainsi le possesseur de cette mesure de blé, soit qu’il veuille l’échanger contre du travail à faire, soit qu’il l’échange contre le travail fait, pourra commander ou obtenir une quantité égale au travail que cette portion de blé peut faire subsister.

C’est le travail qui règle toujours les conditions des échanges ; c’est lui qui constitue le prix naturel de toutes choses.

\section{Méthode pour faciliter l’étude\protect\\ de l’ouvrage de \Smith}

Tel est le résultat de la doctrine de \Smith et le fruit qu’on doit recueillir de son immortel ouvrage. L’évidence du principe et l’enchaînement naturel des conséquences donnent à toute cette doctrine un caractère de simplicité et de vérité qui ne la rend pas moins admirable que convaincante.

Mais cette simplicité ne s’aperçoit pas au premier coup d’œil, et pour la reconnaître il faut beaucoup d’étude et de méditation. On ne peut se dissimuler que le défaut tant de fois reproché aux écrivains anglais de manquer de méthode et de négliger, en traitant les sciences, ces formes didactiques qui soulagent la mémoire du lecteur et guident son intelligence, se fait surtout sentir dans les \publication{Recherches sur le richesse des nations}. Il semble que l’auteur ait pris la plume au moment où il était le plus exalté par l’importance de son sujet et par l’étendu de ses découvertes. Il débute par étaler aux yeux du lecteur les innombrables merveilles opérées par la \emphase{division du travail}. Et c’est par ce tableau imposant et magnifique qu’il ouvre le cours de ses leçons. De là, remontant aux circonstances qui amènent et qui limitent cette division, il est conduit, par la suite de ses idées, à la définition des valeurs, aux lois qui les régissent, à l’analyse des divers éléments qui les composent, et aux rapports qui existent entre les valeurs de différentes nature et origine, toutes notions préliminaires qui devaient naturellement être exposées au lecteur avant de lui mettre sous les yeux la machine compliquée de la multiplication des richesses et de lui découvrir les prodiges du plus puissant de ses ressorts.

D’un autre côté, le fil des leçons est souvent interrompu par de longues digressions qui en font entièrement perdre la trace. Telles sont celle sur \emphase{les variations de la valeur des métaux précieux pendant les quatre derniers siècles}, avec un examen critique de l’opinion qui suppose que cette valeur va en décroissant (liv. I, chap. xi) ; celle sur \emphase{les banques de circulation et sur le papier monnaie} (liv. Il, chap. ii) ; celle sur \emphase{les banques de dépôt, et en particulier celle d’Amsterdam}, dont l’auteur expose les statuts et les opérations dans le plus grand détail (liv. IV, chap. iii) ; celle sur les \emphase{avantages d’un droit de seigneuriage sur la fabrication des monnaies}, insérée dans un chapitre intitulé : des \emphase{Traités de commerce} (liv. IV, chap. vi) ; enfin, celle sur le \emphase{commerce des grains et sur la législation de ce commerce}, tout fait étrangère à l’objet principal du livre dans lequel elle se trouve. Ces traités particuliers, dont chacun est peut-être le meilleur qui ait jamais été fait sur le sujet, sont cependant placés de manière à distraire l’attention du lecteur et à lui faire perdre de vue l’objet principal, et ils nuisent beaucoup à l’effet de l’ensemble. L’auteur ne s’est pas caché les inconvénients de ces digressions, et, en plusieurs endroits, il s’excuse de s’être trop écarté de son sujet, et il reconnaît même que sa digression eût dû être placée dans un autre endroit. On regrette que ces traités particuliers que l’auteur a voulu conserver n’aient pas été par lui rejetés à la fin de son ouvrage en morceaux détachés et par forme d’\emphase{appendices}.

Pour remédier, autant qu’il est en moi, aux difficultés que pourraient présenter à un grand nombre de lecteurs ces irrégularités dans la composition du livre \publication{de la Richesse des nations}, et pour faciliter aux commençants l’étude de la doctrine de \Smith, j’ai cru devoir indiquer l’ordre qui m’a semblé le plus conforme à la marche naturelle des idées, et par cette raison le plus propre à l’enseignement. Je commence par observer que toute la doctrine de \Smith sur la formation, multiplication et distribution des richesses, est renfermée dans ses deux premiers livres, et que les trois autres pourraient être lus à part, comme autant d’ouvrages séparés, qui, à la vérité, confirment et développent sa doctrine, mais qui ne servent pas à la compléter.

En effet, le troisième livre est une discussion politique et historique sur la marche que suivraient les progrès de la richesse dans un pays où le travail et l’industrie seraient librement abandonnés au cours de leur pente naturelle, et sur les circonstances particulières qui, par suite des événements, ont amené dans toutes les contrées de l’Europe une marche directement contraire.

Le quatrième livre est un traité polémique dans lequel l’auteur s’est proposé de combattre les divers systèmes d’économie politique qui ont été en crédit, et principalement celui qu’il désigne sous le nom de \emphase{système mercantile}, dont l’influence a été si forte sur la législation et sur l’administration dans tous les pays de l’Europe, et principalement en Angleterre. Il traite, dans autant de chapitres séparés, des divers expédients que les gouvernements ont mis en œuvre dans la vue de favoriser ce système, tels que les prohibitions et autres entraves à l’importation des marchandises étrangères, les restitutions de droits à la réexportation, les gratifications pour encourager diverses branches de commerce, le monopole du commerce avec les colonies, et enfin les traités de commerce favorables à ce système.

Enfin le cinquième livre traite des revenus de l’État et des dépenses dont il est à propos qu’il demeure chargé, telles que les dépenses nécessaires pour la défense commune, celles pour l’administration de la justice, et celles qui ont pour objet l’instruction de la jeunesse, ainsi que le perfectionnement moral du peuple dans tous les âges et toutes les conditions. Il discute aussi, dans un chapitre particulier, les dépenses que le gouvernement doit faire pour favoriser le commerce en général et la circulation des marchandises par de bonnes routes, et tous les moyens de communication les plus commodes. Il semble que les dépenses publiques relatives à la monnaie, et les soins que doit prendre le gouvernement pour la tenir constamment dans le meilleur état possible, auraient dû trouver leur place dans ce livre. L’auteur a cru devoir, au sujet des dépenses faites pour favoriser le commerce en général, examiner s’il convient au gouvernement de faire des dépenses pour favoriser quelques branches particulières de commerce, ce qui le conduit à donner l’histoire des diverses compagnies autorisées et privilégiées pour le commerce des \emphase{Indes}, de l’\emphase{Afrique}, de la \emphase{mer du Sud}, etc., qui toutes ont occasionné à l’État des pertes sans profit, et à conclure que l’administration se fût épargné de grands frais et eût servi le pays plus utilement en maintenant la liberté de tous ces divers commerces à la généralité de ses sujets. Toute cette discussion sur les compagnies de commerce exclusives ou privilégiées aurait peut-être été mieux placée dans le quatrième livre, parmi les expédients adoptés dans la vue de favoriser le \emphase{système mercantile}.

La seconde partie de ce cinquième et dernier livre est consacrée à l’examen des méthodes les plus équitables et les moins onéreuses au peuple, de pourvoir aux dépenses publiques ; question sur laquelle l’auteur n’est point d’accord avec les économistes français, et approuve les impôts indirects, ou taxes sur les objets de consommation. Ce livre est terminé par un chapitre dans lequel l’auteur traite des dettes publiques et de leur influence sur la prospérité nationale.

D’après ce qu’on vient de voir, ces trois derniers livres peuvent être lus et étudiés tels qu’ils ont été composés, et ils seront aisément compris par tout lecteur qui sera parvenu à bien entendre le corps de doctrine renfermé dans les deux premiers.

Je considère donc ici ces deux premiers livres comme un ouvrage complet, que je divise en trois parties.

La première traite des valeurs en particulier ; elle comprend leur définition, les lois qui les régissent ; l’analyse des éléments qui constituent une valeur ou qui entrent dans sa composition ; enfin les rapports que des valeurs de diverse origine ont à l’égard l’une de l’autre.

La seconde partie traite de la masse générale des richesses. On y divise les richesses en plusieurs classes, selon leur destination ou la fonction qu’elles remplissent.

La troisième et dernière partie expose la manière dont s’opèrent la multiplication et la distribution des richesses. 

\begin{center}
PREMIÈRE PARTIE

Des valeurs en particulier
\end{center}

La qualité essentielle qui constitue les richesses, et sans laquelle elles ne mériteraient pas ce nom, c’est la \emphase{valeur échangeable}.

La \emphase{valeur échangeable} diffère de la \emphase{valeur d’utilité}. (Liv. I, fin du chap. iv.)

Le rapport qui existe entre deux valeurs échangeables, exprimé en une valeur convenue, se nomme \emphase{prix}.

La valeur généralement convenue chez les peuples civilisés, c’est celle des métaux précieux. Motifs de cette préférence, origine de la monnaie (liv. I, chap. iv) ; rapport entre la monnaie et le métal monnayé. (Liv. I, chap. v.)

Le prix en argent ou prix \emphase{nominal} des choses diffère du prix \emphase{réel}, qui est leur évaluation par la quantité de travail qu’elles coûtent ou qu’elles représentent. (\emphase{Idem.})

Lois d’après lesquelles s’établit naturellement le prix des richesses ; des circonstances accidentelles qui font dévier le prix actuel ou courant, du prix naturel, ce qui donne lien à distinguer entre \emphase{prix naturel} et \emphase{prix de marché}. (Liv. I, chap. vii.)

Le prix se compose ordinairement de trois éléments distincts : le \emphase{salaire} du travail ; le \emphase{profit} de l’entrepreneur du travail ; la \emphase{rente} de la terre qui a fourni la matière du travail. Il existe quelques marchandises dans le prix desquelles n’entre point la \emphase{rente} ; d’autres, en plus petit nombre, dans le prix desquelles n’entre point le \emphase{profit} aucune dont le prix ne soit composé de \emphase{salaire}. (Liv. I, chap. vi.)

\emphase{Du salaire}. Lois d’après lesquelles s’établit naturellement le taux du salaire ; des circonstances accidentelles qui le font sortir momentanément des limites de ce taux naturel. (Liv. I, chap. viii.)

\emphase{Du profit} des capitaux. Lois d’après lesquelles s’établit naturellement le taux du profit ; des circonstances accidentelles qui, momentanément, l’élèvent au-dessus ou l’abaissent au-dessous de ce taux naturel. (Liv. I, chap. ix.)

Le travail et les capitaux tendent naturellement à se répandre uniformément dans tous les emplois, et certains emplois, par leur nature, étant accompagnés de désagréments ou de difficultés qui ne se rencontrent pas dans les autres ; certains emplois, an contraire, offrant des avantages réels ou imaginaires qui leur sont particuliers, le salaire et les profits doivent s’élever ou s’abaisser proportionnellement à ces désavantages et à ces avantages, de manière à former un juste équilibre entre tous les divers emplois. La police arbitraire et oppressive de l’Europe s’oppose, en beaucoup de circonstances, à ce que cet équilibre s’établisse conformément à l’ordre naturel. (Liv. I, chap. x.)

De la \emphase{rente} de la terre. Ce que c’est que la rente de la terre ; comme elle entre dans le prix des richesses, et d’après quels principes il arrive que tantôt elle forme et tantôt elle ne forme pas une partie intégrante de ce prix. (Liv. I, chap. xi.)

Division des produits bruts de la terre en deux grandes classes :

1° Les produits qui sont toujours nécessairement vendus de manière à rapporter une rente au propriétaire de la terre.

2° Ceux qui, selon les circonstances, peuvent se vendre sans rapporter de rente de terre, et qui peuvent aussi se vendre quelquefois de manière à en rapporter une.

Les produits bruts de la première classe viennent de la terre propre à fournir de la nourriture a l’homme ou aux animaux dont se nourrit l’homme. La valeur du produit des terres cultivées pour la nourriture de l’homme, détermine la valeur de toutes les autres terres propres à cette culture. Cette règle générale souffre quelques exceptions ; causes de ces exceptions.

Les produits de la seconde classe sont les matières propres au vêtement, au logement, au chauffage, aux meubles et ustensiles du ménage, à l’ornement de la personne et à la décoration de l’habitation. La valeur de ces produits est dépendante de celle des produits de la première classe. Quelles circonstances font que les produits de cette classe peuvent se vendre de manière à fournir une rente au propriétaire de la terre dont ils ont été tirés. Principes d’après lesquels se règle la proportion pour laquelle entre la vente dans le prix de ces produits. (Liv. I, chap. xi.)

Rapport entre les valeurs respectives des produits de la première classe et celles des produits de la seconde. Des variations qui peuvent survenir dans ces rapports et des causes de ces variations. (Liv. I, chap. xi.)

Rapport entre les valeurs des produits bruts des deux classes ci-dessus et celles des produits de manufacture. Des variations qui peuvent survenir dans ce rapport. (Liv. I, chap. xi.) 

Certains produits bruts, tirés de sources très-différentes, sont cependant destinés au même genre de consommation ou à satisfaire le même besoin, procurer la même sorte de commodité, tels que le bois et le charbon de terre pour chauffer, la cire et le suif pour éclairer ; de là il résulte que la valeur de l’un détermine et limite celle de l’autre. (\emphase{Idem}.)

Les rapports des valeurs de diverse nature changent selon l’état dans lequel se trouve la société. Cet état est ou \emphase{progressif}, ou \emphase{décroissant}, ou \emphase{stationnaire}, c’est-à-dire que la société marche vers une plus grande opulence ou décline vers l’appauvrissement, ou se maintient au même état de richesse, sans monter ni descendre.

Des effets que produisent ces diverses conditions de la société sur le prix des salaires (liv. I, chap. viii), sur le taux du profit (liv. I, chap. ix), sur la valeur des produits bruts de la terre et sur celle des divers produits de manufacture (liv. I, chap. xi). Différence à cet égard entre diverses sortes de produits bruts, savoir : 1° ceux que l’industrie humaine ne saurait multiplier ; 2° ceux que cette industrie a le pouvoir de multiplier à proportion des demandes ; 3° ceux sur la multiplication desquels l’industrie humaine n’a qu’une influence incertaine ou limitée. (Liv. I, chap. xi.)

\begin{center}
DEUXIÈME PARTIE

Des richesses considérées en masse et relativement à leurs fonctions
\end{center}

Les richesses accumulées entre les mains d’un particulier sont de deux natures, quant à leur destination ou à la fonction à laquelle il se propose de les employer.

1° Celles réservées pour servir à sa consommation actuelle ou prochaine.
2° Celles employées comme capital pour lui procurer un revenu. (Liv. II, chap. i.)

Le capital employé est aussi de deux espèces différentes : 1° le capital \emphase{fixe} qui produit un revenu sans changer de maître ; 2° le capital \emphase{circulant}, qui ne peut produire de revenu à son possesseur qu’autant que celui-ci l’échange. (Liv. II, chap. i.)

La totalité des richesses accumulées dans une société peut se diviser dans les mêmes trois parties.

\begin{itemize}
	\item [1°] Les fonds destinés à l’actuelle ou prochaine consommation de ceux dans les mains desquels ils se trouvent. 
	\item [2°] le capital \emphase{fixe} de la société.
	\item [3°] Son capital \emphase{circulant}.
\end{itemize}

Le capital \emphase{fixe} de la société, c’est 1° toutes les machines et instruments de travail ; 2° tous les bâtiments et constructions destinés à une exploitation quelconque ; 3° les travaux et améliorations faites à la terre pour la rendre plus productive ; 4° les talents et l’habileté que certains membres de la société ont acquis à force de temps et de dépense, en réalisant ainsi sur eux-mêmes le travail accumulé qui a pourvu à cette dépense.

Le capital \emphase{circulant} de la société, c’est 1° l’argent de la circulation ; 2° les provisions de vivres existantes entre les mains, tant des producteurs que des marchands, et gardées, tant par les uns que par les autres, pour être revendues avec profit ; 3° les matériaux pour logement, vêtement, ameublement, ornement ou décoration, plus ou moins manufacturés, étant entre les mains des ouvriers occupés à les mettre en œuvre et à les rendre tout à fait consommables ; 4° l’ouvrage terminé et propre à la consommation étant dans les magasins et boutiques des marchands qui les gardent pour les revendre avec profit, ou dans les navires et voitures qui les transportent pour le compte du marchand ou du consommateur. (Liv. II, chap. i.)

Des rapports de fonctions qui existent entre ces deux genres de capital. (\emphase{Id.})

De la route que suit le capital \emphase{circulant} en sortant de la circulation pour entrer soit dans le capital \emphase{fixe}, soit dans le fonds de la consommation actuelle et prochaine. (\emphase{Id.})

Les sources qui renouvellent sans cesse le capital circulant, à mesure qu’il diminue en entrant dans le capital fixe ou dans le fonds de consommation, sont : 1° la terre, 2° les mines et carrières, 3° les pêcheries. (\emphase{Id.})

De la fonction que remplit l’argent dans la circulation (liv. II. chap. ii) ; des expédients qui peuvent remplir la même fonction à moins de frais, et des inconvénients auxquels ils sont sujets. (\emphase{Id.})

Des fonds prêtés à intérêt, des circonstances qui règlent la proportion de cette nature de fonds avec la masse totale des fonds existants dans la société. La quantité des fonds à emprunter ne dépend nullement de la quantité du numéraire existant dans la circulation. (Liv. II. chap. iv.) 

Des principes d’après lesquels s’établit le taux commun de l’intérêt de l’argent. (\emphase{Id.}).

Il y a un rapport nécessaire entre ce taux et le prix courant des fonds de terre. (\emphase{Id.})


\begin{center}
TROISIÈME PARTIE

Manière dont s’opèrent la multiplication et la distribution des richesses
\end{center}

Les richesses se multiplient à mesure que la puissance qui les produit augmente, soit en énergie, soit en étendue, (Liv. I, \emphase{Introduction}.)

Le travail, qui est cette puissance, augmente en\emphase{ énergie}, 1° par la division des parties d’un même corps d’ouvrage ou article de manufacture en autant de taches séparées, exécutées par des mains différentes ; 2° par l’invention des machines qui abrégent et facilitent le travail. (Liv. I, chap. i.)

La division augmente l’énergie du travail, 1° par l’habileté et dextérité qu’elle fait acquérir à l’ouvrier ; 2° par l’épargne du temps. (\emphase{Idem.})

L’invention des machines est elle-même un effet de la division du travail. (Id.)

Le penchant particulier à l’espèce humaine, qui porte les individus à échanger entre eux les divers produits de leurs travaux et de leurs talents respectifs, est le principe qui a donné lieu à la division du travail. (Liv. I, chap. ii.)

La division du travail est donc nécessairement limitée par le nombre d’échanges possibles, c’est-à-dire par l’étendue du marché, d’où tout ce qui tend à agrandir le marché d’une nation facilité les progrès de cette nation vers l’opulence. (Liv. I, chap. iii.)

Le travail augmente en \emphase{étendue} en raison, l° de la plus grande accumulation des capitaux ; 2° de la manière dont ces capitaux sont employés. (Liv. I, \emphase{Introduction}.)

Les capitaux s’accumulent d’autant plus vite que la proportion entre les consommateurs productifs et les consommateurs non productifs est plus grande en faveur des premiers. (Liv. II, chap. iii.)

Ce qui détermine la proportion entre ces deux classes de consommateurs c’est la proportion qui se trouve entre la portion du produit annuel destinée à remplacer un capital et celle destinée à servir de revenu. (\emphase{Id.}) 

La proportion entre la portion du produit annuel qui va aux capitaux et celle qui va aux revenus, est forte dans un pays riche et faible dans un pays pauvre. (Liv. II, chap. iii.)

Dans le pays riche, la terre, prise \emphase{absolument}, est beaucoup plus forte que dans un pays pauvre ; prise \emphase{relativement} au capital employé, elle est beaucoup plus faible. (\emphase{Id.})

Dans le pays riche, les profits de capitaux, pris en somme, forment une valeur infiniment plus grande ; mais, relativement au capital, ils sont dans une proportion fort inférieure, c’est-à-dire que le taux du profit est bien plus élevé dans le pays pauvre. (\emphase{Id.})

L’industrie fait le produit, mais c’est l’économie qui, de ce produit, fait aller parmi les capitaux ce qui, sans elle, eût été parmi les revenus. (\emphase{Id.})

L’économie des particuliers naît d’un principe universellement répandu et continuellement en action, le désir inné dans chacun d’améliorer sa condition. Ce principe entretient la vie et l’accroissement de la richesse nationale, malgré les prodigalités de quelques particuliers, et il triomphe même des profusions et des erreurs du gouvernement. (\emphase{Id.})

De plusieurs manières de dépenser, l’une est plus favorable que l’autre à l’accroissement de la richesse nationale. (\emphase{Id.})

Le genre d’emploi auquel sert un capital met plus ou moins de travail national en activité, et par conséquent contribue plus ou moins à ce que le travail national gagne en \emphase{étendue}.

Un capital ne peut s’employer que de quatre manières :
\begin{itemize}
	\item [1°] À faire produire la terre et à l’améliorer, c’est-à-dire à multiplier les produits bruts ;
	\item [2°] À entretenir des ouvriers de manufacture ;
	\item [3°] À acheter en gros pour revendre de la même manière ;
	\item [4°] À acheter en gros pour revendre en détail.
\end{itemize}

Ces quatre sortes d’emplois sont également nécessaires les uns aux autres, et ils s’entretiennent réciproquement. Le premier est celui de tous, sans nulle comparaison, qui entretient un plus grand nombre de bras productifs ; le second en occupe plus que les deux autres ; le quatrième est celui qui en occupe le moins.

Le troisième de ces quatre genres d’emplois peut avoir lieu de trois manières, et alors il contribue dans bien des degrés fort différents à soutenir et à encourager l’industrie nationale. 

Si le capital est employé à échanger des produits de l’industrie nationale contre des produits de même origine, alors il entretient autant de cette industrie que le peut faire tout capital employé dans le commerce.

S’il est employé à échanger des produits de l’industrie nationale contre des produits d’industrie étrangère, alors il sert pour moitié à l’entretien de cette industrie étrangère, et ne rend plus à l’industrie nationale que la moitié du service qu’il eût pu lui rendre s’il eût été employé de l’autre manière, c’est-à-dire à échanger deux produits de l’industrie nationale.

Enfin, s’il est employé à échanger des produits d’industrie étrangère contre des produits d’industrie étrangère, ce qu’on nomme \emphase{commerce de transport ou d’économie}, alors il sert en totalité à entretenir ou encourager l’industrie de deux nations étrangères, et il n’ajoute alors autre chose au produit annuel du pays que le profit fait par le commerçant. (Liv. II, chap. iv.)

L’intérêt privé, laissé à sa pleine liberté, porte nécessairement le possesseur du capital à préférer, toutes choses égales, l’emploi le plus favorable à l’industrie nationale, parce qu’il est aussi le plus profitable pour lui. (\emphase{Id.})

S’il est arrivé souvent que les capitaux aient pris une autre route que celle dans laquelle les eût naturellement conduits l’infaillible instinct de l’intérêt privé, c’est l’effet des circonstances dans lesquelles se sont trouvés les gouvernements de l’Europe, et de l’influence qu’ont prise sur leur système d’administration les intérêts mercantiles et des préjugés généralement répandus. L’exposition de ces circonstances et le développement des vices de ce système d’administration forment, comme il a été observé plus haut, la matière des troisième et quatrième livres.

Ce serait rendre service aux personnes qui commencent l’étude de l’économie politique, que de recomposer l’ouvrage de \Smith pour en classer les différentes parties dans un ordre plus méthodique et pour en détacher toutes ces digressions qui en interrompent la suite, ainsi que plusieurs détails qui s’appliquent exclusivement à l’Angleterre. Ce livre se trouverait par là fort abrégé, et l’instruction qu’on peut y prendre se communiquerait avec plus de facilité. Mais, en cherchant à resserrer cet admirable ouvrage sous un petit volume, il faudrait bien se donner de garde de chercher à rendre plus concis les développements très-étendus que l’auteur a voulu donner aux parties les plus essentielles et les plus délicates de sa doctrine. Pour être mieux compris dans certains endroits dans lesquels il sentait toute la difficulté de son sujet, il a souvent présenté la même idée sous plusieurs faces, et a toujours cherché à la rendre familière en la reproduisant à diverses fois. C’est ce qui a fait dire à quelques critiques qui ne l’avaient lu que superficiellement, qu’il était souvent lourd et diffus. \Smith avait bien prévu qu’il pouvait encourir ce reproche, et il a mieux aimé s’y exposer que de courir un autre risque beaucoup plus grave à ses yeux, celui de n’être pas parfaitement compris. « Je tâcherai, dit-il, de traiter ces trois points avec toute l’étendue et la clarté possibles dans les chapitres suivants, pour lesquels je demande bien instamment la patience et l’attention du lecteur : sa patience, pour me suivre dans des détails dans lesquels je lui paraîtrai peut-être en quelques endroits m’appesantir sans nécessité ; et son attention, pour pouvoir entendre ce qui semblera peut-être encore un peu obscur, malgré tous les efforts que je ferai pour être intelligible. Je courrai volontiers le risque d’être trop long pour chercher à me rendre clair, et après que j’aurai pris toute la peine dont je suis capable, pour répandre de la clarté sur un sujet qui, par sa nature, est si abstrait, je ne suis pas encore sur qu’il n’y reste quelque obscurité. »

Ces craintes de l’auteur n’étaient pas sans fondement ; nous aurons occasion de voir que parmi les critiques auxquelles son ouvrage a été en butte dans ces derniers temps, il en est plusieurs qui ne proviennent que de ce qu’on lui a fait dire tout autre chose que ce qu’il s’était donné la peine de nous expliquer avec tant de soins et de détails.

Lorsque parut le livre des \publication{Recherches sur la richesse des nations}, les vérités neuves et frappantes dont il était rempli jetèrent un si grand éclat, qu’il se fit une révolution complète dans la science de l’économie politique. Tous ceux qui avaient dirigé vers cet objet leurs études et leurs méditations ne songèrent plus qu’à se pénétrer des principes enseignés par \AS ; on se soumit généralement aux leçons de ce nouveau maître. La science semblait être fixée sur des bases désormais inébranlables, et s’être élevée au-dessus de tous les orages de la controverse. Ce livre devint classique pour toute l’Europe. Il fut bientôt traduit en Allemagne, en Italie, en France. Un des partisans les plus zélés et les plus habiles de la secte économique, l’abbé \Morellet, se hâta d’en composer une traduction dont quelques considérations particulières ont arrêté la publication. Le succès ne fut pas moins décidé en Angleterre, où, de tout temps, les matières d'économie politique ont été l’objet de savantes et profondes méditations. On sait qu’un des hommes d’État les plus célèbres de cette nation fit de l’ouvrage de \Smith son étude favorite, et que plus d’une fois il exprima le regret de ce que les préjugés populaires dont il était difficile à l’administration de secouer entièrement le joug, et l’obsession continuelle des riches négociants et des gros manufacturiers ne laissaient pas au gouvernement la liberté de se rapprocher du système le plus raisonnable de tous et le plus propre à affermir et à consolider la prospérité nationale.

Si un génie tel que celui d’\AS se fût montré dans les beaux jours de l’antiquité, ce philosophe aurait été le fondateur d’une grande et illustre école, dont les disciples n’auraient songé qu’à étudier et à propager les leçons de leur maître. Mais dans cet âge on est peu disposé à jurer sur les paroles d’autrui et à reconnaître l’empire de ces hommes supérieurs que la nature crée, à de longues distances, pour éclairer leur siècle.

Depuis une vingtaine d’années environ, des écrivains qui s’annonçaient pour être des disciples de \Smith, et qui protestaient sa doctrine sur presque tous les points, se hasardèrent à contester hautement quelques-uns de ces principes fondamentaux sur lesquels elle repose.

L’un prétendit que le \emphase{travail} ne pouvait être considéré comme une mesure des valeurs, puisque, disait-il, rien n’est plus variable ni plus incertain que la valeur du travail, comme \Smith lui-même l’a reconnu en vingt endroits de son ouvrage, dans lesquels il déclare que le salaire du travail varie d’un moment à l’autre, et souvent même dans les lieux les moins distants. Ce critique, confondant ainsi la valeur que donne l’ouvrier avec celle qu’il reçoit en échange, est justement tombé dans la méprise que \Smith a cherché à prévenir par une distinction parfaitement claire. Le salaire d’un ouvrier, dans l’Inde, n’est peut-être qu’un cinquième de ce que reçoit un ouvrier à Paris pour la même quantité de travail ; cependant l’indien, comme le Parisien, ont fourni, dans l’espace d’une journée, la même quantité de leur temps, de leur force, de leur repos et de leur liberté. Le travail est beaucoup plus productif dans une société civilisée et industrieuse que dans une société naissante et peu avancée, c’est-à-dire que dans la première de ces sociétés, celui qui emploie l’ouvrier et qui paye son travail en retire des produits plus abondants et d’une plus grande valeur ; mais, dans ces deux états de la société, ce que donne l’ouvrier est toujours, quant à lui, la même valeur ; c’est toujours un sacrifice pareil d’une portion de son temps et de sa liberté ; c’est toujours l’emploi de sa force à l’ouvrage quelconque qui lui a été commandé. Au dixième siècle, tout comme au dix-neuvième, l’ouvrier, loué à la journée, a fourni l’action de ses bras pendant un temps convenu ; à une époque comme à l’autre il a dû lui en coûter autant, il a donné la même chose. C’est le travail ainsi défini que \Smith a présenté comme mesure universelle et invariable des valeurs ; celui qui voulait contester ce principe devait au moins prendre la définition de l’auteur telle qu’il l’a donnée.

D’autres ont attaqué la doctrine de \Smith relativement au \emphase{prix}. Ils ont soutenu qu’il n’y avait pas de prix \emphase{naturel}, comme l’a enseigné l’auteur de \publication{la Richesse des nations}, mais qu’il n’y avait d’autres prix que des prix \emphase{courants}, lesquels étaient déterminés par le rapport existant entre les quantités offertes et les quantités demandées de la chose mise au marché, et que le prix n’était autre chose que l'expression de ce rapport ou de la différence entre les deux quantités, en sorte que le prix était haut de tout ce dont la quantité des demandes dépassait celle des offres, et qu’il était bas de tout ce dont la quantité des offres excédait celle des demandes. Il est clair que la conséquence rigoureuse d’une telle théorie, c’est que dans le cas d’équilibre entre les offres et les demandes, ce qui est nécessairement l’état le plus ordinaire, puisque les producteurs ont toujours intérêt à faire monter la quantité de leurs produits au niveau de la somme des demandes et à ne la pas faire monter plus haut, dans ce cas donc le prix de la chose devrait être zéro. Mais, en se conformant aux leçons de \Smith, il faut dire que le prix est trop haut quand la somme des demandes surpasse celle des offres, et qu’il tombe trop bas toutes les fois que la quantité offerte est plus grande que la quantité demandée, en sorte que dans le cas qui doit se rencontrer le plus souvent, où il y a équilibre entre la somme des offres et celle des demandes, le prix se trouve être ni trop haut ni trop bas, et par conséquent est dans son état naturel ; ce qui conduit nécessairement à rechercher ce que c’est que ce prix naturel, et d’après quelles lois il s’établit.

Ces assertions et quelques autres du même genre, incapables de soutenir un examen un peu réfléchi, n’eurent guère d’autres partisans que ceux qui les avaient imaginées. Mais, depuis peu d’années, il s’est élevé en Angleterre une opinion sur le prix du blé en argent, qui n’a pas laissé de prendre quelque crédit, parce qu’elle a été soutenue par des écrivains dont le nom porte avec soi une sorte d’autorité, et peu à peu elle s’est assez étendue pour acquérir de la consistance et même pour se donner le nom de nouvelle école. Cette théorie, comme beaucoup d’autres, dut sa naissance à l’impuissance dans laquelle on se trouva de donner une explication raisonnable à certains phénomènes extraordinaires, et faute de pouvoir rattacher des faits aux principes, on créa un principe exprès pour les faits.

On a vu qu’aucune législation ne se montra plus mobile ni plus incohérente que celle de l’Angleterre sur ce qui concerne le commerce des grains. Après avoir pendant longtemps tiré du commerce étranger une grande partie de son approvisionnement annuel en subsistances, le gouvernement, peu après la révolution de 1688, prit le parti d’encourager la culture du blé dans l’intérieur, et même de provoquer l’exportation au dehors d’une partie du produit, par une gratification accordée sur chaque mesure exportée. Ce régime dura jusqu’au milieu du siècle dernier, et il est à croire que, tant qu’il fut maintenu, il n’y eut guère dans ce pays de terres propres à la culture qui ne fussent consacrées à la production des grains, tant à cause du bon prix que l’exportation soutenait au dedans, que par l’espoir de gagner la gratification s’il y avait lieu.

Toutefois le payement de la gratification devint à la fin tellement onéreux pour le trésor, qu’on renonça à cette mesure. Quelques années ensuite, il s’éleva à ce sujet une très-vive controverse entre le docteur \Price d’une part et le célèbre cultivateur Arthur Young ; le premier soutenait qu’un pays ne devait jamais se mettre, pour sa propre subsistance, dans la dépendance du commerce ou de la production étrangère, et que l’emploi le plus avantageux qu’il pût faire de ses terres labourables était de leur faire produire du blé pour la consommation de ses habitants. Son adversaire prétendait au contraire que le meilleur système que pût adopter l’administration, c’était de laisser au cultivateur la parfaite et entière liberté de préférer le genre de culture qui lui semblait le plus lucratif, ce dont il était à portée de bien juger plus que personne ; qu’en tirant de sa terre le plus gros profit possible, il ne pouvait pas s’enrichir sans contribuer par là à l’accroissement de la richesse publique. Il ajoutait que pour un pays environné de côtes et qui disposait d’un aussi vaste et aussi puissant établissement maritime que l’Angleterre, il ne pouvait jamais y avoir lieu à craindre de manquer de subsistances, attendu que tous les pays qui en produisaient à meilleur marché qu’elle s’empresseraient toujours de lui en envoyer et y trouvaient leur intérêt ; qu’en conséquence il serait toujours plus profitable, tant pour l’agriculteur que pour la chose publique, de consacrer la terre labourable à la production des prairies artificielles et des substances propres à multiplier les bestiaux, dont la peau, le cuir, le suif et la viande salée servaient à donner plus d’emploi aux manufactures et au commerce étranger, les deux grandes sources de la prospérité anglaise. Cette dernière opinion fut celle qui détermina la conduite de l’administration. On autorisa toutes les demandes en clôture ; une grande partie des terres à blé fut convertie en un autre genre d’exploitation, et la nation fut de nouveau obligée à demander aux étrangers une portion considérable des grains nécessaires à sa consommation annuelle. Le gouvernement se trouva même dans la nécessité d’y appeler l’importation du blé étranger et d’y attacher une prime considérable en faveur du marchand importateur : les besoins furent tels, qu’en 1795 un ministre déclara qu’il avait été payé dix millions sterling pour assurer l’approvisionnement des grains pour une seule année.

Il se manifesta alors une révolution extraordinaire dans le prix moyen du blé en argent sur les marchés de l’Angleterre. Ce prix, qui pendant les quatre-vingt-dix premières années du dix-huitième siècle, comme dans tout le cours du dix-septième, avait été de 40 à 42 schellings, s’éleva depuis 1794 à plus du double de ce prix, et resta pendant toutes les dernières années de ce siècle, ainsi que pendant les dix premières du dix-neuvième, à un prix deux fois et demie plus haut qu’on ne l’avait vu dans les précédentes périodes, et hors de toute proportion avec ce qu’il était dans les autres contrées de l’Europe.

On pouvait expliquer ce phénomène par plusieurs causes qui concouraient simultanément à cette élévation du prix nominal du blé en Angleterre.

1° L’accroissement prodigieux de la taxe des pauvres, ainsi que des taxes indirectes sur tous les articles de consommation générale, outre une taxe spécialement assise sur les chevaux de labour, à raison de 17 schellings et demi par tête ; ce qui portait si haut les charges de la culture, ainsi que les dépenses obligées du fermier, que, dans plusieurs fermes, ces charges et dépenses durent absorber la totalité du produit et ne laisser aucun excédant pour le revenu du propriétaire, circonstance qui fit naturellement abandonner la culture des terres trop peu fertiles pour supporter d’aussi fortes dépenses.
2° La grande difficulté des communications maritimes et les risques qui y étaient attachés.
3° Le décri que subit à cette époque le papier-monnaie forcé, qui seul remplissait le service de la circulation, décri qui, ainsi que le démontre le cours du change, ne peut pas être évalué à moins de 30 pour cent, et doit faire monter dans cette proportion tous les prix en argent.

Au lieu de recourir à ces causes, un écrivain qui s’était fait connaître par des observations judicieuses sur la circulation et sur le crédit des billets substitués au numéraire réel, imagina d’établir sur les principes qui règlent le prix du blé en argent, un système tout nouveau, directement contraire à la doctrine de \Smith, dont il semble, sur presque tous les autres points, professer les principes.

M. \Ricardo prétendit que le prix général du blé en argent était déterminé par la quantité du travail qu’exige l’exploitation des terrains les moins fertiles de tous ceux qui concourent à approvisionner le marché intérieur, en sorte que le fermier ne commence a retirer un profit du capital employé à la culture que lorsque ce capital se trouve être appliqué à des terres supérieures en produit (relativement au capital employé) à d’autres terres déjà cultivées, mais douées de moins de fertilité.

Ce système, appuyé de raisonnements spécieux, et qui avait l’apparence de s’accorder avec les faits, entraîna plusieurs autres écrivains, et entre autres un savant professeur dont le nom était devenu célèbre dans toute l’Europe par la publication d’un livre très-remarquable \emphase{sur le principe de la population} ; en sorte qu’il n’est pas surprenant qu’à l’aide de telles autorités, ce système ait pu s’accréditer.

Cependant, pour apprécier cette opinion nouvelle et la réduire à sa juste valeur, il suffit de la rapprocher des principes établis par \AS. Ce rapprochement nous fera bientôt reconnaître que l’erreur des chefs de la moderne école consiste à avoir appliqué aux produits agricoles le principe d’après lequel se règle la valeur des produits tirés des mines et carrières. Notre auteur démontre que la valeur du produit des mines et carrières est déterminée par les frais qu’a coûté l’exploitation de la mine ou de la carrière la moins fertile de toutes celles qui concourent à l’approvisionnement du même marché. Cette règle est fondée sur le principe que les mines et carrières ne donnent de rente au propriétaire du fonds qu’autant que la demande du produit en élève assez le prix pour qu’après le payement de tous les frais d’exploitation, extraction, préparation et transport, il reste encore un excédant de prix capable de fournir une rente ou revenu au propriétaire du fonds. Le travail de la nature, celui que recueille le propriétaire, n’est jamais payé qu’après qu’il a été entièrement satisfait au salaire du travail des hommes, parce que la nature est le seul ouvrier qui travaille sans salaire. Le dernier tonneau de charbon, le dernier cube de pierre qui vient au marché pour compléter la somme des demandes ou l’approvisionnement requis par la consommation, ne se rendrait pas à ce marché si tout le travail humain employé à extraire et à transporter ce tonneau ou ce bloc n’était pas entièrement payé. Or, la demande doit payer tout ce qu’il faut pour qu’elle soit complètement remplie.

Mais les mêmes circonstances ne se rencontrent pas à l’égard des produits agricoles, et particulièrement du blé, qui donne toujours un revenu au propriétaire du fonds sur lequel il a été recueilli, parce que, dans ces sortes de produits, il y a un travail de la nature qui ne manque jamais de trouver son prix, après que tout travail humain a été totalement satisfait. « Dans le travail de la terre, dit \Smith, la nature travaille conjointement avec l’homme ; et quoique son travail ne coûte aucune dépense, ce qu’il produit n’en a pas moins sa valeur, aussi bien que ce que produisent les ouvriers les plus chèrement payés… C’est l’œuvre de la nature qui reste après qu’on a fait la déduction ou la balance de tout ce qu’on peut regarder comme l’œuvre de l’homme. » 

Ainsi, dans le prix du blé en argent, il faut d’abord que le travail de l’homme prenne son salaire avant qu’il puisse être rien réclamé pour le travail de la nature par le propriétaire foncier qui en exerce les droits. Le degré de fertilité de la terre sur laquelle le blé a été recueilli, est une circonstance qui influe sur les conditions du partage à faire entre le fermier et le propriétaire du prix que le produit a rapporté ; mais cette circonstance de fertilité ne saurait influer le moins du monde sur les conditions entre le vendeur et l’acheteur de blé. Il importe fort peu à ce dernier que le blé qu’on lui offre provienne d’un terrain plus ou moins fertile ; c’est ce dont il s’informe le moins, et, à qualité égale dans la denrée, il prisera tout autant le blé produit sur la terre la plus fertile et la moins dispendieuse, que le blé recueilli avec deux fois plus de peine et de dépense sur un terrain ingrat et tout nouvellement mis en culture. La seule circonstance qui règle les conditions du contrat entre le vendeur et l’acheteur du blé, c’est la rareté ou l’abondance de la denrée sur le marché. si elle y est rare, l’acheteur offrira plus d’argent ; et si elle est abondante, il en offrira moins. C’est ce prix, plus ou moins avantageux au produit, qui se partage inégalement entre le fermier et le propriétaire, dans des proportions dont la différence est déterminée par la quantité plus ou moins bonne, plus ou moins améliorée de la terre qui a produit ce blé. Si la terre est d’une bonté ou fertilité moyenne, le fermier ou cultivateur aura droit de réclamer deux tiers seulement dans le prix de la chose produite. Si la terre est des plus mauvaises, le fermier ou cultivateur aura peut-être à réclamer les cinq sixièmes du produit ou du prix de ce produit. Supposez qu’au lieu de faire, comme il est d’usage, le partage de la récolte, soit en nature, soit en argent, d’après un prix convenu dans un bail, le fermier et le propriétaire demeurent en société jusqu’à la vente de la denrée au marché, en sorte que cette vente se fasse par eux en commun et par une sorte de compte en participation : qu’arrivera-t-il ? Si le prix de chaque hectolitre de blé vendu au marché est sur le pied de 16 fr. 50 cent., le fermier de la terre de bonté moyenne retiendra deux tiers ou 11 fr. sur chaque hectolitre vendu, et remettra au propriétaire foncier 5 fr. 50 cent. Mais le fermier qui a eu à cultiver la plus mauvaise terre prélèvera sur ce même prix 13 fr. 75 cent. par chaque hectolitre vendu au prix ci-dessus, et ne laissera au propriétaire du fonds que 2 fr. 75 c. pour son sixième. Que le prix du blé soit plus élevé ou plus bas, le partage se fera toujours dans les mêmes proportions.

Ainsi, ce nouveau système, imaginé par M. \Ricardo, et qui a été embrassé par M. \Malthus et quelques autres écrivains modernes, qui suppose que le prix du blé sur le marché dépend de la nature plus ou moins fertile, plus ou moins améliorée de la terre sur laquelle a été recueilli le blé vendu, est totalement opposé à ce que \Smith enseigne et à ce que la raison nous démontre. Or, dans un système d’économie politique, toutes les vérités s’enchaînent les unes aux autres, et le principe qui règle le prix du blé en argent forme un anneau si important dans cette chaîne de déductions et de conséquences, qu’on ne peut pas le déplacer sans mettre en désordre tous les rapports des valeurs et sans détruire tout le système. Ainsi, ceux qui se permettent de porter ainsi une main destructrice sur l’édifice construit par \AS, s’imposent dès lors la tâche d’en recomposer un autre, et il semble que MM. \Ricardo et \Malthus sont encore fort loin de remplir cette tâche.

Si l’homme a produit du blé au delà de ce qui était nécessaire pour sa propre subsistance, c’était pour employer ce surplus disponible à nourrir d’autres hommes qui travailleraient pour lui ou qui lui consacreraient leur temps et leurs services. Dans les temps du moyen âge, le grand propriétaire nourrissait un nombreux train de domestiques et de vassaux pour le servir dans la paix et à la guerre. De nos jours, il emploie la plus grande partie des subsistances dont il dispose, à commander du travail aux ouvriers des manufactures. Ainsi, dans toutes les périodes de la société, c’est en travail que la subsistance a été payée, et elle a dû recevoir de ce travail une quantité égale à celle qu’elle pouvait faire subsister ; car si la masse de la subsistance destinée au travail d’autrui eût excédé la somme du travail qui s’offrait en échange de la subsistance, cette quantité surabondante de subsistance n’aurait plus été produite, le propriétaire de terre n’ayant plus aucun intérêt à continuer cette production. D’un autre côté, quelque nombreuse que l’on puisse supposer la population indigente qui offre son travail et ses services en échange de subsistance, elle ne peut pas en recevoir moins qu’il ne lui en faut pour subsister, attendu que, par le seul effet des lois de la nature, cette population descendrait bientôt au niveau de la quantité de subsistance capable de l’alimenter.

\Smith a donc été bien fondé à dire : « La nature des choses a imprimé au blé une valeur réelle à laquelle ne peuvent rien changer les révolutions quelconques de son prix \emphase{en argent}… Par tout le monde, en général, la valeur d’une mesure de blé est égale à la quantité de travail qu’elle peut faire subsister, et, dans chaque lieu du monde en particulier, cette valeur est égale à la quantité de travail auquel la mesure de blé peut fournir une subsistance telle que le travail a coutume de la recevoir dans ce lieu. »

Cette proposition, d’une évidence frappante, a été contestée par M. \Malthus dans son \publication{Essai sur le principe de population}, dans lequel il paraît croire que la valeur \emphase{réelle} du blé augmente avec le renchérissement \emphase{en argent} occasionné par une prime à l'exportation. On s’est servi souvent de quelques-uns des principes même de \Smith pour attaquer sa doctrine, faute de les avoir bien compris.

L’argent, a-t-il dit, quand il est venu au marché, y représente tout le travail qu’il en a coûté pour l’extraire de la mine, le préparer et le transporter ; il doit donc commander toute la quantité de travail qu’il représente. Mais représenter du travail, c’est aussi représenter la quantité de subsistance qui alimente ce travail. C’est par représentation des subsistances consommées pour le travail d’extraction, affinage et transport, que l’argent a le pouvoir de commander une quantité égale de travail, car on n’alimente pas le travail avec de l’argent, à moins que préalablement on ne convertisse cet argent en vivres et denrées. Ainsi la subsistance commande le travail directement et non par représentation. Il importe donc peu que par un règlement ou un édit qui crée une prime à l’exportation, le quintal de blé se paye, au cours du marché, 11 fr. au lieu de 10 fr. qu’il eût valu sans cela, ce quintal ne commandera pas un quart d’heure de travail de plus. Il en est d’un surhaussement factice dans le prix pécuniaire du blé comme d’un surhaussement dans la dénomination des monnaies, qui ne change rien à leur valeur réelle.

Lorsque la première édition de cette traduction fut publiée, il y a vingt ans, le livre \publication{de la Richesse des nations} était l’objet de l’admiration générale, et, dans chaque pays, on ne songeait qu’à le naturaliser chez soi. Aujourd’hui il semble qu’il est du devoir d’un traducteur de répondre à toutes les critiques qui se sont, depuis quelque temps, élevées de toutes parts contre la doctrine de l’illustre philosophe d’Édimbourg. C’est ce que je me suis spécialement proposé de faire, à mesure que le texte m’en offrira l’occasion, dans les notes jointes à la présente édition, parmi lesquelles j’en ai au plus conservé trois ou quatre de celles qui accompagnaient la première édition. Je me suis aussi attaché, dans ces notes nouvelles, à recueillir dans l’histoire ancienne et dans notre propre histoire tous les faits qui appartiennent à la richesse des nations, au prix du blé, à la valeur de l’argent, au taux des salaires, aux prix des denrées de consommation générale, etc. ; et comme tous ces faits tendent à confirmer la doctrine de \Smith et à démentir toutes celles qu’on a voulu opposer à la sienne, j’ai tâché de fortifier la discussion par l’autorité la plus imposante de toutes, celle à laquelle on est le plus disposé à se soumettre, la leçon de l’expérience.

La théorie de la formation et de la distribution des richesses, cette science si nouvelle et qui est particulière à notre âge, est aujourd’hui comme un État naissant dont les lois fondamentales n’ont pu prendre encore leur assiette, parce qu’elles ont été sans cesse ébranlées par des dissensions intestines. Il semble que tous ceux qui se dévouent à l’enseignement de cette science cherchent à se faire des principes à part, et qu’ils n’écrivent que pour les faire prévaloir. Rien, sans doute, ne peut nuire davantage à la propagation d’une science que la division entre les maîtres et la diversité des doctrines, parce qu’il en naît une défiance et une incertitude sur le fond même de la science, et que peu de personnes sont tentées de prendre des leçons au milieu de tant d’écoles qui se combattent et qui se contredisent.

Je croirai donc avoir fait quelque bien si je puis contribuer le moins du monde à rallier aux mêmes principes ceux qui se livrent à l’étude et à la culture de l’économie politique, et à les ramener sur les pas de ce grand maître dont on ne peut quitter la trace sans courir le risque de se jeter dans de fausses routes.

\vskip 6.66mm

\hfill Août 1821\hspace{1cm}

%%%%%%%%%%%%%%%%%%%%%%%%%%%%%%%%%%%%%%%%%%%%%%%%%%%%%%%%%%%%%%%%%%%%%%%%%%%%%%%%
%                                                                              %
%                                 Introduction                                 %
%                                                                              %
%%%%%%%%%%%%%%%%%%%%%%%%%%%%%%%%%%%%%%%%%%%%%%%%%%%%%%%%%%%%%%%%%%%%%%%%%%%%%%%%

\chapter*{\textcolor{coquelicot}{Introduction}}
\markboth{Introduction}{}

\notedebasdepagesansmarque{Le docteur \Smith n’a pas établi le sens précis qu’il attachait au mot richesse quoique, le plus souvent, il le définisse comme « l\emphase{e produit annuel de la terre et du travail}. » On a cependant justement reproché à cette définition de se reporter aux sources de la richesse avant qu’on ne sût ce qu’était la richesse elle-même, et de comprendre les produits inutiles de la terre avec ceux que l’homme s’approprie et ceux dont il jouit.

Nous sommes portés à penser qu’on doit considérer la richesse comme désignant tous les articles ou produits qui sont nécessaires, utiles ou agréables à l’homme, et qui, en même temps, sont doués d’une valeur échangeable ; cette dernière qualité exprimant le pouvoir ou la faculté d’\emphase{être échangé} contre une telle quantité de \emphase{travail}, contre une ou plusieurs marchandises ou produits obtenus par les voies seules du travail, ou encore la faculté de \emphase{les acheter}. Cette définition sépare la \emphase{richesse}, de ces objets que la Providence répand gratuitement et a l’infini sur l’homme. Ces derniers, quoique susceptibles d’une très-haute utilité, sont nécessairement tous dépourvus de valeur échangeable ; car il est évident que nul ne fera des efforts ni ne donnera les produits de son \emphase{industrie} pour obtenir ce qu’il peut avoir en tout temps et en quantités illimitées sans travail. C’est pourquoi les bases sur lesquelles on a établi une distinction entre la \emphase{richesse} et les articles ou produits \emphase{non doués de valeur échangeable} sont manifestes et ont été universellement reconnues. On ne dit pas d’un homme qu’il est \emphase{riche} parce qu’il peut puiser sans cesse dans le domaine inépuisable de l’air atmosphérique, ce privilège lui étant commun avec tous et, par cela, ne pouvant être la source d’aucune distinction : mais on lui accordera le titre de riche en proportion directe de la faculté qu’il aura de posséder ces objets de nécessité, d’utilité ou de luxe qui ne peuvent être produits que par l’action du travail ou de l’industrie et qui peuvent devenir la propriété et la jouissance d’un individu à l’exclusion des autres. Ces articles ou ces produits ont seuls une valeur échangeable, et seuls, ils peuvent constituer ce qu’on appelle de la \emphase{richesse}. L’Économie politique n’étudie que les résultats de l’industrie humaine. Cette science peut véritablement être appelée la \emphase{science des valeurs} : car tout ce qui ne possède pas \emphase{valeur échangeable} ou qui ne peut être reçu comme équivalent d’un autre objet dont la production ou l’acquisition a exigé du travail, ne peut jamais être compris dans les limites de \emphase{ses recherches}.}{\MacCulloch}

Le \emphase{travail annuel} d’une nation est le fonds primitif qui fournit à sa consommation annuelle toutes les choses nécessaires et commodes à la vie ; et ces choses sont toujours ou le produit immédiat de ce travail, ou achetées des autres nations avec ce produit.

Ainsi, selon que ce produit, ou ce qui est acheté avec ce produit, se trouvera être dans une proportion plus ou moins grande avec le nombre des consommateurs, la nation sera plus ou moins bien pourvue de toutes les choses nécessaires ou commodes dont elle éprouvera le besoin\notedebasdepage[\Sismondi]{Nous professons, avec \AS, que le travail est la seule origine de la richesse, que l’économie est le seul moyen de l’accumuler ; mais nous ajoutons que la jouissance est le seul but de cette accumulation, et qu’il n’y a accroissement de la richesse nationale que quand il y a aussi accroissement des jouissances nationales.}

Or, dans toute nation, deux circonstances différentes déterminent cette proportion. Premièrement, l’habileté, la dextérité et l’intelligence qu’on y apporte généralement dans l’application du travail ; deuxièmement, la proportion qui s’y trouve entre le nombre de ceux qui sont occupés à un travail utile et le nombre de ceux qui ne le sont pas. Ainsi, quels que puissent être le sol, le climat et l’étendue du territoire d’une nation, nécessairement l’abondance ou la disette de son approvisionnement annuel, relativement à sa situation particulière, dépendra de ces deux circonstances.

L’abondance ou l’insuffisance de cet approvisionnement dépend plus de la première de ces deux circonstances que de la seconde. Chez les nations sauvages qui vivent de la chasse et de la pêche, tout individu en état de travailler est plus ou moins occupé à un travail utile, et tâche de pourvoir, du mieux qu’il peut, à ses besoins et à ceux des individus de sa famille ou de sa tribu qui sont trop jeunes, trop vieux ou trop infirmes pour aller à la chasse ou à la pêche. Ces nations sont cependant dans un état de pauvreté suffisant pour les réduire souvent, ou du moins pour qu’elles se croient réduites, à la nécessité tantôt de détruire elles-mêmes leurs enfants, leurs vieillards et leurs malades, tantôt de les abandonner aux horreurs de la faim ou à la dent des bêtes féroces. Au contraire, chez les nations civilisées et en progrès, quoiqu’il y ait un grand nombre de gens tout à fait oisifs et beaucoup d’entre eux qui consomment un produit de travail décuple et souvent centuple de ce que consomme la plus grande partie des travailleurs, cependant la somme du produit du travail de la société est si grande, que tout le monde y est souvent pourvu avec abondance, et que l’ouvrier, même de la classe la plus basse et la plus pauvre, s’il est sobre et laborieux, peut jouir, en choses propres aux besoins et aux aisances de la vie, d’une part bien plus grande que celle qu’aucun sauvage pourrait jamais se procurer\notedebasdepage[\AB]{Cela devrait être ; mais le développement tout à fait vicieux de l’industrie manufacturière a changé de nos jours ces généreuses espérances eh désappointement bien amer. On en verra plus loin les motifs.}.

Les causes qui perfectionnent ainsi le pouvoir productif du travail et l’ordre suivant lequel ses produits se distribuent naturellement entre les diverses classes de personnes dont se compose la société, feront la matière du \emphase{premier livre} de ces recherches.

Quel que soit, dans une nation, l’état actuel de son habileté, de sa dextérité et de son intelligence dans l’application du travail, tant que cet état reste le même, l’abondance ou la disette de sa provision\notedebasdepage[\AB]{Le mot \emphase{provision} exprime d’une manière très-imparfaite le terme anglais \anglais{supply}, surtout lorsqu’il est employé en qualité de verbe : \anglais{to supply}. \emphase{Pourvoir}, dans ce cas, est une expression plus juste qu’\emphase{approvisionner}. Nous aurons plus d’une occasion de signaler la difficulté de traduire exactement certaines expressions de la langue économique créée par \AS.} annuelle dépendra nécessairement de la proportion entre le nombre des individus employés à un travail utile, et le nombre de ceux qui ne le sont pas. Le nombre des travailleurs utiles et productifs est partout, comme on le verra par la suite, en proportion de la quantité du Capital employé à les mettre en œuvre, et de la manière particulière dont ce capital est employé. Le \emphase{second livre} traite donc de la nature du capital et de la manière dont il s’accumule graduellement, ainsi que des différentes quantités de travail qu’il met en activité, selon les différentes manières dont il est employé.

Des nations qui ont porté assez loin l’habileté, la dextérité et l’intelligence dans l’application du travail, ont suivi des méthodes fort différentes dans la manière de le diriger ou de lui donner une impulsion générale, et ces méthodes n’ont pas toutes été également favorables à l’augmentation de la masse de ses produits. La politique de quelques nations a donné un encouragement extraordinaire à l’industrie des campagnes ; celle de quelques autres, à l’industrie des villes. Il n’en est presque aucune qui ait traité tous les genres d’industrie avec égalité et avec impartialité. Depuis la chute de l’empire romain, la politique de l’Europe a été plus favorable aux arts, aux manufactures et au commerce, qui sont l’industrie des villes, qu’à l’agriculture, qui est celle des campagnes\notedebasdepage[\AB]{Nous ne doutons pas qu’une réaction s’opère bientôt en faveur de l’agriculture, à mesure que la production manufacturière se complique et s’encombre sous l’influence des machines et des prohibitions. On a trop longtemps délaissé cette source importante de la prospérité publique, que les \emphase{physiocrates} avaient trop exaltée ; une ère nouvelle s’ouvre pour elle dans le monde et nous croyons qu’elle sera très-brillante, dès que les capitaux, fatigués des mécomptes de l’industrie, prendront la route de l’agriculture, et surtout quand l’amélioration de notre régime hypothécaire assurera aux cultivateurs les ressources du crédit à des conditions aussi favorables que celles dont jouissent les autres classes de producteurs.}. Les circonstances qui semblent avoir introduit et établi cette politique sont exposées dans le \emphase{troisième livre}\notedebasdepage[\Buchanan]{La pensée du Dr. \Smith sur la préférence accordée par les États modernes au commerce, au détriment de l’agriculture, paraît ne reposer que sur une interprétation erronée des faits. L’état des mœurs était, sans aucun doute, défavorable à l’agriculture, mais il n’y avait là aucune préférence méditée d’une industrie à l’autre, et il ne paraît pas non plus que l’ordre naturel des progrès ait été transformé par la politique de l’Europe. En fait, l’énoncé du Dr. \Smith diffère de sa théorie, car il montre que dans quelques pays au moins, l’agriculture avait, par circonstance, devancé le commerce.}.

Quoique ces différentes méthodes aient peut-être dû leur première origine aux préjugés et à l’intérêt privé de quelques classes particulières, qui ne calculaient ni ne prévoyaient les conséquences qui pourraient en résulter pour le bien-être général de la société, cependant elles ont donné lieu à différentes théories d’économie politique, dont les unes exagèrent l’importance de l’industrie qui s’exerce dans les villes, et les autres celle de l’industrie des campagnes. Ces théories ont eu une influence considérable, non-seulement sur les opinions des hommes instruits, mais même sur la conduite publique des princes et des États. J’ai tâché, dans le \emphase{quatrième livre}, d’exposer ces différentes théories aussi clairement qu’il m’a été possible, ainsi que les divers effets qu’elles ont produits en différents siècles et chez différents peuples.

Ces quatre premiers livres traitent donc de ce qui constitue le revenu de la masse du peuple, ou de la nature de ces fonds qui, dans les différents âges et chez les différents peuples, ont fourni à leur consommation annuelle. Le \emphase{cinquième et dernier livre} traite du revenu du Souverain ou de la République. J’ai tâché de montrer dans ce livre : 1° quelles sont les dépenses nécessaires du souverain ou de la république, lesquelles de ces dépenses doivent être supportées par une contribution générale de toute la société, et lesquelles doivent l’être par une certaine portion seulement ou par quelques membres particuliers de la société ; 2° quelles sont les différentes méthodes de faire contribuer la société entière à l’acquit des dépenses qui doivent être supportées par la généralité du peuple, et quels sont les principaux avantages et inconvénients de chacune de ces méthodes ; 3° enfin, quelles sont les causes qui ont porté presque tous les gouvernements modernes à engager ou à hypothéquer quelque partie de ce revenu, c’est-à-dire à contracter des dettes, et quels ont été les effets de ces dettes sur la véritable richesse de la société, sur le produit annuel de ses terres et de son travail\notedebasdepage[\AB]{Dès le début de son ouvrage, \AS sépare nettement sa doctrine de celle des \emphase{économistes} ou \emphase{physiocrates} du dix-huitième siècle, qui ne voulaient reconnaître d’autre source de richesse que l’agriculture et d’autre produit que le \emphase{produit net}. Les partisans de l’école mercantile avaient commis la même erreur, en attribuant exclusivement au commerce la production de la richesse, représentée à leurs yeux par l’or et par l’argent. En réhabilitant ainsi le travail en qualité d’élément indispensable de la production, \AS a ouvert une carrière nouvelle à l’économie politique ; il a fait connaître la véritable source de tous nos revenus ; il a mis en discussion la grande question de la distribution des profits et des salaires, celle de la liberté des industries et une foule d’autres qu’il n’aura pas eu le bonheur de résoudre, mais qu’il a admirablement posées.}.


%%%%%%%%%%%%%%%%%%%%%%%%%%%%%%%%%%%%%%%%%%%%%%%%%%%%%%%%%%%%%%%%%%%%%%%%%%%%%%%%
%                                                                              %
%                                   Livre 1                                    %
%                                                                              %
%%%%%%%%%%%%%%%%%%%%%%%%%%%%%%%%%%%%%%%%%%%%%%%%%%%%%%%%%%%%%%%%%%%%%%%%%%%%%%%%

\part{Des\\ \textcolor{bourgogne}{causes {\relsize{-3}qui ont} perfectionné {\relsize{-3}les} facultés productives {\relsize{-3}du} travail}, \\et de\\ \textcolor{bourgogne}{l’ordre {\relsize{-3}suivant lequel} ses produits {\relsize{-3}se} distribuent naturellement {\relsize{-3}dans les différentes} classes {\relsize{-3}du} peuple}}
\markboth{Des causes qui ont perfectionné les facultés productives du travail, et de l’ordre suivant lequel ses produits se distribuent naturellement dans les différentes classes du peuple}{}

%%%%%%%%%%%%%%%%%%%%%%%%%%%%%%%%%%%%%%%%%%%%%%%%%%%%%%%%%%%%%%%%%%%%%%%%%%%%%%%%
%                                  Chapitre 1                                  %
%%%%%%%%%%%%%%%%%%%%%%%%%%%%%%%%%%%%%%%%%%%%%%%%%%%%%%%%%%%%%%%%%%%%%%%%%%%%%%%%

\chapter{{\relsize{-5}De la} \\\textcolor{coquelicot}{division \\{\relsize{-5}du} travail}}
\markboth{De la division du travail}{}

Les plus grandes améliorations dans la puissance productive du travail, et la plus grande partie de l’habileté, de l’adresse, de l’intelligence avec laquelle il est dirigé ou appliqué, sont dues, à ce qu’il semble, à la \emphase{division du travail}\notedebasdepage[\AB]{Plusieurs économistes avaient entrevu avant \AS les avantages de la division du travail. Le célèbre publiciste Beccaria les avait signalés, dès l’année 1769, dans son cours d’Économie politique professé à Milan ; mais l’honneur de cette \emphase{observation} appartiendra toujours à \AS, parce que c’est lui qui a le premier démontré l’importance de la division du travail et son influence sur le développement de la production.}.

On se fera plus aisément une idée des effets de la \emphase{division du travail} sur l’industrie générale de la société, si l’on observe comment ces effets opèrent dans quelques manufactures particulières. On suppose communément que cette division est portée le plus loin possible dans quelques-unes des manufactures où se fabriquent des objets de peu de valeur. Ce n’est pas peut-être que réellement elle y soit portée plus loin que dans des fabriques plus importantes ; mais c’est que, dans les premières, qui sont destinées à de petits objets demandés par un petit nombre de personnes, la totalité des ouvriers qui y sont employés est nécessairement peu nombreuse, et que ceux qui sont occupés à chaque différente branche de l’ouvrage peuvent souvent être réunis dans un atelier et placés à la fois sous les yeux de l’observateur. Au contraire, dans ces grandes manufactures destinées à fournir les objets de consommation de la masse du peuple, chaque branche de l’ouvrage emploie un si grand nombre d’ouvriers, qu’il est impossible de les réunir tous dans le même atelier. On ne peut guère voir à la fois que les ouvriers employés à une seule branche de l’ouvrage. Ainsi, quoique dans ces ma­nu­factures l’ouvrage soit peut-être en réalité divisé en un plus grand nombre de parties que dans celles de la première espèce, cependant la division y est moins sensible et, par cette raison, elle y a été moins bien observée.

Prenons un exemple dans une manufacture de la plus petite importance, mais où la \emphase{division du travail} s’est fait souvent remarquer : une manufacture d’épingles. Un homme qui ne serait pas façonné à ce genre d’ouvrage, dont la \emphase{division du travail} a fait un métier particulier, ni accoutumé à se servir des instruments qui y sont en usage, dont l’invention est probablement due encore à la \emphase{division du travail}, cet ouvrier, quelque adroit qu’il fût, pourrait peut-être à peine faire une épingle dans toute sa journée, et certainement il n’en ferait pas une vingtaine. Mais de la manière dont cette industrie est maintenant conduite, non-seulement l’ouvrage entier forme un métier particulier, mais même cet ouvrage est divisé en un grand nombre de branches, dont la plupart constituent autant de métiers particuliers. Un ouvrier \emphase{tire le fil à la bobine}, un autre le \emphase{dresse}, un troisième \emphase{coupe la dressée}, un quatrième \emphase{empointe}, un cinquième est employé à émoudre le bout qui doit recevoir la \emphase{tête}. Cette \emphase{tête} est elle-même l’objet de deux ou trois opérations séparées : la \emphase{frapper} est une besogne particulière ; \emphase{blanchir} les épingles en est une autre ; c’est même un métier distinct et séparé que de \emphase{piquer} les papiers et d’y \emphase{bouter} les épingles ; enfin, l’important travail de faire une épingle est divisé en dix-huit opérations distinctes ou environ, lesquelles, dans certaines fabriques, sont remplies par autant de mains différentes, quoique dans d’autres le même ouvrier en remplisse deux ou trois. J’ai vu une petite manufacture de ce genre qui n’employait que dix ouvriers, et où, par conséquent, quelques-uns d’eux étaient chargés de deux ou trois opérations. Mais, quoique la fabrique fût fort pauvre et, par cette raison, mal outillée, cependant, quand ils se mettaient en train, ils ve­naient à bout de faire entre eux environ douze livres d’épingles par jour ; or, chaque livre contient au delà de quatre mille épingles de taille moyenne. Ainsi, ces dix ouvriers pouvaient faire entre eux plus de quarante-huit milliers d’épingles dans une journée ; donc, chaque ouvrier, faisant une dixième partie de ce produit, peut être considéré comme donnant dans sa journée quatre mille huit cents épingles. Mais s’ils avaient tous travaillé à part et indépendamment les uns des autres, et s’ils n’avaient pas été façonnés à cette besogne particulière, chacun d’eux assurément n’eût pas fait vingt épingles, peut-être pas une seule, dans sa journée, c’est-à-dire pas, à coup sûr, la deux-cent-quarantième partie, et pas peut-être la quatre-mille-huit-centième partie de ce qu’ils sont maintenant en état de faire, en conséquence d’une division et d’une combinaison convenables de leurs différentes opérations\notedebasdepage[\AB]{Le progrès général de toutes les industries est dû aux applications nombreuses de la division du travail. Partout elle se substitue aux anciennes méthodes et transforme le travail individuel en travail d’association : mais chaque ouvrier n’est plus qu’un rouage, au lieu d’être un appareil complet ; il devient plus habile, et encore pas toujours, mais plus dépendant. Qu’est-ce qu’un homme qui ne sait faire, même parfaitement, que des têtes d’épingles ou des pointes d’aiguilles ?}.

Dans tout autre art et manufacture, les effets de la \emphase{division du travail} sont les mêmes que ceux que nous venons d’observer dans la fabrique d’une épingle, quoique dans un grand nombre le travail ne puisse pas être aussi subdivisé ni réduit à des opé­ra­tions d’une aussi grande simplicité. Toutefois, dans chaque art, la \emphase{division du travail}, aussi loin qu’elle peut y être portée, amène un accroissement proportionnel dans la puissance productive du travail. C’est cet avantage qui paraît avoir donné naissance à la séparation des divers emplois et métiers. Aussi, cette séparation est en général poussée plus loin dans les pays qui jouissent du plus haut degré de perfectionnement ; ce qui, dans une société encore un peu grossière, est l’ouvrage d’un seul homme, devient, dans une société plus avancée, la besogne de plusieurs. Dans toute société avancée, un fermier en général n’est que fermier, un fabricant n’est que fabricant. Le travail nécessaire pour produire complètement un objet manufacturé est aussi presque toujours divisé entre un grand nombre de mains. Que de métiers différents sont employés dans chaque branche des ouvrages manufacturés, de toile ou de laine, depuis l’ouvrier qui travaille à faire croître le lin et la laine, jusqu’à celui qui est employé à blanchir et à tisser la toile ou à teindre et à lustrer le drap ! Il est vrai que la nature de l’agriculture ne comporte pas une aussi grande sub­division de travail que les manufactures, ni une séparation aussi complète des travaux. Il est impossible qu’il y ait, entre l’ouvrage du nourrisseur de bestiaux et du fermier, une démarcation aussi bien établie qu’il y en a communément entre le métier du charpentier et celui du forgeron. Le tisserand et le fileur sont presque toujours deux personnes différentes ; mais le laboureur, le semeur et le moissonneur sont souvent une seule et même personne. Comme les temps propres à ces différents genres de tra­vaux dépendent des différentes saisons de l’année, il est impossible qu’un homme puisse trouver constamment à s’employer à chacun d’eux. C’est peut-être l’impossi­bilité de faire une séparation aussi entière et aussi complète des différentes branches du travail appliqué à l’agriculture, qui est cause que, dans cet art, la puissance pro­ductive du travail ne fait pas des progrès aussi rapides que dans les manufactures. À la vérité, les peuples les plus opulents l’emportent, en général, sur leurs voisins aus­si bien en agriculture que dans les autres industries ; mais cependant leur supériorité se fait communément beaucoup plus sentir dans ces dernières. Leurs terres sont, en général, mieux cultivées et, y ayant consacré plus de travail et de dépense, ils en reti­rent un produit plus grand, eu égard à l’étendue et à la fertilité naturelle du sol. Mais la supériorité de ce produit n’excède guère la proportion de la supériorité de travail et de dépense. En agriculture, le travail du pays riche n’est pas toujours beau­coup plus productif que celui du pays pauvre, ou du moins cette différence n’est jamais aussi forte qu’elle l’est ordinairement dans les manufactures. Ainsi, le blé d’un pays riche, à égal degré de bonté, ne sera pas toujours, au marché, à meilleur compte que celui d’un pays pauvre\notedebasdepage[\AB]{Cette grave question a été traitée d’une manière très-remarquable dans l’ouvrage de M. \Ricardo : \publication{Principes de l’économie politique et de l’impôt}, chapitre du fermage, tom. Ier, pages 57-94 de la traduction de Constancio, annotée par \JBS. La fameuse théorie du fermage de \Ricardo est exposée tout entière dans ce chapitre.}. Le blé de Pologne, à bonté égale, est à aussi bon marché que celui de France, malgré la supériorité de ce dernier pays en opulence et en industrie. Le blé de France, dans les provinces à blé, est tout aussi bon, la plupart des années, et presque au même prix que le blé d’Angleterre, quoique peut-être la France soit inférieure à l’Angleterre du côté de l’opulence et de l’industrie\notedebasdepage[\Buchanan]{En opposant l’agriculture d’une nation riche à ses manufactures, le Dr. \Smith méconnaît le principe qui règle le prix du blé. Sa conclusion ferait croire que, parce qu’il aura coûté moins cher d’apporter le blé au marché, ce blé sera vendu à plus bas prix. — Mais le prix de vente du blé n’est pas réglé par les frais de production ; et quoiqu’il pût être produit pour rien, il ne serait pas, pour cela, vendu à plus bas prix. C’est pourquoi ce qu’une riche nation épargne dans les dépenses de culture, ne sert pas à réduire le prix, mais à augmenter le revenu.}. Toutefois, les terres d’Angleterre sont mieux cultivées que celles de France, et celles-ci sont, à ce qu’on dit, beaucoup mieux cultivées que celles de Pologne. Mais quoique les pays pauvres, malgré l’infé­riorité de leur culture, puissent, en quelque sorte, rivaliser avec les pays riches pour la bonté et le bon marché du blé, cependant ils ne peuvent prétendre à la même concur­rence en fait de manufactures, du moins si ces manufactures sont en rapport avec le sol, le climat et la situation du pays riche. Les soieries de France sont plus belles et à meilleur compte que celles d’Angleterre, parce que les manufactures de soie ne conviennent pas au climat d’Angleterre aussi bien qu’à celui de France, du moins sous le régime des forts droits dont on a chargé chez nous l’importation des soies écrues. Mais la quincaillerie d’Angleterre et ses gros lainages sont sans comparaison bien supérieurs à ceux de France\notedebasdepage[\AB]{Les événements ont donné un démenti à cette assertion d’\AS. Depuis les réformes que M. \Huskisson a fait subir au système restrictif de l’Angleterre, l’industrie des soieries a fait les plus grands progrès dans ce pays. L’Angleterre rivalise aujourd’hui avec la France dans la fabrication des tissus de soie unis. Voyez sur cotte importante question les doctrines soutenues par M. \Huskisson, dans la collection de ses œuvres intitulée : \publication{The speeches of the right honourable William \Huskisson}, tom. Il, pages 465-530 ; ainsi que la grande enquête de 1852 publiée par le gouvernement anglais sous ce titre : \publication{Report from select committee on the silk trade}, in-fol. de 1050 pages.}, et beaucoup moins chers à qualité égale. En Pologne, dit-on, à peine y a-t-il des manufactures, si ce n’est quelques fabriques où se font les plus grossiers ustensiles de ménage, et dont aucun pays ne saurait se passer\notedebasdepage[\MacCulloch]{L’opinion de l’auteur sur l’impossibilité de pousser la division du travail aussi loin dans l’agriculture que dans les manufactures pu le commerce, est indubitablement exacte ; mais cette circonstance n’est pas, comme \Smith le suppose, la seule ou même la principale raison pour laquelle le prix du blé, dans les pays de haute agriculture, est généralement aussi élevé, et souvent beaucoup plus, que dans les pays qui sont comparativement mal cultivés et barbares. Si une supériorité agronomique, si une plus grande subdivision des instruments ruraux et une introduction plus vaste des machines dans les travaux des champs suffisaient pour déterminer le prix des produits \emphase{bruts} ou naturels, ce prix serait certainement plus bas en Angleterre qu’en Pologne ou en Russie. Mais il est évident que le prix du blé dans les différents pays ne dépend qu’en partie des systèmes de culture, et qu’il est en même temps sérieusement soumis à la différence de fertilité des terres cultivées. C’est ce fait que \Smith a, par une étrange inadvertance, tournent négligé. La rareté de la population chez les nations peu civilisées, n’attire nécessairement la culture que sur les terres de la plus haute fertilité ; mais à mesure que la société avance et que la population s’accroît, il devient urgent de s’adresser à des terres moins fertiles. Dès lors, le produit de ces terres, par l’accroissement du capital et du travail que leur culture réclame, doit être relativement cher. Il a été établi, par quelques-uns des rapports sur l’état de l’agriculture en 1821, examiné par un comité de la chambre des communes, que le produit des terres livrées à la culture en Angleterre et dans le pays de Galles, évalué en froment, variait depuis trente-six et quarante boisseaux jusqu’à huit et neuf boisseaux par \emphase{acre}. L’alimentation nécessaire ne pourrait pas être acquise si l’on ne cultivait pas ces terres inférieures, et c’est cette nécessité de recourir aux terrains de moindre fertilité, qui devient la cause réelle de l’élévation relative du prix du blé et des autres produits naturels ou \emphase{bruts} dans les pays de grande population. Il serait superflu d’ajouter que ce prix y serait encore bien plus haut s’il ne trouvait pas un palliatif dans la supériorité des méthodes agronomiques et les progrès qui s’y opèrent chaque jour.}.

Cette grande augmentation dans la quantité d’ouvrage qu’un même nombre de bras est en état de fournir, en conséquence de la \emphase{division du travail}, est due à trois circonstances différentes : premièrement, à un accroissement d’habileté chez chaque ouvrier individuellement ; deuxièmement, à l’épargne du temps qui se perd ordinairement quand on passe d’une espèce d’ouvrage à une autre ; et troisièmement enfin, à l’invention d’un grand nombre de machines qui facilitent et abrègent le travail, et qui permet­tent à un homme de remplir la tâche de plusieurs.

Premièrement, l’accroissement de l’habileté dans l’ouvrier augmente la quantité d’ouvrage qu’il peut accomplir, et la \emphase{division du travail}, en réduisant la tâche de cha­que homme à quelque opération très-simple et en faisant de cette opération la seule occupation de sa vie, lui fait acquérir nécessairement une très-grande dextérité. Un forgeron ordinaire qui, bien qu’habitué à manier le marteau, n’a cependant jamais été habitué à faire des clous, s’il est obligé par hasard de s’essayer à en faire, viendra très-difficilement à bout d’en faire deux ou trois cents dans sa journée ; encore seront-ils fort mauvais. Un forgeron qui aura été accoutumé à en faire, mais qui n’en aura pas fait son unique métier, aura peine, avec la plus grande diligence, à en fournir dans un jour plus de huit cents ou d’un millier. Or, j’ai vu des jeunes gens au-dessous de vingt ans, n’ayant jamais exercé d’autre métier que celui de faire des clous, qui, lorsqu’ils étaient en train, pouvaient fournir chacun plus de deux mille trois cents clous par jour. Toutefois, la façon d’un clou n’est pas une des opérations les plus simples. La même personne fait aller les soufflets, attise ou dispose le feu quand il en est besoin, chauffe le fer et forge chaque partie du clou. En forgeant la tête, il faut qu’elle change d’outils. Les différentes opérations dans lesquelles se subdivise la façon d’une épingle ou d’un bouton de métal sont toutes beaucoup plus simples, et la dextérité d’une personne qui n’a pas eu dans sa vie d’autres occupations que celles-là, est ordinairement beaucoup plus grande. La rapidité avec laquelle quelques-unes de ces opérations s’exécutent dans les fabriques passe tout ce qu’on pourrait imaginer ; et ceux qui n’en ont pas été témoins ne sauraient croire que la main de l’homme fût capable d’acquérir autant d’agilité\notedebasdepage[\AB]{Il faut lire dans le spirituel opuscule de \Lemontey, \publication{Raison et folie}, la triste contre-partie de ce brillant tableau.}.

En second lieu, l’avantage qu’on gagne à épargner le temps qui se perd commu­nément en passant d’une sorte d’ouvrage à une autre, est beaucoup plus grand que nous ne pourrions le penser au premier coup d’œil. Il est impossible de passer très-vite d’une espèce de travail à une autre qui exige un changement de place et des outils diffé­rents. Un tisserand de la campagne, qui exploite une petite ferme, perd une grande partie de son temps à aller de son métier à son champ, et de son champ à son métier. Quand les deux métiers peuvent être établis dans le même atelier, la perte du temps est sans doute beaucoup moindre ; néanmoins elle ne laisse pas d’être consi­dérable. Ordinairement, un homme perd un peu de temps en passant d’une besogne à une autre. Quand il commence à se mettre à ce nouveau travail, il est rare qu’il soit d’abord bien en train ; il n’a pas, comme on dit, le cœur à l’ouvrage, et pendant quelques moments il niaise plutôt qu’il ne travaille de bon cœur. Cette habitude de flâner et de travailler sans application et avec nonchalance est naturelle à l’ouvrier de la campagne, ou plutôt il la contracte nécessairement, étant obligé de changer d’ouvrage et d’outils à chaque demi-heure, et de mettre la main chaque jour de sa vie à vingt besognes différentes ; elle le rend presque toujours paresseux et incapable d’un travail sérieux et appliqué, même dans les occasions où il est le plus pressé d’ouvrage. Ainsi, indépendamment de ce qui lui manque en dextérité, cette seule raison diminuera considérablement la quantité d’ouvrage qu’il sera en état d’accomplir.

En troisième et dernier lieu, tout le monde sent combien l’emploi de machines propres à un ouvrage abrège et facilite le travail. Il est inutile d’en chercher des exem­ples. Je ferai remarquer seulement qu’il semble que c’est à la \emphase{division du travail} qu’est originairement due l’invention de toutes ces machines propres à abréger et à faciliter le travail. Quand l’attention d’un homme est toute dirigée vers un objet, il est bien plus propre à découvrir les méthodes les plus promptes et les plus aisées pour l’atteindre, que lorsque cette attention embrasse une grande variété de choses. Or, en conséquen­ce de la \emphase{division du travail}, l’attention de chaque homme est naturellement fixée tout entière sur un objet très-simple. On doit donc naturellement attendre que quelqu’un de ceux qui sont employés à une branche séparée d’un ouvrage, trouvera bientôt la méthode la plus courte et la plus facile de remplir sa tâche particulière, si la nature de cette tâche permet de l’espérer. Une grande partie des machines employées dans ces manufactures où le travail est le plus subdivisé, ont été originairement inventées par de simples ouvriers qui, naturellement, appliquaient toutes leurs pensées à trouver les moyens les plus courts et les plus aisés de remplir la tâche particulière qui faisait leur seule occupation. Il n’y a personne d’accoutumé à visiter les manufactures, à qui on n’ait fait voir une machine ingénieuse imaginée par quelque pauvre ouvrier pour abréger et faciliter sa besogne. Dans les premières machines à feu, il y avait un petit garçon continuellement occupé à ouvrir et à fermer alternativement la communication entre la chaudière et le cylindre, suivant que le piston montait ou descendait. L’un de ces petits garçons, qui avait envie de jouer avec ses camarades, observa qu’en mettant un cordon au manche de la soupape qui ouvrait cette communication, et en attachant ce cordon à une autre partie de la machine, cette soupape s’ouvrirait et se fermerait sans lui, et qu’il aurait la liberté de jouer tout à son aise. Ainsi, une des découvertes qui a le plus contribué à perfectionner ces sortes de machines depuis leur invention, est due à un enfant qui ne cherchait qu’à s’épargner de la peine. 

Cependant il s’en faut de beaucoup que toutes les découvertes tendant à perfectionner les machines et les outils aient été faites par les hommes destinés à s’en servir personnellement. Un grand nombre est dû à l’industrie des constructeurs de machines, de­puis que cette industrie est devenue l’objet d’une profession particulière, et quelques-unes à l’habileté de ceux qu’on nomme \emphase{savants} ou \emphase{théoriciens}, dont la profession est de ne rien faire, mais de tout observer, et qui, par cette raison, se trouvent souvent en état de combiner les forces des choses les plus éloignées et les plus dissemblables. Dans une société avancée, les fonctions philosophiques ou spéculatives deviennent, comme tout autre emploi, la principale ou la seule occupation d’une classe particulière de citoyens. Cette occupation, comme tout autre, est aussi subdivisée en un grand nombre de branches différentes, dont chacune occupe une classe particulière de savants, et cette subdivision du travail, dans les sciences comme en toute autre chose, tend à accroître l’habileté et à épargner du temps. Chaque individu acquiert beaucoup plus d’expérience et d’aptitude dans la branche particulière qu’il a adoptée ; il y a au total plus de travail accompli, et la somme des connaissances en est considérablement augmentée\notedebasdepage[\AB]{\AS aurait fort à rabattre des espérances que lui donnaient les phénomènes de la division du travail, s’il voyait aujourd’hui à quel état de misère et d’abjection l’exagération de ce principe a réduit les classes ouvrières dans son pays. La condition des esclaves dans l’antiquité et celle des noirs dans nos colonies sont mille fois préférables au sort des tisserands et de certains fileurs, en Angleterre. Voyez à ce sujet le livre de M. \Buret, intitulé : \publication{De la misère des classes ouvrières en Angleterre et en France} ; 2 vol. in-8°, 1811, et l’enquête dirigée parle respectable M. \Fletcher, (\publication{Handloom Weavers inquiry}) qui a été publiée en 1840 par les ordres du parlement.}.

Cette grande multiplication dans les produits de tous les différents arts et métiers, résultant de la \emphase{division du travail}, est ce qui, dans une société bien gouvernée, donne lieu à cette opulence générale qui se répand jusque dans les dernières classes du peuple\notedebasdepage[\Garnier]
{Mais quels sont ceux qui recueillent les fruits de cette amélioration dans les facultés productives du travail ? Quelle est, parmi les différentes classes de la société, celle qui ajoute à ses commodités et jouissances personnelles tout ce qui résulte de cette augmentation de produits ? Quelle est celle qui profite directement de l’abondance générale des richesses et qui ne les reverse sur les autres membres de la société qu’en raison du prix qu’elle met à leurs Services et selon la proportion plus ou moins libérale dans laquelle elle juge à propos de payer ces services ? Nous verrons, en suivant les développements de la doctrine de notre auteur, que, dans cette masse toujours croissante des produits du travail, la part assignée au travail, prise en masse et quelle que soit l’inégalité des portions dispensées à chaque individu salarié, reste toujours nécessairement bornée à la quantité de subsistances indispensable pour alimenter le travail et pour l’entretenir ; que la part de ces produits attribuée à l’entrepreneur du travail, comme profit de ses avances, est également limitée par la somme de capital indispensable pour tenir le travail en activité, et que l’accumulation des capitaux étant toujours croissante à mesure que le travail donne plus de produits, le taux de profit afférent à chaque portion de capital employé va toujours en baissant a proportion qu’augmente la somme totale des capitaux qui concourent au même service ; qu’enfin, après la déduction de ces salaires et de ces profils, qui est la charge inhérente au travail, sans laquelle il n’aurait pu être exécuté, tout le surplus des produits appartient exclusivement aux propriétaires fonciers, qui l’appliquent selon qu’il leur plaît à leur satisfaction personnelle ; qu’ils peuvent arbitrairement le consacrer soit à leurs besoins, soit à ces fantaisies qui naissent de l’opulence et d’un pouvoir illimité sur le travail d’autrui ; que tout perfectionnement dans les moyens de travail, tout procédé, toute découverte qui tend à multiplier ses produits, tourne directement ou indirectement au profit de cette classe de la satiété, parce que le droit de propriété lui conférant celui de distribuer à son gré les subsistances et les matières premières, elle s’enrichit de toute l’utilité que le travail peut acquérir et de toute la valeur ajoutée aux matières ; que, dans cette relation entre les diverses classes de la société, qui ne consiste qu’en services rendus et en services commandés, ceux qui vivent de services sont toujours forcés par la concurrence de les offrir au plus bas prix possible, tandis que \emphase{le territoire} qui paye et entretient ces services, restant circonscrit dans les mêmes limites, et ne pouvant croître en étendue et en fertilité à proportion de l’accroissement des consommations de son produit, \emphase{les maîtres de ce territoire sont investis d’un monopole sur tout le travail de la société}, et n’ont a supporter que les charges inhérentes à ce travail ; d’où il suit que si tel individu, non propriétaire foncier, participe à toutes les jouissances que la grande richesse peut procurer, il ne jouit de cet avantage que parce qu’il le reçoit d’une manière plus ou moins immédiate de quelques propriétaires fonciers qui consentent à payer avec libéralité le service de ses talents ou de ses capitaux\footnotemark.}\notedebasdepagesansmarque[\AB]{Cette note du traducteur d’\AS est extrêmement remarquable, parce qu’elle caractérise de la manière la plus naïve les doctrines de l’École économique anglaise, ou plutôt les conséquences qu’on a prétendu en tirer. On n’admet plus aujourd’hui que la part des profits du travailleur demeure toujours nécessairement bornée à la quantité de substances indispensable pour l’alimenter, c’est-à-dire pour l’empêcher de mourir ; personne n’ose plus soutenir que \emphase{tous les profits} reviennent \emphase{exclusivement} aux propriétaires fonciers, en vertu du singulier droit de l’appliquer \emphase{aux fantaisies qui naissent de l’opulence}, et du monopole dont ils sont investis \emphase{sur tout le travail} de la société. Ce sont là des préjugés qui appartiennent à l’école de \Quesnay et dont le livre même d’\AS a fait justice. M. le sénateur \Garnier nous semble avoir eu tort d’imputer à ce grand économiste une telle hérésie. Celle prétention étrange n’a pris naissance que dans les écrits de \Malthus, de \Ricardo, de M. \MacCulloch et de M. \Senior, les vrais représentants de l’école Impitoyable et fataliste, dont l’\publication{Essai sur le principe de population} est la plus habile et la plus énergique expression. La marche naturelle des choses réfute chaque jour celle inique doctrine. Loin que les grands propriétaires fonciers soient les dispensateurs naturels et exclusifs de la richesse, \emphase{en qualité de propriétaires}, ils vivent en réalité du travail de leurs fermiers, lorsqu’ils ne sont pas cultivateurs eux-mêmes, et nous marchons d’un pas assez rapide vers le moment on ils seront tous forcés de cultiver ou de vendre, s’ils veulent avoir un revenu.}. Chaque ouvrier se trouve avoir une grande quantité de son travail dont il peut disposer, outre ce qu’il en applique à ses propres besoins ; et comme les autres ouvriers sont aussi dans le même cas, il est à même d’échanger une grande quantité des marchandises fabriquées par lui contre une grande quantité des leurs, ou, ce qui est la même chose, contre le prix de ces marchandises. Il peut fournir abondamment ces autres ouvriers de ce dont ils ont besoin, et il trouve également à s’accommoder auprès d’eux, en sorte qu’il se répand, parmi les différentes classes de la société, une abondance universelle.

Observez, dans un pays civilisé et florissant, ce qu’est le mobilier d’un simple journalier ou du dernier des manœuvres, et vous verrez que le nombre des gens dont l’industrie a concouru pour une part quelconque à lui fournir ce mobilier, est au-delà de tout calcul possible. La veste de laine, par exemple, qui couvre ce journalier, toute grossière qu’elle paraît, est le produit du travail réuni d’une innombrable multitude d’ou­vriers. Le berger, celui qui a trié la laine, celui qui l’a peignée ou cardée, le teinturier, le fileur, le tisserand, le foulonnier\notedebasdepage[\Garnier]{Ce mot doit s’entendre le plus souvent, dans le cours de cet ouvrage, de l’ouvrier qui tisse le drap, métier qui occupe beaucoup de monde dans certains cantons de l’Angleterre.}, celui qui adoucit, chardonne et unit le drap, tous ont mis une portion de leur industrie à l’achèvement de cette œuvre grossière. Combien, d’ailleurs, n’y a-t-il pas eu de marchands et de voituriers employés à transporter la matière à ces divers ouvriers, qui souvent demeurent dans des endroits distants les uns des autres ! Que de commerce et de navigation mis en mouvement ! Que de constructeurs de vaisseaux, de matelots, d’ouvriers en voiles et en cordages, mis en œuvre pour opérer le transport des différentes drogues du teinturier, rapportées souvent des extrémités du monde\notedebasdepage[\Buchanan]{L’admirable tableau tracé, dans ce chapitre, de l’utilité du commerce, joint avec la division du travail, comme il l’est, semble une réponse suffisante à quelques théories qui mettaient en question l’importance du commerce étranger. Le commerce d’un pays avec un autre ne diffère eu aucune façon de son commerce intérieur. Les échanges intérieurs établissent la division du travail; tandis que les échanges extérieurs étendent le principe bien plus loin, en permettant à une nation, par une direction perfectionnée de ses terres et de son travail, d’en accroître largement le produit annuel. Le commerce étranger et le commerce domestique d’un pays reposent précisément sur un même principe, et le même argument doit conséquemment prouver leur utilité collective.} ! Quelle variété de travail aussi pour produire les outils du moindre de ces ouvriers ! Sans parler des machines les plus compliquées, comme le vaisseau du commerçant, le moulin du foulonnier ou même le métier du tisserand, considérons seulement quelle multitude de travaux exige une des machines les plus simples, les ciseaux avec lesquels le berger a coupé la laine. Il faut que le mineur, le constructeur du fourneau où le minerai a été fondu, le bûcheron qui a coupé le bois de la charpente, le charbonnier qui a cuit le charbon consommé à la fonte, le briquetier, le maçon, les ouvriers qui ont construit le fourneau, la cons­truc­tion du moulin de la forge, le forgeron, le coutelier, aient tous contribué, par la réunion de leur industrie, à la production de cet outil. Si nous voulions examiner de même chacune des autres parties de l’habillement de ce même journalier, ou chacun des meubles de son ménage, la grosse chemise de toile qu’il porte sur la peau, les souliers qui chaussent ses pieds, le lit sur lequel il repose et toutes les différentes parties dont ce meuble est composé ; le gril sur lequel il fait cuire ses aliments, le charbon dont il se sert, arraché des entrailles de la terre et apporté peut-être par de longs trajets sur terre et sur mer, tous ses autres ustensiles de cuisine, ses meubles de table, ses couteaux et ses fourchettes, les assiettes de terre ou d’étain sur lesquelles il sert et coupe ses aliments, les différentes mains qui ont été employées à préparer son pain et sa bière, le châssis de verre qui lui procure à la fois de la chaleur et de la lumière, en l’abritant du vent et de la pluie ; l’art et les connaissances qu’exige la préparation de cette heureuse et magnifique invention, sans laquelle nos climats du nord offriraient à peine des habitations supportables ; si nous songions aux nombreux outils qui ont été nécessaires aux ouvriers employés à produire ces diverses commodités ; si nous examinions en détail toutes ces choses, si nous considérions la variété et la quantité de travaux que suppose chacune d’elles, nous sentirions que, sans l’aide et le concours de plusieurs milliers de personnes, le plus petit particulier, dans un pays civilisé, ne pourrait être vêtu et meublé même selon ce que nous regardons assez mal à propos comme la manière la plus simple et la plus commune. Il est bien vrai que son mobilier paraîtra extrêmement simple et commun, si on le compare avec le luxe extravagant d’un grand seigneur ; cependant, entre le mobilier d’un prince d’Europe et celui d’un paysan laborieux et rangé, il n’y a peut-être pas autant de différence qu’entre les meubles de ce dernier et ceux de tel roi d’Afrique qui règne sur dix mille sauvages nus, et qui dispose en maître absolu de leur liberté et de leur vie.

%%%%%%%%%%%%%%%%%%%%%%%%%%%%%%%%%%%%%%%%%%%%%%%%%%%%%%%%%%%%%%%%%%%%%%%%%%%%%%%%
%                                  Chapitre 2                                  %
%%%%%%%%%%%%%%%%%%%%%%%%%%%%%%%%%%%%%%%%%%%%%%%%%%%%%%%%%%%%%%%%%%%%%%%%%%%%%%%%

\chapter{{\relsize{-5}Du}\\\textcolor{coquelicot}{principe\\ qui donne lieu\\\relsize{-2}{\relsize{-3} à} la division du travail}}
\markboth{Du principe qui donne lieu à la division du travail}{}

Cette \emphase{division du travail}, de laquelle découlent tant d’avantages, ne doit pas être regardée dans son origine comme l’effet d’une sagesse humaine qui ait prévu et qui ait eu pour but cette opulence générale qui en est le résultat ; elle est la conséquence nécessaire, quoique lente et graduelle, d’un certain penchant naturel à tous les hommes qui ne se proposent pas des vues d’utilité aussi étendues : c’est le penchant qui les porte à trafiquer, à faire des trocs et des échanges d’une chose pour une autre.

Il n’est pas de notre sujet d’examiner si ce penchant est un de ces premiers prin­cipes de la nature humaine dont on ne peut pas rendre compte, ou bien, comme cela paraît plus probable, s’il est une conséquence nécessaire de l’usage de la raison et de la parole. Il est commun à tous les hommes, et on ne l’aperçoit dans aucune autre es­pèce d’animaux, pour lesquels ce genre de contrat est aussi inconnu que tous les autres. Deux lévriers qui courent le même lièvre ont quelquefois l’air d’agir de concert. Chacun d’eux renvoie le gibier vers son compagnon ou bien tâche de le saisir au passage quand il le lui renvoie. Ce n’est toutefois l’effet d’aucune convention entre ces animaux, mais seulement celui du concours accidentel de leurs passions vers un même objet. On n’a jamais vu de chien faire de propos délibéré l’échange d’un os avec un autre chien. On n’a jamais vu d’animal chercher à faire entendre à un autre par sa voix ou ses gestes : \emphase{Ceci est à moi, cela est à toi ; je te donnerai l’un pour l’autre}. Quand un animal veut obtenir quelque chose d’un autre animal ou d’un homme, il n’a pas d’autre moyen que de chercher à gagner la faveur de celui dont il a besoin. Le petit caresse sa mère, et le chien qui assiste au dîner de son maître s’efforce par mille manières d’attirer son attention pour en obtenir à manger. L’homme en agit quelquefois de même avec ses semblables, et quand il n’a pas d’autre voie pour les engager à faire ce qu’il souhaite, il tâche de gagner leurs bonnes grâces par des flatteries et des attentions serviles. Il n’a cependant pas toujours le temps de mettre ce moyen en œuvre. Dans une société civilisée, il a besoin à tout moment de l’assistance et du concours d’une multitude d’hommes, tandis que toute sa vie suffirait à peine pour lui gagner l’amitié de quelques personnes. Dans presque toutes les espèces d’animaux, chaque individu, quand il est parvenu à sa pleine croissance, est tout à fait indépendant et, tant qu’il reste dans son état naturel, il peut se passer de l’aide de toute autre créature vivante. Mais l’homme a presque continuellement besoin du secours de ses semblables, et c’est en vain qu’il l’attendrait de leur seule bienveillance. Il sera bien plus sûr de réussir, s’il s’adresse à leur intérêt personnel et s’il leur persuade que leur propre avantage leur commande de faire ce qu’il souhaite d’eux. C’est ce que fait celui qui propose à un autre un marché quelconque ; le sens de sa proposition est ceci : \emphase{Donnez-moi ce dont j’ai besoin, et vous aurez de moi ce dont vous avez besoin vous-même} ; et la plus grande partie de ces bons offices qui nous sont nécessaires s’obtiennent de cette façon. Ce n’est pas de la bienveillance du boucher, du marchand de bière et du boulanger, que nous attendons notre dîner, mais bien du soin qu’ils apportent à leurs intérêts. Nous ne nous adressons pas à leur humanité, mais à leur égoïsme ; et ce n’est jamais de nos besoins que nous leur parlons, c’est toujours de leur avantage. Il n’y a qu’un mendiant qui puisse se résoudre à dépendre de la bienveillance d’autrui ; encore ce mendiant n’en dépend-il pas en tout ; c’est bien la bonne volonté des personnes charitables qui lui fournit le fonds entier de sa subsistance ; mais quoique ce soit là en dernière analyse le principe d’où il tire de quoi satisfaire aux besoins de sa vie, cependant ce n’est pas celui-là qui peut y pourvoir a mesure qu’ils se font sentir. La plus grande partie de ces besoins du moment se trouvent satisfaits, comme ceux des autres hommes, par traité, par échange et par achat. Avec l’argent que l’un lui donne, il achète du pain. Les vieux habits qu’il reçoit d’un autre, il les troque contre d’autres vieux habits qui l’accommodent mieux, ou bien contre un logement, contre des aliments, ou enfin contre de l’argent qui lui servira à se procurer un logement, des aliments ou des habits quand il en aura besoin.

Comme c’est ainsi par traité, par troc et par achat que nous obtenons des autres la plupart de ces bons offices qui nous sont mutuellement nécessaires, c’est cette même disposition à trafiquer qui a dans l’origine donné lieu à la \emphase{division du travail}. Par exemple, dans une tribu de chasseurs ou de bergers, un individu fait des arcs et des flèches avec plus de célérité et d’adresse qu’un autre. Il troquera fréquemment ces objets avec ses compagnons contre du bétail ou du gibier, et il ne tarde pas à s’apercevoir que, par ce moyen, il pourra se procurer plus de bétail et de gibier que s’il allait lui-même à la chasse. Par calcul d’intérêt donc, il fait sa principale occupation des arcs et des flèches, et le voilà devenu une espèce d’armurier. Un autre excelle à bâtir et à couvrir les petites huttes ou cabanes mobiles ; ses voisins prennent l’habitude de l’employer à cette besogne, et de lui donner en récompense du bétail ou du gibier, de sorte qu’à la fin il trouve qu’il est de son intérêt de s’adonner exclusivement à cette besogne et de se faire en quelque sorte charpentier et constructeur. Un troisième devient de la même manière forgeron ou chaudronnier ; un quatrième est le tanneur ou le corroyeur des peaux ou cuirs qui forment le principal revêtement des sauvages. Ainsi, la certitude de pouvoir troquer tout le produit de son travail qui excède sa propre consommation, contre un pareil surplus du produit du travail des autres qui peut lui être nécessaire, encourage chaque homme à s’adonner à une occupation parti­culière, et à cultiver et perfectionner tout ce qu’il peut avoir de talent et d’intelligence pour cette espèce de travail.

Dans la réalité, la différence des talents naturels entre les individus est bien moin­dre que nous ne le croyons, et les aptitudes si différentes qui semblent distinguer les hommes de diverses professions quand ils sont parvenus à la maturité de l’âge, n’est pas tant la cause de l’effet de la \emphase{division du travail}, en beaucoup de circonstances. La différence entre les hommes adonnés aux professions les plus opposées, entre un philosophe, par exemple, et un portefaix, semble provenir beaucoup moins de la nature que de l’habitude et de l’éducation. Quand ils étaient l’un et l’autre au com­men­cement de leur carrière, dans les six ou huit premières années de leur vie, il y avait peut-être entre eux une telle ressemblance que leurs parents ou camarades n’y auraient pas remarqué de différence sensible. Vers cet âge ou bientôt après, ils ont commencé à être employés à des occupations fort différentes. Dès lors a commencé entre eux cette disparité qui s’est augmentée insensiblement, au point qu’aujourd’hui la vanité du philosophe consentirait à peine à reconnaître un seul point de ressemblance. Mais, sans la disposition des hommes à trafiquer et à échanger, chacun aurait été obligé de se procurer lui-même toutes les nécessités et commodités de la vie. Chacun aurait eu la même tâche à remplir et le même ouvrage à faire, et il n’y aurait pas eu lieu à cette grande différence d’occupations, qui seule peut donner naissance à une grande diffé­rence de talents.

Comme c’est ce penchant à troquer qui donne lieu à cette diversité de talents, si remarquable entre hommes de différentes professions, c’est aussi ce même penchant qui rend cette diversité utile. Beaucoup de races d’animaux, qu’on reconnaît pour être de la même espèce, ont reçu de la nature des caractères distinctifs et des aptitudes différentes beaucoup plus sensibles que celles qu’on pourrait observer entre les hommes, antérieurement à l’effet des habitudes et de l’éducation. Par nature, un philosophe n’est pas de moitié aussi différent d’un portefaix, en aptitude et en intelligence, qu’un mâtin l’est d’un lévrier, un lévrier d’un épagneul, et celui-ci d’un chien de berger. Toutefois, ces différentes races d’animaux, quoique de même espèce, ne sont presque d’aucune utilité les uns pour les autres. Le mâtin ne peut pas ajouter aux avantages de sa force en s’aidant de la légèreté du lévrier, ou de la sagacité de l’épagneul, ou de la docilité du chien de berger. Les effets de ces différentes aptitudes ou degrés d’intelligence, faute d’une faculté ou d’un penchant au commerce et à l’échange, ne peuvent être mis en commun, et ne contribuent pas le moins du monde à l’avantage ou à la commodité commune de l’espèce. Chaque animal est toujours obligé de s’entretenir et de se défendre lui-même à part et indépendamment des autres, et il ne peut retirer la moindre utilité de cette variété d’aptitudes que la nature a reparties entre ses pareils. Parmi les hommes, au contraire, les talents les plus disparates sont utiles les uns aux autres ; les différents produits de leur industrie respective, au moyen de ce penchant universel à troquer et à commercer, se trouvent mis, pour ainsi dire, en une masse commune où chaque homme peut aller acheter, suivant ses besoins, une portion quelconque du produit de l’industrie des autres\notedebasdepage[\MacCulloch]{La dissertation du Dr. \Smith sur le principe qui donne naissance à la division du travail, quoique assez ingénieuse, ne paraît pas reposer sur une base solide. Cette division est simplement une conséquence de la faculté que nous possédons d’apercevoir, ou tout au moins de conjecturer avec plus ou moins d’exactitude, ce qui en des circonstances données, nous est le plus avantageux et le plus utile. Les sauvages qui marchent lentement ou qui sont boiteux, n’ont pas une tendance innée à devenir fabricants d’arcs et de flèches, et à échanger par plaisir ces articles avec les sauvages leurs semblables. Mais il ne leur est pas difficile d’apercevoir qu’il sera grandement dans leur intérêt d’opérer ces échanges et que telle est, en réalité, la seule voie qui puisse leur faire obtenir leur nourriture : tandis que, d’un autre côté, les sauvages doués d’agilité voient qu’il leur est profitable de suivre le genre d’industrie pour lequel ils ont une aptitude particulière, et d’échanger une partie de leur proie avec les autres pour en recevoir les instruments dont ils ont besoin. Les dispositions physiques, les talents, les aptitudes, aussi bien que les circonstances dans lesquelles les hommes sont placés, diffèrent essentiellement, et rien ne parait plus naturel que de voir chaque individu adopter de préférence les occupations qu’à tout prendre il juge les plus favorables : échangeant alors les portions de ses produits qui excèdent sa consommation contre les produits de ses voisins qu’il voudrait acquérir et que ceux-ci désireraient céder. Les habitants des Highlands du Pertshire ne s’adonnent pas à l’élève des bestiaux, ni ceux du \publication{Carse of Gowrie} à la culture du froment, ni ceux des îles Shetland à la pèche, seulement parce qu’une tendance instinctive les pousse à s’engager dans de tels travaux ; mais bien parce qu’ils ont appris de l’expérience qu’ils obtiendront une plus large part d’objets nécessaires et utiles à la vie humaine, en se bornant à ces branches d’industrie pour le développement desquelles ils ont une supériorité marquée, et en échangeant le surplus de leurs produits avec les autres.}.

%%%%%%%%%%%%%%%%%%%%%%%%%%%%%%%%%%%%%%%%%%%%%%%%%%%%%%%%%%%%%%%%%%%%%%%%%%%%%%%%
%                                  Chapitre 3                                  %
%%%%%%%%%%%%%%%%%%%%%%%%%%%%%%%%%%%%%%%%%%%%%%%%%%%%%%%%%%%%%%%%%%%%%%%%%%%%%%%%

\chapter{{\relsize{-5}Que}\\\vskip 1.5mm \textcolor{coquelicot}{{\relsize{-2}la division du travail}\\ est limitée par l’étendue du marché}}
\markboth{Que la division du travail est limitée par l’étendue du marché}{}

Puisque c’est la faculté d’échanger qui donne lieu à la \emphase{division du travail}, l’accroissement de cette division doit, par conséquent, toujours être limité par l’étendue de la faculté d’échanger, ou, en d’autres termes, par l’étendue du \emphase{marché}. Si le \emphase{marché} est très-petit, personne ne sera encouragé à s’adonner entièrement à une seule occupation, faute de pouvoir trouver à échanger tout le surplus du produit de son travail qui excédera sa propre consommation, contre un pareil surplus du produit du travail d’autrui qu’il voudrait se procurer\notedebasdepage[\AB]{C’est ce qui explique pourquoi l’industrie est si simple et si patriarcale dans nos villages, où l’on voit souvent le même homme exercer plusieurs métiers, et vendre une infinité de produits qui nécessiteraient, dans une ville, un grand nombre de marchands différents.}.

Il y a certains genres d’industrie, même de l’espèce la plus basse, qui ne peuvent s’établir ailleurs que dans une grande ville. Un portefaix, par exemple, ne pourrait pas trouver ailleurs d’emploi ni de subsistance. Un village est une sphère trop étroite pour lui ; même une ville ordinaire est à peine assez vaste pour lui fournir constamment de l’occupation. Dans ces maisons isolées et ces petits hameaux qui se trouvent épars dans un pays très-peu habité, comme les montagnes d’Écosse, il faut que chaque fermier soit le boucher, le boulanger et le brasseur de son ménage. Dans ces contrées, il ne faut pas s’attendre à trouver deux forgerons, deux charpentiers, ou deux maçons qui ne soient pas au moins à vingt milles l’un de l’autre. Les familles éparses qui se trouvent à huit ou dix milles du plus proche de ces ouvriers sont obligées d’apprendre à faire elles-mêmes une quantité de menus ouvrages pour lesquels on aurait recours à l’ouvrier dans des pays plus peuplés. Les ouvriers de la campagne sont presque partout dans la nécessité de s’adonner à toutes les différentes branches d’industrie qui ont quelque rapport entre elles par l’emploi des mêmes matériaux. Un charpentier de village confectionne tous les ouvrages en bois, et un serrurier de village tous les ouvrages en fer. Le premier n’est pas seulement charpentier, il est encore menuisier, ébé­niste ; il est sculpteur en bois, en même temps qu’il fait des charrues et des voitures. Les métiers du second sont encore bien plus variés. Il n’y a pas de place pour un cloutier dans ces endroits reculés de l’intérieur des montagnes d’Écosse. À raison d’un millier de clous par jour, et en comptant trois cents jours de travail par année, cet ouvrier pourrait en fournir par an trois cents milliers. Or, dans une pareille localité, il lui serait impossible de trouver le débit d’un seul millier, c’est-à-dire du travail d’une seule journée, dans le cours d’un an.

Comme la facilité des transports par eau ouvre un marché plus étendu à chaque espèce d’industrie que ne peut le faire le seul transport par terre, c’est aussi sur les côtes de la mer et le long des rivières navigables que l’industrie de tout genre com­mence à se subdiviser et à faire des progrès ; et ce n’est ordinairement que longtemps après que ces progrès s’étendent jusqu’aux parties intérieures du pays. Un chariot à larges roues, conduit par deux hommes et attelé de huit chevaux, mettra environ six semaines de temps à porter et rapporter de Londres à Édimbourg près de quatre tonneaux pesant de marchandises. Dans le même temps à peu près, un navire de six à huit hommes d’équipage, faisant voile du port de Londres à celui de Leith, porte et rapporte ordinairement le poids de deux cents tonneaux. Ainsi, à l’aide de la naviga­tion, six ou huit hommes pourront conduire et ramener dans le même temps, entre Londres et Édimbourg, la même quantité de marchandises que cinquante chariots à larges roues conduits par cent hommes et traînés par quatre cents chevaux\notedebasdepage[\MacCulloch]{Les frais de transport des marchandises par terre ont été de beaucoup réduits depuis la publication de \publication{La Richesse des nations} ; cependant ils sont toujours bien supérieurs aux frais de transport par mer.}. Par conséquent, deux cents tonneaux de marchandises transportées par terre de Londres à Édimbourg, au meilleur compte possible, auront à supporter la charge de l’entretien de cent hommes pendant trois semaines et, de plus, non-seulement de l’entretien, mais encore, ce qui est à peu près aussi cher, l’entretien et la diminution de valeur de quatre cents chevaux et de cinquante grands chariots ; tandis que la même quantité de marchandises, transportées par eau, ne se trouvera seulement chargée que de l’entretien de six à huit hommes et de la diminution de capital d’un bâtiment du port de deux cents tonneaux, en y ajoutant simplement la valeur du risque un peu plus grand, ou bien la différence de l’assurance entre le transport par eau et celui par terre. S’il n’y avait donc entre ces deux places d’autre communication que celle de terre, on ne pourrait transporter de l’une à l’autre que des objets d’un prix considérable relativement à leur poids, et elles ne comporteraient ainsi qu’une très-petite partie du commerce qui subsiste présentement entre elles ; par conséquent, elles ne se donneraient qu’une très-faible partie de l’encouragement qu’elles fournissent réciproquement à leur industrie. À cette condition, il n’y aurait que peu ou point de commerce entre les parties éloignées du monde. Quelle sorte de marchandise pourrait supporter les frais d’un voyage par terre, de Londres à Calcutta ? Ou, en supposant qu’il y en eût d’assez précieuses pour valoir une telle dépense, quelle sûreté y aurait-il à la voiturer à travers les territoires de tant de peuples barbares ? Cependant, ces deux villes entretiennent aujourd’hui entre elles un commerce très-considérable ; et par le marché qu’elles s’ouvrent l’une à l’autre, elles donnent un très-grand encouragement à leur industrie respective.

Puisque le transport par eau offre de si grands avantages, il est donc naturel que les premiers progrès de l’art et de l’industrie se soient montrés partout où cette facilité ouvre le monde entier pour marché, au produit de chaque espèce de travail, et ces progrès ne s’étendent que beaucoup plus tard dans les parties intérieures du pays\notedebasdepage[\AB]{Ces vérités sont aujourd’hui parfaitement comprises. Sur tous les points du globe, les gouvernements s’efforcent d’améliorer la navigation, soit en perfectionnant le régime des fleuves, soit en creusant des canaux et des ports. La France, l’Angleterre et les États-Unis ont dépensé plusieurs milliards dans ce but depuis un demi-siècle. La seule étude de la canalisation de l’Europe serait un travail immense. Qu’y a-t-il de plus admirable dans le monde que le canal trie aux États-Unis, celui de Gothie en Suède, et le canal Calédonien en Écosse ?}. L’intérieur des terres peut n’avoir pendant longtemps d’autre \emphase{marché} pour la grande partie de ses marchandises, que le pays qui l’environne et qui le sépare des côtes de la mer ou des rivières navigables. Ainsi, l’étendue de son \emphase{marché} doit, pendant long­temps, être en proportion de ce pays et, par conséquent, il ne peut faire de progrès que postérieurement à ceux du pays environnant. Dans nos colonies de l’Amérique septen­trionale, les plantations ont suivi constamment les côtes de la mer ou les bords des rivières navigables, et elles se sont rarement étendues à une distance considérable des unes ou des autres.

D’après les témoignages les plus authentiques de l’histoire, il paraît que les nations qui ont été les premières civilisées sont celles qui ont habité autour des côtes de la Méditerranée. Cette mer, sans comparaison la plus grande de toutes les mers intérieures du globe, n’ayant point de marées et, par conséquent, point d’autres vagues que celles causées par les vents, était extrêmement favorable à l’enfance de la navigation, tant par la tranquillité de ses eaux que par la multitude de ses îles et par la proximité des rivages qui la bordent, alors que les hommes, ignorant l’usage de la boussole, craignaient de perdre de vue les côtes et que, dans l’état d’imperfection où était l’art de la construction des vaisseaux, ils n’osaient s’abandonner aux flots impétueux de l’Océan. Traverser les colonnes d’Hercule, c’est-à-dire naviguer au-delà du détroit de Gibraltar, fut longtemps regardé, dans l’Antiquité, comme l’entreprise la plus périlleuse et la plus surprenante. Les Phéniciens et les Carthaginois, les plus habiles navigateurs et les plus savants constructeurs de vaisseaux dans ces anciens temps, ne tentèrent même ce passage que fort tard, et ils furent longtemps les seuls peuples qui l’osèrent.

L’Égypte semble avoir été le premier de tous les pays, sur les côtes de la Méditerranée, dans lequel l’agriculture ou les métiers aient été cultivés et avancés à un degré un peu considérable. La haute Égypte ne s’étend qu’à quelques milles de distan­ce du Nil, et dans la basse Égypte, ce grand fleuve se partage en plusieurs différents canaux qui, à l’aide de très-peu d’art, ont fourni des moyens de communication et de transport, non-seulement entre toutes les grandes villes, mais encore entre les villages considérables, et même entre plusieurs établissements agricoles, à peu près de la même manière que font aujourd’hui en Hollande le Rhin et la Meuse. L’étendue et la facilité de cette navigation intérieure furent probablement une des causes principales qui ont amené l’Égypte de si bonne heure à l’état d’opulence\notedebasdepage[\AB]{Voyez à ce sujet le magnifique tableau de l’Égypte, tracé par Napoléon dans le Ier volume des \publication{Mémoires dictés à Sainte-Hélène}, pages 42-70, Ire édition. Les événements ont justifié la haute opinion que l’empereur avait conçue des destinées de ce pays, dont la fortune tient aujourd’hui l’Europe entière en suspens.}.

Il paraît aussi que les progrès de l’agriculture et des métiers datent de la plus haute Antiquité dans le Bengale et dans quelques-unes des provinces orientales de la Chine, quoique nous ne puissions cependant avoir sur cette partie du monde aucun témoignage bien authentique pour juger de l’étendue de cette antiquité. Au Bengale, le Gange et quelques autres grands fleuves se partagent en plusieurs canaux, comme le Nil en Égypte. Dans les provinces orientales de la Chine, il y a aussi plusieurs grands fleuves qui forment par leurs différentes branches une multitude de canaux et qui, communiquant les uns avec les autres, favorisent une navigation intérieure bien plus étendue que celle du Nil ou du Gange, ou peut-être que toutes deux à la fois. Il est à remarquer que ni les anciens Égyptiens, ni les Indiens, ni les Chinois, n’ont encouragé le commerce étranger, mais que tous semblent avoir tiré leur grande opulence de leur navigation intérieure.

Toute l’Afrique intérieure, et toute cette partie de l’Asie qui est située à une assez grande distance au nord du Pont-Euxin et de la mer Caspienne, l’ancienne Scythie, la Tartarie et la Sibérie moderne, semblent, dans tous les temps, avoir été dans cet état de barbarie et de pauvreté dans lequel nous les voyons à présent. La mer de Tartarie est la mer Glaciale, qui n’est pas navigable ; et quoique ce pays soit arrosé par quelques-uns des plus grands fleuves du monde, cependant ils sont à une trop grande distance l’un de l’autre pour que la majeure partie du pays puisse en profiter pour les communications et le commerce. Il n’y a en Afrique aucun de ces grands golfes, comme les mers Baltique et Adriatique en Europe, les mers Noire et Méditerranée en Asie et en Europe, et les golfes Arabique, Persique, ceux de l’Inde, du Bengale et de Siam, en Asie, pour porter le commerce maritime dans les parties intérieures de ce vaste continent ; et les grands fleuves de l’Afrique se trouvent trop éloignés les uns des autres, pour donner lieu à aucune navigation intérieure un peu importante. D’ailleurs, le commerce qu’une nation peut établir par le moyen d’un fleuve qui ne se partage pas en un grand nombre de branches ou de canaux et qui, avant de se jeter dans la mer, traverse un territoire étranger, ne peut jamais être un commerce considérable, parce que le peuple qui possède ce territoire étranger est toujours maître d’arrêter la commu­ni­cation entre cette autre nation et la mer. La navigation du Danube est d’une très-faible utilité aux différents États qu’il traverse, tels que la Bavière, l’Autriche et la Hongrie, en comparaison de ce qu’elle pourrait être si quelqu’un de ces États possédait la totalité du cours de ce fleuve jusqu’à son embouchure dans la mer Noire\notedebasdepage[\AB]{Cet état contre nature devra cesser un jour. Déjà le Rhin est rendu à sa destination par des traités entre les nations riveraines, et quoique ces traités laissent encore à désirer, le moment n’est pas loin où ils produiront leur plein et entier effet. La navigation du Danube s’améliore chaque jour, et les trois grandes contrées dont parle \AS participent déjà aux bienfaits de cette navigation. Il en sera bientôt ainsi de tous les grands fleuves de la terre.}.

%%%%%%%%%%%%%%%%%%%%%%%%%%%%%%%%%%%%%%%%%%%%%%%%%%%%%%%%%%%%%%%%%%%%%%%%%%%%%%%%
%                                  Chapitre 4                                  %
%%%%%%%%%%%%%%%%%%%%%%%%%%%%%%%%%%%%%%%%%%%%%%%%%%%%%%%%%%%%%%%%%%%%%%%%%%%%%%%%

\chapter{{\relsize{-5}\addfontfeatures{Letters=SmallCaps}De}\\\vskip -1.5mm \textcolor{coquelicot}{{\relsize{-2}l’origine {\relsize{-3}et de} l’usage}\\{\relsize{-5}de la} \\ \vskip -1mm monnaie}}
\markboth{De l’origine et de l’usage de la monnaie}{}

La \emphase{division du travail} une fois généralement établie, chaque homme ne produit plus par son travail que de quoi satisfaire une très-petite partie de ses besoins. La plus grande partie ne peut être satisfaite que par l’échange du surplus de ce produit qui excède sa consommation, contre un pareil surplus du travail des autres. Ainsi, chaque homme subsiste d’échanges et devient une espèce de marchand, et la société elle-même est proprement une société commerçante.

Mais dans les commencements de l’établissement de la \emphase{division du travail}, cette faculté d’échanger dut éprouver de fréquents embarras dans ses opérations. Un homme, je suppose, a plus d’une certaine denrée qu’il ne lui en faut, tandis qu’un autre en manque. En conséquence, le premier serait bien aise d’échanger une partie de ce superflu, et le dernier ne demanderait pas mieux que de l’acheter. Mais si par malheur celui-ci ne possède rien dont l’autre ait besoin, il ne pourra pas se faire d’échange entre eux. Le boucher a dans sa boutique plus de viande qu’il n’en peut consommer, le brasseur et le boulanger en achèteraient volontiers une partie, mais ils n’ont pas autre chose à offrir en échange que les différentes denrées de leur négoce, et le boucher est déjà pourvu de tout le pain et de toute la bière dont il a besoin pour le moment. Dans ce cas-là, il ne peut y avoir lieu entre eux à un échange. Il ne peut être leur vendeur, et ils ne peuvent être ses chalands ; et tous sont dans l’impossibilité de se rendre mutuellement service. Pour éviter les inconvénients de cette situation, tout homme prévoyant, dans chacune des périodes de la société qui suivirent le premier établissement de la division du travail, dut naturellement tâcher de s’arranger pour avoir par devers lui, dans tous les temps, outre le produit particulier de sa propre industrie, une certaine quantité de quelque marchandise qui fût, selon lui, de nature à convenir à tant de monde, que peu de gens fussent disposés à la refuser en échange du produit de leur industrie.

Il est vraisemblable qu’on songea, pour cette nécessité, à différentes denrées qui furent successivement employées. Dans les âges barbares, on dit que le bétail fut l’instrument ordinaire du commerce ; et quoique ce dût être un des moins commodes, cependant, dans les anciens temps, nous trouvons souvent les choses évaluées par le nombre de bestiaux donnés en échange pour les obtenir. L’armure de Diomède, dit Homère, ne coûtait que neuf bœufs\notedebasdepage[\MacCulloch]{Le marquis Gantier a dépensé beaucoup d’érudition pour contester ce passage de \Smith ; il prétend que les bœufs d’Homère étaient de véritables monnaies métalliques : cela est possible ; mais il n’en est pas moins vraisemblable, qu’à des époques plus reculées encore, on s’est servi de bétail comme de monnaie d’échange.} ; mais celle de Glaucus en valait cent. On dit qu’en Abyssinie le sel est l’instrument ordinaire du commerce et des échanges ; dans quelques contrées de la côte de l’Inde, c’est une espèce de coquillage ; à Terre-Neuve, c’est de la morue sèche ; en Virginie, du tabac ; dans quelques-unes de nos colonies des Indes occidentales, on emploie le sucre à cet usage, et dans quelques autres pays, des peaux ou du cuir préparé ; enfin, il y a encore aujourd’hui un village en Écosse, où il n’est pas rare, à ce qu’on m’a dit, de voir un ouvrier porter au cabaret ou chez le boulanger des clous au lieu de monnaie.

Cependant, des raisons irrésistibles semblent, dans tous les pays, avoir déterminé les hommes à adopter les métaux pour cet usage, par préférence à toute autre denrée. Les métaux non*seulement ont l’avantage de pouvoir se garder avec aussi peu de déchet que quelque autre denrée que ce soit, aucune n’étant moins périssable qu’eux, mais encore ils peuvent se diviser sans perte en autant de parties qu’on veut, et ces parties, à l’aide de la fusion, peuvent être de nouveau réunies en masse ; qualité que ne possède aucune autre denrée aussi durable qu’eux, et qui, plus que toute autre qualité, en fait les instruments les plus propres au commerce et à la circulation. Un homme, par exemple, qui voulait acheter du sel et qui n’avait que du bétail à donner en échange, était obligé d’en acheter pour toute la valeur d’un bœuf ou d’un mouton à la fois. Il était rare qu’il pût en acheter moins, parce que ce qu’il avait à donner en échange pouvait très-rarement se diviser sans perte ; et s’il avait eu envie d’en acheter davantage, il était, par les mêmes raisons, forcé d’en acheter une quantité double ou triple, c’est-à-dire pour la valeur de deux ou trois bœufs ou bien de deux ou trois moutons. Si, au contraire, au lieu de bœufs ou de moutons, il avait eu des métaux à donner en échange, il lui aurait été facile de proportionner la quantité du métal à la quantité précise de denrées dont il avait besoin pour le moment.

Différentes nations ont adopté pour cet usage différents métaux. Le fer fut l’instrument ordinaire du commerce chez les Spartiates, le cuivre chez les premiers Romains, l’or et l’argent chez les peuples riches et commerçants\footnote{Ce serait se livrer à la plus infructueuse de toutes les recherches, que de prétendre remonter à l’époque où la monnaie a été pour la première fois mise en usage parmi les hommes. Autant vaudrait remonter à l’origine même de la civilisation. La monnaie n’est ni une invention ni une découverte ; on ne la doit ni au hasard ni au génie ; elle est née naturellement des besoins de la société. \Smith a parfaitement établi que le penchant à échanger, particulier à l’espèce humaine, est le principe qui donne lieu à la division du travail, et que cette division ne peut se développer et s’étendre qu’à mesure qu’il y a pour chacun plus de moyens d’échanger le surplus des produits de son travail. On peut donc affirmer que partout où l’on voit qu’une grande division de travail a eu lieu, il y a eu nécessairement beaucoup d’activité dans les échanges, et que par conséquent il a existé un instrument destiné à faciliter et à accélérer les échanges, c’est-à-dire une monnaie. Tout peuple a dû choisir pour instrument d’échange la substance la plus propre à remplir cette fonction, de même qu’il a choisi, pour se nourrir et pour se vêtir, ce qu’il a pu trouver de plus convenable et de plus commode.

Les peuples qui n’ont point eu connaissance de l’usage des métaux, se sont fait une monnaie de celle de leurs marchandises qui réunissait aux plus haut degré les deux qualités nécessaires à ce service : 1° celle de pouvoir se diviser de manière à s’approprier aux plus petits échanges ; 2° celle de pouvoir se garder pendant un certain temps sans frais et sans déchet. Il n’est donc pas surprenant qu’en Abyssinie, comme \Smith le rapporte, le sel ait été adopté comme instrument ordinaire des échanges ; que, dans quelques contrées des côtes de l’Inde, ont ait choisi pour cet usage le cauris, petit coquillage d’un blanc et d’un poli remarquables, et qui a une valeur comme objet de parure ; que la morue à Terre-Neuve, le tabac en Virginie, le sucre dans quelques colonies des Antilles, les peaux et cuirs préparés, dans d’autres lieux, enfin des clous même dans un village d’Écosse, puissent assez bien remplir la fonction de monnaie ; tous ces objets ayant une valeur réelle, étant divisibles, et pouvant se garder sans frais et sans déchet jusqu’au moment où le possesseur trouve occasion d’acheter. Mais parmi tous les articles divers qui sont la matière des échanges entre des hommes réunis en société, il n’en est assurément aucun qui soit moins propre à rendre le service de monnaie ou instrument intermédiaire du commerce, que ne l’est le bétail, puisqu’il ne peut être gardé}.

Il paraît que, dans l’origine, ces métaux furent employés à cet usage, en barres informes, sans marque ni empreinte. Aussi Pline\footnote{Pline, \publication{Histoire naturelle}, livre XXXIII, chap. iii.} nous rapporte, d’après l’autorité de Timée, ancien historien, que les Romains, jusqu’au temps de Servius Tullius, n’avaient pas de monnaie frappée, mais qu’ils faisaient usage de barres de cuivre sans em­preinte, pour acheter tout ce dont ils avaient besoin. Ces barres faisaient donc alors fonction de monnaie.

L’usage des métaux dans cet état informe entraînait avec soi deux grands incon­vénients : d’abord, l’embarras de les peser, et ensuite celui de les essayer. Dans les métaux précieux, où une petite différence dans la quantité fait une grande différence dans la valeur, le pesage exact exige des poids et des balances fabriqués avec grand soin. C’est, en particulier, une opération assez délicate que de peser de l’or. À la vérité, pour les métaux grossiers, où une petite erreur serait de peu d’importance, il n’est pas besoin d’une aussi grande attention. Cependant, nous trouverions excessivement incommode qu’un pauvre homme fût obligé de peser un liard chaque fois qu’il a besoin d’acheter ou de vendre pour un liard de marchandise. Mais l’opération de l’essai est encore bien plus longue et bien plus difficile ; et à moins de fondre une portion du métal au creuset avec des dissolvants convenables, on ne peut tirer de l’essai que des conclusions fort incertaines. Pourtant, avant l’institution des pièces monnayées, à moins d’en passer par cette longue et difficile opération, on se trouvait à tout moment exposé aux fraudes et aux plus grandes friponneries, et on pouvait recevoir en échange de ses marchandises, au lieu d’une livre pesant d’argent fin ou de cuivre pur, une composition falsifiée avec les matières les plus grossières et les plus viles, portant à l’extérieur l’apparence de ces métaux. C’est pour prévenir de tels abus, pour faciliter les échanges et encourager tous les genres de commerce et d’industrie, que les pays qui ont fait quelques progrès considérables vers l’opulence ont trouvé nécessaire de marquer d’une empreinte publique certaines quantités des métaux particuliers dont ils avaient coutume de se servir pour l’achat des denrées. De là l’origine de la monnaie frappée et des établissements publics destinés à la fabrication des monnaies ; institution qui est précisément de la même nature que les offices des auneurs et marqueurs publics des draps et des toiles. Tous ces offices ont également pour objet d’attester, par le moyen de l’empreinte publique, la qualité uniforme ainsi que la quantité de ces diverses marchandises quand elles sont mises au marché\notedebasdepage[\AB]{Le lecteur désireux d’approfondir cette importante matière, pourra consulter avec fruit l’excellent ouvrage de M. \Jacob : \publication{An historical Inquiry into the production and consumption of the precious metals}, 2 vol. in-8°, 1831 ; \publication{les Considérations} de \Law sur le numéraire, et le livre V du \publication{Cours d’Économie politique} de H. \Storch, édition de \JBS, avec ses notes. Ou peut lire aussi le chap. xxiv de notre \publication{Histoire de l’Économie politique}, où se trouvent résumés les divers systèmes monétaires qui ont régné en Europe depuis les anciens jusqu’à la découverte des mines du nouveau monde.}.

Il paraît que les premières empreintes publiques qui furent frappées sur les métaux courants n’eurent, la plupart du temps, d’autre objet que de rectifier ce qui était à la fois le plus difficile à connaître et ce dont il était le plus important de s’assurer, savoir la bonté ou le degré de pureté du métal. Elles devaient ressembler à cette marque sterling qu’on imprime aujourd’hui sur la vaisselle et les lingots d’argent, ou à cette empreinte espagnole qui se trouve quelquefois sur les lingots d’or ; ces empreintes, n’étant frappées que sur un côté de la pièce et n’en couvrant pas toute la surface, certifient bien le degré de fin, mais non le poids du métal. Abraham pèse à Éphron les quatre cents sicles d’argent qu’il était convenu de lui payer pour le champ de Macpelah. Quoiqu’ils passassent pour la monnaie courante du marchand, ils étaient reçus néanmoins au poids et non par compte, comme le sont aujourd’hui les lingots d’or et d’argent. On dit que les revenus de nos anciens rois saxons étaient payés, non en monnaie, mais en nature, c’est-à-dire en vivres et provisions de toute espèce. Guillaume le Conquérant introduisit la coutume de les payer en monnaie ; mais pendant longtemps cette monnaie fut reçue, à l’Échiquier, au poids et non par compte.

La difficulté et l’embarras de peser ces métaux avec exactitude donna lieu à l’institution du coin, dont l’empreinte, couvrant entièrement les deux côtés de la pièce et quelquefois aussi la tranche, est censée certifier, non-seulement le titre, mais encore le poids du métal. Alors ces pièces furent reçues par compte, comme aujourd’hui, sans qu’on prît la peine de les peser.

Originairement, les dénominations de ces pièces exprimaient, à ce qu’il me semble, leur poids ou la quantité du métal qu’elles contenaient. Au temps de Servius Tullius, qui le premier fit battre monnaie à Rome, l’\emphase{as} romain ou la \emphase{livre} contenait le poids d’une livre romaine de bon cuivre. Elle était divisée, comme notre \emphase{livre de Troy}\footnote{Voyez à la fin de l’ouvrage, la table pour l’\emphase{évaluation des monnaies, poids et mesures anglaises}.}, en douze onces, dont chacune contenait une once véritable de bon cuivre. La \emphase{livre} sterling d’Angleterre, au temps d’Édouard Ier, contenait une livre (poids de la Tour) d’argent d’un titre connu. La livre de la Tour paraît avoir été quelque chose de plus que la livre romaine, et quelque chose de moins que la livre de Troy. Ce ne fut qu’à la dix-huitième année du règne de Henri VIII que cette dernière fut introduite à la monnaie d’Angleterre. La \emphase{livre} de France, au temps de Charlemagne, contenait une livre, poids de Troyes, d’argent d’un titre déterminé. La foire de Troyes en Champagne était alors fréquentée par toutes les nations de l’Europe, et les poids et mesures d’un marché si célèbre étaient connus et évalués par tout le monde. La monnaie d’Écosse, appelée \emphase{livre} depuis le temps d’Alexandre 1er jusqu’à celui de \RobertBruce, contenait une livre d’argent du même poids et du même titre que la livre sterling d’Angleterre. Le \anglais{penny} ou \emphase{denier} d’Angleterre, celui de France et celui d’Écosse, contenaient tous de même, dans l’origine, un denier réel pesant d’argent, c’est-à-dire la vingtième partie d’une once, et la deux cent quarantième partie d’une livre. Le \anglais{schelling} ou \emphase{sou} semble aussi d’abord avoir été la dénomination d’un poids. « Quand le froment est à 12 schellings le \anglais{quarter}\footnote{Voyez à la fin de l’ouvrage, le rapport des mesures anglaises à celles de France.}, dit un ancien statut de Henri III, alors le pain d’un \anglais{farthing}\footnote{Quart du penny ou du denier sterling.} doit peser 11 schellings 4 pence. » Toutefois, il paraît que le schelling ne garda pas, soit avec le penny d’un côté, soit avec la livre de l’autre, une proportion aussi constante et aussi uniforme que celle que conservèrent entre eux le penny et la livre. Sous la première race des rois de France, le \anglais{schelling} ou \emphase{sou} français paraît en différentes occa­sions avoir contenu cinq, douze, vingt et quarante deniers. Chez les anciens Saxons, on voit le schelling, dans un temps, ne contenir que cinq pence ou deniers, et il n’est pas hors de vraisemblance qu’il aura été aussi variable chez eux que chez leurs voisins les anciens Francs. Chez les Français, depuis Charlemagne, et chez les Anglais, depuis Guillaume le Conquérant, la proportion entre la livre, le schelling et le denier ou penny, paraît avoir été uniformément la même qu’à présent, quoique la valeur de chacun ait beaucoup varie ; car je crois que, dans tous les pays du monde, la cupidité et l’injustice des princes et des gouvernements, abusant de la confiance des sujets, ont diminué par degrés la quantité réelle de métal qui avait été d’abord contenue dans les monnaies. L’as romain, dans les derniers temps de la république, était réduit à un vingt-quatrième de sa valeur primitive, et au lieu de peser une livre, il vint à ne plus peser qu’une demi-once\notedebasdepage{M. le sénateur \Garnier a contesté l’exactitude de ce fait dans une longue note historique, qui n’est qu’un plaidoyer en faveur de la probité des Romains. Nous ne croyons pas devoir la citer, parce qu’elle n’aboutit à aucune conclusion vraiment scientifique.}{\AB}. La livre et le penny anglais\footnote{Dans tout le cours de cet ouvrage, toutes les valeurs exprimées en livres, sous et deniers, doivent toujours s’entendre de livres, sous et deniers sterling. On a employé indifféremment le mot \emphase{sou} ou \anglais{schelling}*, et le mot \emphase{denier} ou \anglais{penny}. \anglais{Pence} est le pluriel de \anglais{penny}.} ne contiennent plus aujourd’hui qu’un tiers environ de leur valeur originaire ; la livre et le penny d’Écosse, qu’un trente-sixième environ, et la livre et le penny ou denier français, qu’à peu près un soixante-sixième. Au moyen de ces opérations, les princes et les gouvernements qui y ont eu recours se sont, en apparence, mis en état de payer leurs dettes et de remplir leurs engagements avec une quantité d’argent moindre que celle qu’il en aurait fallu sans cela ; mais ce n’a été qu’en apparence, car leurs créanciers ont été, dans la réalité, frustrés d’une partie de ce qui leur était dû. Le même privilège se trouva accordé à tous les autres débiteurs dans l’État, et ceux-ci se trouvèrent en état de payer, avec la même somme nominale de cette monnaie nouvelle et dégradée, tout ce qui leur avait été prêté en ancienne monnaie. De telles opérations ont donc toujours été favorables aux débiteurs et ruineuses pour les créanciers, et elles ont quelquefois produit dans les fortunes des particuliers des révolutions plus funestes et plus générales que n’aurait pu faire une très-grande calamité publique\footnote{\MacCulloch renvoie le lecteur à l’article \publication{Monnaie} (Money) de l’\publication{Encyclopédie britannique} ; cet article contient un tableau des altérations pratiquées sur les coins d’Angleterre, de France et des autres pays.}.

C’est de cette manière que la monnaie est devenue chez tous les peuples civilisés l’instrument universel du commerce, et que les marchandises de toute espèce se vendent et s’achètent, ou bien s’échangent l’une contre l’autre, par son intervention.

Il s’agit maintenant d’examiner quelles sont les règles que les hommes observent naturellement, en échangeant les marchandises l’une contre l’autre, ou contre de l’argent. Ces règles déterminent ce qu’on peut appeler la Valeur \emphase{relative} ou \emphase{échangeable} des marchandises.

Il faut observer que le mot \emphase{valeur} a deux significations différentes ; quelquefois il signifie l’utilité d’un objet particulier, et quelquefois il signifie la faculté que donne la possession de cet objet d’en acheter d’autres marchandises\footnote{Il n’y a pas de sujet qui ait plus exercé les économistes et qui ait donné lieu à plus de dissertations que la définition de \emphase{la valeur}. La plupart des écrivains se se sont égarés dans un dédale d’arguties métaphysiques sur le sens économique de ce mot. Nous ne citerons pas la longue nomenclature de ces monographies désormais inutiles. Il est reconnu aujourd’hui par tous les auteurs qui font autorité dans la science, que la distinction établie par \AS a l’inappréciable avantage d’établir nettement le caractère particulier de la \emphase{valeur en échange}, la seule sur laquelle s’exercent les transactions, parce qu’elle est le produit du travail humain.}. On peut appeler l’une, \emphase{valeur en usage}, et l’autre, \emphase{valeur en échange}. Des choses qui ont la plus grande \emphase{valeur en usage} n’ont souvent que peu ou point de \emphase{valeur en échange} ; et au contraire, celles qui ont la plus grande \emphase{valeur en échange} n’ont souvent que peu ou point de \emphase{valeur en usage}. Il n’y a rien de plus utile que l’eau, mais elle ne peut presque rien acheter ; à peine y a-t-il moyen de rien avoir en échange. Un diamant, au contraire, n’a presque aucune valeur\notedebasdepage[\MacCulloch]{Le mot \emphase{valeur} a souvent été employé pour désigner, non seulement le \emphase{prix d’échange} d’un article, ou sa faculté d’être échangé pour d’autres objets que le travail seul peut donner, mais encore son utilité, ou la propriété qu’il possède de satisfaire à nos besoins ou de contribuer à notre bien-être et à nos jouissances. Mais il est évident que l’\emphase{utilité} des marchandises est une qualité tout à fait différente de leur faculté d’échange : témoin le pouvoir qu’a le blé d’apaiser notre faim, et l’eau d’étancher notre soif. Le Dr. \Smith a découvert cette différence, et il a démontré l’importance qui s’attachait à distinguer l’\emphase{utilité} des marchandises ou, comme il le disait, leur \emphase{valeur d’usage} ou \emphase{naturelle}, de leur \emphase{valeur d’échange}. Confondre des qualités si essentiellement distinctes, serait évidemment entrer dans la voie des plus absurdes conclusions. Aussi, pour éviter de méconnaître le sens d’un mol aussi important que celui de valeur, il serait mieux de ne l’appliquer qu’à la \emphase{valeur d’échange}, et de réserver le mot \emphase{utilité} pour exprimer le pouvoir ou la faculté que possède un article de satisfaire à nos besoins ou de répondre à nos désirs.} quant à l’usage, mais on trouvera fréquemment à l’échanger contre une très-grande quantité d’autres marchandises.

Pour éclaircir les principes qui déterminent la \emphase{valeur échangeable} des marchandises, je tâcherai d’établir :

Premièrement, quelle est la véritable mesure de cette \emphase{valeur échangeable}, ou en quoi consiste le \emphase{prix réel} des marchandises.

Secondement, quelles sont les différentes parties intégrantes qui composent ce \emphase{prix réel}.

Troisièmement enfin, quelles sont les différentes circonstances qui tantôt élèvent quelqu’une ou la totalité de ces différentes parties du prix au-dessus de leur taux naturel ou ordinaire, et tantôt les abaissent au-dessous de ce taux, ou bien quelles sont les causes qui empêchent que le \emphase{prix de marché}\notedebasdepage[\JBS]{L’usage veut que l’on appelle \emphase{valeur échangeable} la quantité de toute marchandise que l’on donne en échange de celle qu’on veut avoir, et \emphase{prix} la quantité de monnaie que l’on donne pour le même objet.}, c’est-à-dire le prix actuel des marchandises, ne coïncide exactement avec ce qu’on peut appeler leur prix naturel.

Je tâcherai de traiter ces trois points avec toute l’étendue et la clarté possibles dans les trois chapitres suivants, pour lesquels je demande bien instamment la patience et l’attention du lecteur : sa patience pour me suivre dans des détails qui, en quelques endroits, lui paraîtront peut-être ennuyeux ; et son attention, pour comprendre ce qui semblera peut-être encore quelque peu obscur, malgré tous les efforts que je ferai pour être intelligible. Je courrai volontiers le risque d’être trop long, pour chercher à me rendre clair ; et après que j’aurai pris toute la peine dont je suis capable pour répan­dre de la clarté sur un sujet qui, par sa nature, est aussi abstrait, je ne serai pas encore sûr qu’il n’y reste quelque obscurité.

%%%%%%%%%%%%%%%%%%%%%%%%%%%%%%%%%%%%%%%%%%%%%%%%%%%%%%%%%%%%%%%%%%%%%%%%%%%%%%%%
%                                  Chapitre 5                                  %
%%%%%%%%%%%%%%%%%%%%%%%%%%%%%%%%%%%%%%%%%%%%%%%%%%%%%%%%%%%%%%%%%%%%%%%%%%%%%%%%

\chapter{{\relsize{-5}\addfontfeatures{Letters=SmallCaps}Du }\\\vskip -1.5mm\textcolor{coquelicot}{{\relsize{-1}prix} réel}{\relsize{-5} et du}\\ \textcolor{coquelicot}{{\relsize{-1}prix} nominal}\\{\relsize{-5} des }\textcolor{coquelicot}{marchandises}\\\vskip -1mm{\relsize{-5}ou de leur }\textcolor{coquelicot}{{\relsize{-1}prix}{ \relsize{-2}en }travail}\\\vskip -1mm{\relsize{-5}et de leur }\textcolor{coquelicot}{{\relsize{-1}prix}{\relsize{-2} en }argent}}
\markboth{Du prix réel et du prix nominal des marchandises\\ou de leur prix en travail et de leur prix en argent}{}

Un homme est riche ou pauvre, suivant les moyens qu’il a de se procurer les choses nécessaires, commodes ou agréables de la vie. Mais la division une fois établie dans toutes les branches du travail, il n’y a qu’une partie extrêmement petite de toutes ces choses qu’un homme puisse obtenir directement par son travail ; c’est du travail d’autrui qu’il lui faut attendre la plus grande partie de toutes ces jouissances ; ainsi, il sera riche ou pauvre, selon la quantité de travail qu’il pourra commander ou qu’il sera en état d’acheter.

Ainsi, la valeur d’une denrée quelconque pour celui qui la possède et qui n’entend pas en user ou la consommer lui-même, mais qui a intention de l’échanger pour autre chose, est égale à la quantité de travail que cette denrée le met en état d’acheter ou de commander. Le travail est donc la mesure réelle de la \emphase{valeur échangeable} de toute marchandise\notedebasdepage[\MacCulloch]{Le travail, ou le produit du travail consacré à la création ou à l’acquisition des marchandises, n’est pas cependant, quoique la plus importante de beaucoup, la seule source de la \emphase{valeur}, et sitôt qu’une marchandise ou un produit peut constituer la propriété ou la jouissance d’une ou de plusieurs personnes exclusivement, il est susceptible de \emphase{valeur échangeable} ; c’est-à-dire qu’alors même qu’il n’aurait rien coûté à ses possesseurs, il peut se trouver d’autres personnes disposées à donner en échange une partie de leur travail ou des produits de leur travail. Ceci résulte de ce phénomène, que la marchandise n’existe qu’en un degré d’abondance limité, et conséquemment, sous l’influence de faits qui permettent de la monopoliser. Un individu qui possède une chute d’eau dans sa propriété, peut quelquefois en retirer un revenu considérable, et un individu qui a rencontré un antique joyau, pourra probablement le vendre à un très-haut prix. Le nombre des chutes d’eau et des pierres antiques est limité et ne peut être accru ; on ne peut appliquer à leur production ni travail ni capital, et c’est pourquoi, quoique primitivement la chute d’eau n’ait rien couté, elle peut acquérir une valeur locative égale à celle du travail qu’elle épargnera à ceux qui l’emploieront à transmettre le mouvement aux machines ou à produire tout autre effet désiré : mais la valeur du joyau ne sera pas réglée de même ; car, la \emphase{demande} qu’il excitera dépendant entièrement des caprices du goût et de la mode, ainsi que de la fortune de ceux qui désirent de telles curiosités, cette valeur oscillera entre des extrêmes très-éloignés.}.

Le prix réel de chaque chose, ce que chaque chose coûte réellement à celui qui veut se la procurer, c’est le travail et la peine qu’il doit s’imposer pour l’obtenir. Ce que chaque chose vaut réellement pour celui qui l’a acquise et qui cherche à en disposer ou à l’échanger pour quelque autre objet, c’est la peine et l’embarras que la possession de cette chose peut lui épargner et qu’elle lui permet d’imposer à d’autres personnes. Ce qu’on achète avec de l’argent ou des marchandises est acheté par du travail, aussi bien que ce que nous acquérons à la sueur de notre front. Cet argent et ces marchandises nous épargnent, dans le fait, cette fatigue. Elles contiennent la valeur d’une certaine quantité de travail, que nous échangeons pour ce qui est supposé alors contenir la valeur d’une quantité égale de travail. Le travail a été le premier prix, la monnaie payée pour l’achat primitif de toutes choses. Ce n’est point avec de l’or ou de l’argent, c’est avec du travail que toutes les richesses du monde ont été achetées originairement ; et leur valeur pour ceux qui les possèdent et qui cherchent à les échanger contre de nouvelles productions, est précisément égale à la quantité de travail qu’elles les mettent en état d’acheter ou de commander.

« \emphase{Richesse, c’est pouvoir} », a dit \Hobbes ; mais celui qui acquiert une grande fortune ou qui l’a reçue par héritage, n’acquiert par là nécessairement aucun pouvoir politique, soit civil, soit militaire. Peut-être sa fortune pourra-t-elle lui fournir les moyens d’acquérir l’un ou l’autre de ces pouvoirs, mais la simple possession de cette fortune ne les lui transmet pas nécessairement. Le genre de pouvoir que cette possession lui transmet immédiatement et directement, c’est le pouvoir d’acheter ; c’est un droit de commandement sur tout le travail d’autrui, ou sur tout le produit de ce travail existant alors au marché. Sa fortune est plus ou moins grande exactement en proportion de l’étendue de ce pouvoir, en proportion de la quantité du travail d’autrui qu’elle le met en état de commander, ou, ce qui est la même chose, du produit du travail d’autrui qu’elle le met en état d’acheter. La \emphase{valeur échangeable} d’une chose quelconque doit nécessairement toujours être précisément égale à la quantité de cette sorte de pouvoir qu’elle transmet à celui qui la possède\notedebasdepage{La \emphase{valeur d’échange}, ou cette faculté d’être échangé contre le travail ou ses produits, est une qualité inhérente à tout article qui a provoqué la \emphase{demande}, pourvu toutefois que ce produit ait nécessite un \emphase{travail}, direct ou indirect, nu qu’il n’existe qu’en quantités limitées. Mais cette qualité ne peut ni se manifester ni être appréciée si ce n’est quand les marchandises sont comparées entre elles ou avec le travail. Il est, en effet, totalement impossible d’établir la valeur d’une marchandise sans se reporter à une autre ou sans remonter au travail comme fait initial. Aucun article, aucun produit ne possède de \emphase{valeur échangeable}, si ce n’est par rapport à quelque autre objet qui est ou qui peut être échangé avec lui. Il serait aussi rationnel de parler d’une hauteur ou d’une profondeur absolue que d’une \emphase{valeur absolue}.}.

Mais, quoique le travail soit la mesure réelle de la \emphase{valeur échangeable} de toutes les marchandises, ce n’est pourtant pas celle qui sert communément à apprécier cette valeur. Il est souvent difficile de fixer la proportion entre deux différentes quantités de travail. Cette proportion ne se détermine pas toujours seulement par le temps qu’on a mis à deux différentes sortes d’ouvrages. Il faut aussi tenir compte des différents degrés de fatigue qu’on a endurés et de l’habileté qu’il a fallu déployer. Il peut y avoir plus de travail dans une heure d’ouvrage pénible que dans deux heures de besogne aisée, ou dans une heure d’application à un métier qui a coûté dix années de travail à apprendre, que dans un mois d’application d’un genre ordinaire et à laquelle tout le monde est propre. Or, il n’est pas aisé de trouver une mesure exacte applicable au travail ou au talent. Dans le fait, on tient pourtant compte de l’une et de l’autre quand on échange ensemble les productions de deux différents genres de travail. Toutefois, ce compte-là n’est réglé sur aucune balance exacte ; c’est en marchandant et en débattant \emphase{les prix de marché} qu’il s’établit, d’après cette grosse équité qui, sans être fort exacte, l’est bien assez pour le train des affaires communes de la vie.

D’ailleurs, chaque marchandise est plus fréquemment échangée et, par consé­quent, comparée, avec d’autres marchandises qu’avec du travail. Il est donc plus naturel d’estimer sa \emphase{valeur échangeable} par la quantité de quelque autre denrée que par celle du travail qu’elle peut acheter. Aussi, la majeure partie du peuple entend bien mieux ce qu’on veut dire par telle quantité d’une certaine denrée, que par telle quantité de travail. La première est un objet simple et palpable ; l’autre est une notion abstraite, qu’on peut bien rendre assez intelligible, mais qui n’est d’ailleurs ni aussi commune ni aussi évidente.

Mais quand les échanges ne se font plus immédiatement, et que l’argent est devenu l’instrument général du commerce, chaque marchandise particulière est plus souvent échangée contre de l’argent que contre toute autre marchandise. Le boucher ne porte guère son bœuf ou son mouton au boulanger ou au marchand de bière pour l’échanger contre du pain ou de la bière ; mais il le porte au marché, où il l’échange contre de l’argent, et ensuite il échange cet argent contre du pain et de la bière. La quantité d’argent que sa viande lui rapporte détermine aussi la quantité de pain et de bière qu’il pourra ensuite acheter avec cet argent. Il est donc plus clair et plus simple pour lui d’estimer la valeur de sa viande par la quantité d’argent, qui est la mar­chandise contre laquelle il l’échange immédiatement, que par la quantité de pain et de bière, qui sont des marchandises contre lesquelles il ne peut l’échanger que par l’intermédiaire d’une autre marchandise ; il est plus naturel pour lui de dire que sa viande vaut trois ou quatre pence la livre, que de dire qu’elle vaut trois ou quatre livres de pain, ou trois ou quatre pots de petite bière. De là vient qu’on estime plus souvent la valeur échangeable de chaque marchandise par la quantité d’argent, que par la quantité de travail ou de toute autre marchandise qu’on pourrait avoir en échange.

Cependant l’or et l’argent, comme toute autre marchandise, varient dans leur valeur ; ils sont tantôt plus chers et tantôt à meilleur marché ; ils sont quelquefois plus faciles à acheter, quelquefois plus difficiles. La quantité de travail que peut acheter ou commander une certaine quantité de ces métaux, ou bien la quantité d’autres marchan­dises qu’elle peut obtenir en échange, dépend toujours de la fécondité ou de la stérilité des mines exploitées dans le temps où se font ces échanges. Dans le seizième siècle, la découverte des mines fécondes de l’Amérique réduisit la valeur de l’or et de l’argent, en Europe, à un tiers environ de ce qu’elle avait été auparavant. Ces métaux, coûtant alors moins de travail pour être apportés de la mine au marché, ne purent plus acheter ou commander, quand ils y furent venus, qu’une moindre quantité de travail, et cette révolution dans leur valeur, quoique peut-être la plus forte, n’est pourtant pas la seule dont l’histoire nous ait laissé des témoignages. Or, de même qu’une mesure de quantité, telle que le pied naturel, la coudée ou la poignée, qui varie elle-même de gran­deur dans chaque individu, ne saurait jamais être une mesure exacte de la quantité des autres choses, de même une marchandise qui varie elle-même à tout moment dans sa propre valeur, ne saurait être non plus une mesure exacte de la valeur des autres marchandises.

Des quantités égales de travail doivent être, dans tous les temps et dans tous les lieux, d’une valeur égale pour le travailleur. Dans son état habituel de santé, de force et d’activité, et d’après le degré ordinaire d’habileté ou de dextérité qu’il peut avoir, il faut toujours qu’il sacrifie la même portion de son repos, de sa liberté, de son bonheur. Quelle que soit la quantité de denrées qu’il reçoive en récompense de son travail, le prix qu’il paye est toujours le même. Ce prix, à la vérité, peut acheter tantôt une plus grande, tantôt une moindre quantité de ces denrées ; mais c’est la valeur de celles-ci qui varie, et non celle du travail qui les achète. En tous temps et en tous lieux, ce qui est difficile à obtenir ou ce qui coûte beaucoup de travail à acquérir est \emphase{cher}, et ce qu’on peut se procurer aisément ou avec peu de travail est à bon \emphase{marché}. Ainsi, le travail, ne variant jamais dans sa valeur propre, est la seule mesure réelle et définitive qui puisse servir, dans tous les temps et dans tous les lieux, à apprécier et à comparer la valeur de toutes les marchandises. Il est leur prix \emphase{réel} ; l’argent n’est que leur prix \emphase{nominal}.

Mais, quoique les quantités égales de travail soient toujours d’une valeur égale pour celui qui travaille, cependant, pour celui qui emploie l’ouvrier, elles paraissent tantôt d’une plus grande, tantôt d’une moindre valeur. Le dernier achète ces quantités de travail, tantôt avec une plus grande, tantôt avec une plus petite quantité de mar­chan­dises ; et pour lui le prix du travail paraît varier comme celui de toute autre chose. Il lui semble \emphase{cher} dans un cas, et \emphase{à bon marché} dans l’autre. Dans la réalité pourtant, ce sont les marchandises qui sont à bon marché dans un cas, et chères dans l’autre.

Ainsi, dans cette acception vulgaire, on peut dire du travail, comme des autres marchandises, qu’il a un prix \emphase{réel} et un prix \emphase{nominal}. On peut dire que son prix \emphase{réel} consiste dans la quantité de choses nécessaires et commodes qu’on donne pour le payer, et son prix \emphase{nominal} dans la quantité d’argent. L’ouvrier est riche ou pauvre, il est bien ou mal récompensé, en proportion du prix \emphase{réel}, et non du prix \emphase{nominal}, de son travail.

La distinction entre le prix \emphase{réel} et le prix \emphase{nominal} des marchandises et du travail n’est pas une affaire de pure spéculation, mais elle peut être quelquefois d’un usage important dans la pratique. Le même prix réel est toujours de même valeur ; mais au moyen des variations dans la valeur de l’or et de l’argent, le même prix \emphase{nominal} exprime souvent des valeurs fort différentes. Ainsi, quand une propriété foncière est aliénée sous la réserve d’une rente perpétuelle, si l’on veut que cette rente conserve toujours la même valeur, il est important, pour la famille au profit de laquelle la rente est réservée, que cette rente ne soit pas stipulée en une somme d’argent fixe. Sa valeur, dans ce cas, serait sujette à éprouver deux espèces de variations : premièrement, celles qui proviennent des différentes quantités d’or et d’argent qui sont contenues, en différents temps, dans les monnaies de même dénomination ; secondement, celles qui proviennent des différences dans la valeur des quantités égales d’or et d’argent à différentes époques\notedebasdepage[\Ricardo]{L’argent est une marchandise qu’on tire des pays étrangers ; c’est l’agent d’échange parmi toutes les nations civilisées, et il se distribue en tout pays dans des proportions qui changent par l’effet de tout perfectionnement dans le commerce et dans les machines, et par la difficulté croissante d’avoir des subsistances et des objets de première nécessité pour la consommation d’une population qui va eu augmentant ; voilà la source des variations continuelles auxquelles il est sujet. En posant donc les principes qui règlent la valeur et les prix échangeables, on devrait soigneusement distinguer ce qui, dans ces variations, appartient à la marchandise même, d’avec ce qui provient d’un changement dans l’agent de la circulation qui sert à estimer la valeur, ou à exprimer le prix.}.

Les princes et les gouvernements se sont souvent imaginé qu’il était de leur intérêt du moment de diminuer la quantité de métal pur contenu dans leurs monnaies ; mais on ne voit guère qu’ils se soient jamais imaginé avoir quelque intérêt à l’augmenter. En conséquence, je crois que, chez toutes les nations, la quantité de métal pur conte­nue dans les monnaies a été à peu près continuellement en diminuant, et presque jamais en augmentant. Ainsi, les variations de cette espèce tendent presque toujours à diminuer la valeur d’une rente en argent.

La découverte des mines de l’Amérique a diminué la valeur de l’or et de l’argent en Europe. On suppose communément, je crois, sans preuve bien certaine, que cette diminution continue toujours graduellement et qu’elle doit durer encore pendant longtemps. D’après cette supposition donc, les variations de ce genre sont plus propres à diminuer qu’à augmenter la valeur d’une rente en argent, quand même on la stipulerait payable, non en une quantité de pièces de monnaie de telle dénomination, comme en tant de livres sterling, par exemple, mais en une certaine quantité d’onces d’argent pur ou à un titre déterminé.

Les rentes qu’on s’est réservées en blé ont conservé leur valeur beaucoup mieux que celles stipulées payables en argent, même dans le cas où la dénomination de la monnaie n’a pas souffert d’altération. Par le statut de la dix-huitième année d’Élisabeth\notedebasdepage{L’usage adopté en Angleterre pour citer les lois, est de les désigner par l’année du règne, tous les actes passés pendant le cours d’une session du parlement étant comptés pour un statut, et l’acte particulier étant indiqué par le numéro du chapitre.}, il a été réglé qu’un tiers des rentes de tous les baux des collèges serait réservé en blé, payable soit en nature, soit au prix courant du marché public le plus voisin. Suivant le docteur \Blackstone, l’argent qui provient de la portion payable en blé, quoique dans l’origine il n’ait été qu’un tiers du total de la rente, est ordinairement à peu près le double de ce que rapportent les deux autres tiers. À ce compte, il faut donc que les anciennes rentes des collèges, stipulées en argent, soient descendues environ au quart de leur ancienne valeur, ou ne vaillent guère plus d’un quart du blé qu’elles valaient originairement. Or, depuis le règne de Philippe et Marie, la dénomination de la monnaie anglaise n’a subi que peu ou point d’altération, et le même nombre de livres, schellings et pence a toujours contenu à peu près la même quantité d’argent fin. Cette diminution dans la valeur des rentes des collèges, stipulées en argent, provient donc en totalité de la diminution dans la valeur de l’argent.

Quand l’abaissement de la valeur de l’argent coïncide avec la diminution de la quantité contenue dans des monnaies de même dénomination, la perte est alors beau­coup plus grande. En Écosse, où la monnaie a subi bien plus de changements qu’en Angleterre, et en France où elle en a subi beaucoup plus qu’en Écosse, il y a d’anciennes rentes qui ont été dans l’origine d’une valeur considérable, et qui se sont trouvées réduites presque à rien.

Dans des temps très-éloignés l’un de l’autre, on trouvera que des quantités égales de travail se rapportent de bien plus près dans leur valeur à des quantités égales de blé, qui est la subsistance de l’ouvrier, qu’elles ne le font à des quantités égales d’or et d’argent, ou peut-être de toute autre marchandise. Ainsi, des quantités égales de blé, à des époques très-distantes l’une de l’autre, approcheront beaucoup plus entre elles de la même valeur réelle, ou bien elles mettront beaucoup plus celui qui les possédera en état d’acheter ou de commander une même quantité de travail, que ne le feraient des quantités égales de presque toute autre marchandise que ce puisse être. je dis qu’elles le feront beaucoup plus que des quantités égales de toute autre marchandise ; car mê­me des quantités égales de blé ne le feront pas exactement. La subsistance de l’ouvrier, ou le prix réel du travail, diffère beaucoup en diverses circonstances, com­me je tâcherai de le faire voir par la suite. Il est plus libéralement payé dans une société qui marche vers l’opulence, que dans une société qui reste stationnaire ; il est plus libéralement payé dans une société stationnaire, que dans une société rétrograde. Une denrée quelconque, en quelque temps que ce soit, achètera une plus grande ou une moindre quantité de travail, en proportion de la quantité de subsistances qu’elle pourra acheter à cette époque. Par conséquent, une rente réservée en blé ne sera sujette qu’aux variations dans la quantité de travail que telle quantité de blé peut acheter ; mais une rente stipulée en toute autre denrée sera sujette non-seulement aux variations dans la quantité de travail que telle quantité de blé peut acheter, mais encore aux variations qui surviendront dans la quantité de blé que telle quantité de cette denrée stipulée pourra acheter\notedebasdepage{Le docteur \Smith est tombé à ce sujet, dans quelques sérieuses erreurs. En recherchant la valeur des métaux précieux à différentes époques, il prend le \emphase{blé} comme base de ses évaluations pour l’or et pour l’argent. Il considère le blé, par la fixité de sa valeur propre, comme un garant sûr de la valeur des autres marchandises , et c’est pourquoi il regarde chaque augmentation dans son prix comme étant seulement nominale et prouvant la dépréciation de la valeur des métaux précieux, le blé étant, dans ce cas, livré pour une plus grande quantité de ces métaux, ou en d’autres termes, à un plus haut prix. Mais quand l’or s’est substitué à l’argent dans les grandes opérations, ce n’est pas contre l’argent, mais contre l’or, que les marchandises sont échangées, et c’est pourquoi il est clair que leur prix, ou la quantité d’or donnée pour eux, ne peut pas indiquer la valeur de l’\emphase{argent}. Le prix de vente du blé indique sa valeur en or (le métal qui en solde le prix), et non en argent ; et nulle différence ne peut s’élever, dans la valeur de l’argent, de ce que le prix du blé est haut ou bas, ou en d’autres termes, de ce qu’on donne moins d’or en échange. Ce prix de vente montre seulement la valeur de l’objet contre lequel on échange le blé. Mais quand toutes les grandes opérations du marché sont réglées en or, c’est contre de l’or que l’échange s’opère, et non contre de l’argent. Donc quand le docteur \Smith conclut de la valeur du blé celle de l’argent, il ne remarque pas que, par le déclin de l’argent dans la circulation monétaire de ce pays, sa valeur depuis longtemps a perdu toute influence sur le prix du blé comme sur celui de toute autre marchandise. La \emphase{monnaie} étant la base de tous les contrats de commerce, il est de haute importance qu’elle demeure invariable dans sa valeur. Cette valeur de la monnaie est toutefois sujette à varier : 1° par les fluctuations dans la valeur des métaux dont elle est faite ; 2° par le \emphase{frai} survenu dans la circulation des pièces usées et légères ayant une valeur inférieure à celles qui conservent leur poids normal. On sent encore bien plus ces inconvénients, quand on emploie deux métaux pour les hauts payements ; car les \emphase{types} des deux métaux respectifs sont sujets à varier non-seulement dans leurs rapports avec les marchandises, mais aussi dans leurs rapports entre eux. En effet, une hausse dans la valeur de chaque métal détruit les proportions établies par la circulation, et fait de la réduction en lingots une opération profitable. Quand, par exemple, les cotes de la \emphase{monnaie} indiquent que les pièces d’argent subiront une perte de valeur intrinsèque dans l’échange contre les pièces d’or, il y a un avantage palpable à convertir l’argent en lingots, et à le vendre au prix qu’il pourra atteindre sur le marché. C’est pourquoi si l’un des deux métaux est déprécié sous forme de monnaie, il est immédiatement refondu et ramené à sa véritable valeur sous forme de lingots. Ces inconvénients sont inséparables de toute circulation métallique, puisque, quelque précises que soient les évaluations de l’or et de l’argent dans la circulation, il est impossible d’éviter les variations qui s’établissent dans les rapports de ces métaux entre eux. Dans ce cas, il est avantageux de convertir en lingots la monnaie qui circule sous l’influence d’une \emphase{moins value}.}.

Il est bon d’observer que, quoique la valeur réelle d’une rente en blé varie beaucoup moins que celle d’une rente en argent, d’un siècle à un autre, elle varie pourtant beaucoup plus d’une année à l’autre. Le prix du travail en argent, comme je tâcherai de le faire voir plus loin, ne suit pas, d’une année à l’autre, toutes les fluctuations du prix du blé en argent, mais il paraît se régler partout sur le prix moyen ou ordinaire de ce premier besoin de la vie, et non pas sur son prix temporaire ou accidentel. Le prix moyen ou ordinaire du blé se règle, comme je tâcherai pareillement de le démontrer plus bas, sur la valeur de l’argent, sur la richesse ou la stérilité des mines qui fournis­sent le marché de ce métal, ou bien sur la quantité de travail qu’il faut employer et, par conséquent, de blé qu’il faut consommer pour qu’une certaine quantité d’argent soit transportée de la mine jusqu’au marché. Mais la valeur de l’argent, quoiqu’elle varie quelquefois extrêmement d’un siècle à un autre, ne varie cependant guère d’une année à l’autre, et même continue très-souvent à rester la même ou à peu près la même pendant un demi-siècle ou un siècle entier. Ainsi, le prix moyen ou ordinaire du blé en argent peut continuer aussi, pendant toute cette longue période, à rester le même ou à peu près le même, et avec lui pareillement le prix du travail, pourvu toutefois que la société, à d’autres égards, continue à rester dans la même situation ou à peu près. Pendant le même temps, le prix temporaire ou accidentel du blé pourra souvent dou­bler d’une année à l’autre : par exemple, de vingt-cinq schellings le \anglais{quarter}, s’élever à cinquante. Mais lorsque le blé est à ce dernier prix, non-seulement la valeur nominale, mais aussi la valeur réelle d’une rente en blé est au premier prix, ou bien elle pourra acheter une quantité double, soit de travail, soit de toute autre marchandise, le prix du travail en argent, et avec lui le prix de la plupart des choses, demeurant toujours le même au milieu de toutes ces fluctuations.

Il paraît donc évident que le travail est la seule mesure universelle, aussi bien que la seule exacte, des valeurs, le seul étalon qui puisse nous servir à comparer les valeurs de différentes marchandises à toutes les époques et dans tous les lieux.

On sait que nous ne pouvons pas apprécier les valeurs réelles de différentes mar­chandises, d’un siècle à un autre, d’après les quantités d’argent qu’on a données pour elles. Nous ne pouvons pas les apprécier non plus d’une année à l’autre, d’après les quantités de blé qu’elles ont coûté. Mais, d’après les quantités de travail, nous pou­vons apprécier ces valeurs avec la plus grande exactitude, soit d’un siècle à un autre, soit d’une année à l’autre. D’un siècle à l’autre, le blé est une meilleure mesure que l’argent, parce que, d’un siècle à l’autre, des quantités égales de blé seront bien plus près de commander la même quantité de travail, que ne le seraient des quantités égales d’argent. D’une année à l’autre, au contraire, l’argent est une meilleure mesure que le blé, parce que des quantités égales d’argent seront bien plus près de comman­der la même quantité de travail.

Mais, quoique la distinction entre le prix \emphase{réel} et le prix \emphase{nominal} puisse être utile dans des constitutions de rentes perpétuelles, ou même dans des baux à très-longs termes, elle ne l’est nullement pour les achats et les ventes, qui sont les contrats les plus communs et les plus ordinaires de la vie.

Au même temps et au même lieu, le prix \emphase{réel} et le prix \emphase{nominal} d’une marchan­dise quelconque sont dans une exacte proportion l’un avec l’autre. Selon qu’une denrée quelconque vous rapportera plus ou moins d’argent au marché de Londres, par exemple, elle vous mettra aussi en état d’acheter ou de commander plus ou moins de travail au même temps et au même lieu. Ainsi, quand il y a identité de temps et de lieu, l’argent est la mesure exacte de la \emphase{valeur échangeable} de toutes les marchandises ; mais il ne l’est que dans ce cas seulement.

Quoique, à des endroits éloignés l’un de l’autre, il n’y ait pas de proportion régulière entre le prix réel des marchandises et leur prix en argent, cependant, le marchand qui les transporte de l’un de ces endroits à l’autre n’a pas autre chose à considérer que leur prix en argent, ou bien la différence entre la quantité d’argent pur qu’il donne pour les acheter, et celle qu’il pourra retirer en les vendant. Il se peut qu’une demi-once d’argent à Canton, en Chine, achète une plus grande quantité, soit de travail, soit de choses utiles ou commodes, que ne le ferait une once à Londres. Toutefois, une marchandise qui se vend une demi-once d’argent à Canton peut y être réellement plus chère, être d’une importance plus réelle pour la personne qui la possède en ce lieu, qu’une marchandise qui se vend à Londres une once ne l’est pour la personne qui la possède à Londres. Néanmoins, si un commerçant de Londres peut acheter à Canton, pour une demi-once d’argent, une marchandise qu’il revendra ensuite une once à Londres, il gagne à ce marché cent pour cent, tout comme si l’once d’argent avait exactement la même valeur à Londres et à Canton. Il ne s’embarrasse pas de savoir si une demi-once d’argent à Canton aurait mis à sa disposition plus de travail et une plus grande quantité de choses propres aux besoins et aux commodités de la vie, qu’une once ne pourrait le faire à Londres. À Londres, pour une once d’argent, il aura à sa disposition une quantité de toutes ces choses double de celle qu’il pourrait y avoir pour une demi-once, et c’est là précisément ce qui lui importe.

Comme c’est le prix \emphase{nominal}, ou le prix en argent des marchandises, qui déter­mine finalement, pour tous les acheteurs et les vendeurs, s’ils font une bonne ou mauvaise affaire, et qui règle par là presque tout le train des choses ordinaires de la vie dans lesquelles il est question de prix, il n’est pas étonnant qu’on ait fait beaucoup plus d’attention à ce prix qu’au prix \emphase{réel}.

Mais, dans un ouvrage de la nature de celui-ci, il peut quelquefois être utile de comparer les différentes valeurs réelles d’une marchandise particulière, à différentes époques et en différents lieux, ou d’évaluer les différents degrés de puissance sur le travail d’autrui qu’elle a pu donner en différentes circonstances à celui qui la possé­dait. Dans ce cas, ce n’est pas tant les différentes quantités d’argent pour lesquelles elle a été communément vendue qu’il s’agit de comparer, que les différentes quantités de travail qu’auraient achetées ces différentes quantités d’argent ; mais il est bien difficile de pouvoir jamais connaître avec quelque degré d’exactitude les prix courants du travail dans des temps et des lieux éloignés. Ceux du blé, quoiqu’ils n’aient été régulièrement enregistrés que dans peu d’endroits, sont en général beaucoup plus connus, et on en trouve fréquemment des indications dans les historiens et dans les autres écrivains. Il faut donc, en général, nous contenter de ces prix, non pas comme étant toujours exactement dans les mêmes proportions que les prix courants du travail, mais comme étant l’approximation la meilleure que l’on puisse obtenir communément pour trouver à peu près ces proportions. J’aurai occasion par la suite de faire quelques comparaisons et rapprochements de ce genre\notedebasdepage[\Buchanan]{Principalement dans le chapitre xi de ce livre.}.

Les nations commerçantes, à mesure que leur industrie a fait des progrès, ont trouvé utile de frapper en monnaie plusieurs métaux différents : de l’or pour les gros payements, de l’argent pour les achats de valeur moyenne, et du cuivre ou quelque autre métal grossier pour ceux de la plus petite importance. Cependant, elles ont toujours regardé un de ces métaux comme étant plus particulièrement la mesure des valeurs qu’aucun des deux autres, et il paraît qu’en général elles ont donné cette préférence au métal qui leur avait le premier servi d’instrument de commerce. Ayant commencé une fois à prendre ce métal pour mesure, comme il fallait bien le faire quand elles n’avaient pas d’autre monnaie, elles ont généralement continué cet usage, lors même qu’il n’y avait plus de nécessité.

On dit que les Romains n’ont eu que de la monnaie de cuivre jusques environ cinq ans avant la première guerre punique\notedebasdepage{Pline, livre XXXIII, chap. iii.}, époque à laquelle ils commencèrent à frapper leurs premières monnaies en argent. Aussi, le cuivre paraît toujours avoir continué à servir de mesure de valeur dans cette république. On voit à Rome tous les comptes exprimés et tous les biens évalués en \emphase{as} ou en \emphase{sesterces}. L’\emphase{as} fut toujours la dénomination d’une monnaie de cuivre ; le mot de \emphase{sesterce} veut dire \emphase{deux as et demi}. Ainsi, quoique le sesterce fût toujours une monnaie d’argent, cependant cette monnaie était évaluée sur la monnaie de cuivre. À Rome, on disait d’un homme qui avait beaucoup de dettes qu’\emphase{il avait une grande quantité de cuivre appartenant à autrui}.

Il paraît que les peuples du Nord qui s’établirent sur les ruines de l’Empire romain ont eu de la monnaie d’argent dès le commencement de leur établissement, et que plusieurs siècles se sont écoulés avant qu’ils connussent de monnaies ni d’or ni de cuivre\notedebasdepage[\Storch]{Du moment que l’argent fut introduit, le motif d’épargner reçut une force jusque-là inconnue. Comme le numéraire représente toutes les autres richesses, et que l’argent est le numéraire du monde commerçant, on n’a qu’à amasser de l’argent pour se procurer toutes les différentes espèces de richesses qui existent dans le monde ; et comme l’argent est en même temps la plus inaltérable de toutes lés richesses et la plus facile à conserver, il présente encore le moyen le plus facile pour accumuler. Ainsi l’or et l’argent étant devenus numéraire, ont procuré ce grand avantage à la société, de fournir non-seulement lé motif le plus puissant, mais encore le moyen le plus propre à capitaliser les plus petites économies comme les plus grandes. Une autre circonstance contribua encore à renforcer le motif d’épargner ; ce fut le prêt à intérêt, rendu possible par le numéraire. Avant l’introduction de l’argent dans le commerce, les prêts devaient se réduire à fort peu de chose. Celui qui ne pouvait pas employer lui-même son capital, ou qui en avait plus qu’il ne pouvait employer, rencontrait infiniment de difficultés s’il voulait le louer ou prêter. Il ne lui suffisait pas de trouver des emprunteurs, il lui fallait des emprunteurs pour sa denrée. D’ailleurs comme la denrée se prêtait en nature, elle devait être rendue en nature : et vous sentez bien quelle source de contestations et de pertes cette circonstance devait être dans tous les cas où la denrée ne pouvait pas être restituée identiquement, ou lorsqu’elle était sujette à perdre de sa valeur par l’usage. Ces inconvénients devaient extrêmement borner les prêts et les emprunts, et par conséquent ôter l’envie d’économiser et d’accumuler à tous ceux qui n’étaient pas dans la situation d’employer eux-mêmes leurs capitaux.

Mais du moment que l’argent fut introduit comme numéraire, toutes ces difficultés cessèrent sur-le-champ ; car l’argent rend les mêmes services aux prêts qu’il rend aux échanges ; les préteurs n’ont plus besoin de chercher les emprunteurs qui veulent de telle denrée ; chaque capitaliste peut aider chaque emprunteur et chaque emprunteur trouve ce qu’il cherche chez chaque capitaliste. D’ailleurs, point de contestation Sur la valeur du prêt ; c’est la mesure même de toutes les valeurs qui se prêtent. Dès lors les économies se Sont multipliées et ont donné naissance à une foule de capitaux, qui n’eussent jamais existé sans cette facilité de prêter : et mille entreprises sont devenues possibles, auxquelles ont n’eût jamais pensé sans cette facilité d’emprunter.}. Il y avait en Angleterre de la monnaie d’argent dès le temps des Saxons, mais on n’y frappa guère de monnaie d’or avant Édouard III, ni aucune monnaie de cuivre avant l’avènement de Jacques Ier au trône de la Grande-Bretagne. C’est aussi pour cela qu’en Angleterre et, je crois, chez toutes les autres nations modernes de l’Europe, on a, en général, tenu tous les comptes et évalué tous les biens et marchandises en argent ; et quand nous voulons exprimer le montant de la fortune de quelqu’un, nous ne comptons guère par le nombre de guinées que nous la supposons valoir, mais par le nombre de \emphase{livres}. Je pense que, dans tous les pays, les offres légales de payement ne purent être faites, dans l’origine, que dans la monnaie seulement du métal adopté particulièrement pour signe ou mesure des valeurs. En Angleterre, l’or ne fut pas regardé comme monnaie légale, même longtemps après qu’on y eut frappé des monnaies d’or. Aucune loi ou proclamation publique n’y fixait la proportion entre l’or et l’argent ; on laissait au marché à la déterminer. Si un débiteur faisait des offres en or, le créancier avait le droit de les refuser tout à fait, ou bien de les accepter d’après une évaluation de l’or faite à l’amiable entre lui et son débiteur\notedebasdepage[\MacCulloch]{Selon le feu lord Liverpool, qui a étudié ce sujet avec grand soin (\publication{Treatise on Coins}, p. 128), ces monnaies d’or eurent cours à certains taux fixés de temps en temps par proclamation royale, et par conséquent elles ont été monnaie légale dès l’époque où l’or commença à être monnayé en Angleterre, en 1257, jusqu’en 1661, époque où la guinée, qui fut alors frappée pour la première fois, et les autres monnaies d’or furent déclarées monnaies de circulation, sans aucune évaluation de la valeur relative de l’or et de l’argent sur le marché. Cette ordonnance fut en vigueur jusqu’en 1717, où le taux ou la valeur d’après laquelle une guinée pouvait être échangée , fut fixé à 21 schellings. Depuis cette époque jusqu’à 1774, les monnaies d’or et d’argent furent également monnaies légales ; mais comme l’or avait été évalué trop haut par rapport à l’argent dans la proportion établie en 1717, presque tous les grands payements furent effectués en or, et les monnaies d’argent de poids intégral étaient exportées aussitôt après leur sortie de la monnaie ; celles-là seules qui étaient usées et altérées restaient dans la circulation. Enfin, en 1516, la valeur de l’argent fut élevée au-dessus de sa juste proportion, comparée à l’or, en frappant 66 schellings au lieu de 62 dans une livre poids de Troy, les quatre schellings additionnels étant retenus comme un seigneuriage ou droit de 614,31 pour cent sur le monnayage : mais afin d’empêcher que cette valeur exagérée de l’argent n’expulsât du pays la circulation des monnaies d’or, et que l’argent ne devint ainsi le seul intermédiaire des échanges, on déclara en même temps que l’argent ne serait monnaie légale que pour une somme qui ne dépasserait pas 40 schellings ; et d’un autre côté, pour empêcher que son abondance le fit baisser de valeur, le pouvoir d’émettre de la monnaie d’argent fut placé exclusivement dans les mains du gouvernement. Grâce à ces règlements, l’argent est devenu un agent de circulation tout à fait subordonné, occupant la même place par rapport à l’or, que le cuivre occupe par rapport à lui. L’expérience a démontré que ce système s’opposait parfaitement bien aux inconvénients que l’on voulait prévenir.}. Le cuivre ne peut être aujourd’hui une monnaie légale, excepté pour le change des plus petites pièces d’argent. Dans cet état de choses, la distinction entre le métal qui était réputé signe légal des valeurs et celui qui n’était pas réputé tel, était quelque chose de plus qu’une distinction nominale. Dans la suite des temps, et lorsque le peuple se fut familiarisé par degrés avec l’usage des monnaies de différents métaux et que, par conséquent, il connut mieux le rapport existant entre leur valeur respective, la plupart des nations, je pense, ont jugé convenable de fixer authentiquement le rapport de cette valeur, et de déclarer par un acte public de la loi, qu’une guinée, par exemple, de tel poids et à tel titre, s’échange­rait contre 21 schellings, ou bien serait une offre valable pour une dette de cette som­me. Dans cet état de choses, et tant que dure le rapport établi de cette manière, la dis­tinc­tion entre le métal qui est l’étalon et le métal qui ne l’est pas n’est plus guère qu’une distinction nominale. Toutefois, en cas de quelque changement dans le rapport établi par la loi, cette distinction redevient, ou au moins semble redevenir quelque chose de plus qu’une distinction nominale. Ainsi, par exemple, si la valeur légale d’une guinée était ou réduite à 20 schellings, ou élevée à 22, comme tous les comptes et presque toutes les obligations pour dettes sont réglés et exprimés en monnaie d’argent, la majeure partie des payements pourrait bien se faire avec les mêmes quantités de monnaie d’argent qu’auparavant ; mais ils exigeraient, en monnaie d’or, des quantités fort différentes, de plus grandes dans un cas, et de moindres dans un autre. L’argent paraîtrait plus invariable que l’or dans sa valeur. L’argent semblerait mesurer la valeur de l’or, et l’or ne pas mesurer celle de l’argent. La valeur de l’or aurait l’air de dépendre de la quan­tité d’argent pour laquelle il serait échangeable, tandis que la valeur de l’argent paraîtrait indépendante de la quantité d’or pour laquelle il pourrait être échangé. Cette diffé­rence ne serait pourtant autre chose qu’un effet de la coutume de régler les comptes et d’exprimer le montant des grandes et petites sommes en monnaie d’argent plutôt qu’en monnaie d’or. Un bon de 25 ou de 50 guinées, de M. \Drummond, malgré un change­ment de ce genre, serait toujours payable, comme auparavant, en 25 ou 50 guinées. Après le changement que nous supposons, ce bon serait toujours payable avec la même quantité d’or qu’auparavant, mais il demanderait en argent des quantités fort différentes. Dans le payement d’un pareil billet, ce serait l’or qui paraîtrait plus inva­riable que l’argent dans sa valeur ; ce serait l’or qui semblerait mesurer la valeur de l’argent, tandis que l’argent ne semblerait pas mesurer celle de l’or. Si la coutume de régler les comptes de cette manière et d’exprimer le montant des promesses et autres obligations pour dettes, devenait jamais générale, aussitôt l’or, et non l’argent, serait regardé comme le métal formant particulièrement le signe ou la mesure des valeurs. Dans la réalité, tant que dure le rapport légalement établi entre la valeur respective des différents métaux monnayés, la valeur du plus précieux de ces métaux règle la valeur de la totalité de la monnaie\notedebasdepage[\MacCulloch]{Ceci est une erreur. La valeur de l’argent ne dépend nullement de la valeur de l’or, et réciproquement. C’est dans les règlements de monnayage des différents peuples qu’il faut chercher la raison pour laquelle la circulation de certains pays consiste en monnaies d’or et celle d’autres contrées en monnaies d’argent. La valeur de chacun de ces métaux, comme celles de toutes les autres marchandises, étant exposée à des variations perpétuelles, il en résulte nécessairement que la proportion fixée parle gouvernement pour leur échange, doit promptement cesser d’exprimer leur rapport réel l’un avec l’autre, quelle que soit d’ailleurs l’exactitude avec laquelle cette proportion ait été d’abord établie. Aussitôt que cette différence de valeur s’est introduite , il est de l’intérêt de tous les débiteurs de payer leurs dettes avec la monnaie de métal qui a été surévalué ; alors le métal évalué au-dessus de sa valeur est exporté ; en d’autres pays par les marchands de monnaie. L’histoire des monnaies de France et d’Angleterre fournit des preuves abondantes à l’appui de cette vérité. En France, par exemple, avant la refonte de 1785, le louis d’or était taxé, à la Monnaie, à 24 livres seulement, lorsqu’il valait réellement 25 livres 10 sous. Ceux donc qui auraient acquitté leurs obligations en monnaie d’or au lieu de le faire en monnaie d’argent, auraient perdu 1 livre 10 sous par chaque 24 livres ! La conséquence naturelle fut donc que très-peu de payements étaient faits en or ; que l’or fut presque banni de la circulation, et que l’argent devint la seule espèce de circulation métallique usitée en France. (\Say, \publication{Traité d’Économie politique}, tom. Ier, p. 393, 4e édit.) En Angleterre , la surévaluation de l’or par rapport à l’argent dans la proportion fixée en 1717, produisit, comme on l’a déjà observé, un effet tout opposé. Cette évaluation fut estimée par lord Liverpool avoir été, dans le temps, environ égale à quatre pence par guinée, ou à 119,51 pour cent. Et comme la valeur réelle de l’argent par rapport à l’or, ne fit qu’augmenter pendant la plus grande partie du dernier siècle, l’avantage des payements en or devint plus décidé encore ; il conduisit enfin a l’usage universel de l’or dans les grands payements, et à l’exportation de toutes les monnaies d’argent de poids intégral. }. Douze pence de cuivre contiennent une demi-livre, \emphase{avoir du poids}, d’un cuivre qui n’est pas de meilleure qualité, et qui, avant d’être monnayé, vaut tout au plus 7 pence en argent\notedebasdepage[Note inédite de \JBS.]{Je crois que \Smith se trompe. Quand la monnaie de cuivré sert seulement à faire l’appoint de ce qu’on ne peut payer en monnaie d’argent, la valeur intrinsèque du cuivre n’en est pas changée, son empreinte seule a une valeur monétaire qui représente une coupure d’argent ; ce sont des billets de confiance écrits sur cuivre, émis par la Monnaie, remboursables en argent du moment qu’on en porte à la Monnaie assez pour avoir la plus petite pièce d’argent. Quand la Monnaie ne les rembourse pas à vue, ils perdent sur la place contre de l’argent ; donc l’argent n’en augmente pas la valeur. Quand on force à en recevoir dans les payements pour une certaine proportion, pour 1/40e par exemple, alors c’est comme si l’on mettait pour 1/40e d’alliage de cuivre dans l’argent. Le trésor public n’y gagne rien dans les marchés qu’il fait, puisque les marchés sont faits en conséquence ; quand il paye ainsi une dette contractée en argent, il fait la même banqueroute que s’il dégradait le titre. Dans ce-cas là, le change étranger baisse à proportion ; c’est-à-dire qu’on donne moins de monnaie étrangère pour de la monnaie nationale, en proportion de ce que le titre en est moins bon.}. Mais comme, par les règlements, ces 12 pence doivent s’échanger contre 1 schelling, ils sont regardés au marché comme valant 1 schelling, et on peut en tout temps les échanger pour 1 schelling. Avant même la dernière refonte de la monnaie d’or de la Grande-Bretagne, l’or, ou au moins cette portion qui en circulait dans Londres et les environs, avait en général moins perdu du poids légal que la plus grande partie de l’argent. Cependant, 21 schellings usés et effacés étaient regardés comme l’équivalent d’une guinée, qui peut-être était aussi, à la vérité, usée et effacée, mais presque jamais autant. Les derniers règlements ont porté la monnaie d’or aussi près du poids légal qu’il est possible de porter la monnaie courante d’une nation ; et l’ordre donné de ne recevoir l’or qu’au poids dans les caisses publiques, est fait pour le maintenir dans cet état aussi longtemps qu’on tiendra la main à ce règlement ; tandis que la monnaie d’argent reste toujours usée et continue à se dégrader, comme elle faisait avant la refonte de la monnaie d’or. Et malgré cela, au marché, 21 schellings de cet argent dégradé sont toujours regardés comme valant une guinée de cette excellente monnaie d’or. La refonte de la monnaie d’or a bien évidemment élevé la valeur de la monnaie d’argent qui s’échange avec elle. À la monnaie d’Angleterre, dans une livre pesant d’or, on taille 44 guinées et 112, qui, à 21 schellings la guinée, font 46 livres 14 schellings 6 deniers. Une once de cette monnaie d’or vaut donc 3 livres 17 schellings 10 deniers et 112 en argent. En Angleterre, on ne paye aucun droit ni seigneuriage sur la fabrication de la monnaie, et quiconque porte à la Monnaie une livre ou une once pesant d’or en lingot au titre de la loi, en retire une livre ou une once pesant d’or monnayé, sans aucune déduction. Ainsi, 3 livres 17 schellings 10 deniers et 1,12 l’once passent pour le prix de l’or à la monnaie d’Angleterre, ou bien c’est la quantité de métal monnayé que donne la monnaie pour de l’or en lingot au titre de la loi. Avant la refonte de la monnaie d’or, le prix de l’or en lingot au titre de la loi, pendant plusieurs années, a été, au marché, au-dessus de 3 liv. 18 schellings, quelquefois à 3 livres 19 schellings, et très-souvent à 4 livres l’once ; somme qui probable­ment, dans l’état usé et dégradé où était la monnaie d’or, contenait rarement plus d’une once d’or au titre. Depuis la refonte de la monnaie d’or, les lingots d’or au titre se vendent, au marché, rarement au-delà de 3 livres 17 schellings 7 deniers l’once\notedebasdepage{La raison pour laquelle l’once d’or en lingot vaut 3 den. et demi de moins que l’once d’or monnayé, quoique le monnayage soit franc, est expliquée ci-après.}. Avant la refonte de la monnaie d’or, le \emphase{prix de marché} a toujours été plus ou moins au-des­sus du prix de l’or à la monnaie ; depuis cette refonte, le \emphase{prix de marché} a été constamment au-dessous de celui de la monnaie. Or, ce \emphase{prix de marché} est le même, soit qu’il se paye en monnaie d’or, soit qu’il se paye en monnaie d’argent. Ainsi, la dernière refonte de la monnaie d’or a élevé non-seulement la valeur de la monnaie d’or, mais aussi celle de la monnaie d’argent, relativement à celle de l’or en lingot, et probablement aussi relativement à toutes les autres marchandises, quoique la hausse dans la valeur de l’or et de l’argent, relativement à celles-ci, ne puisse pas être aussi distincte ni aussi sensible, le prix de la plupart des marchandises étant soumis à l’influence d’une infinité d’autres causes. À la monnaie d’Angleterre, une livre pesant d’argent en lingot au titre de la loi, est frappée en 62 schellings, qui contiennent pareillement une livre pesant d’argent au titre. Ainsi\notedebasdepage[\MacCulloch]{Depuis 1816, une livre d’argent en lingot au litre a été frappée en 66 schellings, mais son prix de monnaie est resté fixé à 5 schellings deux pence l’once, les quatre schellings additionnels étant retenus comme seigneuriage ou droit de monnayage.} 5 schellings 2 deniers l’once passent pour être le prix de l’argent à la monnaie d’Angleterre, ou bien la quantité d’argent monnayé que donne la monnaie pour de l’argent en lingot au titre. Avant la refonte de la monnaie d’or, le \emphase{prix de marché} pour l’argent au titre en lingot, a été, en différentes circonstances, 5 schellings 4 deniers, 5 schellings 5 deniers, 5 schellings 6 deniers, 5 schellings 7 deniers, et très-souvent 5 schellings 8 deniers l’once. Cependant, 5 schellings 7 deniers semblent avoir été le prix le plus commun. Depuis la refonte de la monnaie d’or, le \emphase{prix de marché} de l’argent au titre, en lingot, est tombé, en différentes rencontres, à 5 schellings 3 deniers, 5 schellings 4 deniers, et 5 schellings 5 deniers l’once, et il n’a guère jamais dépassé ce dernier prix. Quoique, au marché, l’argent en lingot ait baissé considérablement depuis la refonte de la monnaie d’or, cependant il n’a jamais baissé jusques au prix qu’on en donne à la monnaie. Dans la proportion établie à la monnaie d’Angleterre entre les différents métaux, si le cuivre est à un prix fort au-dessus de sa vraie valeur, l’argent, d’un autre côté, se trouve y être évalué à un taux quelque peu au-dessous de la sienne. Dans le marché général de l’Europe, dans les monnaies de France et dans celles de Hollande, 1 once d’or fin se change contre environ 14 onces d’argent fin. Dans les monnaies anglaises, elle se change contre environ 15 onces, c’est-à-dire pour plus d’argent qu’elle ne vaut selon la commune évaluation de l’Europe. Mais de même que la hausse du cuivre, dans les monnaies anglaises, n’élève pas le prix du cuivre en barre, de même le bas prix de l’argent dans ces monnaies n’a pas fait baisser le prix de l’argent en lingot. L’argent en lingot n’en conserve pas moins sa vraie proportion avec l’or, par la même raison que le cuivre en barre conserve toujours sa vraie proportion avec l’argent\notedebasdepage{La proportion qui subsiste entre la valeur de l’or et celle de l’argent, n’est pas la même dans tout pays ; elle varie d’une année à l’autre, souvent d’une semaine à l’autre. À la fin du quinzième siècle, ou peu de temps avant la découverte de l’Amérique, cette proportion était en Europe comme 1 à 12, et même 1 à 10 ; c’est-à-dire qu’une livre d’or fin était censée valoir dix à douze livres d’argent fin. Depuis cette époque, l’or haussa dans sa valeur numérique, ou dans la quantité d’argent qu’il pouvait acheter. Les deux métaux baissèrent dans leur valeur réelle, ou dans la quantité de nourriture qu’ils pouvaient acheter ; mais l’argent baissa plus que l’or. Quoique les mines d’or d’Amérique, aussi bien que celles d’argent, surpassassent en fécondité toutes les mines connues jusqu’alors, les mines d’argent furent plus fécondes que les mines d’or. À la vérité, jusqu’à l’année 1545, l’Europe parait avoir reçu du nouveau monde beaucoup plus d’or que d’argent ; mais passé cette année, elle a été inondée de l’argent du Pérou. Cette accumulation produisit un effet d’autant plus grand, que la prospérité de l’Europe était alors plus concentrée, que les communications étaient moins fréquentes, et qu’une moindre partie des métaux de l’Amérique refluait en Asie. Depuis le milieu du seizième siècle, la proportion entre l’or et l’argent changea rapidement dans le midi de l’Europe. En Hollande elle était encore, en 1589, comme 1 à 11 3/5 ; mais sous le règne de Louis XIII, en 1641, nous la trouvons déjà en Flandre, comme 1 à 12 1/2 ; en France, comme 1 à 13 1/2 ; en Espagne, comme 1 à 14 et même au delà. En 1751 et 1752, cette proportion était à Amsterdam, alors le grand marché de l’Europe pour les matières-fines, comme 1 à 14 1/2. Aujourd’hui la proportion moyenne est comme 1 à 15, dans la plupart des pays de l’Europe.

Sur la quantité totale d’or et d’argent qu’on retire annuellement, depuis la fin du dix-huitième siècle, de toutes les mines de l’Amérique, de l’Europe et de l’Asie boréale, l’Amérique seule fournit 90/100 du produit total de l’or, et 91/100 du produit total de l’argent. L’abondance relative de ces deux métaux diffère par conséquent très-peu dans les deux continents. La quantité d’or retirée des mines d’Amérique est à celle de l’argent comme 1 à 46 ; en Europe, y compris la Sibérie, cette proportion est comme 1 à 40. Si la quantité offerte d’une denrée influait seule sur son prix, l’argent vaudrait 45 fois 2/3 moins que l’or, parce que la quantité d’argent mise actuellement au marché, est environ 45 fois 2/3 supérieure à la quantité d’or qu’on y amène. Mais l’argent est bien plus demandé que l’or ; il est employé par bien plus de gens et dans plus de cas ; voilà pourquoi sa valeur ne tombe guère au-dessous du quinzième de la valeur de l’or.

Ensuite l’argent étant moins cher que l’or, il trouve aussi bien plus d’acheteurs, et il est employé à bien des usages. Que chacun compare ce qu’il a de vaisselle et de bijoux d’argent avec ce qu’il a en or, et il trouvera que non-seulement la quantité, mais encore la valeur de ce qu’il a en argent, excède beaucoup ce qu’il a en or. Enfin la plupart des États de l’Europe emploient, dans leur monnaie, beaucoup plus d’argent que d’or, non-seulement pour la quantité, mais pour la valeur. Il n’y a que l’Angleterre et peut-être le Portugal qui font exception à cet égard.}.

Lors de la refonte de la monnaie d’argent sous le règne de Guillaume III, le prix de l’argent en lingot continua toujours à être quelque peu au-dessus du prix de cet argent à la monnaie. M. \Locke attribue ce haut prix à la permission d’exporter l’argent en lingot et à la défense d’exporter l’argent monnayé. Cette permission d’exporter, dit-il, rendit les demandes d’argent en lingot plus abondantes que les demandes d’argent monnayé. Mais la quantité de gens qui ont besoin de monnaie d’argent pour l’usage commun des ventes et des achats au-dedans est sûrement beaucoup plus considérable que celle des gens qui ont besoin de lingots d’argent, soit pour les exporter, soit pour tout autre usage. Il subsiste à présent une pareille permission d’exporter l’or en lingot, et une pareille prohibition d’exporter l’or monnayé\notedebasdepage[\MacCulloch]{L’acte qui prohibait l’exportation des monnaies d’or et d’argent a été rappelé en 1819. Toutes deux maintenant peuvent être exportées sans obstacle ou contrôle.} ; cependant le prix de l’or en lingot n’en est pas moins tombé au-dessous du prix qu’on en donne à la monnaie. Mais alors, tout comme aujourd’hui, l’argent, dans les monnaies anglaises, était évalué au-dessous de sa vraie proportion avec l’or ; et la monnaie d’or, que dans ce temps aussi on sup­posa n’avoir pas besoin d’être réformée, réglait alors, aussi bien qu’à présent, la valeur réelle de la monnaie de toute espèce. Comme la refonte de la monnaie d’argent ne fit pas baisser alors le prix de l’argent en lingot au taux auquel il est reçu à la Monnaie, il n’est pas fort probable qu’une semblable refonte puisse le faire aujourd’hui. Si la monnaie d’argent était rapprochée du poids légal autant que l’est la monnaie d’or, il est probable qu’une guinée s’échangerait, selon la proportion actuelle, contre une plus grande quantité d’argent monnayé qu’elle n’en pourrait acheter en lingot. La monnaie d’argent contenant en entier son poids légal, il y aurait, dans ce cas, profit à la fondre, afin de vendre d’abord le lingot pour de la monnaie d’or, et ensuite échanger cette monnaie d’or contre de la monnaie d’argent qu’on remettrait encore de même au creuset. Il semble que la seule méthode de prévenir cet inconvénient, serait de faire quelque changement au rapport actuellement établi entre ces deux métaux\notedebasdepage[\Buchanan]{Le rôle de la circulation métallique se divise naturellement en deux sections distinctes, la fonction d’effectuer les payemens importants étant nécessairement dévolue au métal le plus précieux, tandis que les métaux inférieurs sont toujours conservés pour les échanges ordinaires, et sont ainsi simplement subordonnés à la circulation principale.

Cependant, entre l’introduction première d’un métal plus précieux dans la circulation, et son emploi exclusif dans les hauts payements, il existe un large intervalle : et les payements du petit commerce doivent dans un temps devenir, par l’accroissement des richesses, assez considérables pour qu’il soit possible de les effectuer convenablement, au moins en partie, avec le métal nouveau et plus précieux ; car aucune monnaie ne peut être usitée pour les payements importants, qui ne soit en même temps convenable au commerce de détail. C’est, en dernier ressort, du consommateur que tout commerce, même celui du manufacturier, obtient la rentrée de son capital. L’argent du consommateur se substitue immédiatement au capital du détaillant. Celui-ci transmet ce qu’il reçoit de la vente de ses marchandises au marchand en gros, qui, de même, le reverse aux mains du manufacturier ou du fermier. Si bien que, quoique le métal plus précieux puisse être d’un usage plus favorable dans les forts payements , la circulation que développent les affaires de détail, doit effectivement opérer la fusion de toutes les transactions du commerce. Ce métal sera amassé par le détaillant en quantité suffisante pour ses payements \emphase{en gros}, et sera conséquemment réparti en larges sommes : de cette manière il entrera dans la circulation pour y remplir les fonctions d’agent principal. Quoiqu’il soit plus convenable d’effectuer les gros payements en or, un pays peut cependant conserver une monnaie relative à la petite échelle de ses transactions de détail. Autrement ses objets de consommation ne pourraient être divisés en des portions assez petites pour les besoins de la société. C’est pourquoi le métal, quel qu’il soit, argent ou cuivre, qui pourra répondre à celle exigence, et puis effectuer, quoique moins convenablement, les payements principaux formera la monnaie dominante d’un pays. Dans toutes les circulations métalliques de l’Europe, le cuivre est un agent subalterne, puisque les payements effectués avec ce métal représentent des sommes inférieures à toute monnaie d’argent. Mais quoique l’or ait été depuis longtemps introduit dans tous les systèmes monétaires du continent, il n’a pas supplanté l’argent dans les règlements principaux. La nature des payements sur le continent ne parait pas permettre l’usage de l’argent comme monnaie subsidiaire. Quoique l’or pût clairement répondre d’une manière plus satisfaisante aux grands payements, il n’en est pas moins vrai que l’argent doit être réuni en fortes sommes par les détaillants. Le débit de leurs marchandises doit toujours leur rapporter une masse considérable de monnaie d’argent qu’ils ne peuvent affecter qu’au payement des demandes du marchand en gros. Il est évident qu’en Angleterre la quantité d’argent mise en circulation n’excède pas les besoins des petits payements, puisqu’il est souvent difficile de se procurer le change d’une guinée. La monnaie d’argent a souvent même été vendue \emphase{avec prime}. On voit rarement l’argent en sommes dépassant vingt schellings, et en réalité, on effectue peu de payements de cette valeur en argent. Il serait intéressant, si nous avions les matériaux nécessaires à une enquête aussi curieuse, de retracer le développement et le déclin de tel métal particulier dans la circulation d’un pays : mais le manque de documents oppose une barrière infranchissable à une pareille investigation. L’époque à laquelle l’argent fut substitué au cuivre dans les systèmes monétaires de l’Europe est ensevelie dans l’obscurité : car malheureusement, les historiens accordent peu d’intérêt à ces questions. La guerre et la politique éveillent principalement leur attention et ces faits favoris font rejeter dans l’ombre les détails si précieux de l’histoire économique. L’argent domine encore les hauts payements en Europe, en Amérique et dans toute l’Inde ; et nul document ne peut servir de base à un calcul qui indiquerait l’époque à laquelle l’\emphase{or} prendra le principal rôle dans la circulation monétaire de ces contrées. Une telle ère peut ne pas s’ouvrir dans la révolution des siècles ; et c’est pourquoi, dans tous les cas, la situation de la circulation monétaire anglaise pendant la période à laquelle nous avons fait allusion, mérite d’être interrogée. L’introduction exclusive de l’or dans les gros payements de l’Angleterre, est une preuve évidente que les remboursements du commerce de détail étaient, à cette époque, principalement effectués avec ce métal : fait qui devait se manifester alors même qu’aucun payement particulier n’excédait ou peut-être n’égalait aucun des \emphase{types} en or. Car, en face de l’abondance de l’or et de la rareté de l’argent, il était naturel que l’on présentât les monnaies en or pour de petites sommes en demandant un appoint en argent. Par ce moyen, l’or en aidant et économisant l’emploi de l’argent dans les marchés de détail, même pour les plus petits payements, prévenait son accumulation dans les mains du détaillant. Et comme on avait primitivement trouvé nécessaire, alors que l’argent était employé dans le \emphase{détail}, de l’introduire aussi dans les hauts payements, il en découle d’une manière analogue que l’or ne peut aujourd’hui être employé exclusivement dans les payements considérables sans remplir parfois aussi les fonctions de l’argent dans le petit commerce. Si on ne se servait pas d’\emphase{or} à cet effet, il faudrait plus d’\emphase{argent}, et dès lors ce métal tendrait à s’accumuler entre les mains des détaillants en quantité suffisante pour leurs payements \emphase{en gros} et empiéterait ainsi sur les fonctions de la \emphase{haute circulation}. La substitution de l’or à l’argent dans les forts payements, doit immédiatement produire un excès dans cette dernière monnaie et conséquemment une diminution dans sa valeur ; l’effet de cette substitution est un cas correspondant exactement à celui que produirait un accroissement dans la circulation en général, provoquée sans que les demandes du commerce se soient mises en équilibre avec cette augmentation. Cette dépréciation dans la valeur du \emphase{type d’argent} développera nécessairement la tentation de le refondre. L’argent aura une plus grande valeur en \emphase{lingots} qu’en \emphase{monnaie}, et ceci se perpétuera jusqu’à ce que la quantité mise en circulation ne dépasse pas les besoins des petits payements. }. 

L’inconvénient serait peut-être moindre si l’argent était évalué, dans nos monnaies, autant au-dessus de sa juste proportion avec l’or, qu’il se trouve maintenant éva­lué au-dessous, pourvu qu’en même temps il fût statué que l’argent ne pourrait servir de monnaie légale pour plus que pour le change d’une guinée, de la même manière que le cuivre ne peut servir de monnaie légale pour plus que pour le change d’un schelling. Dans ce cas, aucun créancier ne pourrait être la dupe de la haute évaluation de l’argent monnayé, comme à présent aucun créancier ne peut l’être de la haute évaluation de la monnaie de cuivre. Il n’y aurait que les banquiers qui souffriraient de ce règlement. Quand vient à fondre sur eux une presse imprévue de demandes de payements, ils tâchent de gagner du temps en payant en pièces de 6 pence, et un tel règlement leur ôterait ce moyen d’éluder un payement exigible\notedebasdepage[\AB]{Un banquier qui aurait recours dans notre pays à ce triste expédient pour gagner du temps, serait considéré comme perdu.}. Ils seraient, par conséquent, obligés de tenir en tout temps dans leur caisse une plus grande quantité de fonds qu’à présent ; et quoique ce fût sans doute un inconvénient considérable pour eux, ce serait en même temps une sûreté considérable pour leurs créanciers. Trois livres 17 schellings 10 deniers et demi, qui sont le prix de l’or à la monnaie, ne contiennent certainement pas, même dans notre excellente monnaie d’or d’aujourd’hui, plus d’une once d’or au titre, et on pourrait penser en conséquence qu’ils ne devraient pas acheter une plus grande quantité d’or au même titre, en lingot. Mais l’or monnayé est plus commode que l’or en lingot ; et quoique le monnayage soit franc de tous droits en Angleterre, cependant l’or qu’on porte en lingot à la monnaie ne peut guère rentrer entre les mains du propriétaire, sous forme de monnaie, qu’après un délai de plusieurs semaines. Aujourd’hui même que la monnaie a tant d’occupation, il faudrait que le propriétaire attendît plusieurs mois pour le retirer. Ce délai est équivalent à un léger droit, et il donne à l’or monnayé un peu plus de valeur qu’à l’or en lingot, à quantité égale. Si l’argent était évalué dans les monnaies anglaises selon sa juste proportion avec l’or, il est probable que le prix de l’argent en lingot tomberait au-dessous du prix pour lequel il est reçu à la monnaie, même sans qu’il y eût aucune réforme dans la monnaie d’argent actuelle, tout usée et effacée qu’elle est, se réglant sur la valeur de l’excellente monnaie d’or avec laquelle on peut l’échanger. Un léger droit de seigneuriage ou un impôt sur le monnayage, tant de l’or que de l’argent, augmenterait probablement encore davantage la valeur de ces métaux mon­nayés sur la valeur de quantités égales en lingot. Dans ce cas, le monnayage élèverait la valeur du métal monnayé en proportion de l’étendue de ce léger droit, par la même raison que la façon augmente la valeur de la vaisselle à proportion du prix de cette façon. La supériorité du métal monnayé sur le lingot empêcherait qu’on ne fondît la monnaie et en découragerait l’exportation. Si quelque nécessité publique en rendait l’exportation nécessaire, la majeure partie de cette monnaie rentrerait bientôt d’elle-même. Au-dehors, on ne pourrait la vendre que pour la valeur de son poids comme lingot ; au-dedans, elle vaudrait quelque chose de plus\notedebasdepage[\JBS]{Une seule qualité nous importe dans la monnaie : c’est qu’elle ait de la valeur, et la conserve depuis le moment où nous l’acquérons pour une vente, jusqu’à celui où nous nous en séparons par un achat ; or l’expérience nous prouve que cette qualité peut résider dans des billets de confiance, et même dans une monnaie de papier, qui ont en outre l’avantage sur l’argent d’être plus promptement comptés et plus facilement transportés ; avantage qui, pour un objet qui devient successivement lu propriété de tant de personnes différentes, est digne aussi de quelque considération. Ce sont ces considérations qui engagèrent M. \DavidRicardo à publier son excellente brochure intitulée \publication{Proposals for an economical and secure currency}, dont l’objet est essentiellement de proposer un papier de confiance dont la valeur ne pourrait jamais tomber au-dessous de l’or, parce qu’il serait perpétuellement remboursable à présentation contre de l’or en lingots, et qui resterait forcément dans la circulation, par la nécessité où serait le public de s’en servir comme intermédiaire dans les échanges, à défaut de pièces métalliques qu’on ne lui livrerait pas. Une très-petite quantité d’or en lingots suffirait pour soutenir la valeur d’une fort grande quantité de billets ; car les besoins de la circulation empêcheraient qu’on ne recourût au remboursement extrême ; on irait recevoir l’or en lingots quand on aurait besoin du métal, mais quand on ne voudrait de la monnaie que pour faire des payements, on prendrait incontestablement du papier. D’ailleurs si quelque motif de méfiance menait un grand nombre de porteurs de billets à, la caisse, l’effet d’un pareil remboursement serait de diminuer la somme de la monnaie en circulation, et par conséquent d’en réveiller la demande.}. Il y aurait donc du profit à la faire rentrer. En France, il y a sur la fabrication de la monnaie un seigneuriage d’environ huit pour cent, et on dit aussi que lorsque la monnaie de France est exportée, elle rentre d’elle-même\notedebasdepage{Ceci est une erreur dans laquelle le Docteur \Smith est tombé pour avoir suivi le \publication{Dictionnaire des Monnaies} de \Bazinghen. En 1771, le seigneuriage sur les monnaies d’or fut fixé en France à 1 4/15 pour cent, et sur l’argent à 1 7/24 pour cent. En ce moment il est si peu élevé qu’il couvre à peine les frais de monnayage, qui ne sont que de 1/3 pour cent sur l’or et de 1/2 pour cent sur l’argent. Voyez \Necker, \publication{Administration des finances}, tom. III, p. 8.}. Les variations accidentelles qui surviennent au marché dans le prix de l’or et de l’argent en lingot, proviennent des mêmes causes que celles qui font varier le prix de toutes les autres marchandises. Les pertes fréquentes qui se font de ces métaux par divers accidents sur terre et sur mer, la dépense continuelle qu’on en fait pour dorer en couleur et en or moulu, pour les galons et les broderies, le déchet que produisent l’user et le frottement, tant dans la monnaie que dans les ouvrages d’orfèvrerie\notedebasdepage[\MacCulloch]{Ce déchet est ce qu’on nomme frai en terme de Monnaie.}, tout cela exige, pour réparer ce déchet et ces pertes, une importation continuelle dans tous les pays qui ne possèdent pas de mines en propre. Nous devons présumer que les marchands qui font cette importation cherchent, comme tous les autres marchands, à proportionner chacune de leurs importations aux demandes du moment, autant qu’il leur est possible d’en juger ; cependant, malgré toute leur attention, ils sont quelque­fois au-delà, quelquefois en deçà de la demande. Quand leur importation excède la demande actuelle, plutôt que de courir le risque et la peine de réexporter, ils aiment mieux quelquefois céder une partie de leurs lingots un peu au-dessous du prix moyen ou ordinaire. Quand, d’un autre côté, ils ont importé au-dessous des demandes, ils retirent quelque chose au-delà de ce prix moyen. Mais lorsqu’au milieu de ces fluctuations accidentelles, le \emphase{prix du marché} pour les lingots d’or ou d’argent continue d’une manière durable et constante à baisser plus ou moins au-dessous du prix pour lequel ils sont reçus à la monnaie, ou à s’élever plus ou moins au-dessus, nous pou­vons être certains qu’une infériorité ou supériorité de prix aussi durable et aussi constante est l’effet de quelque changement dans l’état des monnaies, qui rend, pour le moment, une certaine quantité précise de métal brut qu’il devait contenir. La constance et la durée de l’effet supposent une constance et une durée proportionnées dans la cause. La monnaie d’un pays quelconque peut être regardée, dans un temps et un lieu déterminés, comme une mesure plus ou moins exacte des valeurs, selon que la mon­naie courante est plus ou moins exactement conforme au poids et au titre qu’elle annonce, ou bien qu’elle contient plus ou moins exactement la quantité précise d’or fin ou d’argent fin qu’elle doit contenir. Si, par exemple, en Angleterre, quarante-quatre guinées et demie contenaient exactement une livre pesant d’or au titre, ou onze onces d’or et une once d’alliage, la monnaie d’Angleterre serait, dans un temps et un lieu donnés quelconques, une mesure aussi exacte de la valeur actuelle des marchandises que la nature des choses puisse le comporter. Mais si, au moyen du frai, quarante-quatre guinées et demie contiennent, en général, moins d’une livre pesant d’or au titre, le déchet étant toutefois moins fort sur certaines pièces que sur d’autres, alors la mesure de valeur devient sujette à la même espèce d’incertitude à laquelle sont communément exposés tous les autres poids et mesures. Comme il arrive rarement que ceux-ci soient exactement conformes à leur étalon, le marchand ajuste du mieux qu’il peut le prix de ses marchandises, non sur ce que ces poids et mesures devraient être, mais sur ce que, d’après son expérience, il voit qu’ils sont réellement, par évaluation moyenne. En conséquence d’un pareil défaut d’exactitude dans la monnaie, le prix des marchandises se règle de la même manière ; non sur la quantité d’or ou d’argent fin que la monnaie devrait contenir, mais sur ce que, par aperçu, l’expérience fait voir qu’elle en contient pour le moment. Il faut observer que, par le \emphase{prix en argent} des marchandises, j’entends toujours la quantité d’or ou d’argent fin pour laquelle on les vend, sans m’arrêter à la dénomination de la monnaie. Par exemple, je regarde 6 schellings 8 deniers, du temps d’Édouard Ier, comme le même prix en argent qu’une livre sterling d’aujourd’hui, parce que, autant que nous en pouvons juger, ils contenaient environ la même quantité d’argent fin.


%%%%%%%%%%%%%%%%%%%%%%%%%%%%%%%%%%%%%%%%%%%%%%%%%%%%%%%%%%%%%%%%%%%%%%%%%%%%%%%%
%                                  Chapitre 6                                  %
%%%%%%%%%%%%%%%%%%%%%%%%%%%%%%%%%%%%%%%%%%%%%%%%%%%%%%%%%%%%%%%%%%%%%%%%%%%%%%%%

\chapter{Des parties constituantes du prix des marchandises}
\markboth{Des parties constituantes du prix des marchandises}{}

Dans ce premier état informe de la société, qui précède l’accumulation des capitaux et l’appropriation du sol, la seule circonstance qui puisse fournir quelque règle pour les échanges, c’est, à ce qu’il semble, la qualité de travail nécessaire pour acquérir les différents objets d’échange. Par exemple, chez un peuple de chasseurs, s’il en coûte habituellement deux fois plus de peine pour tuer un castor que pour tuer un daim, naturellement un castor s’échangera contre deux daims ou vaudra deux daims. Il est naturel que ce qui est ordinairement le produit de deux jours ou de deux heures de travail, vaille le double de ce qui est ordinairement le produit d’un jour ou d’une heure de travail.

Si une espèce de travail était plus rude que l’autre, on tiendrait naturellement compte de cette augmentation de fatigue, et le produit d’une heure de ce travail plus rude pourrait souvent s’échanger contre le produit de deux heures de l’autre espèce de travail. De même, si une espèce de travail exige un degré peu ordinaire d’habileté ou d’adresse, l’estime que les hommes ont pour ces talents ajoutera naturellement à leur produit une valeur supérieure à ce qui serait dû pour le temps employé au travail. Il est rare que de pareils talents s’acquièrent autrement que par une longue application, et la valeur supérieure qu’on attribue à leur produit n’est souvent qu’une compensation raisonnable du temps et de la peine qu’on a mis à les acquérir.

Dans l’état avancé de la société, on tient communément compte, dans les salaires du travail, de ce qui est dû à la supériorité d’adresse ou de fatigue, et il est vraisemblable qu’on en a agi à peu près de même dans la première enfance des sociétés.

Dans cet état de choses, le produit du travail appartient tout entier au travailleur, et la quantité de travail communément employée à acquérir ou à produire un objet échangeable est la seule circonstance qui puisse régler la quantité de travail que cet objet devra communément acheter, commander ou obtenir en échange.

Aussitôt qu’il y aura des capitaux accumulés dans les mains de quelques particuliers, certains d’entre eux emploieront naturellement ces capitaux à mettre en œuvre des gens industrieux, auxquels ils fourniront des matériaux et des substances, afin de faire un profit sur la vente de leurs produits, ou sur ce que le travail de ces ouvriers ajoute de valeur aux matériaux. Quand l’ouvrage fini est échangé, ou contre de l’argent, ou contre du travail, ou contre d’autres marchandises, il faut bien qu’en outre de ce qui pourrait suffire à payer le prix des matériaux et les salaires des ouvriers, il y ait encore quelque chose de donné pour les profits de l’entrepreneur de l’ouvrage, qui hasarde ses capitaux dans cette affaire. Ainsi, la valeur que les ouvriers ajoutent à la matière se résout alors en deux parties, dont l’une paye leurs salaires, et l’autre les profits que fait l’entrepreneur sur la somme des fonds qui lui ont servi à avancer ces salaires et la matière à travailler. Il n’aurait pas d’intérêt à employer ces ouvriers, s’il n’attendait pas de la vente de leur ouvrage quelque chose de plus que le remplacement de son capital, et il n’aurait pas d’intérêt à employer un grand capital plutôt qu’un petit, si ses profits n’étaient pas en rapport avec l’étendue du capital employé.

Les profits, dira-t-on peut-être, ne sont autre chose qu’un nom différent donné aux salaires d’une espèce particulière de travail, le travail d’inspection et de direction. Ils sont cependant d’une nature absolument différente des salaires ; ils se règlent sur des principes entièrement différents, et ne sont nullement en rapport avec la quantité et la nature de ce prétendu travail d’inspection et de direction. Ils se règlent en entier sur la valeur du capital employé, et ils sont plus ou moins forts, à proportion de l’étendue de ce capital. Supposons, par exemple, que dans une certaine localité où les profits des fonds employés dans les manufactures sont communément de dix pour cent par an, il y ait deux manufactures différentes, chacune desquelles emploie vingt ouvriers à raison de 15 livres par an chacun, soit une dépense de 300 livres par an pour chaque atelier ; supposons encore que la matière première de peu de valeur, employée annuel­lement dans l’une, coûte seulement 700 livres, tandis que dans l’autre on emploie des matières plus précieuses qui coûtent 7,000 livres ; le capital employé annuellement dans l’une sera, dans ce cas, de 1000 livres seulement, tandis que celui employé dans l’autre s’élèvera à 7,300 livres. Or, au taux de 10 pour cent, l’entrepreneur de l’une comptera sur un profit annuel d’environ 100 livres seulement, tandis que l’entre­pre­neur de l’autre s’attendra à un bénéfice d’environ 730 livres. Mais, malgré cette diffé­rence énorme dans leurs profits, il se peut que leur travail d’inspection et de direction soit tout à fait le même ou à peu près l’équivalent. Dans beaucoup de grandes fabriques, souvent presque tout le travail de ce genre est confié à un premier commis. Ses appointements expriment réellement la valeur de ce travail d’inspection et de direction. Quoique, en fixant ce salaire, on ait communément quelque égard, non-seule­ment à son travail et à son degré d’intelligence, mais encore au degré de confiance que son emploi exige, cependant ses appointements ne sont jamais en proportion réglée avec le capital dont il surveille la régie ; et le propriétaire de ce capital, bien qu’il se trouve par là débarrassé de presque tout le travail, n’en compte pas moins que ses profits seront en proportion réglée avec son capital. Ainsi, dans le prix des marchandises, les profits des fonds ou capitaux sont une part constituante dans la valeur, entièrement indifférente des salaires du travail, et réglée sur des principes tout à fait différents.

Dans cet état de choses, le produit du travail n’appartient pas toujours tout entier à l’ouvrier. Il faut, le plus souvent, que celui-ci le partage avec le propriétaire du capital qui le fait travailler. Ce n’est plus alors la quantité de travail communément dépensée pour acquérir ou pour produire une marchandise, qui est la seule circonstance sur laquelle on doive régler la quantité de travail que cette marchandise pourra communément acheter, commander ou obtenir en échange. Il est clair qu’il sera encore dû une quantité additionnelle pour le profit du capital qui a avancé les salaires de ce travail et qui en a fourni les matériaux.

Dès l’instant que le sol d’un pays est devenu propriété privée, les propriétaires, comme tous les autres hommes, aiment à recueillir où ils n’ont pas semé, et ils deman­dent une Rente, même pour le produit naturel de la terre. Il s’établit un prix addi­tionnel sur le bois des forêts, sur l’herbe des champs et sur tous les fruits naturels de la terre, qui, lorsqu’elle était possédée en commun, ne coûtaient à l’ouvrier que la peine de les cueillir, et lui coûtent maintenant davantage. Il faut qu’il paye pour avoir la permission de les recueillir, et il faut qu’il cède au propriétaire du sol une portion de ce qu’il recueille ou de ce qu’il produit par son travail. Cette portion ou, ce qui revient au même, le prix de cette portion constitue le fermage (rent of land) et dans le prix de la plupart des marchandises, elle forme une troisième partie constituante.

Il faut observer que la valeur réelle de toutes les différentes parties constituantes du prix se mesure par la quantité du travail que chacune d’elles peut acheter ou commander. Le travail mesure la valeur, non-seulement de cette partie du prix qui se résout en travail, mais encore de celle qui se résout en fermage et de celle qui se résout en profit\notedebasdepage[Note inédite de \JBS]{Ici je ne peux être de l’avis de Smith, ou plutôt je ne sais quel est l’avis de Smith. Quel travail mesure le profit de la terre et celui du capital qui ont concouru à la création d’un produit ? ce profit est tout à fait indépendant du travail de l’homme et montre que le service rendu par la terre et par le capital, est autre chose que celui rendu par le travail. Même un capital, qui est une accumulation de valeurs dues en partie au travail de l’homme, rend un service où le travail de l’homme n’a plus de part. Le capital représente en partie un travail humain, et partie ; de sa valeur en provient; mais la valeur du service qu’il rend ne représente plus de travail humain. Il n’en entre pas dans le service que rend le capital.}.

Dans toute société, le prix de chaque marchandise se résout définitivement en quelqu’une de ces trois parties ou en toutes trois, et dans les sociétés civilisées, ces parties entrent toutes trois, plus ou moins, dans le prix de la plupart des marchandises, comme parties constituantes de ce prix.

Dans le prix du blé, par exemple, une partie paye la rente du propriétaire, une autre paye les salaires ou l’entretien des ouvriers, ainsi que des bêtes de labour et de charroi employées à produire le blé, et la troisième paye le profit du fermier.

Ces trois parties semblent constituer immédiatement ou en définitive la totalité du prix du blé. On pourrait peut-être penser qu’il faut y ajouter une quatrième partie, nécessaire pour remplacer le capital du fermier ou pour compenser le dépérissement de ses chevaux de labour et autres instruments d’agriculture. Mais il faut considérer que le prix de tout instrument de labourage, tel qu’un cheval de charrue, est lui-même formé de ces mêmes trois parties : la rente de la terre sur laquelle il a été élevé, le travail de ceux qui l’ont nourri et soigné, et les profits d’un fermier qui a fait les avances, tant de cette rente que des salaires de ce travail. Ainsi, quoique le prix du blé doive payer aussi bien le prix du cheval que son entretien, la totalité du prix de ce blé se résout toujours, soit immédiatement, soit en dernière analyse, dans ces mêmes trois parties, rente, travail et profit.

Dans le prix de la farine, il faut ajouter au prix du blé les profits du meunier et les salaires de ses ouvriers ; dans le prix du pain, les profits du boulanger et les salaires de ses garçons, et dans les prix de l’un et de l’autre, le travail de transporter le blé de la maison du fermier à celle du meunier, et de celle du meunier à celle du boulanger, ainsi que les profits de ceux qui avancent les salaires de ce travail.

Le prix du lin se résout dans les mêmes trois parties constituantes que celui du blé. Dans le prix de la toile, il faut comprendre le salaire de ceux qui sérancent le lin, de ceux qui le filent, du tisserand, du blanchisseur, etc., et à tout cela ajouter les profits de ceux qui mettent en œuvre ces différents ouvriers.

À mesure qu’une marchandise particulière vient à être plus manufacturée, cette partie du prix qui se résout en salaires et en profits devient plus grande à proportion de la partie qui se résout en rente\notedebasdepage{Ici la rente est évidemment ce que nous apellons l’intérêt.}. À chaque transformation nouvelle d’un produit, non-seu­le­ment le nombre des profits augmente, mais chaque profit subséquent est plus grand que le précédent, parce que le capital d’où il procède est nécessairement tou­jours plus grand. Le capital qui met en œuvre les tisserands, par exemple, est nécessairement plus grand que celui qui fait travailler les fileurs, parce que non-seulement il remplace ce dernier capital avec ses profits, mais il paye encore, en outre, les salaires des tisserands ; et, comme nous l’avons vu, il faut toujours que les profits soient en certaine proportion avec le capital.

Néanmoins, dans les sociétés les plus avancées, il y a toujours quelques marchandises, mais en petit nombre, dont le prix se résout en deux parties seulement, les salaires du travail et le profit du capital ; et d’autres, en beaucoup plus petit nombre encore, dont le prix consiste uniquement en salaires de travail. Dans le prix du poisson de mer, par exemple, une partie paye le travail des pêcheurs, et l’autre les profits du capital placé dans la pêcherie. Il est rare que la rente fasse partie de ce prix, quoique cela arrive quelquefois, comme je le ferai voir par la suite\notedebasdepage{Chapitre x.}. Il en est autrement, au moins dans la plus grande partie de l’Europe, quant aux pêches de rivière. Une pêcherie de saumon paye une rente, et cette rente, quoiqu’on ne puisse pas trop l’appeler rente de terre, fait une des parties du prix du saumon, tout aussi bien que les salaires et les profits. Dans quelques endroits de l’Écosse, il y a de pauvres gens qui font métier de chercher le long des bords de la mer ces petites pierres tachetées, connues vulgairement sous le nom de cailloux d’Écosse. Le prix que leur paye le lapidaire est en entier le salaire de leur travail ; il n’y entre ni rente ni profit.

Mais la totalité du prix de chaque marchandise doit toujours, en dernière analyse, se résoudre en quelqu’une de ces parties ou en toutes trois, attendu que, quelque partie de ce prix qui reste après le payement de la rente de la terre et le prix de tout le travail employé à la faire croître, à la manufacturer et à la conduire au marché, il faut de toute nécessité que cette partie soit le profit de quelqu’un.

De même que le prix ou la valeur échangeable de chaque marchandise prise séparément, se résout en l’une ou l’autre de ces parties constituantes ou en toutes trois, de même le prix de toutes les marchandises qui composent la somme totale du produit annuel de chaque pays, prises collectivement et en masse, se résout nécessairement en ces mêmes trois parties, et doit se distribuer entre les différents habitants du pays, soit comme salaire de leur travail, soit comme profit de leurs capitaux, soit comme rente de leurs terres. La masse totale de ce que chaque société recueille ou produit annuellement par son travail, ou, ce qui revient au même, le prix entier de cette masse, est primitivement distribué de cette manière entre les différents membres de la société. Salaire, profit et rente sont les trois sources primitives de tout revenu, aussi bien que de toute valeur échangeable. Tout autre revenu dérive, en dernière analyse, de l’une ou de l’autre de ces trois sources.

Quiconque subsiste d’un revenu qui lui appartient en propre, doit tirer ce revenu ou de son travail, ou d’un capital qui est à lui, ou d’une terre qu’il possède. Le revenu qui procède du travail se nomme salaire. Celui qu’une personne retire d’un capital qu’elle dirige ou qu’elle emploie, est appelé profit. Celui qu’en retire une personne qui n’emploie pas elle-même ce capital, mais qui le prête à une autre, se nomme intérêt. C’est une compensation que l’emprunteur paye au prêteur, pour le profit que l’usage de l’argent lui donne occasion de faire. Naturellement, une partie de ce profit appar­tient à l’emprunteur, qui court les risques de l’emploi et qui en a la peine, et une partie au prêteur, qui facilite au premier les moyens de faire ce profit. L’intérêt de l’argent est toujours un revenu secondaire qui, s’il ne se rend pas sur le profit que procure l’usage de l’argent, doit être payé par quelque autre source de revenu, à moins que l’emprunteur ne soit un dissipateur qui contracte une seconde dette pour payer l’intérêt de la première\notedebasdepage{Voyez le chapitre iv du livre II.}. Le revenu qui procède entièrement de la terre est appelé fermage (rent), et appartient au propriétaire. Le revenu du fermier provient en partie de son travail, et en partie de son capital. La terre n’est pour lui que l’instrument qui le met à portée de gagner les salaires de ce travail et de faire profiter ce capital. Tous les impôts et tous les revenus qui en proviennent, les appointements, pensions et annuités de toutes sortes, sont, en dernière analyse, dérivés de l’une ou de l’autre de ces trois sources primitives de revenu, et sont payés, soit immédiatement, soit médiatement, ou avec des salaires de travail, ou avec des profits de capitaux, ou avec des rentes de terre.

Quand ces trois différentes sortes de revenus appartiennent à différentes personnes, on les distingue facilement ; mais quand ils appartiennent à la même personne, on les confond quelquefois l’un avec l’autre, au moins dans le langage ordinaire.

Un propriétaire (gentleman) qui exploite une partie de son domaine, devra gagner, après le payement des frais de culture, et la rente du propriétaire et le profit du fermier. Cependant, tout ce qu’il gagne de cette manière, il est porté à le nommer profit, et il confond ainsi la rente dans le profit, au moins dans le langage ordinaire. C’est le cas de la plupart de nos planteurs de l’Amérique septentrionale et des Indes occidentales ; la plupart d’entre eux exploitent leurs propres terres, et en conséquence on nous parle souvent des profits d’une plantation, mais rarement de la rente qu’elle rapporte.

Il est rare que de petits fermiers emploient un inspecteur pour diriger les princi­pa­les opérations de leur ferme. Ils travaillent eux-mêmes, en général, une bonne partie du temps, et mettent la main à la charrue, à la herse, etc. Ce qui reste de la récolte, la rente payée, doit remplacer, non-seulement le capital qu’ils ont mis dans la culture avec ses profits ordinaires, mais encore leur payer les salaires qui leur sont dus, tant comme ouvriers que comme inspecteurs. Cependant ils appellent profit ce qui reste après la rente payée et le capital remplacé, quoique les salaires y entrent évidemment pour une partie. Le fermier, en épargnant la dépense de ces salaires, les gagne nécessairement pour lui-même. Aussi, dans ce cas, les salaires se confondent avec le profit.

Un ouvrier indépendant qui a un petit capital suffisant pour acheter des matières et pour subsister jusqu’à ce qu’il puisse porter son ouvrage au marché, gagnera à la fois et les salaires du journalier qui travaille sous un maître, et le profit que ferait le maître sur l’ouvrage de celui-ci. Cependant, la totalité de ce que gagne cet ouvrier se nomme profit, et les salaires sont encore ici confondus avec le profit\notedebasdepage{Voyez encore au chapitre x du livre II.}.

Un jardinier qui cultive de ses propres mains son jardin, réunit à la fois dans sa personne les trois différents caractères de propriétaire, de fermier et d’ouvrier. Ainsi, le produit de son jardin doit lui payer la rente du premier, le profit du second et le salaire du troisième. Néanmoins, le tout est regardé communément comme le fruit de son travail. Ici la rente et le profit se confondent avec le salaire.

Comme dans un pays civilisé il n’y a que très-peu de marchandises dont toute la valeur échangeable procède du travail seulement, et que, pour la très-grande partie d’entre elles, la rente et le profit y contribuent pour de fortes portions, il en résulte que le produit annuel du travail de ce pays suffira toujours pour acheter et commander une quantité de travail beaucoup plus grande que celle qu’il a fallu employer pour faire croître ce produit, le préparer et l’amener au marché. Si la société employait annuellement tout le travail qu’elle est en état d’acheter annuellement, comme la quantité de ce travail augmenterait considérablement chaque année, il s’ensuivrait que le produit de chacune des années suivantes serait d’une valeur incomparablement plus grande que celui de l’année précédente. Mais il n’y a aucun pays dont tout le produit annuel soit employé à entretenir des travailleurs\notedebasdepage[\AB]{Ce chapitre, d’une si admirable simplicité, renferme des détails qui semblent aujourd’hui bien vulgaires, mais dont l’exposé constate use véritable découverte. En distinguant d’une manière si juste et si vraie les différentes sources de nos revenus, Adam Smith a rendu un service immense à la science et à l’humanité. Quoi de plus profond et de plus ingénieux à la fois que ces analyses d’où ressort avec tant de clarté la part que prennent aux produits et que doivent prendre aux profils les diverses classes de la société ! « Mais, dit l’auteur, il n’y a aucun pays dont tout le produit annuel soit employé i entretenir des travailleurs ; partout les oisifs en consomment une grande partie… » Voilà le mal nettement indiqué ; notre génération cherche déjà le remède.}. Partout les oisifs en consomment une grande partie ; et selon les différentes proportions dans lesquelles ce produit se partage entre ces deux différentes classes, les travailleurs et les oisifs\notedebasdepage{Voyez plus au long la définition de ces deux classes, selon la théorie de l’auteur, au livre II, chap. iii.}, sa valeur ordinaire ou moyenne doit nécessairement ou augmenter, ou décroître, ou demeurer la même, d’une année à l’autre.

%%%%%%%%%%%%%%%%%%%%%%%%%%%%%%%%%%%%%%%%%%%%%%%%%%%%%%%%%%%%%%%%%%%%%%%%%%%%%%%%
%                                  Chapitre 7                                  %
%%%%%%%%%%%%%%%%%%%%%%%%%%%%%%%%%%%%%%%%%%%%%%%%%%%%%%%%%%%%%%%%%%%%%%%%%%%%%%%%

\chapter{Du prix naturel des marchandises, et de leur prix de marché}
\markboth{Du prix naturel des marchandises, et de leur prix de marché}{}

Dans chaque société, dans chaque localité, il y a un taux moyen ou ordinaire pour les profits dans chaque emploi différent du travail ou des capitaux. Ce taux se règle naturellement, comme je le ferai voir par la suite, en partie par les circonstances géné­rales dans lesquelles se trouve la société, c’est-à-dire sa richesse ou sa pauvreté, son état progressif vers l’opulence, ou stationnaire, ou décroissant, et en partie par la nature particulière de chaque emploi[1].

Il y a aussi, dans chaque société ou canton, un taux moyen ou ordinaire pour les fermages (rents), qui est aussi réglé, comme je le ferai voir[2], en partie par les circons­tances générales dans lesquelles se trouve la société ou la localité dans laquelle la terre est située, et en partie par la fertilité naturelle ou industrielle du sol. 

On peut appeler ce taux moyen et ordinaire le taux naturel du salaire, du profit et du fermage, pour le temps et le lieu dans lesquels ce taux domine communément.

Lorsque le prix d’une marchandise n’est ni plus ni moins que ce qu’il faut pour payer, suivant leurs taux naturels, et le fermage de la terre, et les salaires du travail, et les profits du capital employé à produire cette denrée, la préparer et la conduire au marché, alors cette marchandise est vendue ce qu’on peut appeler son prix naturel[3].

La marchandise est alors vendue précisément ce qu’elle vaut ou ce qu’elle coûte réellement à celui qui la porte au marché ; car bien que, dans le langage ordinaire, on ne comprenne pas dans le prix primitif d’une marchandise le profit de celui qui fait métier de la vendre, cependant, s’il la vendait à un prix qui ne lui rendît pas son profit au taux ordinaire de la localité, il est évident qu’il perdrait à ce métier, puisqu’il aurait pu faire ce profit en employant son capital d’une autre manière. D’ailleurs, son profit constitue son revenu ; c’est, pour le marchand, le fonds d’où il tire sa subsistance. De même que le vendeur avance à ses ouvriers leurs salaires ou leur subsistance pendant que la marchandise se prépare et est conduite au marché, de même il se fait aussi à lui-même l’avance de sa propre subsistance, laquelle, en général, est en raison du profit qu’il peut raisonnablement attendre de la vente de sa marchandise. Ainsi, à moins de lui concéder ce profit, on ne lui aura pas payé le prix qu’on peut regarder, à juste titre, comme celui que cette marchandise lui coûte réellement.

En conséquence, quoique le prix qui lui donne ce profit ne soit pas toujours le plus bas prix auquel un vendeur puisse quelquefois céder sa marchandise, c’est cepen­dant le plus bas auquel, pendant un temps un peu considérable, il soit en état de le faire, au moins s’il jouit d’une parfaite liberté, ou s’il est le maître de changer de mé­tier quand il lui plaît.

Le prix actuel auquel une marchandise se vend communément est ce qu’on appelle son prix de marché. Il peut être ou au-dessus, ou au-dessous, ou précisément au niveau du prix naturel.

Le prix de marché de chaque marchandise particulière est déterminé par la proportion entre la quantité de cette marchandise existant actuellement au marché, et les demandes de ceux qui sont disposés à en payer le prix naturel ou la valeur entière des fermages, profits et salaires qu’il faut payer pour l’attirer au marché. On peut les appeler demandeurs effectifs, et leur demande, demande effective, puisqu’elle suffit pour attirer effectivement la marchandise au marché. Elle diffère de la demande abso­lue. Un homme pauvre peut bien, dans un certain sens, faire la demande d’un carrosse à six chevaux, c’est-à-dire qu’il voudrait l’avoir ; mais sa demande n’est pas une deman­de effective, capable de faire jamais arriver cette marchandise au marché pour le satisfaire.

Quand la quantité d’une marchandise quelconque, amenée au marché, se trouve au-dessous de la demande effective, tous ceux qui sont disposés à payer la valeur entière des fermages, salaires et profits qu’il en coûte pour mettre cette marchandise sur le marché, ne peuvent pas se procurer la quantité qu’ils demandent. Plutôt que de s’en passer tout à fait, quelques-uns d’eux consentiront à donner davantage. Une con­currence s’établira aussitôt entre eux, et le prix de marché s’élèvera plus ou moins au-dessus du prix naturel, suivant que la grandeur du déficit, la richesse ou la fantaisie des concurrents viendront animer plus ou moins cette concurrence. Le même déficit donnera lieu généralement à une concurrence plus ou moins active entre des compé­titeurs égaux en richesse ou en luxe, selon que la marchandise à acheter se trouvera être alors d’une plus ou moins grande importance pour eux ; de là l’élévation exor­bitante dans le prix des choses nécessaires à la vie, pendant le siège d’une ville ou dans une famine.

Lorsque la quantité mise sur le marché excède la demande effective, elle ne peut être entièrement vendue à ceux qui consentent à payer la valeur collective des fermages, salaires et profits qu’il en a coûté pour l’y amener. Il faut bien qu’une partie soit vendue à ceux qui veulent payer moins que cette valeur entière, et le bas prix que donnent ceux-ci réduit nécessairement le prix du tout. Le prix de marché tombera alors plus ou moins au-dessous du prix naturel, selon que la quantité de l’excédent aug­men­tera plus ou moins la concurrence des vendeurs, ou suivant qu’il leur impor­tera plus ou moins de se défaire sur-le-champ de la marchandise. Le même excédent dans l’importation d’une denrée périssable donnera lieu à une concurrence beaucoup plus vive, à cet égard, que dans l’importation d’une marchandise durable : dans une importation d’oranges, par exemple, que dans une de vieux fers. 

Lorsque la quantité mise sur le marché suffit tout juste pour remplir la demande effective, et rien de plus, le prix de marché se trouve naturellement être avec exactitu­de, du moins autant qu’il est possible d’en juger, le même que le prix naturel. Toute la quantité à vendre sera débitée à ce prix, et elle ne saurait l’être à un plus haut prix. La concurrence des différents vendeurs les oblige à accepter ce prix, mais elle ne les oblige pas à accepter moins.

La quantité de chaque marchandise mise sur le marché se proportionne naturel­lement d’elle-même à la demande effective. C’est l’intérêt de tous ceux qui emploient leur terre, leur travail ou leur capital à faire venir quelque marchandise au marché, que la quantité n’en excède jamais la demande effective ; et c’est l’intérêt de tous les autres, que cette quantité ne tombe jamais au-dessous.

Si cette quantité excède pendant quelque temps la demande effective, il faut que quelqu’une des parties constituantes de son prix soit payée au-dessous de son prix naturel.

Si c’est le fermage, l’intérêt des propriétaires les portera sur-le-champ à retirer une partie de leur terre de cet emploi ; et si ce sont les salaires ou les profits, l’intérêt des ouvriers, dans le premier cas, et de ceux qui les emploient, dans le second, les portera à en retirer une partie de leur travail ou de leurs capitaux. La quantité amenée au mar­ché ne sera bientôt plus que suffisante pour répondre à la demande effective. Toutes les différentes parties du prix de cette marchandise se relèveront à leur taux naturel, et le prix total reviendra au prix naturel.

Si, au contraire, la quantité amenée au marché restait pendant quelque temps au-dessous de la demande effective, quelques-unes des parties constituantes de son prix hausseraient nécessairement au-dessus de leur taux naturel. Si c’est le fermage, l’inté­rêt de tous les autres propriétaires les portera naturellement à disposer une plus gran­de quantité de terre pour la production de cette marchandise ; si ce sont les salaires ou les profits, l’intérêt de tous les autres ouvriers et entrepreneurs les portera bientôt à employer plus de travail et de capitaux à la préparer et à la faire venir au marché. La quantité qui y sera portée sera bientôt suffisante pour répondre à la demande effec­tive. Toutes les différentes parties de son prix baisseront bientôt à leur taux naturel, et le prix total retombera au prix naturel.

Le prix naturel est donc, pour ainsi dire, le point central vers lequel gravitent conti­nuellement les prix de toutes les marchandises. Différentes circonstances acci­den­telles peuvent quelquefois les tenir un certain temps élevées au-dessus, et quel­quefois les forcer à descendre un peu au-dessous de ce prix. Mais, quels que soient les obstacles qui les empêchent de se fixer dans ce centre de repos et de perma­nence, ils ne tendent pas moins constamment vers lui.

La somme totale d’industrie employée annuellement pour mettre au marché une marchandise se proportionne ainsi naturellement à la demande effective. Elle tend naturellement à porter toujours au marché cette quantité précise qui peut suffire à la demande, et rien de plus.

Mais, dans certaines branches de productions, la même quantité d’industrie pro­dui­ra, en différentes années, des quantités fort différentes de marchandises, pendant que, dans d’autres branches, elle produira la même quantité ou à peu près. Le même nombre d’ouvriers employés à la culture produira, en différentes années, des quantités fort différentes de blé, de vin, d’huile, de houblon, etc. Mais le même nombre de fi­leurs et de tisserands produira chaque année la même quantité, ou à peu près, de toile ou de drap. Il n’y a que le produit moyen de la première espèce d’industrie qui puisse, en quelque manière, se proportionner à la demande effective ; et comme son produit actuel est souvent beaucoup plus grand et souvent beaucoup moindre que ce produit moyen, la quantité de ces sortes de denrées qui sera mise au marché, tantôt excédera de beaucoup la demande effective, tantôt sera fort au-dessous. Aussi, même en suppo­sant que cette demande continue à rester la même, le prix de ces denrées, au marché, ne sera pas moins sujet à de grandes fluctuations ; il tombera quelquefois bien au-des­sous du prix naturel, et quelquefois s’élèvera beaucoup au-dessus. Dans l’autre espèce d’industrie, le produit de quantités égales de travail étant toujours le même ou à peu près, il s’accordera plus exactement avec la demande effective. Tant que cette deman­de reste la même, le prix de marché pour ces denrées doit vraisemblablement aussi res­­ter le même, et être tout à fait le même que le prix naturel, ou du moins aussi rapproché qu’il est permis d’en juger. Il n’y a personne qui ne sache par expérience que le prix du drap ou de la toile n’est sujet ni à d’aussi fréquentes ni à d’aussi fortes variations que le prix du blé. Le prix des premiers varie seulement en proportion des variations qui surviennent dans la demande ; celui des produits naturels varie non-seulement en proportion des variations de la demande, mais encore il suit les varia­tions beaucoup plus fortes et beaucoup plus fréquentes qui surviennent dans la quantité de ces denrées mise sur le marché pour répondre à la demande.

Les fluctuations accidentelles et momentanées qui surviennent dans le prix de marché d’une denrée, tombent principalement sur les parties de son prix, qui se résol­vent en salaires et en profits. La partie qui se résout en rente en est moins affectée. Une rente qui consiste, ou dans une certaine portion, ou dans une quantité fixe du produit brut, est, sans aucun doute, affectée dans sa valeur annuelle par toutes les fluctuations momentanées et accidentelles qui surviennent dans le prix de marché de ce produit brut ; mais il est rare qu’elles influent sur le taux annuel de cette rente. Quand il s’agit de régler les conditions du bail, le propriétaire et le fermier tâchent, chacun du mieux qu’il peut, de déterminer ce taux d’après le prix moyen et ordinaire du produit, et non pas d’après un prix momentané et accidentel.

Ces sortes de fluctuations affectent les salaires et les profits, tant dans leur valeur que dans leur taux, selon que le marché vient à être surchargé ou à être trop peu fourni de marchandises ou de travail, d’ouvrage fait ou d’ouvrage à faire. Un deuil public fait hausser le prix du drap noir, dont le marché se trouve presque toujours trop peu fourni dans ces occasions, et il augmente les profits des marchands qui en possèdent quelque quantité considérable. Il n’a pas d’effet sur les salaires des ouvriers qui fabriquent le drap. C’est de marchandises, et non pas de travail, que le marché se trouve peu fourni, d’ouvrage fait et non pas d’ouvrage à faire. Mais ce même événement fait hausser les salaires des tailleurs. Dans ce cas, le marché est trop peu fourni de travail ; il y a demande effective de travail, demande d’ouvrage à faire pour plus qu’on ne peut en fournir. Ce deuil fait baisser le prix des soieries et draps de couleurs, et par là il diminue les profits des marchands qui en ont en main une quan­tité considérable. Il réduit aussi les salaires des ouvriers employés à préparer ces sortes de marchandises dont la demande est arrêtée pour six mois, peut-être pour un an. Dans ce cas, le marché est surchargé, tant de marchandise que de travail.

Mais, quoique le prix de marché de chaque marchandise particulière tende ainsi, par une gravitation continuelle, s’il est permis de s’exprimer ainsi, vers le prix naturel, cependant, tantôt des causes accidentelles, tantôt des causes naturelles et tantôt des règlements de police particuliers peuvent, à l’égard de beaucoup de marchandises, tenir assez longtemps de suite le prix de marché au-dessus du prix naturel.
Lorsque, par une augmentation dans la demande effective, le prix de marché de quelque marchandise particulière vient à s’élever considérablement au-dessus du prix naturel, ceux qui emploient leurs capitaux à fournir le marché de cette marchandise ont, en général, grand soin de cacher ce changement. S’il était bien connu, leurs grands profits leur susciteraient tant de nouveaux concurrents engagés par là à em­ployer leurs capitaux de la même manière, que, la demande effective étant pleine­ment remplie, le prix de marché redescendrait bientôt au prix naturel, et peut-être même au-dessous pour quelque temps. Si le marché est à une grande distance de ceux qui le fournissent, ils peuvent quelquefois être à même de garder leur secret pendant plu­sieurs années de suite, et jouir pendant tout ce temps de leurs profits extra­ordi­naires sans éveiller de nouveaux concurrents. Cependant, il est reconnu que des secrets de ce genre sont rarement gardés longtemps, et le profit extraordinaire ne dure guère plus longtemps que le secret.

Les secrets de fabrique sont de nature à être gardés plus longtemps que les secrets de commerce. Un teinturier qui a trouvé le moyen de produire une couleur particu­lière avec des matières qui ne lui coûtent que la moitié du prix de celles qu’on em­ploie communément, peut, avec quelques précautions, jouir du bénéfice de sa décou­verte pendant toute sa vie et la laisser même en héritage à ses enfants. Son gain extraordinaire procède du haut prix qu’on lui paye pour son travail particulier ; ce gain consiste proprement dans les hauts salaires de ce travail[4]. Mais, comme ils se trouvent être répétés sur chaque partie de son capital, et que leur somme totale conserve ainsi une proportion réglée avec ce capital, on les regarde communément comme des profits extraordinaires du capital[5].

De tels renchérissements dans le prix de marché sont évidemment les effets de causes accidentelles particulières, dont néanmoins l’influence peut durer quelquefois pendant plusieurs années de suite.

Il y a de telles productions naturelles qui exigent un sol et une exposition parti­culiers, de sorte que toute la terre propre à les produire dans un grand pays ne suffit pas pour répondre à la demande effective. Ainsi, toute la quantité qui en vient au marché sera livrée à ceux qui consentent à en donner plus qu’ils ne font payer le fermage de la terre qui les produit, les salaires du travail et les profits des capitaux employés à mettre ces produits sur le marché, selon le taux naturel de ces fermages, salaires et profits. Des marchandises de ce genre peuvent continuer, pendant des siècles entiers, à être vendues à ce haut prix ; et, dans ce cas, c’est la partie qui se résout en fermage qui est, en général, celle qu’on paye au-dessus du taux naturel. Le fermage de la terre qui fournit ces productions rares et précieuses, comme le fermage de quelques vignobles de France, renommés par la nature et l’exposition du terrain, est sans proportion ré­glée avec les fermages des autres terres du voisinage, également fertiles et aussi bien cultivées[6]. Au contraire, les salaires du travail et les profits des capitaux employés à mettre sur le marché ces sortes de productions, ne sont guère hors de leur proportion naturelle avec ceux des autres emplois de travail et de capitaux dans le voisinage.

De tels renchérissements dans le prix de marché sont évidemment l’effet de causes naturelles qui peuvent empêcher que la demande effective ne soit jamais pleinement remplie, et qui, par conséquent, peuvent agir toujours.

Un monopole accordé à un individu ou à une compagnie commerçante[7] a le même effet qu’un secret dans un genre de commerce ou de fabrication. Les monopoleurs, en tenant le marché constamment mal approvisionné et en ne répondant jamais pleine­ment à la demande effective, vendent leurs marchandises fort au-dessus du prix natu­rel ; et que leurs bénéfices consistent soit en salaires soit en profits, ils les font monter beaucoup au-delà du taux naturel[8].

Le prix de monopole est, à tous les moments, le plus haut qu’il soit possible de reti­rer. Le prix naturel ou le prix résultant de la libre concurrence est, au contraire, le plus bas qu’on puisse accepter, non pas à la vérité à tous les moments, mais pour un temps un peu considérable de suite. L’un est, à tous les moments, le plus haut qu’on puisse arracher aux acheteurs, ou le plus haut qu’on suppose qu’ils consentiront à donner ; l’autre est le plus bas dont les vendeurs puissent généralement se contenter, pour pouvoir en même temps continuer leur commerce.

Les privilèges exclusifs des corporations, les statuts d’apprentissage[9] et toutes les lois qui, dans les branches d’industrie particulière, restreignent la concurrence à un Plus petit nombre de personnes qu’il n’y en aurait sans ces entraves, ont la même ten­dan­ce que les monopoles, quoique à un moindre degré. Ce sont des espèces de mono­poles, étendus sur plus de monde, et ils peuvent souvent, pendant des siècles et dans des professions tout entières, tenir le prix de marché de quelques marchandises parti­culières au-dessus du prix naturel, et maintenir quelque peu au-dessus du taux naturel tant les salaires du travail que les profits des capitaux qu’on y emploie.

Des renchérissements de ce genre, dans le prix de marché, dureront aussi long­temps que les règlements de police qui y ont donné lieu. 

Quoique le prix de marché d’une marchandise particulière puisse continuer long­temps à rester au-dessus du prix naturel, il est difficile qu’il puisse continuer long­temps à rester au-dessous. Quelle que soit la partie de ce prix qui soit payée au-dessous du taux naturel, les personnes qui y ont intérêt sentiront bientôt le dommage qu’elles éprouvent, et aussitôt elles retireront, ou tant de terre, ou tant de travail, ou tant de capitaux de ce genre d’emploi, que la quantité de cette marchandise qui sera amenée au marché ne sera bientôt plus que suffisante pour répondre à la demande effective. De cette manière son prix de marché remontera bientôt au prix naturel, ou du moins tel sera l’effet général partout où règne une entière liberté.

À la vérité, les mêmes statuts d’apprentissage et autres lois de corporations qui, tant qu’un genre d’industrie prospère, mettent l’ouvrier à même de hausser ses salaires un peu au-dessus de leur taux naturel, l’obligent aussi quelquefois, quand ce même genre vient à déchoir, à les laisser aller bien au-dessous de ce taux. Si, dans le premier cas, elles excluent beaucoup de gens de sa profession, dans l’autre, par la même raison, elles l’excluent lui-même de beaucoup de professions. Cependant, l’effet de ces règlements n’est pas à beaucoup près aussi durable quand il fait baisser les salaires de l’ouvrier au-dessous du taux naturel, que quand il les élève au-dessus. Dans ce dernier cas, cet effet pourrait durer pendant plusieurs siècles ; mais, dans l’autre, il ne peut guère s’étendre au-delà de la vie de quelques ouvriers qui ont été élevés à ce métier dans le temps où il prospérait. Quand ceux-ci ne seront plus, le nombre de ceux qui s’adonneront à cette profession se proportionnera naturellement à la demande effective. Pour tenir les salaires ou les profits au-dessous de leur taux naturel, dans des emplois particuliers, pendant une suite de générations, il ne faut pas moins qu’une police aussi violente que celle de l’Indostan ou de l’ancienne Égypte, où tout homme était tenu, par principe de religion, de suivre les mêmes occupations que son père, et où le changement de profession passait pour le plus horrible sacrilège.

Je ne pense pas qu’il soit nécessaire, pour le moment, d’en dire davantage sur ces déviations accidentelles ou permanentes du prix de marché, de la ligne du prix naturel.

Le prix naturel varie lui-même avec le taux naturel de chacune de ses parties cons­ti­tuantes, le salaire, le profit et le fermage ; et le taux de ces trois parties varie dans chaque société, selon les circonstances où elle se trouve, selon son état de ri­ches­se ou de pauvreté, suivant sa marche Progressive, stationnaire ou rétrograde. je vais tâcher d’exposer, avec autant de développement et de clarté qu’il me sera pos­sible, les causes de ces différentes variations, dans les quatre chapitres suivants.

D’abord, je tâcherai d’expliquer quelles sont les circonstances qui déterminent naturellement le taux des salaires, et de quelle manière peuvent influer sur ces cir­constances l’état de richesse ou de pauvreté de la société, sa marche progressive, stationnaire ou rétrograde.

Secondement, je tâcherai de montrer quelles sont les circonstances qui détermi­nent naturellement le taux des Profits, et de quelle manière aussi les mêmes variations dans l’état de la société influent sur ces circonstances.

Quoique les salaires et les profits pécuniaires soient très-différents dans les divers emplois du travail et des capitaux, cependant il semble en général qu’il s’établit une certaine proportion entre les salaires pécuniaires dans tous les divers emplois du travail, ainsi qu’entre les profits pécuniaires dans tous les divers emplois des capitaux. Cette proportion dépend, comme on le verra par la suite, en partie de la nature des emplois, en partie de la police et des différentes lois de la société dans laquelle s’exer­cent ces emplois. Mais, quoique cette proportion dépende à plusieurs égards des lois et de la police de la société, il ne paraît pas qu’elle dépende beaucoup de l’état de ri­ches­se ou de pauvreté de cette société, de sa marche progressive, stationnaire ou rétro­grade ; il parait au contraire que, dans ces différents états de la société, cette proportion se maintient la même ou à peu près. Je tâcherai donc, en troisième lieu, de développer toutes les différentes circonstances qui règlent cette proportion[10].

Quatrièmement enfin, je tâcherai de faire voir quelles sont les circonstances qui règlent la rente de la terre, et qui tendent à élever ou à abaisser le prix réel de toutes les différentes substances qu’elle produit.
 
 
 
↑ Chap. viii, ix et x de ce livre.
↑ Chap. xi de ce livre.
↑ J.-B. Say, et, après lui, presque tous les économistes ont donné le nom de frais de production à ce qu’Adam Smith appelle prix naturel. On se sert aussi quelquefois des mots prix de revient ; mais on ne peut raisonnablement appeler prix le taux auquel une chose ne se vend pas. A. B.
↑ Le gain qui procède de secrets industriels diffère évidemment et du salaire et du profil ; et quoiqu’on ne lui ait pas appliqué le nom de rente, il en constitue réellement une. La rente de la terre est ce surplus, qui, par le haut prix de ses produits, surpasse les dépenses de culture. Le gain extraordinaire provoqué par un secret industriel est un surplus analogue, que le haut prix de ses produits élève au-dessus des salaires et des profits. Quelle est donc la différence entre ces deux sources de bénéfices ? Le Dr. Smith prétend que ce gain extraordinaire de l’industrie vient du haut prix accordé au travail privé. Mais le détenteur du secret peut ne pas travailler, et généralement même cette hypothèse se vérifie. D’un autre côté il n’accorde pas de hauts salaires pour le travail qu’il réclame. La marchandise n’exige pas plus de frais pour être amenée au marché ; mais elle acquiert un meilleur prix, par cette même raison qui élève le taux du blé ; c’est-à-dire, qu’arec un prix plus bas on le consommerait plutôt que de le produire. Buchannan.
↑ Il est évident que ces produits ne peuvent, en aucune manière, être l’objet d’une proportion avec le capital employé. Le prix d’une marchandise est tel, qu’il puisse laisser au propriétaire du secret un excédant sur les salaires et les profils : c’est pourquoi cet excédant, loin de former aucune proportion avec les fonds et le travail employés, en demeure complètement indépendant. Buchanan.
↑ Voyez chap. xi de ce livre.
↑ Sur les effets des monopoles, voyez la troisième partie du chap. vii du liv. IV, et presque tout le liv. IV, sur les diverses sortes de monopoles.
↑ Le profit découlant d’un monopole repose précisément sur le même principe que la rente : un monopole produit artificiellement ce qui, dans les cas de la rente, est amené par des causes naturelles. Il restreint l’approvisionnement du marché jusqu’à ce que le prix se soit élevé au-dessus du niveau des salaires et du profit. C’est par une semblable restriction dans l’approvisionnement que la rente augmente. Le monopoleur, comme le déclare le Dr. Smith, n’élève pas les salaires et le profit au-dessus de leurs taux naturels ; mais il prélève un excédant au-dessus des salaires et du profit. Buchanan.
↑ Voyez la section deuxième du chap. x de ce livre.
↑ M. Thomas Tooke, qui a rendu de grands services à l’économie politique, a publié, en 1838, un excellent ouvrage en 2 vol., intitulé : History of Prices from 1793 to 1837. Ce livre peut être considéré comme le complément indispensable du chapitre d’Adam Smith sur les prix. 

%%%%%%%%%%%%%%%%%%%%%%%%%%%%%%%%%%%%%%%%%%%%%%%%%%%%%%%%%%%%%%%%%%%%%%%%%%%%%%%%
%                                  Chapitre 8                                  %
%%%%%%%%%%%%%%%%%%%%%%%%%%%%%%%%%%%%%%%%%%%%%%%%%%%%%%%%%%%%%%%%%%%%%%%%%%%%%%%%

\chapter{Des salaires du travail}
\markboth{Des salaires du travail}{}

Ce qui constitue la récompense naturelle ou le salaire du travail, c’est le produit du travail.

Dans cet état primitif qui précède l’appropriation des terres et l’accumulation des capitaux, le produit entier du travail appartient à l’ouvrier. Il n’a ni propriétaire ni maître avec qui il doive partager.

Si cet état eût été continué, le salaire du travail aurait augmenté avec tout cet accrois­sement de la puissance productive du travail, auquel donne lieu la division du tra­vail. Toutes les choses seraient devenues, par degrés, de moins en moins chères. Elles auraient été produites par de moindres quantités de travail, et elles auraient été pareillement achetées avec le produit de moindres quantités, puisque, dans cet état de choses, des marchandises produites par des quantités égales de travail se seraient naturellement échangées l’une contre l’autre.

Mais quoique, dans la réalité, toutes les choses fussent devenues à meilleur mar­ché, cependant il y aurait eu beaucoup de choses qui, en apparence, seraient devenues plus chères qu’auparavant, et qui auraient obtenu en échange une plus grande quantité d’autres marchandises. Supposons, par exemple, que, dans la plupart des branches d’industrie, la puissance productive du travail ait augmenté dans la proportion de dix à un, c’est-à-dire que le travail d’un jour produise actuellement dix fois autant d’ouvrage qu’il en aurait produit dans l’origine ; supposons en outre que, dans un emploi parti­culier, ces facultés n’aient fait de progrès que comme deux à un, c’est-à-dire que, dans une industrie particulière, le travail d’une journée produise actuellement deux fois autant d’ouvrage qu’il en aurait produit dans l’origine ; supposons en outre que, dans un emploi particulier, ces facultés n’aient fait de progrès que comme deux à un, c’est-à-dire que, dans une industrie particulière, le travail d’une journée produise seulement deux fois plus d’ouvrage qu’il n’aurait fait anciennement. En échangeant le produit du travail d’un jour, dans la plupart des industries, contre le produit du travail d’un jour dans cet emploi particulier, on donnera dix fois la quantité primitive d’ouvrage que produisaient ces industries, pour acheter seulement le double de la quantité primitive de l’autre. Ainsi une quantité particulière, une livre pesant, par exemple, de cette der­nière espèce d’ouvrage, paraîtra être cinq fois plus chère qu’auparavant. Dans le fait, pourtant, elle est deux fois à meilleur marché qu’elle n’était dans l’origine. Quoique, pour l’acheter, il faille donner cinq fois autant d’autres espèces de marchandises, cependant il ne faut que la moitié seulement du travail qu’elle coûtait anciennement, pour l’acheter ou la produire actuellement ; elle est donc deux fois plus aisée à acquérir qu’elle n’était alors.

Mais cet état primitif, dans lequel l’ouvrier jouissait de tout le produit de son propre travail, ne put pas durer au-delà de l’époque où furent introduites l’appropria­tion des terres et l’accumulation des capitaux. Il y avait donc longtemps qu’il n’existait plus, quand la puissance productive du travail parvint à un degré de perfection consi­dérable, et il serait sans objet de rechercher plus avant quel eût été l’effet d’un pareil état de choses sur la récompense ou le salaire du travail.

Aussitôt que la terre devient une propriété privée, le propriétaire demande pour sa part presque tout le produit que le travailleur peut y faire croître ou y recueillir. Sa rente est la première déduction que souffre le produit du travail appliqué à la terre[1].

Il arrive rarement que l’homme qui laboure la terre possède par-devers lui de quoi vivre jusqu’à ce qu’il recueille la moisson. En général, sa subsistance lui est avancée sur le capital d’un maître, le fermier qui l’occupe, et qui n’aurait pas d’intérêt à le faire s’il ne devait pas prélever une part dans le produit de son travail, ou si son capital ne devait pas lui rentrer avec un profit. Ce profit forme une seconde déduction sur le produit du travail appliqué à la terre.

Le produit de presque tout autre travail est sujet à la même déduction en faveur du profit. Dans tous les métiers, dans toutes les fabriques, la plupart des ouvriers ont besoin d’un maître qui leur avance la matière du travail, ainsi que leurs salaires et leur subsistance, jusqu’à ce que leur ouvrage soit tout à fait fini. Ce maître prend une part du produit de leur travail ou de la valeur que ce travail ajoute à la matière à laquelle il est appliqué, et c’est cette part qui constitue son profit.

À la vérité, il arrive quelquefois qu’un ouvrier qui vit seul et indépendant, a assez de capital pour acheter à la fois la matière du travail et pour s’entretenir jusqu’à ce que son ouvrage soit achevé. Il est en même temps maître et ouvrier, et il jouit de tout le produit de son travail personnel ou de toute la valeur que ce travail ajoute à la matière sur laquelle il s’exerce. Ce produit renferme ce qui fait d’ordinaire deux revenus dis­tincts, appartenant à deux personnes distinctes, les profits du capital et les salaires du travail.

Ces cas-là, toutefois, ne sont pas communs, et dans tous les pays de l’Europe, pour un ouvrier indépendant, il y en a vingt qui servent sous un maître ; et partout on en­tend, par salaires du travail, ce qu’ils sont communément quand l’ouvrier et le pro­prié­taire du capital qui lui donne de l’emploi sont deux personnes distinctes.

C’est par la convention qui se fait habituellement entre ces deux personnes, dont l’intérêt n’est nullement le même, que se détermine le taux commun des salaires. Les ouvriers désirent gagner le plus possible ; les maîtres, donner le moins qu’ils peuvent ; les premiers sont disposés à se concerter pour élever les salaires, les seconds pour les abaisser.

Il n’est pas difficile de prévoir lequel des deux partis, dans toutes les circonstances ordinaires, doit avoir l’avantage dans le débat, et imposer forcément à l’autre toutes ses conditions. Les maîtres, étant en moindre nombre, peuvent se concerter plus aisément ; et de plus, la loi les autorise à se concerter entre eux, ou au moins ne le leur interdit pas, tandis qu’elle l’interdit aux ouvriers. Nous n’avons point d’actes du parlement contre les ligues qui tendent à abaisser le prix du travail ; mais nous en avons beaucoup contre celles qui tendent à le faire hausser[2]. Dans toutes ces luttes, les maîtres sont en état de tenir ferme plus longtemps. Un propriétaire, un fermier, un maître fabricant ou marchand, pourraient en général, sans occuper un seul ouvrier, vi­vre un an ou deux sur les fonds qu’ils ont déjà amassés. Beaucoup d’ouvriers ne pour­raient pas subsister sans travail une semaine, très-peu un mois et à peine un seul une année entière. À la longue, il se peut que le maître ait autant besoin de l’ouvrier que celui-ci a besoin du maître ; mais le besoin du premier n’est pas si pressant.

On n’entend guère parler, dit-on, de coalitions entre les maîtres, et tous les jours on parle de celles des ouvriers. Mais il faudrait ne connaître ni le monde, ni la matière dont il s’agit, pour s’imaginer que les maîtres se liguent rarement entre eux Les maî­tres sont en tout temps et partout dans une sorte de ligue tacite, mais constante et uniforme, pour ne pas élever les salaires au-dessus du taux actuel. Violer cette règle est partout une action de faux frère et un sujet de reproche pour un maître parmi ses voisins et ses pareils. À la vérité, nous n’entendons jamais parler de cette ligue, parce qu’elle est l’état habituel, et on peut dire l’état naturel de la chose, et que personne n’y fait attention. Quelquefois les maîtres font entre eux des complots particuliers pour faire baisser au-dessous du taux habituel les salaires du travail. Ces complots sont toujours conduits dans le plus grand silence et dans le plus grand secret jusqu’au moment de l’exécution ; et quand les ouvriers cèdent comme ils font quelquefois, sans résistance, quoiqu’ils sentent bien le coup et le sentent fort durement, personne n’en entend parler. Souvent, cependant, les ouvriers opposent à ces coalitions particulières une ligue défensive ; quelquefois aussi, sans aucune provocation de cette espèce, ils se coalisent de leur propre mouvement, pour élever le prix de leur travail. Leurs pré­textes ordinaires sont tantôt le haut prix des denrées, tantôt le gros profit que font les maîtres sur leur travail. Mais que leurs ligues soient offensives ou défensives, elles sont toujours accompagnées d’une grande rumeur. Dans le dessein d’amener l’affaire à une prompte décision, ils ont toujours recours aux clameurs les plus emportées, et quelquefois ils se portent à la violence et aux derniers excès. Ils sont désespérés, et agissent avec l’extravagance et la fureur de gens au désespoir, réduits à l’alternative de mourir de faim ou d’arracher à leurs maîtres, par la terreur, la plus prompte condes­cendance à leurs demandes. Dans ces occasions, les maîtres ne crient pas moins haut de leur côté ; ils ne cessent de réclamer de toutes leurs forces l’autorité des magistrats civils, et l’exécution la plus rigoureuse de ces lois si sévères portées contre les ligues des ouvriers, domestiques et journaliers. En conséquence, il est rare que les ouvriers tirent aucun fruit de ces tentatives violentes et tumultueuses, qui, tant par l’interven­tion du magistrat civil que par la constance mieux soutenue des maîtres et la nécessité où sont la plupart des ouvriers de céder pour avoir leur subsistance du moment, n’aboutissent en général à rien autre chose qu’au châtiment ou à la ruine des chefs de l’émeute[3]. 

Mais quoique les maîtres aient presque toujours nécessairement l’avantage dans leurs querelles avec leurs ouvriers, cependant il y a un certain taux au-dessous duquel il est impossible de réduire, pour un temps un peu considérable, les salaires ordinai­res, même de la plus basse espèce de travail.

Il faut de toute nécessité qu’un homme vive de son travail, et que son salaire suffise au moins à sa subsistance ; il faut même quelque chose de plus dans la plupart des circonstances ; autrement il serait impossible au travailleur d’élever une famille, et alors la race de ces ouvriers ne pourrait pas durer au-delà de la première génération[4]. À ce compte, M. Cantillon[5] paraît supposer que la plus basse classe des simples manœu­vres doit partout gagner au moins le double de sa subsistance, afin que ces travailleurs soient généralement en état d’élever deux enfants ; on suppose que le travail de la femme suffit seulement à sa propre dépense, à cause des soins qu’elle est obligée de donner à ses enfants. Mais on calcule que la moitié des enfants qui naissent meurent avant l’âge viril. Il faut, par conséquent, que les plus pauvres ouvriers tâchent, l’un dans l’autre, d’élever au moins quatre enfants, pour que deux seulement aient la chan­ce de parvenir à cet âge. Or, on suppose que la subsistance nécessaire de quatre enfants est à peu près égale à celle d’un homme fait. Le même auteur ajoute que le travail d’un esclave bien constitué est estimé valoir le double de sa subsistance, et il pense que celui de l’ouvrier le plus faible ne peut pas valoir moins que celui d’un esclave bien constitué. Quoi qu’il en soit, il paraît au moins certain que, pour élever une famille, même dans la plus basse classe des plus simples manœuvres, il faut néces­sairement que le travail du mari et de la femme puisse leur rapporter quelque chose de plus que ce qui est précisément indispensable pour leur propre subsistance ; mais dans quelle proportion ? Est-ce dans celle que j’ai citée, ou dans toute autre ? C’est ce que je ne prendrai pas sur moi de décider. C’est peu consolant pour les indi­vidus qui n’ont d’autre moyen d’existence que le travail.

Il y a cependant certaines circonstances qui sont quelquefois favorables aux ouvriers, et les mettent dans le cas de hausser beaucoup leurs salaires au-dessus de ce taux, qui est évidemment le plus bas qui soit compatible avec la simple humanité[6]. 

Lorsque, dans un pays, la demande de ceux qui vivent de salaires, ouvriers, jour­naliers, domestiques de toute espèce, va continuellement en augmentant ; lorsque cha­que année fournit de l’emploi pour un nombre plus grand que celui qui a été employé l’année précédente, les ouvriers n’ont pas besoin de se coaliser pour faire hausser leurs salaires. La rareté des bras occasionne une concurrence parmi les maîtres, qui mettent à l’enchère l’un sur l’autre pour avoir des ouvriers, et rompent ainsi volontairement la ligue naturelle des maîtres contre l’élévation des salaires.

Évidemment, la demande de ceux qui vivent de salaires ne peut augmenter qu’à proportion de l’accroissement des fonds destinés à payer des salaires. Ces fonds sont de deux sortes : la première consiste dans l’excédant du revenu sur les besoins ; la se­con­de, dans l’excédant du capital nécessaire pour tenir occupés les maîtres du travail.

Quand un propriétaire, un rentier, un capitaliste a un plus grand revenu que celui qu’il juge nécessaire à l’entretien de sa famille, il emploie tout ce surplus ou une partie de ce surplus à entretenir un ou plusieurs domestiques. Augmenter ce surplus, et natu­rellement il augmentera le nombre de ses domestiques[7]. 

Quand un ouvrier indépendant, tel qu’un tisserand ou un cordonnier, a amassé plus de capital qu’il ne lui en faut pour acheter la matière première de son travail per­son­nel et pour subsister lui-même jusqu’à la vente de son produit, il emploie naturel­lement un ou plusieurs journaliers avec ce surplus, afin de bénéficier sur leur travail. Augmentez ce surplus, et naturellement il augmentera le nombre de ses ouvriers.

Ainsi, la demande de ceux qui vivent de salaires augmente nécessairement avec l’accroissement des revenus et des capitaux de chaque pays, et il n’est pas possible qu’elle augmente sans cela. L’accroissement des revenus et des capitaux est l’accrois­se­ment de la richesse nationale ; donc, la demande de ceux qui vivent de salaires aug­mente naturellement avec l’accroissement de la richesse nationale, et il n’est pas possible qu’elle augmente sans cela[8]. 

Ce n’est pas l’étendue actuelle de la richesse nationale, mais c’est son progrès con­tinuel qui donne lieu à une hausse dans les salaires du travail. En conséquence, ce n’est pas dans les pays les plus riches que les salaires sont le plus élevés, mais c’est dans les pays qui font le plus de progrès, ou dans ceux qui marchent le plus vite vers l’opulence. Certainement, l’Angleterre est pour le moment un pays beaucoup plus riche qu’aucune partie de l’Amérique septentrionale ; cependant les salaires du travail sont beaucoup plus élevés dans l’Amérique septentrionale que dans aucun endroit de l’Angleterre. Dans la province de New-York[9], un simple manœuvre gagne par jour 3 schellings 6 deniers, monnaie du pays, valant 2 schellings sterling ; un charpentier de marine, 10 schellings 6 deniers, monnaie du pays, avec une pinte de rhum qui vaut 6 deniers sterling, en tout 6 schellings 6 deniers sterling ; un charpentier en bâtiment et un maçon, 8 schellings, argent courant du pays, valant 4 schellings 6 deniers sterling ; un garçon tailleur, 5 schellings, argent courant, valant environ 2 schellings 10 deniers sterling. Ces prix sont tous au-dessus de ceux de Londres, et on assure que dans les autres colonies les salaires sont aussi élevés qu’à New-York. Dans toute l’Amérique septentrionale, les denrées sont à beaucoup plus bas prix qu’en Angleterre ; on n’y a jamais vu de disette. Dans les années les plus mauvaises, il n’y a que l’exportation qui ait souffert, mais il y a toujours eu assez pour la consommation du pays. Ainsi donc, si le prix du travail en argent y est plus élevé que dans aucun endroit de la mère pa­trie, son prix réel, la quantité réelle de choses propres aux besoins et aisances de la vie, que ce prix met à la disposition de l’ouvrier, s’y trouve supérieur, dans une pro­portion encore bien plus grande.

Mais, quoique l’Amérique septentrionale ne soit pas encore aussi riche que l’An­gle­terre, elle est beaucoup plus florissante et elle marche avec une bien plus grande rapidité vers l’acquisition de nouvelles richesses. La marque la plus décisive de la prospérité d’un pays est l’augmentation du nombre de ses habitants[10]. On suppose que dans la Grande-Bretagne et la plupart des autres pays de l’Europe, ce nombre ne dou­ble guère en moins de cinq cents ans[11]. Dans les colonies anglaises de l’Amérique sep­ten­­trionale, on a trouvé qu’il doublait en vingt ou vingt-cinq ans ; et cet accroissement de population est bien moins dû à l’immigration continuelle de nouveaux habitants, qu’à la multiplication rapide de l’espèce. On dit que ceux qui parviennent à un âge avancé y comptent fréquemment de cinquante à cent, et quelquefois plus, de leurs propres descendants. Le travail y est si bien récompensé, qu’une nombreuse famille d’enfants, au lieu d’être une charge, est une source d’opulence et de prospérité pour les parents. On compte que le travail de chaque enfant, avant qu’il puisse quitter leur maison, leur rapporte par an 100 livres de bénéfice net. Une jeune veuve, avec quatre ou cinq enfants, qui aurait tant de peine à trouver un second mari dans les classes moyennes ou inférieures du peuple en Europe, est là le plus souvent un parti recher­ché comme une espèce de fortune. La valeur des enfants est le plus grand de tous les encouragements au mariage. Il ne faut donc pas s’étonner de ce qu’on se marie en général fort jeune dans l’Amérique septentrionale. Malgré le grand accrois­se­ment de population qui résulte de tant de mariages entre de très-jeunes gens, on s’y plaint néanmoins continuellement de l’insuffisance des bras. Il paraît que, dans ce pays, la demande des travailleurs et les fonds destinés à les entretenir croissent encore trop vite pour qu’on trouve autant de monde qu’on voudrait en employer.

Quand même la richesse d’un pays serait très-grande, cependant, s’il a été long­temps dans un état stationnaire, il ne faut pas s’attendre à y trouver les salaires bien élevés. Les revenus et les capitaux de ses habitants, qui sont les fonds destinés au payement des salaires, peuvent bien être d’une très-grande étendue ; mais s’ils ont con­ti­nué, pendant plusieurs siècles, à être de la même étendue ou à peu près, alors le nombre des ouvriers employés chaque année pourra aisément répondre, et même plus que répondre, au nombre qu’on en demandera l’année suivante. On y éprouvera rare­ment une disette de bras, et les maîtres ne seront pas obligés de mettre à l’enchère les uns sur les autres pour en avoir. Au contraire, dans ce cas, les bras se multiplieront au-delà de la demande. Il y aura disette constante d’emploi pour les ouvriers, et ceux-ci seront obligés, pour en obtenir, d’enchérir au rabais les uns sur les autres[12]. Si, dans un tel pays, les salaires venaient jamais à monter au-delà du taux suffisant pour faire subsister les ouvriers et les mettre en état d’élever leur famille, la concurrence des ouvriers et l’intérêt des maîtres réduiraient bientôt ces salaires au taux le plus bas que puisse permettre la simple humanité. La Chine a été, pendant une longue période, un des plus riches pays du monde, c’est-à-dire un des plus fertiles, des mieux cultivés, des plus industrieux et des plus peuplés ; mais ce pays paraît être depuis très-long­temps dans un état stationnaire. Marco Polo, qui l’observait il y a plus de cinq cents ans, nous décrit l’état de sa culture, de son industrie et de sa population presque dans les mêmes termes que les voyageurs qui l’observent aujourd’hui. Peut-être même cet empire était-il déjà, longtemps avant ce voyageur, parvenu à la plénitude d’opulence que la nature de ses lois et de ses institutions lui permet d’atteindre. Les rapports de tous les voyageurs, qui varient sur beaucoup de points, s’accordent sur les bas prix des salaires du travail et sur la difficulté qu’éprouve un ouvrier en Chine pour élever sa famille[13]. Si, en remuant la terre toute une journée, il peut gagner de quoi acheter le soir une petite portion de riz, il est fort content. La condition des artisans y est encore pire, s’il est possible. Au lieu d’attendre tranquillement dans leurs ateliers que leurs prati­ques les fassent appeler, comme en Europe, ils sont continuellement à courir par les rues avec les outils de leur métier, offrant leurs services et mendiant pour ainsi dire de l’ouvrage. La pauvreté des dernières classes du peuple à la Chine dépasse de beau­coup celle des nations les plus misérables de l’Europe. Dans le voisinage de Canton, plusieurs centaines, on dit même plusieurs milliers de familles, n’ont point d’habita­tions sur la terre et vivent habituellement dans de petits bateaux de pêcheurs, sur les canaux et les rivières. La subsistance qu’ils peuvent s’y procurer y est tellement rare, qu’on les voit repêcher avec avidité les restes les plus dégoûtants jetés à la mer par quelque vaisseau d’Europe. Une charogne, un chat ou un chien mort, déjà puant et à demi pourri, est une nourriture tout aussi bien reçue par eux que le serait la viande la plus saine par le peuple des autres pays. Le mariage n’est pas encouragé à la Chine par le profit qu’on retire des enfants, mais par la permission de les détruire. Dans toutes les grandes villes, il n’y a pas de nuit où l’on n’en trouve plusieurs exposés dans les rues, ou noyés comme on noie de petits chiens. On dit même qu’il y a des gens qui se chargent ouvertement de cette horrible fonction, et qui en font métier pour gagner leur vie.

Cependant la Chine, quoique demeurant toujours peut-être dans le même état, ne paraît pas rétrograder. Nulle part ses villes ne sont désertées par leurs habitants ; nulle part on n’y abandonne les terres une fois cultivées. Il faut donc qu’il y ait annuelle­ment la même, ou environ la même quantité de travail accompli, et que les fonds destinés à faire subsister les ouvriers ne diminuent pas, par conséquent, d’une manière sensible. Ainsi, malgré toutes les peines qu’elles ont à subsister, il faut bien que les plus basses classes d’ouvriers trouvent à se tirer d’affaire d’une manière ou d’une autre, assez du moins pour se maintenir dans leur nombre ordinaire.

Mais il en serait autrement dans un pays où les fonds destinés à faire subsister le travail viendraient à décroître sensiblement. Chaque année la demande de domesti­ques et d’ouvriers, dans les différentes espèces de travail, serait moindre qu’elle n’aurait été l’année précédente. Un grand nombre de ceux qui auraient été élevés dans des métiers d’une classe supérieure, ne pouvant plus se procurer d’ouvrage dans leur emploi, seraient bien aises d’en trouver dans les classes inférieures. Les classes les plus basses se trouvant surchargées non-seulement de leurs propres ouvriers, mais encore de ceux qui y reflueraient de toutes les autres classes, il s’y établirait une si grande concurrence pour le travail, que les salaires seraient bornés à la plus chétive et à la plus misérable subsistance de l’ouvrier[14]. Beaucoup d’entre eux, même à de si dures conditions, ne pourraient pas trouver d’occupation ; ils seraient réduits à périr de faim, ou bien à chercher leur subsistance en mendiant ou en s’abandonnant au crime. La misère, la famine et la mortalité désoleraient bientôt cette classe, et de là s’étendraient aux classes supérieures, jusqu’à ce que le nombre des habitants du pays se trouvât ré­duit à ce qui pourrait aisément subsister par la quantité de revenus et de capitaux qui y seraient restés, et qui auraient échappé à la tyrannie ou à la calamité universelle. Tel est peut-être, à peu de chose près, l’état actuel du Bengale et de quelques autres éta­blis­sements anglais dans les Indes orientales[15]. Dans un pays fertile qui a déjà été extrêmement dépeuplé, où, par conséquent, la subsistance ne devrait pas être très-diffi­­cile, et où, malgré tout cela, il meurt de faim, dans le cours d’une année, trois à quatre cent mille personnes, il n’y a nul doute que les fonds destinés à faire subsister le travail du pauvre ne décroissent avec une grande rapidité. La différence qui se trou­ve entre l’état de l’Amérique septentrionale et celui des Indes orientales est peut-être le fait le plus propre à faire sentir la différence qui existe entre l’esprit de la consti­tution britannique, qui protège et gouverne le premier de ces pays, et l’esprit de la compagnie mercantile qui maîtrise et qui opprime l’autre[16].

Ainsi, un salaire qui donne au travail une récompense libérale est à la fois l’effet nécessaire et le symptôme naturel de l’accroissement de la richesse nationale ; celui qui ne fournit à l’ouvrier pauvre qu’une chétive subsistance est l’indication d’un état stationnaire ; et, enfin, celui qui ne lui donne pas même de quoi subsister et le réduit à mourir de faim signifie que les richesses décroissent avec rapidité[17].

Dans la Grande-Bretagne, le salaire du travail semble, dans le temps actuel, être évidemment au-dessus de ce qui est précisément nécessaire pour mettre l’ouvrier en état d’élever une famille. Pour nous en assurer, il ne sera pas nécessaire de nous jeter dans des calculs longs et incertains sur la somme qu’il faut à l’ouvrier pour lui donner cette possibilité. Il y a plusieurs signes certains qui démontrent que les salaires du travail ne sont, dans aucun endroit de ce pays, réduits à ce taux, qui est le plus bas que la simple humanité puisse accorder.

Premièrement, dans presque toutes les parties de la Grande-Bretagne, il y a une distinction entre les salaires d’été et ceux d’hiver ; même dans les travaux de la dernière classe, les salaires d’été sont toujours les plus élevés. Or, à cause de la dépen­se extraordinaire du chauffage, la subsistance d’une famille est plus coûteuse en hiver. Ainsi, les salaires étant plus élevés lorsque la dépense est moins forte, il paraît clair qu’ils ne sont pas réglés sur ce qu’exige le strict nécessaire, mais bien sur la quantité et la valeur présumées du travail. On dira peut-être que l’ouvrier doit épargner une partie de ses salaires d’été pour subvenir aux dépenses de l’hiver, et que les salaires de toute l’année n’excèdent pas ce qui est nécessaire pour l’entretien de sa famille pendant toute l’année.

Cependant, nous ne traiterions pas de cette manière un esclave ou quelqu’un qui dépendrait absolument et immédiatement de nous pour sa subsistance. Sa subsistance journalière serait proportionnée à ses besoins de chaque jour.

En second lieu, les salaires du travail, dans la Grande-Bretagne, ne suivent pas les fluctuations du prix des denrées. Ce prix varie partout d’une année à l’autre, souvent d’un mois à l’autre. Mais en beaucoup d’endroits le prix pécuniaire du travail reste invariablement le même, quelquefois un demi-siècle de suite[18]. Si, dans ces localités, un pauvre ouvrier peut entretenir sa famille dans les années de cherté, il doit être à son aise dans les temps où les denrées sont à un prix modéré, et dans l’abondance, aux époques de grande abondance. Le haut prix des denrées, pendant ces dix dernières années, n’a entraîné, dans beaucoup d’endroits du royaume, aucune augmentation sen­sible dans le prix pécuniaire du travail ; elle a eu lieu, à la vérité, en quelques en­droits ; mais elle était plutôt due probablement à l’augmentation de la demande du travail qu’à l’augmentation du prix des denrées.

Troisièmement, si le prix des denrées varie plus que les salaires du travail d’une année à l’autre, d’un autre côté les salaires varient plus que le prix des denrées d’un lieu à un autre. Le prix du pain et de la viande de boucherie est, en général, le même ou à peu près le même, dans la plus grande partie du Royaume-Uni. Ces denrées et presque toutes les autres qui se vendent au détail, qui est la manière dont les pauvres ouvriers achètent leurs denrées, sont en général à tout aussi bon ou même à meilleur marché dans les grandes villes que dans les endroits les plus reculés de la campagne, pour des raisons que j’aurai occasion de développer dans la suite[19]. Mais les salaires du travail dans une grande ville et dans son voisinage sont fréquemment d’un quart ou d’un cinquième, de vingt à vingt-cinq pour cent plus élevés qu’ils ne sont à quelques milles de distance ; 18 deniers par jour peuvent passer pour le prix du travail le plus simple à Londres et dans ses environs ; à quelques milles de là, il tombe à 14 ou à 15 ; son prix est sur le pied de 10 deniers à Édimbourg et dans les environs ; à quelques milles de distance, il tombe à 8, qui est le prix ordinaire du plus simple travail dans la plus grande partie du plat pays d’Écosse, où il varie infiniment moins qu’en Angle­terre[20]. Une telle différence dans les prix, qui paraît ne pas suffire toujours pour trans­porter un homme d’une paroisse à une autre, entraînerait infailliblement un si grand transport de denrées même les plus volumineuses, non-seulement d’une paroisse à une autre, mais même d’un bout du royaume, presque d’un bout du monde à l’autre, qu’elles se trouveraient bientôt ramenées à peu près au niveau. Malgré tout ce qu’on a dit de la légèreté et de l’inconstance de la nature humaine, il paraît évidemment, par l’expérience, que rien n’est plus difficile à déplacer que l’homme. Si donc, dans ces parties du royaume où le prix du travail est le plus bas, les ouvriers pauvres peuvent néanmoins soutenir leurs familles, ils doivent être dans l’abondance dans les pays où il est le plus élevé.

Quatrièmement, les variations dans le prix du travail ne correspondent point, quant aux lieux et aux temps, à celles du prix des denrées, mais elles ont lieu souvent dans les directions tout à fait opposées[21]. 

Le grain, qui est la nourriture des gens du peuple, est plus cher en Écosse qu’il ne l’est en Angleterre, d’où l’Écosse tire chaque année des approvisionnements très-consi­dérables[22]. Le blé anglais doit se vendre nécessairement plus cher en Écosse, qui est le pays où il est apporté, qu’il ne se vend en Angleterre, le pays d’où il vient ; et pro­portionnellement à sa qualité, il ne peut pas se vendre plus cher en Écosse que le blé écossais, qui vient au même marché en concurrence avec lui. La qualité du grain dépend principalement de la quantité de farine qu’il rend à la mouture, et à cet égard le blé d’Angleterre est tellement supérieur à celui d’Écosse, que, quoique souvent plus cher en apparence ou en proportion de sa mesure en volume, il est en général à meilleur marché dans la réalité, ou en proportion de sa qualité ou même de sa mesure en poids. Au contraire, le prix du travail est plus cher en Angleterre qu’en Écosse. Si donc le travail du pauvre suffit dans cette partie du Royaume-Uni pour le mettre en état de soutenir sa famille, il doit, dans l’autre, mettre l’ouvrier dans l’abondance. Il est vrai que le bas peuple d’Écosse consomme de la farine d’avoine pour la plus grande et la meilleure partie de sa nourriture, qui, en général, est fort inférieure à celle des per­sonnes de la même classe en Angleterre. Mais cette différence dans leur manière de subsister est seulement l’effet de la différence qui existe dans leurs salaires, quoique, par une étrange méprise, je l’aie souvent entendu considérer comme en étant la cause. Ce n’est pas parce qu’un homme roule carrosse tandis que son voisin va à pied, que l’un est riche et l’autre pauvre ; mais l’un roule carrosse parce qu’il est riche, et l’autre va à pied parce qu’il est pauvre.

Pendant le cours du siècle dernier, une année dans l’autre, le grain a été plus cher dans chacune des parties des deux royaumes, que pendant le cours de celui-ci[23]. C’est là une vérité de fait sur laquelle on ne peut maintenant élever de doute raisonnable, et la preuve en est même bien plus décisive, s’il est possible, pour l’Écosse que pour l’Angleterre ; elle est fondée sur les relevés authentiques des marchés publics, qui sont des évaluations, faites sur serment, d’après l’état actuel des marchés, de toutes les diverses espèces de grains, dans chaque différent comté d’Écosse[24]. Si une preuve aussi directe avait besoin de quelque témoignage accessoire pour la confirmer, je ferais observer que la même chose a eu également lieu en France, et probablement dans la plupart des autres pays de l’Europe. Quant à la France, la preuve est la plus claire pos­sible. Mais s’il est certain que, dans chacune des parties du Royaume-Uni, le grain a été un peu plus cher dans le dernier siècle que dans celui-ci, il est également certain que le travail y a été à beaucoup plus bas prix. Si le travail des individus pauvres a pu leur suffire alors à soutenir leurs familles, il doit donc les mettre aujourd’hui dans une plus grande aisance[25]. Dans le dernier siècle, le salaire journalier du travail d’un ma­nœu­vre était le plus habituellement, dans la majeure partie de l’Écosse, de 6 deniers en été et de 5 en hiver. On continue encore aujourd’hui à payer 3 schellings par semai­ne, ce qui fait, à très-peu de chose près, le même prix, dans les montagnes d’Écosse et dans les îles de l’Ouest. Dans la plus grande partie du plat pays de l’Écosse, les salaires les plus ordinaires du travail de manœuvre sont aujourd’hui à 8 deniers par jour ; 10 deniers, quelquefois 1 schelling, aux environs d’Édimbourg, dans les comtés qui confinent à l’Angleterre, probablement à cause de ce voisinage, et dans un petit nombre d’autres endroits où la demande de travail a considérablement augmenté depuis peu, comme aux environs de Glasgow, de Carron, de Ayrshire, etc. En Angle­terre, l’agriculture, les manufactures et le commerce ont commencé à faire des progrès beaucoup plus tôt qu’en Écosse. La demande de travail et, par conséquent, son prix ont dû nécessairement augmenter avec ces progrès. C’est pour cela que, dans le dernier siècle aussi bien que dans le présent, les salaires du travail ont été plus élevés en Angleterre qu’en Écosse. Ils se sont aussi considérablement élevés depuis ce temps, quoiqu’il soit plus difficile de déterminer de combien, à cause de la plus grande variété des salaires qui y ont été payés en différents endroits. En 1614, la paye d’un soldat d’infanterie était la même qu’à présent, 8 deniers par jour[26]. Quand cette paye fut d’abord établie, elle dut nécessairement être réglée sur les salaires habituels des manœuvres, qui est la classe du peuple dont on tire le plus communément les soldats d’infanterie. Le lord chef de justice Hales, qui écrivait sous Charles II, calcule la dépense nécessaire de la famille d’un ouvrier, consistant en six personnes, le père, la mère, deux enfants en état de faire quelque chose, et deux incapables de travail, et il évalue cette dépense à 10 schellings par semaine, ou 26 livres par an. Si le produit de leur travail n’atteint pas cette somme, il faut qu’ils la complètent, dit-il, en men­diant, ou par le vol. Il paraît avoir fait sur cette matière les recherches les plus exactes[27]. En 1688, M. Grégoire King, dont le docteur Davenant vante si haut l’habi­le­té en arithmétique politique, a calculé le revenu ordinaire des manœuvres et domes­ti­ques de campagne à 15 livres par an pour chaque famille, qu’il suppose consister, l’une dans l’autre, en trois personnes et demie. Son calcul, quoique différent en appa­rence, correspond exactement, au fond, avec celui du juge Hales. Ils évaluent l’un et l’autre la dépense de ces familles, pour une semaine, à environ 20 deniers par tête. Depuis ce temps, le revenu pécuniaire et la dépense de ces familles ont considérable­ment augmenté dans la plus grande partie du royaume ; dans quelques endroits plus, dans d’autres moins, mais presque nulle part autant qu’on l’a avancé dernièrement au public, dans certaines évaluations exagérées de l’état actuel des salaires. On doit observer qu’il n’est possible de déterminer exactement nulle part le prix du travail ; car on paye souvent des prix différents pour le même genre de travail, et cela dans le même temps, non-seulement en raison de l’adresse de l’ouvrier, mais encore en raison de la facilité ou de la dureté du maître. Partout où les salaires ne sont pas fixés par la loi, tout ce que nous pouvons espérer de déterminer, c’est leur taux le plus habituel ; et l’expérience semble démontrer que la loi ne peut jamais les régler convenablement, quoiqu’elle ait eu souvent la prétention de le faire[28].

La récompense réelle du travail, la quantité réelle des choses propres aux besoins et commodités de la vie, qu’il peut procurer à l’ouvrier, a augmenté, dans le cours de ce siècle, dans une proportion bien plus forte encore que son prix en argent. Non-seulement le grain a un peu baissé de prix, mais encore beaucoup d’autres denrées qui fournissent au pauvre économe et laborieux des aliments sains et agréables, sont descendues à un prix infiniment plus bas. Les pommes de terre, par exemple, ne coû­tent pas, dans la plus grande partie du royaume, la moitié du prix qu’elles coû­taient il y a trente ou quarante ans. On en peut dire autant des choux, des navets, des carot­tes, toutes denrées qu’on ne cultivait jamais autrefois qu’à la bêche, mais qu’aujourd’hui on fait venir communément à la charrue. Les produits du jardinage ont aussi beaucoup baissé de prix. Dans le siècle dernier, les pommes et même les oignons consommés dans la Grande-Bretagne étaient en très-grande partie tirés de la Flandre. Les manu­factures de toiles et de draps communs se sont perfectionnées au point de fournir aux ouvriers des habillements meilleurs et à plus bas prix, et les fabriques de métaux communs sont aussi devenues, par leur progrès, en état de leur fournir des outils meilleurs et à moindre prix et, de plus, une quantité d’ustensiles de ménage agréables et commodes[29]. À la vérité, le savon, le sel, la chandelle, le cuir et les liqueurs fermen­tées sont devenus, beaucoup plus chers, principalement à cause des impôts qui ont été établis sur ces denrées[30]. Mais la quantité que les ouvriers pauvres sont obligés d’en consommer est si petite, que l’augmentation de ces prix est loin de compenser la diminution survenue dans le prix d’une infinité d’autres choses. Les plaintes que nous entendons chaque jour sur les progrès du luxe qui gagne les ouvriers les plus pauvres, lesquels ne se contentent plus aujourd’hui de la nourriture, des vêtements et du logement qui leur suffisaient dans l’ancien temps, ces plaintes nous prouvent que ce n’est pas seulement le prix pécuniaire du travail, mais que c’est aussi sa récompense réelle qui a augmenté.

Cette amélioration survenue dans la condition des dernières classes du peuple doit-elle être regardée comme un avantage ou comme un inconvénient pour la socié­té ? Au premier coup d’œil, la réponse paraît extrêmement simple. Les domesti­ques, les ouvriers et artisans de toute sorte composent la plus grande partie de toute société politique. Or, peut-on jamais regarder comme un désavantage pour le tout ce qui améliore le sort de la plus grande partie ? Assurément, on ne doit pas regarder comme heu­reuse et prospère une société dont les membres les plus nombreux sont réduits à la pau­vreté et à la misère. La seule équité, d’ailleurs, exige que ceux qui nourrissent, ha­billent et logent tout le corps de la nation, aient, dans le produit de leur propre travail, une part suffisante pour être eux-mêmes passablement nourris, vêtus et logés[31]. 

Quoique, sans aucun doute, la pauvreté décourage le mariage, cependant elle ne l’em­pê­che pas toujours ; elle paraît même être favorable à la génération. Une monta­gnarde à demi mourante de faim a souvent plus d’une vingtaine d’enfants ; tandis qu’une belle dame qui a grand soin de sa personne, est quelquefois incapable d’en avoir un seul et, en général, se trouve épuisée par deux ou trois couches. La stérilité, qui est si fréquente chez les femmes du grand monde, est extrêmement rare parmi celles d’une condition inférieure. Dans le beau sexe, le luxe, qui enflamme peut-être la passion pour les jouissances, semble toujours affaiblir et souvent détruire les facultés de la génération.

Mais si la pauvreté n’empêche pas d’engendrer des enfants, elle est un très-grand obstacle à ce qu’on puisse les élever. Le tendre rejeton est produit, mais il est placé dans un sol si froid, dans un climat si rigoureux, que bientôt il se dessèche et périt. J’ai souvent entendu dire qu’il n’est pas rare, dans les montagnes d’Écosse, qu’une mère qui a eu vingt enfants n’en ait pas conservé deux vivants. Plusieurs officiers fort expérimentés m’ont assuré que, bien loin de trouver à recruter leur régiment parmi les enfants de soldats qui y naissent, ils n’ont même jamais pu s’y fournir de tambours et de fifres. Cependant il est rare de voir nulle part un plus grand nombre de jolis enfants que dans les environs d’une caserne. Un très-petit nombre d’entre eux arrivent à l’âge de treize ou quatorze ans. Dans quelques endroits, la moitié des enfants qui naissent meurt avant quatre ans ; dans beaucoup d’autres avant sept, et dans presque tous avant neuf ou dix. Cette grande mortalité, toutefois, se rencontrera principalement parmi les enfants des basses classes, que leurs parents ne peuvent pas soigner comme le font ceux d’une condition plus élevée. Quoique leurs mariages soient, en général, plus féconds que ceux des gens du monde, cependant la proportion d’enfants qui arrivent jusqu’à l’âge fait y est beaucoup moindre. Dans les hôpitaux d’enfants trouvés, et parmi les enfants élevés à la charité des paroisses, la mortalité est encore beaucoup plus grande que parmi ceux du bas peuple[32].

Naturellement, toutes les espèces d’animaux multiplient à proportion de leurs moyens de subsistance, et aucune espèce ne peut jamais multiplier au-delà. Mais dans les sociétés civilisées, ce n’est que parmi les classes inférieures du peuple que la disette de subsistance peut mettre des bornes à la propagation de l’espèce humaine ; et cela ne peut arriver que d’une seule manière, en détruisant une grande partie des enfants que produisent les mariages féconds de ces classes du peuple[33].

Ces bornes tendront naturellement à s’agrandir par une récompense plus libérale du travail, qui mettra les parents à portée de mieux soigner leurs enfants et, par consé­quent, d’en élever un plus grand nombre. Il est bon d’observer encore qu’elle opérera nécessairement cet effet, à peu près dans la proportion que déterminera la demande de travail. Si cette demande va continuellement en croissant, la récompense du travail doit nécessairement donner au mariage et à la multiplication des ouvriers un encoura­gement tel, qu’ils soient à même de répondre à cette demande toujours croissante par une population aussi toujours croissante.

Supposez dans un temps cette récompense moindre que ce qui est nécessaire pour produire cet effet, le manque de bras la fera bientôt monter ; et si vous la supposez, dans un autre temps, plus forte qu’il ne faut pour ce même effet, la multiplication excessive d’ouvriers la rabaissera bientôt à ce taux nécessaire. Dans l’un de ces cas, le marché serait tellement dépourvu de travail, et il s’en trouverait tellement surchargé dans l’autre, qu’il forcerait bientôt le prix du travail à revenir à un taux qui s’accordât avec ce qu’exigeraient les circonstances où se trouverait la société. C’est ainsi que la demande d’hommes règle nécessairement la production des hommes, comme fait la demande à l’égard de toute autre marchandise ; elle hâte la production quand celle-ci marche trop lentement, et l’arrête quand elle va trop vite. C’est cette demande qui règle et qui détermine l’état où est la propagation des hommes, dans tous les différents pays du monde, dans l’Amérique septentrionale, en Europe et à la Chine ; qui la fait marcher d’un pas si rapide dans la première de ces contrées ; qui lui donne dans l’autre une marche lente et graduelle, et qui la rend tout à fait stationnaire dans la troisième[34].

C’est aux dépens du maître, a-t-on dit, que les esclaves s’usent et vieillissent, tan­dis que les serviteurs libres s’usent et vieillissent à leurs propres dépens. Cependant, cette espèce de déchet qui provient du temps et du service, est, pour les uns comme pour les autres, une charge ou une dépense qui doit être également supportée par le maître. Les salaires qu’on paye à des gens de journée et domestiques de toute espèce, doivent être tels que ceux-ci puissent, l’un dans l’autre, continuer à maintenir leur po­pu­lation, suivant que peut le requérir l’état croissant ou décroissant, ou bien station­naire, de la demande qu’en fait la société. Mais quoique le maître paye également ce qu’il faut pour remplacer un jour le domestique libre, il lui en coûte bien moins que pour un esclave. Le fonds destiné à remplacer et à réparer, pour ainsi dire, le déchet ré­sultant du temps et du service dans la personne de l’esclave, est ordinairement sous l’administration d’un maître peu attentif ou d’un inspecteur négligent. Celui qui est destiné au même emploi, à l’égard du serviteur fibre, est économisé par les mains mê­mes du serviteur libre. Dans l’administration du premier s’introduisent naturellement les désordres qui règnent, en général, dans les affaires du riche ; la frugalité sévère et l’attention parcimonieuse du pauvre s’établissent aussi naturellement dans l’adminis­tration du second. Avec une administration différente, il faudra, pour remplir le même objet, des degrés de dépense fort différents. En conséquence, l’expérience de tous les temps et de tous les pays s’accorde, je crois, pour démontrer que l’ouvrage fait par des mains libres revient définitivement à meilleur compte que celui qui est fait par des esclaves[35]. C’est ce qui se voit même à Boston, à New-York et à Philadelphie, où les salaires du travail le plus simple sont si élevés[36].

La récompense libérale du travail, qui est l’effet de l’accroissement de la richesse nationale, devient donc aussi la cause d’accroissement de la population. Se plaindre de la libéralité de cette récompense, c’est se plaindre de ce qui est à la fois l’effet et la cause de la plus grande prospérité publique.

Il est peut-être bon de remarquer que c’est dans l’état progressif de la société, lors­qu’elle est en train d’acquérir successivement plus d’opulence, et non pas lorsqu’elle est parvenue à la mesure complète de richesse dont elle est susceptible, que véritable­ment la condition de l’ouvrier pauvre, celle de la grande masse du peuple, est plus heu­reuse et plus douce ; elles est dure dans l’état stationnaire ; elle est misérable dans l’état de déclin. L’état progressif est, pour tous les différents ordres de la société, l’état de la vigueur et de la santé parfaites ; l’état stationnaire est celui de la pesanteur et de l’inertie ; l’état rétrograde est celui de la langueur et de la maladie.

De même que la récompense libérale du travail encourage la population, de même aussi elle augmente l’industrie des classes inférieures. Ce sont les salaires du travail qui sont l’encouragement de l’industrie, et celle-ci, comme tout autre qualité de l’hom­me, se perfectionne à proportion de l’encouragement qu’elle reçoit. Une subsistance abondante augmente la force physique de l’ouvrier ; et la douce espérance d’améliorer sa condition et de finir peut-être ses jours dans le repos et dans l’aisance, l’excite à tirer de ses forces tout le parti possible. Aussi verrons-nous toujours les ouvriers plus actifs, plus diligents, plus expéditifs là où les salaires sont élevés, que là où ils sont bas ; en Angleterre, par exemple, plus qu’en Écosse, dans le voisinage des grandes villes, plus que dans des campagnes éloignées. Il y a bien quelques ouvriers qui, lors­qu’ils peuvent gagner en quatre jours de quoi subsister toute la semaine, passeront les trois autres jours dans la fainéantise. Mais, à coup sûr, ce n’est pas le fait du plus grand nombre. Au contraire, on voit souvent les ouvriers qui sont largement payés à la pièce, s’écraser de travail, et ruiner leur santé et leur tempérament en peu d’années. À Londres et dans quelques autres endroits, un charpentier passe pour ne pas conser­ver plus de huit ans sa pleine vigueur. Il arrive la même chose à peu près dans beau­coup d’autres métiers où les ouvriers sont payés à la pièce, comme ils le sont, en général, dans beaucoup de professions, et même dans les travaux des champs, partout où les salaires sont plus au-dessus du taux habituel. Il n’y a presque aucune classe d’artisans qui ne soit sujette à quelque infirmité particulière, occasionnée par une application excessive à l’espèce de travail qui la concerne. Ramuzzini, célèbre méde­cin italien, a écrit un traité particulier sur ce genre de maladies. Nous ne regardons pas chez nous les soldats comme la classe du peuple la plus laborieuse ; cependant, quand on a employé les soldats à quelque espèce particulière d’ouvrage où on les payait bien et à la pièce, il est arrivé souvent que les officiers ont été obligés de convenir avec l’entrepreneur qu’on ne leur laisserait pas gagner par jour plus d’une certaine somme, fixée d’après le taux auquel ils étaient payés. Avant qu’on eût pris cette précaution, l’émulation réciproque et le désir de gagner davantage les poussaient souvent à forcer le travail et à s’exténuer par un labeur excessif. Cette fainéantise de trois jours de la semaine, dont on se plaint tant et si haut, n’a souvent pour véritable cause qu’une application forcée pendant les quatre autres. Un grand travail de corps ou d’esprit, continué pendant plusieurs jours de suite, est naturellement suivi, chez la plupart des hommes, d’un extrême besoin de relâche qui est presque irrésistible, à moins qu’il ne soit contenu par la force ou par une impérieuse nécessité. C’est le cri de la nature qui veut impérieusement être soulagée, quelquefois seulement par du repos, quelquefois aussi par de la dissipation et de l’amusement. Si on lui désobéit, il en résulte souvent des conséquences dangereuses, quelquefois funestes, qui presque tou­jours amènent un peu plus tôt ou un peu plus tard le genre d’infirmité qui est particu­lière au métier. Si les maîtres écoutaient toujours ce que leur dictent à la fois la raison et l’humanité, ils auraient lieu bien souvent de modérer plutôt que d’exciter l’applica­tion au travail, chez une grande partie de leurs ouvriers. je crois que, dans quelque métier que ce soit, on trouvera que celui qui travaille avec assez de modération pour être en état de travailler constamment, non-seulement conserve le plus longtemps sa santé, mais encore est celui qui, dans le cours d’une année, fournit la plus grande quantité d’ouvrage.

On a prétendu que, dans les années d’abondance, les ouvriers étaient, en général, plus paresseux et que, dans les années de cherté, ils étaient plus laborieux que dans les temps ordinaires. On en a conclu qu’une subsistance abondante énervait leur acti­vité, et qu’une subsistance chétive les animait au travail. Qu’un peu plus d’aisance qu’à l’ordinaire puisse rendre certains ouvriers paresseux, c’est ce qu’on ne saurait nier ; mais que cette aisance produise le même effet sur la plupart d’entre eux, ou bien que les hommes, en général, soient mieux disposés à travailler quand ils sont mal nour­ris que quand ils sont bien nourris ; quand ils ont le cœur abattu, que quand ils sont contents et animés ; quand ils sont souvent malades, que quand ils jouissent générale­ment d’une bonne santé, c’est ce qui ne paraît pas fort probable. Il est à remar­quer que les années de cherté sont en général des années de maladies et de mortalité pour les basses classes, et qui ne peuvent manquer de diminuer le produit de leur travail.

Dans les années d’abondance, les domestiques quittent souvent leurs maîtres, et se fient à leur propre industrie pour gagner par eux-mêmes leur subsistance. Mais ce bas prix des vivres, en augmentant le fonds qui est destiné à entretenir des domestiques, encourage les maîtres, et principalement les fermiers, à en employer un plus grand nombre. Dans ces circonstances-là, les fermiers trouvent que leur blé leur rapporte plus en l’employant à entretenir de nouveaux travailleurs, que s’ils le vendaient au bas prix du marché. La demande de domestiques augmente, tandis que le nombre de ceux qui peuvent prétendre à cette demande diminue. Le prix du travail doit donc souvent hausser dans les années de bon marché.

Dans les années de cherté, la difficulté et l’incertitude de se procurer des subsis­tances rendent tous ces gens-là très-empressés à se remettre en service. Mais le haut prix des vivres, en diminuant le fonds destiné à entretenir des domestiques, dispose les maîtres à réduire plutôt qu’à augmenter le nombre de ceux qu’ils emploient. Il arrive aussi que, dans les années de cherté, de pauvres ouvriers indépendants mangent souvent le petit capital qui leur servait à se procurer la matière du travail, et qu’ils sont obligés de se remettre à la journée pour gagner leur subsistance. Le nombre de ceux qui cherchent de l’emploi est plus grand que le nombre des hommes qui peuvent en trouver facilement ; beaucoup d’entre eux sont disposes a en accepter à des conditions inférieures aux conditions ordinaires, et les salaires, tant des domestiques que des journaliers, baissent souvent dans les années de cherté.

Ainsi, les maîtres de tout genre font souvent des marchés plus avantageux avec leurs domestiques et ouvriers dans des années de cherté, que dans celles d’abondance, et dans les premières ils les trouvent plus soumis et plus dociles. Ils doivent donc naturellement vanter ces années comme plus favorables à l’industrie. D’ailleurs, les propriétaires et les fermiers, deux des classes de maîtres les plus étendues, ont une autre raison pour aimer les années de cherté. Les rentes des uns et les profits des autres dépendent beaucoup du prix des denrées. On ne peut rien imaginer de plus absurde que de croire qu’en général les hommes travailleront moins quand ils travail­le­ront pour leur propre compte, que quand ils travailleront pour le compte d’autrui. Un pauvre ouvrier indépendant sera généralement plus laborieux que ne le sera même un ouvrier qui travaille à la pièce. L’un jouit de tout le produit de son industrie, l’autre le partage avec un maître. L’un, dans son état d’isolement et d’indépendance, est moins exposé à être tenté par les mauvaises compagnies qui perdent si souvent les mœurs de l’autre, dans les grandes manufactures. La supériorité de l’ouvrier indépendant doit être encore bien plus grande sur ces ouvriers qui sont loués au mois ou à l’année, et qui ont toujours les mêmes salaires et la même subsistance, qu’ils fassent soit beau­coup, soit peu d’ouvrage. Or, les années d’abondance tendent à augmenter la propor­tion des ouvriers indépendants sur les domestiques et journaliers, et les années de cherté tendent à la diminuer.

Un auteur français de beaucoup de savoir et de sagacité, M. Messance, receveur des tailles de l’élection de Saint-Étienne[37], essaie de démontrer que les pauvres travail­lent plus dans les années de bas prix que dans les années de cherté, et pour cela il com­pare la quantité et la valeur des marchandises fabriquées, dans ces deux circons­tances contraires, en trois différentes manufactures : l’une de gros draps établie à Elbeuf, une de toiles et une autre de soieries, établies toutes trois dans l’étendue de la généralité de Rouen. Il paraît, d’après son calcul relevé sur les registres des bureaux publics, que la quantité et la valeur des marchandises fabriquées dans ces trois manu­factures ont généralement été plus grandes dans les années de bas prix que dans les années de cherté, et qu’elles ont toujours été plus grandes dans les années du prix le plus bas, et plus faibles dans les années de la plus grande cherté. Toutes les trois paraissent être des manufactures dans un état stationnaire, c’est-à-dire que, si leur produit varie quelque peu d’une année à l’autre, au total elles ne marchent ni en avant ni en arrière.

La fabrication des toiles en Écosse, et celle de gros draps dans la partie occiden­tale du comté d’York, sont des manufactures croissantes dont le produit en général, à quelques variations près, va toujours en augmentant en quantité et en valeur. En examinant cependant les comptes qu’on a publiés de leur produit annuel, je n’ai pas remarqué que les variations aient eu quelque rapport sensible avec le bas prix ou la cherté des temps. En 1740, année de grande disette, ces deux manufactures paraissent, dans le fait, avoir déchu d’une manière fort considérable. Mais en 1756, autre année de grande cherté, la manufacture d’Écosse fit un pas plus rapide qu’à l’ordinaire. La manufacture de la province d’York, à la vérité, alla en déclinant, et son produit fut au-dessous de ce qu’il avait été en 1755, et cela jusqu’à l’année 1766, après la révocation de l’acte du timbre de l’Amérique. Dans cette année et dans la suivante, il s’éleva alors beaucoup plus haut qu’il n’avait jamais été auparavant, et il a toujours continué ainsi depuis.

Quant aux grandes manufactures dont les marchandises doivent être vendues au loin, leur produit doit nécessairement dépendre beaucoup moins du bon marché ou de la cherté des temps dans les endroits où elles sont établies, que des circonstances qui influent sur la demande dans les endroits où s’en fait la consommation ; telles que la paix ou la guerre, la prospérité ou la décadence de quelque autre manufacture rivale, et les bonnes ou mauvaises dispositions des principaux consommateurs. D’ailleurs, une grande partie du travail extraordinaire, qui se fait probablement dans les années de bon marché, ne paraît jamais sur les registres publics des manufactures. Les salariés qui quittent leurs maîtres s’établissent à leur propre compte. Les femmes retournent chez leurs parents, et ordinairement elles filent pour se vêtir elles et leur famille. Les ouvriers indépendants ne travaillent même pas toujours pour vendre au public, mais ils se trouvent employés par leurs voisins à des ouvrages destinés à l’usage de la famille. Ainsi, il arrive fort souvent que le produit de leur travail ne figure point dans ces registres dont on publie quelquefois les relevés avec tant d’éta­lage, et sur lesquels nos marchands et nos manufacturiers prétendent souvent, assez mal-à-propos, calculer la prospérité ou la décadence des empires.

Quoique les variations dans le prix du travail, non-seulement ne correspondent pas toujours avec celles du prix des vivres, mais se manifestent en sens opposé, il ne faudrait pas pourtant s’imaginer, d’après cela, que le prix des vivres n’a pas d’influence sur le prix du travail. Le prix pécuniaire du travail est nécessairement réglé par deux circonstances, la demande du travail et le prix des choses propres aux besoins et commodités de la vie. La quantité des choses propres aux besoins et commodités de la vie qu’il faut donner à l’ouvrier, est déterminée par l’état où se trouve la demande du travail, selon que cet état est croissant, stationnaire ou décroissant, ou bien selon qu’il exige une population croissante, stationnaire ou décroissante ; et c’est ce qu’il faut d’argent pour acheter cette quantité déterminée de choses, qui règle le prix pécuniaire du travail. Si donc le prix pécuniaire du travail se trouve quelquefois élevé, tandis que le prix des denrées a baissé, il serait encore plus élevé si les denrées étaient chères, en supposant la demande du travail toujours la même.

C’est parce que la demande du travail augmente dans les années d’une abondance soudaine et extraordinaire, et parce qu’elle décroît dans les années d’une cherté soudaine et extraordinaire, que le prix pécuniaire du travail s’élève quelquefois dans les unes et baisse dans les autres.

Dans les années d’une abondance soudaine et extraordinaire, il se trouve dans les mains des entrepreneurs de travail des fonds qui peuvent suffire à entretenir et à employer un plus grand nombre de travailleurs qu’il n’en a été employé l’année pré­cé­dente ; et ce nombre extraordinaire n’est pas toujours facile à trouver. Ainsi ces maî­tres, qui voudraient avoir plus d’ouvriers, enchérissent les uns sur les autres pour en avoir ; ce qui permet aux ouvriers de hausser à la fois le prix réel et le prix pécuniaire de leur travail.

Il arrive tout le contraire dans une année de cherté soudaine et extraordinaire. Les fonds destinés à alimenter l’industrie sont alors moindres qu’ils n’étaient l’année précédente. Un grand nombre de gens se trouvent privés d’occupation, et ils enchéris­sent au rabais les uns sur les autres pour s’en procurer ; ce qui baisse à la fois le prix réel et le prix pécuniaire du travail[38]. En 1740, année de disette extraordinaire, un grand nombre d’ouvriers consentaient à travailler pour la seule nourriture. Dans les années d’abondance qui succédèrent, il fut plus difficile de se procurer des domestiques et des ouvriers.

La disette d’une année de cherté, en diminuant la demande du travail, tend à en faire baisser le prix, comme la cherté des vivres tend à le hausser. Au contraire, l’abon­­dance d’une année de bon marché, en augmentant cette demande, tend à élever le prix du travail, comme le bon marché des vivres tend à le faire baisser. Dans les varia­tions ordinaires du prix des vivres, ces deux causes opposées semblent se contre­balancer l’une l’autre ; et c’est là probablement ce qui explique pourquoi les salaires du travail sont partout beaucoup plus fixes et plus constants que le prix des vivres.

L’augmentation qui survient dans les salaires du travail augmente nécessairement le prix de beaucoup de marchandises en haussant cette partie du prix qui se résout en salaires, et elle tend d’autant à diminuer la consommation tant intérieure qu’extérieure de ces marchandises. Cependant, la même cause qui fait hausser les salaires du tra­vail, l’accroissement des capitaux, tend à augmenter sa puissance productive, et à faire produire à une plus petite quantité de travail une plus grande quantité d’ouvrage. Le propriétaire du capital qui alimente un grand nombre d’ouvriers essaye nécessaire­ment, pour son propre intérêt, de combiner entre eux la division et la distribution des tâches de telle façon qu’ils produisent la plus grande quantité possible d’ouvrage. Par le même motif, il s’applique à les fournir des meilleures machines que lui ou eux peu­vent imaginer. Ce qui s’opère parmi les ouvriers d’un atelier particulier, s’opérera pour la même raison parmi ceux de la grande société. Plus leur nombre est grand, plus ils tendent naturellement à se partager en différentes classes et à subdiviser leurs tâches. Il y a un plus grand nombre d’intelligences occupées à inventer les machines les plus propres à exécuter la tâche dont chacun est chargé et, dès lors, il y a d’autant plus de probabilités que l’on viendra à bout de les inventer. Il y a donc une infinité de mar­chandises qui, en conséquence de tous ces perfectionnements de l’industrie, sont obte­nues par un travail tellement inférieur à celui qu’elles coûtaient auparavant, que l’aug­men­tation dans le prix de ce travail se trouve plus que compensée par la diminution dans la quantité du même travail.
 
 
 
↑ Nous avons dû rétablir avec soin, dans plusieurs parties de ce chapitre, le texte littéral d’Adam Smith, qui a été singulièrement altéré par M. le sénateur Garnier. A. B.
↑ Les lois destinées à réprimer les coalitions volontaires des ouvriers, pour élever les salaires ou régler les conditions auxquelles ils vendent leur travail, ont été rapportées par un statut de la cinquième année de George IV, chapitre lxxv. Mac Culloch.
↑ Si ce rapport sur les ouvriers est exact, ils ont depuis lors fait de grands progrès en application et en conduite ; car, loin de recourir à leurs maîtres pour la subsistance aux jours de leurs contestations sur un salaire plus élevé, on les a vus quelquefois accumuler, pendant la prospérité de leur industrie, un fonds commun destiné à les secourir dans leurs nécessités. La loi leur est contraire en réalité ; et cela, malgré la question qu’ils soulèvent, n’appelle en aucune manière l’intervention législative. Les rapports d’ouvrier à maître forment un contrat volontaire : et si la loi intervient pour forcer une des parties à se soumettre à l’autre, la nature même du contrat est changée ; le consentement mutuel, base de toute transaction, est détruit, et un individu se trouve dépouillé au profit d’un autre. D’ailleurs, le législateur n’a aucun intérêt positif à intervenir violemment dans les transactions des particuliers. Les ouvriers s’unissent pour provoquer, aux dépens de leurs maîtres, une hausse dans les salaires. Eh pourquoi pas ? Qu’importe au public, qu’en définitive, le gain revienne aux ouvriers ou aux chefs ? Si la société accorde un bon prix à un objet, il ne manquera pas d’abonder sur le marché ; et il n’est d’aucune importance, en tout ce qui peut la concerner, que ce prix soit divisé dans telle ou telle proportion entre les salaires et le profit. La marchandise acquiert son prix sur le marché et les deux parties intéressées se contestent le partage du butin ; mais, qu’a le public à faire de tout cela ? et pourquoi favoriserait-on une des parties au détriment de l’autre ? La vérité est que les coalitions de chefs et d’ouvriers sont amenées par la rareté du travail ou de l’ouvrage. Ce sont les effets naturels d’une cause plus générale ; et jusqu’à ce que la loi ait atteint cette cause générale, jusqu’à ce qu’elle ait créé un supplément de travail ou d’ouvrage, elle ne servira qu’à l’oppression des particuliers. Buchanan.
↑ Le salaire n’est pas seulement une compensation du travail, calculée à tant par heure d’après sa durée : c’est le revenu du pauvre ; et, en conséquence, il doit suffire non-seulement à son entretien pendant l’activité, mais aussi pendant la rémission du travail ; il doit pourvoir à l’enfance et à la vieillesse comme à l’âge viril, à la maladie comme à la santé, et aux jours de repos nécessaires au maintien des forces ou ordonnés par la loi ou le culte public, comme aux jours de travail. Sismondi.
↑ Auteur de l’Essai sur la nature du commerce.
↑ En général, on croit avoir fait quelque chose pour la prospérité d’une nation, quand on a trouvé le moyen d’employer l’activité îles enfants, et de les associer, dès leur plus bas âge, au travail de leurs pères dans les manufactures. Cependant, il résulte toujours de la lutte entre la classe ouvrière et celle qui paye, que la première donne, en retour du salaire qui lui est alloué, tout ce qu’elle peut donner de travail sans dépérir. Si les enfants ne travaillaient point, il faudrait que leurs pères gagnassent assez pour les entretenir jusqu’à ce que leurs forces fussent développées ; sans cela, les enfants mourraient en bas âge, et le travail cesserait bientôt. Mais depuis que les enfants gagnent une partie de leur vie, le salaire des pères a pu être réduit. Il n’est point résulté de leur activité une augmentation de revenu pour la classe pauvre, mais seulement une augmentation de travail, qui s’échange toujours pour la même somme, ou une diminution dans le prix des journées ; tandis que le prix total du travail national est resté le même. C’est donc sans profit pour la nation que les enfants des pauvres ont été privés du seul bonheur de leur vie, la jouissance de l’âge où les forces de leur corps et de leur esprit se développaient dans la gaieté et la liberté. C’est sans profil pour la richesse ou l’industrie, qu’on les a fait entrer, dés six ou huit ans, dans ces usines à coton, où ils travaillent douze et quatorze heures au milieu d’une atmosphère constamment chargée de poils et de poussière, et où ils périssent successivement de consomption avant d’avoir atteint vingt ans.
On aurait bonté de calculer la somme qui pourrait mériter le sacrifice de tant de victimes humaines ; mais ce crime journalier se commet gratuitement. Sismondi.
↑ L’erreur du docteur Smith consiste à représenter tout accroissement dans le revenu ou le capital d’une société comme proportionnel à l’accroissement de ces fonds. Certes, l’individu qui jouira de cet excédant de capital ou de revenu le considérera comme un fonds additionnel lui permettant de maintenir plus de travail ; mais, à moins qu’une partie n’en soit convertible en une quantité additionnelle de provision», cet excédant ne pourra, pour tout un pays, être un fonds réellement propre à maintenir un nombre supplémentaire de travailleurs : et partout où l’accroissement proviendra, non du produit de la terre, mais de celui du travail, cette conversion ne pourra avoir lieu. Il pourra s’établir dans ce cas une distinction entre le nombre de bras que le capital de la société pourrait employer et celui que son territoire peut soutenir.
Le docteur Smith définit la richesse d’un État, le produit annuel de ton territoire et de son travail. Il est évident que cette définition embrasse les produits manufacturiers comme les produits de la terre.
Maintenant, supposons qu’une nation, pendant un certain nombre d’années, ajoute ce qu’elle a épargné de son revenu annuel à son capital manufacturier seulement, et à l’exclusion du capital réparti à la terre ; il est évident que, suivant la définition précédente, elle croîtra en richesse sans, pour cela, pouvoir entretenir un plus grand nombre de travailleurs et, dès lors, sans accroissement réel dans le fonds de subsistance du travail. Cependant, l’extension du capital manufacturier développerait la demande de travail. Cette demande élèverait naturellement le prix de la main-d’œuvre ; mais si la masse annuelle des subsistances du pays ne croissait pas, cette hausse deviendrait bientôt purement nominale, puisque le prix des subsistances s’élèverait avec elle. Malthus.
↑ Le salaire nécessaire n’est pas le même dans tous les emplois du travail ; au contraire il varie suivant les métiers. Pour découvrir la base ou le taux normal du salaire nécessaire dans tous les métiers, il faut chercher quel serait le prix nécessaire d’un travail simple qui ne demanderait que les facultés naturelles les plus ordinaires ; d’un travail qui ne coulerait que des efforts médiocres, qui ne serait accompagné d’aucun désagrément particulier, d’aucun danger palpable pour la vie du travailleur ; d’un travail enfin qui pourrait se continuer tous les jours de l’année sans interruption. Tel est par exemple le travail d’un journalier de ferme. Un pareil travail étant le plus commun et le moins pénible sous tous les rapports, il serait aussi le moins coûteux : son salaire nécessaire se réduirait à l’entretien le plus indispensable. Cependant, quelque chétif que nous admettions cet entretien, il faut qu’il suffise pour faire subsister les travailleurs. Or, dans ce calcul, il ne s’agit pas seulement des individus dont le travail est actuellement en demande : il faut que la classe des travailleurs soit conservée ; autrement elle finirait par décroître, et par une suite immanquable, le prix du travail monterait bien au delà du taux normal. Ainsi ce taux comprend non-seulement ce qui est requis pour la subsistance du travailleur lui-même, mais encore la nourriture de ses enfants jusqu’à ce qu’ils puissent travailler comme lui. Sur cette base, on suppose que le taux normal doit rapporter au travailleur au moins le double de sa subsistance personnelle, en admettant que le travail de la femme suffise seulement pour sa propre dépense, à cause des soins qu’elle est obligée de donner à son ménage et à ses enfants. À la vérité, la subsistance nécessaire de deux enfants n’augmenterait pas du double de celle de leur père ; mais on calcule qu’une moitié des enfants qui naissent, meurent avant l’âge viril. Ainsi il faut, selon Adam Smith, que les travailleurs tâchent, l’un dans l’autre, d’élever au moins quatre enfants, pour que deux aient égalité de chance de parvenir à cet âge. Or on suppose que la subsistance nécessaire de quatre enfants est à peu près égale à celle d’un homme fait. Cette considération n’est pas la seule qui entre dans l’évaluation du taux normal. Le travailleur qui ne gagne que le strict nécessaire, ne peut pas perdre un seul jour de travail sans manquer de subsistance pour ce jour-là. Or les maladies, les accidents inévitables, lui enlèvent bien des journées, et cette perte doit être compensée par un surplus sur le salaire qu’il gagne les jours où il peut travailler. D’ailleurs sa consommation n’est point la même dans un jour d’été que dans un jour d’hiver, car il faut au travailleur, pour ce dernier, plus de combustible et plus de vêtements. Ce n’est donc pas sa consommation d’une année qu’on doit considérer pour établir le taux normal. Cependant, quelque soin qu’on mette à déterminer rigoureusement ce taux, il est toujours une mesure variable. Nous avons déjà observé ailleurs combien la nature du pays influe sur l’étendue des besoins du travailleur ; ainsi le taux normal varie d’un pays à l’antre et quelquefois même d’un canton à l’autre. Un climat froid fait naître plus de besoins indispensables qu’un climat chaud, et celui-ci en occasionne plus qu’un climat tempéré. Chez nous par exemple (en Russie), les fourrures, le combustible, l’huile et la chandelle que les longues nuits d’hiver font consommer, augmentent le taux normal des salaires, comparativement à la France et à l’Italie, où ces besoins ont beaucoup moins d’étendue. Dans la région chaude du Mexique, un journalier a besoin annuellement, pour subsister avec sa famille, de 72 piastres ; cette dépense est moindre de près de 20 piastres dans la région tempérée de ce pays. Quelquefois l’air vif qu’un peuple respire semble aiguiser son appétit, tandis qu’ailleurs une température douce parait le rendre sobre et frugal. Les peuples du nord de l’Europe consomment en proportion plus d’aliments que ceux du midi. Telle est la base qui sert à évaluer le salaire nécessaire dans tous les différents métiers. C’est en partant du salaire qui est le prix nécessaire du travail le plus commun et le moins pénible, qu’on peut remonter graduellement aux salaires qui sont lu récompense des travaux les plus relevés, les plus difficiles ou les plus désagréables. Comme il y a très-peu de métiers absolument exempts de difficultés, d’inconvénients, il n’y en a aussi que très-peu dans lesquels le salaire nécessaire soit exactement de niveau avec son taux normal ; dans la plupart des métiers, les difficultés et les inconvénients élèvent ce salaire plus ou moins au-dessus du taux normal. Storch.
↑ Ceci a été écrit en 1773, avant le commencement des derniers troubles. (Note de l’auteur.)
↑ Et l’Irlande !
↑ Depuis la publication de la Richesse es nations en 1776, le mouvement de la population s’est accru dans une grande proportion. Mac Culloch.
↑ On ne peut remédier à une rareté générale de travail que par l’accroissement des fonds destines à l’industrie : et tout plan qui n’aura pas cet effet, ne pourra même amener une amélioration dans l’état du travailleur. L’emploi des pauvres dans les Workhouses est le remède fourni par la loi ; mais ce remède ne crée aucun fonds nouveau pour l’entretien du travail : il détourne seulement une partie du capital ancien vers un canal différent. D’ailleurs, si ces établissements n’existaient pas, les matières qui y sont manipulées serviraient à donner, sous la direction prévoyante du manufacturier privé, une vive impulsion à l’industrie. Les effets d’un tel système sont, non pas d’augmenter les fonds industriels, mais de changer leur gestion, de les enlever à ceux qui ont un intérêt immédiat à les administrer fidèlement, pour les placer entre les mains de surveillants moins soigneux qui peuvent les dilapider ou les dévorer entièrement. Ces effets sont clairement indiqués dans un vieux traité de Daniel de Foe, intitulé : Giving alms no charity (Aumônes ne sont pas charité) : traité cité par MM. Malthus et sir F. Morton Eden, où l’on observe que chaque écheveau de laine filé dans le Workhouse retire l’équivalant de ce travail à quelque famille pauvre qui le filait auparavant, et que, pour chaque pièce de lainage fabriquée à Londres, il doit y en avoir une pièce de moins fabriquée à Colchester ou ailleurs.
C’est pourquoi il ajoute que « l’emploi des indigents dans les Workhouses tend seulement à transporter les manufactures de Colchester à Londres et à ravir le pain à la bouche du pauvre de l’Essex pour le mettre à la bouche du pauvre de Middlesex.
Il est évident que de pareils systèmes de bienfaisance générale sont incompatibles avec l’ordre établi des sociétés humaines. Les calamités auxquelles sont exposées les classes laborieuses naissent de causes que la législation ne peut atteindre ; et c’est pourquoi qui homme d’État connaissant les limites de son pouvoir, n’abordera des projets tendant à fournir au travailleur de l’ouvrage ou à régler son salaire*. Cependant les législateurs des siècles précédents, semblent avoir considéré la loi comme un remède efficace pour tous les maux ; et au lieu de livrer le travailleur aux ressources de son industrie et de son talent, ils ont généralement essayé de fixer son sort par des règlements arbitraires.
On avait primitivement décrété en Écosse, pour le soutien du pauvre, une provision légale ; mais l’esprit de la population, qui attache du déshonneur à une pauvreté dépendante, a fait abandonner ce système, et maintenant les pauvres sont secourus par des aumônes volontaires.
En Angleterre, les lois en faveur des pauvres, au lieu de tomber en désuétude, ont été poussées bien plus loin que le plan primitif. Par le 43e acte d’Élisabeth, celui sur lequel repose le présent système, on ordonne aux juges de paix de mettre au travail les enfants pauvres ou ceux qui, pouvant travailler, n’ont point d’ouvrage. Ils sont aussi autorisés à lever tous les impôts qu’ils jugeront nécessaires pour le soulagement du pauvre et à désigner ceux qu’ils croient devoir être les objets de la charité publique. L’opinion de sir F. M. Eden, est que cet acte de la reine Élisabeth n’avait aucun rapport avec le corps actif des travailleurs, mais était seulement destiné au soulagement de ceux qui n’avaient pas d’ouvrage ou qui étaient incapables de travailler. Dans les temps modernes cependant, ce secours a été étendu à toutes les classes de travailleurs** et on en a réglé le montant d’après le haut prix des subsistances ; quoiqu’il soit évident que, ajouter à l’actif du travailleur quand les salaires sont bas ou quand les subsistances sont chères, revienne en réalité à hausser violemment les salaires ou à fixer un maximum aux prix des subsistances. Buchanan.
↑ Si le commerce intérieur et extérieur étaient hautement honorés en Chine, il est évident que par le grand nombre des travailleurs et le bas prix du travail, ce pays pourrait donner à ses manufactures d’exportation un immense développement. Il est également évident que, vu la grande masse de ses subsistances et l’étendue étonnante de son territoire intérieur, la Chine ne pourrait, en retour, par ses importations, augmenter d’une manière sensible son fonds annuel d’approvisionnement. C’est pourquoi elle échangerait le montant de ses produits manufacturés contre des objets de luxe réunis de toutes les parties du monde. À présent, il parait qu’on n’épargne aucun travail dans la production alimentaire. Le pays est surabondamment peuplé par rapport aux bras que son capital peut employer : et c’est pourquoi le travail est si abondant que l’on ne fait aucun effort pour l’abréger. La conséquence de ce système est probablement la production alimentaire la plus considérable que le sol puisse déployer, car il est a observer que généralement, quoique les procédés qui abrègent le travail agricole permettent au fermier d’apporter sur le marché une certaine quantité de grains à plus bas prix, ils tendent plutôt à diminuer qu’à accroître le produit total. On ne pourrait pas, eu Chine, consacrer un capital immense à la fabrication des objets d’exportation, sans enlever à l’agriculture des bras, en nombre suffisant pour altérer cet état de choses et, à un certain point, diminuer le produit du pays. La demande de travailleurs industriels élèverait naturellement le prix du travail; mais comme la quantité de subsistances ne s’accroîtrait pas, le prix des provisions atteindrait une proportion égale et même supérieure, dans le cas où la quantité d’aliments viendrait à diminuer chaque jour. Cependant le pays avancerait sans cesse en fortune : la valeur échangeable du produit annuel de son territoire et de son travail se trouverait annuellement augmentée : d’un autre côté, le fonds effectif destiné au maintien du travail serait stationnaire ou même irait en déclinant, et conséquemment la richesse ascendante de la nation tendrait plutôt à déprimer qu’à rehausser la condition du pauvre. Quant à leur indépendance des besoins de la vie, elle serait la même, ou plutôt elle aurait diminué ; et la plupart se trouveraient avoir échangé les travaux fortifiants de l’agriculture contre les occupations malsaines de l’industrie manufacturière*.
Malthus.
↑ Le grand mal de la condition du travailleur est la pauvreté, développée par la rareté de la nourriture ou du travail; aussi, dans tous les pays, a-t-on décrété des lois sans nombre pour la soulager. Mais il est, dans l’état social, de ces misères que lu loi ne peut soulager : et c’est pourquoi il est utile de connaître la limite de leurs effets, afin de ne pas manquer le bien réellement possible en voulant atteindre ce qui est hors de notre portée.
Par un accroissement de population, qui ne serait pas suivi d’un accroissement de nourriture, la même somme alimentaire devrait être répartie entre un plus grand nombre de consommateurs, qui tous, naturellement, auraient une moindre portion ; les mêmes effets résulteraient d’une diminution dans les subsistances sans diminution proportionnée dans la population. La loi est généralement intervenue dans ces circonstances, soit pour régler les salaires, soit pour extorquer aux riches des sommes destinées à être données aux pauvres. Mais le mal git dans la rareté des subsistances : et il n’y a d’autre remède que dans un surcroit d’approvisionnement. Les lois et les donations en argent sont également impuissantes, car ce n’est pas de l’argent, mais de la nourriture qu’il faut. Quand même le salaire de chaque travailleur serait doublé ou triplé, ce besoin subsisterait, car aucune classe ne peut obtenir de secours au sein de la détresse générale, sans que ce soit aux dépens d’une autre. D’ailleurs, toute société est principalement composée de travailleurs, et comme il n’est aucun ordre d’individus qui puisse les remplacer, c’est par leurs privations seulement, qu’on peut transformer un approvisionnement imparfait en un produit égal à celui des années ordinaires.
Buchanan.
↑ Tel est aussi l’état de l’Irlande, cette hideuse plaie de l’Angleterre. (Consulter sur l’état économique et social de cette malheureuse contrée, le beau livre de M. Gustave de Beaumont : l’Irlande sociale, politique et religieuse, 2 vol. in-8°.) A. B.
↑ Voyez le livre IV, chapitre vii, section troisième, sur la fin.
↑ La population irlandaise ne meurt pas de faim parce que les richesses décroissent en Irlande, car elles y augmentent tous les jours ; mais parce qu’elles y sont distribuées avec iniquité. Un grand seigneur y dévore la substance de dix mille travailleurs. L’esclavage ancien et l’esclavage moderne n’offrent pas d’exemple d’un tel mépris de l’espèce humaine. A. B.
↑ Le travail, ainsi que toutes choses que l’on peut acheter et vendre, et dont la qualité peut augmenter ou diminuer, a un prix naturel et un prix courant. Le prix naturel du travail est celui qui fournit aux ouvriers, en général, les moyens de subsister et de perpétuer leur espèce sans accroissement ni diminution. Les facultés qu’a l’ouvrier pour subvenir à son entretien et à celui de la famille nécessaire pour maintenir le nombre des travailleurs, ne tiennent pas à la quantité d’argent qu’il reçoit pour son salaire, mais à la quantité de subsistances et d’autres objets nécessaires ou utiles, dont l’habitude lui a fait un besoin. Le prix naturel du travail dépend donc du prix des subsistances et de celui des choses nécessaires ou utiles à l’entretien de l’ouvrier et de sa famille. Une hausse dans les prix de ces objets fera hausser le prix naturel du travail, lequel baissera par la baisse des prix.
Plus la société fait de progrès, plus le prix naturel tend à hausser, parce qu’une des principales denrées qui règlent le prix naturel, tend à renchérir en raison de la plus grande difficulté de l’acquérir. Néanmoins, les améliorations dans l’agriculture, la découverte de nouveaux marchés d’où l’on peut tirer des vivres, peuvent, pendant un certain temps, s’opposer à la hausse du prix des denrées, et peuvent même faire baisser leur prix naturel. Les mêmes causes produiront un semblable effet sur le prix naturel du travail.
Le prix naturel de toute denrée, les matières premières et le travail exceptés, tend à baisser, par suite de l’accroissement des richesses et de la population ; car, quoique d’un côté leur valeur réelle augmente par la hausse du prix naturel des matières premières, cela est plus que compensé par les perfectionnements des machines, par une meilleure division et distribution du travail, et par l’habileté toujours croissante des producteurs dans les sciences et les arts.
Le prix courant du travail est le prix réel qu’on le paye, selon la proportion dans laquelle il est offert ou demandé, le travail étant cher quand les bras sont rares, et a bon marché quand ils abondent. Quelque grande que puisse être la différence entre le prix courant et le prix naturel du travail, ces deux prix tendent, ainsi que toutes les denrées, à se rapprocher. C’est lorsque le prix courant du travail s’élève au-dessus de son prix naturel, que le sort de l’ouvrier est réellement prospère et heureux, et qu’il peut se procurer en plus grande quantité tout ce qui est utile ou agréable à la vie, et par conséquent élever et maintenir une famille robuste et nombreuse. Quand, au contraire, le nombre des ouvriers est augmenté par l’accroissement de la population, que la hausse des salaires a encouragé, les salaires baissent de nouveau à leur prix naturel, et quelquefois même, l’effet de la réaction est tel, qu’ils tombent encore plus bas.
Quand le prix courant du travail est au-dessous de son prix naturel, le sort des ouvriers est déplorable, la pauvreté ne leur permettant plus de 6e procurer les objets que l’habitude leur a rendus absolument nécessaires. Ce n’est que lorsqu’à force de privations le nombre des ouvriers se trouve réduit, ou que la demande de bras s’accroit, que le prix courant du travail remonte de nouveau à son prix naturel. L’ouvrier peut alors se procurer encore une fois les jouissances modérées qui faisaient son bonheur. Ricardo.
↑ Chap. x de ce livre, première partie.
↑ L’énorme accroissement des communications entre les différentes provinces de l’empire, depuis les dernières années, par suite de l’amélioration des routes, de la construction des canaux et des chemins de fer, de l’établissement de la navigation à la vapeur, a ramené le prix des différents produits et les salaires du travail, dans les diverses localités, à un niveau beaucoup plus égal qu’à l’époque où fut publiée la Richesse des nations.
Mac Culloch.
↑ Les salaires du travail ne consistent point dans l’argent, mais dans ce qu’on peut acheter avec cet argent, c’est-à-dire dans les denrées et autres objets nécessaires; et la part qui sera accordée au travailleur sur le fonds commun, sera toujours proportionnée a l’offre. Là où les subsistances sont à bas prix et abondantes, son lot sera plus fort ; et là où elles sont rares et chères, sa part le sera moins. Son salaire sera toujours exactement ce qui doit lui revenir, et jamais au delà. Le docteur Smith et la plupart des autres auteurs ont, il est vrai, prétendu que le prix en argent du travail était réglé par le prix en argent des subsistances, et que toutes les fois que les vivres montaient, les salaires haussaient dans la même proportion. Il est cependant clair que le prix du travail n’a point de rapport nécessaire avec le prix des subsistances, puisqu’il dépend entièrement de l’offre du travail industriel comparée avec la demande. D’ailleurs, il faut faire attention que le haut prix des subsistances est un indice certain de la diminution de l’approvisionnement, et a lieu dans le cours ordinaire des choses, afin d’en retarder la consommation. Une moindre quantité de vivres, partagée entre le même nombre de consommateurs, en laissera évidemment à chacun une moindre portion, et le travailleur sera forcé de supporter sa part de privations dans la disette. Pour que ce fardeau soit distribué également, et pour empêcher que le travailleur ne consomme autant de subsistances qu’auparavant, le prix monte. On prétend que les salaires doivent monter en même temps, pour mettre le travailleur en état de se procurer la même quantité d’une denrée devenue plus rare. Mais si cela était ainsi, la nature contrarierait elle-même ses propres desseins, en faisant d’abord monter le prix des subsistances, afin d’en diminuer la consommation, et en faisant ensuite hausser les salaires, pour fournir au travailleur le même approvisionnement qu’auparavant.
Buchanan.
↑ C’est le contraire qui arrive aujourd’hui : le grain n’est pas, généralement parlant, aussi cher en Écosse qu’en Angleterre; et l’excédant des exportations est presque toujours en faveur de l’Écosse. Mac Culloch.
↑ Cela est vrai du dix-huitième siècle (le siècle contemporain de la publication du texte), seulement jusqu’à 1765. De 1765 à 1820, les prix de toutes les espèces de grains eu Angleterre, dépassèrent de beaucoup la moyenne des époques précédentes. Depuis 1820, ils sont retombés à peu près à l’ancien niveau. Mac Culloch.
↑ Ces relevés se nomment fiars en Écosse, et c’est la même chose que les registres des marchés et prix des grains ou mercuriales, tenus dans la plupart des villes et bourgs de France par les officiers de police.
↑ Il ne faut pas oublier que ces remarques du docteur Smith ne s’appliquent qu’à la condition des travailleurs en 1775. Mac Culloch.
↑ Elle est maintenant de 13 deniers par jour, à partir du moment où le soldat est enrôlé ; après sept ans de service, elle est portée à i4 deniers ; à 15, après quatorze ans de service. Mac Culloch.
↑ Voyez son Projet sur les moyens de faire subsister les pauvres, dans l’Histoire de la législation sur les pauvres, par le docteur Burn. Note de l’auteur.
↑ Pour tout ce qui concerne le salaire des ouvriers en France, il est indispensable de consulter le consciencieux ouvrage de M. le docteur Villermé : Tableau de l’état physique et moral des ouvriers, 2vol., 1840. Cet ouvrage, empreint des plus nobles sympathies pour les classes ouvrières, appartient à une école bien éloignée de la dureté systématique de Malthus et de Ricardo. A. B.
↑ Chap. xi de ce livre : Des effets des progrès, etc.
↑ Liv. V, chap. ii : Impôts sur les objets de consommation. — Il est très-important aussi de consulter sur ce sujet le chapitre de Ricardo : de l’Impôt sur les salaires.
↑ L’excellent docteur Smith se croit presque obligé de démontrer que le bien-être des classes laborieuses n’est pas un mal. Un jour viendra où l’art de gouverner consistera surtout dans les moyens d’assurer le bien-être de ces classes. Ab Jove principium. A. B.
↑ Consulter, sur la question des enfants trouvés, les ouvrages de MM. de Gouroff, Monfalcon, et le livre de Malthus, sur la Population. A. B.
↑ Malthus a été plus loin ; il a érigé en principe et proclamé comme une fatalité nécessaire, les tristes conséquences de ces mariages prématurés et féconds. Ses cruelles doctrines ont été adoptées un moment par tous les économistes de l’Europe, et elles dominent encore aujourd’hui toute la législation de l’Angleterre. C’est au système de Malthus que l’Angleterre doit le maintien des lois céréales, l’exagération des taxes indirectes, la révocation des lois sur les pauvres, et la création de ces affreux repaires qui, sous le nom de Work houses, ont pour but de châtier la misère comme un crime, et de punir en secourant. Peu à peu, il s’est établi en Angleterre cette étrange maxime, que toute pauvreté est l’œuvre de celui qui en souffre, et qu’il en faut pourchasser les victimes, au lieu d’en poursuivre les causes. Les législateurs, imbus de cette maxime, font des lois draconiennes contre les pauvres, plutôt que de s’attaquer à la pauvreté ; et nous craignons qu’ils ne préparent à leur pays de bien grandes tempêtes. M. de Sismondi a protesté le premier, dans ses Nouveaux principes d’économie politique, contre l’invasion de cette barbarie économique ; et M. de Villeneuve-Bargemont a très-bien démontré, dans son Économie politique chrétienne, l’impuissance des rigueurs malthusiennes pour arrêter le débordement de la population. Il faut changer de voie ; il faut prendre pour but, sincèrement et courageusement, l’amélioration du sort des classes laborieuses, et non pas leur exclusion du banquet de la vie. Telle est la tendance de l’économie politique de nos jours. A. B.
↑ Et l’Irlande !
↑ C’est ce qui est démontré aujourd’hui par toutes les observations consignées dans les enquêtes sur l’émancipation des esclaves dans les colonies. Partout où l’affranchi travaille, son travail est plus productif que celui de l’esclave ; mais malheureusement il ne travaille pas toujours. A. B.
↑ Voyez liv. III, chap. ii, et liv. IV, chap. ix.
↑ Recherches sur la population de Lyon, etc., imprimées en 1768.
↑ Lorsque les denrées sont chères en même temps que la main-d’œuvre est à bas prix ; lorsque, par conséquent, les ouvriers, forcés par la concurrence, se contentent du nécessaire pour vivre ; lorsqu’ils retranchent sur toutes leurs jouissances et toutes leurs heures de repos ; que leur existence est un combat continuel contre la misère, les prix sont réellement bas, et leur ténuité est une calamité nationale. De tels ouvriers créent bien aussi une portion de richesse échangeable; ils emploient bien le capital national, et ils donnent au fabricant des bénéfices; mais cet accroissement de richesses est acheté trop cher aux dépens de l’humanité. On a reconnu, dès longtemps, que la trop grande division du terrain amenait dans la population agricole un état de misère universelle, dans lequel l’ouvrier, par le plus grand travail, n’obtenait pas un salaire suffisant pour vivre ; et quoique de l’activité à laquelle il était forcé il résultât une augmentation de produit brut, on a reconnu que cette richesse, insuffisante pour ceux qu’elle devait nourrir, était une calamité nationale. La même chose est vraie de la même manière pour les ouvriers des manufactures. La nation s’appauvrit au lieu de s’enrichir, lorsque son revenu augmente comme un, et sa population comme deux. Sismondi.

*. M. Malthus suppose ici, pour l’honneur de l’argument qu’il développe, une hypothèse qui ne peut jamais se réaliser ; c’est-à-dire, une société avançant dans tous les arts qui contribuent au bien-être, à l’élégance, et négligeant l’agriculture, sur laquelle repose sa subsistance. Il suppose que lorsqu’une nation est rapidement entraînée vers l’accumulation de la richesse commerciale, le capital le trouve détourné de l’agriculture : tandis qu’il est positif qu’une partie du capital additionnel acquis par le commerce sera employée à l’amélioration des terres ; qu’un accroissement de produits en sera la conséquence, et qu’ainsi des Tonds additionnels pour le maintien du travail pourront être amassés. D’après cela , il n’est pas facile de comprendre comment, même en Chine, un accroissement dans la richesse commerciale tend à abaisser la condition du pauvre. 

%%%%%%%%%%%%%%%%%%%%%%%%%%%%%%%%%%%%%%%%%%%%%%%%%%%%%%%%%%%%%%%%%%%%%%%%%%%%%%%%
%                                  Chapitre 9                                  %
%%%%%%%%%%%%%%%%%%%%%%%%%%%%%%%%%%%%%%%%%%%%%%%%%%%%%%%%%%%%%%%%%%%%%%%%%%%%%%%%

\chapter{Des profits du capital}
\markboth{Des profits du capital}{}

La hausse et la baisse dans les profits du capital dépendent des mêmes causes que la hausse et la baisse dans les salaires du travail, c’est-à-dire de l’état croissant ou dé­crois­sant de la richesse nationale ; mais ces causes agissent d’une manière très-diffé­rente sur les uns et sur les autres.

L’accroissement des capitaux qui fait hausser les salaires tend à abaisser les pro­fits. Quand les capitaux de beaucoup de riches commerçants sont versés dans un mê­me genre de commerce, leur concurrence mutuelle tend naturellement à en faire bais­ser les profits, et quand les capitaux se sont pareillement grossis dans tous les diffé­rents commerces établis dans la société, la même concurrence doit produire le même effet sur tous.

Nous avons déjà observé qu’il était difficile de déterminer quel est le taux moyen des salaires du travail, dans un lieu et dans un temps déterminés. On ne peut guère, même dans ce cas, déterminer autre chose que le taux le plus habituel des salaires ; mais cette approximation ne peut guère s’obtenir à l’égard des profits des capitaux. Le profit est si variable, que la personne qui dirige un commerce particulier ne pourrait pas toujours vous indiquer le taux moyen de son profit annuel. Ce profit est affecté, non-seulement de chaque variation qui survient dans le prix des marchandises qui sont l’objet de ce commerce, mais encore de la bonne ou mauvaise fortune des con­currents et des pratiques du commerçant, et de mille autres accidents auxquels les marchandises sont exposées, soit dans leur transport par terre ou par mer, soit même quand on les tient en magasin. Il varie donc, non-seulement d’une année à l’autre, mais même d’un jour à l’autre, et presque d’heure en heure. Il serait encore plus difficile de déterminer le profit moyen de tous les différents commerces établis dans un grand royaume et, quant à prétendre juger avec un certain degré de précision de ce qu’il peut avoir été anciennement ou à des époques reculées, c’est ce que nous regardons comme absolument impossible.

Mais quoiqu’il soit peut-être impossible de déterminer avec quelque précision quels sont ou quels ont été les profits moyens des capitaux, soit à présent, soit dans les temps anciens, cependant on peut s’en faire une idée approximative d’après l’inté­rêt de l’argent. On peut établir pour maxime que partout où on pourra faire beaucoup de profits par le moyen de l’argent, on donnera communément beaucoup pour avoir la faculté de s’en servir, et qu’on donnera en général moins, quand il n’y aura que peu de profits à faire par son emploi[2]. Ainsi, suivant que le taux ordinaire de l’intérêt varie dans un pays, nous pouvons compter que les profits ordinaires des capitaux varient en même temps ; qu’ils baissent quand il baisse, et qu’ils montent quand il monte. Les progrès de l’intérêt peuvent donc nous donner une idée du profit du capital.

Par le statut de la trente-septième année du règne de Henri VIII, tout intérêt au-dessus de 10 p. 100 fut déclaré illégitime. Il paraît qu’avant ce statut, on prenait quel­quefois un intérêt plus fort. Sous le règne d’Edouard VI, le zèle religieux proscrivit tout intérêt. On dit cependant que cette prohibition, comme toutes les autres de ce genre, ne produisit aucun effet, et il est probable qu’elle augmenta le fléau de l’usure, plutôt que de le diminuer. Le statut de la treizième année d’Élisabeth, chapitre VIII, fit revivre celui de Henri VIII, et le taux légal de l’intérêt demeura fixé à 10 p. 100, jusqu’à la vingt et unième année du règne de Jacques Ier, où il fut réduit à 8 p. 100. Bientôt après la restauration, il fut réduit à 6 p. 100, et par le statut de la deuxième année de la reine Anne, à 5 p. 100. Tous ces différents règlements paraissent avoir été faits avec beaucoup d’égards aux circonstances. Ils semblent avoir suivi et non précédé le taux de l’intérêt de la place, ou le taux auquel empruntaient habituellement les gens qui avaient bon crédit. Il paraît que, depuis le temps de la reine Anne, 5 p. 100 a été un taux plutôt au-dessus qu’au-dessous de celui de la place. Avant la dernière guerre, le gouvernement empruntait à 3 p. 100 ; et dans la capitale, ainsi qu’en beaucoup d’autres lieux du royaume, les gens qui avaient bon crédit empruntaient à 3 1/2, 4 et 4 1/2 p. 100.

Depuis le règne de Henri VIII, la richesse et le revenu national ont toujours été en croissant et, dans le cours de leurs progrès, leur mouvement paraît avoir été graduel­lement accéléré plutôt que retardé. Ils paraissent non-seulement avoir toujours avancé, mais encore avoir toujours avancé de plus vite en plus vite. Durant la même période, les salaires du travail ont été continuellement en augmentant, et les profits des capi­taux, dans la plus grande partie des différentes branches de commerce et de ma­nu­factures. continuellement en diminuant.

Il faut, en général, un plus grand capital pour faire aller un genre quelconque de commerce dans une grande ville que dans un village. Dans une ville importante, les grands capitaux versés dans chaque branche de commerce, et le nombre des riches concurrents, réduisent généralement le taux du profit au-dessous de ce qu’il est dans un village. Mais les salaires du travail sont, en général, plus hauts dans une grande ville que dans un village. Dans une ville qui s’enrichit, ceux qui ont de gros capitaux à employer ne peuvent souvent trouver autant d’ouvriers qu’ils voudraient ; et pour s’en procurer le plus qu’ils peuvent, ils enchérissent les uns sur les autres, ce qui fait hausser les salaires et baisser les profits. Dans les campagnes éloignées, les capitaux ne suffisent pas d’ordinaire pour occuper tout le monde, en sorte que les ouvriers s’of­frent au rabais pour se procurer de l’emploi, ce qui fait baisser les salaires et hausser les profits.

En Écosse, quoique le taux légal de l’intérêt soit le même qu’en Angleterre, cepen­dant le taux de la place est plus élevé. Les gens les plus accrédités y empruntent rare­ment au-dessous de 5 p. 100 ; les banquiers d’Édimbourg donnent même 4 p. 100 sur leurs bons, payables en tout ou en partie à la volonté du porteur. Les banquiers, à Londres, ne donnent pas d’intérêt pour l’argent déposé chez eux. Il y a peu d’industries qui ne puissent s’exercer en Écosse avec de plus faibles capitaux qu’en Angleterre ; le taux commun du profit y doit donc être un peu plus élevé. On a déjà observé que les salaires étaient plus bas en Écosse qu’en Angleterre ; aussi le pays est-il non-seule­ment beaucoup plus pauvre, mais encore ses progrès vers un état meilleur, car il est clair qu’il en fait, semblent-ils être bien plus lents et bien plus tardifs[3].

En France, le taux légal de l’intérêt, pendant le cours de ce siècle, n’a pas toujours été réglé sur le taux de la place. En 1720, l’intérêt fut réduit du denier 20 au denier 50, ou de 5 à 2 p. 100. En 1724, il fut porté au denier 30, ou à 3 1/2 p. 100. En 1725, il fut remis au denier 20, ou à 5 p. 100. En 1766, sous l’administration de M. Laverdy, il fut réduit au denier 25, ou à 6 p. 100. L’abbé Terray le porta ensuite à l’ancien taux de 5 p. 100. On suppose que l’objet de la plupart de ces réductions forcées de l’intérêt était d’amener la réduction de l’intérêt des dettes publiques, et ce projet a été quel­quefois mis à exécution. La France est peut-être pour le moment un pays moins riche que l’Angleterre ; et quoique le taux légal de l’intérêt ait souvent été plus bas en France qu’en Angleterre, le taux de la place a été généralement plus élevé ; car là, comme ailleurs, on a beaucoup de moyens faciles et sûrs d’éluder la loi. Des commerçants anglais, qui ont fait le commerce dans les deux pays, m’ont assuré que les profits du négoce étaient plus élevés en France qu’en Angleterre ; et c’est là, sans aucun doute, le motif pour lequel beaucoup de sujets anglais emploient de préférence leurs capitaux dans un pays où le commerce est peu considéré, plutôt que de les employer dans leur propre pays où il est en grande estime. Les salaires du travail sont plus bas en France qu’en Angleterre. Quand on passe d’Écosse en Angleterre, la différence que l’on re­mar­que dans l’extérieur et la tenue des gens du peuple des deux pays indique suffi­samment la différence de leur condition. Le contraste est encore plus frappant quand on revient de France. La France, quoique indubitablement plus riche que l’Écos­se, ne paraît pas avancer d’un pas aussi rapide. C’est une opinion générale, et même vulgaire dans chacun de ces pays, que l’opulence y va en déclinant ; opinion mal fondée, à ce que je crois, même à l’égard de la France. Quant à l’Écosse, quiconque l’aura vue il y a vingt ou trente ans et l’observera aujourd’hui, ne supposera jamais assurément qu’elle aille en déclinant.

D’un autre côté, la Hollande est plus riche que l’Angleterre proportionnellement à sa population et à l’étendue de son territoire. Le gouvernement y emprunte à 2 p. 100, et les particuliers qui ont bon crédit, à 3. On dit que les salaires y sont plus élevés qu’en Angleterre, et l’on sait généralement que les Hollandais sont, de tous les peuples de l’Europe, celui qui se contente des moindres bénéfices. Quelques personnes ont prétendu que le commerce déclinait en Hollande, et cela est peut-être vrai de quelques branches particulières. Mais ces symptômes semblent indiquer assez que la décadence n’y est pas générale. Quand les profits baissent, les commerçants sont très-disposés à se plaindre de la décadence du commerce, quoique cependant la diminution des pro­fits soit l’effet naturel de sa prospérité ou d’une plus grande masse de fonds qui y est versée[4]. Pendant la dernière guerre, les Hollandais ont gagné tout le commerce de trans­port de la France, dont ils conservent encore la plus grande partie. Les fortes sommes dont ils sont propriétaires dans les fonds publics de France et d’Angleterre, qu’on évalue pour ces derniers à environ 40,000,000 l. sterl. (en quoi je soupçonne pourtant beaucoup d’exagération[5]), la quantité de fonds qu’ils prêtent à des particuliers, dans les pays où le taux de l’intérêt est plus élevé que chez eux, sont des circonstances qui, sans aucun doute, démontrent la surabondance de leurs capitaux, ou bien leur accroissement au-delà de ce qu’ils peuvent employer avec un profit convenable dans les affaires de leur pays ; mais cela ne prouve nullement que ces affaires aillent en diminuant. Ne peut-il pas en être des capitaux d’une grande nation comme de ceux d’un particulier, lesquels, bien qu’ils aient été acquis par le moyen de son commerce, s’augmentent souvent au-delà de ce qu’il peut y employer, tandis qu’en même temps son commerce n’en va pas moins toujours en augmentant ?

Dans nos colonies de l’Amérique septentrionale et des Indes occidentales, non-seulement les salaires du travail, mais encore l’intérêt de l’argent et, par conséquent, les profits du capital, sont plus élevés qu’en Angleterre. Dans ces différentes colonies, le taux légal de l’intérêt, ainsi que le taux de la place, s’élève de 6 à 8 p. 100. Cepen­dant de forts salaires et de gros profits sont naturellement des choses qui vont rare­ment ensemble, si ce n’est dans le cas particulier d’une colonie nouvelle. Dans une co­lo­nie nouvelle, à la différence de tout autre pays, les capitaux sont naturellement peu abondants en proportion de l’étendue de son territoire, et peu nombreux en proportion de sa population et de l’étendue de son capital. Les colons ont plus de terre qu’ils n’ont de capitaux à consacrer à la culture ; aussi les capitaux qu’ils possèdent sont-ils appli­qués seulement à la culture des terres les plus fertiles et les plus favorablement si­tuées, à celles qui avoisinent les côtes de la mer ou qui bordent les rivières naviga­bles. Ces terres s’achètent très-souvent au-dessous même de la valeur de leur produit naturel. Le capital employé à l’achat et à l’amélioration de ces terres doit rendre un très-gros profit et, par conséquent, fournir de quoi payer un très-gros intérêt. L’accu­mu­lation rapide du capital dans un emploi aussi profitable met le planteur dans le cas d’augmenter le nombre des bras qu’il occupe, beaucoup plus vite qu’un établissement récent ne lui permet d’en trouver ; aussi les travailleurs qu’il peut se procurer sont-ils très-libéralement payés. À mesure que la colonie augmente, les profits des capitaux dimi­nuent. Quand les terres les plus fertiles et les mieux situées se trouvent toutes occupées, la culture de celles qui sont inférieures, tant pour le sol que pour la situa­tion, devient de moins en moins profitable et, par conséquent, l’intérêt du capital employé se trouve nécessairement réduit. C’est pour cela que le taux de l’intérêt, soit légal, soit courant, a considérablement baissé dans la plupart de nos colonies, pendant le cours de ce siècle. À mesure de l’augmentation des richesses de l’industrie et de la population, l’intérêt a diminué.

Les salaires du travail ne baissent pas comme les profits des capitaux. La de­man­de de travail augmente avec l’accroissement du capital, quels que soient les profits ; et après que ces profits ont baissé, les capitaux n’en augmentent pas moins ; ils conti­nuent même à augmenter bien plus vite qu’auparavant. Il en est des nations indus­tri­euses qui sont en train de s’enrichir, comme des individus industrieux. Un gros capi­tal, quoique avec de petits profits, augmente en général plus promptement qu’un petit capital avec de gros profits. L’argent fait l’argent, dit le proverbe. Quand vous avez gagné un peu, il vous devient souvent facile de gagner davantage. Le difficile est de gagner ce peu.

J’ai déjà exposé en partie la liaison qu’il y a entre l’accroissement du capital et celui de l’industrie ou de la demande de travail productif ; mais je la développerai avec plus d’étendue par la suite, en traitant de l’accumulation des capitaux[6].

L’acquisition d’un nouveau territoire ou de quelques nouvelles branches d’indus­trie peut quelquefois élever les profits des capitaux et, avec eux, l’intérêt de l’argent, même dans un pays qui fait des progrès rapides vers l’opulence. Les capitaux du pays ne suffisant pas à la quantité des affaires que ces nouvelles acquisitions offrent aux possesseurs de ces capitaux, on les applique alors seulement aux branches particu­liè­res qui donnent le plus gros profit. Une partie de ceux qui étaient auparavant em­ployés dans d’autres industries en est nécessairement retirée, pour être versée dans les entreprises nouvelles qui sont plus profitables ; la concurrence devient donc moins active qu’auparavant, dans toutes les anciennes branches d’industrie. Le marché se trouve moins abondamment fourni de plusieurs différentes sortes de marchandises. Le prix de celles-ci hausse nécessairement plus ou moins, et rend un plus gros profit à ceux qui en trafiquent, ce qui les met dans le cas de payer un intérêt plus fort des prêts qu’on leur fait. Pendant quelque temps, après la fin de la dernière guerre, non-seule­ment des particuliers du meilleur crédit, mais même quelques-unes des premières compagnies de Londres, qui auparavant ne payaient pas habituellement plus de 4 et 4 et 1/2 p. 100, empruntèrent communément alors à 5. Cela s’explique suffisamment par la grande augmentation de territoire et de commerce, qui fut la conséquence de nos acquisitions dans l’Amérique septentrionale et les Indes occidentales, sans qu’il soit besoin de supposer aucune diminution dans la masse des capitaux de la société. La masse des anciens capitaux étant attirée dans une foule de nouvelles entreprises, il en est résulté nécessairement une diminution dans la quantité employée auparavant dans les autres industries, où la diminution de la concurrence fit nécessairement baisser les profits. J’aurai lieu, par la suite, d’exposer les raisons qui me portent à croi­re que la masse des capitaux de la Grande-Bretagne n’a pas souffert de diminu­tion, même par les dépenses énormes de la dernière guerre[7].

Toutefois, une diminution survenue dans la masse des capitaux d’une société, ou dans le fonds destiné à alimenter l’industrie, en amenant la baisse des salaires, amène pareillement une hausse dans les profits et, par conséquent, dans le taux de l’intérêt. Les salaires du travail étant baissés, les propriétaires de ce qui reste de capitaux dans la société peuvent établir leurs marchandises à meilleur compte qu’auparavant ; et comme il y a moins de capitaux employés à fournir le marché qu’il n’y en avait aupa­ra­vant, ils peuvent vendent plus cher. Leurs marchandises leur coûtent moins et se vendent plus cher. Leurs profits, croissant ainsi en raison double, peuvent suffire à payer un plus gros intérêt. Les grandes fortunes faites si subitement et si aisément, au Bengale et dans les autres établissements anglais des Indes orientales, nous témoi­gnent assez que les salaires sont très-bas et les profits très-élevés dans ces pays ruinés. L’intérêt de l’argent suit la même proportion. Au Bengale, on prête fréquemment aux fermiers à raison de 40, 50 et 60 p. 100, et la récolte suivante répond du payement. De même que les profits capables de payer un pareil intérêt doivent réduire presque à rien la rente du propriétaire, de même une usure aussi énorme doit à son tour emporter la majeure partie de ces profits. Dans les temps qui précédèrent la chute de la république romaine, il paraît qu’une usure de la même espèce régnait dans les provinces, sous l’ad­mi­nistration ruineuse de leurs proconsuls. Nous voyons, dans les lettres de Cicé­ron, que le vertueux Brutus prêtait son argent, en Chypre, à quarante-huit pour cent.

Dans un pays qui aurait atteint le dernier degré de richesse auquel la nature de son sol et de son climat et sa situation à l’égard des autres pays peuvent lui permettre d’at­teindre, qui, par conséquent, ne pourrait plus ni avancer ni reculer ; dans un tel pays, les salaires du travail et les profits des capitaux seraient probablement très-bas tous les deux. Dans un pays largement peuplé en proportion du nombre d’hommes que peut nourrir son territoire ou que peut employer son capital, la concurrence, pour obtenir de l’occupation, serait nécessairement telle, que les salaires y seraient réduits à ce qui est purement suffisant pour entretenir le même nombre d’ouvriers ; et comme le pays serait déjà pleinement peuplé, ce nombre ne pourrait jamais augmenter. Dans un pays richement pourvu de capitaux, en proportion des affaires qu’il peut offrir en tout genre, il y aurait, dans chaque branche particulière de l’industrie, une aussi grande quan­tité de capital employé, que la nature et l’étendue de ce commerce pourraient le permettre ; la concurrence y serait donc partout aussi grande que possible et, consé­quemment, les profits ordinaires aussi bas que possible.

Mais peut-être aucun pays n’est encore parvenu à ce degré d’ opulence. La Chine paraît avoir été longtemps stationnaire, et il y a probablement longtemps qu’elle est arrivée au comble de la mesure de richesse qui est compatible avec la nature de ses lois et de ses institutions ; mais cette mesure peut être fort inférieure à celle dont la nature de son sol, de son climat et de sa situation serait susceptible avec d’autres lois et d’autres institutions. Un pays qui néglige ou qui méprise tout commerce étranger, et qui n’admet les vaisseaux des autres nations que dans un ou deux de ses ports seulement, ne peut pas faire la même quantité d’affaires qu’il ferait avec d’autres lois et d’autres institutions. Dans un pays d’ailleurs où, quoique les riches et les posses­seurs de gros capitaux jouissent d’une assez grande sûreté, il n’y en existe presque aucune pour les pauvres et pour les possesseurs de petits capitaux, où ces derniers sont, au contraire, exposés en tout temps au pillage et aux vexations des mandarins infé­rieurs, il est impossible que la quantité du capital engagée dans les différentes bran­ches d’industrie soit jamais égale à ce que pourraient comporter la nature et l’étendue de ces affaires. Dans chacune des différentes branches d’industrie, l’oppres­sion qui frappe les pauvres établit nécessairement le monopole des riches, qui, en se rendant les maîtres de tout le commerce, se mettent à même de faire de très-gros profits ; aussi dit-on que le taux ordinaire de l’intérêt de l’argent à la Chine est de 12 pour 100, et il faut que les profits ordinaires des capitaux soient assez forts pour solder cet intérêt exorbitant.

Un vice dans la loi peut quelquefois faire monter le taux de l’intérêt fort au-dessus de ce que comporterait la condition du pays, quant à sa richesse ou à sa pauvreté. Lorsque la loi ne protège pas l’exécution des contrats, elle met alors tous les emprun­teurs dans une condition équivalente à celle de banqueroutiers ou d’individus sans crédit, dans les pays mieux administrés. Le prêteur, dans l’incertitude où il est de re­cou­vrer son argent, exige cet intérêt énorme qu’on exige ordinairement des banque­routiers. Chez les peuples barbares qui envahirent les provinces occidentales de l’em­pire romain, l’exécution des contrats fut, pendant plusieurs siècles, abandonnée à la bonne foi des contractants. Il était rare que les cours de justice de leurs rois en pris­sent connaissance. Il faut peut-être attribuer en partie à cette cause le haut intérêt qui régna dans les anciens temps.

Lorsque la loi défend toute espèce d’intérêt, elle ne l’empêche pas. Il y a toujours beaucoup de gens dans la nécessité d’emprunter, et personne ne consentira à leur prê­ter sans retirer de son argent un intérêt proportionné, non-seulement au service que cet argent peut rendre, mais encore aux risques auxquels on s’expose en éludant la loi. M. de Montesquieu attribue le haut intérêt de l’argent chez tous les peuples mahomé­tans, non pas à leur pauvreté, mais en partie au danger de la contravention, et en partie à la difficulté de recouvrer la dette.

Le taux le plus bas des profits ordinaires des capitaux doit toujours dépasser un peu ce qu’il faut pour compenser les pertes accidentelles auxquelles est exposé chaque emploi de capital. Ce surplus constitue seulement, à vrai dire, le profit ou le bénéfice net. Ce qu’on nomme profit brut comprend souvent, non-seulement ce surplus, mais encore ce qu’on retient pour la compensation de ces pertes extraordinaires. L’intérêt que l’emprunteur peut payer est en proportion du bénéfice net seulement.

Il faut encore que le taux le plus bas de l’intérêt ordinaire dépasse aussi de quelque chose ce qui est nécessaire pour compenser les pertes accidentelles qui résultent du prêt, même quand il est fait sans imprudence. Sans ce surplus, il n’y aurait que l’amitié ou la charité qui pourraient engager à prêter.

Dans un pays qui serait parvenu au comble de la richesse, où il y aurait dans cha­que branche particulière d’industrie la plus grande quantité de capital qu’elle puisse absorber, le taux ordinaire du profit net serait très-peu élevé ; par conséquent, le taux de l’intérêt ordinaire que ce profit pourrait payer serait trop bas pour qu’il fût possible, excepté aux personnes riches, extrêmement riches, de vivre de l’intérêt de leur argent. Tous les gens de fortune bornée ou médiocre seraient obligés de diriger eux-mêmes l’emploi de leurs capitaux. Il faudrait absolument que tout homme fût occupé dans les affaires ou intéressé dans quelque genre d’industrie. Tel est, à peu près, à ce qu’il paraît, l’état de la Hollande. Là, le bon ton ne défend pas à un homme de pratiquer les affaires. La nécessité en a fait presque à tout le monde une habitude, et partout c’est la coutume générale qui règle le bon ton. S’il est ridicule de ne pas s’habiller comme les autres, il ne l’est pas moins de ne pas faire la chose que tout le monde fait. De même qu’un homme d’une profession civile paraît fort déplacé dans un camp ou dans une garnison, et court même risque d’y être peu respecté, il en est de même d’un homme désœuvré au milieu d’une société de gens livrés aux affaires.

Le taux le plus élevé auquel puissent monter les profits ordinaires est celui qui, dans le prix de la grande partie des marchandises, absorbe la totalité de ce qui devait revenir à la rente de la terre, et qui réserve seulement ce qui est nécessaire pour sala­rier le travail de préparer la marchandise et de la conduire au marché, au taux le plus bas auquel le travail puisse jamais être payé, c’est-à-dire la simple subsistance de l’ouvrier. Il faut toujours que, d’une manière ou d’une autre, l’ouvrier ait été nourri pendant le temps que le travail lui a pris ; mais il peut très-bien se faire que le proprié­taire de la terre n’ait pas eu de rente. Les profits du commerce que pratiquent au Bengale les employés de la compagnie des Indes orientales ne sont peut-être pas très-éloignés de ce taux excessif[8].

La proportion que le taux ordinaire de l’intérêt, au cours de la place, doit garder avec le taux ordinaire du profit net, varie nécessairement, selon que le profit hausse ou baisse. Dans la Grande-Bretagne, on porte au double de l’intérêt ce que les com­mer­çants appellent un profit honnête, modéré, raisonnable ; toutes expressions qui, à mon avis, ne signifient autre chose qu’un profit commun et d’usage. Dans un pays où le taux ordinaire du profit net est de 8 ou 10 p. 100, il peut être raisonnable qu’une moitié de ce profit aille à l’intérêt, toutes les fois que l’affaire se fait avec de l’argent d’emprunt. Le capital est au risque de l’emprunteur, qui, pour ainsi dire, est l’assureur de celui qui prête ; et dans la plupart des genres de commerce, 4 ou 5 p. 100 peuvent être à la fois un profit suffisant pour le risque de cette assurance, et une récompense suffisante pour la peine d’employer le capital. Mais dans le pays où le taux ordinaire des profits est beaucoup plus bas ou beaucoup plus élevé, la proportion entre l’intérêt et le profit net ne saurait être la même ; s’il est beaucoup plus bas, peut-être ne pourrait-on pas en retrancher une moitié pour l’intérêt ; s’il est plus élevé, il faudra peut-être aller au-delà de la moitié.

Dans les pays qui vont en s’enrichissant avec rapidité, le faible taux des profits peut compenser le haut prix des salaires du travail dans le prix de beaucoup de den­rées, et mettre ces pays à portée de vendre à aussi bon marché que leurs voisins, qui s’enrichiront moins vite, et chez lesquels les salaires seront plus bas.

Dans le fait, des profits élevés tendent, beaucoup plus que des salaires élevés, à faire monter le prix de l’ouvrage. Si, par exemple, dans la fabrique des toiles, les salaires des divers ouvriers, tels que les séranceurs du lin, les fileuses, les tisserands, etc., venaient tous à hausser de deux deniers par journée, il deviendrait nécessaire d’élever le prix d’une pièce de toile, seulement d’autant de fois deux deniers qu’il y au­rait eu d’ouvriers employés à la confectionner, en multipliant le nombre des ouvriers par le nombre des journées pendant lesquelles ils auraient été ainsi employés. Dans chacun des différents degrés de main-d’œuvre que subirait la marchandise, cette partie de son prix, qui se résout en salaires, hausserait seulement dans la proportion arithmé­tique de cette hausse des salaires. Mais si les profits de tous les différents maîtres qui emploient ces ouvriers venaient à monter de 5 p. 100, cette partie du prix de la mar­chandise qui se résout en profits, s’élèverait dans chacun des différents degrés de la main-d’œuvre, en raison progressive de cette hausse du taux des profits ou en pro­por­tion géométrique. Le maître des séranceurs demanderait, en vendant son lin, un sur­croît de 5 pour 100 sur la valeur totale de la matière et des salaires par lui avancés à ses ouvriers. Le maître des fileuses demanderait un profit additionnel de 5 pour 100, tant sur le prix du lin sérancé dont il aurait fait l’avance, que sur le montant du salaire des fileuses. Et enfin, le maître des tisserands demanderait aussi 5 pour 100, tant sur le prix par lui avancé du fil de lin, que sur les salaires de ses tisserands. La hausse des salaires opère sur le prix d’une marchandise, comme l’intérêt simple dans l’accumulation d’une dette. La hausse des profits opère comme l’intérêt composé. Nos marchands et nos maîtres manufacturiers se plaignent beaucoup des mauvais effets des hauts salaires, en ce que l’élévation des salaires renchérit leurs marchan­dises, et par là en diminue le débit, tant à l’intérieur qu’à l’étranger ; ils ne parlent pas des mauvais effets des hauts profits ; ils gardent le silence sur les conséquences fâcheuses de leurs propres gains ; ils ne se plaignent que de celles du gain des autres[9].
 
 
 
↑ Smith s’est jeté dans un grand embarras, faute d’avoir sépare en deux parties ce qu’il appelle profits du fonds. Il y a dans cette valeur deux éléments qu’il a distingués ailleurs, sans maintenir cette distinction dans le reste de son ouvrage. Ces deux éléments sont le profit de l’industrie ou, si l’on veut, le salaire du travail et l’intérêt du capital. Pourquoi vouloir établir la valeur de l’un d’après la valeur de l’autre ? Leur valeur se règle d’après des principes différents. Celle du profit de l’industrie se règle sur le degré d’habileté, la longueur des études, etc. ; celle de l’intérêt du capital se règle sur l’abondance des capitaux , la sûreté du placement, etc.
Ce qu’il y a de singulier, c’est que Smith lui-même, en traitant son sujet, a fini par s’apercevoir qu’il avait eu tort, ainsi qu’on peut le voir plus loin dans un passage où il dit : « La différence apparente dans les profits des capitaux, suivant les diverses professions, est en général une erreur provenant de ce que nous ne distinguons pas toujours ce qui doit être regardé comme salaires du travail de ce qui doit passer pour profits des capitaux.
Note inédite de J.-B. Say.
↑ Voyez liv. II, chap. iv.
↑ Depuis la guerre d’Amérique, le progrès a été plus rapide en Écosse qu’en Angleterre, et peut-être que dans aucun autre pays. Voyez l’article Agriculture et le chapitre qui traite de l’amélioration qu’ont éprouvée le régime alimentaire, les vêtements, etc., dans la Statistique de la Grande-Bretagne, par Mac. Culloch. A. B.
↑ Le taux peu élevé des profits en Hollande doit être attribué entièrement, ou presque entièrement, à l’élévation oppressive des impôts. (Voyez Principles of Political economy, 2e édit., p. 494.) A. B.
↑ Le docteur Smilh semble ignorer ici qu’un relevé officiel, fait en 1762, porte les différents fonds transférables à la banque d’Angleterre, inscrits sous des noms d’étrangers ou de leurs agents, seulement à 14,956,395 livres sterling. Ce relevé ne comprend pas les fonds étrangers engagés dans le capital de la Compagnie de la mer du Sud et dans les annuités, ni ceux engagés dans le capital de la Compagnie des Indes orientales ; mais en évaluant à 18,000,000 livres sterling la totalité des capitaux étrangers engagés dans les fonds anglais, on sera assurément au-dessus de la vérité.
En 1806, la part des étrangers dans les fonds anglais s’élevait à 18,598,606 livres sterling, auxquels on pouvait ajouter 2,000,000 de surplus pour avoir l’évaluation la plus forte. Depuis la paix, le montant du capital possédé par des étrangers diminue rapidement. En août 1818, il ne s’élevait plus qu’à 12,486,000 livres sterling ; et en ce moment (1838) on ne suppose pas qu’il excède 8,000,000 livres sterling. (Voyez Fairman, on the Funds, 7e édit., p. 229.) Mac Culloch.
↑ Liv. II, chap. iii.
↑ Liv. II, chap. iii ; liv. IV, chap. i ; liv. V, chap. iii.
↑ Sur la nature oppressive de ce commerce, voyez liv. IV, chap. vii, sect. 3.
↑ Qu’on ne suppose pas qu’en affirmant que l’accumulation du capital entre les mains des personnes qui ne les créent ni ne les emploient arrête la marche de la Société, j’aie méconnu ce principe, que si les fonds des capitalistes ne donnaient aucun profit, il n’existerait plus de but à l’épargne, de stimulant pour l’industrie, et d’accroissement dans la richesse nationale. Loin de méconnaître cette loi, c’est précisément en raison de l’importance que j’y attache que je n’ai pu statuer avec promptitude et dogmatiquement à ce sujet. Cependant il est évident que le principe qui affirme que l’intérêt du capital est nécessaire pour stimuler l’économie et l’industrie, est de tous points incompatible avec celui qui proportionne l’énergie et l’habileté du travail au taux de la récompense, puisque l’intérêt doit être prélevé sur le produit du travailleur.
Je puis bien comprendre comment le droit de s’approprier, sous le nom d’intérêt ou de profit, le produit d’autres individus, devient un aliment à la cupidité ; mais je ne puis imaginer qu’en diminuant la récompense du travailleur pour ajouter à l’opulence de l’homme oisif, on puisse accroître l’industrie ou accélérer les progrès de la société en richesse. L’intérêt sur le capital était salutaire alors qu’il tendait à réduire la puissance des seigneurs féodaux, ces maîtres absolus de tous les travailleurs esclaves d’un pays ; mais c’est une grave erreur d’assigner comme cause générale un fait propre seulement à transformer ou à altérer une usurpation particulière.
En réfléchissant sur le principe de la population relativement à nos affections, et observant ce qui a lieu dans les déserts de l’Amérique, nous sommes conduits, je crois, à une solution différente de celles données généralement sur cette question : « S’il se peut ou non que la société se développe là où le capital n’appelle pas l’intérêt. » La première réflexion nous démontrera que le produit de tout travailleur est impérieusement réclamé par la nourriture de sa propre famille. Élever ses enfants, pourvoir à leurs besoins est, en général, pour le travailleur, un motif suffisant de travailler. Comme ceux-ci ont été élevés et ont appris un art manuel, ils deviennent travailleurs, étendent la division du travail, provoquent l’accroissement des lumières, et développent, à leur tour, la population et le produit annuel d’une société. Dans l’état actuel, les épargnes du capitaliste sont aussi bien consommées que toute autre partie du produit annuel, et cela, généralement par des travailleurs ; mais passant premièrement par les mains du capitaliste, celui-ci en prélève pour lui-même une large part qui eût été répartie aux travailleurs, et leur eût permis d’entretenir de plus grandes familles, tout en étendant la division du travail ; ce que les capitaux croissants n’opèrent point. Les motifs qui portent à épargner, dit un rédacteur du Westminster Review, existent entièrement en dehors de l’accroissement des épargnes elles-mêmes. L’affection paternelle est, je crois, une source féconde d’industrie et d’économies qui permet à l’homme d’élever une famille en partageant avec elle le produit de son travail ; et là où de nombreuses familles se forment, la nation croit en richesse et en population.
Dans le fait, c’est une misérable illusion que d’appeler capital un objet économisé. La plus grande partie n’en est pas destinée à la consommation et échappe à jamais aux désirs de bien-être. Lorsqu’un sauvage a besoin de nourriture, il ramasse ce que la nature lui offre spontanément. Plus lard, il découvre qu’un arc ou une fronde lui offre la possibilité de tuer des animaux sauvages à distance ; et, aussitôt, il se décide à les fabriquer, subsistant comme il le peut, pendant que ce travail s’achève. Il n’épargne rien ; car, quoique par sa nature l’instrument soit plus durable que la chair de daim, il ne fut jamais destiné à la consommation. Cet exemple représente ce qui a lieu dans tous les rangs de la société, si ce n’est que les divers travaux sont faits par différentes personnes. L’un fabrique l’arc ou la charrue, tandis que l’autre tue les animaux ou cultive la terre, afin de pourvoir aux besoins de ceux qui font les instruments et les machines. Un excédant de travail peut seul, hormis des cas particuliers et des circonstances passagères, permettre d’amasser ou d’épargner des marchandises dont l’utilité diminue généralement, en raison directe de leur monopole ou accaparement. Les économies du capitaliste, comme on les appelle, sont consommées par le travailleur, et l’on n’entend point parler d’une accumulation actuelle dans les marchandises. Hodgskins.

%%%%%%%%%%%%%%%%%%%%%%%%%%%%%%%%%%%%%%%%%%%%%%%%%%%%%%%%%%%%%%%%%%%%%%%%%%%%%%%%
%                                  Chapitre 10                                 %
%%%%%%%%%%%%%%%%%%%%%%%%%%%%%%%%%%%%%%%%%%%%%%%%%%%%%%%%%%%%%%%%%%%%%%%%%%%%%%%%

\chapter{De la rente de la terre et du fermage}
\markboth{De la rente de la terre et du fermage}{}

Chacun des divers emplois du travail et du capital, dans un même canton, doit nécessairement offrir une balance d’avantages et de désavantages qui établisse ou qui tende continuellement à établir une parfaite égalité entre tous ces emplois. Si, dans un même canton, il y avait quelque emploi qui fût évidemment plus ou moins avanta­geux que tous les autres, tant de gens viendraient à s’y jeter dans un cas, ou à l’aban­don­ner dans l’autre, que ses avantages se remettraient bien vite de niveau avec ceux des autres emplois. Au moins en serait-il ainsi dans une société où les choses sui­vraient leur cours naturel, où l’on jouirait d’une parfaite liberté, et où chaque individu serait entièrement le maître de choisir l’occupation qui lui conviendrait le mieux et d’en changer aussi souvent qu’il le jugerait à propos. L’intérêt individuel porterait chacun à rechercher les emplois avantageux et à négliger ceux qui seraient désa­vantageux.

À la vérité, les salaires et les profits pécuniaires sont, dans tous les pays de l’Europe, extrêmement différents, suivant les divers emplois du travail et des capi­taux. Mais cette différence vient en partie de certaines circonstances attachées aux emplois mêmes, lesquelles, soit en réalité, soit du moins aux yeux de l’imagi­nation, suppléent, dans quelques-uns de ces emplois, à la modicité du gain pécuniaire, ou en contrebalancent la supériorité dans d’autres ; elle résulte aussi en partie de la police de l’Europe, qui nulle part ne laisse les choses en pleine liberté.

Pour examiner particulièrement et ces circonstances, et cette police, je diviserai ce chapitre en deux sections.

\section{Des inégalités qui procèdent de la nature même des emplois}

Autant qu’il m’a été possible de l’observer, les circonstances principales qui suppléent à la modicité du gain pécuniaire dans quelques emplois, et contre-balancent sa supériorité dans d’autres, sont les cinq suivantes : 1° l’agrément ou le désagrément des emplois en eux-mêmes ; 2° la facilité ou le bon marché avec lequel on peut les apprendre, ou la difficulté et la dépense qu’ils exigent pour cela ; 3° l’occupation cons­tante qu’ils procurent, ou les interruptions auxquelles ils sont exposés ; 4° le plus ou moins de confiance dont il faut que soient investis ceux qui les exercent ; 5° la probabilité ou improbabilité d’y réussir.

Premièrement, les salaires du travail varient suivant que l’emploi est aisé ou péni­ble, propre ou malpropre, honorable ou méprisé. Ainsi, dans la plupart des endroits, à prendre l’année en somme, un garçon tailleur gagne moins qu’un tisserand ; son ouvrage est plus facile[1]. Le tisserand gagne moins qu’un forgeron ; l’ouvrage du premier n’est pas toujours plus facile, mais il est beau­coup plus propre ; le forgeron, quoiqu’il soit un artisan, gagne rarement autant, en dou­­ze heures de temps qu’un charbonnier travaillant aux mines, qui n’est qu’un jour­nalier[2], gagne en huit. Son ouvrage n’est pas tout à fait aussi malpropre ; il est moins dangereux, il ne se fait pas sous terre et loin de la clarté du jour. La considération entre pour beaucoup dans le salaire des professions honorables. Sous le rapport de la rétribution pécuniaire, tout bien considéré, elles sont, en général, trop peu payées, comme je le ferai voir bientôt[3]. La défaveur attachée à un état produit un effet con­traire. Le métier de boucher a quelque chose de cruel et de repoussant ; mais, dans la plupart des endroits, c’est le plus lucratif de presque tous les métiers ordinaires. Le plus affreux de tous les emplois, celui d’exécuteur public, est, en proportion de la quantité de travail, mieux rétribué que quelque autre métier que ce soit.

La chasse et la pêche, les occupations les plus importantes de l’homme dans la première enfance des sociétés, deviennent, dans l’état de civilisation, ses plus agréa­bles amusements, et il se livre alors par plaisir à ce qu’il faisait jadis par nécessité. Ainsi, dans une société civilisée, il n’y a que de très-pauvres gens qui fassent par métier ce qui est pour les autres l’objet d’un passe-temps. Telle a été la condition des pêcheurs depuis Théocrite[4]. Dans la Grande-Bretagne, un braconnier est un homme fort pauvre. Dans le pays où la rigueur des lois ne permet pas le braconnage, le sort d’un homme qui fait son métier de la chasse, moyennant une permission, n’est pas beaucoup meilleur. Le goût naturel des hommes pour ce genre d’occupation y porte beaucoup plus de gens qu’elle ne peut en faire vivre dans l’aisance, et ce que produit un tel travail, en proportion de sa quantité, se vend toujours à trop bon marché pour fournir aux travailleurs au-delà de la plus chétive subsistance.

Le désagrément et la défaveur de l’emploi influent de la même manière sur les profits des capitaux. Le maître d’une auberge ou d’une taverne, qui n’est jamais le maî­tre chez lui, et qui est exposé aux grossièretés du premier ivrogne, n’exerce pas une industrie très-agréable ni très-considérée ; mais il y a peu de commerce ordinaire dans lequel on puisse, avec un petit capital, réaliser d’aussi gros profits.

Secondement, les salaires du travail varient suivant la facilité et le bon marché de l’apprentissage, ou la difficulté et la dépense qu’il exige.

Quand on a établi une machine coûteuse, on espère que la quantité extraordinaire de travail qu’elle accomplira avant d’être tout à fait hors de service remplacera le capital employé à l’établir, avec les profits ordinaires tout au moins. Un homme qui a dépensé beaucoup de temps et de travail pour se rendre propre à une profession qui demande une habileté et une expérience extraordinaires, peut être comparé à une de ces machines dispendieuses. On doit espérer que la fonction à laquelle il se prépare lui rendra, outre les salaires du simple travail, de quoi l’indemniser de tous les frais de son éducation, avec au moins les profits ordinaires d’un capital de la même valeur. Il faut aussi que cette indemnité se trouve réalisée dans un temps raisonnable, en ayant égard à la durée très-incertaine de la vie des hommes, tout comme on a égard à la durée plus certaine de la machine.

C’est sur ce principe qu’est fondée la différence entre les salaires du travail qui demande une grande habileté, et ceux du travail ordinaire.

La police de l’Europe considère comme travail demandant de l’habileté celui de tous les ouvriers, artisans et manufacturiers, et comme travail commun celui de tous les travailleurs de la campagne. Elle paraît supposer que le travail des premiers est d’une nature plus délicate et plus raffinée que celui des autres. Il peut en être ainsi dans certains cas ; mais le plus souvent il en est autrement, comme je tâcherai bientôt de le faire voir[5]. Aussi les lois et coutumes d’Europe, afin de rendre l’ouvrier capable d’exercer la première de ces deux espèces de travail, lui imposent la nécessité d’un apprentissage, avec des conditions plus ou moins rigoureuses, selon les différents pays[6] ; l’autre reste libre et ouvert à tout le monde, sans condition. Tant que dure l’apprentissage, tout le travail de l’apprenti appartient à son maître ; pendant ce même temps, il faut souvent que sa nourriture soit payée par ses père et mère ou quelque autre de ses parents, et presque toujours il faut au moins qu’ils l’habillent. Ordinaire­ment aussi, on donne au maître quelque argent pour qu’il enseigne son métier à l’apprenti. Les apprentis qui ne peuvent donner d’argent donnent leur temps, ou s’en­ga­gent pour un plus grand nombre d’années que le temps d’usage ; convention toujours très-onéreuse pour l’apprenti, quoiqu’elle ne soit pas toujours, à cause de l’indolence habituelle de celui-ci, très-avantageuse pour le maître[7]. Dans les travaux de la campa­gne, au contraire, le travailleur se prépare peu à peu aux fonctions les plus difficiles tout en s’occupant des parties les plus faciles de la besogne ; et son travail suffit à sa subsistance dans tous les différents degrés de sa profession. Il est donc juste qu’en Europe les salaires des artisans, gens de métier et ouvriers de manufactures, soient un peu plus élevés que ceux des ouvriers ordinaires ; ils le sont aussi et, à cause de la supériorité de leurs salaires, les artisans sont regardés presque partout comme faisant partie d’une classe plus relevée. Cependant cette supériorité est bien peu considérable ; le salaire moyen d’un ouvrier à la journée, dans les fabriques les plus communes, com­me celles de draps et de toiles unies, n’est guère supérieur, dans la plupart des lieux, aux salaires journaliers des simples manœuvres. À la vérité, l’artisan est plus constamment et plus uniformément occupé, et la supériorité de son gain paraîtra un peu plus forte si on le calcule pour toute l’année ensemble. Toutefois, cette supériorité ne s’élève pas au-dessus de ce qu’il faut pour compenser la dépense plus forte de son éducation[8].

L’éducation est encore bien plus longue et plus dispendieuse dans les arts qui exigent une grande habileté, et dans les professions libérales. La rétribution pécu­niaire des peintres, des sculpteurs, des gens de loi et des médecins doit donc être beaucoup plus forte, et elle l’est aussi.

Quant aux profits des capitaux, ils semblent être très-peu affectés par la facilité ou la difficulté d’apprentissage de la profession dans laquelle ils sont employés. Les différents emplois des capitaux dans les grandes villes paraissent offrir communément chacun la même somme de facilités et de difficultés. Une branche quelconque de com­merce, soit étranger, soit domestique, ne saurait être beaucoup plus compliquée qu’une autre.

Troisièmement, les salaires du travail varient dans les différentes professions, suivant la constance ou l’incertitude de l’occupation.

Dans certaines professions, l’occupation est plus constante que dans d’autres. Dans la plus grande partie des ouvrages de manufactures, un journalier est à peu près sûr d’être occupé tous les jours de l’année où il sera en état de travailler ; un maçon en pierres ou en briques, au contraire, ne peut pas travailler dans les fortes gelées ou par un très-mauvais temps et, dans tous les autres moments, il ne peut compter sur de l’occupation qu’autant que ses pratiques auront besoin de lui ; conséquemment, il est sujet à se trouver souvent sans occupation. Il faut donc que ce qu’il gagne quand il est occupé, non-seulement l’ entretienne pour le temps où il n’a rien à faire, mais le dédommage encore en quelque sorte des moments de souci et de découragement que lui cause quelquefois la pensée d’une situation aussi précaire[9]. Aussi, dans les lieux où le gain de la plupart des ouvriers de manufactures se trouve être presque au niveau des salaires journaliers des simples manœuvres, celui des maçons, est en général, de la moitié ou du double plus élevé. Quand les simples manœuvres gagnent 4 et 5 schellings par semaine, les maçons en gagnent fréquemment 7 et 8 ; quand les premiers en gagnent 6, les autres en gagnent souvent 9 et 10 ; et quand ceux-là en gagnent 9 ou 10, comme à Londres, ceux-ci communément en gagnent 15 et 18. Cependant, il n’y a aucune espèce de métier qui paraisse plus facile à apprendre que celui d’un maçon. On dit que pendant l’été, à Londres, on emploie quelquefois les porteurs de chaises comme maçons en briques. Les hauts salaires de ces ouvriers sont donc moins une récompense de leur habileté, qu’un dédommagement de l’interruption qu’ils éprouvent dans leur emploi.

Le métier de charpentier en bâtiment paraît exiger plus de savoir et de dextérité que celui de maçon. Cependant, en plusieurs endroits, car il n’en est pas de même par­tout, le salaire journalier du charpentier est un peu moins élevé. Quoique son occupa­tion dépende beaucoup du besoin accidentel que ses pratiques ont de lui, cependant elle n’en dépend pas entièrement, et elle n’est pas sujette à être interrompue par les mauvais temps.

Quand il arrive que, en certaines localités, l’ouvrier n’est pas occupé constamment dans les mêmes métiers où, en général, il l’est constamment ailleurs, alors son salaire s’élève bien au-dessus de la proportion ordinaire avec le salaire du simple travail. À Londres, presque tous les compagnons de métier sont sujets à être arrêtés et renvoyés par leurs maîtres, d’un jour à l’autre ou de semaine en semaine, de la même manière que les journaliers dans les autres endroits. La plus basse classe d’artisans, celle des garçons tailleurs, y gagne en conséquence une demi-couronne[10] par jour, quoique 18 deniers[11] y puissent passer pour le salaire du simple travail. Dans les petites villes et les villages, au contraire, les salaires des garçons tailleurs sont souvent à peine au niveau de ceux des simples manœuvres ; mais c’est qu’à Londres ils restent souvent plusieurs semaines sans occupation, particulièrement pendant l’été.

Quand l’incertitude de l’occupation se trouve réunie à la fatigue, au désagrément et à la malpropreté de la besogne, alors elle élève quelquefois les salaires du travail le plus grossier au-dessus de ceux du métier le plus difficile. Un charbonnier des mines, qui travaille à la pièce, passe pour gagner communément, à Newcastle, environ le dou­ble, et dans beaucoup d’endroits de l’Écosse environ le triple des salaires du travail de manœuvre. Ce taux élevé provient entièrement de la dureté, du désagrément et de la malpropreté de la besogne. Dans la plupart des cas, cet ouvrier peut être occupé autant qu’il le veut. Le métier des déchargeurs de charbon à Londres égale presque ce­lui des charbonniers pour la fatigue, le désagrément et la malpropreté ; mais l’occu­pation de la plupart d’entre eux est nécessairement très-peu constante, à cause de l’irrégularité dans l’arrivée des bâtiments de charbon. Si donc les charbonniers des mines gagnent communément le double et le triple des salaires du manœuvre, il ne doit pas sembler déraisonnable que les déchargeurs de charbon gagnent quatre et cinq fois la valeur de ces mêmes salaires. Aussi, dans les recherches que l’on fit, il y a quelques années, sur le sort de ces ouvriers, on trouva que sur le pied auquel on les payait alors, ils pouvaient gagner 6 à 10 schellings par jour ; or, 6 schellings sont environ le quadruple des salaires du simple travail à Londres, et dans chaque métier particulier on peut toujours regarder les salaires les plus bas comme ceux de la très-majeure partie des ouvriers de ce métier. Quelque exorbitants que ces gains puissent paraître, s’ils étaient plus que suffisants pour compenser toutes les circonstances désa­gréables qui accompagnent cette besogne, il se jetterait bientôt tant de concur­rents dans ce métier, qui n’a aucun privilège exclusif, que les gains y baisseraient bien vite au taux le plus bas.

Les profits ordinaires des capitaux ne peuvent, dans aucune industrie, être affectés par la constance ou l’incertitude de l’emploi. C’est la faute du commerçant, et non celle des affaires, si le capital n’est pas constamment employé.

Quatrièmement, les salaires du travail peuvent varier suivant la confiance plus ou moins grande qu’il faut accorder à l’ouvrier.

Les orfèvres et les joailliers, en raison des matières précieuses qui leur sont confiées, ont partout des salaires supérieurs à ceux de beaucoup d’autres ouvriers dont le travail exige non-seulement autant, mais même beaucoup plus d’habileté.

Nous confions au médecin notre santé, à l’avocat et au procureur notre fortune, et quelquefois notre vie et notre honneur ; des dépôts aussi précieux ne pourraient pas, avec sûreté, être remis dans les mains de gens pauvres et peu considérés. Il faut donc que la rétribution soit capable de leur donner dans la société le rang qu’exige une confiance si importante. Lorsque à cette circonstance se joint encore celle du long temps et des grandes dépenses consacrés à leur éducation, on sent que le prix de leur travail doit s’élever encore beaucoup plus haut.

Quand une personne n’emploie au commerce d’autres capitaux que les siens pro­pres, il n’y a pas lieu à confiance, et le crédit qu’elle peut d’ailleurs se faire dans le public ne dépend pas de la nature de son commerce, mais de l’opinion qu’on a de sa fortune, de sa probité et de sa prudence. Ainsi, les différents taux du profit dans les diverses branches d’industrie ne peuvent pas résulter des différents degrés de confiance accordés à ceux qui les exercent.

Cinquièmement, les salaires du travail dans les différentes occupations varient suivant la chance de succès.

Dans les divers genres d’occupation, il est plus ou moins probable, à divers degrés, qu’un apprenti acquerra la capacité nécessaire pour remplir l’emploi auquel on le destine. Dans la plus grande partie des métiers, le succès est à peu près sûr, mais il est très-incertain dans les professions libérales. Mettez votre fils en apprentissage chez un cordonnier, il n’est presque pas douteux qu’il apprendra à faire une paire de souliers ; mais envoyez-le à une école de droit, il y a au moins vingt contre un à parier qu’il n’y fera pas assez de progrès pour être en état de vivre de cette profession. Dans une loterie parfaitement égale, ceux qui tirent les billets gagnants doivent gagner tout ce que perdent ceux qui tirent les billets blancs. Dans une profession où vingt person­nes échouent pour une qui réussit, celle-ci doit gagner tout ce qui aurait pu être gagné par les vingt qui échouent. L’avocat, qui ne commence peut-être qu’à l’âge de quarante ans à tirer parti de sa profession, doit recevoir la rétribution, non-seulement d’une éducation longue et coûteuse, mais encore de celle de plus de vingt autres étudiants, à qui probablement cette éducation ne rapportera jamais rien. Quelque exorbitants que semblent quelquefois les honoraires des avocats, leur rétribution réelle n’est jamais égale à ce résultat. Calculez la somme vraisemblable du gain annuel de tous les ou­vriers d’un métier ordinaire, dans un lieu déterminé, comme cordonniers ou tisse­rands, et la somme vraisemblable de leur dépense annuelle, vous trouverez qu’en général la première de ces deux sommes l’emportera sur l’autre ; mais faites le même calcul à l’égard des avocats et étudiants en droit dans tous les différents collèges de jurisconsultes, et vous trouverez que la somme de leur gain annuel est en bien petite proportion avec celle de leur dépense annuelle, en évaluant même la première au plus haut, et la seconde au plus bas possible. La loterie du droit est donc bien loin d’être une loterie parfaitement égale, et cette profession, comme la plupart des autres profes­sions libérales, est évidemment très-mal récompensée, sous le rapport du gain pécuniaire.

Ces professions cependant ne sont pas moins suivies que les autres, et malgré ces motifs de découragement, une foule d’esprits élevés et généreux s’empressent d’y en­trer. Deux causes différentes contribuent à cette vogue : la première, c’est le désir d’ac­quérir la célébrité qui est le partage de ceux qui s’y distinguent ; et la seconde, c’est cette confiance naturelle que tout homme a plus ou moins, non-seulement dans ses talents, mais encore dans son étoile.

Exceller dans une profession dans laquelle très-peu atteignent la médiocrité, est la marque la plus décisive de ce qu’on appelle génie ou mérite supérieur. L’admiration publique, qui accompagne des talents aussi distingués, compose toujours une partie de leur récompense, ou plus grande ou plus faible, selon que cette admiration publi­que est d’un genre plus ou moins élevé ; elle forme une partie considérable de la récom­pense dans la profession de médecin, une plus grande encore peut-être dans celle d’avocat, et elle est presque la seule rémunération de ceux qui cultivent la poésie et la philosophie.

Il y a des talents très-brillants et très-agréables qui entraînent une certaine sorte d’admiration pour celui qui les possède, mais dont l’exercice, quand il est fait en vue du gain, est regardé, soit raison ou préjugé, comme une espèce de prostitution publi­que. Il faut donc que la récompense pécuniaire de ceux qui les exercent ainsi soit suffisante pour indemniser, non-seulement du temps, de la peine et de la dépense d’acquérir ces talents, mais encore de la défaveur qui frappe ceux qui en font un moyen de subsistance. Les rétributions exorbitantes que reçoivent les comédiens, les chanteurs et danseurs d’opéra, etc., sont fondées sur ces deux principes : l° la rareté et la beauté du talent ; 2° la défaveur attachée à l’emploi lucratif que l’on en fait. Il paraît absurde, au premier coup d’œil, de mépriser leurs personnes et en même temps de récompenser leurs talents avec une extrême prodigalité. C’est pourtant parce que nous faisons l’un, que nous sommes obligés de faire l’autre. Si l’opinion publique ou le préjugé venait jamais à changer à l’égard de ces professions, leur récompense pécu­niaire tomberait bientôt après. Beaucoup plus de gens s’y adonneraient, et la con­currence y ferait baisser bien vite le prix du travail. Ces talents, quoique bien loin d’être communs, ne sont pourtant pas aussi rares qu’on le pense. Il y a bien des gens qui les possèdent dans la dernière perfection, mais qui regarderaient comme au-dessous d’eux d’en tirer parti ; et il y en a encore bien davantage qui seraient en état de les acquérir, si ces talents étaient plus considérés.

L’opinion exagérée que la plupart des hommes se forment de leurs propres talents est un mal ancien qui a été observé par les philosophes et les moralistes de tous les temps. Leur folle confiance en leur bonne étoile a été moins remarquée ; c’est cepen­dant un mal encore plus universel, s’il est possible. Il n’y a pas un homme sur terre qui n’en ait sa part, quand il est bien portant et un peu animé. Chacun s’exagère plus ou moins la chance du gain ; quant à celle de la perte, la plupart des hommes la comptent au-dessous de ce qu’elle est, et il n’y en a peut-être pas un seul, bien dispos de corps et d’esprit, qui la compte pour plus qu’elle ne vaut.

Le succès général des loteries nous montre assez que l’on s’exagère naturellement les chances du gain. On n’a jamais vu et on ne verra jamais une loterie au monde qui soit parfaitement égale, ou dans laquelle la somme du gain compense celle de la perte, parce que l’entrepreneur n’y trouverait pas son compte. Dans les loteries établies par les gouvernements, les billets ne valent pas, en réalité, le prix que payent les premiers souscripteurs, et cependant ils sont communément revendus sur la place, à 20, 30 et quelquefois 40 p. 100 de bénéfice. Le vain espoir de gagner quelqu’un des gros lots est la seule cause de la demande. Les gens les plus sages ont peine à regarder comme une folie ce fait de payer une petite somme pour acheter la chance de gagner 10 ou 20,000 livres, quoiqu’ils sachent bien que cette petite somme est peut-être 20 ou 30 p. 100 plus que la chance ne vaut. Dans une loterie où il n’y aurait pas de lot au-dessus de 20 livres, mais qui se rapprocherait plus d’une parfaite égalité que les loteries pu­bli­ques ordinaires, les billets ne seraient pas aussi courus. Afin de s’assurer une meil­leure chance pour quelques-uns des gros lots, il y a des gens qui achètent beaucoup de billets, et d’autres qui s’associent pour de petites portions dans un beau­coup plus grand nombre de billets. C’est pourtant une des propositions les mieux dé­mon­trées en mathématiques, que plus on prend de billets, plus on a de chances de perte contre soi. Prenez tous les billets de la loterie, et vous serez sûr de perdre ; or, plus le nombre des billets pris sera grand, plus on approchera de cette certitude.

Les profits extrêmement modérés des assureurs nous font bien voir le plus sou­vent que les chances de perte sont calculées au-dessous de ce qu’elles sont, et presque jamais au-dessus. Pour que l’assurance, ou contre l’incendie, ou contre les risques de mer, soit une industrie, il faut que la prime ordinaire soit suffisante pour compenser les pertes ordinaires, payer les frais de l’établissement et fournir le profit qu’aurait pu rapporter le même capital employé à tout autre commerce. La personne qui ne paye pas plus que cela ne paye évidemment que la vraie valeur du risque ou le prix le plus bas auquel elle puisse raisonnablement s’attendre qu’on voudra le lui garantir. Mais, quoique beaucoup de gens aient gagné un peu d’argent dans le commerce des assu­ran­ces, il y en a très-peu qui y aient fait de grandes fortunes, et de cette seule considé­ration il paraît résulter assez clairement que la balance ordinaire des profits et des pertes n’est pas plus avantageuse dans ce genre d’affaires que dans tout autre genre de commerce, où tant de gens font leur fortune. Et encore, toute modérée qu’est la prime d’assurance, beaucoup de gens font si peu de compte du risque, qu’ils ne se soucient pas de la payer. À prendre tout le royaume en masse, il y a dix-neuf maisons sur vingt, ou peut-être même quatre-vingt-dix-neuf sur cent, qui ne sont pas assurées contre les incendies[12]. Les risques de mer sont plus alarmants pour la plupart des intéressés, aussi la proportion des vaisseaux assurés à ceux qui ne le sont pas est-elle beaucoup plus forte. Il en est cependant un grand nombre, dans tous les temps et même en temps de guerre, qui font voile sans être assurés ; et quelquefois cela peut se faire sans imprudence. Quand une grande compagnie ou même un gros négociant a vingt ou trente vaisseaux en mer, ils s’assurent pour ainsi dire l’un l’autre. Il se peut que la prime épargnée sur tous fasse compensation avec les pertes qu’il est probable de rencontrer d’après le cours ordinaire des chances diverses. Toutefois, dans la plupart des cas, c’est moins par suite d’un calcul aussi approfondi que l’on néglige d’assurer les vaisseaux, que par l’effet de cette insouciance et de cette présomption qui portent à mépriser le danger, comme pour l’assurance des maisons.

L’âge où les jeunes gens font le choix d’un état est, de toutes les époques de la vie, celle où ce mépris du danger et cette confiance présomptueuse qui se flatte toujours de réussir agissent le plus puissamment. C’est là qu’on peut observer combien peu la crainte d’un événement malheureux est capable de balancer l’espoir d’un bon succès. Si l’empressement avec lequel on embrasse les professions libérales en est une preuve, cette preuve est encore bien plus sensible dans l’ardeur que mettent les gens du peuple à s’enrôler comme soldats ou comme matelots.

On voit tout d’un coup l’étendue des risques que court un soldat. Cependant, sans réfléchir au danger, les jeunes volontaires ne sont jamais si empressés de s’enrôler qu’au commencement d’une guerre ; et quoiqu’il n’y ait pour eux presque aucune chance d’avancement, leurs jeunes têtes se figurent mille occasions, qui n’arrivent jamais, d’acquérir de la gloire et des distinctions. Ces espérances romanesques sont le prix auquel ils vendent leur sang. Leur paye est au-dessous du salaire des simples manœuvres, et quand ils sont en activité de service, leurs fatigues sont beaucoup plus grandes que celles de ces derniers.

La loterie de la marine n’est pas tout à fait aussi désavantageuse que celle de l’armée. Le fils d’un ouvrier ou d’un bon artisan se met en mer souvent avec le consen­te­ment de son père ; mais c’est toujours sans ce consentement qu’il s’enrôle comme soldat. Dans le premier de ces métiers, d’autres personnes que lui voient quelque possibilité à ce qu’il y fasse quelque chose ; dans l’autre, cette chance n’est visible que pour lui seul. Un grand amiral excite moins l’admiration publique qu’un général, et les plus grands succès dans le service de mer promettent moins de gloire et d’honneurs que de pareils succès sur terre. On retrouve la même différence dans tous les grades inférieurs des deux services. Par les règlements sur la préséance, un capitaine dans la marine a le même rang qu’un colonel dans l’armée, mais dans l’opinion générale il ne tient pas la même place. Comme dans cette loterie les premiers lots sont moindres, il faut que les petits soient plus nombreux. Aussi les simples matelots sont plus souvent dans le cas d’avancer et de se faire un sort que les simples soldats, et c’est l’espoir de ces lots qui met principalement ce métier en crédit. Qu’il exige bien plus de savoir et de dextérité que presque tout autre métier d’artisan, et quoique toute la vie d’un matelot soit une suite continue de travaux et de dangers, cependant, tant qu’il reste simple matelot, pour tout ce savoir et toute cette dextérité, pour tous ces travaux et ces dangers, il reçoit à peine d’autre récompense que le plaisir d’accomplir les uns et de surmonter les autres. Le salaire du matelot n’est pas plus fort que celui d’un simple manœuvre, dans le port qui règle le taux de ces salaires. Comme les matelots passent continuellement d’un port à un autre, la paye mensuelle de ceux qui partent de tous les différents ports de la Grande-Bretagne se rapproche bien plus du même niveau, que celle des autres ouvriers dans tous ces endroits différents, et le taux du port d’où part et où arrive le plus grand nombre de matelots, qui est le port de Londres, règle le taux de tous les autres ports. À Londres, les salaires de la majeure partie des différentes classes d’ouvriers sont environ le double de ceux de la même classe à Édimbourg. Mais les matelots qui partent du port de Londres, gagnent rarement au-delà de 3 ou 4 schellings par mois de plus que ceux qui partent du port de Leith, et souvent même la différence n’est pas si grande. En temps de paix et dans le service marchand, le prix de Londres est de 1 guinée à 27 schellings environ, par mois de trente jours[13]. Un ouvrier ordinaire à Londres, sur le pied de 9 à 10 schellings par semaine, peut gagner, dans le même mois, de 40 à 45 schellings. À la vérité le matelot est fourni de vivres en sus de sa paye ; cependant leur valeur n’excède peut-être pas toujours la différence de sa paye avec celle de l’ouvrier ordinaire ; et quand cela serait quelquefois, ce sur­plus ne forme pas un gain net pour le matelot, puisqu’il ne peut pas le partager avec sa femme et ses enfants, qu’il est toujours obligé de faire subsister chez lui sur ses salaires[14].

Cette vie, pleine d’aventures et de périls, où l’on se voit sans cesse à deux doigts de la mort, loin de décourager les jeunes gens, semble donner à la profession un attrait de plus pour eux. Dans les rangs inférieurs du peuple, une mère tremble sou­vent d’envoyer son fils à l’école dans une ville maritime, de peur que la vue des vaisseaux et le récit des aventures des matelots ne l’excitent à se mettre en mer. La perspective lointaine de ces hasards, dont nous espérons triompher par courage et par adresse, n’a rien de désagréable pour nous, et elle ne fait nullement hausser les salai­res dans un emploi. Mais il n’en est pas de même des risques où le courage et l’adresse ne peuvent rien. Dans les métiers qui sont connus pour être très-malsains, les salaires du travail sont très-élevés. Le défaut de salubrité est une sorte de désagrément, et c’est sous cette distinction générale qu’il faut ranger les effets qu’il produit sur les salaires.

Dans les divers emplois du capital, le taux ordinaire du profit varie plus ou moins, suivant le plus ou moins de certitude des rentrées. Il y a, en général, moins d’incer­titude dans le commerce intérieur que dans le commerce étranger, et dans certaines branches du commerce étranger que dans d’autres ; dans le commerce de l’Amérique septentrionale, par exemple, que dans celui de la Jamaïque. Le taux ordinaire du pro­fit s’élève toujours plus ou moins avec le risque. Il ne paraît pas pourtant qu’il s’élève en proportion du risque, ou de manière à le compenser parfaitement. C’est dans les commerces les plus hasardeux que les banqueroutes sont les plus fréquentes. Le métier du contrebandier, le plus hasardeux de tous, mais aussi le plus lucratif quand l’affaire réussit, conduit infailliblement à la banqueroute. Cette confiance présomptu­euse dans le succès paraît agir ici comme partout ailleurs, et entraîner tant de gens à s’aventurer dans les affaires périlleuses, que la concurrence y réduit le profit au-dessous de ce qui serait nécessaire pour compenser le risque. Pour le compenser tout à fait, il faudrait que les rentrées ordinaires, outre les profits ordinaires du capital, pussent non-seulement remplacer toutes les pertes accidentelles, mais encore qu’elles rapportassent aux coureurs d’aventures un surcroît de profit du même genre que le profit des assureurs. Mais si les rentrées ordinaires suffisaient à tout cela, les ban­que­routes ne seraient pas plus fréquentes dans ce genre de commerce que dans les autres.

Ainsi, des cinq circonstances qui font varier les salaires du travail, il n’y en a que deux qui influent sur les profits du capital : l’agrément ou le désagrément, la sûreté ou le risque qui accompagnent le genre d’affaire auquel le capital est employé. Sous le rapport de l’agrément ou désagrément, il n’y a, dans la très-grande partie des emplois du capital, que peu ou point de différence, mais il y en a beaucoup dans les emplois du travail ; et quant au risque, quoiqu’il fasse hausser les profits du capital, il ne paraît pas que cette hausse ait toujours lieu en proportion du risque. De tout cela il résulte nécessairement que, dans une même société ou dans une même localité, le taux moyen des profits ordinaires dans les différents emplois du capital se rappro­chera bien plus de l’égalité que celui des salaires pécuniaires des diverses espèces de travail ; aussi est-ce bien ce qui arrive. La différence entre le gain d’un simple ma­nœuvre et celui d’un avocat ou d’un médecin, est évidemment bien plus grande que celle qui peut exister entre les profits ordinaires de deux différentes branches d’indus­trie, quelles qu’elles soient. D’ailleurs, la différence apparente qui semble exister entre deux différents genres d’industrie est, en général, une illusion qui provient de ce que nous ne distinguons pas toujours ce qui doit être regardé comme salaire, de ce qui doit être regardé comme profit.

Les profits des apothicaires passent, par une sorte de dicton, pour quelque chose de singulièrement exorbitant. Cependant ce profit, en apparence énorme, n’est souvent autre chose qu’un salaire fort raisonnable. Le savoir d’un apothicaire est d’une nature bien plus délicate et plus raffinée que celui d’aucun autre ouvrier, et la confiance dont il doit être investi est aussi d’une bien plus haute importance. Il est, dans toutes les circonstances, le médecin des pauvres, et celui des riches, quand le danger ou la souffrance ne sont pas très-considérables. Il faut que ses salaires soient en raison de ce savoir et de cette confiance, et il ne peut les trouver en général que dans le prix de ses drogues. La totalité des drogues que l’apothicaire le plus achalandé pourra vendre dans le cours d’une année, dans la ville qui fournit le marché le plus étendu, ne lui coûtera peut-être pas plus de 30 ou 40 livres. Quand même il ne les vendrait à 3 ou 400, ou même à 1000 pour 100 de profit, ce prix ne serait souvent que le salaire rai­sonnable de son travail, qui se trouve ainsi compris dans le prix de ses drogues, car il ne peut pas être prélevé autrement. La plus grande partie de ses profits apparents n’est qu’un véritable salaire déguisé sous la forme de profits.

Dans un port de mer peu considérable, un petit épicier se fera 40 ou 50 p. 100 d’un capital d’une centaine de livres sterling, tandis qu’un fort marchand en gros, dans le même heu, pourra à peine faire rendre 8 à 10 p. 100 à un capital de 10,000 livres. Le commerce de l’épicier y est nécessaire à la consommation des habitants ; mais un marché aussi resserré ne peut pas comporter l’emploi d’un plus gros capital dans ce négoce. Cependant il faut non-seulement qu’un homme vive de son commerce, il faut encore qu’il puisse en vivre convenablement aux conditions que ce commerce exige de lui. Outre celle de posséder un petit capital, il faut encore ici celle de savoir lire, écrire et compter ; il faut celle de pouvoir aussi juger assez passablement de peut-être cinquante ou soixante espèces de marchandises différentes, de leurs prix, de leurs qualités, et des marchés où on peut se les procurer au meilleur compte ; il faut, en un mot, avoir toutes les connaissances nécessaires a un gros marchand, et rien n’empêche celui-ci de le devenir, que le manque d’un capital suffisant. On ne peut pas dire que 30 ou 40 livres par an soient une récompense trop forte pour le travail d’un homme qui réunit toutes ces connaissances. Si vous déduisez cela des gros profits apparents de son capital, il ne restera plus guère que les profits ordinaires. La plus grande partie du profit apparent est donc aussi, dans ce cas, un véritable salaire[15]. 

La différence entre le profit apparent de la vente en détail et celui de la vente en gros est bien moindre dans une capitale que dans une petite ville ou dans un village. Quand il est possible d’employer un fonds de 10,000 livres au commerce d’épicerie, les salaires du travail de l’épicier ne sont qu’une bagatelle à ajouter à ce qui est réelle­ment le profit d’un aussi gros capital. Ainsi les profits apparents d’un très-fort détail­lant, dans une grande ville, se rapprochent beaucoup plus de ceux du marchand en gros ; c’est pour cette raison que les marchandises qui se vendent en détail sont, en général, à aussi bon marché, et souvent à bien meilleur marché dans la capitale que dans les petites villes ou dans les villages. Les épiceries, par exemple, y sont géné­ralement à bien meilleur marché ; le pain et la viande de boucherie y sont souvent à aussi bon marché. Il n’en coûte pas plus pour transporter des épiceries dans une gran­de ville que dans un village ; mais il en coûte bien davantage pour transporter dans la première du blé et du bétail, dont la plus grande partie est amenée d’une grande dis­tance. Le premier prix des épiceries étant le même dans les deux endroits, elles seront à meilleur marché là où elles sont chargées d’un moindre profit. Le premier prix du pain et de la viande de boucherie est plus fort dans la grande ville que dans le village ; quoique chargés d’un profit moindre, ils n’y sont pas toujours à meilleur marché, mais ils s’y vendent souvent au même prix. Dans des articles comme le pain et la viande, la même cause qui diminue le profit apparent augmente les frais de premier achat. C’est l’étendue du marché qui, offrant de l’emploi à de plus gros capitaux, diminue le profit apparent ; mais aussi c’est elle qui, obligeant de se fournir à de plus grandes distances, augmente le premier coût. Cette diminution d’une part, et cette augmenta­tion de l’autre, semblent, en beaucoup de cas, se contrebalancer à peu près ; et c’est là probablement la raison pour laquelle les prix du pain et de la viande de boucherie sont en général, à très-peu de chose près, les mêmes dans la plus grande partie du royaume, quoiqu’en différents endroits il y ait ordinairement de grandes différences dans les prix du blé et du bétail.

Quoique les profits des capitaux, tant pour la vente en détail que pour la vente en gros, soient, en général, plus faibles dans la capitale que dans de petites villes ou dans des villages, cependant on voit fort souvent dans la première de grandes fortunes faites avec de petits commencements, et on n’en voit presque jamais dans les autres. Dans de petites villes et dans des villages, le peu d’étendue du marché empêche le commerce de s’étendre à mesure que grossit le capital ; aussi, dans de pareils lieux, quoique le taux des profits d’une personne en particulier puisse être très-élevé, cepen­dant la masse ou la somme totale de ces profits et, par conséquent, le montant de son accumulation annuelle, ne peuvent pas être très-considérables. Au contraire, dans les grandes villes, on peut étendre son commerce à mesure que le capital augmente, et le crédit d’un homme qui est économe et en prospérité augmente encore bien plus vite que son capital. Suivant que l’un et l’autre augmentent, il agrandit la sphère de ses opérations ; la somme ou le montant total de ses profits est en proportion de l’étendue de son commerce, et ce qu’il accumule annuellement est proportionné à la somme totale de ses profits. Toutefois il arrive rarement que, même dans les grandes villes, on fasse des fortunes considérables dans une industrie régulière fixée et bien connue, si ce n’est par une longue suite d’années d’une vie appliquée, économe et laborieuse. À la vérité il se fait quelquefois, dans ces endroits, des fortunes soudaines dans ce qu’on appelle proprement le commerce ou la spéculation. Le négociant qui s’abandonne à ce genre d’affaires n’exerce pas d’industrie fixe, régulière, ni bien connue. Il est cette année marchand de blé, il sera marchand de vin, l’année prochaine, et marchand de sucre, de tabac ou de thé, l’année suivante. Il se livre à toute espèce de commerce qu’il présume pouvoir donner quelque profit extraordinaire, et il l’abandonne quand il prévoit que les profits en pourront retomber au niveau de ceux des autres affaires : ses profits et ses pertes ne peuvent donc garder aucune proportion régulière avec ceux de toute autre branche de commerce fixe et bien connue. Un homme qui ne craint pas de s’aventurer peut quelquefois faire une fortune considérable en deux ou trois spécu­lations heureuses ; mais il est tout aussi probable qu’il en perdra autant par deux ou trois spéculations malheureuses. Un tel commerce ne peut s’entreprendre que dans les grandes villes. Ce n’est que dans les endroits où les affaires et les correspondances sont extrêmement étendues, qu’on peut se procurer toutes les connaissances qu’il exige.

Les cinq circonstances qui viennent d’être exposées en détail occasionnent bien des inégalités très-fortes dans les salaires et les profits, mais elles n’en occasionnent aucune dans la somme totale des avantages et désavantages réels ou imaginaires de chacun des différents emplois du travail ou des capitaux ; elles sont de nature seulement à compenser, dans certains emplois, la modicité du gain pécuniaire, et à en balancer la supériorité dans d’autres[16].

Cependant, pour que cette égalité ait lieu dans la somme totale des avantages et désavantages des emplois, trois choses sont nécessaires, en supposant même la plus entière liberté : la première, que l’emploi soit bien connu et établi depuis longtemps dans la localité ; la seconde, qu’il soit dans son état ordinaire, ou ce qu’on peut appeler son état naturel, et la troisième, qu’il soit la seule ou la principale occupation de ceux qui l’exercent.

Premièrement, cette égalité ne peut avoir lieu que dans ces emplois qui sont bien connus et qui existent depuis longtemps dans la localité.

Toutes choses égales d’ailleurs, une entreprise nouvelle donne de plus hauts salaires que les anciennes. Quand un homme forme le projet d’établir une manufac­ture nouvelle, il faut, dans le commencement, qu’il attire les ouvriers et les détourne des autres emplois par l’attrait de salaires plus forts que ceux qu’ils gagneraient dans leurs propres professions, supérieurs à ceux que mériterait le nouveau travail, et il se passera un temps considérable avant qu’il puisse risquer de les remettre au niveau commun. Les manufactures dont le débit est entièrement fondé sur la mode et la fantaisie changent continuellement, et elles ne durent presque jamais assez longtemps pour qu’on puisse les regarder comme d’anciens établissements. Au contraire, celles dont le débit tient principalement à la nécessité ou à l’utilité sont moins sujettes au changement, et peuvent conserver, pendant des siècles entiers de suite, la même for­me et le même genre de fabrication. Les salaires doivent donc naturellement être plus forts, dans les manufactures de la première espèce, que dans celles de la dernière. Birmingham produit principalement des ouvrages de la première sorte ; Sheffield, des ouvrages de la seconde ; et l’on dit que les salaires du travail dans chacune de ces places sont conformes à cette différence dans la nature de leurs produits.

Tout établissement nouveau en manufacture, toute branche nouvelle de com­mer­ce, toute pratique nouvelle en agriculture, est toujours une spéculation dont l’entrepre­neur se promet des profits extraordinaires. Ces profits sont quelquefois très-forts ; plus souvent peut-être, c’est tout le contraire qui arrive ; mais, en général, ils ne sont pas en proportion régulière avec ceux que donnent dans le voisinage les anciennes industries. Si le projet réussit, les profits sont ordinairement très-élevés d’abord. Quand ce genre de trafic ou d’opération vient à être tout à fait établi et bien connu, la concurrence réduit les profits au niveau de ceux des autres emplois.

Secondement, cette égalité dans la somme totale des avantages et désavantages des divers emplois du travail et des capitaux ne peut avoir lieu que dans les emplois qui sont dans leur état ordinaire, ou dans ce qu’on peut appeler leur état naturel.

Dans presque chaque espèce différente de travail, la demande est tantôt plus gran­de, tantôt moindre que de coutume. Dans le premier cas, les avantages de ce genre d’emploi montent au-dessus du niveau commun ; dans l’autre, ils descendent au-dessous. La demande de travail champêtre est plus forte dans le temps des foins et de la moisson que pendant le reste de l’année, et les salaires haussent avec ce surcroît de demande[17]. En temps de guerre, lorsque quarante ou cinquante matelots sont forcés de passer de la marine marchande au service du roi, la demande de matelots pour le com­merce s’élève nécessairement en proportion de leur rareté ; et dans ces cas-là, leurs salaires montent communément d’une guinée ou de 27 schellings par mois, à 40 schellings et à 3 livres. Au contraire, dans une manufacture qui décline, beaucoup d’ouvriers, plutôt que de quitter leur ancien métier, se contentent de salaires plus faibles que ceux que comporterait sans cela la nature de leur travail.

Le profit du capital varie avec le prix des marchandises qui font l’objet de l’em­ploi. Quand le prix d’une marchandise s’élève au-dessus du taux ordinaire ou moyen, les profits d’une partie au moins du capital employé à la mettre sur le marché montent au-dessus du niveau général ; et quand ce prix baisse, au contraire, ils tombent au-dessous de ce niveau. Toutes les marchandises sont plus ou moins sujettes à des variations dans leur prix, mais quelques-unes le sont beaucoup plus que d’autres. Dans toutes les choses qui sont le produit de l’industrie humaine, la demande annuelle règle nécessairement la quantité d’industrie qui s’y porte annuellement, de telle sorte que le produit moyen annuel puisse égaler, d’aussi près qu’il est possible, la consom­mation annuelle. On a déjà observé[18] que, dans quelques emplois, la même quantité d’industrie produira toujours la même ou presque la même quantité de marchandises. Par exemple, dans une manufacture de toiles ou de draps, le même nombre de bras fabriquera, dans une année, la même quantité, à fort peu de chose près, de toiles ou de draps. Le prix de marché de ce genre de marchandises n’a donc varié qu’en consé­quence de variations accidentelles dans la demande, par exemple, en conséquence d’un deuil public qui fera hausser le prix du drap noir. Mais comme, en général, la demande de la plupart des espèces de drap ou de toile est assez uniforme, il en est de même de leur prix. Au contraire, il y a d’autres emplois où la même quantité d’indus­trie ne produira pas toujours la même quantité de marchandises. Par exemple, la même quantité d’industrie produira, en différentes années, des quantités fort diffé­rentes de blé, de vin, de houblon, de sucre, de tabac, etc. Aussi, le prix de ce genre de marchandises varie, non-seulement d’après les variations de la demande mais encore d’après les variations bien plus fortes et bien plus fréquentes de la quantité produite, et il est, par conséquent, extrêmement mobile. Or, il faut nécessairement que le profit de quelques-uns de ceux qui font commerce de ces denrées se ressente de la mobilité du prix. Ceux qui se livrent au commerce de spéculation établissent leurs principales opérations sur ces sortes de marchandises. Quand ils prévoient que le prix pourra monter, ils en accaparent autant qu’ils peuvent, et ils cherchent à vendre quand il y a apparence de baisse.

Troisièmement, cette égalité dans la somme totale des avantages et désavantages de divers emplois de travail et de capitaux ne peut avoir lieu que dans les emplois qui sont la seule ou la principale occupation de ceux qui les exercent.

Lorsqu’une personne tire sa subsistance d’un emploi qui n’occupe pas la plus grande partie de son temps, elle consent volontiers, dans ses intervalles de loisir, à travailler à quelque autre emploi pour un salaire moindre que ne le comporterait sans cela la nature de ce travail.

Il existe encore, dans plusieurs endroits de l’Écosse, une classe de gens qu’on nom­me cotters ou cottagers[19], qui étaient, il y a quelques années, encore plus nom­breux qu’aujourd’hui[20]. Ce sont des espèces de domestiques externes des propriétaires et des fermiers. La rétribution d’usage qu’ils reçoivent de leur maître, c’est une maison, un petit jardin potager, autant d’herbe qu’il en faut pour nourrir une vache, et peut-être une acre ou deux de mauvaise terre labourable. Quand le maître les emploie, il leur donne en outre quatre gallons[21] de farine d’avoine par semaine, valant environ 16 deniers sterling. Pendant une grande partie de l’année, il ne les emploie pas ou les emploie très-peu, et la culture de leur petite possession ne suffit pas pour occuper tout le temps qu’on leur laisse libre. Quand ces tenanciers étaient plus nombreux qu’ils ne sont maintenant, on dit que, moyennant une très-faible rétribution, ils donnaient vo­lon­tiers le superflu de leur temps à quiconque les voulait employer, et qu’ils travail­laient pour de moindres salaires que les autres ouvriers. Il paraît qu’ils ont été autre­fois très-communs dans toute l’Europe. Dans des pays mal cultivés et encore plus mal peuplés, la plus grande partie des propriétaires et des fermiers n’auraient pas pu sans cela se pourvoir des bras extraordinaires qu’exigent, dans certains saisons, les travaux de la campagne. Il est évident que la rétribution journalière ou hebdomadaire que ces ouvriers recevaient accidentellement de leurs maîtres n’était pas le prix entier de leur travail ; leur petite possession en formait une partie considérable. Cependant plusieurs écrivains qui ont recueilli les prix du travail et des denrées dans les temps anciens, et qui se sont plu à les représenter tous deux prodigieusement bas, ont regardé cette rétribution accidentelle comme formant tout le salaire de ces ouvriers.

Le produit d’un travail fait de cette manière se présente souvent sur le marché à meilleur compte que la nature de ce travail ne le permettrait sans cette circonstance. Dans plusieurs endroits d’Écosse, on a des bas tricotés à l’aiguille à beaucoup meilleur marché qu’on ne pourrait les établir au métier partout ailleurs ; c’est l’ouvrage de domestiques et d’ouvrières qui trouvent dans une autre occupation la principale partie de leur subsistance. On importe par an, à Leith, plus de mille paires de bas de Shetland, dont le prix est de 5 à 7 deniers la paire. À Learwick, la petite capitale des îles de Shetland, le prix ordinaire du simple travail est, à ce qu’on m’a assuré, de 10 deniers par jour. Dans les mêmes îles, on tricote des bas d’estame[22], de la valeur d’une guinée la paire, et au-dessus.

La filature de toile se fait en Écosse de la même manière à peu près que les bas à l’aiguille, c’est-à-dire par des femmes qui sont louées principalement pour d’autres services. Celles qui essayent de vivre uniquement de l’un ou de l’autre de ces métiers gagnent à peine de quoi ne pas mourir de faim. Dans la plus grande partie de l’Écosse, il faut être une bonne fileuse pour gagner 20 deniers par semaine[23].

Dans les pays opulents, le marché est, en général, assez étendu pour qu’une seule occupation suffise à employer tout le travail et tout le capital de ceux qui s’y livrent. Ce n’est guère que dans les pays pauvres qu’on trouve des exemples de gens qui vivent d’un emploi et retirent en même temps quelques petits bénéfices d’un autre. Cependant, la capitale d’un pays très-riche peut nous fournir un exemple de quelque chose de semblable. Il n’y a pas, je crois, de ville en Europe où les loyers de maisons soient plus chers qu’à Londres, et je ne connais pourtant pas de capitale où on puisse trouver des chambres garnies à si bon marché ; non-seulement les logements à Lon­dres sont moins chers qu’à Paris, ils le sont même beaucoup moins qu’à Édimbourg, au même degré de commodité ; et ce qui pourra paraître singulier, c’est la cherté des loyers[24], qui est la cause du bon marché des logements. La cherté des loyers de maison à Londres ne procède pas seulement des causes qui les rendent chers dans toutes les capitales, c’est-à-dire la cherté du travail, la cherté des matériaux de construction qu’il faut en général transporter de fort loin, et par-dessus tout la cherté de la rente ou loyer du sol[25] ; chaque propriétaire de sol agit en monopoleur, et exige très-souvent, pour une seule acre de mauvaise terre dans la ville, une plus forte rente que ne pourraient lui en produire cent acres des meilleures terres de la campagne. Mais la cherté de ces loyers provient encore en partie de la coutume du pays, qui oblige tout chef de famille à prendre à loyer une maison entière, de la cave au grenier. En Angleterre, on com­prend sous le nom d’habitation ou domicile tout ce qui est renfermé sous le même toit ; tandis que ce mot, en France, en Écosse et dans beaucoup d’autres endroits de l’Europe, ne signifie souvent rien de plus qu’un seul étage. Un industriel à Londres est obligé de prendre à loyer une maison entière, dans le quartier où demeurent ses pratiques. Il tient sa boutique au rez-de-chaussée, et il couche, ainsi que sa famille, dans les combles ; ensuite il tâche de regagner une partie de son loyer en prenant des locataires dans les deux étages du milieu. C’est sur son industrie, et non sur ses locataires, qu’il compte pour entretenir sa famille, tandis qu’à Paris et à Édimbourg les gens qui fournissent les logements n’ont ordinairement pas d’autres moyens de subsistance, et qu’il faut que le prix du logement paye non-seulement le loyer de la maison, mais encore toute la dépense de la famille. 

\section{Inégalités causées par la police de l’Europe}

Telles sont les inégalités qui se trouvent dans la somme totale des avantages et désavantages des divers emplois de travail et de capitaux, même dans les pays où règne la plus entière liberté, lesquelles proviennent du défaut de quelqu’une des trois conditions ci-dessus expliquées ; mais la police qui domine en Europe, faute de laisser les choses dans une entière liberté, donne lieu à d’autres inégalités d’une bien plus grande importance.

Elle produit cet effet principalement de trois manières : la première, en restrei­gnant la concurrence, dans certains emplois, à un nombre inférieur à celui des indi­vi­dus qui, sans cela, seraient disposés à y entrer ; la seconde, en augmentant dans d’au­tres le nombre des concurrents au-delà de ce qu’il serait dans l’état naturel des choses ; et la troisième, en gênant la libre circulation du travail et des capitaux, tant d’un emploi à un autre, que d’un lieu à un autre.

Premièrement, la police qui règne en Europe donne lieu à une inégalité considé­rable dans la somme totale des avantages et désavantages des divers emplois du travail et des capitaux, en restreignant, dans certains endroits, la concurrence à un plus petit nombre d’individus que ceux qui s’y porteraient sans cela.

Pour cet objet, les principaux moyens qu’elle emploie, ce sont les privilèges exclusifs des corporations.

Le privilège exclusif d’un corps de métier restreint nécessairement la concurrence, dans la ville où il est établi, à ceux auxquels il est libre d’exercer ce métier. Ordi­nai­re­ment, la condition requise pour obtenir cette liberté est d’avoir fait son apprentissage sous un maître ayant qualité pour cela. Les statuts de la corporation règlent quel­quefois le nombre d’apprentis qu’il est permis à un maître d’avoir, et presque toujours le nombre d’années que doit durer l’apprentissage. Le but de ces règlements est de restreindre la concurrence à un nombre d’individus beaucoup moindre que celui qui, sans cela, embrasserait cette profession. La limitation du nombre des apprentis res­treint directement la concurrence ; la longue durée de l’apprentissage la restreint d’une manière plus indirecte, mais non moins efficace, en augmentant les frais de l’édu­cation industrielle.

À Sheffield, un statut de la corporation interdit à tout maître coutelier d’avoir plus d’un apprenti à la fois. À Norwich et à Norfolk, aucun maître tisserand[26] ne peut avoir plus de deux apprentis, sous peine d’une amende de 5 livres par mois envers le roi. Dans aucun endroit de l’Angleterre ou des colonies anglaises, un maître chapelier ne peut avoir plus de deux apprentis, sous peine de 5 livres d’amende par mois, applica­bles, moitié au roi, moitié au dénonciateur. Quoique ces deux derniers règlements aient été confirmés par une loi du royaume, ils n’en sont pas moins évidemment dictés par ce même esprit de corporation qui a imaginé le statut de Sheffield. À peine les fabri­cants d’étoffes de soie à Londres ont-ils été une année érigés en corporation, qu’ils ont porté un statut qui défendait à tour maître d’avoir plus de deux apprentis à la fois ; il a fallu un acte exprès du Parlement pour casser ce statut.

La durée de l’apprentissage, dans la plupart des corps de métiers, paraît avoir été anciennement fixée, dans toute l’Europe, au terme ordinaire de sept ans. Ces corpora­tions se nommaient autrefois universités, d’un mot latin qui désigne en effet une corporation quelconque. Dans les vieilles chartes des villes anciennes, nous trouvons souvent ces expressions : l’université des forgerons, l’université des tailleurs, etc. Lors du premier établissement de ces corporations particulières, qui sont aujourd’hui désignées spécialement sous le nom d’universités, le terme des années d’étude qui fut jugé nécessaire pour obtenir le degré de maître és arts, paraît évidemment avoir été fixé d’après le terme d’apprentissage dans les métiers dont les corporations étaient beaucoup plus anciennes. De même qu’il était nécessaire d’avoir travaillé sept ans sous un maître dûment qualifié pour acquérir le droit de devenir maître dans un métier ordinaire et d’y tenir ainsi des apprentis, de même il fut nécessaire d’avoir étudié sept ans sous un maître pour être en état de devenir, dans les professions libérales, maître, professeur ou docteur (termes autrefois synonymes), et pour prendre sous soi des étudiants ou apprentis (termes qui furent aussi synonymes dans l’origine).

Le statut de la cinquième année d’Élisabeth, appelé communément le statut des apprentis, décida que nul ne pourrait à l’avenir exercer aucun métier, profession ou art pratiqué alors en Angleterre, à moins d’y avoir fait préalablement un apprentissage de sept ans au moins ; et ce qui n’avait été jusque-là que le statut de quelques corpora­tions particulières devint la loi générale et publique de l’Angleterre, pour tous les métiers établis dans les villes de marché ; car quoique les termes de la loi soient très-généraux et semblent renfermer sans distinction la totalité du royaume, cependant, en l’interprétant, on a limité son effet aux villes de marché seulement, et on a tenu que, dans les villages, une même personne pouvait exercer plusieurs métiers différents, sans avoir fait un apprentissage de sept ans pour chacun[27].

De plus, par une interprétation rigoureuse des termes du statut, on en a limité l’effet aux métiers seulement qui étaient établis en Angleterre avant la cinquième année d’Élisabeth, et on ne l’a jamais étendu à ceux qui y ont été introduits depuis cette époque.

Cette limitation a donné lieu à plusieurs distinctions qui, considérées comme règlements de police, sont bien ce qu’on peut imaginer de plus absurde. Par exemple, on a décidé qu’un carrossier ne pouvait faire, ni par lui-même, ni par des ouvriers employés par lui à la journée, les roues de ses carrosses, mais qu’il était tenu de les acheter d’un maître ouvrier en roues, ce dernier métier étant pratiqué en Angleterre antérieurement à la cinquième année d’Élisabeth. Mais l’ouvrier en roues, sans avoir jamais fait d’apprentissage chez un ouvrier en carrosses, peut très-bien faire des carrosses, soit par lui-même, soit par des ouvriers à la journée, le métier d’ouvrier en carrosses n’étant pas compris dans le statut, parce qu’à cette époque il n’était pas pratiqué en Angleterre. Il y a pour la même raison un grand nombre de métiers dans les industries de Manchester, Birmingham et Wolverhampton, qui, n’ayant pas été exercés en Angleterre antérieurement à la cinquième année d’Élisabeth, ne sont pas compris dans le statut.

En France, la durée de l’apprentissage varie dans les différentes villes et dans les différents métiers. Le terme fixé pour un grand nombre, à Paris, est de cinq ans ; mais dans la plupart, avant que l’ouvrier puisse avoir le droit d’exercer comme maître, il faut qu’il travaille encore cinq ans de plus comme ouvrier à la journée ; pendant ce dernier terme il est appelé le compagnon du maître, et ce temps s’appelle son compa­gnonnage[28].

En Écosse, il n’y a pas de loi générale qui règle universellement la durée de l’apprentissage. Le terme est différent dans les différentes corporations. Quand le terme est long, on peut, en général, en racheter une partie en payant un léger droit. En outre, dans beaucoup de villes, on achète la maîtrise dans un corps de métier quel­conque, moyennant un droit très-faible. Les tisserands en toiles de lin et de chanvre, qui sont les principales fabrications du pays, ainsi que tous les autres ouvriers qui en dépendent, ouvriers en rouets, ouvriers en dévidoirs, etc., peuvent exercer leur métier dans toute ville incorporée[29], sans payer aucun droit. Dans les villes de corporation, toute personne est libre de vendre de la viande de boucherie à tous les jours de la semaine où il est permis d’en vendre. Le terme ordinaire de l’apprentissage en Écosse est de trois ans, même dans quelques métiers très-difficiles ; et, en général, je ne connais pas de pays où les lois de corporation soient moins oppressives.

La plus sacrée et la plus inviolable de toutes les propriétés est celle de son propre travail, parce qu’elle est la source originaire de toutes les autres propriétés. Le patrimoine du pauvre est dans sa force et dans l’adresse de ses mains ; et l’empêcher d’employer cette force et cette adresse de la manière qu’il juge la plus convenable, tant qu’il ne porte de dommage à personne, est une violation manifeste de cette propriété primitive. C’est une usurpation criante sur la liberté légitime, tant de l’ouvrier que de ceux qui seraient disposés à lui donner du travail ; c’est empêcher à la fois l’un, de travailler à ce qu’il juge à propos, et l’autre, d’employer qui bon lui semble. On peut bien en toute sûreté s’en fier à la prudence de celui qui occupe un ouvrier, pour juger si cet ouvrier mérite de l’emploi, puisqu’il y va assez de son propre intérêt. Cette sollicitude qu’affecte le législateur, pour prévenir qu’on n’ emploie des personnes inca­pables, est évidemment aussi absurde qu’oppressive.

Ce n’est pas l’institution de longs apprentissages qui pourra vous garantir qu’on n’exposera pas très-souvent en vente des ouvrages défectueux. Quand on en produit de ce genre, c’est en général l’effet de la fraude, et non du manque d’habileté ; et les plus longs apprentissages ne sont pas des préservatifs contre la fraude. Pour prévenir cet abus, il faut avoir recours à des règlements d’une tout autre nature. La marque sterling sur la vaisselle, ou l’empreinte sur les draps et sur les toiles, donne aux ache­teurs une garantie beaucoup plus sûre que tous les statuts d’apprentissage possibles. Aussi fait-on, en général, attention à ces marques quand on achète, tandis qu’on ne songe guère à s’informer si l’ouvrier a rempli ou non ses sept années d’apprentissage.

L’institution des longs apprentissages ne tend nullement à rendre les jeunes gens industrieux. Un journalier qui travaille à la pièce est bien plus disposé à devenir laborieux, parce que l’exercice de son industrie lui procure un bénéfice. Un apprenti doit naturellement être paresseux, et il l’est aussi presque toujours, attendu qu’il n’a pas d’intérêt immédiat au travail. Dans les emplois inférieurs, la récompense du tra­vail est le seul attrait du travail. Ceux qui seront le plus tôt à portée de jouir de cette récompense prendront vraisemblablement le plus tôt le goût de leur métier et en acquerront les premiers l’habitude. Naturellement, un jeune homme conçoit du dégoût pour le travail, quand il travaille longtemps sans en retirer aucun bénéfice. Les en­fants qu’on met en apprentissage sur les fonds des charités publiques sont presque toujours engagés pour un terme plus long que le nombre d’années ordinaires et, en général, ils deviennent très-paresseux et très-mauvais sujets.

L’apprentissage était totalement inconnu chez les anciens, tandis que les devoirs réciproques du maître et de l’apprenti forment un article important dans nos codes modernes. La loi romaine n’en parle pas. je ne connais pas de mot grec ou latin, et je pourrais bien avancer, je crois, qu’il n’en existe point, qui réponde à l’idée que nous attachons aujourd’hui au mot d’apprenti, c’est-à-dire un serviteur engagé à travailler à un métier particulier pour le compte d’un maître, pendant un terme d’années, sous la condition que le maître lui enseignera ce métier[30]. 

De longs apprentissages ne sont nullement nécessaires. Un art bien supérieur aux métiers ordinaires, celui de faire des montres et des pendules, ne renferme pas de secrets qui exigent un long cours d’instruction. À la vérité, la première invention de ces belles machines, et même celle de quelques instruments qu’on emploie pour les faire, doit être le fruit de beaucoup de temps et d’une méditation profonde, et elle peut passer avec raison pour un des plus heureux efforts de l’industrie humaine. Mais les uns et les autres étant une fois inventés et parfaitement connus, expliquer à un jeune homme, le plus complètement possible, la manière d’appliquer ces instruments et de construire ces machines, cela doit être au plus l’affaire de quelques semaines de leçons, peut-être même serait-ce assez de quelques jours. Dans les arts mécaniques ordinaires, quelques jours pourraient certainement suffire. À la vérité, la dextérité de la main, même dans les métiers les plus simples, ne peut s’acquérir qu’à l’aide de beaucoup de pratique et d’expérience. Mais un jeune homme travaillerait avec bien plus de zèle et d’attention, si dès le commencement il le faisait comme ouvrier, en recevant une paye proportionnée au peu d’ouvrage qu’il exécuterait, et en payant à son tour les matières qu’il pourrait gâter par maladresse ou défaut d’habitude. Par ce moyen son éducation serait, en général, plus efficace, et toujours moins longue et moins coûteuse. Le maître, il est vrai, pourrait perdre à ce compte ; il y perdrait tous les salaires de l’apprenti, qu’il épargne à présent pendant sept ans de suite ; peut-être bien aussi que l’apprenti lui-même pourrait y perdre. Dans un métier appris aussi aisément, il aurait plus de concurrents, et quand l’apprenti serait devenu ouvrier parfait, ses salaires seraient beaucoup moindres qu’ils ne sont aujourd’hui. La même augmentation de concurrence abaisserait les profits des maîtres, tout comme les salaires des ouvriers. Les gens de métier et artisans de toute sorte, ceux qui exploitent des procédés secrets, perdraient sous ce rapport, mais le public y gagnerait, car tous les produits de la main-d’œuvre arriveraient alors au marché à beaucoup meilleur compte. 

C’est pour prévenir cette réduction de prix et, par conséquent, de salaires et de profits, en restreignant la libre concurrence qui n’eût pas manqué d’y donner lieu, que toutes les corporations et la plus grande partie des lois qui les concernent ont été établies. Autrefois, dans presque toute l’Europe, il ne fallait pas d’autre autorité pour ériger un corps de métier, que celle de la ville incorporée où il était établi. À la vérité, en Angleterre, il fallait aussi une charte du roi. Mais il paraît que cette prérogative a été réservée à la couronne plutôt comme moyen de tirer de l’argent des sujets que comme moyen de défendre la liberté générale contre ces monopoles oppresseurs. On voit qu’en payant un droit au roi, la charte était, en général, accordée sur-le-champ ; et lorsque quelque classe d’artisans ou de marchands s’était avisée d’agir comme corpo­ration sans avoir pris de charte, ces communautés de contrebande, comme on les appelait, ne perdaient pas toujours pour cela leurs franchises, mais elles étaient tenues de payer au roi un droit annuel, pour la permission d’exercer les privilèges qu’elles avaient usurpés[31]. La surveillance immédiate de toutes les corporations et des statuts qu’elles jugeaient à propos de faire pour leur propre régime, appartenait aux villes incorporées où elles étaient établies, et toute discipline qui s’exerçait sur elles procédait ordinairement, non du roi, mais de la grande corporation municipale, dont ces corporations subordonnées n’étaient que des membres ou des dépendances.

Le régime des villes incorporées se trouva tout à coup dans la main des mar­chands et artisans, et l’intérêt évident de chacune de leurs classes particulières fut d’em­pêcher que le marché ne fût surchargé, comme ils disent ordinairement, des objets de leur commerce particulier, c’est-à-dire, en réalité, de l’en tenir toujours dégarni. Chaque classe travailla avec ardeur à fabriquer les règlements les plus propres à ce but et, pourvu qu’on la laissât faire, elle fut très-disposée à laisser faire de même les autres classes. Chaque classe, il est vrai, au moyen de ses règlements, se trouvait obligée d’acheter les marchandises dont elle avait besoin dans la ville même, chez les marchands et artisans des autres classes, et de les payer un peu plus cher qu’elle n’aurait fait sans cela ; mais en revanche elle se trouvait aussi à même de vendre les siennes plus cher, dans la même proportion, ce qui revenait à peu près au même ; dans les affaires que les classes différentes faisaient entre elles dans la ville, aucune d’elles ne perdait à ces règlements. Mais dans les affaires qu’elles faisaient avec la campagne, toutes également trouvaient de gros bénéfices, et c’est dans ce dernier genre d’affaires que consiste tout le trafic qui soutient et qui enrichit les villes.

Chaque ville tire de la campagne toute sa subsistance et tous les matériaux de son industrie. Elle paye ces deux objets de deux manières : la première, en renvoyant à la campagne une partie de ces matériaux travaillés et manufacturés et, dans ce cas, le prix en est augmenté du montant des salaires des ouvriers, et du montant des profits de leurs maîtres ou de ceux qui les emploient immédiatement ; la seconde, en envoyant à la campagne le produit brut manufacturé ou importé dans la ville, soit des autres pays, soit des parties éloignées du même pays ; et dans ce cas aussi, le prix originaire de ces marchandises s’accroît des salaires des voituriers ou matelots, et du profit des marchands qui les emploient. Le gain résultant de la première de ces deux branches d’industrie compose tout le bénéfice que la ville retire de ses manufactures. Le gain résultant de la seconde compose tout le bénéfice de son commerce intérieur et de son commerce étranger. La totalité du gain, dans l’une et dans l’autre branche d’industrie, consiste en salaires d’ouvriers et dans les profits de ceux qui les emploient. Ainsi, tous règlements, qui tendent à faire monter ces salaires et ces profits au-dessus de ce qu’ils devaient être naturellement, tendent à permettre à la ville d’acheter, avec une moindre quantité de son travail, le produit d’une plus grande quantité du travail de la campagne. Ils donnent aux marchands et artisans de la ville un avantage sur les propriétaires, fermiers et ouvriers de la campagne, et ils rompent l’égalité naturelle, qui s’établirait sans cela dans le commerce entre la ville et la campagne. La totalité du produit annuel du travail de la société se divise annuelle­ment entre ces deux classes de la nation. L’effet de ces règlements est de donner aux habitants des villes une part de ce produit plus forte que celle qui leur reviendrait sans cela, et d’en donner une moindre aux habitants des campagnes.

Le prix que payent les villes pour les denrées et matières qui y sont annuellement importées, consiste dans tous les objets de manufactures et autres marchandises qui en sont annuellement exportés. Plus ces derniers sont vendus cher, plus les autres sont achetés bon marché. L’industrie des villes en est plus favorisée au détriment de l’industrie des campagnes. 

Pour nous convaincre que l’industrie qui s’exerce dans les villes est, dans toute l’Europe, plus favorisée que celle qui s’exerce dans les campagnes, il n’est pas besoin de se livrer à des calculs compliqués, il suffit d’une observation très-simple et à la portée de tout le monde. Il n’y a pas un pays en Europe où nous ne trouvions au moins cent personnes qui auront fait de grandes fortunes avec peu de chose, par le moyen du commerce et des manufactures, autrement par l’industrie des villes, contre une seule qui aura fait fortune par l’industrie agricole, par celle qui obtient les produits de la terre par la culture et l’amélioration du sol. Il faut donc que l’industrie soit mieux récompensée, que les salaires du travail et les profits des capitaux soient évidemment plus forts dans les villes que dans les campagnes. Or, le travail et les capitaux cher­chent naturellement les emplois les plus avantageux. Naturellement donc, ils se jetteront dans les villes le plus qu’ils pourront, et abandonneront les campagnes[32].

Les habitants d’une ville, étant rassemblés dans un même lieu, peuvent aisément communiquer et se concerter ensemble. En conséquence, les métiers les moins impor­tants qui se soient établis dans les villes ont été presque partout érigés en corporations et, même quand ils ne l’ont pas été, l’esprit de corporation, la jalousie contre les étran­gers, la répugnance à prendre des apprentis ou à communiquer les secrets du métier, y ont toujours généralement dominé, et les différentes professions ont bien su empê­cher, par des associations et des accords volontaires, cette libre concurrence qu’elles ne pouvaient gêner par des statuts. Les métiers qui n’emploient qu’un petit nombre de bras sont ceux où on se livre le plus aisément à ces sortes de complots. Il ne faut peut-être qu’une demi-douzaine de cardeurs de laine pour fournir de l’ouvrage à un millier de fileuses et de tisserands. En convenant entre eux de ne pas prendre d’apprentis, non-seulement ils peuvent se ménager plus d’occupation, mais encore tenir en quelque sorte dans leur dépendance toute la fabrique des draps, et faire monter le prix de leur travail fort au-dessus de ce que vaut la nature de leur emploi. 

Les habitants de la campagne, qui vivent dispersés et éloignés l’un de l’autre, ne peuvent pas facilement se concerter entre eux. Non-seulement ils n’ont jamais été réunis en corps de métier, mais même l’esprit de corporation n’a jamais régné parmi eux. On n’a jamais pensé qu’un apprentissage fût nécessaire pour l’agriculture, qui est la grande industrie de la campagne. Cependant, après ce qu’on appelle les beaux-arts et les professions libérales, il n’y a peut-être pas de profession qui exige une aussi grande variété de connaissances et autant d’expérience. La quantité innombrable de volumes qui ont été écrits sur cet art, dans toutes les langues, prouve bien que les nations les plus sages et les plus éclairées ne l’ont jamais regardé comme un sujet de facile étude. Et nous aurions peine encore à trouver dans tous ces volumes ensemble autant de connaissances sur les opérations si diverses et si compliquées de cette profession, qu’en possède communément un cultivateur même ordinaire, malgré tout le dédain avec lequel affectent de parler de lui certains auteurs inconsidérés qui ont écrit sur cette matière. Au contraire, il n’y a presque pas une profession mécanique ordinaire dont on ne puisse expliquer toutes les opérations dans une brochure de quelques pages, aussi complètement et aussi clairement que de pareilles choses peuvent se rendre à l’aide du discours et des figures. Il y en a plusieurs qui sont expliquées de cette manière dans l’Histoire des Arts et Métiers, publiée par l’Acadé­mie des sciences de France. En outre, il faut bien plus de jugement et de prudence pour diriger des opérations qui doivent varier à chaque changement de saison, et suivant une infinité d’autres circonstances, que pour des travaux qui sont toujours les mêmes ou à peu près les mêmes.

Non-seulement l’art du cultivateur, qui consiste dans la direction générale des opérations de la culture, mais même plusieurs des branches inférieures des travaux de la campagne, exigent beaucoup plus de savoir et d’expérience que la majeure partie des arts mécaniques. Un homme qui travaille sur le cuivre ou sur le fer travaille avec des outils et sur des matières dont la nature est toujours la même ou à peu près ; mais celui qui laboure la terre avec un attelage de chevaux ou de bœufs travaille avec des instruments dont la santé, la force et le tempérament sont très-différents, selon les di­ver­ses circonstances. La nature des matériaux sur lesquels il travaille n’est pas moins sujette à varier que celle des instruments dont il se sert, et les uns et les autres veulent être maniés avec beaucoup de jugement et de prudence ; aussi est­-il rare que ces qua­lités manquent à un simple laboureur, quoiqu’on le prenne, en général, pour un mo­dèle de stupidité et d’ignorance. À la vérité, il est moins accoutumé que l’artisan au commerce de la société ; son langage et le son de sa voix ont quelque chose de plus grossier et de plus choquant pour ceux qui n’y sont pas accoutumés ; toutefois, son intelligence, habituée à s’exercer sur une plus grande variété d’objets, est en général bien supérieure à celle de l’autre, dont toute l’attention est ordinairement du matin au soir bornée à exécuter une ou deux opérations très-simples. Tout homme qui, par relation d’affaires ou par curiosité, a un peu vécu avec les dernières classes du peuple de la campagne et de la ville, connaît très-bien la supériorité des unes sur les autres[33]. Aussi dit-on qu’à la Chine et dans l’Indostan, les ouvriers de la campagne sont mieux traités, pour la considé­ration et les salaires, que la plupart des artisans et ouvriers de manufactures. Il en serait probablement de même partout, si les lois et l’esprit de corporation n’y mettaient obstacle.

Ce n’est pas seulement aux corporations et à leurs règlements qu’il faut attribuer la supériorité que l’industrie des villes a usurpée dans toute l’Europe sur celle des cam­pa­gnes, il y a encore d’autres règlements qui la maintiennent ; les droits élevés dont sont chargés tous les produits de manufacture étrangère et toutes les mar­chan­dises importées par des marchands étrangers, tendent tous au même but[34]. Les lois de corpo­ration mettent les habitants des villes à même de hausser leurs prix, sans crainte d’être supplantés par la libre concurrence de leurs concitoyens ; les autres règlements les garantissent de celle des étrangers. Le renchérissement de prix qu’occasionnent ces deux espèces de règlements est partout supporté, en définitive, par les propriétaires, les fermiers et les ouvriers de la campagne, qui se sont rarement opposés à l’éta­blissement de ces monopoles. Ordinairement, ils n’ont ni le désir ni les moyens de se concerter entre eux pour de pareilles mesures[35] ; les marchands, par leurs clameurs et leurs raisonnements captieux, viennent aisément à bout de leur faire prendre pour l’intérêt général ce qui n’est que l’intérêt privé d’une partie, et encore d’une partie subordonnée de la société[36].

Il paraît qu’anciennement, dans la Grande-Bretagne, l’industrie des villes avait sur celle des campagnes plus de supériorité qu’à présent ; aujourd’hui, les salaires du tra­vail de la campagne se rapprochent davantage de ceux du travail des manufactures, et les profits des capitaux employés à la culture, de ceux des capitaux employés au commerce et aux manufactures, plus qu’ils ne s’en rapprochaient, à ce qu’il semble, dans le dernier siècle ou dans le commencement de celui-ci. Ce changement peut être regardé comme la conséquence nécessaire, quoique très-tardive, de l’encouragement forcé donné à l’industrie des villes. Le capital qui s’y accumule devient, avec le temps, si considérable, qu’il ne peut plus y être employé avec le même profit à cette espèce d’industrie qui est particulière aux villes ; cette industrie a ses limites comme toute autre, et l’accroissement des capitaux, en augmentant la concurrence, doit nécessaire­ment réduire les profits. La baisse des profits dans les villes force les capitaux à refluer dans les campagnes, où ils vont créer de nouvelles demandes de travail, et font hausser, par conséquent, les salaires du travail agricole ; alors ces capitaux se répandent, pour ainsi dire, sur la surface du sol, et l’emploi qu’on en fait en culture les rend en partie à la campagne, aux dépens de laquelle ils s’étaient originairement accu­mulés dans les villes. Je tâcherai de faire voir, par la suite[37], que c’est à ces déborde­ments de capitaux ordinairement accumulés dans les villes que, dans toute l’Europe, on est redevable des plus grandes améliorations faites dans la culture du pays, et je tâcherai de démontrer en même temps que, quoique ce cours des choses ait amené quelques pays à un degré considérable d’opulence, néanmoins une telle marche est nécessairement par elle-même lente, incertaine, sujette à être interrompue par une foule innombrable d’accidents, et qu’elle est à tous égards contraire à l’ordre de la nature et de la raison. Dans les troisième et quatrième livres de ces Recherches, je tâcherai de développer, avec autant de clarté et d’étendue qu’il me sera possible, quels sont les intérêts, les préjugés, les lois et coutumes qui ont donné lieu à ce fait.

Il est rare que des gens du même métier se trouvent réunis, fût-ce pour quelque partie de plaisir ou pour se distraire, sans que la conversation finisse par quelque conspiration contre le public, ou par quelque machination pour faire hausser les prix[38]. Il est impossible, à la vérité, d’empêcher ces réunions par une loi qui puisse s’exé­cuter, ou qui soit compatible avec la liberté et la justice ; mais si la loi ne peut pas empêcher des gens du même métier de s’assembler quelquefois, au moins ne devrait-elle rien faire pour faciliter ces assemblées, et bien moins encore pour les rendre nécessaires.

Un règlement qui oblige tous les gens du même métier, dans une ville, à faire inscrire sur un registre public leurs noms et demeures, facilite ces assemblées ; il établit une liaison entre des individus qui autrement ne se seraient peut-être jamais connus, et il donne à chaque homme de métier une indication pour trouver toutes les autres personnes de sa profession.

Un règlement qui autorise les gens du même métier à se taxer entre eux pour pourvoir au soulagement de leurs pauvres, de leurs malades, de leurs veuves et orphe­lins, en leur donnant ainsi des intérêts communs à régir, rend ces assemblées nécessaires.

Une corporation rend non-seulement les assemblées nécessaires, mais elle fait encore que la totalité des membres se trouve liée par le fait de la majorité. Dans un métier libre, on ne peut former de ligue qui ait son effet, que par le consentement unanime de chacun des individus de ce métier, et encore cette ligue ne peut-elle durer qu’autant que chaque individu continue à être du même avis. Mais la majorité d’un corps de métier peut établir un statut, avec des dispositions pénales, qui limitera la concurrence d’une manière plus efficace et plus durable que ne pourrait faire aucune ligue volontaire quelconque.

C’est sans le moindre fondement qu’on a prétexté que les corporations étaient nécessaires pour régir sagement l’industrie. La discipline véritable et efficace qui s’exer­ce sur un ouvrier, ce n’est pas celle de la corporation, mais bien celle de ses prati­ques. C’est la crainte de perdre l’ouvrage qu’elles lui donnent qui prévient ses fraudes et corrige sa négligence. Une corporation exclusive diminue nécessairement la force de cette discipline. On vous oblige alors d’employer une classe particulière de gens, qu’ils se comportent bien ou mal. C’est pour cette raison que, dans plusieurs grande villes de corporation, on ne trouve quelquefois pas d’ouvriers passables, même dans les métiers les plus indispensables. Si vous voulez avoir de l’ouvrage fait avec quelque soin, il faut le commander dans les faubourgs, où les ouvriers, n’ayant pas de privilège exclusif, ne peuvent compter que sur la bonne réputation qu’ils se font, et ensuite il faut l’introduire en contrebande dans la ville.

C’est ainsi que la police des pays de l’Europe, en restreignant dans quelques loca­li­tés la concurrence à un plus petit nombre de personnes que celui qui s’y serait porté sans cela, donne lieu à une inégalité très-considérable dans la somme totale des avantages et désavantages des divers emplois du travail et des capitaux.

Secondement, la police des pays de l’Europe, en augmentant la concurrence dans quelques emplois au-delà de ce qu’elle serait naturellement, occasionne une inégalité d’une espèce contraire dans la somme totale des avantages et désavantages des diffé­rents emplois du travail et des capitaux.

On a regardé comme une chose de si grande importance qu’il y eût un nombre convenable de jeunes gens élevés dans certains professions, qu’il a été institué dans cette vue, tantôt par l’État, tantôt par la piété de quelques fondateurs particuliers, une quantité de pensions, de bourses, de places dans les collèges et séminaires, etc., qui attirent dans ces professions beaucoup plus de gens qu’il n’y en aurait sans cela. je crois que, dans tous les pays chrétiens, l’éducation de la plupart des ecclésiastiques est défrayée de cette manière. Il y en a très-peu parmi eux qui aient été élevés entière­ment à leurs propres frais ; ceux qui sont dans ce cas ne trouveront donc pas toujours une récompense proportionnée à une éducation qui exige tant de temps, d’études et de dépense, les emplois ecclésiastiques étant obsédés par une foule de gens qui, pour se procurer de l’occupation, sont disposés à accepter une rétribution fort au-dessous de celle à laquelle ils auraient pu prétendre sans cela, avec une pareille éducation ; et ainsi la concurrence du pauvre emporte la récompense du riche. Sans doute, il ne serait pas convenable de comparer un curé ou un chapelain à un artisan à la journée. On peut bien pourtant, sans blesser les convenances, considérer les honoraires d’un curé ou d’un chapelain comme étant de la même nature que les salaires de cet artisan. Tous les trois sont payés de leur travail en vertu de la convention qu’ils ont-faite avec leurs supérieurs respectifs. Or, jusques après le milieu du quatorzième siècle, les honoraires ordinaires d’un curé ou d’un prêtre gagé dans une paroisse, en Angleterre, ont été de 5 marcs d’argent (contenant environ autant d’argent que 10 livres de notre monnaie actuelle), ainsi que nous le trouvons réglé par les décrets de plusieurs con­ciles nationaux. À la même époque, il est déclaré que la paye d’un maître maçon est de 4 deniers par jour, contenant la même quantité d’argent qu’un schelling de notre monnaie actuelle, et celle d’un compagnon maçon, de 3 deniers par jour, égaux à 9 d’aujourd’hui[39]. Ainsi, les salaires de ces ouvriers, en les supposant constamment employés, étaient fort au-dessus des honoraires du curé ; et en supposant le maître maçon sans ouvrage pendant un tiers de l’année, ses salaires étaient encore aussi élevés que ces honoraires. Dans le statut de la douzième année de la reine Anne, chapitre xii, il est dit : « Qu’attendu qu’en plusieurs endroits les cures ont été mal desservies faute de fonds suffisants pour entretenir et encourager les curés, l’évêque sera autorisé à leur allouer, par acte revêtu de sa signature et de son sceau, des émoluments ou une rétribution fixe et suffisante, qui n’excède pas 50 livres, et qui ne soit pas au-dessous de 20 livres par an. » On regarde aujourd’hui 40 livres par an comme une rétribution suffisante pour un curé[40], et malgré cet acte du Parlement, il y a beaucoup de cures au-dessous de 20 livres[41]. Or, il y a à Londres des cordonniers à la journée qui gagnent 40 livres par an, et il n’y a presque pas un ouvrier laborieux, de quelque genre que ce soit, dans cette capitale, qui n’en gagne plus de 20. Cette dernière somme n’excède même pas ce que gagnent très-souvent de simples manœu­vres dans plusieurs paroisses de campagne, Toutes les fois que la loi a cherché à régler les salaires des ouvriers, c’est toujours pour les faire baisser plutôt que pour les élever ; mais en maintes occasions la loi a tâché d’élever les honoraires des curés, en obligeant les recteurs des paroisses, pour maintenir la dignité de l’Église, à leur donner quelque chose de plus que la misérable subsistance qu’ils se seraient volon­tiers soumis à accepter. Dans l’un comme dans l’autre cas, la loi a également manqué son but, et elle n’a jamais eu le pouvoir d’élever le salaire des curés, non plus que d’abaisser celui des ouvriers jusqu’au degré qu’elle s’était proposé, parce qu’elle n’a jamais pu empêcher que les premiers, vu leur état d’indigence et la multitude des concurrents, ne consentissent à accepter moins que la rétribution fixée par la loi ; ni que les autres, vu la concurrence contraire de ceux qui trouvent leur profit et leur plaisir à les employer, ne reçussent davantage.

Les grands bénéfices et les autres honneurs ecclésiastiques soutiennent la dignité de l’Église, malgré la chétive condition de ses membres inférieurs. La considération que l’on porte à cette profession, même pour ces derniers, dédommage de la modicité de leur récompense pécuniaire. En Angleterre et dans tous les pays catholiques ro­mains, la chance de fortune dans l’Église est, en réalité, plus avantageuse qu’il ne le faut. Il suffit de l’exemple des Églises d’Écosse, de Genève et de plusieurs autres de la communion protestante, pour nous convaincre que dans une profession aussi recom­mandable, où l’on a tant de facilité pour se procurer l’éducation nécessaire, la seule perspective de bénéfices beaucoup plus modiques attirerait dans les ordres sacrés un nombre suffisant d’hommes instruits, bien nés et respectables[42].

Si l’on élevait proportionnellement une aussi grande quantité de personnes aux frais du public, dans les professions où il n’y a pas de bénéfices, telles que le droit et la médecine, la concurrence y serait bientôt si grande, que la récompense pécuniaire baisserait considérablement : personne alors ne voudrait prendre la peine de faire élever son fils à ses dépens dans l’une ou l’autre de ces professions. Elles seraient aban­données uniquement à ceux qui y auraient été préparés par cette espèce de chari­té publique, et ces deux professions, aujourd’hui si honorées, seraient tout à fait dé­gra­­dées par la misérable rétribution dont ces élèves si nombreux et si indigents se verraient en général forcés de se contenter. 

La classe d’hommes peu fortunés qu’on appelle communément gens de lettres est à peu près dans la même position que celle où se trouveraient probablement les juri­consultes et les médecins, dans la supposition ci-dessus. La plupart d’entre eux, dans toutes les parties de l’Europe, ont été élevés pour l’Église, mais ils ont été détournés, par différentes raisons, d’entrer dans les ordres. Ils ont donc, en général, reçu leur édu­cation aux frais du public, et leur nombre est partout trop grand pour que le prix de leur travail ne soit pas réduit communément à la plus mince rétribution.

Avant l’invention de l’imprimerie, les gens de lettres n’avaient d’autre emploi, pour tirer parti de leurs talents, que celui d’enseigner publiquement, ou de communiquer a d’autres les connaissances utiles et curieuses qu’ils avaient acquises ; et cet emploi est encore assurément plus utile, plus honorable et même, en général, plus lucratif que celui d’écrire pour des libraires, emploi auquel l’imprimerie a donné naissance. Le temps et l’étude, le talent, le savoir et l’application nécessaires pour former un profes­seur distingué dans les sciences sont au moins équivalents à ce qu’en possèdent les premiers praticiens en médecine et en jurisprudence ; mais la rétribution ordinaire d’un savant professeur est, sans aucune proportion, au-dessous de celle d’un bon avocat ou d’un bon médecin, parce que la profession du premier est surchargée d’une foule d’indigents qui ont été instruits aux dépens du public, tandis que dans les deux autres il n’y a que très-peu d’élèves qui n’aient pas fait eux-mêmes les frais de leur éducation. Cependant, toute faible qu’elle est, la récompense ordinaire des professeurs publics et particuliers serait indubitablement beaucoup au-dessous même de ce qu’elle est, s’ils ne se trouvaient débarrassés de la concurrence de cette portion plus indigente encore de gens de lettres qui écrivent pour avoir du pain. Avant l’invention de l’imprimerie, étudiant et mendiant étaient, à ce qu’il semble, des termes à peu près synonymes, et il paraît qu’avant cette époque les différents recteurs des universités ont souvent accordé à leurs écoliers la permission de mendier.

Dans l’Antiquité, où l’on n’avait aucun de ces établissements charitables destinés à élever des personnes indigentes dans les professions savantes, les professeurs étaient, à ce qu’il semble, bien plus richement récompensés. Isocrate, dans son Discours contre les sophistes, reproche aux professeurs de son temps leur inconséquence. « Ils font à leurs écoliers, dit-il, les promesses les plus magnifiques ; ils se chargent de leur enseigner à être sages, à être heureux, à être justes, et en retour d’un service d’une telle importance, ils stipulent une misérable récompense de 4 ou 5 mines. Ceux qui ensei­gnent la sagesse, continue-t-il, devraient certainement être sages eux-mêmes ; cepen­dant, si l’on voyait un homme vendre à si bas prix une telle marchandise, il serait convaincu de la folie la plus manifeste. » Sans doute, il n’entend pas ici exagérer le montant de la rétribution, et nous pouvons être bien sûrs qu’elle n’était pas moindre qu’il ne la représente. Quatre mines étaient égales à 13 livres 6 sous 8 deniers[43] ; 5 mi­nes, à 16 livres 13 sous 4 deniers ; ainsi, il faut que, dans ce temps, on ne payât guère moins que la plus forte de ces deux sommes aux premiers professeurs d’Athènes. Isocrate lui-même exigeait de chacun de ses élèves 10 mines, ou 33 livres 6 sous 8 deniers. Quand il enseignait à Athènes, on dit qu’il avait une centaine d’écoliers. J’entends par là le nombre auquel il enseignait à la fois, ou ceux qui assistaient à ce que nous appellerions un cours de leçons, et ce nombre ne paraîtra pas extraordinaire dans une si grande ville pour un professeur aussi célèbre, et qui enseignait celle de toutes les sciences qui était alors le plus en vogue, la rhétorique. Il faut donc que chacun de ses cours lui ait valu 1,000 mines, ou 3,333 livres 6 sous 8 deniers. Aussi, Plutarque nous dit-il ailleurs que 1,000 mines étaient son didactron ou le revenu ordi­naire de son école. Beaucoup d’autres grands professeurs de ces temps-là paraissent avoir fait des fortunes considérables. Gorgias fit présent au temple de Delphes de sa propre statue en or massif ; cependant il ne faut pas, à ce que je crois, la supposer de grandeur naturelle. Son genre de vie, aussi bien que celui d’Hippias et de Protagoras, deux autres professeurs distingués du même temps, nous est représenté par Platon comme d’un luxe qui allait jusqu’à l’ostentation. On dit que Platon lui-même vivait d’une manière très-somptueuse. Aristote, après avoir été le précepteur d’Alexandre et avoir été magnifiquement récompensé, comme chacun sait, tant par ce prince que par Philippe, trouva que les leçons de son école valaient bien encore la peine qu’il revînt à Athènes pour les reprendre. Les professeurs des sciences étaient probablement moins communs à cette époque qu’ils ne le devinrent un siècle ou deux après, lorsque la concurrence eut sans doute diminué le prix de leur travail et l’admiration qu’on avait pour leurs personnes. Cependant, les premiers d’entre eux paraissent toujours avoir joui d’un degré de considération bien supérieur à tout ce que pourrait espérer aujour­d’hui un homme de cette profession. Les Athéniens envoyèrent en ambassade solen­nelle à Rome Carnéade l’académicien et Diogène le stoïcien ; et quoique leur ville fût alors déchue de sa première grandeur, c’était encore une république considérable et indépendante. Carnéade, d’ailleurs, était Babylonien de naissance ; et comme jamais aucun peuple ne se montra plus jaloux que les Athéniens d’écarter les étrangers des emplois publics, il faut que leur considération pour lui ait été très-grande.

En somme, cette inégalité est peut-être plus avantageuse que nuisible au public. Elle tend bien à dégrader un peu la profession de ceux qui s’adonnent à l’enseigne­ment ; mais ce léger inconvénient est à coup sûr grandement contre-balancé par l’avan­tage qui résulte du bon marché de l’éducation littéraire. Cet avantage serait encore d’une bien autre importance pour le public, si la constitution des collèges et des maisons d’éducation était plus raisonnable qu’elle ne l’est aujourd’hui dans la plus grande partie de l’Europe[44].

Troisièmement, la police des pays de l’Europe, en gênant la libre circulation du travail et des capitaux, tant d’un emploi à un autre que d’un lieu à un autre, occasionne en certains cas une inégalité fort nuisible dans la somme totale des avantages de leurs différents emplois.

Les statuts d’apprentissage gênent la libre circulation du travail d’un emploi à un autre, même dans le même lieu. Les privilèges exclusifs des corporations la gênent d’un lieu à un autre, même dans le même emploi.

Il arrive fréquemment que, tandis que des ouvriers gagnent de gros salaires dans une manufacture, ceux d’une autre sont obligés de se contenter de la simple subsis­tance. L’une sera dans un état d’avancement et, par conséquent, demandera sans cesse de nouveaux bras ; l’autre sera dans un état de décadence, et les bras y devien­dront de plus en plus surabondants. Ces deux manufactures seront quelquefois dans la même ville, quelquefois dans le même voisinage, sans pouvoir se prêter l’une à l’autre la moindre assistance. L’obstacle qui s’y oppose peut résulter de la loi d’apprentissage dans un cas ; il peut résulter, dans l’autre, et de cette loi et de l’institution des corpo­rations exclusives. Cependant, dans plusieurs manufactures différentes, les opérations ont tant d’analogie, que les ouvriers pourraient aisément changer de métier les uns avec les autres, si ces lois absurdes n’y mettaient empêchement. Par exemple, l’art de tisser la toile unie et celui de tisser les étoffes de soie sont presque entièrement la même chose. Celui de tisser la laine en uni est un peu différent ; mais la différence est si peu de chose, qu’un tisserand, soit en toile, soit en soie, y deviendrait en quelques jours un ouvrier passable. Si l’une de ces trois manufactures capitales venait à dé­choir, les ouvriers pourraient trouver une ressource dans l’une des deux autres qui serait dans un état de prospérité et, de cette manière, leurs salaires ne pourraient ja­mais s’élever trop haut dans l’industrie en progrès, ni descendre trop bas dans l’industrie en décadence. À la vérité, les manufactures de toile, en Angleterre, par un statut particulier, sont ouvertes à tout le monde ; mais, comme ce genre n’est pas très-cultivé dans une grande partie du pays, il ne peut pas fournir une ressource générale aux ouvriers des autres manufactures en déclin ; partout où la loi de l’apprentissage est en vigueur, ces ouvriers n’ont donc d’autre parti à prendre que de se mettre à la charge de la paroisse, ou de travailler comme simples manœuvres, ce à quoi ils sont bien moins propres, par leurs habitudes, qu’à tout autre genre d’industrie qui aurait quelque rapport avec leur métier ; aussi, en général, ils préfèrent se mettre à la charge de la paroisse.

Tout ce qui gêne la libre circulation du travail d’un emploi à un autre gêne pareil­lement celle des capitaux, la quantité de fonds qu’on peut verser dans une branche de commerce dépendant beaucoup de celle du travail qui peut y être employé. Cepen­dant, les lois des corporations apportent moins d’obstacles à la libre circulation des capitaux d’un lieu à un autre, qu’à celle du travail. Partout un riche marchand trouvera plus de facilité pour obtenir le privilège de s’établir dans une ville de corporation, qu’un pauvre artisan pour avoir la permission d’y travailler.

La gêne que les lois des corporations apportent à la libre circulation du travail est, je pense, commune à tous les pays de l’Europe ; celle qui résulte des lois sur les pau­vres est, d’autant que je puis le savoir, particulière à l’Angleterre. Elle vient de la diffi­culté qu’un homme pauvre trouve à obtenir un domicile (settlement)[45], ou même la permission d’exercer son industrie dans une autre paroisse que celle à laquelle il appartient. Les lois des corporations ne gênent que la libre circulation du travail des artisans et ouvriers de manufacture seulement ; la difficulté d’obtenir un domicile gêne jusqu’à la circulation du travail de simple manœuvre. Il ne sera pas hors de propos de donner à ce sujet quelques éclaircissements sur l’origine, le progrès et l’état actuel de ce mal, l’un des plus fâcheux peut-être de l’administration de l’Angleterre.

Lors de la destruction des monastères, quand les pauvres furent privés des secours charitables de ces maisons religieuses, après quelques tentatives infructueuses pour leur soulagement, le statut de la quarante-troisième année d’Élisabeth, chapitre II, ré­gla que chaque paroisse serait tenue de pourvoir à la subsistance de ses pauvres, et qu’il y aurait des inspecteurs des pauvres établis annuellement, lesquels, conjointe­ment avec les marguilliers, lèveraient, par une taxe paroissiale, les sommes suffi­san­tes pour cet objet.

Ce statut imposa à chaque paroisse l’obligation indispensable de pourvoir à la subsistance de ses pauvres. Ce fut donc une question de quelque importance, de sa­voir quels étaient les individus que chaque paroisse devait regarder comme ses pauvres. Après quelques variations, cette question fut enfin décidée dans les treizième et quatorzième années de Charles II, où il fut statué qu’une résidence non contestée de quarante jours ferait acquérir le domicile dans une paroisse, mais que, pendant ce terme, deux juges de paix pourraient, sur la réclamation des marguilliers ou inspec­teurs des pauvres, renvoyer tout nouvel habitant à la paroisse sur laquelle il était légalement établi en dernier lieu, à moins que cet habitant ne tînt à loyer un bien de 10 livres de revenu annuel, ou bien qu’il ne fournît, pour la décharge de la paroisse où il était actuellement résident, une caution fixée par ces juges.

On dit que ce statut donna lieu à quelques fraudes, les officiers de paroisse ayant quelquefois engagé par connivence leurs propres pauvres à aller clandestinement dans une autre paroisse, et à s’y tenir cachés pendant les quarante jours pour y gagner le domicile, à la décharge de la paroisse à laquelle ils appartenaient réellement. En consé­quence, il fut établi par le statut de la première année de Jacques II, que les qua­rante jours de résidence non contestée exigés pour gagner le domicile ne commen­ce­raient à courir que du jour où le nouveau résident aurait donné à l’un des marguilliers ou inspecteurs de la paroisse où il venait habiter, une déclaration par écrit du lieu de sa demeure et du nombre d’individus dont sa famille était composée.

Mais les officiers de paroisse, à ce qu’il paraît, furent quelquefois aussi peu scru­puleux à l’égard de leurs propres paroisses qu’à l’égard des autres paroisses, et ils prêtèrent la main à ces intrusions en recevant la déclaration, sans faire ensuite aucune des démarches qu’il convenait de faire. En conséquence, comme on supposa que chaque habitant avait intérêt d’empêcher, autant qu’il était en lui, l’admission de ces intrus qui augmentaient la charge de la paroisse, le statut de la troisième année de Guillaume III ajouta aux précédentes dispositions, que le terme de quarante jours de résidence ne courrait que de la date de la publication faite le dimanche à l’église, immédiatement après le service divin, de la déclaration donnée par écrit.

« Après tout, dit le docteur Burn, cette espèce de domicile, par une résidence de quarante jours continuée après la publication de la déclaration par écrit, s’obtient fort rarement, et le but de la loi est bien moins de faire gagner les droits de domicile que d’annuler ceux des personnes qui s’introduisent clandestinement dans une paroisse ; car donner une déclaration, c’est seulement mettre la paroisse dans la nécessité de vous renvoyer. Mais lorsque la position de la personne est telle qu’il est incertain si elle est actuellement dans le cas de renvoi ou non, en donnant sa déclaration elle forcera la paroisse ou à lui accorder le domicile sans contestation, en lui laissant continuer ses quarante jours, ou à faire juger la chose en lui signifiant son renvoi. »

Ainsi, ce statut rendit à peu près impraticable pour les pauvres l’ancienne voie de gagner le domicile par quarante jours d’habitation. Mais, pour ne pas paraître ôter tout à fait aux gens d’une paroisse la possibilité de jamais s’établir tranquillement sur une autre, ce statut ouvrit quatre autres voies par lesquelles on pourrait gagner le domicile sans déclaration par écrit ni publication. La première fut d’être porté sur les rôles de la paroisse et de payer la taxe ; la deuxième, d’être élu à un des emplois annuels de la paroisse et de l’exercer pendant un an ; la troisième, de faire son temps d’apprentissage dans la paroisse ; la quatrième, d’y être engagé pour servir un an, et de rester au même service pendant tout ce temps.

Ce n’est qu’un acte public de la paroisse entière qui peut faire gagner le domicile par les deux premières voies ; en effet, lorsqu’un nouveau venu n’a que son travail pour subsister, la paroisse connaît trop les conséquences qui en résulteraient, pour consentir à l’adopter, soit en l’imposant aux taxes paroissiales, soit en le nommant à un office.

Un homme marié ne peut guère gagner le domicile par les deux dernières voies. Un apprenti est presque toujours garçon, et il est expressément statué qu’aucun do­mes­tique marié ne pourra gagner le domicile en s’engageant pour un an au service de quelqu’un. Le principal effet qu’ait produit l’introduction de cette voie de gagner le domicile par service a été de détruire en grande partie l’ancienne méthode de louer les domestiques pour une année, méthode auparavant si ordinaire en Angleterre que, même encore aujourd’hui, quand il n’y a pas de terme particulier de convenu, la loi suppose que tout domestique est loué pour l’année. Mais les maîtres ne sont pas tou­jours dans l’intention de donner le domicile à leurs domestiques en les louant de cette manière ; et les domestiques, de leur côté, ne sont pas non plus toujours d’avis de se louer ainsi, parce que, le dernier domicile emportant déchéance de tous les précé­dents, ils pourraient perdre par là leur domicile originaire dans le lieu de leur nais­sance, ou de celui où résident leurs parents et leur famille.

Il est bien évident qu’un ouvrier indépendant, soit manœuvre, soit artisan, ne voudra jamais gagner le domicile par apprentissage ni par service. Aussi, quand un de ces ouvriers venait porter son industrie dans une nouvelle paroisse, il était sujet, quelque bien portant et laborieux qu’il pût être, à être renvoyé, selon le bon plaisir d’un marguillier ou d’un inspecteur, à moins qu’il ne tint un loyer de 10 livres par année, chose impossible à un ouvrier qui n’a que son travail pour vivre, ou bien qu’il ne pût fournir pour la décharge de la paroisse une caution, à l’arbitrage de deux juges de paix. Le montant de cette caution est, à la vérité, laissé entièrement à leur pruden­ce, mais ils ne peuvent guère l’exiger au-dessous de 30 livres, puisqu’il a été statué que l’acquisition, même en pleine propriété, d’un bien valant moins de 30 livres, ne pourrait faire gagner le domicile, cette somme n’étant pas suffisante pour la décharge de la paroisse. Or, c’est encore une caution que ne pourrait presque jamais fournir un homme vivant de son travail, et très-souvent on en exige une beaucoup plus forte.

Pour rétablir donc en quelque sorte la libre circulation du travail, que ces diffé­rents statuts avaient presque totalement détruite, on imagina les certificats. Dans les huitième et neuvième années de Guillaume III, il fut statué que lorsqu’une personne aurait obtenu de la paroisse où elle avait son dernier domicile légal un certificat signé des marguilliers et inspecteurs des pauvres, et approuvé par deux juges de paix, tout autre paroisse serait tenue de la recevoir ; qu’elle ne pourrait être renvoyée sur le simple prétexte qu’elle était dans le cas de devenir à la charge de la paroisse, mais seulement pour le fait d’y être actuellement à charge, auquel cas la paroisse qui avait accordé le certificat serait tenue de rembourser tant la subsistance du pauvre que les frais de son renvoi. Et à l’effet de donner à la paroisse, sur laquelle le porteur d’un tel certificat venait demeurer, la sûreté la plus complète, il fut réglé de plus, par le même statut, que ce porteur de certificat ne pourrait y gagner le domicile par quelque voie que ce fût, excepté celle de tenir un loyer de 10 livres par an, ou de remplir person­nellement, pendant une année entière, un des offices annuels de la paroisse ; en consé­quence, cette personne ne pouvait pas gagner le domicile par déclaration, ni par service, ni par apprentissage, ni par le payement des taxes. Il fut même encore statué, dans la douzième année de la reine Anne, statut Ier[46], chap. xviii, que les domestiques et les apprentis du porteur d’un tel certificat ne pourraient gagner aucun droit de domicile dans la paroisse où celui-ci demeurerait à la faveur de ce certificat.

Une observation fort judicieuse du docteur Burn peut nous apprendre jusqu’à quel point l’invention des certificats a rétabli cette libre circulation du travail, presque entièrement anéantie par les statuts précédents. « Il est évident, dit-il, qu’il y a plu­sieurs bonnes raisons pour exiger des certificats des personnes qui viennent s’établir dans un endroit : d’abord, c’est afin que celles qui résident à la faveur de ces certificats ne puissent gagner le domicile, ni par apprentissage, ni par service, ni par déclaration, ni par le payement des taxes ; c’est afin qu’elles ne puissent donner le domicile ni à leurs apprentis, ni à leurs domestiques ; c’est afin que, si elles deviennent à la charge de la paroisse, on sache où on doit les renvoyer, et que la paroisse soit remboursée de la dépense du renvoi et de celle de leur subsistance pendant ce temps ; enfin, que si elles tombent malades de manière à ne pouvoir être transportées, la paroisse qui a donné le certificat soit tenue de les entretenir ; toutes choses qui ne pourraient avoir lieu sans la formalité du certificat. Ces raisons, d’un autre côté, seront à proportion tout aussi puissantes pour empêcher les paroisses d’accorder des certificats dans les cas ordinaires ; car il y a une chance infiniment plus qu’égale pour que les porteurs de leurs certificats leur reviennent, et encore dans une condition pire. » Le sens de cette observation, à ce qu’il semble, c’est que la paroisse où un homme pauvre vient demeurer devrait toujours exiger le certificat, et que celle qu’il se propose de quitter ne devrait presque jamais en accorder. « Il y a quelque chose de révoltant dans cette institution », dit encore ce judicieux auteur, dans son Histoire de la législation des pauvres, « c’est d’attribuer à un officier de paroisse le pouvoir de tenir ainsi un hom­me, pour toute sa vie, dans une espèce de prison, quelque inconvénient qu’il puisse y avoir pour lui à rester dans l’endroit où il aura eu le malheur de gagner ce qu’on appelle un domicile, ou quelque avantage qu’il puisse trouver à aller vivre ailleurs[47]. »

Quoiqu’un certificat n’emporte avec soi aucune attestation de bonne conduite et ne certifie autre chose, sinon que la personne appartient à la paroisse à laquelle elle appartient réellement, cependant il est absolument laissé à l’arbitraire des officiers de paroisse de l’accorder ou de le refuser. On demanda une fois, dit le docteur Burn, une ordonnance de Mandamus pour enjoindre à des marguilliers et inspecteurs de signer[48] un certificat ; mais la requête fut rejetée par la cour du banc du roi[49], comme une prétention très-étrange. 

C’est probablement aux obstacles qu’un pauvre ouvrier trouve, dans la loi du do­mi­cile, à porter son industrie d’une paroisse à une autre sans l’aide d’un certificat, qu’il faut attribuer cette inégalité si forte qu’on remarque fréquemment en Angleterre dans les prix du travail, à des endroits assez peu distants l’un de l’autre. Un ouvrier garçon qui est bien portant et laborieux pourra quelquefois résider, par tolérance, sans cette formalité ; mais un homme ayant femme et enfants, qui se risquerait à le faire, serait sûr, dans la plupart des paroisses, d’être renvoyé et, en général, il en serait de même du garçon s’il venait par la suite à se marier ; ainsi, la disette de bras dans une paroisse ne peut pas toujours être soulagée par la surabondance dans une autre, comme cela se fait constamment en Écosse et, je présume, dans tous les pays où il n’existe pas d’entraves à la liberté de s’établir. Dans ces pays-là, quoique les salaires s’y élèvent quelquefois un peu dans le voisinage des grandes villes et partout ailleurs où il y a demande extraordinaire de travail, ils baissent ensuite par degrés à mesure que la distance de ces endroits vient à augmenter, jusqu’à ce qu’ils retombent au taux ordi­nai­re des campagnes ; mais nous n’y rencontrons jamais ces différences tranchantes et inexplicables que nous remarquons quelquefois dans les salaires de deux places voisines en Angleterre, où les barrières artificielles d’une paroisse sont bien plus difficiles à franchir pour un pauvre ouvrier, que des limites naturelles, telles qu’un bras de mer ou une chaîne de montagnes qui forment ailleurs une démarcation très-sensible entre les différents taux des salaires.

C’est un attentat manifeste contre la justice et la liberté naturelles, que de renvoyer un homme qui n’est coupable d’aucun délit de la paroisse où il choisit de demeurer ; cependant le peuple, en Angleterre, qui est si jaloux de sa liberté, mais qui, comme le peuple de la plupart des autres pays, n’entend jamais bien en quoi elle consiste, est resté, déjà depuis plus d’un siècle, assujetti à cette oppression sans y chercher de remède. Quoique les gens sages se soient aussi quelquefois plaints de la loi du domi­cile comme d’une calamité publique, néanmoins elle n’a jamais été l’objet d’une réclamation universelle du peuple, comme celle qu’ont occasionnée les Warrants généraux[50] ; pratique sans contredit très-abusive, mais qui pourtant ne peut donner lieu à une oppression générale ; tandis qu’on peut affirmer qu’il n’existe pas en Angleterre un seul pauvre ouvrier, parvenu à l’âge de quarante ans, qui n’ait eu à éprouver, dans un moment ou dans un autre de sa vie, des effets excessivement durs de cette oppressive et absurde loi du domicile[51].

Je terminerai ce long chapitre en observant que s’il a été d’usage anciennement de fixer le taux des salaires, d’abord par des lois générales qui s’étendaient sur la totalité du royaume, et ensuite par des ordonnances particulières des juges de paix pour chaque comté particulier, aujourd’hui ces deux pratiques sont tout à fait tombées en désuétude. « Après une expérience de plus de quatre cents ans, dit le docteur Burn, il est bien temps enfin de ne plus se tourmenter pour assujettir à des règlements précis ce qui, par sa nature, ne paraît être susceptible d’aucune exacte limitation ; car s’il fallait que toutes les personnes reçussent des salaires égaux dans le même genre de travail, il n’y aurait plus d’émulation, et ce serait fermer la voie à l’industrie et au talent. » 

Toutefois, on essaye encore de temps à autre, dans des actes du Parlement, de fixer le taux des salaires dans quelques métiers et dans quelques endroits particuliers ; ainsi, le statut de la huitième année de Georges III défend, sous de graves peines, à tous maîtres tailleurs à Londres et à cinq milles à la ronde, de donner à leurs ouvriers, et à ceux-ci d’accepter plus de 2 s. 7 d. et demi par jour, excepté en cas de deuil public. Toutes les fois que la législature essaye de régler les démêlés entre les maîtres et leurs ouvriers, ce sont toujours les maîtres qu’elle consulte ; aussi, quand le règle­ment est en faveur des ouvriers, il est toujours juste et raisonnable ; mais il en est quelquefois autrement quand il est en faveur des maîtres : ainsi, la loi qui oblige les maîtres, dans plusieurs métiers, à payer leurs ouvriers en argent et non en denrées, est tout à fait juste et raisonnable, elle ne fait aucun tort aux maîtres ; elle les oblige seulement à payer en argent la même valeur que celle qu’ils prétendaient payer, mais qu’ils ne payaient pas toujours réellement, en marchandises. Cette loi est en faveur des ouvriers, mais celle de la huitième année de Georges III est en faveur des maîtres. Quand les maîtres se concertent entre eux pour réduire les salaires de leurs ouvriers, ordinairement ils se lient, par une promesse ou convention secrète, à ne pas donner plus que tel salaire, sous une peine quelconque. Si les ouvriers faisaient entre eux une ligue contraire de la même espèce, en convenant, sous certaines peines, de ne pas accepter tel salaire, la loi les en punirait très-sévèrement. Si elle agissait avec impar­tialité, elle traiterait les maîtres de la même manière ; mais le statut de la huitième année de Georges III donne force de loi à cette taxation que les maîtres cherchaient quelquefois à établir par des ligues secrètes. Les plaintes des ouvriers semblent par­fai­tement bien fondées, quand ils disent que ce statut met l’ouvrier le plus habile et le plus laborieux sur le même pied qu’un ouvrier ordinaire[52].

Il était aussi d’usage, dans les anciens temps, de chercher à borner les profits des marchands et autres vendeurs, en taxant le prix des vivres et de quelques autres marchandises. La taxe du pain est, autant que je sache, le seul vestige qui reste de cet ancien usage. Partout où il existe une corporation exclusive, il est peut-être à propos de régler le prix des choses de première nécessité ; mais où il n’y en a point, la concur­rence le réglera bien mieux que toutes les taxes possibles. La méthode établie par le statut de la trente-unième année de Georges II, pour régler le prix du pain, ne peut se pratiquer en Écosse, à cause d’une omission à la loi, son exécution dépendant de l’office de clerc du marché[53], qui n’existe pas dans ce pays. On ne remédia à cette omission qu’à la troisième année de Georges III. Le défaut de taxe n’occasionna pas d’inconvénient remarquable, et son établissement, dans un petit nombre d’endroits où elle eut lieu, ne produisit aucun avantage sensible. Il y a pourtant, dans la plus grande partie des villes d’Écosse, une corporation de boulangers qui réclame des privilèges exclusifs, mais ceux-ci ne sont pas, au reste, très-sévèrement observés[54].

J’ai déjà remarqué[55] que la proportion entre les taux différents, tant des salaires que des profits, dans les divers emplois du travail et des capitaux, ne paraissait pas être beaucoup affectée par l’état de richesse ou de pauvreté de la société, par son état crois­sant, stationnaire ou décroissant. Ces révolutions dans la propriété publique ont bien une influence générale sur l’universalité des salaires et des profits ; mais, en définitive, cette influence agit également sur tous, quels que soient les différents emplois. Ainsi, la proportion qui règne entre eux subsiste toujours la même, et aucune de ces révolutions ne doit guère y apporter de changements, au moins pour un temps considérable.
 
 
 
↑ Il s’est opéré de bien grands changements dans la condition relative des diverses classes d’ouvriers. Où trouverait-on aujourd’hui le salaire d’un tailleur inférieur à celui d’un malheureux tisserand ? A. B.
↑ Sur la différence entre le travail d’artisan et celui de manœuvre et journalier, voyez ci-après, p. 136.
↑ Voyez ci-après, même chapitre.
↑ Voyez sa 21e Idylle.
↑ Seconde section de ce chapitre.
↑ Il n’est pas besoin d’avertir que les lois qui réglaient le temps et les conditions de l’apprentissage n’existent plus aujourd’hui, du moins en France.
↑ Voyez la 2e section de ce chapitre.
↑ Tout cela pouvait être vrai en Angleterre, à l’époque d’Adam Smith ; mais aujourd’hui ces principes souffrent de nombreuses exceptions, surtout en France, où l’agriculture tend à se relever de son infériorité, depuis que le travailleur agricole peut être propriétaire.
↑ Ce taux du strict nécessaire ne varie pas uniquement à raison du genre de vie plus ou moins passable de l’ouvrier et de sa famille, mais encore à raison de toutes les dépenses regardées comme indispensables dans le pays où il vit. C’est ainsi que l’on met d’abord au rang des dépenses nécessaires celle d’élever des enfants ; il en est d’autres moins impérieusement commandées par la nature des choses, quoiqu’elles le soient au même degré par le sentiment : tel est le soin des vieillards. Dans la classe ouvrière, il est trop négligé. La nature, pour perpétuer le genre humain, ne s’en est rapportée qu’aux impulsions d’un appétit violent, et aux sollicitudes de l’amour paternel ; les vieillards dont elle n’a plus besoin, elle les abandonne à la reconnaissance de leur postérité, après les avoir rendus victimes de l’imprévoyance de leur âge. Si les mœurs d’une nation rendaient indispensable l’obligation de préparer, dans chaque famille, quelque provision pour la vieillesse, comme elle en accorde en général à l’enfance, les besoins de première nécessité étant ainsi un peu plus étendus, le taux naturel des plus bas salaires serait un peu plus fort. J.-B. Say.
↑ Environ 5 francs.
↑ 18 décimes, ou 1 franc 80 centimes
↑ La proportion des maisons assurées au nombre total est aujourd’hui infiniment plus grande qu’à l’époque de la publication de la Richesse des nations.
↑ Aujourd’hui la paye mensuelle des matelots peut être évaluée de 50 à 60 schellings. Mac Culloch.
↑ Il est étrange que le docteur Smith n’ait pas fait allusion à l’usage de la presse, en énumérant les désavantages du service de mer. Mac Culloch.
↑ La part du travailleur est la rémunération de ses efforts, et forme son salaire.
↑ Cette conclusion de l’auteur nous semble bien obscurs et bien hasardeuse. Les choses ne se passent pas ainsi dans la réalité. A. B.
↑ C’est ce qui fait que dans les pays où les habitants des campagnes ne sont que journaliers, comme en Angleterre, la population agricole est réduite à l’état de misère. Elle n’est pas sûre d’être occupée toute l’année, et les salaires de son travail, suffisant à l’entretenir quand on a besoin d’elle comme à l’époque de la fenaison et de la moisson, tombent au-dessous des besoins les plus grossiers à d’autres époques. A. B.
↑ Chap. vii.
↑ Du mot cottage, qui veut dire chaumière.
↑ Aujourd’hui cette classe a entièrement disparu. Mac Culloch.
↑ En anglais pecks. Le peck anglais vaut 2 gallons, ou 9 litres 86 millilitres en mesures françaises.
↑ Worsted, c’est une laine filée beacuoup plus torse que l’autre, et préparée exprès pour faire des bas.
↑ Le fil de toile est généralement filé aujourd’hui à la mécanique. Mac Culloch.
↑ C’est-à-dire, du loyer ou prix du bail d’une maison entière, par opposition au prix d’un logement ou appartement qui ne comprend qu’une partie de la maison.
↑ Voyez sur ce loyer du sol, liv. V, chap. ii.
↑ Ce sont des tisserands en laine, ou ouvriers qui tissent les draps.
↑ Le statut d’apprentissage a été rapporté en 1814 par le statut 54 Geo. III, ch. 96. Cet acte ne louchait pas aux droits, privilèges ou statuts des différentes corporations légalement constituées ; mais là où d’anciens privilèges légaux ne font pas obstacle, les conditions de l’apprentissage et sa durée sont complètement abandonnées a la discrétion des parties intéressées. Mac Culloch.
↑ Il n’est pas besoin d’avertir que ces privilèges n’existent plus.
↑ Les villes de corporation, ou villes incorporées, ont acquis par charte royale, par acte du parlement, ou par usage immémorial, ce privilège, qui consiste à agir, posséder, etc., en corps ou nom collectif, et à s’organiser ou se nommer des chefs, syndics, etc. Plusieurs villes qui sont incorporées n’ont pas pour cela le droit de députer au parlement ; et plusieurs villes ou bourgs qui députent au parlement, ne sont pas incorporés, tels, par exemple, que la cité de Westminster.
↑ Nous ne connaissons pas assez l’état de l’industrie dans l’antiquité pour affirmer que l’apprentissage des métiers était absolument libre ou soumis à des conditions. Une grande partie des travaux industriels était faite par des esclaves, et c’est ce qu’Adam Smith a l’air d’oublier ici. Nous savons que chez les Romains l’industrie était organisée en collèges (collegia), soumis à des règlements particuliers. La loi des Douze Tables reconnaît aux collèges le droit d’établir des statuts, pourvu que ces statuts ne blessent eu rien les lois.
Mac Culloch.
↑ Voyez le Firma Burgi de Madox, pag. 26, etc. Note de l’auteur.
↑ L’industrie n’est pas réellement, en moyenne, mieux récompensée dans les villes que dans la campagne ; mais les marchands et les manufacturiers résidant dans une ville ont, comme le docteur Smith l’a déjà expliqué, un champ plus large pour l’exercice de leur industrie, ou plus d’occasions de faire fortune par l’emploi d’un grand capital. Mac Culloch.
↑ Ce passage sur la supériorité morale de la population agricole comparée à la population ouvrière des villes, est un de ceux qui révèlent le mieux la bonne foi et le génie du fondateur de l’économie politique. Les disciples d’Adam Smith, en Angleterre, n’ont pas voulu admettre le fait incontestable si bien exposé par leur maître. Mac Culloch prétend, dans une note, que si jamais la population agricole a été supérieure en intelligence et en moralité à la population industrielle, il n’en est plus de même aujourd’hui. Il soutient que les ouvriers de l’industrie anglaise sont aujourd’hui plus intelligents que les paysans agriculteurs. Selon lui, l’intelligence du paysan, continuellement occupée par les faits nombreux et variés qui passent sous ses yeux, n’a pas le temps de réfléchir, et elle reste endormie ; tandis que la monotonie des occupations industrielles sert d’excitation à l’intelligence de l’ouvrier des villes. Il prétend que la nature même de leurs occupations provoque les ouvriers de l’industrie à exercer leur intelligence, et il cite pour exemple les tisserands de Glasgow, de Manchester, etc. Ici, Mac Culloch est complètement démenti par les enquêtes récentes faites sur la condition des tisserands à la main. Les commissaires de l’enquête ont constaté que les tisserands étaient autrefois une classe intelligente et morale ; mais que, sous l’influence de la misère, ils sont descendus à l’abrutissement et à la dégradation morale, qui est la condition des basses classes de la nation anglaise. (Voyez, pour les résultats de celle enquête, l’ouvrage intitulé De la Misère des classes laborieuses en Angleterre et en France, Paris 1841.) Mac Culloch confond ici évidemment l’esprit plus éveillé qui se montre chez les ouvriers de l’industrie, avec le solide bon sens. Comme du temps d’Adam Smith, et plus encore, les populations agricoles sont supérieures en bon sens, en raison pratique, à celles des grandes villes d’industrie ; c’est un fait incontestable, qui n’est que trop démontré par la moralité comparée des agriculteurs et des ouvriers des villes. Il est aussi vrai que l’avantage intellectuel et moral est de leur côté, qu’il est vrai qu’ils vivent plus longtemps. La différence n’est peut-être pas aussi sensible qu’en France, parce que, en Angleterre, la plupart des ouvriers de l’agriculture sont réduits à la misère et à la dégradation morale qui en est la conséquence. Adam Smith a donc encore raison aujourd’hui en soutenant que les travaux agricoles sont plus favorables à la moralité, à la raison de l’homme, à la santé, que les travaux de l’industrie telle qu’elle est constituée aujourd’hui, surtout en Angleterre.
↑ Voyez le liv. IV, et surtout les chap ii, iii et viii.
↑ Si Adam Smith avait été témoin de ce qui s’est passé depuis 1791, relativement aux lois céréales, il aurait assurément modifié cette opinion. Mac Culloch.
↑ Voyez ci-après chap. xi, sur la fin.
↑ Voyez le liv. III, et notamment le chap. iv.
↑ Mac Culloch fait observer que ces coalitions ne peuvent jamais atteindre le but qu’elles se proposent. Du moment où une coalition élève les prix à un taux artificiel, l’intérêt que les individus ont à se séparer de cette coalition devient trop grand pour permettre que cette élévation de prix soit durable. A. B.
↑ Voyez le statut des ouvriers, vingt-cinquième année d’Édouard III. Note de l’auteur.
↑ Un curé est le dernier grade ecclésiastique dans l’église d’Angleterre ; c’est un ministre gagé pour desservir la cure pendant la vacance du bénéfice ou l’empêchement du titulaire. Garnier.
↑ Un acte passé en 1817 (57, Geo. III, ch. xcxix) autorise les évèques à nommer des curés et à leur assigner une pension qui, en aucun cas, ne doit être au-dessous de 80 livres sterling par an, et qui doit s’élever à 150 livres sterling, suivant l’accroissement de la population dans la paroisse. Mais, bien que cet acte ait certainement amélioré la condition des curés, on peut douter encore, par les raisons que donne le docteur Smith, que les dispositions de cet acte ne soient éludées par des conventions privées entre les curés et ceux qui les emploient. Mac Culloch.
↑ Voyez liv. V, chap. i, sect. 3, art. 3.
Un acte passé en 1812 élève à 150 livres sterling les honoraires des ecclésiastiques d’Écosse, qui étaient au-dessous de cette somme, sans compter le logement et le casuel. On admet généralement que cette somme est insuffisante pour entretenir un ecclésiastique d’une manière conforme à sa condition, et que le minimum des honoraires, outre le logement et le casuel, devrait être élevé à 250 ou 300 livres sterling par an. Mac Culloch.
↑ L’auteur évalue ici le denier ou la drachme des anciens à 8 den. st. Ainsi la mine, qui valait 100 drachmes, répond, dans son calcul, à 3 liv. 6 sous 8den. st. Nos auteurs français, qui suivent la même opinion sur les monnaies anciennes, pensent que le denier des Romains ou la drachme des Grecs contenait 66 de nos grains d’argent fin ; ce qui donnerait environ, pour la valeur de la mine, 79 fr. 50 cent. G. Garnier.
↑ Voyez liv. V, chap. i, sect. 3, art. 2.
↑ Garnier traduit le mot anglais settlement, domicile, par le mot insignifiant et incompréhensible d’établissement.
↑ Quand il y a eu, dans le cours de la même année, deux sessions du parlement, on désigne par le statut premier les actes passés dans la première de ces sessions.
↑ L’acte de Guillaume III, qui obligeait un pauvre à se procurer un certificat avant de pouvoir sortir d’une paroisse, a été rappelé en 1795 ; et il fut déclaré en même temps que les pauvres ne pourraient jamais être renvoyés de la paroisse, ou du lieu qu’ils habitaient, à l’endroit où ils avaient leur dernier domicile légal, avant d’être devenus présentement à charge u la paroisse.
Mac Culloch.
↑ Garnier traduit signifier un certificat, pour signer un certificat.
↑ Cour suprême de justice, à laquelle est spécialement attribuée la connaissance
↑ Les mandats d’arrêt ou warrants généraux sont ceux qui portent commission d’arrêter en général toutes personnes suspectes d’un tel délit, sans autre désignation de personnes. Ils ont été pratiqués surtout dans les poursuites contre les libelles et autres délits résultant de la presse. Enfin, cette forme a été déclarée illégale en 1766, et tout warrant doit être spécial, à peine de nullité, c’est-à-dire qu’il doit désigner spécialement et nominalement l’individu qu’il s’agit d’arrêter.
↑ On a accusé le docteur Smith d’avoir exagéré les effets pernicieux des lois de domicile, et peut-être ce reproche est-il fondé dans une certaine proportion. Mais malgré les améliorations apportées à ces lois par l’acte de 1795, qui abolit les certificats, et qui défend de renvoyer un pauvre avant qu’il soit devenu à charge à la paroisse, ces lois n’en ont pas moins donné lieu à une immense quantité de litiges. Les sommes dépensées en actions légales concernant les domiciles ou l’expulsion des pauvres, avant les changements opérés dans les lois des pauvres en 1834, furent rarement au-dessous de 300,000 à 350,000 livres sterling par an ! Aussi longtemps qu’existera un système de charité forcée pour l’entretien des pauvres, les paroisses répugneront toujours extrêmement à donner à un pauvre le droit de domicile, et seront toujours disposées à l’empêcher de l’obtenir par tous les moyens.
Mac Culloch.
↑ Ces lois et d’autres sur les salaires ont été rappelées par le statut de la cinquième années de Georges IV, chap. xcxv. Les maîtres et les ouvriers sont libres maintenant, en Angleterre, de se concerter pour abaisser ou élever les salaires. Mac Culloch.
↑ Officier de justice, dont la fonction est de juger criminellement tous délits incidents aux foires et marchés, tels que la vente à faux poids et mesures, etc. Comme il était anciennement commis par l’évêque, il a conservé le nom de clerc, quoique aujourd’hui ce juge soit presque toujours un laïque.
↑ Les lois relatives à la taxe du pain, à Londres et dans ses environs, ont été rappelées par un acte local passé en 1813 (55, Geo. III, ch. xix), et celles relatives à la taxe du pain en d’autres lieux, sont rarement exécutées. Mac Culloch.

%%%%%%%%%%%%%%%%%%%%%%%%%%%%%%%%%%%%%%%%%%%%%%%%%%%%%%%%%%%%%%%%%%%%%%%%%%%%%%%%
%                                  Chapitre 11                                 %
%%%%%%%%%%%%%%%%%%%%%%%%%%%%%%%%%%%%%%%%%%%%%%%%%%%%%%%%%%%%%%%%%%%%%%%%%%%%%%%%

\chapter{Des parties constituantes du prix des marchandises}
\markboth{Des parties constituantes du prix des marchandises}{}

Le fermage, considéré comme le prix payé pour l’usage de la terre, est naturellement le prix le plus élevé que le fermier est en état de payer, dans les circonstan­ces où se trouve la terre pour le moment. Lors de la stipulation des clauses du bail, le propriétaire fait tout ce qu’il peut pour ne lui laisser d’autre part dans le produit que celle qui est nécessaire pour remplacer le capital qui fournit la semence, paye le travail, achète et entretient les bestiaux et autres instruments de labourage, et pour lui donner, en outre, les profits ordinaires que rendent les fermes dans le canton. Cette part est évidemment la plus petite dont le fermier puisse se contenter sans être en perte, et le propriétaire entend rarement lui en laisser davantage. Tout ce qui reste du produit ou de son prix, ce qui est la même chose, au-delà de cette portion, quel que puisse être ce reste, le propriétaire tâche de se le réserver comme rente de sa terre ; ce qui est évidemment la rente la plus élevée que le fermier puisse payer, dans l’état ac­tuel de la terre. Quelquefois, à la vérité, par générosité, et plus souvent par ignoran­ce, le propriétaire consent à recevoir quelque chose de moins que ce surplus, et quel­quefois aussi, quoique plus rarement, le fermier se soumet par ignorance à payer quel­que chose de plus que ce reste, ou se contente de quelque chose de moins que les profits ordinaires des fermes du canton. Néanmoins, ce surplus peut toujours être regardé comme la rente naturelle de la terre, ou la rente moyennant laquelle on peut naturellement penser que seront louées la plupart des terres.

On pourrait se figurer que la Rente de la terre n’est souvent autre chose qu’un Profit et un Intérêt raisonnable du capital que le propriétaire a employé à l’améliora­tion de la terre. Sans doute, il y a des circonstances où la rente pourrait être regardée en partie comme telle ; car il ne peut presque jamais arriver que cela ait lieu pour plus que pour une partie[2]. Le propriétaire exige une rente même pour la terre non amélio­rée, et ce qu’on pourrait supposer être un intérêt ou profit des dépenses d’amélioration n’est, en général, qu’une addition à cette rente primitive. D’ailleurs, ces améliorations ne sont pas toujours faites avec les fonds du propriétaire, mais quelquefois avec ceux du fermier ; cependant, quand il s’agit de renouveler le bail, le propriétaire exige ordi­nai­rement la même augmentation de rente que si toutes ces améliorations eussent été faites de ses propres fonds.

Il exige quelquefois une rente pour ce qui est tout à fait incapable d’être amélioré par la main des hommes. La salicorne[3] est une espèce de plante marine qui donne, quand elle est brûlée, un sel alcalin dont on se sert pour faire du verre, du savon, et pour plusieurs autres usages ; elle croît en différents endroits de la Grande-Bretagne, particulièrement en Écosse, et seulement sur des rochers situés au-dessous de la haute marée, qui sont, deux fois par jour, couverts par les eaux de la mer, et dont le produit, par conséquent, n’a jamais été augmenté par l’industrie des hommes. Cependant, le propriétaire d’un domaine borné par un rivage où croît cette espèce de salicorne en exige une rente tout aussi bien que de ses terres à blé. 

Dans le voisinage des îles de Shetland, la mer est extraordinairement abondante en poisson, ce qui fait une grande partie de la subsistance des habitants ; mais, pour tirer parti du produit de la mer, il faut avoir une habitation sur la terre voisine. La rente du propriétaire est en proportion, non de ce que le fermier peut tirer de la terre, mais de ce qu’il peut tirer de la terre et de la mer ensemble. Elle se paye partie en poisson, et ce pays nous offre un de ces exemples, très-peu communs, où la rente cons­titue une des parties du prix de cette espèce de denrée[4].

La rente de la terre, considérée comme le prix payé pour l’usage de la terre, est donc naturellement un prix de monopole. Il n’est nullement en proportion des amélio­rations que le propriétaire peut avoir faites sur sa terre, ou de ce qu’il lui suffirait de prendre pour ne pas perdre, mais bien de ce que le fermier peut consentir à donner.

On ne peut porter ordinairement au marché que les parties seulement du produit de la terre dont le prix ordinaire suffit à remplacer le capital qu’il faut employer pour les y porter, et les profits ordinaires de ce capital. Si le prix ordinaire est plus que suffisant, le surplus en ira naturellement à la rente de la terre. S’il n’est juste que suffi­sant, la marchandise pourra bien être portée au marché, mais elle ne pourra fournir une rente au propriétaire. Le prix sera-t-il ou ne sera-t-il pas plus que suffisant ? C’est ce qui dépend de la demande.

Il y a certaines parties du produit de la terre dont la demande doit toujours être telle, qu’elles rapporteront un prix plus élevé que ce qui suffit pour les faire venir au marché ; il en est d’autres dont la demande peut être alternativement telle, qu’elles rap­por­tent ou ne rapportent pas ce prix plus fort que le prix suffisant. Les premières doi­vent toujours fournir de quoi payer une rente au propriétaire ; les dernières quelquefois suffiront à l’acquittement d’une rente, et d’autre fois non, suivant la différence des circonstances.

Il faut donc observer que la rente entre dans la composition du prix des mar­chandises d’une tout autre manière que les salaires et les profits. Le taux élevé ou bas des salaires et des profits est la cause du prix élevé ou bas des marchandises ; le taux élevé ou bas de la rente est l’effet du prix ; le prix d’une marchandise particulière est élevé ou bas, parce qu’il faut, pour la faire venir au marché, payer des salaires et des profits élevés ou bas ; mais c’est parce que son prix est élevé ou bas, c’est parce qu’il est ou beaucoup ou très-peu plus, ou pas du tout plus élevé que ce qui suffit pour payer ces salaires et ces profits, que cette denrée fournit de quoi payer une forte ou une faible rente, ou ne permet pas d’en acquitter une.

Je considérerai en particulier : 1° les parties du produit de la terre qui fournissent toujours de quoi payer une rente ; 2° celles qui peuvent quelquefois fournir de quoi en payer une, et quelquefois non ; 3° les variations qui, dans les différentes périodes de développement des sociétés, s’opèrent naturellement dans la valeur relative de ces deux différentes sortes de produits, soit qu’on les compare l’une avec l’autre, soit qu’on les compare avec les marchandises manufacturées. Ces trois objets diviseront ce chapitre en trois sections.

\section{Du produit qui fournit toujours de quoi payer une rente}

Les hommes, comme toutes les autres espèces d’animaux, se multipliant naturelle­ment en proportion des moyens de subsistance, les denrées alimentaires sont toujours plus ou moins demandées. En tout temps, la nourriture pourra acheter ou commander une quantité plus ou moins grande de travail, et toujours il se trouvera des individus disposés à faire quelque chose pour la gagner. À la vérité, ce qu’elle peut acheter de travail n’est pas toujours égal à ce qu’elle pourrait faire subsister de travailleurs si elle était distribuée de la manière la plus économique, et cela à cause des salaires élevés qui sont quelquefois donnés au travail. Mais elle peut toujours acheter autant de travail qu’elle peut en entretenir au taux auquel ce genre de travail est communément entretenu dans le pays.

Or, la terre, dans presque toutes les situations possibles, produit plus de nourriture que ce qu’il faut pour entretenir tout le travail qui concourt à mettre cette nourriture au marché, et même l’entretenir de la manière la plus libérale qui ait jamais eu lieu pour ce genre de travail. Le surplus de cette nourriture est aussi toujours plus que suffisant pour remplacer avec profit le capital qui emploie ce travail. Ainsi, il reste toujours quelque chose pour donner une rente au propriétaire.

Les marais les plus déserts d’Écosse et de Norvège forment une espèce de pâtu­rage pour des bestiaux qui, avec leur lait et l’accroissement du troupeau, suffisent toujours, non-seulement à faire subsister le travail qu’exigent leur garde et leur entre­tien, ainsi qu’à payer au fermier ou maître du troupeau les profits ordinaires de son capital, mais encore à fournir une petite rente au propriétaire. La rente augmente en proportion de la bonté du pâturage. Non-seulement la même étendue de terre nourrit un plus grand nombre de bestiaux, mais comme ils sont rassemblés dans un petit espace, ils exigent moins de travail pour leur garde et pour la récolte de leur produit. Le propriétaire gagne de deux manières : par l’augmentation du produit, et par la diminution du travail qu’il faut faire subsister sur ce produit.

La rente varie selon la fertilité de la terre, quel que soit son produit, et selon sa situation, quelle que soit sa fertilité. La terre située dans le voisinage d’une ville don­ne une rente plus élevée qu’une terre également fertile, située dans un endroit éloi­gné de la campagne. Quoique l’une et l’autre n’exigent peut-être pas plus de travail pour leur culture, il en coûte toujours nécessairement davantage pour amener au marché le produit de la terre éloignée. Il faut donc que ce dernier produit entretienne une plus grande quantité de travail et, par conséquent, que le surplus, dont le profit du fermier et la rente du propriétaire sont tirés tous les deux, en soit d’autant diminué. Mais, com­me on l’a déjà fait voir[5], dans les endroits éloignés de la campagne, le taux du pro­fit est généralement plus élevé que dans le voisinage d’une grande ville. Ainsi, dans ce surplus déjà diminué, il ne doit revenir qu’une part d’autant plus petite au proprié­taire.

Les grandes routes bien entretenues, les canaux et les rivières navigables, en diminuant les frais de transport, rapprochent du niveau commun les parties reculées de la campagne et celles qui avoisinent la ville. Ce sont aussi, par cette raison, les plus importantes des améliorations ; elles encouragent la culture des terres les plus éloi­gnées, qui forment nécessairement dans un pays la portion la plus étendue de sa surface. Elles sont avantageuses à la ville, en détruisant le monopole des campagnes situées dans son voisinage ; elles sont même avantageuses à cette dernière partie des campagnes. Si elles donnent lieu à introduire dans l’ancien marché quelques denrées rivales du produit de ces campagnes voisines, elles ouvrent aussi à ce produit plu­sieurs marchés nouveaux. Le monopole d’ailleurs est un des grands ennemis d’une bonne gestion, laquelle ne peut jamais s’établir universellement dans un pays, qu’au­tant que chacune se voit forcé, par une concurrence libre et générale, d’y avoir recours pour la défense de ses propres intérêts. Il n’y a pas plus de cinquante ans que quelques-uns des comtés voisins de Londres présentèrent au Parlement une pétition con­tre le projet d’étendre les routes entretenues[6] aux comtés plus éloignés de la capitale. Ces provinces éloignées, disaient-ils, en conséquence du bas prix de la main-d’œuvre, pourraient vendre leurs grains et fourrages à meilleur compte que nous au marché de Londres, et par ce moyen réduiraient nos fermages et ruineraient notre cul­ture. Cependant, depuis ce temps, ces réclamants ont vu leurs fermages s’augmenter et leur culture s’améliorer.

Une pièce de blé, d’une fertilité médiocre, produit une beaucoup plus grande quan­­tité de nourriture pour l’homme, que la meilleure prairie d’une pareille étendue. Quoique sa culture exige plus de travail, cependant le surplus qui reste après le rem­placement de la semence et la subsistance de tout ce travail, est encore beaucoup plus considérable. Ainsi, en supposant qu’une livre de viande de boucherie ne valût jamais plus qu’une livre de pain, cet excédent plus fort serait partout d’une plus grande valeur et formerait un fonds plus abondant, tant pour le profit du fermier que pour la rente du propriétaire. C’est ce qui semble avoir eu lieu partout généralement dans les pre­miers commencements de l’agriculture.

Mais la valeur relative de ces deux espèces de nourriture, le pain et la viande de boucherie, est fort différente, selon les différentes périodes de l’agriculture. Dans l’enfance grossière de cet art, les terres inhabitées et sans culture, qui forment alors la majeure partie du pays, sont toutes abandonnées au bétail. Il y a plus de viande que de pain ; par conséquent, le pain est la nourriture pour laquelle la concurrence est plus grande, et qui, en raison de cela, se vend à plus haut prix. Ulloa nous dit qu’à Buénos-Ayres, il y a quarante ou cinquante ans, le prix ordinaire d’un bœuf, choisi parmi un troupeau de deux ou trois cents était de 4 réaux, équivalant à 21 deniers et demi sterling. Il ne dit rien du prix du pain, sans doute parce qu’il n’y avait rien trouvé de remarquable. Un bœuf, dit-il, n’y coûte guère plus que la peine de le prendre. Mais nulle part le blé ne peut croître sans une grande quantité de travail ; et dans un pays situé sur les bords de la Plata, qui était alors la route directe de l’Europe aux mines d’ar­gent du Potosi, le prix pécuniaire du travail ne devait pas être à très-bon marché. Il en est autrement quand la culture s’est étendue à la majeure partie du pays ; il y a alors plus de pain que de viande. La concurrence prend une autre direction, et c’est le prix de la viande qui devient plus fort que celui du pain.

En outre, à mesure que la culture s’étend, les terres incultes deviennent insuffi­san­tes pour répondre à la demande de viande de boucherie. Une grande partie des terres cultivées est nécessairement employée à élever et à engraisser du bétail, dont il faut, par conséquent, que le prix suffise à payer, non-seulement le travail de le soigner et de le garder, mais encore les profits et la rente que cette terre mise en labour aurait pu rapporter au fermier et au propriétaire. Lorsqu’on amène les bestiaux au même mar­ché, ceux qui ont été nourris au milieu des friches les plus incultes sont, à proportion du poids et de la qualité, vendus au même prix que ceux qui ont été élevés sur la terre la mieux cultivée. Les propriétaires de ces friches en profitent, et ils haussent la rente de leurs terres en proportion du prix du bétail qu’elles nourrissent. Il n’y a pas plus d’un siècle que, dans plusieurs endroits des montagnes d’Écosse, la viande de bou­cherie était à aussi bon ou à meilleur marché que le pain même de farine d’avoine. Par l’union des deux royaumes, le marché d’Angleterre a été ouvert au bétail de ces mon­ta­gnes. Leur prix ordinaire est à présent environ trois fois plus élevé qu’au commen­cement du siècle, et pendant le même temps les rentes de la plupart des biens situés dans ce pays ont triplé et quadruplé. Dans presque toute la Grande-Bretagne, une livre de la meilleure viande de boucherie vaut aujourd’hui, en général, plus de deux livres du meilleur pain blanc, et dans les années d’abondance, elle en vaut quelquefois trois ou quatre.

C’est ainsi que, par les progrès de l’amélioration des terres, les rentes et profits des pâtures incultes se règlent, en quelque sorte, sur les rentes et profits de celles qui sont cultivées, et celles-ci, à leur tour, sur les rentes et profits des terres à blé. Le blé est une récolte annuelle. La viande de boucherie est une récolte qui met quatre ou cinq années à croître. Or, comme une acre de terre produira une beaucoup plus petite quan­tité d’une de ces deux espèces de nourriture que l’autre,’ il faut que l’infériorité de la quantité soit compensée par la supériorité du prix. S’il y avait plus que compensation, on remettrait des terres à blé en prairie ; et si la compensation n’était pas obtenue, une partie des prés serait remise en terres à blé.

On doit cependant entendre que ce n’est que dans la plus grande partie seulement des terres cultivées d’un grand pays, que peut avoir lieu cette égalité entre les rentes et profits fournis par les prés et prairies, et ceux fournis par le blé ; entre la terre dont le produit nourrit immédiatement le bétail, et celle dont le produit nourrit immé­dia­te­ment les hommes. Il y a des situations locales particulières où il en est tout à fait autrement, et où la rente et le profit que donne l’herbe des prés sont fort au-dessus de ceux que le blé pourrait rendre.

Ainsi, dans le voisinage d’une grande ville, la demande de lait et de fourrage con­tribue plus souvent, avec le haut prix de la viande de boucherie à élever la valeur de l’herbe des prés au-dessus de ce qu’on peut appeler sa proportion naturelle avec la valeur du blé. Il est évident que cet avantage local ne peut se communiquer aux terres situées à quelque distance.

Des circonstances particulières ont quelquefois rendu certains pays si peuplés, que tout le territoire, semblable à celui du voisinage d’une grande ville, n’a pu suffire à produire à la fois et le fourrage et le blé qu’exigeait la consommation. Ils ont donc par préférence employé leurs terres à la production du fourrage, comme la denrée la plus volumineuse et la plus difficile à transporter au loin ; et la nourriture de la masse du peuple, le blé, a été principalement importée des pays étrangers. Telle est à présent la situation de la Hollande, et telle semble avoir été celle d’une partie considérable de l’ancienne Italie, pendant la prospérité des Romains. Au rapport de Cicéron[7], Caton l’Ancien disait que le premier genre d’exploitation et le plus profitable dans une fer­me, c’était de faire d’excellents pâturages ; le second, d’en faire de médiocres, et le troisième, d’en faire de mauvais. Il ne mettait le labourage qu’au quatrième rang pour le profit et l’avantage. À la vérité, dans cette partie de l’Italie, voisine de Rome, le labourage doit avoir été extrêmement découragé par les fréquentes distributions de blé qu’on y faisait au peuple, gratuitement ou à très-bas prix. Ce blé était amené des provinces conquises, dont plusieurs étaient obligées de fournir à la république, par forme d’impôt, le dixième de leur produit à un prix fixe d’environ six deniers le quart de boisseau[8]. Le bas prix auquel ce blé était distribué au peuple doit nécessairement avoir fait baisser, sur le marché de Rome, le blé qui était porté du Latium ou de l’an­cien territoire de Rome, et il doit avoir découragé dans ce pays la culture des céréales.

De même, dans un canton ouvert dont la production principale est le blé, une prairie bien enclose fournira souvent une plus forte rente qu’aucune pièce de blé du voisinage. Elle est utile à la subsistance du bétail employé à la culture du blé et, dans ce cas, la forte rente qu’elle rend n’est pas tant payée, à proprement parier, par la va­leur de son propre produit, que par celle des terres à blé qui sont cultivées à l’aide de ce produit. Si jamais les terres voisines venaient à être généralement encloses, il est probable que cette rente baisserait. La forte rente que rendent aujourd’hui en Écosse les terres encloses paraît être un effet de la rareté des clôtures, et il est probable qu’elle ne durera pas plus longtemps que cette rareté. L’avantage de la clôture est plus grand pour un pré que pour une terre à blé ; elle épargne la peine de garder le bétail qui, d’ailleurs, se nourrit bien mieux quand il n’est pas sujet à être troublé par le berger ou par son chien.

Mais partout où il n’y a pas d’avantage local de ce genre, la rente et le profit que donne le blé ou tout autre végétal qui sert à la nourriture générale du peuple, doivent naturellement régler la rente et le profit que donnera une terre propre à cette produc­tion, et qui sera mise en prairie.

L’usage des prairies artificielles, des turneps, carottes, choux, etc., et tous les autres expédients dont on s’est avisé pour qu’une même quantité de terre pût nourrir un plus grand nombre de bestiaux que ne faisait la pâture naturelle, ont dû contribuer, à ce qu’il semble, à diminuer un peu la supériorité que le prix de la viande a natu­rel­lement sur celui du pain dans un pays bien cultivé. Aussi paraissent-ils avoir produit cet effet ; et il y a quelque raison de croire, au moins pour le marché de Londres, que le prix de la viande de boucherie est aujourd’hui moins élevé proportionnellement au prix du pain qu’il ne l’était au commencement du siècle dernier.

Dans le Supplément à la vie du prince Henri, le docteur Birch nous a rapporté les prix auxquels ce prince payait ordinairement sa viande de boucherie. Il y est dit que les quatre quartiers d’un bœuf pesant six cents livres lui coûtaient communément 9 livres 10 schellings ou environ, ce qui fait 31 schellings 8 deniers par chaque cent livres pesant. Le prince Henri est mort le 6 novembre 1612, dans la dix-neuvième année de son âge.

En mars 1764, le Parlement fit une enquête sur les causes de la cherté qui régnait alors dans le prix des denrées. Entre plusieurs preuves relatives à l’objet de cette en­quê­te, un marchand de Virginie, entendu en témoignage, déclara qu’au mois de mars 1763 il avait approvisionné ses équipages en bœuf, à 24 ou 25 schellings le cent pe­sant, ce qu’il regardait comme le prix ordinaire ; tandis que, dans cette année de cherté, il avait payé 27 schellings pour le même poids et la même qualité de viande. Cepen­dant, ce haut prix de 1764 est de 4 schellings 8 deniers inférieur au prix paye ordinai­rement par le prince Henri, et il faut observer qu’il n’y a que la meilleure viande qui soit propre à être salée pour ces voyages de long cours.

Le prix payé par le prince Henri s’élève à 3 deniers quatre cinquièmes pour cha­que livre pesant de tout le corps de la bête, en prenant l’un dans l’autre la basse viande et les morceaux de choix, et à ce compte les morceaux de choix n’auraient pas pu être vendus en détail à moins de 4 deniers et demi ou 5 deniers la livre.

Dans l’enquête parlementaire de 1764, les témoins établirent que les morceaux de choix du meilleur bœuf revenaient au consommateur au prix de 4 deniers et de 4 deniers un quart la livre, et la basse viande en général, de 7 farthings[9] à 2 deniers et demi et 2 deniers trois quarts ; et ils ajoutèrent que ces prix étaient généralement d’un demi-penny plus chers que les mêmes sortes de viande n’avaient été vendues habi­tu­ellement dans le mois de mars, les autres années. Or, ce haut prix lui-même est encore de beaucoup meilleur marché que ne paraît l’avoir été le prix ordinaire de la viande en détail, dans le temps du prince Henri.

Pendant les douze premières années du dernier siècle, le prix moyen du meilleur froment, au marché de Windsor, a été de 1 livre 18 schellings 3 deniers un sixième le quarter de neuf boisseaux de Winchester.

Mais dans les douze années qui ont précédé 1764, y compris cette même année, le prix moyen de la même mesure du meilleur froment au même marché a été de 2 livres 1 schelling 9 deniers et demi.

Ainsi, il paraît que, dans les douze premières années du dernier siècle, le froment a été bien meilleur marché, et la viande bien plus chère que dans les douze années antérieures à 1764 inclusivement.

Dans tous les grands pays, la majeure partie des terres cultivées est employée à produire ou de la nourriture pour les hommes, ou de la nourriture pour les bestiaux. La rente et le profit de ces terres règlent les rentes et profits de toutes les autres terres cultivées. Si quelque produit particulier fournissait moins, la terre serait bientôt remise en blé ou en prairie ; et s’il y avait quelque produit qui fournît plus, on consacrerait bientôt à ce genre de produit une partie des terres à blé ou des prairies.

À la vérité, les genres de productions qui exigent ou une plus grande dépense pri­mitive, ou une plus grande dépense annuelle de culture, pour que la terre y soit appro­priée, paraissent ordinairement rapporter, les uns une plus forte rente, les autres un plus gros profit que le blé ou l’herbe des prés. Néanmoins, on trouvera rarement que cette supériorité aille au-delà d’un intérêt raisonnable ou d’une juste compensation de cette plus forte dépense.

Une houblonnière, un verger, un potager, paraissent généralement rendre, tant au propriétaire qu’au fermier, en rente et en profit, plus qu’un pré ou une pièce de blé ; mais il faut aussi plus de dépense pour mettre la terre en cet état ; de là il est dû une plus forte rente au propriétaire ; elle exige aussi plus de soin, d’attention et d’habileté dans la culture : de là, un plus gros profit est dû au fermier ; la récolte aussi est plus précaire, du moins pour le houblon et les fruits ; il faut donc que le prix de cette récol­te, outre une compensation pour les pertes accidentelles, fournisse encore quelque cho­se, comme une espèce de prime d’assurance. La condition des jardiniers, bien peu aisée en général et toujours médiocre, nous prouve assez que, pour l’ordinaire, un métier aussi difficile n’est pas trop payé. Il y a tant de gens riches qui se livrent par amusement à cet art agréable, qu’il y a peu de profit à faire pour ceux qui le pratiquent pour vivre, parce que les personnes qui naturellement seraient leurs meilleurs cha­lands se procurent par elles-mêmes les productions les plus précieuses de ce genre de travail.

Il paraît que, dans aucun temps, l’avantage que le propriétaire retire de ces sortes d’améliorations n’a excédé ce qu’il lui fallait pour l’indemniser de la dépense qu’elles avaient originairement coûtée. Dans l’agriculture ancienne, après la vigne, c’était un potager bien arrosé, qui était, à ce qu’il semble, la partie de la ferme qu’on supposait rendre le meilleur produit. Mais Démocrite, qui écrivait sur l’agriculture, il y a environ deux mille ans, et qui était regardé par les anciens comme un des créateurs de l’art, pensait que ce n’était pas agir en homme sage que d’enclore un potager. Le profit, dit-il, n’indemniserait pas de la dépense d’un mur de pierres ; et ceux de briques (je présume qu’il entend parler de briques cuites au soleil) se dégradent par la pluie et les mauvais temps de l’hiver, et exigent des réparations continuelles. Columelle, qui rapporte ce sentiment de Démocrite, ne le contredit pas, mais il indique une méthode très-économique d’enclore avec une haie d’épines et de ronces, qu’il a trouvée, dit-il, par expérience, former une défense à la fois durable et impénétrable, mais qui, à ce qu’il semble, n’était guère connue du temps de Démocrite. Palladius adopte l’opinion de Columelle, qui avait été auparavant fort approuvée par Varron. Au jugement de ces anciens agriculteurs, le produit d’un jardin potager n’aurait été, à ce qu’il paraît, guère plus que suffisant pour payer les frais de la culture extraordinaire et de l’arro­sement ; car, dans ces pays méridionaux, on pensait alors, comme on le pense encore aujourd’hui, qu’il était à propos d’avoir à sa disposition un courant d’eau que l’on pût conduire dans chaque partie du jardin. Aujourd’hui, dans presque toute l’Europe, on ne juge pas qu’un potager mérite une meilleure clôture que celle indiquée par Columelle. Dans la Grande-Bretagne, et dans quelques autres pays du Nord, les bons fruits ne peuvent venir à maturité qu’à l’abri d’un mur ; en conséquence, dans ces pays-là, il faut que leur prix suffise à payer la dépense de bâtir et d’entretenir ce mur sans lequel on ne les obtiendrait pas. Souvent, le mur à fruit environne le potager, qui jouit ainsi de l’avantage d’une clôture que son propre produit ne pourrait presque jamais payer.

C’était, à ce qu’il paraît, une maxime reconnue dans l’ancienne agriculture, comme elle l’est encore dans tous les pays vignobles, que la vigne est la partie la plus profi­table de la ferme, quand elle est plantée convenablement et amenée à sa perfec­tion ; mais quant à savoir s’il était avantageux de planter une nouvelle vigne, c’était là un sujet de controverse parmi les anciens agriculteurs d’Italie, ainsi que nous l’apprend Columelle. Comme amateur de toute culture savante, il décide en faveur de la vigne, et il tâche de démontrer, en comparant le profit et la dépense, que c’était une des amé­li­orations les plus avantageuses. Toutefois, ces sortes de comparaisons entre le profit et la dépense d’entreprises nouvelles sont ordinairement sujettes à de grandes erreurs, et en agriculture plus qu’en toute autre affaire. Si de telles plantations eussent alors donné autant de bénéfice qu’il prétend qu’elles devaient le faire, il n’y aurait pas eu matière à dispute. La même question est aussi souvent débattue aujourd’hui dans les pays vignobles. Les écrivains en économie rurale dans ces contrées, amateurs et parti­sans de la grande culture, paraissent, il est vrai, généralement disposés à décider, comme Columelle, en faveur de la vigne. Ce qui paraît favoriser encore leur opinion, ce sont les sollicitudes que se sont données en France les propriétaires des anciennes vignes, pour empêcher qu’on n’en plantât de nouvelles ; ce fait semble indiquer, chez ceux qui en ont le plus l’expérience, une reconnaissance tacite que cette espèce de culture est pour le moment, dans ce pays, plus profitable qu’aucune autre. Cependant, on pourrait tirer du même fait une autre opinion, qui est que ce profit supérieur ne devrait pas survivre aux lois qui restreignent présentement la libre culture de la vigne. En 1731, ces propriétaires obtinrent un arrêt du conseil, qui défendit de planter de nouvelles vignes et de renouveler les anciennes arrachées depuis deux ans, à moins d’une permission particulière du roi, laquelle ne serait accordée que sur le rapport de l’intendant de la province, et son certificat portant que la terre, d’après l’examen, n’était susceptible d’aucune autre culture. Le prétexte de cet arrêt du conseil était la rareté du blé et des fourrages et la surabondance du vin ; mais si cette surabondance eût réellement existé, elle aurait très-efficacement empêché, sans le secours d’aucun arrêt du conseil, la plantation de nouvelles vignes, en réduisant les profits de ce genre d’exploitation au-dessous de leur proportion naturelle avec ceux des blés et des prairies. Et pour répondre à cette prétendue rareté de blés occasionnée par la multi­pli­cation des vignes, le blé n’est nulle part mieux cultivé en France que dans les provin­ces vignobles qui ont des terres propres à cette culture, telles que la Bourgogne, la Guyenne et le Haut-Languedoc. La quantité de bras qu’emploie une espèce de culture en­courage nécessairement l’autre, parce que la première fournit un marché tout prêt pour le produit de la seconde. C’est à coup sûr l’expédient le moins propre à encou­rager la culture du blé, que de diminuer le nombre de ceux qui sont en état de le payer ; c’est une politique aussi sage que celle qui voudrait donner de l’extension à l’agri­culture en décourageant les manufactures.

Ainsi, les rentes et profits des productions qui exigent ou de plus fortes avances primitives pour y approprier la terre, ou une plus grande dépense pour leur culture annuelle, quoique souvent fort supérieurs aux rentes et profits des blés et de l’herbe des prés, sont réglés par les rentes et profits de ces deux espèces ordinaires de récoltes, dans tous les cas où ils ne font que compenser, les avances et dépenses extra­ordinaires.

A la vérité, il arrive quelquefois que la quantité de terres qui peut être appropriée à une certaine production est trop petite pour répondre à la demande effective. Tout le produit en pourra alors être vendu à ceux qui sont disposés à donner quelque chose au-delà de ce qui est suffisant pour payer la totalité des rentes, salaires et profits em­ployés à le faire croître et à le mettre sur le marché, selon leurs taux naturels, ou selon les taux auxquels on les paye dans la majeure partie des terres cultivées. Dans ce cas, et dans ce seul cas, la portion restante du prix après le remboursement total des frais d’amélioration et de culture peut bien communément ne garder aucune proportion régulière avec le surplus correspondant dans le prix du blé ou de l’herbe des prés ; elle peut même l’excéder à un degré presque sans bornes et la majeure partie de cet excé­dent va naturellement à la rente du propriétaire.

Par exemple, ce que nous avons dit de la proportion naturelle et ordinaire entre les rentes et profits que rapporte le vin et ceux que donnent le blé et l’herbe des prés, ne doit s’entendre seulement que pour ces vignes qui ne produisent autre chose qu’un bon vin ordinaire, tel qu’il en peut croître à peu près partout où il se trouve un terrain léger, pierreux ou sablonneux, un vin qui n’a d’autre qualité que de la force et de la salubrité. Ce n’est qu’avec ces sortes de vignes seulement que les terres ordinaires du pays peuvent être mises en concurrence ; mais il est évident que cela ne peut avoir lieu à l’égard des vins d’une qualité particulière.

La vigne est, de tous les arbres à fruits, celui sur lequel la différence du terroir a le plus d’effet. Certains terroirs, à ce qu’on suppose, donnent au vin un bouquet qu’aucune espèce de culture ou de soins ne pourrait obtenir sur tout autre sol. Cet avantage, réel ou imaginaire, est quelquefois particulier au produit d’un petit nombre de vignes ; quelquefois il s’étend sur la majeure partie d’un petit canton, et quelquefois sur une partie considérable d’une vaste province. La quantité de ces vins qui va au marché est au-dessous de la demande effective ou de la demande de ceux qui seraient disposés à payer la totalité des rentes, profits et salaires nécessaires pour les faire croître et les mettre sur le marché, suivant le taux ordinaire, ou suivant le taux auquel ces rentes, profits et salaires sont payés dans les vignobles ordinaires. Toute cette quantité peut donc trouver son débit parmi ceux qui sont disposés à payer au-delà ; et cela élève nécessairement le prix de ces vins au-dessus des vins ordinaires. La diffé­rence est plus ou moins grande, selon que la vogue ou la rareté du vin donne plus ou moins d’activité à la concurrence des acheteurs. Quelle que soit cette différence, la majeure partie revient à la rente du propriétaire ; car, quoiqu’en général ces sortes de vignes soient cultivées avec plus de soin que la plupart des autres, cependant le haut prix du vin paraît être moins l’effet que la cause de cette culture plus soignée. Dans un produit aussi précieux, la perte que causerait la négligence est assez forte pour obliger même les plus négligents à être soigneux. Ainsi, il ne faut qu’une petite partie de ce haut prix pour payer les salaires du travail extraordinaire donné à la culture de ces vins, ainsi que les profits du capital extraordinaire qui alimente ce travail.

Les colonies à sucre que possèdent les nations européennes dans les Indes occid­entales peuvent être comparées à ces vignobles précieux. La totalité de leur produit est au-dessous de la demande effective de l’Europe, et elle peut trouver son débit parmi ceux qui consentent à en payer plus que la totalité des rentes, profits et salaires nécessaire à sa préparation et à son transport au marché, sur le pied où on les paye communément pour tout autre produit. M. Poivre, qui a observé avec le plus grand soin l’agriculture aux Indes, nous dit[10] que le plus beau sucre blanc, à la Cochin­chine, se vend communément 3 piastres le quintal, environ 13 schellings 6 deniers de notre monnaie. Ce qu’on appelle là le quintal pèse de 130 à 200 livres de Paris, ou en prenant le terme moyen, 175 livres de Paris ; ce qui, réduisant le prix du cent pesant d’Angleterre à environ 8 schellings sterl., ne fait pas le quart de ce qu’on paye com­mu­nément les sucres bruts ou moscouades[11] qu’on importe de nos colonies, et pas la sixième partie du prix du plus beau sucre blanc. À la Cochinchine, la majeure partie des terres cultivées est employée à produire du blé et du riz, la nourriture de la masse du peuple. Les prix respectifs du blé, du riz et du sucre ont probablement entre eux une proportion naturelle, celle qui s’établit naturellement entre les différents produits de la plupart des terres cultivées, et qui est capable d’indemniser le propriétaire et le fermier, aussi exactement qu’il est possible de l’apprécier, des dépenses qu’ils ont cou­tume de faire pour l’amélioration primitive et pour la culture annuelle. Mais, dans nos colonies à sucre, le prix du sucre n’a pas cette proportion avec le prix du produit d’un champ de blé ou de riz, en Europe ou en Amérique. On dit ordinairement qu’un plan­teur compte, pour se défrayer de toutes ses dépenses de culture, sur le rhum et les mélasses seulement, et que son sucre est pour lui un profit net[12]. Si cela est vrai, car je ne prétends pas l’affirmer, c’est comme si le fermier d’une terre à blé comptait payer tous ses frais de culture avec ses pailles, et que le grain fût tout profit pour lui. Nous voyons souvent des compagnies de négociants, à Londres et dans d’autres villes de commerce, acheter dans nos colonies à sucre des terres incultes, qu’ils se proposent de mettre en valeur, et de cultiver avec profit par le moyen des facteurs ou gérants, et cela malgré la grande distance, l’incertitude des retours et la mauvaise administration de la justice en ces contrées. Or, personne n’entreprendra de mettre en valeur et de cultiver de la même manière les terres les plus fertiles de l’Écosse, de l’Irlande ou des provinces à blé de l’Amérique septentrionale, quoique la justice, mieux administrée en ces pays, donne lieu de compter sur une plus grande régularité dans les retours.

Dans la Virginie et au Maryland, on préfère la culture du tabac à celle du blé, com­me étant plus lucrative. Le tabac pourrait se cultiver avec avantage dans la plus grande partie de l’Europe ; mais presque partout on en a fait un objet capital d’impôt, et on a pensé qu’il serait plus difficile de lever cet impôt sur chaque différente ferme de pays où cette plante serait cultivée, que de le lever, par les douanes, sur l’impor­tation de la denrée. C’est pour cette raison que, dans la plus grande partie de l’Europe, la plus absurde des prohibitions empêche la culture du tabac et donne nécessairement une espèce de monopole aux pays où cette culture est permise ; et comme c’est la Virginie et le Maryland qui produisent le plus de cette denrée, ces provinces ont une part considérable, quoique avec quelques concurrents, dans les bénéfices de ce mono­pole. Toutefois, la culture du tabac ne paraît pas aussi avantageuse que celle du sucre. je n’ai jamais entendu dire qu’aucune plantation de tabac ait été mise en valeur et cultivée avec des capitaux de commerçants résidant en Angleterre, et nous ne voyons guère arriver de nos colonies à tabac des planteurs aussi opulents que ceux qui nous arrivent souvent de nos îles à sucre. Si la préférence qu’on donne dans ces colonies à la culture du tabac sur celle du blé semble indiquer que la demande effective du tabac faite par l’Europe n’est pas complètement remplie, cependant cette demande est plus près de l’être, à ce qu’il semble, que celle du sucre ; et quoique le prix actuel du tabac soit probablement au-delà de ce qui suffit au payement de la totalité des rentes, profits et salaires qu’exigent sa préparation et son transport au marché, sur le pied auquel on les paye communément dans les terres à blé, il ne doit pas dépasser ce taux dans la même proportion que le prix actuel du sucre. Aussi, nos planteurs de tabac ont-ils témoigné les mêmes craintes sur la surabondance du tabac, que les propriétaires des anciennes vignes de France sur la surabondance du vin. Par un arrêté de leur assem­blée, ils ont restreint sa culture à six mille pieds (qu’on suppose rendre un millier pesant de tabac) par chaque nègre de l’âge de seize à soixante ans. Ils comptent qu’un nègre de cet âge, outre la quantité de tabac qu’il a à fournir, peut encore cultiver qua­tre acres de maïs. Pour empêcher aussi que le marché ne soit surchargé, ils ont quel­quefois, dans les années surabondantes, à ce que nous dit le docteur Douglas[13] que je soupçonne pourtant avoir été mal informé), brûlé une certaine quantité de tabac, par nègre, de la même manière qu’on nous dit que font les Hollandais pour les épices. S’il faut employer des moyens aussi violents pour maintenir le prix actuel du tabac, il est vraisemblable que la supériorité des avantages de cette culture sur celle du blé, s’il y en a encore quelqu’une, ne sera pas de longue durée.

C’est ainsi que la rente des terres cultivées pour produire la nourriture des hom­mes règle la rente de la plupart des autres terres cultivées. Aucun produit particulier ne peut longtemps rendre moins, parce que la terre serait aussitôt mise en autre nature de rapport ; et s’il y a quelque production particulière qui rende ordinairement plus, c’est parce que la quantité de terre qui peut lui être propre ne suffit pas pour remplir la demande effective.

En Europe, c’est le blé qui est la principale production de la terre servant immé­dia­tement à la nourriture de l’homme. Ainsi, excepté quelques circonstances particu­liè­res, la rente des terres à blé règle en Europe celles de toutes les autres terres cul­tivées. L’Angleterre n’est donc pas dans le cas d’envier à la France ses vignobles, ni à l’Italie ses plantations d’olives. À l’exception de circonstances particulières, le rapport de ces sortes de cultures se règle sur le rapport du blé ; et en blé, la fertilité de l’Angleterre n’est pas inférieure à celle de ces deux pays.

Si, dans un pays quelconque, la nourriture végétale ordinaire et favorite du peuple était tirée de quelque plante dont la terre la plus commune, avec la même ou presque la même culture, pût produire une beaucoup plus grande quantité que les terres les plus fertiles ne produisent de blé, alors la rente du propriétaire ou l’excédent de nour­riture qui lui restait après le payement du travail et le remboursement du capital et profits ordinaires du fermier, serait nécessairement beaucoup plus considérable. Quel que pût être, dans ce pays-là, le taux de la subsistance ordinaire du travail, ce plus grand excédent de la nourriture en ferait toujours subsister davantage et, par con­sé­quent, mettrait le propriétaire en état d’en acheter ou d’en commander une plus grande quantité. Il recevrait nécessairement une rente d’une valeur réelle plus considérable ; il aurait réellement plus de pouvoir et d’autorité sur le travail d’autrui ; il aurait à sa disposition et à son commandement plus de ces choses que fournit le travail d’autrui, et qui servent aux besoins et aux commodités de la vie.

Une rizière produit une plus grande quantité de nourriture que le champ de blé le plus fertile. Le produit ordinaire d’une acre s’élève, à ce qu’on dit, à deux récoltes par an de trente à soixante boisseaux chacune. Ainsi, quoique sa culture exige plus de tra­vail, quand tout ce travail a subsisté, il reste un plus grand excédent. Par conséquent, dans les pays à riz, où ce végétal est la nourriture ordinaire et favorite du peuple, et où il compose la principale subsistance du travail qui le cultive, il doit appartenir au propriétaire, dans ce plus grand excédent, une portion plus forte que celle qui lui re­vient dans les pays à blé. Dans la Caroline, où les planteurs sont généralement, com­me dans les autres colonies anglaises, fermiers et propriétaires à la fois et où, par conséquent, la rente se confond dans le profit, la culture du riz est regardée comme plus profitable que celle du blé, quoique leurs rizières ne produisent qu’une récolte par année, et quoique la coutume d’Europe y ait conservé assez d’empire pour que le peuple n’y fasse point du riz sa nourriture végétale ordinaire et favorite.

Un bon champ de riz est un vrai marécage dans toutes les saisons de l’année et, dans une saison, c’est un marécage entièrement couvert d’eau. Ce champ ne peut être propre ni au blé, ni au pâturage, ni à la vigne, ni dans le fait à aucune autre production végétale bien utile aux hommes ; et toutes les terres propres à ces diverses cultures ne le sont nullement à celle du riz, Ainsi, même dans les pays à riz, la rente des terres qui le produisent ne peut pas régler la rente des autres terres cultivées qu’il est impossible de mettre en cette nature de rapport.

Un champ de pommes de terre produit en quantité autant de nourriture qu’un champ de riz, et beaucoup plus qu’un champ de blé. Douze milliers pesant de pommes de terre ne sont pas un produit plus grand pour une acre de terre que deux milliers pe­sant de froment. À la vérité, la nourriture réelle ou la subsistance nourrissante qu’on peut tirer de chacun de ces deux végétaux n’est pas tout à fait en raison de leur poids, à cause de la nature aqueuse de la pomme de terre. Toutefois, en accordant la moitié du poids pour l’eau contenue dans cette racine, ce qui est beaucoup, il restera toujours six milliers pesant de nourriture solide produits par une acre de pomme de terre, c’est-à-dire trois fois la quantité produite par l’acre de blé. Une acre de pommes de terre coûte moins à cultiver qu’une acre de blé, l’année de jachères, qui précède en général les semailles, étant plus qu’une compensation du travail à la houe et des autres façons extraordinaires qu’on donne toujours aux pommes de terre. Si cette racine devenait jamais, dans quelque partie de l’Europe, comme le riz dans certains pays à riz, la nour­riture végétale ordinaire et favorite du peuple, au point d’occuper la même quan­tité de terres labourables, en proportion, qu’en occupe aujourd’hui le blé ou toute autre espèce de grain qui nourrit l’homme, il en résulterait que la même quantité de terres cultivées ferait subsister une bien plus grande quantité d’individus, et que ceux qui travailleraient étant généralement nourris de pommes de terre, il se trouverait un excédent bien plus considérable après le remplacement du capital et la subsistance de tout le travail employé de la culture. Il appartiendrait aussi au propriétaire une plus gran­de portion dans cet excédent. La population augmenterait, et les fermages s’élève­raient beaucoup au-dessus de ce qu’ils sont aujourd’hui[14].

La terre propre à produire des pommes de terre est propre à presque tous les végétaux utiles. Si donc les pommes de terre occupaient autant de terres cultivées que le blé en occupe actuellement, elles régleraient, comme lui, la rente de la plupart des terres cultivées.

On m’a dit que, dans quelques endroits du comté de Lancastre, le pain de farine d’avoine était regardé comme plus nourrissant pour les gens de travail que le pain de froment, et j’ai souvent entendu dire qu’on avait en Écosse la même opinion. Avec cela je doute un peu qu’elle soit vraie. En général, le bas peuple d’Écosse, qui se nour­rit de pain d’avoine, n’est ni aussi fort ni aussi beau que celui d’Angleterre, qui se nourrit de pain de froment. Il ne travaille pas aussi bien et n’a pas aussi bonne mine ; et comme la même différence ne se fait pas remarquer chez les personnes d’un rang supérieur de l’un et de l’autre pays, il semblerait, d’après l’expérience, que la nourri­ture des gens du peuple, en Écosse, ne convient pas aussi bien à la constitution de l’homme que celle des gens de la même classe en Angleterre. Mais il paraît qu’il n’en est pas de même des pommes de terre. On dit que les porteurs de chaises à bras, les portetaix, les porteurs de charbon à Londres, et ces malheureuses femmes qui vivent de prostitution, c’est-à-dire, les hommes les plus robustes et les plus belles femmes peut-être des trois royaumes, viennent pour la plupart des dernières classes du peuple d’Irlande, qui vivent, en général, de cette racine. Il n’y a pas de nourriture dont la qua­lité nourrissante, ou dont l’analogie particulière avec la constitution de l’homme soit démontrée d’une manière plus décisive.

Il est difficile de conserver les pommes de terre toute l’année, et impossible de les garder en magasin, comme le blé, pendant deux ou trois années de suite. La crainte qu’on a de ne pouvoir les débiter avant qu’elles ne se gâtent, en décourage la culture, et c’est peut-être le principal obstacle à ce qu’elles deviennent jamais, dans un grand pays, la première nourriture végétale de toutes les classes du peuple, comme l’est le pain.

\section{Du produit qui tantôt fournit et tantôt ne fournit pas de quoi payer une rente}

La nourriture de l’homme paraît être le seul des produits de la terre qui fournisse toujours et nécessairement de quoi payer une rente quelconque au propriétaire. Les autres genres de produits peuvent tantôt rapporter une rente, et tantôt non, selon les circonstances.

Les deux plus grands besoins de l’homme, après la nourriture, sont le vêtement et le logement.

La terre, dans son état primitif et inculte, peut fournir des matières premières de vêtement et de logement pour beaucoup plus de personnes qu’elle ne peut en nourrir. Dans son état de culture, au contraire, elle ne peut guère fournir de ces denrées à toutes les personnes qu’elle serait dans le cas de nourrir, du moins telles que ces personnes voudraient les avoir et consentiraient à les payer. Ainsi, dans le premier état, il y a toujours surabondance de ces matières, qui n’ont souvent, par cette raison, que peu ou point de prix. Dans l’autre, il y en a souvent disette, ce qui augmente néces­sairement leur valeur. Dans le premier état, une grande partie de ces matières est jetée comme inutile, et le prix de celles dont on fait usage est regardé comme repré­sen­tant seulement le travail et la dépense nécessaire pour les mettre en état de servir. Elles ne peuvent, en conséquence, fournir aucune rente au propriétaire du sol. Dans l’autre, elles sont toutes mises en œuvre, et il y a souvent demande pour plus qu’on n’en peut avoir. Il se trouve toujours quelqu’un disposé à donner, de chaque portion de ces matières, plus que ce qu’il en faut pour payer la dépense de les transporter au marché ; aussi, leur prix peut toujours fournir quelque chose pour acquitter une rente au propriétaire de la terre.

Les premières matières qu’on employa pour le vêtement furent les peaux des plus gros animaux. Ainsi, chez les peuples chasseurs et les peuples pasteurs, qui font leur principale nourriture de la chair de ces animaux, chaque homme, en pourvoyant à sa nourriture, se pourvoit en même temps de plus de matière de vêtements qu’il n’en pourra porter. S’il n’y avait pas de commerce étranger, on en jetterait la plus grande partie comme chose sans valeur. C’est ce qui avait lieu vraisemblablement chez les peuples chasseurs de l’Amérique septentrionale avant la découverte de leur pays par les Européens, avec lesquels ils échangent maintenant l’excédent de leurs pelleteries contre des couvertures, des armes à feu et de l’eau-de-vie, ce qui donne quelque va­leur à cet excédent. Dans l’état actuel du commerce du monde connu, les nations les plus barbares, à ce que je pense, chez lesquelles la propriété soit établie, ont quelque commerce étranger de cette espèce, et trouvent, pour toutes les matières de vêtement que leur terre produit, et qu’elles ne peuvent ou travailler ou consommer chez elles, une demande assez forte de la part de voisins plus riches qu’elles, pour en élever le prix au-delà de ce que coûte la dépense de les envoyer au marché. Ce prix fournit donc quelque rente au propriétaire de la terre. Lorsque les montagnards d’Écosse con­sommaient la majeure partie de leurs bestiaux dans leurs montagnes, l’exportation des peaux de ces animaux était l’article le plus considérable du commerce de ce pays, et ce qu’on recevait en échange ajoutait quelque chose à la rente des propriétés du lieu. La laine d’Angleterre, qui, dans les anciens temps, n’aurait pu ni se consommer ni se manufacturer dans le pays, trouvait un marché dans la Flandre, alors bien supé­rieure à l’Angleterre en richesse et en industrie, et son prix contribuait à augmenter un peu la rente du pays qui la produisait. Dans des pays qui n’auraient pas été mieux cultivés que n’était alors l’Angleterre, ou que ne sont aujourd’hui les montagnes d’Écosse, et qui n’auraient pas de commerce étranger, les matières de vêtement se­raient tellement surabondantes, qu’une grande partie en serait jetée comme inutile, et que le reste ne fournirait aucune rente au propriétaire.

Les matières de logement ne peuvent pas toujours se transporter à une aussi gran­de distance que celles de vêtement, et ne deviennent pas non plus aussi promp­tement un objet de commerce étranger. Lorsqu’elles sont surabondantes dans le pays qui les produit, il arrive fréquemment, même dans l’état actuel du commerce du monde, qu’elles ne soient d’aucune valeur pour le propriétaire de la terre. Une bonne carrière de pierres, dans le voisinage de Londres, fournirait une rente considérable. Dans beau­coup d’endroits de l’Écosse et de la province de Galles, elle n’en rapportera aucu­ne. Le bois propre à la charpente est d’une grande valeur dans un pays bien peuplé et bien cultivé, et la terre qui le produit rapporte une forte rente. Mais, dans plusieurs endroits de l’Amérique septentrionale, le propriétaire saurait très-bon gré à qui voudrait le débarrasser de la plupart de ses gros arbres. Dans quelques endroits des montagnes d’Écosse, faute de moyens de transport par eau ou par terre, l’écorce est la seule partie du bois qu’on puisse envoyer au marché. On y laisse les bois pourrir sur place. Quand il y a une telle surabondance dans les matières de logement, la partie dont on fait usage n’a d’autre valeur que le travail et la dépense employés à la rendre propre au service. Elle ne rapporte aucune rente au propriétaire, qui, en général, en abandonne l’usage à quiconque prend seulement la peine de le lui demander. Cepen­dant, il peut quelquefois être dans le cas d’en retirer une rente, s’il y a demande de la part de nations plus riches. Le pavé des rues de Londres a fourni aux propriétaires de quelques rochers stériles de la côte d’Écosse le moyen de tirer une rente de ce qui n’en avait jamais rapporté auparavant. Les bois de la Norvège et des côtes de la mer Baltique trouvent dans beaucoup d’endroits de la Grande-Bretagne un marché qu’ils ne pourraient pas trouver sur les lieux, et par ce moyen ils fournissent quelque rente à leurs propriétaires.

Les pays ne se peuplent pas en proportion du nombre d’hommes que leur produit pourrait vêtir et loger, mais en raison du nombre que ce produit peut nourrir. Quand la nourriture ne manque pas, il est aisé de trouver les choses nécessaires pour se vêtir et pour se loger ; mais on peut avoir celles-ci en abondance, et éprouver souvent de grandes difficultés à se procurer la nourriture. Dans quelques localités, même en Angleterre, le travail d’un seul homme, dans une seule journée, peut bâtir ce qu’on y appelle une maison. Le vêtement de l’espèce la plus simple, celui de peaux de bêtes, exige un peu plus de travail pour le préparer et le mettre en état de servir ; toutefois, il n’en exige pas encore beaucoup. Chez les peuples sauvages et barbares, la centième partie, ou guère plus, du travail de toute l’année, suffira pour les pourvoir de ce genre de vêtement et les loger de manière à satisfaire la majorité de la nation ; les quatre-vingt-dix-neuf autres centièmes suffisent souvent à peine pour les pourvoir de nourriture.

Mais lorsque, au moyen de la culture et de l’amélioration de la terre, le travail d’une seule famille peut fournir à la nourriture de deux, alors le travail d’une moitié de la société suffit pour nourrir le tout. Ainsi, l’autre moitié, ou au moins la majeure partie de cette autre moitié, peut être employée à faire autre chose ou à satisfaire les autres besoins et fantaisies des hommes. Les objets principaux de la plupart de ces besoins et fantaisies, ce sont le vêtement, le logement, le mobilier de la maison, et ce qu’on appelle le train ou l’équipage. Un riche ne consomme pas plus de nourriture que le plus pauvre de ses voisins. Elle peut être fort différente quant à la qualité, et exiger beaucoup plus de travail et d’art pour la choisir et l’apprêter ; mais quant à la quantité, il n’y a presque aucune différence. Comparez, au contraire, le vaste palais et la nombreuse garde-robe de l’un avec la cabane de l’autre, et le peu de guenilles qui le couvrent, et vous trouverez que, quant au vêtement, au logement et au mobilier, la différence est presque tout aussi grande en quantité qu’elle l’est en qualité. Dans tout homme, l’appétit pour la nourriture est borné par l’étroite capacité de son estomac ; mais on ne saurait mettre de bornes déterminées au désir des commodités et orne­ments qu’on peut rassembler dans ses bâtiments, sa parure, ses équipages et son mobilier. C’est pourquoi ceux qui ont à leur disposition plus de nourriture qu’ils ne peuvent en consommer personnellement cherchent toujours à en échanger le surplus, ou, ce qui revient au même, le prix de ce surplus, pour se procurer des jouissances d’un autre genre. Quand on a donné aux besoins limités ce qu’ils exigent, tout le sur­plus est consacré à ces besoins du superflu, qui ne peuvent jamais être remplis et qui semblent n’avoir aucun terme. Les pauvres, pour obtenir de la nourriture, s’occupent à satisfaire ces fantaisies des riches, et, pour être plus sûrs d’en obtenir, ils cherchent à l’emporter les uns sur les autres par le bon marché et la perfection de leur ouvrage. Le nombre des ouvriers augmente à mesure qu’augmente la quantité de nourriture, ou que la culture et l’amélioration de la terre vont en croissant ; et comme la nature de leur besogne admet une extrême subdivision de travail, la quantité des matières sur lesquelles ils s’exercent augmente dans une proportion infiniment plus forte que leur nombre. De là, naît la demande de toute espèce de matières que puisse mettre en œuvre l’invention des hommes, soit pour l’utilité, soit pour la décoration des bâti­ments, de la parure, de l’équipage ou du mobilier : de là, la demande des fossiles et des minéraux renfermés dans les entrailles de la terre ; de là, la demande des métaux précieux et des pierres précieuses.

Ainsi, non-seulement c’est de la nourriture que la rente tire sa première origine, mais encore, si quelque autre partie du produit de la terre vient aussi par la suite à rapporter une rente, elle doit cette addition de valeur à l’accroissement de puissance qu’a acquis le travail pour produit de la nourriture, au moyen de la culture et de l’amé­lioration de la terre.

Cependant, les autres parties du produit de la terre, qui, par la suite, viennent à rap­porter une rente, n’en rapportent pas toujours. La demande n’en est pas toujours assez forte, même dans les pays cultivés et améliorés, pour que le prix qu’elles rendent soit au-delà de ce qu’exigent le payement du travail dépensé pour les amener au marché et le remplacement du capital employé pour le même objet avec ses profits ordinaires. La demande sera ou ne sera pas assez forte pour cela, d’après différentes circonstances.
Par exemple, la rente que rapportera une mine de charbon de terre dépend en partie de sa fécondité, et en partie de sa situation.

On peut dire d’une mine, en général, qu’elle est féconde ou qu’elle est stérile, selon que la quantité de minerai que peut en tirer une certaine quantité de travail est plus ou moins grande que celle qu’une même quantité de travail tirerait de la plupart des autres mines de la même espèce.

Quelques mines de charbon de terre, avantageusement situées, ne peuvent être exploitées à cause de leur stérilité ; le produit n’en vaut pas la dépense ; elles ne peu­vent rapporter ni profit ni rente.

Il y en a dont le produit est purement suffisant pour payer le travail et remplacer, avec les profits ordinaires, le capital employé à leur exploitation ; elles donnent quel­que profit à leur entrepreneur, mais point de rente au propriétaire. Personne ne peut les exploiter plus avantageusement que le propriétaire, qui, en faisant lui-même l’en­tre­prise, gagne les profits ordinaires sur le capital qu’il y emploie. Il y a, en Écosse, beaucoup de mines de charbon qui sont exploitées ainsi, et qui ne pourraient pas l’être autrement. Le propriétaire n’en permettrait pas l’exploitation à d’autres sans exiger une rente, et personne ne trouverait moyen de lui en payer une.

Dans le même pays, il y a d’autres mines de charbon qui seraient bien assez fer­tiles, mais qui ne peuvent être exploitées à cause de leur situation. La quantité de minerai suffisante pour défrayer la dépense de l’exploitation pourrait bien être tirée de la mine avec la quantité ordinaire, ou même encore moins que la quantité ordinaire de travail ; mais dans un pays enfoncé dans les terres, peu habité, et qui n’a ni bonnes routes ni navigation, cette quantité de minerai ne pourrait être vendue.

Le charbon de terre est un chauffage moins agréable que le bois ; on dit, de plus, qu’il n’est pas aussi sain. Il faut donc qu’en général la dépense d’un feu de charbon de terre soit de quelque chose moindre, pour le consommateur, que celle d’un feu de bois.

Le prix du bois varie encore suivant l’état de l’agriculture, presque de la même manière, et précisément pour la même raison que le prix du bétail. Quand l’agriculture est encore dans sa première enfance, le pays est en grande partie couvert de bois, qui ne sont alors pour le propriétaire qu’un embarras, et qu’il donnerait volontiers pour la peine de les couper. À mesure que l’agriculture fait des progrès, les bois s’éclaircissent par l’extension du labourage et, d’un autre côté, dépérissent par suite de la multi­pli­cation des bestiaux. Quoique ces animaux ne multiplient pas dans la même proportion que le blé, qui est entièrement le fruit de l’industrie humaine, cependant la propa­gation de leur espèce est favorisée par les soins et la protection de l’homme, qui leur ménage, dans la saison de l’abondance, de quoi les faire subsister dans celle de la disette, leur fournit pendant tout le cours de l’année une plus grande quantité de nourriture que n’en fait naître pour eux la nature sauvage, et leur assure la plus libre jouissance de tout ce que leur offre la nature, en détruisant et en extirpant tout ce qui pourrait leur nuire. Des troupeaux nombreux qu’on laisse errer à travers les bois, quoiqu’ils ne détruisent pas les arbres âgés, empêchent la pousse des jeunes et, de cette manière, dans le cours d’un siècle ou deux, toute la forêt s’en va en ruine. Alors, la rareté du bois en élève le prix ; ce produit rapporte une forte rente, et quelquefois le propriétaire trouve qu’il ne peut guère employer plus avantageusement ses meilleures terres qu’en y faisant croître du bois propre à la charpente, qui, par l’importance du profit, compense souvent la lenteur des retours. Tel est à peu près aujourd’hui, à ce qu’il semble, l’état des choses dans plusieurs endroits de la Grande-Bretagne, où l’on trouve autant de profit à faire des plantations qu’à exploiter la terre en blé ou en prairies. Nulle part le bénéfice que le propriétaire retire d’une plantation de bois ne peut l’emporter, au moins pour longtemps, sur la rente que rapportent ces deux der­niers genres de produits ; mais, dans un pays enfoncé dans les terres, très-bien cultivé, il arrivera souvent que ce bénéfice n’y sera pas inférieur. À la vérité, dans un pays bien florissant, situé sur les côtes, si l’on peut se procurer facilement du charbon pour le chauffage, on trouvera quelquefois plus de profit à tirer le bois de charpente, pour ses bâtiments, des pays étrangers moins cultivés, que de les faire croître chez soi. Dans la nouvelle ville Édimbourg, bâtie il y a peu d’années, il n’y a peut-être pas une seule pièce de bois coupée en Écosse.

Quel que puisse être le prix du bois, si celui du charbon est tel, qu’un feu de charbon coûte presque autant qu’un feu de bois, nous pouvons être sûrs que, dans cette localité, et tant que les choses seront ainsi, le prix du charbon est aussi haut qu’il puisse être ; c’est ce qui existe apparemment dans quelques endroits de l’intérieur de l’Angleterre, spécialement dans le comté d’Oxford, où il est d’usage, même chez les gens du peuple, de mêler du bois et du charbon ensemble dans le foyer et où, par conséquent, il ne peut y avoir grande différence entre la dépense de ces deux sortes de chauffage.

Le charbon, dans les pays à mines de charbon, est partout fort au-dessous de ce prix extrême ; sans cela il ne pourrait pas supporter un transport éloigné, par terre ni même par eau. On ne pourrait en vendre qu’une petite quantité, et les maîtres char­bonniers et propriétaires des mines trouvent bien mieux leur compte à en vendre une grande quantité à quelque chose au-dessus du plus bas prix, qu’une petite quantité au prix le plus élevé. En outre, le prix de la mine de charbon la plus féconde règle le prix du charbon pour toutes les autres mines de son voisinage[15]. Le propriétaire et l’entre­preneur trouvent tous deux qu’ils pourront se faire, l’un une plus forte rente, l’autre un plus gros profit en vendant à un prix un peu inférieur à celui de leurs voisins. Les voisins sont bientôt obligés de vendre au même prix, quoiqu’ils soient moins en état d’y suffire, et quoique ce prix aille toujours en diminuant et leur enlève même quel­quefois toute leur rente et tout leur profit. Quelques exploitations se trouvent alors entièrement abandonnées ; d’autres ne rapportent plus de rente, et ne peuvent plus être continuées que par le propriétaire de la mine.

Le prix le plus bas auquel le charbon de terre puisse se vendre pendant un certain temps est, comme celui de toutes les autres marchandises, le prix qui est simplement suffisant pour remplacer, avec ses profits ordinaires, le capital employé à le faire venir au marché[16]. À une mine dont le propriétaire ne retire pas de rente, et qu’il est obligé d’exploiter lui-même ou d’abandonner tout à fait, le prix du charbon doit, en général, approcher beaucoup de ce prix.

La rente, quand le charbon en rapporte une, compose pour l’ordinaire une plus petite portion du prix qu’elle ne le fait dans la plupart des autres produits durs de la terre. La rente d’un bien à la surface de la terre s’élève communément à ce qu’on sup­po­se être le tiers du produit total, et c’est pour l’ordinaire une rente fixe et indépen­dan­te des variations accidentelles de la récolte. Dans les mines de charbon, un cinquième du produit total est une très-forte rente ; un dixième est la rente ordinaire, et cette rente est rarement fixe, mais elle dépend des variations accidentelles dans le produit. Ces variations sont si fortes, que, dans un pays où les propriétés foncières sont censées vendues à un prix modéré, au denier trente, c’est-à-dire moyennant trente années du revenu, une mine de charbon vendue au denier dix est réputée vendue à un bon prix.

La valeur d’une mine de charbon pour le propriétaire dépend souvent autant de sa situation que de sa fécondité. Celle d’une mine métallique dépend davantage de sa fécondité et moins de sa situation. Les métaux même grossiers, et à plus forte raison les métaux précieux, quand ils sont séparés de leur gangue, ont assez de valeur pour pouvoir, en général, supporter les frais d’un long transport par terre et du trajet le plus lointain par mer. Leur marché ne se borne pas aux pays voisins de la mine, mais il s’étend au monde entier. Le cuivre du japon est un des articles du commerce de l’Europe, le fer d’Espagne est un de ceux du commerce du Chili et du Pérou ; l’argent du Pérou s’ouvre un chemin, non-seulement jusqu’en Europe, mais encore de l’Europe à la Chine.

Au contraire, le prix des charbons du Westmoreland ou du Shropshire ne peut influer que peu sur leur prix à Newcastle, et leur prix dans le Lyonnais n’exercera sur celui des premiers aucune espèce d’influence. Les produits de mines de charbon aussi distantes ne peuvent se faire concurrence l’un à l’autre ; mais les produits des mines métalliques les plus distantes peuvent souvent le faire et le font en réalité communé­ment. Ainsi, le prix des métaux même grossiers, et plus encore celui des métaux pré­cieux ; dans les mines les plus fécondes qui existent, influe nécessairement sur le prix de ces métaux dans toute autre mine du monde. Le prix du cuivre au Japon a néces­sai­rement quelque influence sur le prix de ce métal aux mines de cuivre d’Europe. Le prix de l’argent au Pérou, ou la quantité, soit de travail, soit de toute autre marchan­dise qu’il peut y acheter, doit avoir quelque influence sur le prix de l’argent, non-seu­le­ment aux mines d’Europe, mais même à celles de la Chine. Après la découverte des mines du Pérou, les mines d’argent d’Europe furent pour la plupart abandonnées. La valeur de l’argent fut tellement réduite, que le produit de ces dernières ne pouvait plus suffire à payer les frais de leur exploitation, ou remplacer, avec un profit, les dépenses de nourriture, vêtement, logement et autres choses nécessaires qui étaient consom­mées pendant cette opération. La même chose arriva à l’égard des mines de Cuba et de Saint-Domingue, et même à l’égard des anciennes mines du Pérou, après la décou­verte de celles du Potosi.

Ainsi, le prix de chaque métal à chaque mine étant réglé en quelque sorte par le prix de ce métal à la mine la plus féconde qui soit pour le moment exploitée dans le monde, il en résulte que, dans la plus grande partie des mines, ce prix ne doit guère s’élever au-delà de la dépense de l’exploitation, et qu’il peut rarement fournir une bien forte rente au propriétaire. Aussi, dans la plupart des mines, la rente ne compose-t-elle qu’une petite part du prix du métal, et une bien plus petite encore lorsqu’il s’agit de métaux précieux. Le travail et le profit forment la majeure partie de ce prix.

La rente moyenne des mines d’étain de Cornouailles, les plus fécondes du monde, s’élève à un sixième du produit total, à ce que nous assure M. Borlace, garde des mines : quelques-unes, dit-il, rendent plus, et quelques autres moins ; la rente de plu­sieurs mines de plomb très-fécondes, situées en Écosse, est aussi d’un sixième du produit total.

Frézier et Ulloa nous disent qu’aux mines du Pérou, le propriétaire n’impose sou­vent pas d’autre condition à l’entrepreneur de la mine, que celle de venir broyer ou bocarder le minerai à son moulin, en lui payant le prix d’usage pour cette prépara­tion ; à la vérité, jusqu’en 1738, la taxe due au roi d’Espagne s’élevait au cinquième de l’argent au titre, ce qu’on pouvait regarder alors comme la véritable rente de la plupart des mines d’argent du Pérou, les plus riches que l’on connût dans le monde. S’il n’y avait pas eu de taxe imposée sur le produit, ce cinquième aurait appartenu naturelle­ment au propriétaire, et il y aurait eu beaucoup de mines exploitées, qui ne pouvaient l’être alors parce qu’elles n’auraient pas pu suffire à payer cette taxe. La taxe du duc de Cornouailles sur l’étain est supposée de plus de cinq pour cent, ou d’un vingtième de la valeur et, quelle que puisse être sa proportion avec le produit total, le montant de cette taxe appartiendrait naturellement aussi au propriétaire si l’étain était franc de droits ; or, si vous ajoutez un vingtième à un sixième, vous trouverez que la totalité de la rente moyenne des mines d’étain de Cornouailles était à la totalité de la rente moyenne des mines d’argent du Pérou comme 13 est à 12 ; mais les mines d’argent du Pérou ne sont pas en état aujourd’hui de payer même cette faible rente, et en 1736 la taxe sur l’argent a été réduite d’un cinquième à un dixième ; de plus, cette taxe, même telle qu’elle est, donne plus de tentation à la fraude que la taxe d’un vingtième sur l’étain, et la contrebande est bien plus facile sur une marchandise précieuse que sur une denrée d’un volume considérable ; aussi dit-on que la taxe du roi d’Espagne est fort mal payée, et que celle du duc de Cornouailles l’est fort bien. Il est donc vrai­semblable que la rente entre pour une plus grande partie dans le prix de l’étain aux mines d’étain les plus fécondes, qu’elle ne le fait dans le prix de l’argent aux mines d’argent les plus fécondes qui existent. Il semble que ce qui reste au propriétaire après le remplacement du capital employé à exploiter ces différentes mines, avec ses profits ordinaires, est plus considérable dans le métal grossier, qu’il ne l’est dans le métal précieux.

Les profits des entrepreneurs des mines d’argent ne sont pas non plus ordinai­rement très-considérables au Pérou. Les auteurs estimables et très-bien informés que nous avons déjà cités nous apprennent que, lorsqu’un capitaliste entreprend au Pérou l’exploitation d’une nouvelle mine, il est généralement regardé comme un homme à peu près ruiné et presque en banqueroute ; ce qui fait que tout le monde le fuit et évite d’avoir des relations avec lui. L’entreprise d’une nouvelle mine est considérée, dans ce pays, comme l’est ici une loterie dans laquelle le montant des lots ne compense pas la perte des billets blancs, quoique l’importance de quelques-uns de ces lots pousse beaucoup de joueurs téméraires à y aventurer la totalité de leur fortune[17].

Cependant, comme le souverain tire une partie considérable de son revenu du produit des mines d’argent, les lois du Pérou encouragent par tous les moyens possi­bles la découverte et l’exploitation des mines nouvelles. Quiconque découvre une mine est autorisé à prendre une longueur de terrain de 246 pieds, dans la direction qu’il suppose à la veine, et moitié autant en largeur ; il devient propriétaire de cette portion de la mine, et il peut l’exploiter sans payer aucune redevance au propriétaire du terrain. L’intérêt des ducs de Cornouailles a donné lieu à un règlement à peu près de même genre dans cet ancien duché. Toute personne qui découvre une mine d’étain dans des terres incultes et sans clôture, peut en marquer les limites à une certaine éten­due ; c’est ce qu’on appelle borner une mine, celui qui borne ainsi devient le vrai propriétaire de la mine, et il peut ou l’exploiter lui-même, ou la donner à bail à un autre sans le consentement du propriétaire du soi, sauf une très-légère redevance qu’il faut payer à celui-ci lors de l’exploitation. Dans l’un et l’autre de ces règlements, on a sacrifié les droits sacrés de la propriété privée à l’intérêt prétendu du revenu public.

On a donné, au Pérou, les mêmes encouragements à la découverte et à l’exploi­tation des nouvelles mines d’or ; sur l’or, la taxe du roi ne s’élève qu’à un vingtième du métal au titre ; elle était autrefois d’un cinquième, et elle a été réduite à un dixième comme la taxe sur l’argent, mais on trouva que l’entreprise n’était pas en état de supporter même la plus faible de ces deux taxes. Néanmoins, s’il est rare, disent les mêmes auteurs (Frézier et Ulloa), de trouver quelqu’un qui ait fait sa fortune dans l’exploitation d’une mine d’argent, il est encore bien plus rare d’en trouver qui l’aient faite avec une mine d’or. Ce vingtième paraît être la totalité de la rente qui se paye par la plus grande partie des mines d’or au Chili et au Pérou ; l’or est, en effet, plus facile à passer en fraude que l’argent, non-seulement par rapport à la supériorité de la valeur du métal comparée à son volume, mais encore par rapport à la manière particulière dont la nature le produit. On trouve très-rarement l’argent à l’état natif, mais, comme la plupart des autres métaux, il est ordinairement mêlé à une autre substance, dont il est impossible de le séparer en quantité assez grande pour payer la dépense, sinon par des opérations lentes et pénibles qui ne peuvent s’établir que dans des laboratoires construits exprès pour cela et, par conséquent, exposés à l’inspection des officiers du roi. On trouve, au contraire, presque toujours l’or à l’état natif : on le trouve quelque­fois en morceaux d’un certain volume ; et même, quand il se trouve mêlé en parties fort petites et presque insensibles avec du sable, de la terre et d’autres matières étran­gères, on vient à bout de l’en séparer par une opération très-courte et très-simple, que tout le monde peut faire dans sa propre maison, avec quelque peu de mercure. Si donc la taxe du roi est mal payée sur l’argent, elle l’est vraisemblablement encore bien plus mal sur l’or, et la rente doit faire encore une bien plus petite partie du prix, dans celui de l’or, que dans celui de l’argent[18].

Le plus bas prix auquel on puisse, pendant un certain temps, vendre les métaux précieux, c’est-à-dire la plus petite quantité d’autres marchandises pour laquelle on puisse les échanger, se règle sur les mêmes principes qui déterminent le plus bas prix ordinaire de toute autre marchandise. Ce qui le détermine, c’est le capital qu’il faut com­munément employer pour le faire venir de la mine au marché, c’est-à-dire la quan­tité de nourriture, vêtement et logement qu’il faut communément consommer pour cela. Il faut que le prix soit tout au moins suffisant pour remplacer ce capital avec les profits ordinaires.

Toutefois, leur plus haut prix ne paraît pas être déterminé nécessairement par aucune autre circonstance que celle de la rareté ou de l’abondance actuelle de ces mé­taux eux-mêmes. Il n’est déterminé par le prix d’aucune marchandise, comme le prix du charbon de terre se trouve l’être par celui du bois, au-delà duquel il ne peut jamais s’élever, quelque rare que puisse être ce minéral. Augmentez, au contraire, la rareté de l’or à un certain degré, et la plus petite parcelle pourra devenir plus précieuse qu’un diamant, et obtenir en échange une plus grande quantité d’autres marchandises,

La demande de ces métaux provient en partie de leur utilité et en partie de leur beauté. Ils sont plus utiles peut-être qu’aucun autre métal, si l’on en excepte le fer. Comme ils sont moins sujets que tout autre à se rouiller et à se corrompre, il est bien plus aisé de les tenir propres, et c’est par cette raison qu’on trouve plus d’agrément à s’en servir pour les ustensiles de la table et de la cuisine. Une bouilloire d’argent est plus propre qu’une de plomb, de cuivre ou d’étain et la même qualité rendra une bouil­loire d’or encore bien préférable. Cependant, le principal mérite de ces métaux vient de leur beauté, qui les rend particulièrement propres à l’ornement de la parure et des meubles du ménage. Il n’y a pas de peinture ou de vernis qui puisse donner une couleur aussi magnifique que la dorure. Leur rareté ajoute encore extrêmement au mérite de leur beauté. Pour la plupart des gens riches, la principale jouissance de la richesse consiste à en faire parade, et cette jouissance n’est jamais plus complète que lorsqu’ils étalent ces signes incontestables d’opulence, que personne qu’eux seuls ne peut posséder. À leurs yeux, le mérite d’un objet qui a quelque degré d’utilité ou de beauté est infiniment rehaussé par sa rareté ou par le grand travail qu’il faut employer pour en rassembler une quantité considérable, travail qu’eux seuls sont en état de payer. Ils achèteront volontiers de tels objets à un prix plus haut que des choses beau­coup plus utiles ou plus belles, mais qui seraient plus communes. Ce sont ces qualités d’utilité, de beauté et de rareté qui sont la première source du haut prix de ces métaux, c’est-à-dire de la grande quantité d’autres marchandises qu’ils peuvent obtenir partout en échange. Cette valeur a précédé leur usage comme monnaie, et elle en est indépen­dante ; elle est la qualité qui les a rendus propres à cet usage. Cet usage cependant, en occasionnant une nouvelle demande de ces métaux, et en diminuant la quantité qui pourrait en être employée de toute autre manière, peut avoir, par la suite, contribué à maintenir leur valeur ou même à l’augmenter.

La demande de pierres précieuses vient entièrement de leur beauté. Elles ne ser­vent à rien qu’à l’ornement, et le mérite de leur beauté est extrêmement rehaussé par leur rareté ou par la difficulté et la dépense de les extraire de la mine. En consé­quen­ce, c’est de salaires et de profits qu’est composée le plus souvent la presque totalité de leur haut prix. La rente n’y entre que pour une très-faible part, très-souvent elle n’y entre pour rien, et il n’y a que les mines les plus fécondes qui puissent suffire à payer une rente un peu considérable. Lorsque le célèbre joaillier Tavernier alla visiter les mines de diamants de Golconde et de Visapour, on lui dit que le souverain du pays, pour le compte duquel on les exploitait, avait donné ordre de les fermer toutes, à l’ex­cep­­tion de celles qui produisaient les pierres les plus grosses et les plus belles. Les autres, à ce qu’il semble, ne valaient pas pour le propriétaire la peine de les faire exploiter.

Le prix des métaux précieux et des pierres précieuses étant réglé pour le monde entier par le prix qu’ils ont à la mine la plus féconde, il s’ensuit que la rente que peut rapporter au propriétaire une mine des uns ou des autres est en proportion, non de la fécondité absolue de la mine, mais de ce qu’on peut appeler sa fécondité relative, c’est-à-dire de sa supériorité sur les autres mines du même genre. Si l’on découvrait de nouvelles mines qui fussent aussi supérieures à celles du Potosi que celles-ci se sont trouvées être supérieures aux mines de l’Europe, la valeur de l’argent pourrait par là se dégrader au point que les mines, même du Potosi, ne vaudraient pas la peine d’être exploitées. Avant la découverte des Indes occidentales espagnoles, les mines les plus fécondes de l’Europe peuvent avoir rapporté à leurs propriétaires une rente aussi forte que celle que rapportent à présent aux leurs les plus riches mines du Pérou. Quoique la quantité d’argent produit fût beaucoup moindre, elle s’échangeait peut-être contre tout autant de marchandises, et la part revenant au propriétaire mettait à sa disposition ou à son commandement une quantité égale, soit de travail, soit de toute autre mar­chan­dise. La valeur tant du produit total que de la rente, le revenu réel qu’elles don­naient tant au public qu’au propriétaire, pouvait être le même.

Les mines les plus abondantes, soit en métaux précieux, soit en pierres précieuses, ne pourraient qu’ajouter très-peu à la richesse du monde. L’abondance dégrade néces­sairement la valeur d’un produit, qui ne tire sa principale valeur que de sa rareté. Un service de vaisselle d’argent et tous les autres ornement frivoles de la parure et du mobilier pourraient alors s’acheter moyennant une moindre quantité de marchan­dises ; et c’est en cela seulement que consisterait tout l’avantage que cette abondance procurerait au monde[19].

Il en est autrement des biens qui existent à la surface de la terre. La valeur, tant de leur produit que de leur rente, est en proportion de leur fertilité absolue et non de leur fertilité relative. La terre qui produit une certaine quantité de nourriture ou de matiè­res propres au vêtement ou au logement, peut toujours nourrir, vêtir et loger un certain nombre de personnes ; et quelle que soit la proportion dans laquelle le pro­priétaire prendra sa part de ce produit, cette part mettra toujours à sa disposition une quantité proportionnée du travail de ces personnes et des commodités que ce travail peut lui procurer. La valeur des terres les plus stériles n’éprouve aucune diminution par le voisinage des terres les plus fertiles. Au contraire, elle y gagne, en général, une augmentation. Le grand nombre de personnes que les terres fertiles font subsister, procure à plusieurs parties du produit des terres stériles un marché qu’elles n’auraient jamais trouvé parmi les personnes que leur propre produit eût pu faire subsister.

Tout ce qui tend à rendre la terre plus fertile en subsistances, augmente non-seule­ment la valeur des terres sur lesquelles se fait l’amélioration, mais encore contribue à augmenter pareillement la valeur de plusieurs autres terres, en faisant naître de nou­velles demandes de leur produit. Cette abondance de subsistance, dont en consé­quen­ce de l’amélioration de la terre plusieurs personnes se trouvent avoir à disposer au-delà de leur propre consommation, est la grande cause qui donne lieu à la demande de métaux précieux, de pierres précieuses, aussi bien que de tout autre objet de commo­dité ou d’ornement pour la parure, le logement, l’ameublement et l’équipage. La nourriture de l’homme constitue non-seulement le premier et principal article des richesses du monde, mais c’est encore l’abondance de cette nourriture qui donne à plusieurs autres genres de richesse la plus grande partie de leur valeur. Lors de la première découverte de Cuba et de Saint-Domingue par les Espagnols, les pauvres habi­tants de ces îles avaient coutume de porter de petits morceaux d’or en guise d’ornement dans leurs cheveux et sur différentes parties de leurs vêtements. Ils semblaient en faire le cas que nous pourrions faire de quelques petits cailloux un peu plus jolis que les autres, que nous regarderions comme valant tout juste la peine de les ramasser, mais comme de trop peu de prix pour les refuser a quiconque nous les demanderait. Ils donnaient cet or à leurs nouveaux hôtes à la première demande, sans paraître se douter seulement qu’ils leur eussent fait là un présent de quelque valeur. Ils furent très-surpris de voir la fureur des Espagnols pour ce métal ; ils ne pouvaient pas soupçonner qu’il y eût un pays au monde où la nourriture, qui est toujours si rare parmi eux, se trouvât entre les mains et à la disposition de quelques personnes en telle surabondance, qu’elles consentissent à en céder de quoi faire subsister toute une famille pendant plusieurs années pour se procurer seulement une petite quantité de ces colifichets brillants. S’ils avaient pu concevoir cette idée, la passion des Espagnols n’aurait plus été pour eux un objet de surprise.

\section{Des variations dans la proportion entre les valeurs respectives de l’espèce de produit qui fournit toujours une Rente, et l’espèce de produit qui quelquefois en rapporte une et quelquefois n’en rapporte point}

L’abondance de plus en plus grande de subsistances, en conséquence des amélio­rations successives de la culture, doit nécessairement faire augmenter sans cesse la demande de chaque partie du produit de la terre qui n’est pas la nourriture, et qu’on peut faire servir pour la commodité ou pour l’ornement. On pourrait donc s’attendre à ce que, pendant le cours des progrès, il ne pût y avoir qu’une seule espèce de variation dans les valeurs comparatives des deux différentes sortes de produits. Cette espèce de produit qui quelquefois rapporte une rente, et quelquefois n’en rapporte pas, devrait constamment augmenter de valeur, relativement à l’espèce de produit qui rapporte toujours une rente. À mesure que les arts et l’industrie vont en avançant, les matières propres à vêtir et à loger, les fossiles et les minéraux utiles, les métaux précieux et les pierres précieuses devraient successivement être de plus en plus demandés ; ils de­vraient successivement s’échanger contre une quantité de subsistances de plus en plus grande, ou, en d’autres termes, ils devraient toujours renchérir de plus en plus. C’est aussi ce qui est arrivé à l’égard de presque toutes ces choses, le plus souvent, et il en serait arrivé de même pour toutes, dans toutes les circonstances, si à certaines épo­ques des événements particuliers n’avaient pas augmenté l’approvisionnement de quelques-unes de ces choses dans une proportion encore beaucoup plus forte que la demande.

La valeur d’une carrière de pierre de taille, par exemple, augmentera nécessaire­ment à mesure qu’augmenteront l’industrie et la population du pays environnant, surtout si elle se trouve la seule dans le voisinage. Mais la valeur d’une mine d’argent, fût-elle la seule à cinq cents lieues à la ronde, n’augmentera pas nécessairement en conséquence des progrès du pays où elle est située. Le marché, pour le produit d’une carrière de pierre de taille, ne s’étend guère au-delà de quelques milles à l’entour, et la demande en sera généralement en proportion des progrès et de la population de ce petit arrondissement ; mais le marché, pour le produit d’une mine d’argent, peut s’éten­dre à tous les pays du monde connu. Ainsi, à moins que le monde entier n’aug­men­te généralement en richesse et en population, les progrès survenus même dans une grande étendue de pays autour de la mine n’augmenteront pas la demande de l’argent ; et dans le cas même où le monde entier irait en s’enrichissant, lors même que, dans le cours de ces progrès, on découvrirait de nouvelles mines beaucoup plus fécondes qu’aucune de celles connues jusqu’alors, et bien que, dans ce cas, la demande d’argent allât toujours en augmentant, cependant il pourrait se faire que l’approvisionnement augmentât de son côté dans une proportion tellement supérieure, que le prix réel de ce métal tombât de plus en plus ; c’est-à-dire que, de plus en plus, une quantité donnée, une livre pesant, par exemple, de ce métal, ne pût acheter ou commander qu’une quan­tité de travail toujours moindre, ou s’échangeât contre une quantité toujours suc­ces­sivement plus petite de blé, la principale subsistance du travailleur.

Le grand marché pour l’argent, c’est la partie du monde civilisée et commerçante. 

Si la demande de ce marché venait à s’accroître par le progrès de l’amélioration générale, et que l’approvisionnement n’augmentât pas en même temps dans la même proportion, alors la valeur de l’argent s’élèverait successivement, par rapport à celle du blé. Une quantité donnée d’argent s’échangerait contre une quantité de blé toujours de plus en plus grande, ou, en d’autres termes, le prix moyen du blé en argent devien­drait successivement de plus en plus bas.

Si, au contraire, l’approvisionnement venait à s’accroître, par suite de quelque évé­nement, pendant plusieurs années de suite, dans une proportion beaucoup plus forte que la demande, ce métal deviendrait successivement de moins en moins cher, ou, en d’autres termes, le prix moyen du blé en argent, en dépit de toutes les améliorations possibles, deviendrait successivement de plus en plus cher.

Mais, d’un autre côté, si la quantité ou l’offre de ce métal venait à augmenter à peu près dans la même proportion que la demande, il continuerait alors à acheter ou à obtenir en échange la même ou à peu près la même quantité de blé, et le prix moyen du blé, en argent, resterait toujours à peu près le même, malgré tous les progrès.

Ces trois différents cas épuisent, à ce qu’il semble, toutes les combinaisons d’évé­ne­ments qui peuvent avoir lieu à cet égard dans le cours des progrès de l’amélioration générale ; et si nous pouvons en juger par ce qui s’est passé tant en France que dans la Grande-Bretagne, pendant le cours des quatre siècles qui ont précédé celui-ci, chacune de ces trois combinaisons différentes semble avoir eu lieu dans le marché de l’Europe, et à peu près suivant l’ordre dans lequel je viens de les exposer[20].

\section*{Digression\\
Sur les variations dans la valeur de l’argent pendant le cours des quatre derniers siècles}

\subsection*{Première période}

En 1350, et quelque temps antérieurement à cette époque, le prix moyen d’un quar­ter de froment en Angleterre n’était estimé valoir, à ce qu’il paraît, pas moins de qua­tre onces d’argent, poids de la Tour, environ 20 schellings de notre monnaie actuelle. De ce prix il paraît être tombé successivement à deux onces d’argent, égales environ à 10 schellings de notre monnaie actuelle, prix auquel nous le trouvons évalué dans le commencement du seizième siècle, et auquel il semble avoir toujours été estimé jusqu’à l’an 1570 environ.
En 1350, qui était la vingt-cinquième année d’Édouard III, fut porté le statut qu’on appelle des ouvriers. On s’y plaint beaucoup, dans le préambule, de l’insolence des domestiques, qui tâchaient de tirer de leurs maîtres une augmentation de salaires ; en conséquence, le statut ordonne que tous domestiques et ouvriers seront tenus à l’ave­nir de se contenter des mêmes salaires et livrées (on entendait alors par ce mot livrées, non-seulement les habits mais les vivres) qu’ils avaient été dans l’usage de recevoir dans la vingtième année de ce règne et les quatre années précédentes, qu’à ce compte, leur livrée en blé ne pourrait être nulle part évaluée au-dessus de 10 deniers le boisseau, et qu’il serait toujours au choix du maître de la leur payer en blé ou en argent. On regardait donc, dans la vingt-cinquième année d’Édouard III, 10 deniers le boisseau comme un prix très-modéré du blé, puisqu’il fallait un statut particulier pour forcer les domestiques à le recevoir à la place de leur livrée ordinaire de vivres, et on avait regardé ce même prix comme un prix raisonnable, dix années avant celle-ci, dans la seizième année du même règne, époque à laquelle se réfère le statut ; or, dans la seizième année d’Édouard III, 10 deniers contenaient environ une demi-once d’argent, poids de la Tour, et valaient environ une demi-couronne de notre monnaie actuelle. Ainsi, quatre onces d’argent, poids de la Tour, égales à 6 schellings 8 deniers de la monnaie d’alors, et à près de 20 schellings de celle d’aujourd’hui, étaient regardées comme un prix modéré pour le quarter de huit boisseaux.

Ce statut indique plus sûrement, sans aucun doute, ce qu’on regardait alors comme le prix modéré du grain, que ces prix de certaines années recueillies en général par les historiens et par d’autres écrivains, à cause de leur cherté ou de leur bon marché extraordinaire, et qui ne peuvent, en conséquence, servir à établir aucun jugement sur ce que peut avoir été le prix moyen. Il y a, d’ailleurs, d’autres raisons de croire que, dans le commencement du quatorzième siècle, et quelque temps encore auparavant, le prix commun du froment était au moins de quatre onces d’argent le quarter, et celui des autres grains en proportion.

En 1309, Raoul de Born, prieur de Saint-Augustin de Cantorbéry, donna, au jour de son installation, un festin dont Guillaume Thorn nous a conservé le détail, non-seulement quant au service, mais même quant aux prix des denrées. Dans ce repas on consomma : 1° 53 quarters de froment, valant 19 livres, ou bien à 7 schellings 2 de­niers le quarter, ce qui fait environ 21 schellings 6 deniers de notre monnaie actuelle ; 2° 58 quarters de drèche[21] valant 17 livres 10 schellings, ou à 6 schellings le quarter, qui en font environ 18 de notre monnaie d’aujourd’hui ; 3° 20 quarters d’avoine, valant 4 livres, ou à 4 schellings le quarter, qui font environ 12 schellings de notre monnaie actuelle. Les prix de la drèche et de l’avoine paraissent être ici au-dessus de leur proportion ordinaire avec le prix du froment.

Ces prix n’ont pas été recueillis pour leur cherté ou leur bon marché extraordi­naire, mais ils sont rapportés accidentellement comme les prix payés dans le temps pour les immenses quantités de grain consommés dans un festin qui était fameux par sa magnificence.

En 1262, la cinquante-unième année de Henri III, on fit revivre un ancien statut sur la taxe du pain et de l’ale[22], qui avait été porté, dit le roi dans le préambule, au temps de ses ancêtres, rois d’Angleterre. Ce statut est donc vraisemblablement au moins aussi ancien que le règne de son grand-père Henri II, et il peut même remonter au temps de la conquête. Il règle le prix du pain d’après les prix du blé, depuis un schelling jusqu’à 20 le quarter, argent de ce temps-là ; or, il est à présumer, en général, que les statuts de ce genre cherchent à pourvoir également à toutes les déviations du prix moyen, autant au-dessous qu’au-dessus de ce prix. D’après cette supposition, dans le temps où ce statut fut porté d’abord, et depuis cette époque jusqu’à la cinquante-unième année de Henri III, le prix moyen d’un quarter de blé serait évalué à 10 schellings contenant six onces d’argent, poids de la Tour, faisant environ 30 schellings de notre monnaie d’aujourd’hui. Ce n’est donc pas nous écarter beaucoup de la vérité, que de supposer que le prix moyen n’était pas au-dessous du tiers du plus haut prix auquel ce statut règle le prix du pain, ou moins de 6 schellings 8 deniers argent de ce temps-là, contenant quatre onces d’argent, poids de la Tour[23].

Différents faits nous autorisent, en quelque sorte, à conclure que, vers le milieu du quatorzième siècle, et encore un certain temps auparavant, le prix moyen ou ordinaire du quarter de blé n’était pas au-dessus de quatre onces d’argent, poids de la Tour.

Depuis le milieu environ du quartorzième siècle jusqu’au commencement du seizième, ce qu’on regarde comme le prix raisonnable et modéré du blé, c’est-à-dire son prix moyen ou ordinaire, paraît avoir baissé successivement jusqu’à la moitié envi­ron du prix ci-dessus, de manière à être tombé enfin à près de deux onces d’ar­gent, poids de la Tour, faisant environ 10 schellings de notre monnaie actuelle. Il est demeure a ce prix jusqu’à 1570 environ.

Dans le journal de dépense de Henri, cinquième comte de Northumberland, à la date de 1512, il y a deux différentes estimations du blé ; dans l’une, il est compté à 6 schellings 8 deniers le quarter, et dans l’autre, à 5 schellings 8 deniers seulement ; en 1512, 6 schellings 8 deniers ne contenaient que deux onces d’argent, poids de la Tour, et valaient environ 10 schellings de notre monnaie d’à présent.

D’après plusieurs différents statuts, il paraît que, depuis la vingt-cinquième année d’Édouard III, jusqu’au commencement du règne d’Élisabeth, pendant un espace de plus de deux cents ans, ce qu’on appelle le prix modéré et raisonnable du blé, c’est-à-dire son prix moyen et ordinaire, a toujours continué à s’évaluer à 6 schellings 8 de­niers. À la vérité, la quantité d’argent contenue dans cette somme nominale alla conti­nuel­lement en diminuant pendant le cours de cette période, par suite des altérations qui se firent dans la monnaie ; mais l’augmentation de la valeur de l’argent avait, à ce qu’il semble, tellement compensé la diminution de la quantité contenue dans la même somme nominale, que la législature ne pensa pas que ces altérations valussent la peine de s’en occuper.

Ainsi, en 1436, il fut statué qu’on pourrait exporter du blé sans permission, quand le prix serait descendu jusqu’à 6 sch. 8 deniers, et en 1463 il fut statué qu’on ne pourrait pas importer de blé quand le prix ne serait pas au-dessus de 6 schellings 8 deniers le quarter. La législature avait pensé que lorsque le prix était aussi bas, il n’y avait aucun inconvénient à laisser exporter ; mais lorsqu’il s’élevait plus haut, il deve­nait imprudent de permettre l’importation ; ainsi, on avait considéré, dans ce temps, que ce qu’on appelle le prix modéré et raisonnable du blé était 6 schellings 8 deniers, contenant environ la même quantité d’argent que 13 schellings 4 deniers de notre monnaie actuelle, un tiers de moins que n’en contenait la même somme nominale au temps d’Édouard III.

En 1554, par actes des première et seconde années de Philippe et Marie et, en 1558, par acte de la première Élisabeth, l’exportation du blé fut de même prohibée toutes les fois que le prix du quarter excéderait 6 schellings 8 deniers, qui, alors, ne contenaient pas pour plus de 2 deniers d’argent au-delà de ce qu’en contient aujour­d’hui la même somme nominale[24] ; mais on reconnut bientôt que c’était prohiber tout à fait l’exportation du blé, que de la restreindre au temps où le blé tomberait à un prix aussi excessivement bas. En conséquence, en 1562, par acte de la cinquième année d’Élisabeth, on permit l’exportation du blé par certains ports, toutes les fois que le prix du quarter n’excéderait pas 10 schellings, contenant à peu près la même quantité d’argent qu’en contient à présent la même somme nominale. Ce prix a donc été alors regardé comme étant ce qu’on nomme le prix modéré et raisonnable du blé. Il s’accor­de de très-près avec l’estimation du journal de Northumberland, de 1512.

En France, M. Dupré de Saint-Maur[25] et le judicieux auteur de l’Essai sur la police des grains[26], ont observé l’un et l’autre que le prix moyen du blé y avait de même été beaucoup plus bas à la fin du quinzième siècle et au commencement du seizième, que dans les deux siècles précédents. Son prix a vraisemblablement baissé de la même manière, pendant la même période, dans la majeure partie de l’Europe. 

Cette augmentation dans la valeur de l’argent, relativement à celle du blé, pourrait être attribuée entièrement à une augmentation survenue dans la demande de ce métal, en conséquence des progrès et de l’amélioration de la culture, la quantité ou l’offre demeurant toujours la même pendant ce temps-là ; ou bien elle peut être entièrement due à la diminution successive de l’approvisionnement, la plupart des mines alors con­nues dans le monde étant fort épuisées et, par conséquent, plus dispendieuses à exploiter ; ou bien, enfin, elle peut être attribuée en partie à l’une et en partie à l’autre de ces deux circonstances. Sur la fin du quinzième siècle et au commencement du seizième, la plus grande partie de l’Europe s’avançait déjà vers une forme de gouver­nement plus stable que celle dont elle avait pu jouir depuis plusieurs siècles. Une plus grande sécurité devait naturellement accroître l’industrie et tous les genres d’amé­lio­ration, et la demande des métaux précieux, comme celle de tout autre objet de luxe et d’ornement, devait naturellement augmenter à mesure de l’augmentation des richesses. Un produit annuel plus considérable exigeait, pour sa circulation, une masse d’argent plus considérable, et une plus grande quantité de gens riches deman­daient une plus grande quantité de vaisselle et autres meubles en argent. Il est aussi naturel de suppo­ser que la plupart des mines qui fournissaient alors d’argent le marché d’Europe devaient être extrêmement épuisées et que, par conséquent, leur exploitation entraî­nait plus de dépenses. Plusieurs d’entre elles avaient été exploitées dès le temps des Romains.

Cependant, la plupart de ceux qui ont écrit sur les prix des denrées, dans les temps anciens, ont été d’opinion que, depuis la conquête, peut-être même depuis l’invasion de Jules César, jusques à la découverte des mines d’Amérique, la valeur de l’argent a été continuellement en décroissant. Ils semblent avoir été amenés à cette opinion, en partie par les observations qu’ils ont eu occasion de faire sur les prix, tant du blé que de quelques autres parties du produit brut de la terre, et en partie par cette notion vulgaire que la quantité d’argent augmentant naturellement dans chaque pays à mesu­re que les richesses y augmentent, sa valeur doit diminuer à mesure qu’il augmente en quantité.

A l’égard de leurs observations sur le prix du blé, trois circonstances différentes les ont souvent induits en erreur.

D’abord, dans l’ancien temps, presque toutes les rentes se payant en nature, en une certaine quantité de blé, de bétail, de volailles, etc., il arrivait quelquefois que le pro­prié­taire stipulait avec le tenancier qu’il aurait la faculté de demander son payement annuel en nature, ou une certaine somme d’argent à la place. Le prix auquel le paye­ment en nature est ainsi échangé pour une somme d’argent s’appelle, en Écosse, prix de conversion. Comme c’est toujours au propriétaire qu’appartient l’option de prendre la chose ou le prix, il est nécessaire, pour la sûreté du tenancier, que le prix de conversion soit plutôt au-dessous qu’au-dessus du prix ordinaire du marché. Aussi, dans beaucoup d’endroits, il n’est guère au-dessus de la moitié de ce prix. Dans la majeure partie de l’Écosse, cette coutume subsiste encore à l’égard de la volaille, et dans quelques endroits à l’égard du bétail. Elle aurait aussi vraisemblablement subsisté à l’égard du blé, si l’institution des fiars ou mercuriales des marchés ne l’eût fait cesser. Ce sont des évaluations annuelles, faites au jugement d’une assise, du prix moyen de toutes les sortes de grains de différentes qualités, suivant les prix actuels du marché pour chaque comté différent. Au moyen de cette institution, le tenancier trouve assez de sûreté, et le propriétaire beaucoup plus de commodité à convertir, comme on dit, la rente de blé dans le prix des fiars de chaque année ; plutôt qu’à stipuler un prix fixe et constant ; mais les auteurs qui ont recueilli le prix du blé dans les temps anciens ont souvent pris, par erreur, pour le prix actuel du marché, ce qu’on appelle en Écosse le prix de conversion. Fleetwood reconnaît quelque part qu’il est tombé dans cette méprise ; néanmoins, comme il écrivait son ouvrage dans une autre vue, il ne jugea à propos de faire cet aveu de son erreur qu’après avoir déjà cité quinze fois ce prix de conversion, qu’il porte à 8 schellings le quarter. En 1423, l’année à laquelle il com­mence à le citer, cette somme contenait la même quantité d’argent que 16 schellings de notre monnaie actuelle ; mais, en 1562, qui est l’année où il le rapporte pour la dernière fois, cette somme de 8 schellings ne contenait pas plus d’argent que n’en contient aujourd’hui la même somme nominale.

En second lieu, ces auteurs ont été induits en erreur par la négligence avec laquelle quelques anciens statuts pour la taxe du prix des denrées ont été quelquefois transcrits par des copistes paresseux, et même quelquefois rédigés par la législature du temps.

Il paraît que les anciens statuts de taxe des denrées ont toujours commencé par déterminer quels devaient être les prix du pain et de l’ale, lorsque les prix du blé et de l’orge étaient au taux le plus bas, et qu’ils ont procédé successivement à déterminer ce que ces prix devaient être, suivant que les prix de ces deux sortes de grains vien­draient successivement à s’élever au-dessus de ce taux le plus bas. Mais les copistes qui ont transcrit ces statuts ont pensé souvent qu’il suffisait de copier seulement les articles de la taxe, qui étaient les trois ou quatre premiers des bas prix, s’épargnant par là une partie de leur peine, et jugeant, à ce que je présume, que c’en était assez pour montrer quelle proportion il fallait observer à l’égard des prix élevés.

Ainsi, dans la taxe du pain et de l’ale de la cinquante et unième année de Henri III, le prix du pain fut taxé, selon les différents prix de blé, depuis 1 schelling jusqu’à 20 le quarter, monnaie de ce temps-là ; mais, dans les manuscrits sur lesquels ont été imprimées toutes les différentes éditions des statuts, avant celle de M. Ruffhead[27], les copistes n’ont jamais transcrit les articles de la taxe au delà du prix de 12 schellings. Aussi plusieurs écrivains, induits en erreur par cette transcription inexacte, en ontils tiré la conclusion fort naturelle que le prix moyen ou ordinaire du blé, dans ce temps-là, était la moyenne de ceux énoncés au statut, c’est-à-dire 6 schellings le quarter, qui en font environ 18 de notre monnaie actuelle.

Dans le statut du tombereau[28] et du pilori, porté à peu près à la même époque, le prix de l’ale est taxé en proportion de l’élévation du prix de l’orge de 6 deniers en 6 deniers, depuis 2 jusques à 4 schellings le quarter. Cependant ces 4 schellings n’étaient pas regardés comme le plus haut prix que l’orge pût atteindre dans ce temps ; et ce qui peut bien nous porter à croire que ces prix n’étaient donnés que comme un exemple de la proportion à observer pour tous les autres prix, soit plus élevés, soit plus bas, ce sont ces derniers mots du statut : Et sic deinceps crescetur vel diminuetur per sex denarios. L’expression est fort peu élégante, mais la signification est assez claire : « Qu’il faut ainsi élever ou diminuer le prix de l’ale par chaque 6 deniers de hausse ou de baisse que le prix de l’orge viendra à subir. » Il paraît que la législature elle-même a mis aussi peu de soin à la rédaction de ce statut que les copistes en avaient mis à transcrire l’autre.

Dans un vieux manuscrit du Regiam Majestatem, qui est un ancien livre du roi d’Écosse, il y a un statut dans lequel est taxé le prix du pain d’après tous les différents prix du blé, depuis 10 deniers jusques à 3 schellings le boll d’Écosse, qui fait environ moitié du quarter anglais. Au temps où cette taxe paraît avoir été faite, 3 sous d’Écosse valaient environ 6 schellings sterling de notre monnaie actuelle. Il paraît que M. Rudiman[29] en a conclu que 3 sous étaient le plus haut prix auquel le blé s’élevât jamais dans ce temps-là, et que le prix ordinaire était de 10 à 12 deniers, tout au plus de 2 schellings. Cependant il est bien évident, en consultant le manuscrit, que tous ces prix ne sont mis là que comme des exemples de la proportion à observer entre les prix respectifs du blé et du pain. Les derniers mots du statut sont : Reliqua judicabis secundum prœscripta, habendo respectum ad preetum bladi. « Vous jugerez dans les autres cas d’après ce qui a été marqué ci-dessus, ayant égard au prix du blé. »

En troisième lieu, ces auteurs paraissent encore avoir été induits en erreur par le très-bas prix auquel le blé s’est quelquefois vendu dans les temps fort anciens, et il paraît qu’ils se sont imaginé que son plus bas prix ayant été alors beaucoup moindre qu’il ne l’est dans les temps postérieurs, son prix ordinaire doit pareillement avoir été bien plus bas. Ils auraient pourtant dû s’apercevoir que, dans ces temps reculés, le plus haut prix s’élevait au-dessus de tout ce qu’on a pu connaître des temps subséquents, autant que le plus bas prix descendait au-dessous. Ainsi, en 1270, Fleetwood nous donne deux prix du quarter de blé : l’un est de 4 livres 16 schellings, argent de ce temps-là, faisant 14 livres 8 schellings de celui d’aujourd’hui ; l’autre est de 6 livres 6 schellings, faisant 19 livres 4 schellings de notre monnaie actuelle[30]. On ne trouve rien qui approche de ces prix excessifs, à la fin du quinzième ou au commencement du seizième siècle. Quoique en tout temps le prix du blé soit sujet à des variations, cependant il varie infiniment plus dans ces sociétés, livrées aux troubles et aux désor­dres, où l’interruption de tout commerce et de toute communication empêche que l’abondance d’une province ne vienne suppléer à la disette de l’autre. Dans l’état de confusion où était l’Angleterre sous les Plantagenets, qui la gouvernèrent depuis le milieu environ du douzième siècle jusque vers la fin du quinzième, un canton pouvait se trouver dans l’abondance, tandis qu’un autre, qui n’était pas très-éloigné, ayant eu sa récolte détruite par quelque accident naturel, ou ravagée par les excursions d’un baron voisin, souffrait toutes les horreurs de la famine ; et cependant, s’ils étaient séparés par les terres de quelque seigneur ennemi, l’un d’eux ne pouvait pas donner le moindre secours à l’autre. Sous l’administration vigoureuse des Tudors, qui gouvernèrent l’An­gle­terre pendant le reste du quinzième siècle et dans tout le cours du seizième, il n’y avait pas de baron assez puissant pour oser troubler la tranquillité publique.

Le lecteur trouvera à la fin de ce chapitre tous les prix du blé, qui ont été recueillis par Fleetwood, depuis 1202 inclusivement jusques à 1597 aussi inclusivement, réduits au cours de la monnaie actuelle, et disposés, suivant l’ordre des temps, en sept séries de douze années chacune. Il trouvera aussi à la fin de chaque série le prix moyen des douze années qui la composent. Fleetwood n’a pu recueillir, dans cette longue période, que les prix de quatre-vingts années seulement, de manière que, pour com­pléter la dernière série, il ne manquerait que quatre années ; j’ai donc ajouté, d’après les comptes du collège d’Eton, les prix des années 1598, 1599, 1600 et 1601 ; c’est la seule addition que j’aie faite. Le lecteur verra que, depuis le commencement du trei­zième siècle, jusque passé le milieu du seizième, le prix moyen de chaque série de douze années va successivement en baissant de plus en plus, et que, vers la fin du seizième siècle, il commence à se relever. Il est vrai que les prix que Fleetwood a pu recueillir paraissent être principalement ceux qui ont été remarquables par leur bon marché ou leur cherté extraordinaire, et je ne pense pas qu’on en puisse tirer aucune conclusion bien décisive.

Cependant, s’ils prouvent quelque chose, ils confirment absolument la proposition que j’ai cherché à établir. Fleetwood lui-même paraît néanmoins avoir pensé, avec la plupart des autres écrivains, que, pendant toute cette période, l’argent, attendu son abondance sans cesse croissante, a été toujours en diminuant de valeur. Les prix du blé, qu’il a recueillis lui-même, ne s’accordent certainement pas avec cette opinion. Ils s’accordent parfaitement avec celle de M. Dupré de Saint- Maur, et avec celle que j’ai tâché de démontrer. L’évêque Fleetwood et M. Dupré de Saint-Maur sont les deux auteurs qui semblent avoir recueilli, avec le plus de soin et de fidélité, les prix des choses dans les temps anciens. Il est assez remarquable que, malgré la grande diffé­rence de leurs opinions, les faits recueillis par chacun d’eux se trouvent coïncider avec tant d’exactitude, au moins pour ce qui regarde les prix du blé.

Cependant c’est moins du bas prix du blé que de celui de quelques autres parties du produit brut de la terre, que les écrivains les plus judicieux ont inféré la grande valeur qu’ils attribuent à l’argent dans ces anciens temps. Le blé, ont-ils dit, étant une espèce de produit de main-d’œuvre, a été, dans ces temps grossiers, beaucoup plus cher, en proportion, que la plupart des autres marchandises ; je présume qu’ils ont voulu dire la plupart des autres marchandises qui n’étaient pas produit de main-d’œuvre, telles que le bétail, la volaille, le gibier de toute espèce, etc. En effet, que dans ces temps de pauvreté et de barbarie ces sortes de choses fussent en proportion à beaucoup meilleur marché que le blé, c’est une vérité indubitable ; mais ce bon marché n’était pas l’effet de la haute valeur de l’argent, mais bien du peu de valeur de ces denrées. Ce n’était pas que dans ce temps-là l’argent fût en état d’acheter ou de repré­sen­ter une plus grande quantité de travail, mais c’est que ces sortes de denrées n’en pouvaient acheter ou représenter qu’une quantité beaucoup plus petite que dans les temps où la richesse et l’industrie eurent fait plus de progrès. L’argent doit certai­ne­ment être à meilleur marché dans l’Amérique espagnole que dans l’Europe ; dans le pays qui le produit, que dans celui où on l’apporte chargé de la dépense d’un long transport tant par terre que par mer, de celle du chargement et de l’assurance. Cepen­dant, il n’y a pas beaucoup d’années qu’à Buenos-Ayres, à ce que nous dit Ulloa, 21 deniers et demi sterling étaient le prix d’un bœuf choisi dans un troupeau de trois ou quatre cents têtes. M. Byron nous rapporte que, dans la capitale du Chili, le prix d’un bon cheval est de 16 schellings sterling. Dans un pays naturellement fertile, mais dont la plus grande partie est tout à fait inculte, le bétail, la volaille, le gibier de toute espè­ce peuvent s’acquérir au moyen d’une très-petite quantité de travail ; il en résulte qu’ils ne peuvent en acheter ou en commander qu’une très-petite quantité. Le très-bas prix auquel ils y sont vendus en argent n’est pas une preuve que la valeur de l’argent y soit très-haute, mais c’est une preuve que la valeur de ces marchandises y est fort basse. 

Il faut toujours se rappeler que c’est le travail, et non aucune autre marchandise ou classe de marchandises particulières, qui est la mesure réelle de valeur, tant pour l’argent que pour toute autre marchandise quelconque[31].

Mais dans des pays presque déserts ou très-peu habités, le bétail, la volaille, le gibier de toute espèce, etc., étant des productions spontanées de la nature, s’y multi­plient souvent en beaucoup plus grande quantité que ne l’exige la consommation des habitants. Dans un tel état de choses, l’offre excède communément la demande. Ainsi, ces sortes de denrées, suivant les différents états où sera la société, suivant les diffé­rents degrés d’opulence ou d’industrie où elle se trouvera, représenteront ou vaudront des quantités de travail fort différentes.

Quel que soit l’état de la société, quel que soit son degré de civilisation, le blé est toujours une production de l’industrie des hommes. Or, le produit moyen de toute espèce d’industrie se subordonne toujours avec plus ou moins de précision à la con­som­mation moyenne, la quantité moyenne de l’approvisionnement à la quantité moyen­ne de la demande ; d’ailleurs, dans les différents degrés d’amélioration d’un pays, il faudra toujours, en moyenne, des quantités de travail à peu près égales, pour faire croître des quantités égales de blé dans un même sol et sous un même climat, l’augmentation continuelle qui a lieu dans la puissance productive du travail, à mesure que la culture va en se perfectionnant, étant plus ou moins contre-balancée par l’ac­crois­se­ment continuel du prix des bestiaux, qui sont les principaux instruments de l’agriculture. Nous devons donc, d’après ceci, être bien certains qu’en tout état possible de la société, dans tout degré de civilisation, des quantités égales de blé seront une représentation ou un équivalent plus juste de quantités égales de travail, que ne le seraient des quantités égales de toute autre partie du produit brut de la terre. En conséquence, le blé, ainsi qu’il a été déjà observé, est, dans tous les différents degrés de richesse et d’amélioration de la société, une mesure de valeur plus exacte que toute autre marchandise ou que toute autre classe de marchandises ; ainsi, dans tous ces différents degrés, nous pouvons mieux juger de la valeur réelle de l’argent, en le comparant avec le blé, qu’en prenant pour objet de comparaison une autre marchandise quelconque ou plusieurs autres sortes de marchandises conjointement[32].

En outre, le blé ou tout autre végétal faisant la nourriture ordinaire et favorite du peuple, constitue, dans tout pays civilisé, la principale partie de la subsistance du travailleur. Par suite de l’extension de la culture, la terre d’un pays quelconque produit une bien plus grande quantité de nourriture végétale que de nourriture animale, et partout l’ouvrier se nourrit principalement de l’aliment qui joint à la salubrité l’avan­tage d’être le plus abondant et le moins cher. Excepté dans les contrées les plus florissantes, et dans lesquelles le travail est le plus libéralement récompensé, la viande de boucherie n’est qu’une bien faible partie de la subsistance de l’ouvrier ; la volaille en est encore une bien moindre, et le gibier n’y entre pour rien. En France, et même en Écosse, où le travail est un peu mieux rétribué qu’en France, l’ouvrier pauvre ne mange guère de viande que les jours de fêtes et dans quelques circonstances extra­ordinaires. Le prix du travail en argent dépend donc beaucoup plus du prix moyen du blé, qui est la subsistance de l’ouvrier, que de celui de la viande ou de toute autre partie du produit brut de la terre ; par conséquent, la valeur réelle de l’or et de l’argent, la quantité réelle de travail qu’ils peuvent acheter ou commander, dépend beaucoup plus de la quantité qu’ils peuvent acheter ou représenter, que de celle de viande ou de toute espèce de produit brut dont ils pourraient disposer.

Cependant, des observations aussi peu approfondies sur les prix du blé ou des autres denrées n’auraient vraisemblablement pas induit en erreur tant d’auteurs éclai­rés, si elles ne se fussent pas trouvées conformes à cette notion vulgaire, que la quantité d’argent augmentant naturellement dans un pays où la richesse augmente, ce métal doit diminuer de valeur à mesure qu’il augmente en quantité. Cette notion paraît pourtant tout à fait dénuée de fondement.

Deux causes différentes peuvent augmenter dans un pays la quantité des métaux précieux. La première, c’est une augmentation dans l’abondance des mines qui en four­nissent à ce pays ; la seconde, c’est un accroissement dans la richesse du peuple, une augmentation du produit annuel de son travail. Sans nul doute, la première de ces deux causes entraîne nécessairement avec elle une diminution dans la valeur des métaux précieux, mais non pas la seconde.

Quand des mines plus abondantes viennent à être découvertes, une plus grande quantité de métaux précieux est apportée au marché, et la quantité des autres choses propres aux besoins et aux commodités de la vie contre laquelle ils doivent s’échan­ger, étant la même qu’auparavant, des quantités égales de ces métaux s’échangeront nécessairement contre des quantités plus petites de ces choses. Ainsi, l’augmentation de la quantité des métaux précieux dans un pays, en tant qu’elle provient d’une plus grande abondance dans les mines, emporte de toute nécessité avec elle quelque diminution dans la valeur de ces métaux.

Quand, au contraire, la richesse d’un pays augmente, quand le produit annuel de son travail devient successivement de plus en plus considérable, il lui faut nécessai­re­ment une plus grande quantité d’argent monnayé pour faire circuler cette plus grande quantité de marchandises ; d’un autre côté, les gens de ce pays achèteront naturelle­ment de la vaisselle d’argent et d’autres ouvrages d’orfèvrerie en quantités de plus en plus fortes, à mesure qu’ils se trouveront en état de faire cette dépense et qu’ils auront à leur disposition plus de marchandises pour la payer. La quantité de leur monnaie augmentera pour cause de nécessité ; celle de leur orfèvrerie pour cause de vanité et d’ostentation, ou pour la même raison qui fera que les belles statues, les tableaux et tous les autres objets de luxe et de curiosité deviendront probablement en plus grand nombre parmi eux. Mais comme il n’est pas vraisemblable que les peintres et les sculpteurs soient plus mal payés dans des temps de richesse et de prospérité que dans des temps de pauvreté et de décadence, de même il n’est pas vraisemblable que l’or et l’argent soient aussi moins bien payés[33].

Comme le prix de l’or et de l’argent hausse naturellement dans une nation à mesure qu’elle s’enrichit, à moins que la découverte accidentelle de mines plus abondan­tes ne le fasse baisser, il s’ensuit que, quel que puisse être l’état des mines, ce prix sera naturellement plus élevé dans un pays riche que dans un pays pauvre. L’or et l’argent, comme tout autre marchandise, cherchent naturellement le marché où l’on en offre le meilleur prix et, pour quelque denrée que ce soit, le meilleur prix sera toujours offert par le pays qui est le plus en état de le donner. Le travail, comme il faut toujours se le rappeler, est le prix qui, en dernière analyse, paye tout, et dans deux pays où le travail sera également bien salarié, le prix du travail en argent sera en proportion du prix de la subsistance de l’ouvrier ; or, l’or et l’argent s’échangeront naturellement contre une plus grande quantité de subsistances dans un pays riche que dans un pays pauvre ; dans un pays où les subsistances abondent, que dans un pays qui n’est que médiocre­ment fourni. Si les deux pays sont à une grande distance l’un de l’autre, la différence pourra être très-grande, parce que, quoique les métaux soient naturellement attirés du marché le moins avantageux vers celui où ils trouvent le plus d’avantages, cependant il peut y avoir de la difficulté à les y transporter en quantité suffisante pour que les prix s’équilibrent à peu près dans les deux marchés. Si les pays sont rapprochés, la différence sera moindre, et quelquefois même elle sera à peine sensible, parce que, dans ce cas, le transport sera facile. La Chine est un pays bien plus riche qu’aucune contrée de l’Europe[34], et la différence du prix des subsistances est très-grande entre la Chine et l’Europe. Le riz est à beaucoup meilleur marché en Chine que ne l’est le blé en aucun lieu de l’Europe. L’Angleterre est beaucoup plus riche que l’Écosse ; mais la différence du prix du blé en argent entre ces deux pays est beaucoup moindre, et elle est à peine sensible. Eu égard à la quantité ou mesure, le blé d’Écosse paraît, en général, à bien meilleur marché que le blé d’Angleterre ; mais eu égard à la qualité, il est certainement un peu plus cher. L’Écosse tire presque tous les ans de l’Angleterre de grosses provisions de blé, et il faut bien qu’une marchandise soit un peu plus chère dans le pays où on l’apporte, que dans celui d’où elle vient[35]. Le blé d’Angleterre est donc nécessairement plus cher en Écosse qu’en Angleterre même, et cependant, en proportion de sa qualité ou de la quantité et bonté de la farine qu’on peut en retirer, il ne peut pas s’y vendre communément à plus haut prix que le blé d’Écosse, qui vient avec lui en concurrence au marché.

La différence du prix du travail en argent entre la Chine et l’Europe est encore bien plus forte que celle du prix des subsistances en argent, parce que la récompense réelle du travail est plus élevée en Europe qu’à la Chine, la plus grande partie de l’Europe étant dans un état de progrès, pendant que la Chine semble rester toujours au même point. Le prix du travail en argent est plus élevé en Écosse qu’en Angleterre, parce que la récompense réelle du travail y est beaucoup moindre, l’Écosse, quoique dans un état de progrès vers une plus grande opulence, avançant néanmoins beaucoup plus lentement que l’Angleterre. Une bonne preuve que la demande du travail est fort différente dans ces deux pays, c’est la quantité de personnes qui émigrent d’Écosse, et le peu qui émigrent d’Angleterre. Il faut se rappeler que ce qui règle naturellement la proportion de la récompense réelle du travail entre différents pays, ce n’est pas leur richesse ou leur pauvreté actuelle, mais leur condition progressive, stationnaire ou rétrograde,

Comme l’or et l’argent, n’ont nulle part naturellement plus de valeur que parmi les nations les plus riches, ils n’en ont aussi nulle part moins que parmi les plus pauvres. Chez les nations sauvages, les plus pauvres de toutes, ces métaux ont à peine une valeur. 

Le blé est toujours plus cher dans une grande ville que dans les campagnes éloi­gnées. Cependant, ce n’est pas que l’argent y soit à meilleur marché, mais c’est que le blé y est réellement plus cher. Il n’en coûte pas moins de travail pour apporter l’argent à une grande ville qu’aux campagnes éloignées, mais il en coûte beaucoup plus de travail pour y apporter le blé.

Le blé est cher dans quelques pays riches et commerçants, tels que la Hollande et le territoire de Gênes, par la même raison qu’il est cher dans une grande ville. Ces pays ne produisent pas de quoi nourrir leurs habitants ; leur richesse consiste dans l’industrie et l’habileté de leurs artisans manufacturiers, dans une foule de machines et d’instruments de toute espèce, propres à faciliter et abréger le travail ; dans leurs navires et dans tout l’attirail qui augmente les moyens de transport et de commerce ; mais ces pays sont pauvres en blé, lequel se trouve nécessairement chargé, en sus de son prix, du prix du transport des endroits éloignés d’où il faut absolument le faire venir. Il n’en coûte pas moins de travail pour apporter de l’argent à Amsterdam qu’à Dantzick, mais il en coûte bien plus de travail pour y apporter du blé. Le coût réel de l’argent doit être à peu près le même dans ces deux places, mais celui du blé y doit être très-différent. Diminuez l’opulence réelle de la Hollande ou du territoire de Gênes, le nombre des habitants y restant toujours le même ; diminuez le pouvoir qu’ont ces pays de payer des approvisionnements au loin, et vous verrez que, bien loin de baisser avec cette diminution dans la quantité de l’argent, laquelle., soit comme cause, soit comme effet, accompagnera nécessairement cet état de décadence, le prix du blé va s’y élever au taux d’une famine. Quand nous venons à manquer des choses nécessaires, il faut bien alors renoncer aux choses superflues ; et de même que la valeur de celles-ci hausse dans le temps de prospérité et d’opulence, de même elle baisse dans les temps de pauvreté et de détresse. Il en est autrement des choses néces­saires. Leur prix réel, la quantité de travail qu’elles peuvent commander ou acheter, s’élève dans les temps de pauvreté et de détresse, et baisse dans les temps d’opulence et de prospérité, qui sont toujours des moments de grande abondance, sans quoi ils ne seraient pas des temps d’opulence et de prospérité. Le blé est la chose nécessaire, l’argent est la chose superflue.

Ainsi, quelle qu’ait pu être l’augmentation de quantité dans les métaux précieux, survenue pendant cette première période du milieu du quatorzième siècle au milieu du seizième, en conséquence d’un accroissement de richesse et d’amélioration, cette aug­men­tation n’a pu tendre à diminuer la valeur de ces métaux dans la Grande-Bretagne ou dans toute autre partie de l’Europe. Par conséquent, si ceux qui ont recueilli les prix des denrées dans les temps anciens n’ont pas eu raison d’inférer des observations qu’ils ont faites sur les prix du blé ou des autres marchandises, qu’il y ait eu, pendant cette période, une diminution dans la valeur de l’argent, ils sont encore bien moins fondés à l’inférer de cet accroissement de richesse et d’amélioration qu’ils supposent avoir eu lieu.

\subsection*{Deuxième période}

Autant les opinions des écrivains ont varié sur les progrès de la valeur de l’argent pendant cette première période, autant elles se trouveront unanimes sur ces progrès pendant la seconde.

De 1570 environ, jusqu’à 1640, pendant une période d’environ soixante-dix ans, la proportion entre la valeur de l’argent et celle du blé a varié dans un sens tout à fait opposé.

L’argent a baissé dans sa valeur réelle, ou s’est échangé contre une moindre quan­tité de travail qu’auparavant, et le blé s’est élevé dans son prix nominal, et au lieu de se vendre communément environ 2 onces d’argent le quarter, ou environ 10 schellings de notre monnaie actuelle, il s’est vendu jusqu’à 6 et 8 onces d’argent, c’est-à-dire environ 30 et 40 schellings de notre monnaie actuelle.

Cette diminution de la valeur de l’argent, relativement à celle du blé, ne paraît pas avoir eu d’autre cause que la découverte des mines abondantes de l’Amérique. Aussi, chacun en donne-t-il la même raison, et il n’y a jamais eu, à cet égard, de dispute ni sur le fait ni sur sa cause. Pendant cette période, la majeure partie de l’Europe avan­çait en industrie et en opulence et, par conséquent, la demande d’argent y doit avoir été toujours en augmentant. Mais l’augmentation de l’approvisionnement, à ce qu’il semble, a tellement surpassé celle de la demande, que la valeur de ce métal n’en a pas moins baissé considérablement. Il faut observer que la découverte des mines de l’Amérique ne paraît pas avoir influé d’une manière sensible sur le prix des choses en Angleterre, avant 1570, quoiqu’il y eût déjà plus de vingt ans que les mines, même du Potosi, fussent découvertes.

De 1595 inclusivement à 1620 aussi inclusivement, le prix moyen du quarter de neuf boisseaux du plus beau froment au marché de Windsor paraît, suivant les comp­tes du collège d’Eton, avoir été de 2 livres 1 schelling 6 deniers 9/13. En négligeant la fraction, et déduisant de cette somme un neuvième ou 4 schellings 7 deniers 1/3, le prix du quarter de 8 boisseaux serait revenu à 1 livre 16 schellings 10 deniers 2/3. Et en négligeant encore la fraction, et déduisant de cette somme un neuvième ou 4 schellings 1 denier 1/9 pour la différence de prix entre le plus beau froment et celui de qualité moyenne, le prix de ce dernier serait revenu à environ 1 livre 12 schellings 8 deniers 8/9, ou environ 6 onces et un tiers d’once d’argent.

De 1621 inclusivement à 1636 aussi inclusivement, le prix moyen de la même mesure du plus beau froment au même marché, d’après les mêmes comptes, paraît avoir été de 2 livres 10 schellings, de laquelle somme faisant les mêmes déductions que dans le cas précédent, le prix moyen du quarter de huit boisseaux de blé de moyenne qualité serait revenu à 1 livre 19 schellings 6 deniers, ou à environ 7 onces et deux tiers d’once d’argent.

\subsection*{Troisième période}

C’est entre 1630 et 1640, ou vers l’année 1636, que la découverte des mines de l’Amérique paraît avoir exercé tout son effet sur la réduction de la valeur de l’argent, et il paraît que la valeur de ce métal n’a jamais autant baissé, relativement à celle du blé, qu’elle l’a fait à cette époque. Elle paraît s’être relevée de quelque chose dans le cours de ce siècle ; elle avait même vraisemblablement commencé à hausser quelque temps avant la fin du siècle dernier.

De 1637 inclusivement à 1700 aussi inclusivement, dans les soixante-quatre dernières années du siècle passé, le prix moyen du quarter de neuf boisseaux du plus beau froment au marché de Windsor, d’après les mêmes comptes d’Eton, paraît avoir été de 2 livres 11 schellings 0 denier 1/3 ; ce qui est seulement un schelling et un tiers de denier plus cher qu’il n’avait été pendant les seize années précédentes ; mais dans le cours de ces soixante-quatre années, il était arrivé deux événements qui ont dû pro­duire une rareté du blé beaucoup plus grande que celle qu’aurait sans cela occasionnée la seule influence des saisons, et qui sont plus que suffisants pour expliquer ce petit renchérissement, sans qu’il soit besoin de supposer une réduction ultérieure dans la valeur de l’argent. 

Le premier de ces événements fut la guerre civile, qui, en décourageant l’agri­culture et en interrompant le commerce, a dû élever le prix du blé fort au-dessus de ce qu’aurait pu faire sans cela l’intempérie des saisons. Elle doit avoir produit cet effet, plus ou moins, sur tous les différents marchés du royaume, mais particulièrement sur ceux du voisinage de Londres, qui sont obligés de s’approvisionner dans une étendue de terrain plus grande. Aussi, en 1648, d’après les mêmes comptes, le plus beau froment, au marché de Windsor, paraît-il avoir coûté jusqu’à 4 livres 5 schellings le quarter de neuf boisseaux, et en 1649 jusqu’à 4 livres. Ces deux années ensemble font un excédent de 3 livres 5 schellings sur les 2 livres 10 schellings, prix moyen des seize années antérieures à 1637, lequel excédent, réparti sur les soixante-quatre dernières années du siècle dernier, suffirait presque seul pour rendre raison du léger renchérissement qui paraît s’y faire remarquer. Cependant ces deux prix, quoique les plus hauts, ne sont pas les seuls hauts prix que les guerres civiles paraissent avoir occasionnés.

Le second de ces événements, ce fut la prime accordée, en 1688, à l’exportation du blé. Beaucoup de gens se sont figuré que la prime, en encourageant la culture, pouvait, dans une longue suite d’années, occasionner une plus grande abondance et, par conséquent, une plus grande diminution du prix du blé dans le marché intérieur que celle qui aurait eu lieu sans cela. J’examinerai dans la suite[36] jusques à quel point les primes peuvent, dans aucun temps, produire un semblable effet ; j’observerai seulement, pour le présent, qu’elles n’auraient pas eu le temps de produire cet effet de 1688 à 1700. Pendant cette courte période, le seul effet qu’elles aient dû produire, c’est d’avoir fait monter le prix du blé sur le marché intérieur, en encourageant l’ex­por­tation du produit surabondant de chaque année, et en empêchant par là que la disette d’une année ne se trouvât compensée par l’abondance d’une autre. La prime doit avoir ajouté quelque chose à l’effet de la disette qui a eu lieu en Angleterre de 1693 inclusivement jusques à 1699 aussi inclusivement, quoique cette disette doive être sans doute attribuée principalement aux mauvaises saisons et que, par consé­quent, elle se soit fait sentir dans une partie considérable de l’Europe. Aussi, en 1699 l’exportation du blé fut-elle prohibée pour neuf mois.

Il y a encore un troisième événement qui eut lieu dans le cours de la même période et qui, sans occasionner sans doute aucune rareté dans le blé, ni peut-être aucune augmentation dans la quantité réelle d’argent communément payée pour le prix du blé, doit avoir nécessairement amené quelque augmentation dans la somme nominale de ce prix. Cet événement, ce fut la grande dégradation causée dans la monnaie d’argent par le frai et par les rogneurs. Ce mal commença sous le règne de Charles II, et alla toujours en augmentant jusques en 1693, auquel temps, à ce que nous dit M. Lowndes, la monnaie d’argent courante était, en moyenne, à près de 25 pour 100 au-dessous de sa valeur légale. Mais la somme nominale qui constitue le prix de marché des choses se règle nécessai­rement bien moins sur la quantité d’argent que la monnaie devrait contenir d’après la loi, que sur celle que cette monnaie contient alors effectivement, autant qu’on en peut juger par l’expérience. Cette somme nominale doit donc être nécessairement plus forte, quand la monnaie est dégradée par le frai et par les rogneurs, que quand elle approche de sa valeur légale.

Dans le cours de ce siècle, la monnaie d’argent n’a été dans aucun temps aussi fort au-dessous de son poids légal, qu’elle l’est aujourd’hui. Mais, tout usée qu’elle est, sa valeur est soutenue par la monnaie d’or contre laquelle on peut l’échanger ; car, quoi­que la monnaie d’or fût aussi très-usée avant la dernière refonte, elle l’était beaucoup moins que la monnaie d’argent. En 1695, au contraire, la valeur de la monnaie d’argent n’était pas soutenue par la monnaie d’or, une guinée s’échangeant alors com­mu­­né­ment contre 30 schellings de cette monnaie usée et rognée. Avant la dernière refonte de la monnaie d’or, le prix du lingot d’argent ne s’élevait guère au-dessus de 5 sch. 7 deniers l’once, ce qui n’est que 5 deniers au-dessus du prix qu’on en donne à la Monnaie[37]. Mais en 1695, le lingot d’argent était communément à 6 sch. 5 deniers l’once[38], ce qui fait 15 deniers au-dessus du prix pour lequel il était reçu à la Monnaie. Ainsi, même avant la dernière refonte de la monnaie d’or, les monnaies d’or et d’argent, prises ensemble, n’étaient pas estimées à plus de 8 pour 100 au-dessous de leur valeur légale, en les comparant avec le prix du lingot d’argent. Au contraire, en 1695, elles étaient estimées à 25 pour 100 au-dessous de cette valeur. Mais au com­men­cement de ce siècle, immédiatement après la grande refonte, sous le roi Guil­laume, la majeure partie de la monnaie courante doit avoir été beaucoup plus près de son poids légal qu’elle ne l’est aujourd’hui. Dans le cours de ce siècle, il n’y a pas eu non plus de grande calamité publique, telle qu’une guerre civile, qui ait pu décourager la culture ou interrompre le commerce au-dedans. Et quoique la prime, qui a eu son effet dans la plus grande partie de ce siècle, ait toujours élevé nécessaire­ment le prix du blé un peu plus haut qu’il n’eût été sans elle (l’état de la culture supposé le même dans les deux cas), néanmoins, comme dans le cours de ce siècle la prime a eu tout le temps de produire tous les bons effets qu’on lui attribue communé­ment, comme d’en­cou­rager le labourage et d’augmenter par là la quantité de blé dans le marché intérieur, on peut supposer, d’après les principes d’un système que j’expose ici et que j’examinerai dans la suite[39], qu’elle a contribué en quelque chose à faire baisser le prix de cette denrée, aussi bien qu’elle a contribué d’un autre côté à l’élever. Beaucoup de gens prétendent qu’elle a plus agi dans le premier sens que dans l’autre. En consé­quence, dans les soixante-quatre premières années de ce siècle, le prix moyen du quarter de neuf boisseaux du plus beau froment, au marché de Windsor, suivant les comptes du collège d’Eton, paraît avoir été de 2 liv. 0 schelling 6 den. 19/32, ce qui fait environ 10 schellings 6 den., ou plus de 25 pour 100 meilleur marché qu’il n’avait été pendant les soixante-quatre dernières années du siècle précédent ; environ 9 schellings 6 den. meilleur marché qu’il n’avait été pendant les seize années antérieures à 1636, lorsque la découverte des mines abondantes de l’Amérique avait, on doit le supposer, produit tout son effet ; et environ 1 schelling meilleur marché qu’il n’avait été dans les vingt-six années antérieures à 1620, avant qu’on puisse raisonnablement supposer que cette découverte eût pleinement produit son effet. À ce compte, le prix moyen du froment de moyenne qualité, pendant les soixante-quatre premières années de ce siècle, a dû revenir à environ 32 schellings le quarter de huit boisseaux.

Ainsi, la valeur de l’argent paraît s’être élevée de quelque chose relativement à celle du blé pendant le cours de ce siècle, et elle a vraisemblablement commencé à le faire quelque temps avant la fin du siècle dernier. 

En 1687, le prix du quarter de neuf boisseaux du plus beau froment, au marché de Windsor, a été de 1 livre 5 sch. 2 deniers, le plus bas prix auquel on l’eût jamais vu depuis 1595.

En 1688, M. Grégoire King, homme célèbre par ses connaissances dans ces sortes de matières, estimait que le prix moyen du blé, dans les années d’abondance moyen­ne, était, pour le producteur, de 3 schellings 6 deniers le boisseau, ou 28 schellings le quarter. J’imagine que ce prix du producteur est le même que celui qu’on nomme quelquefois prix de contrat, ou le prix auquel un fermier s’engage, pour un certain nombre d’années, à livrer à un marchand une certaine quantité de blé. Comme un contrat de ce genre épargne au fermier la peine et la dépense de courir les marchés, le prix de contrat est en général, plus bas que le prix de marché ordinaire. M. King a juré que le prix du contrat ordinaire, dans les années d’abondance moyenne, était, à cette époque, de 28 schellings le quarter. Avant la cherté occasionnée par la suite extraordinaire de mauvaises années que nous venons d’avoir, c’était là le prix ordi­naire de contrat dans toutes les années communes.

En 1688, le Parlement accorda la prime à l’exportation du blé. Les propriétaires fonciers[40], qui étaient en beaucoup plus grand nombre qu’à présent dans la législature, s’étaient aperçus que le prix du blé en argent commençait à baisser. La prime était un expédient pour le faire monter artificiellement à ce haut prix auquel il s’était souvent vendu sous les règnes de Charles Ier et de Charles II. Il fut donc établi qu’elle aurait lieu jusqu’à ce que le blé fût monté à 48 sch. le quarter, ce qui était 20 sch. ou 5/7 plus cher que M. King n’avait, cette même année, évalué le prix du producteur, dans les temps d’abondance moyenne. Si ses calculs méritent tant soit peu la réputation qu’on leur accorde généralement, 48 schell. le quarter étaient un prix qu’à moins de quelque expédient, tel que la prime, on ne devait pas attendre alors, excepté dans les années de cherté extraordinaire. Mais le gouvernement du roi Guillaume n’était pas encore bien solidement établi ; il n’était pas en mesure de refuser quelque chose aux propriétaires fonciers, auprès desquels il sollicitait dans ce temps même le premier établissement de la taxe foncière annuelle[41].

Ainsi, la valeur de l’argent, relativement à celle du blé, s’est probablement un peu élevée avant la fin du siècle dernier, et elle semble avoir continué à s’élever pendant le cours de la majeure partie de ce siècle, quoique la prime ait dû avoir nécessairement l’effet de rendre cette hausse moins sensible qu’elle ne l’eût été sans cela, dans l’état actuel de la culture.

Dans les années d’abondance, la prime, en occasionnant une exportation extraor­dinaire, élève nécessairement le prix du blé au-dessus de ce qu’il serait sans cela dans ces années. Le but avoué de cette institution, c’était d’encourager le labourage en tenant le blé à un bon prix, même dans les années de la plus grande abondance.

Dans les années de cherté, il est vrai, la prime était suspendue ; cependant, elle n’en a pas moins produit son effet, même sur les prix de la plupart de ces années. Au moyen de l’exportation extraordinaire qu’elle occasionne dans les années d’abondance, elle doit souvent empêcher que l’abondance d’une année ne compense la disette de l’autre.

Ainsi, la prime élève le prix du blé, tant dans les années abondantes que dans les mauvaises, au-delà du prix où il s’arrêterait naturellement, dans l’état où est alors la culture. Si donc, pendant les soixante-quatre premières années de ce siècle, le prix moyen a été plus bas que pendant les soixante-quatre dernières du précédent, il l’aurait été encore bien davantage sans l’effet de la prime, en supposant que l’état de la culture fût resté le même.

Mais, dira-t-on, sans la prime, l’état de la culture n’eût pas été le même. Quand j’en serai à traiter, dans la suite, des primes en particulier, je tâcherai d’expliquer quels peuvent avoir été les effets de cette institution sur l’agriculture. je me contenterai, pour le moment, d’observer que cette hausse dans la valeur de l’argent, relativement à celle du blé, n’a pas été particulière à l’Angleterre. Trois écrivains qui ont recueilli avec beaucoup de soin et d’exactitude le prix du blé en France, M. Dupré de Saint-Maur, M. Messance et l’auteur de l’Essai sur la police des grains, ont tous observé cette hausse dans leur pays, pendant la même période et presque dans la même proportion ; or, en France, l’exportation des grains a été défendue par les lois jusqu’en 1764, et il me semble assez difficile de supposer que la même diminution de prix à peu près, qui a eu lieu dans un pays, nonobstant la défense d’exporter, doive être attribuée dans l’autre à l’encouragement extraordinaire donné à l’exportation.

Il serait peut-être plus convenable de regarder cette variation dans le prix moyen du blé comme étant plutôt l’effet de quelque hausse graduelle de la valeur réelle de l’argent sur le marché de l’Europe, que de quelque baisse dans la valeur réelle du blé. On a déjà observé que le blé, dans des périodes de temps distantes l’une de l’autre, est une mesure de valeur plus exacte que l’argent, ou peut-être que toute autre marchan­dise. Lorsque, après la découverte des mines abondantes de l’Amérique, le blé vint à monter trois ou quatre fois au-dessus de son ancien prix en argent, ce changement fut attribué généralement à une baisse dans la valeur réelle de l’argent, et non pas à une hausse quelconque dans la valeur réelle du blé. Si donc, pendant les soixante-quatre premières années de ce siècle, le prix moyen du blé en argent a baissé de quelque chose au-dessous de ce qu’il avait été pendant la majeure partie du siècle dernier, nous devrions de même attribuer ce changement, non à quelque baisse dans la valeur réelle du blé, mais à une hausse survenue dans la valeur de l’argent, dans le marché général de l’Europe.

À la vérité, le haut prix du blé, pendant ces dix ou douze dernières années, a fait naître le soupçon que la valeur réelle de l’argent continuait toujours à baisser sur le marché -général de l’Europe. Pourtant ce haut prix du blé paraît avoir été l’effet d’une succession extraordinaire d’années défavorables, et ne devrait pas, par conséquent, être regardé comme une chose permanente, mais comme un événement passager et accidentel. Pendant ces dix ou douze dernières années, les récoltes ont été très-mau­vaises dans la plus grande partie de l’Europe, et les troubles de la Pologne ont aug­men­té extrêmement la disette dans tous les pays qui avaient coutume de s’y approvi­sionner pendant les années de cherté. Quoiqu’une aussi longue suite de mauvaises années ne soit pas un événement ordinaire, il n’est pourtant pas sans exemple, et quiconque a fait des recherches sur l’histoire des prix du blé, dans les anciens, ne sera pas embarrassé de trouver plusieurs autres exemples de la même nature ; d’ailleurs, dix années de disette extraordinaire n’ont rien de plus étonnant que dix années d’une extraordinaire abondance. Le bas prix du blé depuis 1741 inclusivement, jusqu’à 1750 aussi inclusivement, peut très-bien être mis en opposition avec son haut prix, pendant ces huit ou dix dernières années. De 1741 à 1750, le prix moyen du quarter de neuf boisseaux du plus beau froment, au marché de Windsor, paraît, sui­vant les comptes du collège d’Eton, avoir été de 1 livre 13 schellings 9 deniers 4/5 seulement, ce qui fait près de 6 schellings 3 deniers au-dessous du prix moyen des soixante-quatre premières années de ce siècle. À ce compte, le prix moyen du blé de moyenne qualité, pendant ces dix années, a dû revenir à 1 livre 6 schellings 8 deniers seulement le quarter de 8 boisseaux.

Cependant, de 1741 à 1750, la prime a dû empêcher le prix du blé de tomber aussi bas dans le marché intérieur, qu’il l’aurait fait naturellement. Il paraît, d’après les registres des douanes, que la quantité de grains de toutes sortes, exportés pendant ces dix années, ne s’élève pas à moins de 8029 156 quarters. La prime payée pour cet objet s’élève à 1514 962 livres 17 schellings 4 deniers 1/2. Aussi, en 1749, M. Pelham, alors premier ministre, fit-il observer à la chambre des communes qu’on avait payé, dans les trois années précédentes, une somme exorbitante en primes pour l’ex­por­tation du blé. Il était très-bien fondé à faire cette observation, et l’année suivante il l’aurait encore été bien davantage. Dans cette seule année, la prime payée ne s’éleva pas à moins de 324 176 livres 10 schellings 6 deniers[42]. Il est inutile d’observer combien cette exportation forcée doit avoir fait hausser le prix du blé au-dessus de ce qu’il aurait été sans cela sur le marché intérieur.

À la fin de la table des prix annexée à ce chapitre, le lecteur trouvera le compte particulier des dix années, séparé du reste. Il y trouvera aussi le compte particulier des dix années précédentes, dont le taux moyen, beaucoup moins bas, l’est néanmoins encore plus que le taux moyen des soixante-quatre premières années de ce siècle. L’année 1740 fut pourtant une année de disette extraordinaire. Ces vingt années qui ont précédé 1750 peuvent très-bien être mises en opposition avec les vingt qui ont précédé 1770. De même que les premières ont été beaucoup plus bas que le taux moyen du siècle, malgré une ou deux années de cherté qui s’y trouvent, de même les dernières ont été beaucoup au-dessus de ce taux, quoiqu’elles soient coupées par une ou deux années de bon marché, telle, par exemple, que l’année 1759. Si les premières n’ont pas été autant au-dessous du taux moyen général, que les dernières ont été au-dessus, il faut vraisemblablement l’attribuer à la prime. Le changement a évidemment été trop subit pour pouvoir l’attribuer à une modification dans la valeur de l’argent, qui est toujours un événement lent et graduel. Un effet soudain ne peut être expliqué que par une cause qui agisse d’une manière soudaine, telle que la variation accidentelle des saisons.

Il est vrai que le prix du travail en argent s’est élevé, en Angleterre, pendant le cours de ce siècle. Cette hausse cependant paraît être bien moins l’effet d’une dimi­nu­tion de la valeur de l’argent dans le marché général de l’Europe, que l’effet d’une augmentation de la demande de travail en Angleterre, provenant de la grande pros­périté de ce pays et des progrès qui s’y sont accomplis presque universellement. On a observé qu’en France, où la prospérité n’est pas aussi grande, le prix du travail en argent a baissé graduellement avec le prix moyen du blé, depuis le milieu du dernier siècle. On dit que, dans le siècle dernier, ainsi que dans celui-ci, le salaire journalier du travail de manœuvre y a été presque uniformément à environ un vingtième du prix moyen du setier de blé froment, mesure qui contient un peu plus de quatre boisseaux de Winchester. J’ai déjà fait voir[43] qu’en Angleterre la récompense réelle du travail, la quantité réelle de choses propres aux besoins et aisances de la vie qui est donnée à l’ouvrier, a augmenté considérablement pendant le cours de ce siècle. La hausse de son prix pécuniaire paraît avoir eu pour cause, non une diminution de la valeur de l’argent dans le marché général de l’Europe, mais une hausse du prix réel du travail dans le marché particulier de l’Angleterre, due aux circonstances heureuses dans les­quelles se trouve ce pays.

L’argent a dû continuer, pendant quelque temps après la première découverte de l’Amérique, à se vendre à son ancien prix ou très-peu au-dessous. Les profits de l’exploitation des mines furent très-forts, et excédèrent de beaucoup leur taux naturel ; mais ceux qui importaient ce métal en Europe s’aperçurent bientôt qu’ils ne pouvaient pas débiter à ce haut prix la totalité de l’importation annuelle. L’argent dut s’échanger successivement contre une quantité de marchandises toujours de plus petite en plus petite ; son prix dut baisser graduellement de plus bas en plus bas, jusqu’à ce qu’il fût tombé à son prix naturel, c’est-à-dire à ce qui était précisément suffisant pour acquit­ter, suivant leurs taux naturels, les salaires de travail, les profits de capitaux et la rente de terre qu’il faut payer pour que ce métal vienne de la mine au marché. On a déjà observé[44] que, dans la plus grande partie des mines d’argent du Pérou, la taxe du roi d’Espagne, s’élevant à un dixième du produit total, emporte en totalité la rente de la terre. Cette taxe était originairement de moitié ; elle tomba bientôt après au tiers, ensuite au cinquième, et enfin au dixième, taux auquel elle est restée depuis. Cette taxe représente, à ce qu’il semble, dans la plus grande partie des mines d’argent du Pérou, tout le bénéfice qui reste après le remplacement du capital de l’entrepreneur, avec ses profits ordinaires ; et c’est une chose généralement reconnue, que ces profits, qui étaient autrefois très-hauts, sont maintenant aussi bas qu’ils peuvent l’être, pour que son entreprise puisse se soutenir.

En 1504[45], quarante et un ans avant 1545, époque de la découverte des mines du Potosi, la taxe du roi d’Espagne fut réduite à un cinquième de l’argent enregistré. Dans le cours de quatre-vingt-dix ans, ou avant 1636, ces mines, les plus fécondes de toute l’Amérique[46], avaient eu tout le temps de produire leur plein effet, ou de réduire la valeur de l’argent dans le marché de l’Europe aussi bas qu’elle pouvait tomber, tant que ce métal continuait de payer cette taxe au roi d’Espagne. Un espace de quatre-vingt-dix ans est un temps suffisant pour réduire une marchandise quelconque qui n’est pas en monopole à son taux naturel ou au prix le plus bas auquel, tant qu’elle paye une taxe particulière, elle peut continuer de se vendre pendant un certain temps de suite.

Le prix de l’argent, dans le marché de l’Europe, aurait encore peut-être baissé da­van­tage, et il aurait été indispensable, ou de réduire encore la taxe la taxe jusqu’à un vingtième, comme on a fait de celle sur l’or, ou bien de cesser l’exploitation de la plus grande partie des mines d’Amérique qui s’exploitent maintenant. La cause qui a empêché que cela n’arrivât, c’est vraisemblablement l’accroissement progressif de la demande d’argent ou l’agrandissement continuel du marché pour le produit des mines d’argent d’Amérique ; c’est ce qui a non-seulement soutenu la valeur de l’argent dans le marché de l’Europe, mais qui l’a même élevée un peu plus haut qu’elle n’était au milieu du siècle dernier.

Depuis la première découverte de l’Amérique, le marché pour le produit de ses mines d’argent a été continuellement en s’agrandissant de plus en plus.

Premièrement, le marché de l’Europe est devenu successivement de plus en plus étendu. Depuis la découverte de l’Amérique, la plus grande partie de l’Europe a l’ait des progrès considérables. L’Angleterre, la France, la Hollande, l’Allemagne, la Suède même, le Danemark et la Russie, ont tous avancé d’une manière remarquable dans leur agriculture et leur industrie. L’Italie ne paraît pas avoir rétrogradé ; sa décadence avait précédé la conquête du Pérou ; depuis cette époque, elle paraît plutôt s’être un peu relevée. À la vérité, on croit que l’Espagne et le Portugal sont restés un peu en arrière. Toutefois le Portugal n’est qu’une très-petite partie de l’Europe, et la décaden­ce de l’Espagne n’est peut-être pas aussi grande qu’on se l’imagine communément. Au commencement du seizième siècle, l’Espagne était un pays très-pauvre, même en comparaison de la France, qui s’est si fort enrichie depuis cette époque. Tout le monde sait le mot de l’empereur Charles V, que tout abondait en France, et que tout manquait en Espagne. Le produit toujours croissant de l’agriculture et des manufac­tures d’Europe a nécessairement demandé un accroissement successif dans la quantité de monnaie d’argent employée à faire circuler ce produit, et le nombre toujours croissant des individus opulents a fait naître aussi nécessairement la même augmen­tation dans la demande d’argent pour vaisselle, bijoux et autres objets de luxe.

En second lieu, l’Amérique est elle-même un nouveau marché pour le produit de ses propres mines d’argent ; et comme ses progrès en agriculture, en industrie et en population sont beaucoup plus rapides que ceux des nations de l’Europe les plus florissantes, la demande doit augmenter chez elle avec beaucoup plus de rapidité. Les colonies anglaises sont un marché tout à fait nouveau, qui, tant pour la monnaie que pour l’orfèvrerie, exige une fourniture toujours de plus en plus forte pour appro­visionner d’argent un vaste continent où l’on n’en demandait point du tout auparavant. La plus grande partie aussi des colonies espagnoles et portugaises sont des marchés entièrement nouveaux. Avant la découverte faite par les Européens, la Nouvelle-Grenade, l’Yucatan, le Paraguay et le Brésil étaient habités par des peuples sauvages qui n’avaient ni art ni agriculture. Dans tous ces pays, les arts et l’agriculture se sont introduits à un degré considérable. Le Mexique même et le Pérou, quoiqu’on ne puisse les considérer comme des marchés tout à fait nouveaux, sont certainement des marchés bien autrement étendus qu’ils ne l’étaient auparavant. Malgré tous les contes merveilleux qu’on s’est plu à débiter sur l’état de magnificence de ces pays dans leur ancien temps, quiconque veut lire avec un jugement un peu rassis l’histoire de leur première découverte et de leur conquête, sera à même de discerner très-clairement que leurs habitants étaient beaucoup plus ignorants en arts, en agriculture et en commerce, que ne le sont aujourd’hui les Tartares de l’Ukraine. Les Péruviens mêmes, la plus civilisée des deux nations, quoiqu’ils fissent usage d’or et d’argent pour ornements, n’avaient cependant aucune espèce de métaux monnayés. Tout le commerce se faisait par troc et, par conséquent, il n’y avait chez eux presque aucune division de travail. Ceux qui cultivaient la terre étaient obligés de se bâtir leurs maisons, de faire eux-mêmes leurs ustensiles de ménage, leurs habits, leurs chaussures et leurs outils d’agriculture. Le peu d’artisans qu’il y eût parmi eux étaient tous, dit-on, entretenus par le souverain, les nobles et les prêtres, dont ils étaient vraisemblablement les domestiques ou les esclaves. Tous les anciens arts du Mexique et du Pérou n’ont jamais donné à l’Europe un seul genre de manufacture ; les armées espagnoles, qui s’élevaient à peine au-delà de cinq cents hommes, et très-souvent n’atteignaient pas la moitié de ce nombre, trouvèrent presque partout beaucoup de difficulté à se procurer leur subsistance. Les famines qu’elles occasionnaient, à ce qu’on dit, dans presque tous les endroits où elles passaient, dans des pays qu’on veut en même temps repré­senter comme très-peuplés et comme très-bien cultivés, sont une preuve suffisante que ce qu’on a raconté de cette grande population et de cette riche culture est en grande partie fabuleux. Les colonies espagnoles sont sous un gouvernement, à beaucoup d’égards, moins favorable à l’agriculture, à la prospérité et à la population, que celui des colonies anglaises. Néanmoins, elles font, à ce qu’il semble, des progrès dans toutes ces choses, avec bien plus de rapidité qu’aucun pays d’Europe. Dans un sol fertile et sous un heureux climat, la grande abondance des terres et leur bon marché, circonstances qui sont communes à toutes les nouvelles colonies, sont, à ce qu’il semble, un assez grand avantage pour compenser bien des abus dans le gouvernement civil[47]. Frézier, qui observa le Pérou en 1713, représente Lima comme contenant entre vingt-cinq et vingt-huit mille habitants. Ulloa, qui demeura dans le même pays entre 1740 et 1746, la représente comme en renfermant plus de cinquante mille. Les rapports de ces deux voyageurs sur la population de plusieurs autres villes principales du Chili et du Pérou varient à peu près dans la même proportion, et comme on ne voit pas de raison de douter qu’ils n’aient été bien instruits l’un et l’autre, on peut en conclure un accroissement de popu­lation qui ne le cède guère à celui des colonies anglaises. L’Amérique ouvre donc elle-même au produit de ses propres mines d’argent un nouveau marché, où la deman­de augmente encore beaucoup plus rapidement que dans celui des pays de l’Europe qui avance le plus.

En troisième lieu, les Indes orientales sont un autre marché pour le produit des mines d’argent de l’Amérique, et un marché qui, depuis l’époque de la première découverte de ces mines, a continuellement absorbé une quantité d’argent de plus en plus considérable. Depuis cette époque, le commerce direct entre l’Amérique et les Indes orientales, qui se fait par les vaisseaux d’Acapulco, a été sans cesse en augmentant, et le commerce indirect qui se fait par l’entremise de l’Europe s’est accru dans une proportion encore bien plus forte. Pendant le seizième siècle, les Portugais étaient le seul peuple d’Europe qui entretînt un commerce régulier avec les Indes orientales. Dans les dernières années de ce siècle, les Hollandais commencèrent à s’emparer d’une partie de ce monopole et les expulsèrent en peu d’années de leurs principaux établissements dans ces contrées. Pendant la plus grande partie du siècle dernier, ces deux nations partagèrent entre elles la portion la plus considérable du commerce de l’Inde, le commerce des Hollandais augmentant continuellement dans une proportion encore plus grande que ne déclinait celui des Portugais. Les Anglais et les Français firent quelque commerce avec l’Inde dans le dernier siècle, mais il a prodigieusement augmenté dans le cours de celui-ci. C’est aussi dans le cours de ce siècle que les Suédois et les Danois commencèrent à commercer dans l’Inde. Les Russes mêmes font actuellement d’une manière régulière le commerce à la Chine par des espèces de caravanes qui vont à Pékin par terre, à travers la Sibérie et la Tartarie. Si nous en exceptons le commerce des Français, que la dernière guerre a anéanti ou peu s’en faut, celui de tous ces autres peuples aux Indes orientales a été presque continuellement en augmentant. La consommation toujours croissante en Europe des marchandises de l’Inde est assez forte, à ce qu’il paraît, pour leur fournir à tous les moyens d’augmenter successivement leurs affaires. L’usage du thé, par exemple, était très-peu répandu en Europe avant le milieu du siècle dernier. Aujourd’hui, la valeur du thé importé annuellement par la Compagnie des Indes anglaises, pour la consom­ma­tion de l’Angleterre, s’élève à plus d’un million et demi par an, et ce n’est même pas assez dire ; une quantité très-considérable entrant habituellement en fraude par les ports de Hollande, de Gottenbourg en Suède, et aussi des côtes de France, tant qu’a prospéré la Compagnie des Indes de ce dernier pays[48]. La consommation de la Chine, des épiceries des Moluques, des étoffes du Bengale et d’une infinité d’autres articles, a augmenté à peu près dans la même proportion. Aussi, le tonnage de la totalité des vaisseaux employés par l’Europe au commerce de l’Inde, à quelque époque que ce soit du dernier siècle, ne dépassait peut-être pas celui des seuls vais­seaux employés par la Compagnie des Indes anglaises avant la dernière réduction de sa marine.

Mais la valeur des métaux précieux était bien plus élevée dans les Indes, et surtout dans la Chine et dans l’Indoustan, quand les Européens commencèrent à trafiquer dans ces pays, qu’elle n’était en Europe, et il en est encore de même aujourd’hui. Dans des pays à riz, où l’on fait communément deux et quelquefois trois récoltes par an, dont chacune est plus abondante qu’aucune récolte ordinaire de blé, il se trouve nécessairement une beaucoup plus grande abondance de nourriture que dans quelque pays à blé que ce soit, d’une égale étendue. En conséquence, ces pays à riz sont bien plus peuplés ; et de plus les riches, y ayant à leur disposition, au-delà de leur propre consommation, une surabondance infiniment plus grande de subsistances, ont les moyens d’acheter une beaucoup plus grande quantité du travail d’autrui. Aussi, sui­vant tous les rapports, le train d’un grand seigneur à la Chine ou dans l’Indoustan est-il beaucoup plus nombreux et plus magnifique que celui des plus riches particuliers de l’Europe. Cette même surabondance de nourriture dont ils peuvent disposer, les met en état d’en donner une quantité bien plus grande pour toutes productions rares et singulières que la nature n’accorde qu’en très-petites quantités, telles que les métaux précieux et les pierres précieuses, qui sont les grands objets de convoitise entre les riches. Ainsi, en supposant même les mines qui fournissent le marché de l’Inde aussi abondantes que celles qui fournissent le marché de l’Europe, ces marchandises pré­cieuses ne s’en seraient pas moins naturellement échangées, dans l’Inde, contre une plus grande quantité de subsistances qu’en Europe. Mais il paraît que les mines qui ont fourni de métaux précieux le marché de l’Inde ont été beaucoup moins abondan­tes, et que celles qui l’ont fourni de pierres précieuses l’ont été beaucoup plus que les mines qui ont fourni le marché de l’Europe. Ainsi, les métaux précieux ont dû naturellement, dans l’Inde, obtenir en échange une quantité de pierres précieuses plus grande qu’en Europe, et une quantité de nourriture encore beaucoup plus considé­rable. Le prix des diamants, la plus grande des superfluités, a dû être, en argent, un peu plus bas dans un pays que dans l’autre, et celui de la nourriture, la première des choses nécessaires infiniment moindre. Mais, comme on l’a déjà remarqué[49], le prix réel du travail, la quantité réelle de choses propres aux besoins et commodités de la vie qu’on donne à l’ouvrier, est moindre à la Chine et dans l’Indoustan, les deux grands marchés de l’Inde, qu’elle n’est dans la plus grande partie de l’Europe. Le salaire qu’y reçoit l’ouvrier et achètera une moindre quantité de nourriture ; et comme le prix en argent de la nourriture est bien plus bas dans l’Inde qu’en Europe, le prix du travail en argent y est plus bas sous un double rapport ; sous le rapport de la petite quantité de nourriture qu’il peut acheter, et encore sous le rapport du bas prix de cette nourriture. Or, dans des pays où il y a égalité d’art et d’industrie, le prix pécuniaire de la plupart des ouvrages de main-d’œuvre sera proportionné au prix pécuniaire du travail ; et en fait d’ouvrages de manufactures, l’art et l’industrie, quoique inférieurs, ne sont pas, à la Chine et dans l’Indoustan, fort au-dessous de ce qu’ils sont en quelque endroit de l’Europe que ce soit. Par conséquent, le prix en argent de la plupart des ouvrages de main-d’œuvre sera naturellement beaucoup plus bas dans ces grands empires, qu’il ne le sera en aucun endroit de l’Europe. D’ailleurs, dans la majeure partie de l’Europe, les frais de transport par terre augmentent beaucoup le prix tant réel que nominal de la plupart des ouvrages de main-d’œuvre. Il en coûte plus de travail et, par conséquent, plus d’argent, pour transporter au marché, d’abord les matières premières, et ensuite l’ouvrage manufacturé. À la Chine et dans l’Indostan, l’étendue et la multiplicité des moyens de navigation intérieure épargnent la plus grande partie de ce travail et, par conséquent, de cet argent, et par là réduisent à un taux encore plus bas le prix, tant réel que nominal, de la plupart des objets de manu­facture de ces deux pays. D’après tout ceci, les métaux précieux sont une marchandise qu’il a toujours été et qu’il est encore extrêmement avantageux de porter de l’Europe aux Indes orientales[50]. Il n’y a presque aucune marchandise qui y rapporte davantage, ou qui, à proportion de la quantité de travail et de choses qu’elle coûte en Europe, puisse commander ou acheter une plus grande quantité de travail et de choses dans l’Inde. Il est aussi plus avantageux d’y porter de l’argent que de l’or, parce qu’à la Chine et dans la plupart des autres marchés de l’Inde, la proportion entre l’argent fin et l’or fin n’est que comme 10 ou au plus comme 12 est à 1, tandis qu’en Europe elle est comme 14 ou 15 est à 1. À la Chine et dans la plupart des autres marchés de l’Inde, dix onces ou au plus douze onces d’argent achèteront une once d’or. En Europe, il en faut de quatorze à quinze onces. Aussi, dans les cargaisons de la majeure partie des vaisseaux d’Europe qui font voile pour l’Inde, l’argent a été, en général, un des articles principaux. C’est le principal article de la cargaison des vaisseaux d’Acapulco qui font voile pour Manille. Ainsi, l’argent du nouveau continent est, à ce qu’il paraît, le grand objet du commerce qui se fait entre les deux extrémités de l’ancien ; il forme le principal anneau de la chaîne qui unit l’une à l’autre ces deux parties du monde si distantes[51].

Pour pouvoir fournir aux besoins d’un marché d’une aussi vaste étendue, il faut que la quantité d’argent qu’on tire annuellement des mines suffise non-seulement à cette augmentation toujours croissante de demande, pour monnaie, pour vaisselle et pour bijoux, qui vient de tous les pays où l’opulence est progressive, mais encore à réparer la consommation et le déchet continuel d’argent qui a lieu dans tous les pays où l’on fait usage de ce métal.

Il est aisé de se former une idée de la consommation de métaux précieux qui se fait continuellement dans les monnaies et dans les ouvrages d’orfèvrerie, par le frottement résultant du service, particulièrement dans la vaisselle par le nettoyage ; et dans une marchandise dont l’usage est si prodigieusement étendu, cet article seul exigerait tous les ans un remplacement considérable. Mais une consommation encore plus sensible, parce qu’elle est bien plus rapide, quoique au total elle ne soit peut-être pas plus forte que l’autre, qui se fait successivement, c’est celle qui a lieu dans certaines manufactures. Dans celles de Birmingham seulement, la quantité d’or et d’argent qui s’emploie annuellement en feuilles et dans des ouvrages de dorure, et qui est mise par là hors d’état de reparaître jamais sous la forme de métal, s’élève, dit-on, à plus de 50,000 livres sterling. Nous pouvons juger, d’après cela, quelle énorme consommation il s’en fait chaque année dans toutes les différentes parties du monde, tant en ouvrages du genre de ceux de Birmingham, qu’en galons, broderies, étoffes d’or et d’argent, dorure de livres, de meubles, etc. Il se perd aussi tous les ans une quantité considérable de ces métaux dans le transport qui s’en fait par terre et par mer. Enfin, il s’en perd encore une bien plus grande quantité par la pratique, presque universellement usitée dans la majeure partie des pays asiatiques, de cacher dans les entrailles de la terre des trésors dont la connaissance meurt souvent avec la personne qui les a enfouis.

D’après les meilleurs rapports, la quantité d’or et d’argent importée tant à Cadix qu’à Lisbonne, en comptant non-seulement ce qui est enregistré, mais encore ce qu’on peut supposer passer en fraude, s’élève par an à 6,000,000 sterling environ.

Selon M. Meggens[52], l’importation annuelle des métaux précieux en Espagne, en prenant la moyenne de six ans, de 1748 inclusivement à 1753 aussi inclusivement ; et en Portugal, en prenant la moyenne de 7 ans, de 1747 inclusivement à 1753 aussi inclusivement, s’est élevée, pour l’argent, à 1,101,107 livres pesant, et pour l’or, à 49,940 livres pesant. L’argent importé, à 62 schellings la livre de Troy, donne 3,413,431 livres 10 schellings sterling. L’or, à 44 guinées et demie la livre de Troy, donne 2,333,446 livres 14 schellings sterling. Les deux, ensemble, font une somme de 5,746,878 livres 4 schellings sterling. Le même auteur assure que le compte de l’impor­tation, pour ce qui a été enregistré, est exact. Il donne le détail des endroits d’où l’or et l’argent ont été apportés, et de la quantité de chaque métal que chacun de ces endroits a fournie, suivant les registres. Il passe aussi en compte la quantité de chacun de ces métaux qu’il présume avoir pu venir en fraude. La grande expérience de cet habile négociant donne un très-grand poids à son opinion.

Suivant l’auteur éloquent de l’Histoire philosophique et politique de l’établisse­ment des Européens dans les deux Indes, qui a eu quelquefois de bonnes infor­mations, l’importation annuelle en Espagne de l’or et de l’argent enregistrés, en prenant la moyenne de onze ans, de 1754 inclusivement à 1764 aussi inclusivement, s’est élevée à 13,984,185 3/4 piastres de 10 réaux[53]. En tenant compte cependant de ce qui peut avoir passé en fraude, il suppose que toute l’importation annuelle s’élève à 17,000,000 de piastres ; ce qui, à 4 schellings 6 deniers la piaste, fait, 3,825,000 livres sterling. Raynal donne aussi le détail des différents endroits d’où ont été tirés l’or et l’argent, et les quantités de chaque métal que chaque endroit a fournies, suivant les registres. Il prétend aussi[54] que s’il fallait juger de la quantité d’or annuellement importée du Brésil à Lisbonne par le montant de la taxe payée au roi de Portugal, qui paraît être d’un cinquième du métal au titre, on pourrait évaluer cette quantité à 18,000,000 de cruzades ou 45,000,000 de livres en France, faisant environ 2,000,000 sterling. Cependant, dit-il, en comptant ce qui peut être passé en fraude, nous pouvons en toute sûreté ajouter à cette somme un huitième en sus, ou 250,000 livres sterling, de sorte que le total sera de 2,250,000 livres sterling. Ainsi, d’après ce compte, la totalité de l’importation annuelle des métaux précieux, tant en Espagne qu’en Portu­gal, s’élève à 6,075,000 livres sterling environ.

Plusieurs autres rapports manuscrits, mais appuyés de pièces très-authentiques, s’accordent, à ce qu’on m’a assuré, à évaluer la totalité de cette importation annuelle des métaux précieux à environ 6,000,000 sterling, tantôt un peu plus, tantôt un peu moins.

À la vérité, l’importation des métaux précieux à Cadix et à Lisbonne ne compose pas la totalité du produit annuel des mines d’Amérique. Il y en a une partie qui est envoyée tous les ans à Manille par les vaisseaux d’Acapulco ; une autre partie est employée au commerce de contrebande que font les colonies espagnoles avec celles des autres nations européennes, et enfin il en reste indubitablement une autre partie dans le pays même. En outre, les mines d’or et d’argent de l’Amérique ne sont pas les seules qui existent au monde ; elles sont toutefois les plus abondantes, à beaucoup près. Il n’y a nul doute que le produit de toutes les autres mines connues n’est rien en comparaison du leur, et il est pareillement certain que la plus grande partie de leur produit est importée annuellement à Cadix et à Lisbonne. Mais la consommation de Birmingham seulement, en la comptant sur le pied de 50,000 livres par an, emporte déjà une cent vingtième partie de cette importation annuelle, évaluée à 6 millions. Ainsi, la totalité de la consommation annuelle d’or et d’argent, dans tous les divers pays du monde où ces métaux sont en usage, est peut-être égale, à peu de chose près, à la totalité de ce produit annuel. Le reste ne suffit peut-être que tout juste pour répondre à l’augmentation de demande de la part de tous les pays qui s’enrichissent. Il peut même s’être trouvé assez au-dessous de cette demande, pour que le prix de ces métaux se soit élevé de quelque chose dans le marché de l’Europe[55].

La quantité de cuivre et de fer qui va de la mine au marché, est, sans aucune com­paraison, plus grande que celle de l’or et de l’argent. Nous n’allons pas nous imaginer, pour cela, qu’il doive en arriver que ces métaux grossiers se multiplient au-delà de la demande, et qu’ils deviendront plus à meilleur marché. Pourquoi donc nous figurerions-nous que cela dût arriver à l’égard des métaux précieux ? Il est vrai que les métaux grossiers, s’ils sont plus durs, sont aussi employés à des usages bien plus rudes et, qu’attendu leur moindre valeur, on apporte beaucoup moins de soin à les conserver. Mais néanmoins, les métaux précieux ne sont pas plus qu’eux de nature indestructible et, comme eux, ils sont sujets à être perdus, dissipés et consom­més de mille manières différentes.

Le prix des métaux en général, quoique sujet à des variations lentes et succes­sives, varie moins d’une année à l’autre, que celui de presque toute autre partie du pro­duit brut de la terre, et le prix des métaux précieux est même moins sujet à de brus­ques variations que celui des métaux grossiers. La durée des métaux est la cause qui donne à leur prix cette stabilité extraordinaire. Le blé qui a été mis au marché l’année dernière sera tout entier ou presque tout entier consommé longtemps avant la fin de cette année. Mais il peut y avoir quelque portion du fer apporté de la mine, il y a deux ou trois cents ans, qui soit aujourd’hui mis en usage, et peut-être se sert-on encore de quelque portion de l’or qui a été extrait de la mine il y a deux ou trois mille ans. Les différentes masses de blé qui, chaque année, doivent fournir à la consomma­tion du genre humain, seront toujours à peu près proportionnées au produit respectif de cha­cune de ces différentes années. Mais la proportion entre les différentes masses de fer qu’on peut mettre en consommation dans deux années différentes ne souffrira qu’ex­ces­si­vement peu d’une variation accidentelle dans le produit des mines de fer pendant ces deux différentes années, et la proportion entre les différentes masses d’or mises en consommation dépendra encore bien moins de quelque variation semblable dans le produit des mines d’or. Ainsi, quoique le produit de la plupart des mines métalliques varie peut-être encore plus d’une année à l’autre que le produit de la plupart des champs de blé, ces variations ne font pas le même effet sur le prix de la première de ces deux espèces de marchandises que sur celui de l’autre.

\subsection*{Des variations dans la proportion entre les valeurs respectives de l’or et de l’argent}

Avant la découverte des mines de l’Amérique, la valeur de l’or fin, relativement à l’argent fin, était réglée, dans les différentes administrations des Monnaies en Europe, dans la proportion de 10 à 1 et de 12 à 1, c’est-à-dire qu’une once d’or fin était censée valoir de dix à douze onces d’argent fin. Vers le milieu du siècle dernier, cette valeur de l’or fut réglée dans la proportion de 14 à 1 et de 15 à 1, c’est-à-dire qu’une once d’or pur fut censée valoir entre quatorze et quinze onces d’argent pur. L’or haussa dans sa valeur nominale ou dans la quantité d’argent avec laquelle on l’échangea. Les deux métaux baissèrent dans leur valeur réelle ou dans la quantité de travail qu’ils pou­vaient acheter ; mais l’argent baissa plus que l’or. Quoique les mines d’or et d’argent d’Amérique, les unes comme les autres, surpassent en fécondité toutes les mines con­nues jusqu’alors, les mines d’argent, à ce qu’il semble, surpassèrent les anciennes dans une proportion encore plus forte que ne le firent les mines d’or.

Les grandes quantités d’argent portées annuellement de l’Europe aux Indes orien­tales ont réduit par degrés la valeur de ce métal, relativement à l’or, dans quel­ques établissements anglais. À la Monnaie de Calcutta[56], une once d’or pur est censée valoir quinze onces d’argent pur, comme en Europe. Peut-être la Monnaie le taxe-t-elle trop haut comparativement à la valeur qu’il a dans le marché du Bengale. À la Chine, la pro­portion de l’or à l’argent est toujours de 10 à 1 ou de 12 à 1. On dit qu’au japon elle est comme 8 est à 1.

Entre les quantités d’or et d’argent annuellement importées en Europe, la pro­por­tion est à peu près comme 1 est à 22, au rapport de Meggens, c’est-à-dire que pour une once d’or il y a un peu plus de vingt- deux onces d’argent importées[57]. Il présume que la grande quantité d’argent qui va annuellement aux Indes orientales réduit les quantités de ces métaux qui restent en Europe à la proportion de 1 à 14 ou 15, qui est l’inverse de leurs valeurs respectives. Il paraît penser que la proportion entre leurs valeurs doit nécessairement être en raison réciproque de leurs quantités et que, par conséquent, elle serait comme 22 à 1, si ce n’était cette plus forte exportation d’argent.

Mais il n’est pas vrai que la proportion ordinaire entre les valeurs respectives de deux marchandises doive être nécessairement en raison de celle des quantités qui s’en trouvent communément au marché. Le prix du bœuf, en le mettant à 10 guinées, est environ soixante fois le prix d’un agneau, en mettant celui-ci à 3 schellings 6 deniers. Il serait pourtant absurde d’en conclure qu’il y a communément au marché soixante agneaux contre un bœuf ; et il serait tout aussi absurde d’inférer, de ce qu’une once d’or achètera communément de quatorze à quinze onces d’argent, qu’il y a communément au marché quatorze à quinze onces seulement d’argent contre une once d’or.

Il est vraisemblable que la quantité d’argent qui est communément au marché excède celle de l’or dans une proportion beaucoup plus forte que la valeur d’une cer­taine quantité d’or ne surpasse la valeur d’une égale quantité d’argent. La masse totale, mise au marché, d’une denrée à bas prix, est ordinairement, non-seulement plus gran­de en quantité, mais aussi d’une plus grande valeur que la masse totale d’une denrée chère. La masse totale du pain qui va annuellement au marché est non-seule­ment plus grande, mais encore d’une plus grande valeur que toute la masse de viande de bou­cherie ; la masse totale de viande de boucherie vaut plus que la masse totale de volaille ; et la masse totale de volaille, plus que la masse totale de gibier à plumes. Le nombre des acheteurs d’une denrée à bas prix excède tellement celui des acheteurs de la denrée chère, qu’on pourra non-seulement débiter une plus grande quantité de la première, mais en débiter pour une plus grande valeur. Ainsi, la masse totale de la denrée à bas prix excédera communément la masse totale de la denrée chère, dans une proportion plus forte que la valeur d’une certaine quantité de la seconde n’excédera la valeur d’une quantité égale de la première. Quand nous comparons les métaux pré­cieux l’un avec l’autre, l’argent est la denrée à bas prix, et l’or est la denrée chère. Nous devons donc naturellement compter qu’il y aura toujours au marché, non-seule­ment plus d’argent que d’or, mais encore une plus grande valeur en argent qu’en or[58]. Qu’un homme qui possède un peu de ces deux métaux compare ce qu’il a de vaisselle et de bijoux d’argent avec ce qu’il en a en or, et il trouvera probablement que non-seulement la quantité, mais même la valeur de ce qu’il a en argent excède de beau­coup ce qu’il en possède en or. Beaucoup de gens, d’ailleurs, qui ne laissent pas que d’avoir des objets d’argent, n’en ont point du tout en or ; et pour ceux même qui en ont, ces objets se bornent à une boîte de montre, une tabatière et quelques autres colifi­chets, dont la somme totale est rarement de grande valeur. À la vérité, dans la totalité des monnaies anglaises, l’or l’emporte de beaucoup en valeur sur l’argent ; mais il n’en est pas de même dans les monnaies de tous les pays. Dans celles de quelques-uns, la valeur est à peu près, égale dans un métal et dans l’autre. Dans les Monnaies d’Écosse, avant l’union avec l’Angleterre, l’or l’emportait de très-peu en valeur, quoiqu’il l’em­por­tât pourtant de quelque chose[59], comme on le voit par les comptes de l’adminis­tration des Monnaies L’argent l’emporte dans la monnaie de plusieurs pays. En Fran­ce, les plus grosses sommes se payent communément en argent, et il est difficile de s’y procurer plus d’or que ce qu’on a besoin d’en porter dans sa bourse. En outre, la supériorité de la valeur de la vaisselle et des bijoux d’argent sur ceux d’or, qui est généralement dans tous les pays, fait beaucoup plus que compenser la supériorité de valeur de l’or dans les monnaies, qui est particulière à quelques pays seulement[60].

Quoique l’argent ait toujours été et doive toujours être vraisemblablement beau­coup moins cher que l’or, dans le sens ordinaire de ce mot ; cependant, dans un autre sens, il est possible de dire que l’or est de quelque chose moins cher que l’argent, dans l’état actuel où est le marché de l’Espagne. On peut dire d’une marchandise qu’elle est chère ou à bon marché, non-seulement en raison de ce que son prix habituel fait une grosse ou une petite somme, mais aussi en raison de ce que ce prix habituel se trouve plus ou moins au-dessus du prix le plus bas auquel il soit possible de le mettre au marché pendant un certain temps de suite. Ce prix le plus bas est celui qui remplace purement, avec un profit modique, le capital qu’il faut employer pour mettre cette marchandise au marché. Ce prix est celui qui ne fournit rien pour le propriétaire de la terre, celui dans lequel la rente n’entre pas pour une partie constituante, et qui se résout tout entier en salaires et en profits. Or, dans l’état actuel du marché de l’Espa­gne, l’or est certainement un peu plus rapproché de ce prix le plus bas que ne l’est l’argent. La taxe du roi d’Espagne sur l’or n’est que d’un vingtième du métal au titre, ou de 5 pour 100, tandis que la taxe sur l’argent s’élève à un dixième du métal, ou à 10 pour 100. De plus, comme nous l’avons déjà observé, c’est dans ces taxes que consiste toute la rente de la plupart des mines d’or et d’argent de l’Amérique espagnole, et celle sur l’or est toujours beaucoup plus mal payée que celle sur l’argent. Il faut bien aussi que les profits des entrepreneurs des mines d’or soient, en général, encore plus modiques que ceux des entrepreneurs des mines d’argent, puisqu’il est plus rare que les premiers fassent fortune[61]. Ainsi, puisque l’or d’Espagne fournit et moins de rente, et moins de profit, il faut donc que son prix, dans le marché de l’Espagne, soit un peu plus rapproché que celui de l’argent d’Espagne, du prix le plus bas auquel on puisse le mettre à ce marché. Si l’on déduisait toutes les dépenses, la masse totale du premier de ces métaux ne trouverait pas, à ce qu’il semble, dans le marché d’Espagne, un débit aussi avantageux que la masse totale de l’autre. Il est vrai que la taxe du roi de Portu­gal sur l’or du Brésil est la même que la taxe ancienne du roi d’Espagne sur l’argent du Mexique et du Pérou, c’est-à-dire un cinquième du métal, au titre. Il est donc douteux de savoir si, sur le marché général de l’Europe, la totalité de l’or d’Amérique revient à un prix plus voisin du prix le plus bas auquel il soit possible de l’y amener, que n’y revient la totalité de l’argent d’Amérique. 

Le prix des diamants et des autres pierres précieuses est peut-être encore plus rapproché que le prix de l’or du prix le plus bas auquel il soit possible de les mettre au marché.

Quoiqu’il n’y ait pas trop d’apparence qu’on veuille jamais rien abandonner d’une taxe qui non-seulement est établie sur un des articles les plus propres à être imposés, un article purement de luxe et de superfluité, mais qui d’ailleurs rapporte un revenu aussi important, tant qu’on verra de la possibilité à la faire payer ; cependant la même impossibilité de la payer, qui a, en 1736, obligé de la réduire d’un cinquième à un dixième, peut quelque jour obliger à la réduire encore davantage, de la même manière qu’elle a obligé à réduire au vingtième la taxe sur l’or. Quiconque a observé l’état des mines de l’Amérique espagnole, a reconnu que, comme toutes les autres mines, elles deviennent de jour en jour d’une exploitation plus dispendieuse, à cause de la plus grande profondeur à laquelle il faut établir les travaux, et des plus fortes dépenses qu’il faut faire pour tirer l’eau et fournir de l’air frais à ces grandes profondeurs[62].

Ces causes, qui équivalent à une rareté qui se ferait sentir dans l’argent (car on peut dire d’une denrée qu’elle devient plus rare quand il devient plus difficile et plus coûteux d’en recueillir une certaine quantité), doivent produire un jour l’un ou l’autre des trois effets suivants : il faut nécessairement que cette augmentation de dépense soit compensée, ou entièrement par une augmentation proportionnée dans la valeur du métal, ou entièrement par une diminution proportionnée de la taxe sur l’argent, ou enfin partie par l’un, partie par l’autre de ces deux moyens. Le dernier de ces trois cas est très-possible. Comme l’or a haussé de prix relativement à l’argent, nonobstant la grande diminution de la taxe sur l’or, de même l’argent pourrait hausser de prix relativement au travail et autres marchandises, nonobstant une pareille diminution de la taxe sur l’argent.

Cependant, si de telles réductions successives de la taxe ne peuvent pas totale­ment empêcher la hausse de la valeur de l’argent dans le marché de l’Europe, au moins elles doivent certainement la retarder plus ou moins. Ces réductions permettent d’exploiter beaucoup de mines qui n’auraient pas pu être exploitées auparavant, parce que leur produit n’aurait pas pu suffire à payer l’ancienne taxe ; la quantité d’argent annuellement portée au marché doit être nécessairement un peu plus grande, et la va­leur, par conséquent, d’une quantité donnée d’argent doit être un peu moindre qu’elle ne l’aurait été sans cela. Quoique la valeur de l’argent, dans le marché de l’Euro­pe, ne soit peut-être pas aujourd’hui au-dessous, malgré la réduction de taxe qui a eu lieu en 1736, de ce qu’elle était avant cette réduction, néanmoins elle est probablement au moins de 10 pour 100 plus bas qu’elle n’aurait été si la cour d’Espagne eût continué à exiger l’ancienne taxe.

Mais que, malgré cette réduction, la valeur de l’argent ait commencé à hausser de quelque chose dans le marché de l’Europe pendant le cours du siècle actuel, c’est ce que les faits et les raisonnements rapportés ci-dessus me portent à croire, ou plutôt à conjecturer, à soupçonner ; car je ne peux donner que comme conjecture l’opinion la plus sûre que je me suis faite à ce sujet. La hausse, il est vrai, en supposant qu’il y en ait une, a été si faible, jusqu’à présent, que malgré tout ce qui a été dit, il pourra peut-être paraître incertain à beaucoup de personnes, non-seulement si cet événement a réellement eu lieu, mais même si le contraire n’est pas arrivé, ou si la valeur de l’ar­gent ne continue pas toujours à baisser dans le marché de l’Europe[63].

Il faut toutefois observer que, quelle que puisse être l’importation annuelle d’or et d’argent, il doit nécessairement arriver une certaine période à laquelle la consomma­tion annuelle de ces métaux sera égale à leur importation annuelle. Leur consom­mation doit augmenter à mesure qu’augmente leur masse totale, ou plutôt elle doit augmenter dans une proportion beaucoup plus forte. À mesure que leur masse aug­men­te, leur valeur diminue ; on en fait un plus grand usage ; on en a moins de soin, et con­séquemment leur consommation croît dans une plus grande proportion que leur masse. Ainsi, après une certaine période, la consommation annuelle de ces métaux doit devenir égale à leur importation annuelle, à moins que cette importation n’aille con­ti­nuellement en augmentant ; ce qui n’est pas le cas qu’on puisse supposer dans les circonstances actuelles.

La consommation annuelle une fois arrivée au niveau de l’importation annuelle, si celle-ci venait à diminuer par degrés, alors la consommation annuelle pourrait excé­der pendant quelque temps l’importation annuelle ; mais alors la masse de ces métaux diminuerait insensiblement et par degrés, et leur valeur hausserait aussi insensible­ment et par degrés, jusqu’à ce que, l’importation annuelle rétablissant le niveau, la consommation annuelle vînt à se régler insensiblement et par degrés à ce que cette importation annuelle peut fournir.

\subsection*{Des motifs qui ont fait soupçonner que la valeur de l’argent continuait toujours à baisser[64]}

Ce qui porte beaucoup de gens à croire que la valeur des métaux précieux conti­nue toujours à baisser dans le marché de l’Europe, c’est l’accroissement d’opulence de l’Europe, joint à cette notion vulgaire, que la quantité de ces métaux augmentant natu­rellement à mesure que l’opulence augmente, ils doivent diminuer en valeur en aug­mentant en quantité ; et ce qui les confirme encore davantage dans cette opinion, c’est le prix toujours croissant de plusieurs parties du produit brut de la terre[65].

J’ai déjà tâché de démontrer précédemment que l’augmentation de quantité des métaux précieux dans un pays, quand elle avait sa source dans l’augmentation de richesse, ne tendait nullement à diminuer leur valeur. L’or et l’argent vont se rendre naturellement dans un pays riche, par la même raison que s’y vont rendre tous les objets de luxe et de curiosité ; ce n’est pas parce qu’ils y sont à meilleur marché que dans des pays plus pauvres, mais parce qu’ils y sont plus chers, ou parce qu’on donne un meilleur prix pour les obtenir. C’est cette supériorité de prix qui les y attire, et ils cessent nécessairement d’y aller aussitôt que cette supériorité vient à cesser.

J’ai déjà cherché à faire voir qu’à l’exception du blé et des autres végétaux qui sont entièrement le fruit de l’industrie des hommes, toutes les autres espèces de produit brut, le bétail, la volaille, le gibier de tout genre, les fossiles et les minéraux utiles, etc., devenaient naturellement plus chers à mesure que la société s’enrichit et gagne en industrie. Ainsi, quoique ces sortes de denrées viennent à s’échanger contre une plus grande quantité d’argent qu’auparavant, il ne s’ensuit nullement de là que l’argent soit devenu naturellement moins cher ou qu’il achètera moins de travail qu’auparavant, mais seulement que ces denrées sont elles-mêmes devenues réellement plus chères, ou qu’elles achèteront plus de travail qu’auparavant. Ce n’est pas seulement leur prix nominal, c’est encore leur prix réel, qui s’élève en proportion des progrès de l’amélio­ration du pays. La hausse de leur prix nominal n’est pas l’effet d’une dégradation dans la valeur de l’argent, mais l’effet d’une hausse dans leur prix réel.

\subsection*{Des effets différents des progrès de la richesse nationale sur trois sortes différentes de produit brut}

On peut diviser en trois classes les différentes sortes de produit brut dont nous venons de parler.

La première comprend ces sortes de produits sur la multiplication desquels l’influ­ence de l’industrie humaine est nulle ou à peu près nulle.

La seconde comprend ceux qu’on peut multiplier en proportion de la demande.

La troisième, ceux sur la multiplication desquels l’industrie humaine n’a qu’une puissance bornée ou incertaine.

Dans les progrès que fait un pays en opulence et en industrie, le prix réel de cette première classe de produits peut s’élever jusqu’au degré le plus haut qu’on puisse imaginer, et il ne semble contenu par aucunes bornes. Celui de la seconde classe peut bien monter très-haut, mais il a de certaines limites qu’on ne peut guère dépasser pen­dant une suite de temps un peu longue. Celui de la troisième classe, quoiqu’il ait une tendance naturelle à s’élever dans l’avancement progressif du pays vers l’opulence, cependant le degré d’avancement du pays étant le même, ce prix peut venir quel­que­fois à baisser, quelquefois demeurer fixe, et quelque fois hausser plus ou moins ; le tout selon que les efforts de l’industrie humaine réussiront plus ou moins, d’après les diverses circonstances, à multiplier cette sorte de produit brut.
première classe.

La première source de produit brut dont le prix s’élève dans les progrès que fait l’opulence nationale, c’est celle sur la multiplication de laquelle l’industrie humaine n’a presque aucune espèce de pouvoir ; elle consiste dans ces choses que la nature ne donne qu’en certaines quantités, et qui, étant très-périssables, ne peuvent s’accumuler pendant plusieurs saisons de suite. Telles sont la plupart des poissons et oiseaux rares et singuliers ; plusieurs différentes espèces de gibier ; presque tous les oiseaux sauva­ges, particulièrement ceux de passage, ainsi que beaucoup d’autres choses. À mesure que croissent la richesse et le luxe qui l’accompagne, la demande de ces sortes de choses doit vraisemblablement croître aussi en même temps, et il n’y a pas d’efforts de l’industrie humaine capables d’augmenter l’approvisionnement de ces choses fort au-delà de ce qu’il était avant l’accroissement de la demande. Ainsi, la quantité de ces marchandises restant la même ou à peu près la même, tandis que la concurrence des acheteurs va toujours croissant, leur prix peut monter à tous les degrés possibles d’excès, et il ne paraît pas y avoir de bornes pour l’arrêter. Quand même les bécasses deviendraient en vogue pour se vendre vingt guinées la pièce, il n’y a pas d’efforts de l’industrie humaine capables d’en faire venir au marché un plus grand nombre, à peu de chose près, que ce qui y vient aujourd’hui. C’est ainsi qu’on peut facilement expliquer le haut prix de quelques oiseaux et poissons rares, chez les Romains, au temps de leur plus grande opulence. Ces prix n’étaient point l’effet d’une baisse de la valeur de l’argent à cette époque, mais de la grande valeur de ces choses rares et curieuses que l’industrie humaine ne pouvait pas multiplier à son gré. À l’époque qui précéda la chute de la république romaine et dans les temps qui suivirent de près cet événement, la valeur de l’argent, à Rome, était plus élevée qu’elle ne l’est aujourd’hui dans la plus grande partie de l’Europe. Le prix auquel la république payait le modius, ou quart de boisseau de blé de dîme de Sicile, était de 3 sesterces, valant environ 6 den. sterling. Ce prix était pourtant vraisemblablement plus bas que le prix moyen du marché, puisque l’obligation de livrer le blé à ce prix était regardée comme un tribut imposé sur les labours de la Sicile. Aussi, quand les Romains avaient besoin de requé­rir plus de blé que le montant des dîmes, ils devaient payer le surplus, ainsi qu’ils s’y étaient engagés par capitulation, sur le pied de 4 sesterces ou 8 deniers sterling le modius ; et ce prix avait été sans doute regardé comme le prix modéré et raisonnable, c’est-à-dire le prix de contrat moyen et ordinaire de ce temps-là ; il revient à environ 2 sch. le quarter. Or, le prix de contrat ordinaire du blé d’Angleterre, qui, en qualité, est inférieur au blé de Sicile, et se vend, en général, à un plus bas prix dans le marché de l’Europe, était de 28 sch. le quarter avant les dernières années de cherté. Ainsi, la valeur de l’argent, dans ces temps anciens, doit avoir été, à sa valeur actuelle, dans la valeur inverse de 3 à 4, c’est-à-dire que 3 onces d’argent auraient alors acheté la même quantité de travail et de choses consommables que 4 onces en achèteraient aujour­d’hui. Lors donc que nous lisons dans Pline[66] que Seius acheta un rossignol blanc pour en faire présent à l’impératrice Agrippine, au prix de 6,000 sesterces, valant environ 50 livres de notre monnaie, et qu’Asinius Celer[67] acheta un surmulet 8,000 sesterces, valant environ 66 liv. 13 sch. 4 den. de notre monnaie, toute surprenante que nous pa­raisse l’élévation prodigieuse de ces prix, nous la voyons pourtant encore, au premier coup d’œil, d’un tiers au-dessous de ce qu’elle était réellement. Le prix réel de ces deux choses, la quantité de travail et de subsistances qu’on a cédée pour les avoir, était en­co­re d’un tiers environ plus fort que ce que nous exprime aujourd’hui le prix nominal. Seius céda pour le rossignol le droit de disposer d’autant de travail et de subsistances qu’en pourraient acheter maintenant 66 liv. 13 sch. 4 den. ; et Asinius Celer, pour son surmulet, se dessaisit du pouvoir d’en commander autant qu’en achèteraient aujour­d’hui 88 liv. 13 sch. 9 den. 1/3. L’élévation excessive de ces prix provenait bien moins d’une abondance d’argent que de l’abondance de travail et de subsistances que ces deux Romains avaient à leur disposition, au-delà de ce qu’exigeait leur consommation personnelle ; mais la quantité d’argent qu’ils avaient à leur disposition était beaucoup moindre que celle que leur procurerait aujourd’hui la faculté de commander une pareille quantité de travail et de subsistances. 

\subsubsection*{Deuxième classe}

La seconde sorte de produit brut dont le prix s’élève dans les progrès de la civili­sation est celle que l’industrie humaine peut multiplier à proportion de la demande ; elle consiste dans ces plantes et ces animaux utiles que la nature produit dans les pays incultes avec tant de profusion qu’ils n’ont que peu ou point de valeur, et qui, à mesure que la culture s’étend, sont forcés par elle de céder le terrain à quelque produit plus profitable. Pendant une longue période dans le cours des progrès de la civilisa­tion, la quantité des produits de cette classe va toujours en augmentant. Ainsi, leur valeur réelle, la quantité réelle de travail qu’ils peuvent acheter ou commander, s’élève par degrés jusqu’à ce qu’enfin elle monte assez haut pour en faire un produit aussi avantageux que toute autre production venue, à l’aide de l’industrie humaine, sur les terres les plus fertiles et les mieux cultivées. Quand elle est arrivée là, elle ne peut guère aller plus haut ; autrement, pour augmenter la quantité du produit, on y consa­crerait bientôt plus de terre et plus d’industrie.

Par exemple, quand le prix du bétail s’élève assez haut pour qu’il y ait autant de profit à cultiver la terre en nature de subsistances pour le bétail qu’en nature de sub­sis­tances pour l’homme, ce prix ne peut plus guère hausser ; si cela arrivait, une plus grande partie de terre à blé serait bientôt convertie en pâturages. L’extension du labourage, en diminuant la quantité des vaines pâtures, diminue la quantité de viande de boucherie que le pays produisait naturellement sans travail ou sans culture ; et elle en accroît la demande, parce qu’elle augmente le nombre de ceux qui ont du blé, ou, ce qui revient au même, qui ont le prix du blé à donner en échange pour de la viande. Ainsi, le prix de la viande de boucherie et, par conséquent du bétail, doit s’élever par degrés, jusqu’à ce qu’il monte assez haut pour qu’on trouve autant de profit à em­ployer les terres les plus fertiles et les mieux cultivées à y faire venir de la nourriture pour le bétail, qu’à y faire venir du blé ; mais il faut que l’avancement ait déjà fait bien des progrès, avant que le labourage ne soit assez étendu pour faire monter à ce point le prix du bétail ; et jusqu’à ce que ce prix ait atteint un tel degré, il ira toujours en s’élevant, si le pays est constamment dans un état progressif. Il y a peut-être en Europe des endroits où le prix du bétail n’a pas encore atteint cette hauteur ; il n’y était parvenu dans aucune partie de l’Écosse avant l’union. Si le bétail d’Écosse eût toujours été confiné au marché du pays, il aurait peut-être été presque impossible (dans une contrée où il y a tant de terre qui n’est bonne qu’à nourrir les bestiaux, comparative­ment à celle qu’on peut employer à d’autres usages) que jamais le prix du bétail se fût élevé assez haut pour qu’il y eût profit à cultiver la terre dans le but d’en nourrir. On a observé que, en Angleterre, dans le voisinage de Londres, le prix du bétail semblait avoir déjà atteint cette hauteur vers le commencement du siècle dernier ; mais il n’y est parvenu vraisemblablement que bien plus tard dans la plus grande partie des comtés qui sont plus éloignés, et il y en a peut-être quelques-uns où il n’y est pas encore arrivé. Au reste, de tous les différents articles qui composent cette seconde classe de produit brut, le bétail est peut-être celui dont le prix s’élève le premier à cette hauteur, dans le cours des progrès de la civilisation matérielle.

À la vérité, jusqu’à ce que le prix du bétail soit parvenu à ce point, il ne paraît guère possible que la plus grande partie des terres, même de celles qui sont suscepti­bles de la meilleure culture, soient tout à fait cultivées. Dans toutes les fermes qui sont trop éloignées d’une grande ville pour s’y fournir d’engrais, c’est-à-dire dans la très-majeure partie des fermes de tous les pays étendus, la quantité de terres bien cultivées doit être nécessairement proportionnée à la quantité d’engrais que produit la ferme, et cette quantité d’engrais est aussi elle-même proportionnée au fonds de bétail que la ferme entretient. On engraisse la terre, soit en y laissant paître les bestiaux, soit en les nourrissant dans l’étable et en transportant de là leur fumier sur la terre. Or, à moins que le prix des bestiaux ne soit suffisant pour payer et la rente, et le profit d’une terre cultivée, le fermier ne peut trouver son compte à les mettre en pâture sur une pareille terre, et il l’y trouverait encore moins à les nourrir dans l’étable. Le bétail ne peut être nourri dans l’étable qu’avec le produit d’une terre cultivée et améliorée, parce qu’il faudrait beaucoup trop de travail et de dépense pour faire la récolte du produit maigre et épars des terres vagues et incultes. Ainsi, si le prix des bestiaux ne suffit pas à payer le produit d’une terre mise en valeur et cultivée, quand on les y laisse paître, à plus forte raison ne suffira-t-il pas à payer ce produit, s’il faut ajouter encore à la dépense un surcroît de travail pour le recueillir et le porter à l’étable. Dans cet état de choses donc, on ne peut nourrir dans l’étable, avec profit, plus de bestiaux que ce qui est nécessaire au labourage ; or, ceux-ci ne peuvent jamais donner assez d’engrais pour tenir constamment en bon état toutes les terres qui sont capables de recevoir de la culture. Ce qu’ils en donnent étant insuffisant pour toute la ferme, on le réservera naturellement pour les terres sur lesquelles il y a plus de profit ou de commodité à l’employer ; ce sera pour les plus fertiles, ou peut-être pour celles qui se­ront voisines des bâtiments de la ferme. Celles-ci seront tenues constamment en bon état et mises en culture ; le reste des terres, pour la plus grande partie, sera laissé en friche, et ne produira guère autre chose qu’une chétive pâture, à peine suffisante pour empêcher de mourir de faim quelques bêtes bien maigres, qu’on y laisse errer de côté et d’autre, attendu que la ferme quoique très-dégarnie de bestiaux, eu égard à ce qui serait nécessaire pour la cultiver complètement, s’en trouve le plus souvent surchar­gée, eu égard à son produit actuel. Cependant une portion de cette terre en friche, après avoir servi de pâture de cette manière misérable pendant six ou sept années de suite, pourra être labourée, et alors rendre peut-être une ou deux maigres récoltes de mauvaise avoine ou de quelques autres menus grains ; et ensuite se trouvant tout à fait épuisée, elle sera laissée en repos et abandonnée en vaine pâture, comme auparavant, tandis qu’une autre portion sera pareillement labourée, pour être ainsi épuisée et abandonnée à son tour de la même manière. Tel était le système général d’exploitation dans tout le plat pays d’Écosse avant l’union. Les terres qu’on tenait constamment bien fumées et en bon état ne faisaient presque jamais plus d’un tiers ou d’un quart de la totalité de la ferme, et quelquefois n’en formaient pas le cinquième ou même le sixième. Le reste n’était jamais fumé ; mais néanmoins il y en avait une certaine portion qui était à son tour régulièrement cultivée et épuisée. Il est évident que sous une pareille gestion, cette partie même des terres d’Écosse qui est susceptible d’une bonne culture ne pouvait produire que très-peu en comparaison de ce qu’elle est en état de produire. Mais, quelque désavantageux que puisse paraître ce système, cepen­dant, avant l’union, le bas prix des bestiaux le rendait, à ce qu’il semble presque inévitable. Si, malgré la hausse considérable du prix du bétail, ce système continue encore à dominer dans une assez grande partie du pays, il faut l’attribuer sans doute, en beaucoup d’endroits, à l’ignorance du peuple et à son attachement à ses anciens usages ; mais dans la plupart des endroits, c’est l’effet des obstacles inévitables que le cours naturel des choses oppose à l’établissement prompt et immédiat d’un meilleur système d’économie. Ces obstacles sont : 1° la pauvreté des tenanciers, qui n’ont pas encore eu le temps d’acquérir un fonds de bétail suffisant pour une culture plus com­plète de leurs terres, cette même hausse du prix des bestiaux, qui leur ferait trouver du profit à en entretenir un plus grand fonds, leur en rendant aussi l’acquisition plus difficile ; et 2° en supposant qu’ils eussent été dans le cas de l’acquérir, le défaut de temps, qui ne leur a pas encore permis de mettre leurs terres en état d’entretenir convenablement ce plus grand fonds de bétail. L’augmentation du fonds de bétail et l’amélioration de la terre sont deux choses qui doivent marcher de pair, et dont l’une ne peut guère aller plus vite que l’autre. On ne saurait améliorer la terre sans quelque augmentation de bestiaux ; mais on ne peut pas faire d’augmentation de bestiaux un peu importante, à moins d’une amélioration déjà considérable de la terre, autrement la terre ne pourrait pas les nourrir. Ces obstacles naturels à l’établissement d’un meilleur système d’exploitation ne peuvent céder qu’à une longue suite de travaux et d’écono­mies ; et il faut qu’il s’écoule plus d’un demi-siècle, plus d’un siècle peut-être, avant que l’ancien système, qui se détruit de jour en jour, puisse être complètement aboli dans toutes les différentes parties du pays[68]. Au reste, de tous les avantages commer­ciaux que l’Écosse a retirés de son union avec l’Angleterre, la hausse du prix de son bétail est peut-être le plus grand. Non-seulement cette hausse a ajouté à la valeur de toutes les propriétés des montagnes, mais elle a peut-être encore été la cause prin­cipale de l’amélioration des terres dans le plat pays.

Dans toutes les colonies nouvelles, la grande quantité de terres incultes qui ne peuvent pendant beaucoup d’années être employées à autre chose qu’à nourrir des bestiaux, les rend bientôt extrêmement abondants, et en toutes choses la grande abon­dance engendre nécessairement le bon marché. Quoique tous les bestiaux des colonies européennes de l’Amérique aient été originairement transportés d’Europe, ils y ont bientôt multiplié à un tel point, et y sont devenus de si peu de valeur, qu’on a laissé même les chevaux en liberté dans les bois et sans maître, sans qu’aucun propriétaire voulût prendre la peine de les réclamer. Dans de telles colonies, il faut qu’il se soit écoulé bien du temps depuis le premier établissement, pour qu’on en vienne à y trouver du profit à nourrir le bétail avec le produit d’une terre cultivée. Ainsi les mê­mes causes, c’est-à-dire le défaut d’engrais et la disproportion entre le capital employé à la culture, et la terre que ce capital est destiné à cultiver, y doivent probablement introduire un système d’exploitation assez semblable à celui qui continue encore à avoir lieu dans plusieurs endroits de L’Écosse. Aussi M. Kalm, voyageur suédois, en rendant compte de l’état de la culture de quelques-unes des colonies anglaises de l’Amérique septentrionale telle qu’il la trouva en 1749, observe-t-il qu’il lui fut difficile d’y reconnaître la nation anglaise, si habile dans toutes les diverses branches d’agriculture. « À peine, dit ce voyageur, fument-ils leurs terres à blé ; mais quand une pièce de terre a été épuisée par des récoltes successives, ils défrichent et cultivent une autre nouvelle pièce de terre, et quand celle-ci est épuisée, ils passent à une troisième. Ils laissent errer leurs bestiaux à travers les bois et les terres incultes où ces animaux meurent presque de faim, ayant déjà depuis longtemps détruit presque toutes les plantes annuelles des pâturages, en les broutant de trop bonne heure au printemps, avant que l’herbe ait eu le temps de pousser sa fleur et de jeter ses semences[69]. » Les plantes annuelles formaient, à ce qu’il semble, les meilleurs prés naturels de cette partie de l’Amérique septentrionale, et lors des premiers établissements des Euro­péens, elles croissaient ordinairement fort épais à la hauteur de 3 ou 4 pieds. Une pièce de terre qui, dans le temps où écrivait ce voyageur, ne pouvait nourrir une vache, aurait pu aisément, dans ces premiers temps, à ce qu’on lui assura, en nourrir quatre, dont chacune eût donné quatre fois autant de lait que celle-là pouvait en donner. Suivant lui, c’était cette chétive pâture qui causait la dégradation de leur bétail, dont la race s’abâtardissait sensiblement d’une génération à l’autre. Vraisem­blablement ce devait être à peu près comme cette espèce rabougrie si commune dans toute L’Écosse, il y a trente ou quarante ans, et qui s’est si fort amendée aujourd’hui dans la plus grande partie du plat pays, moins par un changement de race (quoiqu’on ait employé cet expédient dans quelques endroits), que par une meilleure méthode de nourrir.

Ainsi, quoique dans les progrès de l’amélioration le bétail n’arrive que tard à ce prix qui fait trouver du profit à cultiver la terre exprès pour le nourrir, cependant, de tous les différents articles qui composent cette seconde classe de produit brut, c’est peut-être le premier qui atteigne ce prix, parce que jusqu’à ce qu’il l’ait atteint, il paraît impossible que l’amélioration approche jamais même de ce degré de perfection auquel elle est portée dans plusieurs parties de l’Europe.

Dans cette classe de produit brut, si le bétail est une des premières parties qui attei­gne ce prix, le gibier est peut-être une des premières. Quelque exorbitant que puis­se paraître le prix de la venaison en Angleterre, il s’en faut encore qu’il puisse compenser la défense d’un parc de bêtes fauves, comme le savent très-bien tous ceux qui se sont occupés de la conservation de ce genre de gibier. S’il en était autrement, ce serait bientôt un objet de fermage ordinaire que d’élever des bêtes fauves, comme c’en était un, chez les anciens Romains, d’élever de ces petits oiseaux qu’ils nommaient turdi. Varron et Columelle nous assurent que c’était une industrie très-lucrative. On dit que c’en est une en certains endroits de la France que d’engraisser des ortolans, sortes d’oiseaux de passage, qui arrivent maigres dans le pays. Si la chair de daim continue à être en vogue, et que la richesse et le luxe augmentent encore en Angle­terre, comme ils ont fait depuis quelque temps, le prix de cette sorte de viande pourra vraisemblablement monter encore plus haut qu’il n’est à présent[70].

Entre cette période des progrès de l’amélioration qui porte à son plus haut point le prix d’un article aussi nécessaire que le bétail, et celle qui y porte le prix d’un article aussi superflu que la venaison, il y a un intervalle immense dans le cours duquel plu­sieurs autres espèces de produit brut arrivent par degrés au plus haut point de leur prix, les unes plus tôt, les autres plus tard, selon les différentes circonstances.

Ainsi, dans toutes les fermes, les rebuts de la grange et de l’étable peuvent entre­tenir un certain nombre de volailles. Comme celles-ci sont nourries de ce qui serait perdu autrement, on les a seulement pour faire profit de tout ; et comme elles ne coûtent presque rien au fermier, il peut trouver encore son compte à les vendre à très-bas prix. Presque tout ce qu’il en retire est gain, et leur prix ne peut guère être assez bas pour le décourager d’en nourrir le même nombre. Mais dans des pays mal cultivés et, par conséquent, faiblement peuplés, les volailles qu’on élève ainsi sans frais sont souvent suffisantes pour fournir largement à toute la demande. Ainsi, dans cet état de choses, elles sont souvent à aussi bon marché que la viande de boucherie ou que toute autre nourriture animale. Mais toute la quantité de volaille que la ferme produit ainsi sans frais doit toujours être beaucoup moindre que toute la quantité de viande de boucherie qui s’y élève ; et dans les temps d’opulence et de luxe, à mérite presque égal, ce qui est rare est toujours pour cela seul préféré à ce qui est commun. À mesure donc qu’en conséquence de l’amélioration et de l’extension de la culture, l’opulence et le luxe viennent s’accroître, le prix de la volaille vient aussi à s’élever par degrés au-dessus de la viande de boucherie, jusqu’à ce qu’enfin il s’élève assez haut pour qu’il y ait profit à cultiver la terre exprès pour en nourrir. Quand le prix est arrivé à ce point, il ne peut plus monter davantage, autrement on consacrerait plus de terre à cet usage. Dans plusieurs provinces de la France, la nourriture des volailles est regardée comme un article très-important de l’économie rurale, et comme suffisamment lucratif pour encourager le fermier à cultiver une quantité considérable de blé d’Inde et de sarrasin exprès pour les nourrir. Un fermier médiocre aura quelquefois quatre cents têtes de volaille dans sa basse-cour. En Angleterre, la nourriture des volailles ne paraît pas être encore regardée généralement comme un objet aussi important. Elles sont cepen­dant certainement plus chères en Angleterre qu’en France, puisque l’Angleterre en tire une quantité considérable de ce pays. Dans le cours des progrès de l’amélioration, l’époque à laquelle chaque espèce particulière de viande est la plus chère doit natu­rellement être l’époque qui précède immédiatement la pratique générale de cultiver exprès pour multiplier cette viande ; car, quelque temps avant que cette pratique ne s’établisse généralement, il faut bien nécessairement que la rareté ait élevé le prix de cet article de produit. Lorsque la pratique est généralement établie, on découvre com­mu­n­ément de nouvelles méthodes d’élever les animaux qui donnent cette viande particulière, de manière que le fermier se trouve en état d’en élever une plus grande quantité sur la même étendue de terre.

Non-seulement l’abondance de cet article l’oblige à vendre à meilleur marché, mais encore les méthodes perfectionnées le mettent à même de trouver son compte en vendant à meilleur marché ; car, s’il ne l’y trouvait pas, l’abondance ne pourrait durer longtemps. C’est vraisemblablement ainsi que l’introduction des luzernes, des turneps, des choux, des carottes, etc., a contribué à faire baisser le prix ordinaire de la viande de boucherie au marché de Londres, un peu au-dessous de ce qu’elle était vers le commencement du siècle dernier[71].

Le cochon, qui trouve à se nourrir dans l’ordure, et qui dévore avidement mille choses que rebutent les autres animaux utiles, est un animal qu’on a, dans l’origine, comme la volaille, pour faire profit de tout. Tant que le nombre de ceux qu’on peut nourrir ainsi pour rien, ou presque rien, suffit complètement à remplir la demande, cette sorte de viande vient au marché à beaucoup plus bas prix qu’aucune autre. Mais quand la demande excède ce que cette quantité-là peut fournir, quand il devient nécessaire de faire croître de la nourriture exprès pour nourrir et engraisser des porcs, comme on fait pour d’autre bétail, alors le prix de cette viande hausse nécessairement, et devient en proportion ou plus élevé ou plus bas que le prix de l’autre viande de boucherie, suivant que, par la nature du pays ou l’état de sa culture, il se trouvera que les cochons coûteront ou plus ou moins à nourrir que d’autre bétail. Selon M. de Buffon, le prix du porc, en France, est à peu près le même que celui du bœuf. Dans la plupart des endroits de la Grande-Bretagne, il est à présent un peu plus élevé.

On a souvent attribué la hausse considérable du prix des cochons et de la volaille en Angleterre à la diminution du nombre des cottagers et autres petits tenanciers ; dimi­nution qui, dans tous les endroits de l’Europe, a été le prélude immédiat de l’amélioration et de la meilleure culture, mais qui, en même temps, peut bien avoir contribué à élever le prix de ces deux articles un peu plus tôt et un peu plus rapide­ment qu’il n’aurait fait sans cela. De même que le ménage le plus pauvre peut souvent nourrir un chat et un chien sans aucune dépense ; de même les plus pauvres tenanciers pouvaient ordinairement nourrir presque pour rien quelque peu de volaille ou une truie avec quelques petits. Les restes de leur table, leur petit-lait, le lait écrémé et le lait de beurre, faisaient une partie de la nourriture de ces animaux, qui trouvaient à vivre pour le surplus dans les champs du voisinage, sans faire à personne de domma­ge sensible. Ainsi, la diminution du nombre de ces petits tenanciers a dû certainement diminuer de beaucoup la quantité de ces sortes de denrées, qui se produisaient pour rien ou presque rien ; et par conséquent, le prix a dû s’en élever et plus tôt et plus rapidement qu’il ne l’eût fait sans cela. Néanmoins, il faut toujours, dans le cours des progrès de l’amélioration, que ce prix monte, plus tôt ou plus tard, à son maximum, c’est-à-dire au prix qui peut payer le travail et la dépense de cultiver la terre par laquelle ces sortes d’animaux sont nourris, tout comme ce travail et cette dépense sont payés par la majeure partie des autres terres cultivées.

C’est aussi, originairement, pour mettre tout à profit que l’on établit la laiterie, tout comme on nourrit d’abord des cochons et de la volaille. Le bétail qu’on est obligé de tenir dans la ferme donne plus de lait qu’il n’en faut pour élever les petits et pour la consommation du ménage du fermier, et cet excédent est encore plus fort dans une saison particulière. Or, de toutes les productions de la terre, le lait est peut-être la plus périssable. Dans les temps chauds, où il est le plus abondant, à peine se garde-t-il vingt-quatre heures. Le fermier en convertit une petite partie en beurre frais, laquelle, par ce moyen, pourra se garder pendant une semaine ; une autre en beurre sale, qui se conservera pendant une année, et une beaucoup plus grande partie en fromage, qui pourra se garder plusieurs années. Il réserve une partie de toutes ces choses pour l’usage de sa famille ; le reste va au marché pour y être donné au meilleur prix qu’on pourra trouver, et ce prix ne peut guère être assez bas pour le décourager d’y envoyer tout ce qui excède la consommation de son ménage. À la vérité, si le prix est extrê­mement bas, il est probable que le fermier tiendra tout ce qui concerne le laitage d’une manière fort négligée et fort malpropre ; il ne pensera guère que cet article vaille la peine d’avoir exprès un bâtiment ou une pièce particulière, mais il laissera faire tout le travail de la laiterie dans sa cuisine, au milieu de la fumée, de la mauvaise odeur et des ordures, comme cela se pratiquait dans presque toutes les fermes d’Écosse il y a trente ou quarante ans, et comme cela se fait encore dans plusieurs. Les mêmes causes qui font monter par degrés le prix de la viande de boucherie, c’est-à-dire l’accrois­sement de la demande et la diminution de la quantité de bétail qu’on peut nourrir pour rien ou presque rien, résultat nécessaire de l’amélioration du pays, concourent de la même manière à faire monter en valeur le produit de la laiterie, produit dont le prix a une connexion naturelle avec celui de la viande de boucherie ou avec la dépense que coûte la nourriture du bétail. L’augmentation de ce prix paye un surcroît de peine, de soins et de propreté. Le laitage mérite alors davantage l’attention du fermier, et sa qualité se perfectionne de plus en plus. Le prix enfin en monte assez haut pour qu’il vaille la peine qu’on emploie quelques-unes des terres les plus fertiles et les mieux cultivées à nourrir des bestiaux exprès pour en avoir le lait ; et quand le prix a monté jusqu’à ce point, il ne peut guère aller plus haut. S’il montait davantage, on n’y consa­crerait bientôt plus de terre[72]. Il paraît qu’il a atteint ce maximum dans la majeure partie de l’Angleterre, où communément on emploie à cet objet beaucoup de bonnes terres. Si vous en exceptez le voisinage d’un petit nombre de grandes villes, il ne paraît pas encore être arrivé à ce point dans aucun autre endroit de l’Écosse, où il est rare de voir les fermiers ordinaires consacrer beaucoup de bonnes terres à nourrir des bestiaux, uniquement pour leur lait. Le prix de ce produit est vraisemblablement enco­re trop bas pour le permettre, quoique depuis quelques années il ait considérable­ment monté. Il est vrai que, comparé au laitage d’Angleterre, l’infériorité de qualité répond bien en entier à celle du prix ; mais cette infériorité de qualité est peut-être plutôt l’effet de la modicité du prix qu’elle n’en est la cause. Quand même la qualité serait beaucoup meilleure qu’elle n’est, j’imagine que, dans l’état actuel du pays, la plus grande partie de ce qu’on en porte au marché ne pourrait pas y trouver le débit à un beaucoup plus haut prix ; et il est vraisemblable, d’un autre côté, que le prix actuel ne pourrait pas payer la dépense de terre et de travail nécessaire pour produire du lait d’une beaucoup meilleure qualité. Quoique le laitage soit à un prix plus élevé dans la majeure partie de l’Angleterre, cependant cette nature d’emploi de la terre ne de la terre ne passe pas pour avoir une supériorité d’avantages sur la culture du blé ou l’engrais du bétail, qui sont les deux principaux objets de l’agriculture ; donc cette nature d’emploi ne peut pas même avoir encore l’égalité d’avantages dans la majeure partie de l’Écosse.

Il est évident que les terres d’un pays ne peuvent jamais parvenir à un état d’amé­lioration et de culture complète, avant que le prix de chaque produit que l’industrie humaine se propose d’y faire croître ne soit d’abord monté assez haut pour payer la dépense d’une amélioration et d’une culture complètes. Pour que les choses en soient là, il faut que le prix de chaque produit particulier suffise à payer d’abord la rente d’une bonne terre à blé, qui est celle qui règle la rente de la plupart des autres terres cultivées, et à payer en second lieu le travail et la dépense du fermier, aussi bien qu’ils se payent communément sur une bonne terre à blé, ou bien, en autres termes, à lui rendre, avec les profits ordinaires, le capital qu’il y emploie. Cette hausse dans le prix de chaque produit particulier doit évidemment précéder l’amélioration et la culture de la terre destinée à faire naître ce produit. Le gain est le but qu’on se propose dans toute amélioration, et rien de ce qui entraîne à sa suite une perte nécessaire ne peut s’appeler amélioration. Or, préparer et cultiver la terre dans la vue d’y faire naître un produit dont le prix ne rapporterait jamais la dépense, est une chose qui entraîne avec soi une perte nécessaire. Ainsi, si l’amélioration et la culture complète d’un pays est le plus grand de tous les avantages publics, comme on n’en peut faire aucun doute, cette hausse dans le prix de toutes ces différentes sortes de produit brut, bien loin d’être regardée comme une calamité publique, doit être regardée comme l’avant-coureur et comme la suite nécessaire du plus grand de tous les avantages pour la société.

Ce n’est pas non plus par l’effet d’une dégradation dans la valeur de l’argent, mais c’est par l’effet d’une hausse dans leur prix réel, que le prix nominal ou en argent de toutes ces différentes sortes de produit brut a haussé. Elles en sont venues à valoir, non une plus grande somme d’argent, mais une plus grande quantité de travail et de subsistances qu’auparavant. Comme il en coûte une plus grande dose de travail et de subsistances pour les faire venir au marché, par cela même elles en représentent ou en valent une plus grande quantité quand elles y sont venues[73].

\subsubsection*{Troisième classe}

La troisième et dernière classe de produit brut dont le prix s’élève naturellement dans les progrès que fait l’amélioration d’un pays, c’est cette espèce de produit sur la multiplication duquel l’industrie humaine n’a qu’un pouvoir limité ou incertain. Ainsi, quoique le prix réel de cette classe de produit brut tende naturellement à s’élever dans le cours des progrès de l’amélioration, néanmoins, selon que, d’après différentes circonstances accidentelles, les efforts de l’industrie humaine se trouvent réussir plus ou moins à augmenter la quantité de ce produit, il peut se faire que ce prix vienne quelquefois à baisser, qu’il se soutienne quelquefois au même taux dans des périodes d’amélioration très-différentes, et quelquefois qu’il hausse plus ou moins dans une même période d’amélioration.

Il y a certaines espèces de produits bruts qui sont par leur nature en quelque sorte accessoires et dépendant d’une autre espèce ; de manière que la quantité de l’une de ces espèces qu’un pays peut fournir est nécessairement limitée par la quantité de l’autre. Par exemple, la quantité de laines ou de peaux crues qu’un pays peut fournir est nécessairement limitée par le nombre du gros et menu bétail qu’on y entretient, et ce nombre est encore déterminé nécessairement par l’état de l’amélioration de ce pays et la nature de son agriculture.

On pourrait penser que les mêmes causes qui, dans le progrès de l’amélioration, font hausser par degrés le prix de la viande de boucherie, devraient produire le même effet sur le prix des laines et des peaux crues, et faire monter aussi ce prix à peu près dans la même proportion. Il en serait ainsi vraisemblablement si, dans les premiers commencements informes de la civilisation, le marché pour les dernières de ces pro­duc­tions était renfermé dans des bornes aussi étroites que le marché pour la première ; mais communément leurs marchés respectifs sont d’une étendue extrêmement différente.

Presque partout le marché pour la viande de boucherie est borné au pays qui la produit. À la vérité, l’Irlande et quelques parties de l’Amérique anglaise font un com­mer­ce assez important en viandes salées, mais ce sont, je pense, les seuls pays du monde commerçant qui exportent dans d’autres pays une partie considérable de leur viande de boucherie. 

Au contraire, le marché pour la laine et les peaux crues, dans ces commencements informes, est bien rarement borné au pays qui les produit. La laine, sans qu’il soit besoin d’aucun apprêt, les peaux crues, avec fort peu d’apprêt, se transportent facile­ment dans les pays éloignés ; et comme ce sont les matières de beaucoup d’ouvrages de manufactures, l’industrie des autres pays peut donner lieu à une demande pour ces denrées, quand même celle du pays qui les produit n’en occasionnerait aucune.

Dans les pays mal cultivés et qui, par conséquent, ne sont que très-faiblement peuplés, le prix de la laine et de la peau est toujours beaucoup plus grand, relative­ment à celui de la bête entière, que dans les pays qui, étant plus avancés en richesse et en population, ont une plus grande demande de viande de boucherie. M. Hume obser­ve que du temps des Saxons la toison était estimée valoir deux cinquièmes de la valeur de la brebis entière, et que cette proportion est fort au-dessus de l’estimation actuelle. On m’a assuré que dans quelques provinces d’Espagne il arrivait fréquem­ment de tuer une brebis uniquement pour avoir la toison et le suif ; on laisse le corps pourrir sur la terre, ou servir de pâture aux bêtes et aux oiseaux de proie. Si cela arrive quelquefois même en Espagne, c’est presque toujours le cas au Chili, à Buenos-Ayres et dans plusieurs autres endroits de l’Amérique espagnole, où on ne tue guère les bêtes à cornes que pour leur cuir et leur suif seulement. C’était aussi l’usage presque constant dans la partie espagnole et à Saint-Domingue, quand cette île était infestée par les boucaniers, et avant que l’établissement des colonies françaises, qui s’étendent maintenant autour des côtes de presque toute la moitié occidentale de cette île, eussent, par leur industrie et leur population, donné quelque valeur au bétail des Espagnols, qui sont encore en possession, non-seulement de la partie orientale de la côte, mais encore de toute la partie intérieure et des montagnes[74].

Quoique, dans l’avancement des arts et de la population, le prix de la bête entière s’élève nécessairement, cependant il est vraisemblable que cette hausse portera beaucoup plus sur le prix du corps de la bête que sur celui de la laine et de la peau. Le marché pour le corps de la bête, qui, dans l’état d’une civilisation ébauchée, se trouve toujours borné au pays qui la produit, doit nécessairement s’agrandir dans la même proportion que l’industrie et la population du pays. Mais comme le marché pour la laine et les peaux, même dans un pays encore barbare, s’étend souvent à tout le monde commerçant, il ne peut presque jamais s’agrandir dans cette même proportion. L’amé­lio­ration d’un pays particulier ne peut plus guère influer sur l’état du monde commer­çant, et après cette amélioration le marché pour ces denrées peut rester le même ou à peu près le même qu’il était auparavant. Cependant, dans le cours naturel des choses, cette amélioration doit bien lui donner quelque avantage, surtout si les manufactures dont ces denrées sont les matières premières venaient à fleurir dans le pays ; le marché, sans être fort agrandi par là, se trouverait au moins beaucoup plus rapproché qu’auparavant du lieu où croissent ces matières, et le prix de celles-ci augmenterait au moins de ce qu’on avait eu coutume de dépenser pour les transporter au loin. Ainsi, quoique ce prix ne puisse pas s’élever dans la même proportion que celui de la viande de boucherie, il doit naturellement s’élever de quelque chose, ou du moins il ne doit certainement pas baisser.

Cependant, malgré l’état florissant des manufactures en Angleterre, le prix des laines anglaises a baissé considérablement dans ce pays depuis le temps d’Édouard III. Il y a une foule de preuves authentiques qui démontrent que pendant le règne de ce prince (vers le milieu du quatorzième siècle, ou environ vers 1339), ce qui était regardé comme le prix modéré et raisonnable de la balle ou de vingt-huit livres de laine, poids d’Angleterre, n’était pas à moins de 10 schellings de l’argent d’alors[75], contenant, sur le pied de 20 deniers l’once, 6 onces d’argent, poids de la Tour, valant environ 30 schellings de notre monnaie actuelle. Aujourd’hui, on regarderait 21 schellings comme un bon prix pour la balle de vingt-huit livres de la plus belle laine d’Angleterre. Ainsi, le prix pécuniaire de la laine, au temps d’Édouard III, était à son prix pécuniaire d’aujourd’hui comme 10 est à 7. La supériorité de son prix réel était encore plus forte. Dix schellings étaient dans ce temps-là le prix de douze boisseaux de blé, sur le pied de 6 schellings 8 deniers le quarter. Aujourd’hui, à raison de 28 schellings le quarter, 21 schellings ne valent que six boisseaux. Ainsi la proportion des deux prix réels est comme 12 est à 6, comme 2 est à 1. Une balle de vingt-huit livres de laine aurait acheté, dans ces temps anciens, deux fois la quantité de sub­sistance qu’elle achèterait à présent et, par conséquent, deux fois la quantité de travail, si dans l’une et l’autre de ces deux périodes la récompense du travail eût été la même[76].

Cette dégradation, tant dans le prix nominal que dans le prix réel des laines, n’aurait jamais pu arriver dans le cours naturel des choses. Aussi, a-t-elle été l’effet de la contrainte et de l’artifice. Elle procède : 1° de la prohibition absolue d’exporter de la laine d’Angleterre[77] ; 2° de la permission de l’importer de l’Espagne sans payer de droits ; 3° de la défense de l’exporter de l’Irlande en tout autre pays qu’en Angleterre[78]. En conséquence de ces règlements, le marché pour la laine d’Angleterre, au lieu d’avoir reçu quelque extension par l’amélioration de ce pays, a été confiné au marché inté­rieur, où la laine de tous les autres pays peut venir en concurrence avec elle, et où celle d’Irlande est forcée d’y venir[79]. Comme d’ailleurs les manufactures de laine en Irlande sont aussi découragées que peuvent le permettre la justice et la bonne foi du commerce, les Irlandais ne peuvent mettre en œuvre qu’une très-petite quantité de leurs laines, et ils sont en conséquence obligés d’en envoyer une plus grande quantité en Angleterre, qui est le seul marché où il leur soit permis d’en expédier.

Je n’ai pas trouvé des renseignements aussi authentiques sur le prix des peaux crues dans les temps anciens. La laine était ordinairement payée au roi comme subside, et son évaluation dans le subside atteste au moins, à un certain point, quel était son prix ordinaire. Mais il ne paraît pas qu’il en fût de même pour les peaux crues ; cependant on trouve dans Fleetwood un compte de 1425, entre le prieur de Burcester-Oxford et un de ses chanoines, qui nous donne ce prix, du moins tel qu’il fut établi dans cette circonstance particulière ; savoir : pour cinq cuirs de bœuf, 12 schellings ; pour cinq cuirs de vache, 7 schellings 3 deniers ; pour trente-six peaux de mouton de deux ans, 9 schellings ; pour seize peaux de veau, 2 schellings. En 1425, 12 schellings contenaient environ la même quantité d’argent que 24 de notre monnaie actuelle. Ainsi, à ce compte, un cuir de bœuf valait autant d’argent qu’il y en a dans 4 schellings quatre cinquièmes de notre monnaie actuelle ; son prix nominal était de beaucoup plus bas qu’aujourd’hui. Mais dans ces temps-là, où le blé était sur le pied de 6 schellings 8 deniers le quarter, 12 schellings en auraient acheté 14 boisseaux et quatre cinquièmes de boisseau, lesquels coûteraient aujourd’hui, à raison de 3 schellings 6 deniers le boisseau, 51 schellings 4 deniers. Ainsi un cuir de bœuf, dans ces temps-là, aurait acheté autant de blé que 10 schellings 3 deniers en achèteraient à présent. Sa valeur réelle était égale à 10 schellings 3 deniers de notre monnaie actuelle. Dans ces temps anciens, où le bétail était à demi mourant de faim pendant la plus grande partie de l’hiver, il n’est pas à présumer qu’il fût d’une très-belle taille. Un cuir de bœuf qui pèse quatre stones[80] de seize livres avoir du poids, n’est pas regardé actuellement comme très-chétif, et aurait vraisemblablement passé pour très-beau dans ces temps-là. Or, un cuir de cette espèce, à raison d’une demi-couronne[81] le stone, qui est en ce moment (février 1773) le prix ordinaire, ne coûterait aujourd’hui que 10 schellings. Ainsi, quoique son prix nominal soit maintenant un peu plus haut qu’il n’était dans ces anciens temps, son prix réel, la quantité réelle de subsistances qu’il achètera ou dont il disposera, est plutôt de quelque chose au-dessous. Le prix des cuirs de vache, tel qu’il est porté dans le compte ci-dessus, approche beaucoup de sa proportion ordinaire avec le prix des cuirs de bœuf. Le prix des peaux de mouton" est fort au-dessus de cette proportion ; probablement qu’elles furent vendues avec leur laine ; celui des peaux de veau, au contraire, est de beaucoup au-dessous. Dans les pays où le prix du bétail est fort bas, les veaux qu’on n’a pas dessein d’élever pour entretenir le fonds de bétail de la ferme sont en général tués très-jeunes ; on en usait ainsi en Écosse il y a vingt ou trente ans. On épargne le lait que leur prix ne suffirait pas à payer ; en conséquence, leurs peaux ne sont ordinairement pas bonnes à grand-chose.

Le prix des peaux crues est de beaucoup plus bas aujourd’hui qu’il n’était il y a quelques années ; ce qui vient vraisemblablement de la suppression du droit sur les peaux de veau-marin[82], et de la permission qui a été donnée, en 1769, pour un temps limité, d’exporter les peaux crues de l’Irlande et des colonies, franches de droits. En faisant un taux moyen pour tout ce siècle, leur prix réel a vraisemblablement été tant soit peu plus haut qu’il n’était dans ces anciens temps. La nature de cette denrée ne la rend pas tout à fait aussi propre à être transportée au loin que la laine. Elle risque plus à être gardée. Un cuir salé est regardé comme inférieur à un cuir frais, et se vend moins cher. Cette circonstance doit nécessairement tendre à faire baisser le prix des peaux crues produites dans un pays qui ne les manufacture point, mais qui est obligé de les exporter, et à élever comparativement le prix de celles qui sont produites dans un pays où on les manufacture. Elle doit tendre à faire baisser leur prix dans un pays barbare, et à le faire hausser dans un pays riche et manufacturier. Elle doit donc avoir eu quelque tendance à le faire baisser dans l’ancien temps, et à le faire monter dans celui-ci. Et puis, nos tanneurs n’ont pas tout à fait aussi bien réussi que nos fabricants de draps à persuader à la sagesse nationale que le salut de la chose publique dépendait de la prospérité de leur manufacture particulière. Ils ont en conséquence été bien moins favorisés. À la vérité, on a prohibé l’exportation des peaux crues, et on l’a déclarée délit public[83]. Mais leur importation des pays étrangers a été assujettie à un droit ; et quoique ce droit ait été supprime pour cinq années seulement sur celles importées de l’Irlande et de nos colonies, cependant l’Irlande n’a pas été restreinte au seul marché de la Grande-Bretagne pour la vente de l’excédent de ses peaux ou de celles qui ne sont pas manufacturées chez elle. Il n’y a que très-peu d’années que les peaux de bétail commun ont été mises au nombre des marchandises que les colonies ne peuvent envoyer ailleurs qu’à la mère patrie[84], et enfin le commerce d’Irlande n’a pas, sur cet article, été opprimé jusqu’à présent dans la vue de soutenir les manufac­tures de la Grande-Bretagne.

Tous les règlements, quels qu’ils soient, qui tendent à abaisser le prix, soit de la laine, soit de la peau crue, au-dessous de ce qu’il serait naturellement, doivent néces­sai­rement, dans un pays cultivé et amélioré, avoir quelque tendance à faire hausser le prix de la viande de boucherie. Il faut que le prix du bétail qu’on nourrit sur une terre améliorée et cultivée, soit gros, soit menu bétail, suffise à payer la rente et le profit que le propriétaire et le fermier sont en droit d’attendre d’une terre améliorée et culti­vée. Sans cela, ceux-ci cesseront bientôt d’en nourrir. Ainsi, toute partie de ce prix qui ne se trouve pas payée par la laine et la peau, il faut que le corps la paye. Moins on paye pour l’un de ces articles, plus il faut payer pour l’autre. Pourvu que le pro­prié­taire et le fermier trouvent tout leur prix, il leur importe peu comment il sera réparti sur les différentes parties de la bête. Ainsi, comme propriétaires et comme fermiers, dans tout pays cultivé et amélioré, ils ne peuvent guère être lésés par de tels règle­ments, quoiqu’ils puissent en souffrir comme consommateurs, par la hausse du prix des vivres. Cependant il en serait tout autrement dans un pays sans amélioration et sans culture, où la majeure partie des terres ne pourrait être employée qu’à nourrir des bestiaux, et où la laine et la peau feraient la plus grande partie de la valeur de l’animal. Dans ce cas leur intérêt, comme propriétaires et comme fermiers, souffrirait extrêmement de semblables règlements, et leur intérêt, comme consommateurs, en souffrirait très-peu. Dans ce cas, la baisse du prix de la laine et de la peau ne ferait pas monter le prix du corps de la bête, parce que, la majeure partie des terres du pays ne pouvant servir qu’à nourrir du bétail, on en nourrirait toujours tout autant. Il viendrait toujours au marché la même quantité de viande de boucherie. La demande n’en serait pas plus forte qu’auparavant ; ainsi son prix resterait le même. Le prix total de l’animal baisserait et, avec lui, tant la rente que le profit de toutes ces terres dont le bétail faisait le produit principal, c’est-à-dire de la majeure partie des terres du pays. Dans de telles circonstances, la prohibition perpétuelle d’exporter la laine, qu’on attribue communément, mais à tort, à Édouard III, eût été le règlement le plus des­tructif qu’on eût pu jamais imaginer. Non-seulement il aurait diminué la valeur alors actuelle de la majeure partie des terres, mais encore, en abaissant le prix de l’espèce de menu bétail la plus importante, il aurait prodigieusement retardé l’amélioration ultérieure du royaume.

Le prix des laines d’Écosse baissa considérablement par suite de l’union avec l’Angleterre, par laquelle elles furent exclues du grand marché de l’Europe, et con­fi­nées dans les bornes étroites du marché de la Grande-Bretagne. Cet événement aurait extrêmement influé sur la valeur de la majeure partie des terres des comtés méridio­naux d’Écosse, qui sont principalement des pays de moutons, si la baisse du prix de la laine n’eût été largement compensée par la hausse de celui de la viande de boucherie. 

Si, d’un côté, le pouvoir de l’industrie humaine sur la multiplication de la laine et des peaux crues est limité dans ses effets, en tant qu’il dépend du produit du pays dans lequel s’exerce cette industrie, d’un autre côté, ce pouvoir est incertain dans ses effets en tant qu’il dépend du produit des autres pays ; et à cet égard il en dépend bien moins à raison de la quantité que ces autres pays produisent, qu’à raison de la quantité qu’ils ne manufacturent pas eux-mêmes, et aussi à raison des restrictions qu’ils jugeront ou ne jugeront pas à propos de mettre à l’exportation de cette espèce de produit brut. Ces circonstances, étant entièrement indépendantes de l’industrie nationale, apportent nécessairement plus ou moins d’incertitude dans les efforts qu’elle peut faire pour multiplier cette espèce de produit brut.

Ses efforts sont également bornés et incertains pour la multiplication d’une autre sorte de produit brut très-important ; c’est la quantité du poisson mise au marché. Ils sont bornés par la situation locale du pays, par la distance ou la proximité où ses différentes provinces sont de la mer, par le nombre de ses lacs et de ses rivières, et enfin par ce qu’on peut appeler la fertilité ou stérilité de ces mers, lacs et rivières, quant à cette espèce de produit brut. À mesure que la population augmente, à mesure que le produit annuel de la terre et du travail du pays grossit de plus en plus, le nom­bre des acheteurs de poisson doit augmenter, et ces acheteurs possèdent une plus grande quantité et une plus grande diversité d’autres marchandises, ou, ce qui est la même chose, le prix d’une plus grande quantité et d’une plus grande diversité d’autres marchandises pour acheter ce poisson. Mais, en général, il sera impossible d’appro­visionner ce marché ainsi agrandi et étendu, sans employer pour cela une quantité de travail qui croisse au-delà de la proportion de celle qu’exigeait l’approvisionnement de ce marché quand il était circonscrit dans des limites plus étroites. Un marché qui, d’abord approvisionné avec mille tonneaux de poisson, vient par la suite à en absorber dix mille tonneaux, ne pourra guère être alors approvisionné à moins d’un travail qui sera plus que décuplé du travail qu’il fallait pour l’approvisionner dans le premier état. Il faut alors, en général, aller chercher le poisson à de plus grandes distances, il faut employer de plus grands bâtiments et mettre en œuvre des machines plus dispen­dieuses en tout genre. Ainsi le prix réel de cette denrée doit augmenter naturellement dans les progrès que fait l’amélioration ; c’est aussi, à ce que je crois, ce qui est arrivé, plus ou moins, en tout pays.

Quoique ce soit une chose fort incertaine que le succès de tel ou tel jour de pêche en particulier, cependant, la situation locale du pays une fois donnée, si vous prenez, en somme, tout le cours d’une année ou de plusieurs années ensemble, l’effet général du travail nécessaire, dira-t-on, pour amener au marché telle ou telle quantité de poisson, paraît être assez certain ; et dans le fait, il n’y a pas de doute que cela ne soit. Cependant, comme cet effet dépend plus de la situation locale du pays que de l’état de sa richesse et de son industrie ; comme d’après cela cet effet, dans les différents pays, peut être le même, quoique les degrés d’amélioration de ces pays soient très-diffé­rents, ou être fort différent, les degrés d’amélioration étant les mêmes, il en résulte que sa liaison avec l’état d’amélioration du pays est une chose incertaine, et c’est de cette espèce d’incertitude que j’entends parler ici.

Quant à l’augmentation de quantité des divers minéraux et métaux qu’on tire des entrailles de la terre, et particulièrement des plus précieux, le pouvoir de l’industrie humaine ne paraît pas être borné, mais il paraît être tout à fait incertain dans ses effets,

La quantité de métaux précieux qui peut exister dans un pays n’est bornée par rien qui tienne à la situation locale de ce pays, comme la fertilité ou la stérilité de ses pro­pres mines. Ces métaux se trouvent en abondance dans les pays qui ne possèdent point de mines. Leur quantité, dans chaque pays en particulier, paraît dépendre de deux circonstances différentes. La première, c’est le pouvoir qu’il a d’acheter, c’est l’état de son industrie, c’est le montant du produit annuel de ses terres et de son travail : circonstance qui le met en état d’employer une quantité plus grande ou plus petite de travail et de subsistances, à faire venir ou à acheter des superfluités, telles que sont l’or et l’argent, soit de ses propres mines, soit de celles des autres pays. La seconde, c’est l’état de fécondité ou de stérilité des mines qui, au moment dont il s’agit, approvisionneront de ces métaux le monde commerçant. Cette fécondité ou cette stérilité des mines doit influer plus ou moins sur la quantité de ces métaux dans les pays les plus éloignés des mines, à cause de la facilité et du peu de frais du transport de cette marchandise, résultant de son peu de volume et de sa grande valeur. L’abondance des mines de l’Amérique a dû avoir plus ou moins d’effet sur la quantité de ces métaux à la Chine et dans l’Indoustan. 

En tant que la quantité de ces métaux dans un pays particulier dépend de la première de ces deux circonstances (le pouvoir d’acheter), leur prix réel, comme celui de toute autre chose superflue et de luxe, doit vraisemblablement monter à mesure de la richesse et de l’amélioration du pays, et baisser à mesure de sa pauvreté et de sa décadence. Les pays qui ont une grande quantité de travail et de subsistances au-delà de leur besoin sont en état de dépenser, pour avoir une certaine quantité de ces métaux, une plus grande somme de travail et de subsistances que les pays qui en ont moins au-delà du nécessaire.

En tant que la quantité de ces métaux, dans un pays particulier, dépend de la seconde de ces circonstances (la fécondité ou stérilité des mines qui se trouvent alors approvisionner le monde commerçant), leur prix réel, la quantité réelle de travail et de subsistances qu’ils achèteront, ou pour laquelle on les échangera, baissera sans aucun doute plus ou moins en proportion de la fécondité de ces mines, et haussera plus ou moins en proportion de la fécondité de ces mines, et haussera plus ou moins en proportion de leur stérilité.

La fécondité ou la stérilité des mines qui se trouvent, à une époque donnée, approvisionner le monde commerçant, est toutefois une circonstance qui évidemment ne peut avoir aucune sorte de liaison avec l’état de l’industrie dans un pays quelcon­que. Elle semble même n’avoir aucune liaison nécessaire avec l’état de l’industrie du monde en général. Il est vrai qu’à mesure que les arts et le commerce viennent à se répandre sur une plus grande partie du globe, la recherche des mines nouvelles offrant aux spéculateurs une plus vaste surface, il peut y avoir quelques chances de plus de réussite que lorsqu’elle est circonscrite dans des bornes plus étroites.

Toutefois, rien n’est plus incertain au monde que de savoir si l’on parviendra à découvrir des mines nouvelles à mesure que les anciennes viendront successivement à s’épuiser, et il n’y a pas d’industrie ou de savoir humain qui puisse en répondre. Il est reconnu que toutes les indications sont douteuses, et que la seule chose qui puisse assurer la valeur réelle d’une mine ou même son existence, c’est sa découverte actu­elle et le succès de son exploitation. Dans une recherche de cette nature, on ne peut fixer jusqu’à quel point l’industrie humaine peut être heureuse ou trompée dans ses efforts. Il peut se faire que, dans le cours d’un siècle ou deux, on découvre de nouvelles mines plus fécondes que toutes celles connues jusqu’alors ; et il est tout aussi possible que les mines les plus fécondes, connues à cette même époque, soient plus stériles qu’aucune de celles qu’on exploitait avant la découverte des mines de l’Amérique. Quelle que soit l’une ou l’autre de ces deux hypothèses qui eût lieu, elle serait de très-peu d’importance pour la richesse et la prospérité réelle du monde, pour la valeur réelle du produit annuel de la terre et du travail parmi les hommes. Sans doute, la valeur nominale de ce produit, la somme d’or ou d’argent par laquelle il serait exprime ou représenté, serait très-différente dans les deux cas ; mais la valeur réelle du produit, la quantité réelle de travail qu’il pourrait commander ou acheter, serait toujours précisément la même. Il se pourrait, dans l’un de ces cas, qu’un schelling ne représentât pas plus de travail qu’un penny n’en représente aujourd’hui et que, dans l’autre, un penny en représentât autant que fait à présent un schelling. Mais, dans le premier cas, celui qui aurait un schelling dans sa poche ne serait pas plus riche que celui qui a aujourd’hui un penny ; et dans le second cas, celui qui aurait alors un penny serait tout aussi riche que celui qui a un schelling à présent. Le seul avantage que le monde pourrait retirer de l’une de ces hypothèses, et le seul inconvénient qui résulterait pour lui de l’autre, ce serait, dans la première, l’abondance et le bon marché de la vaisselle et des bijoux d’or et d’argent et, dans la seconde, la rareté et la cherté de ces frivoles superfluités.

\subsection*{Conclusion de la digression sur les variations dans la valeur de l’argent}

La plupart des écrivains qui ont recueilli les prix en argent des denrées, dans les temps anciens, paraissent avoir regardé le bas prix, en argent, du blé et des mar­chan­dises en général, ou, en d’autres termes, la haute valeur de l’or ou de l’argent, comme une preuve non-seulement de la rareté de ces métaux, mais encore de la pauvreté et de la barbarie du pays à cette époque. Cette notion se lie à ce système d’économie poli­tique qui fait consister la richesse nationale dans l’abondance de l’or et de l’argent, et la pauvreté générale dans leur rareté, système que je tâcherai d’expliquer et d’exa­mi­ner fort au long dans le quatrième livre de ces Recherches. je me contenterai d’obser­ver, pour le moment, que la grande valeur des métaux précieux ne peut pas être la preuve de la pauvreté et de la barbarie d’un pays à l’époque où a lieu cette grande valeur. C’est seulement une preuve de la stérilité des mines qui se trouvent à cette époque approvisionner le monde commerçant. Comme un pays pauvre n’est pas en état d’acheter plus d’or et d’argent qu’un pays riche, par la même raison il n’est pas en état de les payer plus cher ; ainsi il n’est pas vraisemblable que ces métaux aient plus de dans le premier de ces pays que dans le dernier. À la Chine, qui est un pays beaucoup plus riche qu’aucun endroit de l’Europe, la valeur des métaux précieux est aussi beaucoup plus élevée qu’en aucun endroit de l’Europe[85]. À la vérité, la richesse de l’Europe s’est grandement accrue depuis la découverte des mines de l’Amérique, et la valeur de l’or et de l’argent y a aussi diminué successivement depuis la même époque. Toutefois, cette diminution de leur valeur n’est pas due à l’accroissement de la riches­se réelle de l’Europe, à l’accroissement du produit annuel de ses terres et de son travail, mais elle est due à la découverte accidentelle de mines plus abondantes qu’au­cu­ne de celles connues auparavant. L’augmentation de la quantité d’or et d’argent en Europe, et l’extension de son agriculture et de ses manufactures, sont deux événe­ments qui, pour être arrivés presque à la même époque, ont eu cependant leur source dans des causes très-différentes, et n’ont presque pas la moindre liaison l’un avec l’autre. L’un est provenu du pur effet du hasard, dans lequel la prudence ni la politique n’ont eu ni n’ont pu avoir aucune part ; l’autre est la conséquence de la chute du système féodal et de l’établissement d’une forme de gouvernement qui a donné à l’industrie le seul encouragement dont elle ait besoin, c’est-à-dire une confiance assez bien établie qu’elle pourra jouir du fruit de ses efforts. La Pologne, qui n’est pas délivrée du système féodal, est encore aujourd’hui un pays aussi misérable qu’il l’était avant la découverte de l’Amérique. Cependant le prix du blé a haussé en Pologne ; la valeur réelle des métaux précieux y a baissé, comme dans tous les autres endroits de l’Europe. La quantité de ces métaux a donc dû y augmenter comme ailleurs et à peu près dans la même proportion, relativement au produit annuel de ses terres et de son travail. Néanmoins, cette augmentation dans la quantité de ces métaux n’a pas, à ce qu’il semble, augmenté ce produit annuel, ni étendu l’agriculture et les manufactures du pays, ni amélioré le sort de ses habitants. L’Espagne et le Portugal, qui possèdent possèdent les mines, sont peut-être, après la Pologne, les deux plus pauvres pays de l’Europe ; cependant il s’en faut bien que la valeur des métaux précieux soit moins élevée en Espagne et en Portugal que dans tout autre endroit de l’Europe, puisque de ces deux pays ils viennent se rendre dans tous les autres, avec la charge non-seulement du fret et de l’assurance, mais encore avec la dépense de la contrebande, leur exportation étant ou prohibée ou soumise à des droits. Leur quantité, comparée au produit annuel des terres et du travail, doit donc nécessairement être plus grande dans ces deux pays qu’en aucun autre endroit de l’Europe ; cependant ces pays sont plus pauvres que la plupart des autres pays de l’Europe. C’est que si le système féodal a été aboli en Espagne et en Portugal, il y a été remplacé par un système qui ne vaut guère mieux.

De même donc que la faible valeur de l’or et de l’argent n’est pas une preuve de la richesse ni de l’état florissant du pays où elle a lieu, de même on ne peut, de la haute valeur de ces métaux, dans un pays ou bien du bas prix en argent, soit des mar­chandises en général, soit du blé en particulier, inférer en aucune manière que ce pays soit pauvre ou qu’il soit dans un état de barbarie.

Mais si le bas prix en argent, soit des marchandises en général, soit du blé en par­ti­culier, ne prouve nullement la pauvreté et la barbarie d’une époque, d’un autre côté, le bas prix en argent de quelques espèces particulières de marchandises, telles que le bétail, la volaille, les différentes sortes de gibier, relativement à celui du blé, en est une des preuves les plus décisives. Il démontre clairement, d’abord la grande abon­dance de ces sortes de denrées relativement au blé et, par conséquent, la grande étendue de terre qu’elles occupent relativement à celle qui est occupée par le blé ; il démontre, en second lieu, le peu de valeur de ces terres relativement à la valeur des terres à blé et, par conséquent, l’état négligé et inculte de la très-grande partie des terres du pays. Il démontre clairement que la population du pays et son capital ne sont pas, relativement à son territoire, dans la proportion où ils sont ordinairement dans les pays civilisés et que, dans un tel pays ou dans un tel temps, la société n’en est encore qu’à son enfance. Du haut ou du bas prix en argent des marchandises en général ou du blé en particulier, nous ne pouvons inférer autre chose, sinon que les mines d’or et d’argent qui, à cette époque, approvisionnaient le monde de ces métaux étaient fécon­des ou étaient stériles, mais non pas que le pays fût riche ou pauvre. Mais du haut ou bas prix en argent de certaines espèces de denrées relativement au prix de certaines autres, nous pouvons inférer, avec un degré de probabilité qui approche presque de la certitude, que le pays était riche ou pauvre, que la majeure partie de ses terres étaient améliorées ou abandonnées, et qu’il était alors ou dans un état plus ou moins barbare, ou dans un état plus ou moins civilisé.

Toute hausse dans le prix pécuniaire des denrées qui proviendrait entièrement d’une dégradation dans la valeur de l’argent, tomberait également sur toutes les espèces de denrées et marchandises et ferait hausser universellement leur prix d’un tiers, d’un quart ou d’un cinquième, selon que L’argent viendrait à perdre un tiers, un quart ou un cinquième de sa première valeur. Mais cette hausse du prix des denrées, sur laquelle on a tant raisonné, ne tombe pas également sur tous les genres de denrées. En prenant le cours de ce siècle pour faire un taux moyen, il est bien reconnu, même par ceux qui rapportent cette hausse à une dégradation dans la valeur de l’argent, que le prix du blé a beaucoup moins haussé que celui de quelques autres sortes de den­rées. On ne peut donc pas entièrement attribuer à la dégradation de la valeur de L’argent la hausse du prix de ces autres sortes de denrées. Il faut bien y faire entrer en compte quelque autre cause, et peut-être que celles que j’ai indiquées ci-dessus pourraient bien, sans recourir à cette prétendue dégradation de la valeur de l’argent, fournir une raison suffisante de la hausse de ces espèces particulières de denrée, dont le prix se trouve actuellement avoir haussé relativement à celui du blé.

Quant au prix du blé lui-même, il a été, pendant les soixante-quatre premières années de ce siècle et avant cette dernière suite extraordinaire de mauvaises années, un peu plus bas qu’il ne l’avait été pendant les soixante-quatre dernières années du siècle précédent. Ce fait est attesté non-seulement par les comptes du marché de Windsor, mais encore par les registres publics de tous les différents comtés d’Écosse, et par les relevés des prix de plusieurs différents marchés de France, qui ont été recueillis avec beaucoup de soin et d’exactitude par M. Messance[86] et par M. Dupré de Saint-Maur[87]. La preuve est plus complète qu’on ne devrait s’y attendre dans une ma­tière qui est naturellement si difficile à connaître avec quelque degré de certitude.

À l’égard du haut prix du blé dans ces dix à douze dernières années, on peut suffisamment l’expliquer par la suite de mauvaises récoltes qui a eu lieu, sans avoir recours à aucune dégradation de valeur de l’argent.

Ainsi aucune bonne observation, soit sur le prix du blé, soit sur le prix des autres denrées, ne paraît pouvoir fonder l’opinion, qui a été souvent avancée, que l’argent continuait à baisser de valeur.

Mais, me dira-t-on peut-être, d’après votre propre compte, la même quantité d’argent, dans le temps présent, achètera une moindre quantité de plusieurs espèces de denrées qu’elle n’en aurait acheté pendant une partie du siècle dernier ; et chercher à démontrer si ce changement est dû à une hausse dans la valeur de ces denrées, ou à une baisse dans la valeur de l’argent, c’est établir une distinction oiseuse et qui ne peut être d’aucune utilité pour celui qui n’a qu’une certaine quantité d’argent à porter au marché, ou qui a son revenu fixe en argent. je ne prétends assurément pas que la connaissance de cette distinction puisse mettre cette personne en état d’acheter à meilleur marché ; elle n’est cependant pas pour cela tout à fait inutile.

Elle peut être de quelque avantage pour le public, en fournissant une preuve facile de l’état de prospérité du pays. Si la hausse dans le prix de quelques espèces de denrées est due uniquement à une baisse dans la valeur de l’argent, elle est due alors à une circonstance de laquelle on ne peut inférer autre chose que la fécondité des mines de l’Amérique. Malgré cette circonstance, la richesse réelle du pays, le produit annuel de ses terres et de son travail peuvent aller, soit en déclinant successivement comme en Portugal et en Pologne, soit en avançant par degrés, comme dans la plupart des autres contrées de l’Europe. Mais si cette hausse dans le prix de quelques espèces de denrées est due à une hausse dans la valeur réelle de la terre qui les produit, à un accroissement dans la fertilité, ou à ce qu’en conséquence des progrès d’amélioration et de bonne culture elle a été rendue propre à la production du blé, alors cette hausse est due à une circonstance qui indique de la manière la plus évidente l’état de pros­périté et d’avancement du pays. La terre constitue la plus grande, la plus importante et la plus durable partie, sans comparaison, de la richesse de tout pays étendu. Il peut sûrement être de quelque utilité pour la nation, ou au moins il peut résulter quelque satisfaction pour elle, d’avoir une preuve aussi décisive que la partie, de beaucoup la plus grande, la plus importante et la plus durable de la richesse nationale, va en augmentant de valeur.

Cette distinction peut aussi être de quelque utilité à l’État, lorsqu’il s’agit de régler la récompense pécuniaire de quelques-uns des fonctionnaires qui le servent. Si cette hausse dans le prix de quelques espèces de denrées est due à une baisse dans la valeur de l’argent, il faut certainement augmenter en proportion de cette baisse leur récom­pense pécuniaire, à moins qu’elle ne fût trop forte auparavant. Si on ne l’augmente point leur récompense réelle en sera évidemment diminuée d’autant. Mais si cette hausse de prix est due à une hausse de valeur occasionnée par l’accroissement de fertilité de la terre qui produit ces sortes de denrées, c’est alors une affaire beaucoup plus délicate de juger dans quelle proportion il faut augmenter cette récompense pécuniaire, ou bien de juger si c’est même le cas de l’augmenter en rien. Si l’extension de l’amélioration et de la culture élève nécessairement le prix de chaque espèce de nourriture animale relativement au prix du blé, d’un autre côté elle fait aussi néces­sairement baisser celui de toute espèce, je crois, de nourriture végétale. Elle élève le prix de la nourriture animale, parce qu’une grande partie de la terre qui produit cette nour­riture, étant rendue plus propre à la production du blé, doit rapporter au pro­prié­taire et au fermier la rente et le profit d’une terre à blé. Elle fait baisser le prix de la nourriture végétale, parce qu’en ajoutant à la fertilité de la terre elle accroît l’abon­dance de cette sorte de nourriture. Les améliorations dans la culture introduisent aussi plusieurs espèces de nourriture végétale, qui, exigeant moins de terre que le blé et pas plus de travail, viennent au marché à beaucoup meilleur compte que le blé. Telles sont les pommes de terre et le maïs, ou ce qu’on appelle blé d’Inde, les deux plus im­por­tantes acquisitions que l’agriculture de l’Europe, et peut-être l’Europe elle-même, aient faites par la grande extension du commerce et de la navigation de celle-ci. D’ailleurs, il y a beaucoup d’espèces d’aliments du genre végétal qui, dans l’état grossier de l’agriculture, sont confinés dans le jardin potager et ne croissent qu’à l’aide de la bêche, mais qui, lorsque l’agriculture s’est perfectionnée, peuvent se semer en plein champ et croître à l’aide de la charrue : tels sont les turneps, les carottes, les choux, etc. Si donc, dans les progrès que fait l’amélioration, le prix réel d’une espèce de nourriture vient nécessairement à hausser, celui de l’autre espèce vient à baisser tout aussi nécessairement, et c’est alors une matière qui devient très-épineuse que de savoir jusqu’à quel point la hausse de l’une peut se trouver compensée par la baisse de l’autre. Quand une fois le prix réel de la viande de boucherie a atteint son maximum (ce qu’il paraît avoir déjà fait depuis plus d’un siècle dans une grande partie de l’Angleterre pour toutes sortes de viande de boucherie, excepté peut-être pour la chair de porc), alors quelque hausse qui puisse arriver par la suite dans le prix de toute autre espèce que ce soit de nourriture animale, elle ne peut guère influer sur le sort des gens de la classe inférieure du peuple. Certainement un adoucissement dans le prix des pommes de terre contribuera infiniment plus à améliorer la condition du pauvre, en bien des endroits de l’Angleterre, que ne pourrait l’aggraver une hausse quelconque dans le prix de la volaille, du poisson ou du gibier de toute espèce.

Dans le moment de la cherté actuelle, le haut prix du blé est sans contredit un fardeau pour les pauvres. Mais dans les temps d’une abondance médiocre, quand le blé est à son prix moyen ordinaire, la hausse naturelle qui a lieu dans le prix de toute autre espèce de produit brut ne peut guère tomber sur eux. Ils souffrent bien plus peut-être de cette hausse artificielle qu’ont occasionnée les impôts dans le prix de quelques denrées manufacturées, tel que celui du sel, du savon, du cuir, des chan­delles, de la drèche, de la bière et de l’ale, etc[88].

\subsubsection*{Des effets et des progrès de la richesse nationale sur le prix réel des ouvrages de manufacture}

L’effet naturel de l’amélioration générale est cependant de faire baisser par degrés le prix réel de presque tous les ouvrages des manufactures. Le prix de la main-d’œuvre diminue peut-être dans toutes, sans exception. De meilleures machines, une plus grande dextérité et une division et distribution de travail mieux entendues, toutes choses qui sont les effets naturels de l’avancement du pays, sont cause que, pour exécuter une pièce quelconque, il ne faut qu’une bien moindre quantité de travail ; et quoique, par suite de l’état florissant[89] de la société, le prix réel du travail doive s’élever considérablement, cependant la grande diminution dans la quantité du travail que chaque chose exige fait bien plus, en général, que compenser quelque hausse que ce soit dans le prix de ce travail.

Il y a, à la vérité, certains genres de manufactures dans lesquels la hausse néces­saire du prix réel des matières premières fait plus que compenser tous les avantages que les progrès de l’industrie peuvent introduire dans l’exécution de l’ouvrage. Dans les ouvrages de charpenterie et de menuiserie, et dans l’espèce la plus grossière de meubles en bois, la hausse nécessaire du prix réel du bois, résultant de l’amélioration de la terre, fera plus que compenser tous les avantages qu’on pourra retirer de la perfection des outils, de la plus grande dextérité de l’ouvrier, et de la division et de la distribution les plus convenables du travail.

Mais pour tous les ouvrages où le prix réel des matières premières ne hausse point ou ne hausse pas extrêmement, celui de la chose manufacturée baisse d’une manière considérable.

Cette diminution de prix a été la plus remarquable, durant le cours de ce siècle et du précédent, dans les manufactures qui emploient les métaux grossiers. On aurait peut-être aujourd’hui pour 20 schellings un meilleur mouvement de montre que celui qu’on aurait payé 20 livres vers le milieu du siècle dernier. Quoique moindre que dans l’horlogerie, il y a eu aussi une grande réduction de prix, pendant la même période, dans les ouvrages de coutellerie et de serrurerie, dans tous les petits ustensiles faits de métaux grossiers et dans toutes ces marchandises connues communément sous le nom de quincaillerie de Birmingham et de Sheffield. Elle a été assez forte pour étonner les ouvriers de tous les autres endroits de l’Europe, qui, à l’égard de beaucoup d’articles, conviennent qu’ils ne pourraient pas faire d’aussi bon ouvrage pour le double ou même le triple du prix. Il n’y a peut-être pas de manufactures où l’on puisse porter aussi loin la division du travail et où les instruments qu’on emploie soient susceptibles d’être perfectionnés d’autant de manières différentes, que les manufactures dont les métaux grossiers sont la matière première. 

Il n’y a pas eu dans les fabriques de draps, pendant la même période, une réduc­tion de prix aussi forte. On m’a assuré au contraire que, depuis vingt-cinq ou trente ans, le prix du drap superfin avait un peu dépassé sa proportion avec les autres, eu égard à la qualité, ce qui provient, dit-on, d’une hausse considérable dans le prix de la matière première, qui consiste entièrement en laine d’Espagne. À la vérité, on dit que les draps du comté d’York, qui sont en entier de laine d’Angleterre, ont considérable­ment baissé de prix, eu égard à la qualité, pendant le cours de ce siècle. Mais la qualité est une chose si sujette à dispute, que je tiens assez peu de compte de tous les renseignements de ce genre. Dans les fabriques de draps, la division du travail est à peu près la même aujourd’hui qu’elle était il y a un siècle, et les instruments qu’on y emploie ne sont pas très-différents de ce qu’ils étaient alors. Cependant il peut y avoir eu, sur ces deux articles, de légères améliorations qui ont pu occasionner quelque réduction de prix.

La réduction paraîtra pourtant bien plus incontestable et bien plus sensible, si l’on compare le prix de ce genre de fabrique, dans le temps actuel, avec ce qu’il était dans une période beaucoup plus reculée, comme vers la fin du quinzième siècle, où vraisemblablement le travail était beaucoup moins subdivisé et où l’on employait des machines bien moins perfectionnées qu’à présent.

En 1487, la quatrième de Henri VII, il fut statué que « quiconque vendrait au détail l’aune large de la plus belle écarlate, ou d’autre drap cramoisi[90] de la plus fine fabrique, au-dessous de 16 schellings serait à l’amende de 40 schellings par chaque aune ainsi vendue ». Donc, à cette époque, 16 schellings, qui contenaient environ la même quantité d’argent que 24 schellings de notre monnaie actuelle, étaient regardés comme un prix assez raisonnable pour un aune du plus beau drap ; et comme ce statut est une loi somptuaire, il est vraisemblable qu’un pareil drap s’était vendu habituelle­ment un peu plus cher. Aujourd’hui une guinée peut être considérée comme le prix le plus haut d’un tel drap. Ainsi, quand on supposerait même la qualité égale (et vraisemblablement celle de nos draps actuels est très-supérieure), cependant, dans cette supposition même, il paraîtrait y avoir eu une réduction considérable dans le prix pécuniaire du plus beau drap depuis la fin du quinzième siècle. Mais son prix réel a été beaucoup plus réduit encore. Le prix moyen du quartier de froment était alors et a été longtemps encore après à 16 schellings 8 deniers ; 16 schellings étaient donc le prix de deux quarters trois boisseaux de froment. En évaluant aujourd’hui le quarter de froment à 28 schellings, le prix réel d’une aune de drap fin doit avoir été, dans ce temps-là, l’équivalent de 3 livres 6 schellings 6 deniers, au moins, de notre monnaie actuelle. Il fallait que la personne qui l’achetait se dessaisît du pouvoir de disposer de la quantité de travail et de subsistance qu’on pourrait acheter aujourd’hui avec cette somme.

La réduction qui a eu lieu dans le prix réel des fabriques de gros draps, quoiqu’elle ait été considérable, n’a pas été aussi forte que celle qui a eu lieu pour les draps fins.

En 1463, la troisième année d’Edouard IV, il fut statué « qu’aucun domestique de labour, aucun manœuvre, aucun ouvrier servant chez un artisan, hors des villes ou des bourgs, ne pourrait porter dans son habillement un drap de plus de 2 schellings l’aune large »[91]. Dans la troisième année d’Edouard IV, 2 schellings contenaient, à fort peu de chose près, la même quantité d’argent que 4 schellings de notre monnaie actuelle. Or, le drap du comté d’York, qui se vend aujourd’hui 4 schellings l’aune, est probablement fort supérieur à tout ce qu’on pouvait faire alors à l’usage de la classe des plus pauvres manœuvres. Ainsi, eu égard à la qualité, le prix même pécuniaire de leur vêtement peut être regardé comme tant soit peu meilleur marché qu’il n’était dans ces anciens temps. Quant au prix réel, il est assurément de beaucoup meilleur marché. Ce qu’on appelle le prix modéré et raisonnable d’un boisseau de froment était alors de 10 deniers. Ainsi, 2 schellings étaient le prix de deux boisseaux et demi à peu près, lesquels aujourd’hui, à 3 schellings 6 deniers le boisseau, vaudraient 8 schellings 9 deniers. Il fallait donc que, pour acheter une aune de ce drap, un pauvre manœuvre renonçât au pouvoir d’acheter autant de subsistances qu’on pourrait maintenant en acheter pour 8 schellings 9 deniers. Ce statut est aussi une loi somptuaire, qui a pour objet de prévenir, chez les pauvres, toute dépense immodérée. Il faut donc que, pour l’ordinaire, leur vêtement ait été plus coûteux.

Par ce statut, il est défendu à la même classe du peuple de porter des bas dont le prix irait au-delà de 14 deniers la paire, valant environ 28 deniers de notre monnaie actuelle. Or, ces 14 deniers étaient alors le prix d’un boisseau et demi à peu près de froment, qui aujourd’hui, à 3 schellings 6 deniers le boisseau, coûteraient 5 schellings 3 deniers. Nous regarderions maintenant ce prix comme excessivement élevé pour une paire de bas à l’usage d’un domestique de la plus basse et de la plus pauvre classe. Il fallait pourtant alors qu’il les payât en réalité l’équivalent de ce prix.

L’art de faire les bas à l’aiguille n’était probablement connu en aucun endroit de l’Europe au temps d’Édouard IV. Ces bas étaient faits de drap ordinaire, ce qui peut avoir été une des causes de leur cherté. On dit que la reine Élisabeth est la première personne qui ait porté en Angleterre des bas tricotés ; elle les avait reçus en présent de l’ambassadeur d’Espagne.

Les machines que l’on employait dans les manufactures de lainages, tant pour le gros que pour le fin, étaient bien plus imparfaites dans ces anciens temps qu’elles ne le sont maintenant. Elles ont depuis acquis trois degrés principaux de perfection, sans compter vraisemblablement beaucoup de légères améliorations dont le nombre ou l’importance serait difficile à constater. Les trois améliorations capitales sont : 1° le rouet à filer, substitué au fuseau, ce qui avec le même travail, met à même de faire deux fois autant d’ouvrage ; 2° l’usage de plusieurs machines fort ingénieuses qui facilitent et abrègent, dans une proportion encore plus grande, le dévidage des laines filées ou l’arrangement convenable de la trame et de la chaîne avant qu’elles soient placées sur le métier, opération qui a dû être extrêmement lente et pénible ; 3° l’usage du moulin à avant l’invention de ces machines ; foulon, pour donner du corps au drap, au lieu de fouler dans l’eau. Avant le commencement du seizième siècle, on ne connaissait en Angleterre ni, autant que je sache, en nul autre endroit de l’Europe, au nord des Alpes, aucune sorte de moulins à vent ni à eau ; ils avaient été introduits en Italie quelque temps auparavant.

Ces circonstances peuvent peut-être nous expliquer, à un certain point, pourquoi le prix réel de ces manufactures, tant de gros que de fin, était si haut dans ces anciens temps, en proportion de ce qu’il est aujourd’hui. Il en coûtait une bien plus grande quantité de travail pour mettre la marchandise au marché ; aussi, quand elle y était venue, il fallait bien qu’elle achetât ou qu’elle obtînt en échange le prix d’une plus grande quantité de travail.

Il est vraisemblable qu’à cette époque les fabriques grossières étaient établies, en Angleterre, sur le même pied où elles l’ont toujours été dans les pays où les arts et les manufactures sont dans l’enfance. C’était probablement une fabrique de ménage où, dans presque chaque famille particulière, tous les différents membres de la famille exécutaient au besoin chacune des différentes parties de l’ouvrage, mais de manière qu’ils n’y travaillaient que quand ils n’avaient pas autre chose à faire, et que ce n’était pour aucun d’eux la principale occupation dont il tirât la plus grande partie de sa subsistance. Il a déjà été observé[92] que l’ouvrage qui se fait de cette manière est tou­jours mis en vente à meilleur marché que celui qui fait le fonds unique ou principal de la subsistance de l’ouvrier. D’un autre côté, les fines fabriques n’étaient pas alors établies en Angleterre, mais dans le pays riche et commerçant de la Flandre, et vraisemblablement elles étaient alors servies, comme elles le sont aujourd’hui, par des gens qui en tiraient toute ou la principale partie de leur subsistance. C’étaient, d’ailleurs, des fabriques étrangères et assujetties à quelques droits envers le roi, au moins à l’ancien droit de tonnage et de pondage[93]. Ce droit, il est vrai, n’était proba­blement pas très-fort. La politique de l’Europe ne consistait pas alors à gêner, par de forts droits, l’importation des marchandises étrangères, mais plutôt à l’encourager, afin que les marchands se trouvassent plus en état de fournir aux grands, au meilleur compte possible, les objets de luxe et de commodité qu’ils désiraient et que l’industrie de leur propre pays ne pouvait leur fournir.

Ces circonstances peuvent nous expliquer peut-être, jusqu’à un certain point, pourquoi, dans ces anciens temps, le prix réel des fabriques grossières était, relative­ment à celui des fabriques fines, beaucoup plus bas qu’il ne l’est aujourd’hui[94]. 

\section*{Conclusion du chapitre}

Je terminerai ce long chapitre en remarquant que toute amélioration qui se fait dans l’état de la société tend, d’une manière directe ou indirecte, à faire hausser la rente réelle de la terre, à augmenter la richesse réelle du propriétaire, c’est-à-dire, son pouvoir d’acheter le travail d’autrui ou le produit du travail d’autrui[95].

L’extension de l’amélioration des terres et de la culture y tend d’une manière directe. La part du propriétaire dans le produit augmente nécessairement à mesure que le produit augmente.

La hausse qui survient dans le prix réel de ces sortes de produits bruts, dont le renchérissement est d’abord l’effet de l’amélioration et de la culture et devient ensuite la cause de leurs progrès ultérieurs, la hausse, par exemple, du prix du bétail tend aussi à élever, d’une manière directe, la rente du propriétaire et dans une proportion encore plus forte. Non-seulement la valeur réelle de la part du propriétaire, le pouvoir réel que cette part lui donne sur le travail d’autrui, augmentent avec la valeur réelle du produit, mais encore la proportion de cette part, relativement au produit total, aug­mente aussi avec cette valeur. Ce produit, après avoir haussé dans son prix réel, n’exige pas plus de travail, pour être recueilli, qu’il n’en exigeait auparavant. Par conséquent, il faudra une moindre portion qu’auparavant de ce produit pour suffire à remplacer le capital qui fait mouvoir ce travail, y compris les profits ordinaires de ce capital. La portion restante du produit, qui est la part du propriétaire, sera donc plus grande, relativement au tout, qu’elle ne l’était auparavant. 

Tous les progrès, dans la puissance productive du travail, qui tendent directement à réduire le prix réel des ouvrages de manufacture, tendent indirectement à élever la rente réelle de la terre. C’est contre des produits manufacturés que le propriétaire échange cette partie de son produit brut qui excède sa consommation personnelle, ou, ce qui revient au même, le prix de cette partie. Tout ce qui réduit le prix réel de ce premier genre de produit élève le prix réel du second ; une même quantité de ce pro­duit brut répond dès lors à une plus grande quantité de ce produit manufacturé, et le propriétaire se trouve à portée d’acheter une plus grande quantité des choses de com­mo­dité, d’ornement ou de luxe qu’il désire se procurer[96].

Toute augmentation dans la richesse réelle de la société, toute augmentation dans la masse de travail utile qui y est mis en couvre, tend indirectement à élever la rente réelle de la terre. Une certaine portion de ce surcroît de travail va naturellement à la terre. Il y a un plus grand nombre d’hommes et de bestiaux employés à sa culture ; le produit croît à mesure que s’augmente ainsi le capital destiné à le faire naître, et la rente grossit avec le produit.

Les circonstances opposées, c’est-à-dire le défaut d’amélioration, la culture négli­gée, la baisse du prix réel de quelque partie du produit brut de la terre, la hausse du prix réel des manufactures, causée par le déclin de l’industrie et de l’art des fabricants, enfin, le décroissement de la richesse réelle de la société, toutes ces choses tendent, d’un autre côté, à faire baisser la rente réelle de la terre, à diminuer la richesse réelle du propriétaire, c’est-à-dire à lui retrancher de son pouvoir sur le travail d’autrui ou sur le produit de ce travail. La masse totale du produit annuel de la terre et du travail d’un pays, ou, ce qui revient au même, la somme totale du prix de ce produit annuel, se divise naturellement, comme on l’a déjà observé[97], en trois parties : la rente de la terre, les salaires du travail, les profits des capitaux, et elle constitue un revenu à trois différentes classes du peuple : à ceux qui vivent de rentes, à ceux qui vivent de salaires, a ceux qui vivent de profits. Ces trois grandes classes sont les classes primitives et constituantes de toute société civilisée, du revenu desquelles toute autre classe tire en dernier résultat le sien.

Ce que nous venons de dire plus haut fait voir que l’intérêt de la première de ces trois grandes classes est étroitement et inséparablement lié à l’intérêt général de la société. Tout ce qui porte profit ou dommage à l’un de ces intérêts, en porte aussi nécessairement à l’autre. Quand la nation délibère sur quelque règlement de commer­ce ou d’administration, les propriétaires des terres ne la pourront jamais égarer, même en n’écoutant que la voix de l’intérêt particulier de leur classe, au moins si on leur suppose les plus simples connaissances sur ce qui constitue cet intérêt[98]. À la vérité, il n’est que trop ordinaire qu’ils manquent même de ces simples connaissances. Des trois classes, c’est la seule à laquelle son revenu ne coûte ni travail ni souci, mais à laquelle il vient, pour ainsi dire, de lui-même, et sans qu’elle y apporte aucun dessein ni plan quelconque. Cette insouciance, qui est l’effet naturel d’une situation aussi tranquille et aussi commode, ne laisse que trop souvent les gens de cette classe, non-seulement dans l’ignorance des conséquences que peut avoir un règlement général, mais les rend même incapables de cette application d’esprit qui est nécessaire pour comprendre et pour prévoir ces conséquences[99].

L’intérêt de la seconde classe, celle qui vit de salaires, est tout aussi étroitement lié que celui de la première à l’intérêt général de la société. On a déjà fait voir[100] que les salaires de l’ouvrier n’étaient jamais si élevés que lorsque la demande d’ouvriers va toujours en croissant, et quand la quantité de travail mise en œuvre augmente con­si­dé­rablement d’année en année. Quand cette richesse réelle de la société est dans un état stationnaire, les salaires de l’ouvrier sont bientôt réduits au taux purement suf­fisant pour le mettre en état d’élever des enfants et de perpétuer sa race. Quand la société vient à déchoir, ils tombent même au-dessous de ce taux. La classe des propriétaires peut gagner peut-être plus que celle-ci à la prospérité de la société ; mais aucune ne souffre aussi cruellement de son déclin que la classe des ouvriers. Cepen­dant, quoique l’intérêt de l’ouvrier soit aussi étroitement lié avec celui de la société, il est incapable, ou de connaître l’intérêt général, ou d’en sentir la liaison avec le sien propre. Sa condition ne lui laisse pas le temps de prendre les informations nécessai­res ; et en supposant qu’il pût se les procurer complètement, son éducation et ses habitudes sont telles, qu’il n’en serait pas moins hors d’état de bien décider. Aussi, dans les délibérations publiques, ne lui demande-t-on guère son avis, bien moins encore y a-t-on égard, si ce n’est dans quelques circonstances particulières où ses clameurs sont excitées, dirigées et soutenues par les gens qui l’emploient, et pour servir en cela leurs vues particulières plutôt que les siennes.

Ceux qui emploient l’ouvrier constituent la troisième classe, celle des gens qui vivent de profits. C’est le capital qu’on emploie en vue d’en retirer du profit, qui met en mouvement la plus grande partie du travail utile d’une société. Les opérations les plus importantes du travail sont réglées et dirigées d’après les plans et les spéculations de ceux qui emploient les capitaux ; et le but qu’ils se proposent dans tous ces plans et ces spéculations, c’est le profit. Or, le taux des profits ne hausse point, comme la rente et les salaires, avec la prospérité de la société, et ne tombe pas, comme eux, avec sa décadence. Au contraire, ce taux est naturellement bas dans les pays riches, et élevé dans les pays pauvres ; jamais il n’est aussi élevé que dans ceux qui se précipitent le plus rapidement vers leur ruine[101]. L’intérêt de cette troisième classe n’a donc pas la même liaison que celui des deux autres avec l’intérêt général de la société[102]. Les marchands et les maîtres manufacturiers sont, dans cette classe, les deux sortes de gens qui emploient communément les plus gros capitaux et qui, par leurs richesses, s’y attirent le plus de considération. Comme dans tout le cours de leur vie ils sont occupés de projets et de spéculations, ils ont, en général, plus de subtilité dans l’enten­dement que la majeure partie des propriétaires de la campagne. Cependant, comme leur intelligence s’exerce ordinairement plutôt sur ce qui concerne l’intérêt de la branche particulière d’affaires dont ils se mêlent, que sur ce qui touche le bien général de la société, leur avis, en le supposant donné de la meilleure foi du monde (ce qui n’est pas toujours arrivé), sera beaucoup plus sujet à l’influence du premier de ces deux intérêts, qu’à celle de l’autre. Leur supériorité sur le propriétaire de la campagne ne consiste pas tant dans une plus parfaite connaissance de l’intérêt général, que dans une connaissance de leurs propres intérêts, plus exacte que celle que celui-ci a des siens. C’est avec cette connaissance supérieure de leurs propres intérêts qu’ils ont souvent surpris sa générosité, et qu’ils l’ont induit à abandonner à la fois la défense de son propre intérêt et celle de l’intérêt public, en persuadant à sa trop crédule honnêteté que c’était leur intérêt, et non le sien, qui était le bien général.

Cependant, l’intérêt particulier de ceux qui exercent une branche particulière de commerce ou de manufacture est toujours, à quelques égards, différent et même contraire à celui du public. L’intérêt du marchand est toujours d’agrandir le marché et de restreindre la concurrence des vendeurs. Il peut souvent convenir assez au bien général d’agrandir le marché, mais de restreindre la concurrence des vendeurs lui est toujours contraire, et ne peut servir à rien, sinon à mettre les marchands à même de hausser leur profit au-dessus de ce qu’il serait naturellement, et de lever, pour leur propre compte, un tribut injuste sur leurs concitoyens. Toute proposition d’une loi nouvelle ou d’un règlement de commerce, qui vient de la part de cette classe de gens, doit toujours être reçue avec la plus grande défiance, et ne jamais être adoptée qu’après un long et sérieux examen, auquel il faut apporter, je ne dis pas seulement la plus scrupuleuse, mais la plus soupçonneuse attention. Cette proposition vient d’une classe de gens dont l’intérêt ne saurait jamais être exactement le même que l’intérêt de la société, qui ont, en général, intérêt à tromper le public et même à le surcharger et qui, en conséquence, ont déjà fait l’un et l’autre en beaucoup d’occasions[103].

TABLE DES PRIX DU BLÉ
ANNONCÉE DANS LA DIGRESSION SUR LES VARIATIONS DANS LA VALEUR DE L’ARGENT, CHAP. XI, PREMIÈRE PÉRIODE.


 

TABLE
Des prix du quarter de neuf boisseaux du plus beau froment ou du plus haut vendu au marché de Windsor, les jours de marché de la Notre-Dame (25 mars) et de la Saint-Michel (29 septembre), depuis l’année 1595 inclusivement, jusqu’à l’année 1764 inclusivement, les prix de chaque année étant le medium entre les prix les plus hauts de ces deux jours de marché.

 

OBSERVATIONS DE GARNIER SUR LES TABLES PRÉCÉDENTES.

L’auteur, en publiant ces tables, a prévenu le lecteur du peu de confiance qu’elles méritent. Il annonce qu’elles ont été, pour la plus grande partie, relevées sur le Chronicon pretiosum de Fleetwood, et il ne dissimule pas que cet écrivain est tombé dans une erreur qu’il n’a pu s’empêcher de reconnaître lui-même. Cette erreur consiste à avoir confondu avec le prix courant du blé, le prix de conversion stipulé, d’après la coutume du lieu, entre le propriétaire et son fermier ; et cette méprise, qui est répétée jusqu’à quinze fois depuis l’année 1423 jusqu’à 1562, n’a point été réformée dans les tables. D’un autre côté, l’évêque Fleetwood a jugé à propos de rapporter certains prix d’une élévation tellement hors de toute mesure, qu’elle est presque incroyable, et qu’il a recueillis dans les chroniques du temps, où ils ont été cités comme un fait extraordinaire. Tel est le prix de 6 livres 8 schellings en 1270, époque à laquelle la livre d’Angleterre était encore la livre de Charlemagne, du poids de 12 de nos onces : en sorte que ce prix est égal, en poids d’argent, à 19 livres 4 schellings sterling, ou à 180 francs de notre monnaie ; et comme l’argent, avant l’introduction de celui du Nouveau- Monde, avait six fois plus de valeur réelle qu’il n’en a actuellement, ce n’est qu’en multipliant par six cette somme d’argent qu’on peut se faire une juste idée de ce prix du quarter. Un tel prix porté en ligne de compte dans une série, fût-il réparti sur cent années, ferait plus que doubler le prix moyen résultant du calcul.
Les prix des temps modernes ne présentent pas non plus l’exactitude qu’on attend d’une recherche de cette nature. L’auteur nous y donne les prix, du blé vendu le plus cher au marché de Windsor, en mars et en septembre de chaque année, en faisant seulement une moyenne de ces deux prix les plus hauts. Mais ce prix le plus haut est une fausse indication du prix moyen de chaque marché, surtout à l’époque des semailles, où il y a toujours une petite quantité de froment de choix que les fermiers achètent pour semer, et-qu’ils payent beaucoup plus cher que le blé destiné à la consommation, qui est celui dont il importe de connaître le prix. M. Dupré de Saint-Maur, W dans ses Recherches, distingue le blé acheté pour la semence et celui acheté pour la provision de la maison. Nous avons vu au marché d’Étampes, l’un des plus considérables des environs de Paris, en juin 1820, le blé provenant de la récolte de 1818, une des années les plus favorables qu’on ait eues depuis longtemps pour la qualité des grains, vendu jusqu’à 45 francs l’hectolitre et demi[104], tandis que, le même jour, le blé recueilli en 1819 n’a été payé que 23 à 24 francs. La différence de poids entre la première et la seconde espèce de blé, à mesure égale, était de 240 livres à 200. Aussi doit-on remarquer que Smith, dans le cours de son ouvrage, à chaque fois qu’il vient à citer le prix du blé, au moment où il écrit (1773), ne se conforme point aux tables qu’il a publiées. Selon ces tables, le prix moyen du blé, pendant les soixante-quatre premières années du dix-huitième siècle, est indiqué il 2 livres 7 deniers sterling, ce qui est à 4 schellings 6 deniers le boisseau, où à 1 livre 16 schellings pour le quarter de huit boisseaux. Toutefois, Smith évalue dans tout son ouvrage le prix moyen actuel du quarter à 1 livre 8 schellings, et celui du boisseau à 3 schellings 6 deniers. Ce prix de 1 livre 8 schellings (35 de nos francs) s’accorde parfaitement avec l’indication que présentent les tables suivantes. Si l’on y relève les prix du setier de Paris de 1701 à 1764 pour en former un prix moyen, on trouve 19 francs 27 centimes. Le quarter anglais étant à notre setier comme 46 est à 25, ce quarter aurait coûté en France, pendant cette période, 35 francs 46 centimes,
Une table du prix des grains est un recueil de faits qui ont entre eux tous, une parfaite connexité, et s’il a le mérite d’être fidèle, c’est un des documents les plus importants que puissent consulter ceux qui s’occupent d’étudier l’économie politique. Il doit porter sur une suite d’années un peu longue, afin de compenser les chances des bonnes et mauvaises récoltes, et d’ol1’rir un terme moyen. Comme chaque prix agit sur les autres, on manquerait le but qu’on se propose si on n’écartait pas de ce tableau les hauts prix qui ont eu pour cause quelque événement particulier, totalement étranger à l’inconstance des saisons.
On a publié, en France, plusieurs tables du prix des grains ; la plus récente est celle qui a été donnée par l’estimable auteur de la Balance du Commerce (feu M. Arnould), et qu’il annonce avoir copiée sur celle de Messance. Mais les tables de Messance, qui ne comprennent que la période. écoulée entre 1674 et 1765, sont un relevé des prix du meilleur froment vendu au marché de Paris, et les prix de ce marché n’ont pas toujours été le résultat naturel de l’influence des saisons et des libres efforts du commerce. On sait que, dans les années de grande cherté, le gouvernement a employé des moyens extraordinaires pour maintenir le prix des grains dans le marché de la capitale à un taux fort inférieur à celui qui aurait eu lieu sans l’intervention du pouvoir. Ainsi, pour en citer un exemple, dans la grande cherté de 1694, le prix du setier de Paris se trouve indiqué, dans les tables de Messance, à 52 livres 2 sous 6 deniers, environ un marc d’argent ; le prix du bichet de Lyon, pour la même année, est porté sur le même pied ; et comme cet auteur a dressé ses tables sur les prix du froment de première qualité, qui est communément de 20 pour cent plus cher que le blé destiné à la consommation générale, le prix courant de 1694 aurait été, si l’on s’en rapporte aux tables de Messance, de 40 francs environ de notre monnaie actuelle pour le setier de Paris, et 8 francs pour le bichet de Lyon, · cinquième du setier de Paris. Mais la réduction du prix de marché dans l’une et l’autre de ces grandes villes a été l’effet de dépenses extraordinaires faites par leurs magistrats municipaux pour adoucir le poids de la disette et le rendre plus supportable à leurs administrés. Il est prouvé par tous les monuments de cette époque, que le prix du blé a monté fort au-dessus des sommes indiquées dans les tables de Messance. Lamarre, dans son Traité de la police (livre V, titre IV, chap. xvi et xvii), expose les moyens violents qui furent mis en œuvre pour faire arriver des blés à Paris, et notamment l’envoi de commissaires chargés de faire des recherches chez les laboureurs et les marchands, et d’informer contre tous ceux qui se trouveraient avoir des provisions au delà des besoins de leur consommation. « Cette mesure (qui eut lieu au mois de juillet 1694) fit baisser, dit-il, le prix du blé au marché de Paris, a de 54 livres qu’il avait été jusque alors, à 36 livres, puis à 32 livres. » Il rapporte aussi qu’en 1698, où la même mesure fut renouvelée, les commissaires trouvèrent chez les laboureurs du blé de 1693 « que ceux-ci avaient laissé gâter, plutôt que de le vendre à 50 livres, prix alors courant dans la province, dans l’espérance que la denrée s’élèverait encore au-dessus de ce prix exorbitant. » Or, il ne faut pas perdre de vue que, dans le temps où Lamarre écrivait ceci, le marc d’argent étant à 50 livres, une somme nominale de 50 livres contenait un marc et deux tiers, et, par conséquent, répondait à 90 francs de notre monnaie actuelle. Ainsi, la table suivante, dans laquelle le prix moyen du blé, pour l’année 1694, a été formé sur quatre prix de la même année recueillis par Dupré de Saint-Maur, et indiqué à 60 francs 99 centimes, est encore plutôt au-dessous qu’au-dessus du véritable prix de l’année, quoique de plus de moitié supérieur à celui de la table de Messance.
Une table fidèle du prix des blés en France était une pièce nécessaire dans un ouvrage tel que celui-ci. Aucuns soins n’ont été négligés pour que celle qui suit fût aussi étendue et aussi exacte qu’il était possible de le désirer. Les années de cherté excessive en ont été retranchées, lorsque cet accident a été évidemment causé par des circonstances tout à fait indépendantes du cours naturel des valeurs : elle a été continuée jusqu’à l’année 1788 inclusivement. La fameuse cherté de 1789, qui ne fut pas uniquement produite par une rareté réelle de la denrée, les désordres que les assignats de 1790 ont jetés ensuite dans le rapport nominal des valeurs, et les variations brusques et multipliées que le prix des subsistances a subies pendant cette longue suite de troubles civils et de guerres extérieures qui ont désolé le royaume pendant tout le reste du dix-huitième siècle, sont des événements qui appartiennent à l’histoire, et non aux froides et paisibles méditations de l’économie politique.
Mai 1821.

TABLEAU DU PRIX DU SETIER DE BLÉ,
MESURE DE PARIS,
PENDANT LES XIIIe, XIVe, XVe, XVIe, XVIIe et XVIIIe SIÈCLES,
Divisé par séries de dix en dix années.

 
 
 
 
↑ Mac Culloch fait observer ici que ce chapitre d’Adam Smith est défectueux. Selon le commentateur, l’auteur de la Richesse des nations n’aurait pas connu la nature, l’origine et les causes du fermage. Il conteste la proposition d’Adam Smith, d’après laquelle certaines espèces de produits produiraient toujours une rente. « S’il en était ainsi, dit Mac Culloch, le fermage existerait toujours, tandis qu’il est inconnu dans les époques primitives des sociétés. La vérité est que le fermage est exclusivement la conséquence de la diminution des pouvoirs productifs des terres successivement mises en culture à mesure que la société se développe, ou plutôt de la diminution du pouvoir productif des capitaux successivement appliqués à la culture de ces terres. On n’a jamais entendu parler de fermage dans les contrées nouvellement peuplées, comme la Nouvelle-Hollande, l'Illinois, l’Indiana, et tous les autres pays où l’on ne cultive que les meilleures « terres. Le fermage n’apparaît qu’au moment où la culture s’est étendue aux « terres de qualité inférieure, etc. » *
↑ La rente de la terre, proprement dite, est la somme que l’on paye pour user de la puissance productive naturelle inhérente au sol, et elle est entièrement distincte de la somme payée pour l’usage des constructions, chemins, clôtures et autres améliorations faites sur le sol. La dernière somme n’est que le profit ou l’intérêt du capital engagé sur le sol. En pratique, ces deux sommes sont confondues sous le terme général de fermage, comme elles l’ont été ici par le Dr Smith ; mais elles sont essentiellement distinctes, et elles doivent être considérées ainsi dans les recherches de cette nature. Mac Culloch.
↑ Le nom anglais est kelp. Cette plante est du genre des salicornia, de Linné.
↑ Voyez ci-dessus, chap. vi, p. 103.
↑ Chap. ix et x, sect. ii.
↑ Ce sont les routes sur lesquelles sont placées des barrières nommées turn-pikes, où se perçoivent des droits dont le produit est exclusivement destiné à l’entretien des routes.
↑ De Officiis, lib. ii, § 25.
↑ Le boisseau anglais pèse environ 57 livres un quart de notre poids de marc ; ainsi le quart approche du modius des Romaisn, qui pesait 24 livres romaines, répondant à 15 trois quarts du même poids de marc.
↑ Le farthing répond à 2 centimes et demi.
↑ Voyages d’un Philosophe.
↑ Le sucre brut ou moscouade, moscovado ou crude sugar, est celui qui n’a subi d’autre préparation que la clarification ordinaire des chaudières. Le sucre brun ou passé, strained sugar, a de plus été filtré à travers la chausse ; mais ni l’un ni l’autre n’a été terré.
↑ Les renseignements que contient le texte concernant les profits des planteurs de sucre étaient probablement très-exagérés à l’époque où le Dr Smith écrivait. Il y a longtemps d’ailleurs qu’ils ont cessé d’être vrais. Loin d’être profitable à l’excès, l’industrie des planteurs de sucre a été généralement le contraire depuis trente ans. La culture a été trop étendue, et la quantité de produit apportée au marché a été si grande, qu’elle a fréquemment réduit le prix à un taux qui ne dépasse guère la somme nécessaire pour couvrir les dépenses de culture et l’acquittement des droits. Mac Culloch.
↑ Douglas’s Summary, vol. II, pages 372, 373.
↑ L’état actuel de l’Irlande justifie tristement cette prévision du Dr Smith. A. B.
↑ Mac Culloch voit dans cette opinion une erreur. Il prétend que c’est la mine de charbon la moins fertile que l’on est forcé d’exploiter pour répondre à la demande, qui règle le prix du charbon des autres mines.
(Voyez éd. de M. C., p. 55.)
↑ Smith enseigne que le prix auquel le propriétaire de la mine de charbon la plus féconde vend sa marchandise, règle le prix de cette denrée pour toutes les autres mines du voisinage. Il démontre cette proposition en observant que le propriétaire et l’entrepreneur de cette mine féconde trouvent tous deux qu’ils pourront se procurer plus de rente et plus de profit en vendant à un prix un peu au-dessous de celui de leurs voisins, et qu’alors ceux-ci, quoique moins en état de supporter une diminution, sont forcés de vendre au même prix et de baisser toujours leur prix de plus eu plus, jusqu’à ce qu’ils soient descendus au point où l’exploitation serait absolument sans profit.
M. Ricardo a jugé à propos de s’attacher à la proposition contraire. Il soutient que le prix du charbon est réglé par la mine la plus pauvre du voisinage, par celle dont le produit ne fait que rendre l’équivalent du capital employé à son exploitation, avec le profit ordinaire de ce capital, et ne peut suffire à payer un loyer ou prix de ferme quelconque au propriétaire du sol.
Ceci est une pure dispute de mots, qui ne procède que de deux manières différentes d’exprimer le même principe, et elle disparaît dès qu’on veut définir la chose. Il est constant que la mine la plus pauvre, celle dont le produit est trop peu abonda ut ou d’une extraction trop dispendieuse pour que l’entrepreneur puise payer un prix de ferme, est celle qui pose les limites du prix du charbon, le taux au-dessous duquel il ne peut pas être vendu dans le canton. C’est ce que Smith reconnaît formellement lorsqu’il dit que le prix le plus bas auquel puisse se vendre le charbon de mine, pendant un certain temps, est comme celui de toutes les autres marchandises, c’est-à-dire, le prix qui ne suffit qu’à remplacer, avec le profit courant, le capital employé à faire aller l’entreprise. Il ajoute que tel doit être à peu près le prix du charbon dans une mine que le propriétaire est forcé d’exploiter lui-même, faute de trouver un entrepreneur qui consente a lui payer un loyer ou prix de ferme.
Mais pourquoi ce prix est-il si bas ? Pourquoi le propriétaire est-il forcé de vendre sa denrée à un prix qui ne lui donne point de rente ? C’est parce que l’a ainsi voulu et réglé le propriétaire voisin qui possède une mine plus féconde. C’est le propriétaire de la mine la plus féconde qui fait la loi aux autres propriétaires de mines du voisinage, et qui leur prescrit en quelque sorte le taux auquel ils peuvent vendre, dans l’état actuel ouest la demande de la denrée. En effet, si la demande est peur dix mille muids de charbon, et que la mine la plus riche puisse les fournir à un prix qui serait trop bas pour les autres mines, et qui ne rendrait pas le profit de leur exploitation, le propriétaire de cette mine plus riche profitera de son avantage naturel pour s’attribuer le monopole, et il baissera ses prix de vente jusqu’au point où il sera nécessaire pour tenir fermées toutes les autres mines du voisinage. Mais si ta demande est de trente mille muids, et que pour se les procurer il faille avoir recours jusqu’à la mine la moins riche de toutes celles du canton, comme celle-ci ne peut mettre ses produits au marché qu’autant que l’entrepreneur retrouvera, dans le prix de la denrée, son capital et son profit, il faudra bien que le prix du charbon de cette mine soit assez élevé pour y suffire. Sous ce rapport, la mine la moins riche fixe le ’’minimum’’ au-dessous duquel le prix du charbon ne peut descendre, et à défaut duquel la mine se ferme jusqu’à ce que les consommateurs consentent à donner ce prix. Les bénéfices des mines s’élèvent de toute la différence qui existe entre ce prix le plus bas et le prix naturel auquel revient leur charbon, attendu qu’il ne peut y avoir au même marché deux prix différents pour le meute denrée, à qualité égale.
À cet égard, les mines et carrières ont un point de ressemblance avec les manufactures, le prix de leur produit étant déterminé par la quantité des demandes de la consommation, et ne pouvant jamais descendre au-dessous des avances de l’entrepreneur, augmentées du profit courant et ordinaire. Mais la différence entre ces sortes de manufactures territoriales et les manufactures industrielles, c’est que les bénéfices de ces dernières tendent toujours à se mettre de niveau, sauf le cas où le manufacturier serait possesseur d’un secret qui lui donnerait un avantage sur ses concurrents ; au lieu que, dans les mines et carrières, une inégalité naturelle contre laquelle l’industrie humaine ne peut lutter, empêche que les bénéfices résultant de l’exploitation ne s’égalisent, et les tient constamment à des hauteurs différentes.
Mais une grave erreur de M. Ricardo, et celle qu’il nous reproduit sans cesse, c’est d’appliquer ce principe du ’’minimum’’ de prix résultant de la moindre fertilité des mines aux terres cultivées en blé, et de supposer qu’il règle de même le prix du blé en argent. Sans doute les besoins actuels de la consommation déterminent bien quel sera le degré de fertilité auquel une terre i blé pourra être cultivée avec avantage ; ils règlent le ’’minimum’’ du produit qu’il faut au propriétaire pour qu’il se décide à préférer la production des subsistances à toute autre production. Si les besoins de la population en subsistances sont satisfaits par des terrains plus productifs, le propriétaire d’une terre inférieure en qualité trouvera plus d’avantage i y faire croître des menus grains, des plantes oléagineuses, des bois, etc., ou même à y faire paître du bétail ; mais le prix moyen du blé en argent n’est nullement affecté par le plus ou le moins de fertilité des terres qui produisent cette denrée. M. Ricardo parait s’être fait une idée extrêmement fausse sur le prix du blé en argent ; il semble avoir écrit sous l’influence de ce préjugé dont l’illusion est entretenue et fortifiée par les habitudes communes de la vie, mais dont il faut se garder, quand on veut considérer d’un œil philosophique les phénomènes de la circulation des richesses. L’argent lui parait être le régulateur du prix du blé, parce que l’argent est la mesure des variations accidentelles et temporaires que le blé éprouve, comme marchandise. Mais il semble perdre de vue que la valeur de l’argent lui-même est, comme celle de toutes les autres valeurs, mesurée par le blé ; que l’argent est un produit du travail, et que ce travail a sa mesure dans la quantité de subsistances qui l’alimentent ; que les hommes ont adopté l’argent comme instrument des échanges et comme mesure de toutes les valeurs commerçables, pour la commodité et l’activité de la circulation ; mais que cet instrument ne peut remplir sa fonction qu’après qu’il a été lui-même ajusté sur l’étalon primitif et originaire des valeurs échangeables, c’est-à-dire la subsistance de l’ouvrier. Garnier.
↑ Au Mexique, antérieurement à la guerre de la révolution, les spéculateurs sur les mines étaient généralement des personnes riches et de distinction, capables de taire de larges avances de leurs propres fonds pour conduire leur industrie, et alors cette industrie était regardée comme aussi sûre, était aussi considérée que tout autre genre de commerce. Mais au Pérou, auquel s’appliquent les observations de Smith, les spéculateurs en raines appartenaient à une classe toute différente : c’étaient des gens nécessiteux dont le capital était emprunté à un intérêt exorbitant, et qui étaient par conséquent à la merci de leurs créanciers et des marchands d’argent. On ne pouvait attendre ni prudence, ni économie de personnes placées dans des circonstances aussi défavorables, et la grande majorité d’entre elles nous est représentée comme ayant été à la fois malhonnêtes, pauvres et prodigues. (Consulter des détails extraits du Mercurio Peruano, feuille périodique publiée à Lima, de 1791 a 1794, dans la Revue d’Édimbourg, t. IX, p. 444.)
Les associations formées en Angleterre en 1824 et 1825, pour l’exploitation des mises d’Amérique, n’ont eu aucun succès, et ont été en grande partie abandonnées.Mac Culloch.
↑ Voyez ci-après la iiie section de ce chapitre.
↑ Voyez le livre IV, et surtout le chap. i.
↑ La valeur des métaux précieux a été bien plus sérieusement altérée par une autre circonstance que le Dr Smith a négligé de signaler : je veux parler de la substitution du papier aux espèces comme intermédiaire des échanges.
Buchanan.
↑ La drèche ou le malt est de l’orge infusée dans l’eau, légèrement fermentée, et préparée par différentes manipulations, pour en composer diverses boissons, dont la bière est une des principales.
↑ L’ale est une sorte de boisson composée d’orge et de houblon, et qui diffère de la bière commune, principalement en ce qu’on y met beaucoup moins de houblon ; ce qui la rend bien moins amère que la bière, et moins propre à être gardée longtemps.
↑
Smith ayant une fois admis une prétendue augmentation de la valeur de l’argent, qui serait survenue à la fin du quinzième siècle, éprouve quelque embarras à en rechercher la cause.
Une grande difficulté qu’il ne songe point à résoudre, c’est de savoir comment, depuis les huitième et neuvième siècles, la valeur de l’argent aurait éprouvé une baisse considérable qui aurait duré pendant trois à quatre siècles, et pour quelle raison ce métal aurait valu au treizième siècle deux ou trois fois moins qu’il ne valait au temps de Charlemagne. Mais Smith ne s’est point fait cette objection, parce qu’il n’a pas remonté jusques à cette époque, qui cependant fournit sur cette matière les informations les plus sûres et les plus complètes qu’on puisse désirer. Il se borne donc à examiner ce qui a pu élever la valeur de l’argent à la fin du quinzième siècle.
La cause, dit-il, en peut être attribuée à une plus grande demande d’argent occasionnée par le progrès de la richesse, et à laquelle l’ancien approvisionnement n’avait pu suffire ; ou bien on peut l’attribuer à l’épuisement des mines d’argent de l’ancien Monde, qui seraient devenues hors d’état de fournir la même quantité de produits que dans les temps antérieurs ; peut-être enfin, ajoute-l-il, au concours de ces deux causes.
Sans doute, vers la fin du quinzième siècle, le retour à un état plus tranquille et une administration plus puissante et mieux ordonnée furent de grands encouragements pour le commerce et pour l’industrie, et il est très-vraisemblable que les peuples de l’Europe se trouvèrent dans une situation qui amena une plus abondante consommation de métaux précieux. Mais tout ce qui résulta de cet accroissement dans la consommation de cet article, c’est que le travail de l’exploitation s’étendit dans la même proportion que les demandes, sans que pour cela ce travail fût plus dispendieux, et sans que la valeur de l’argent en fût augmentée. Les mines qui alors fournissaient à l’approvisionnement du monde étaient si peu dans un état d’épuisement, qu’encore aujourd’hui quelques-unes d’entre elles suffisent à payer leurs frais, malgré l’énorme dépréciation de l’argent, et peuvent concourir avec celles du Nouveau-Monde à fournir les marchés de l’Europe. Sur un total de 150 millions et demi auxquels on évalue le produit des mines d’argent des deux Mondes, dans une année moyenne de 1803 à 1809, 13 millions et demi, plus du douzième de la provision annuelle, proviennent des mines de l’Europe et de l’Asie. On peut regarder les mines des métaux précieux comme un fonds inépuisable, capable de fournir a toutes les demandes de la consommation, à quelque point qu’elles s’étendent, et sans qu’on puisse assigner de bornes à ce produit, tant que les consommateurs consentiront à donner la quantité de subsistances nécessaires pour alimenter et entretenir le travail de l’exploitation. Depuis la découverte de l’Amérique, la fécondité de cette contrée en métaux précieux ne semble pas avoir éprouvé la plus légère diminution, quoiqu’on en ail retiré pour une valeur déplus de 33 milliards de francs. Si la presque totalité des anciennes mines d’Europe sont aujourd’hui fermées, ce n’est pas pour cause d’épuisement, c’est parce que la plus féconde de celles qu’on a été forcé d’abandonner était d’un produit inférieur à h moins fertile de celles qui sont actuellement en exploitation.
Garnier.
↑ La livre était alors taillée en 60 sous ; elle le fut depuis en 62. Ainsi, 6 schellings 8 deniers d’alors étaient juste la neuvième partie d’une livre, et contenaient 1/279 de livre d’argent de plus que 6 schellings 8 deniers d’aujourd’hui ; différence qui équivaut à environ 2 deniers 1/2 sterling.
↑ Essai sur les monnaies et sur le prix des denrées, dans les temps anciens. Paris, 1746, in-4°.
↑ M. Herbert. Son ouvrage a été imprimé en 1755.
↑ Édition donnée en 1762, en 11 volumes in-4°, et augmentée de 8 autres volumes eu 1796.
↑ Tumbrel ou Dungeart, cathedra stercoris, instrument de correction pour châtier les brasseurs qui contrevenaient au statut de l’assiette de l’ale, comme le pilori pour les boulangers qui vendaient au-dessus du prix de la taxe. Voyez Blackstone, liv. IV, chap. xii.
↑ Voyez sa préface au Recueil des Chartres d’Écosse, d’Anderson.
↑ Voyez, à la fin de ce chapitre, la note sur la table des prix du blé.
↑ Voyez le chap. v de ce livre.
↑ « Le prix du blé ne règle pas le prix en argent de tous les autres produits bruts de la terre ; il ne règle ni le prix des métaux, ni celui de beaucoup d’autres matières premières ; et comme il ne règle pas le prix du travail, il ne règle pas non plus celui des objets manufacturés**. » (Buchanan.)
↑ La valeur des choses est une qualité positive et absolue qui existe en elles, indépendamment de tout échange. Ce qui constitue le prix originaire d’une chose, c’est la quantité de travail fait ou la grandeur des difficultés vaincues pour l’obtenir. Voilà ce qui la rend plus ou moins chère pour celui qui la possède. C’est bien de là qu’elle tient la faculté de pouvoir s’échanger contre tant de blé, tant d’argent, tant de sucre, tant de laine, etc. Mais certainement il n’est pas nécessaire que cette faculté soit exercée, pour qu’on puisse la considérer en elle-même et s’en former une idée. Robinson, dans son île, sans relation avec aucune autre créature de son espèce avec laquelle il pût faire un échange, avait cependant, parmi les meubles de sa cabane, des objets qu’il considérait comme plus chers que les autres, parce qu’ils lui avaient coûté plus de travail ou plus de peine à acquérir. Qu’un voyageur, en parcourant la mer du Sud, ait occasion d’observer deux îles différentes, dont les habitants n’ont entre eux aucune sorte de rapports ; que dans l’une, les naturels du pays vivent uniquement des produits d’une chasse pénible, difficile et fort incertaine, tandis que les autres insulaires cultivent le bananier, le manioc ou quelque autre plante alimentaire qui croît facilement et presque sans travail dans leur sol : ce voyageur s’exprimerait sans doute d’une manière fort exacte, s’il disait que la nourriture est fort chère chez les uns et qu’elle est à très-bon marché chez les autres. Cette manière de mesurer la valeur des choses par la pensée et d’après le travail qu’elles coûtent à celui qui en fait usage, est extrêmement utile pour connaître, dans les transactions du commerce, d’où procède le renchérissement de certains articles, et pour juger dans un échange quel est celui des deux termes dont se compose cet échange qui gagne le plus et qui procure le plus d’avantages à celui qui le produit ou le possède.
C’est dans ce sens absolu qu’Adam Smith a considéré l’or et l’argent quand il a observé que ces métaux étaient nécessairement moins chers en Portugal et en Espagne que dans les contrées où on les transporte de l’un ou de l’autre de ces pays, puisque dans les contrées qui importent cet or ou cet argent, il faut ajouter à la valeur que les matières avaient déjà acquise en Espagne ou en Portugal, nation à mesure qu’elle s’enrichit, à moins que la découverte accidentelle de mines plus abondantes ne le fasse baisser, il s’ensuit que, quel que puisse être l’état des mines, ce prix sera naturellement plus élevé dans un pays riche que dans un pays pauvre. L’or et l’argent, comme toute autre marchandise, cherchent naturellement le marché où on en le surcroît de travail et de risques qui dorme lieu aux frais de transport et d’assurance.
Voilà ce que M. Ricardo a jugé à propos de contredire: « Quand on parle, dit-il, du plus ou moins de valeur de l’or et de l’argent, ou de toute autre marchandise, en différents pays, on devrait toujours choisir une mesure pour estimer cette valeur, si l’on veut être intelligible. Par exemple, quand ou dit que l’or est plus cher en Angleterre qu’en Espagne, si on ne l’évalue pas en le comparant avec d’autres marchandises, quel peut être le sens de cette assertion ? Si le blé, les olives, l’huile, le vin et la laine sont à meilleur marché en Espagne qu’en Angleterre, l’or estimé au moyen de ces denrées se trouvera être plus cher en Espagne***. »
L’argument de M. Ricardo consiste à ne pas prendre les mots dans le sens qu’Adam Smith a voulu leur donner. Le critique ne veut voir que le résultat des échanges, sans considérer séparément chacun des deux termes de l’échange. Je conviens que pour celui qui trafique et qui calcule le profit qu’il peut faire, le résultat de l’échange est le plus souvent le seul point important. Maïs il n’en est pas de même pour l’observateur qui s’occupe de l’intérêt général, et qui a souvent besoin de reconnaître quel rôle joue dans un échange chacune des marchandises qui y concourent. Ainsi, dans l’hypothèse que nous offre M. Ricardo, il peut très-bien arriver qu’une livre pesant d’argent achète, en Espagne, un quintal d’une espèce de laine, tandis que le même poids d’argent, en Angleterre, n’achètera que 95 livres de cette même sorte de laine. Que doit-on en conclure, si l’on tient à la doctrine enseignée par Adam Smith ? C’est que la laine, en Espagne, est à meilleur marché qu’en Angleterre, non pas seulement de 5 pour 100, mais encore de toute la différence du prix de l’argent dans ces deux pays. On pourra avec assurance partir de ce point, qu’une livre d’argent transportée en Angleterre, et y étant venue d’Espagne, chargée du fret et de l’assurance, y a plus de valeur réelle qu’elle n’en avait en Espagne ; et cependant cette livre d’argent ainsi renchérie, n’achetant que 95 livres d’une laine dont on a eu en Espagne 100 livres pour un pareil poids d’argent, la conséquence nécessaire est que la différence du prix de la laine eu Angleterre est sensiblement de plus de 5 pour 100 avec cette même laine en Espagne. Si, au contraire, on se borne à la théorie de M. Ricardo, on ne saura autre chose, sinon que, dans ce marché, la laine en Espagne gagne 5 pour 100 sur sou prix en Angleterre, mais on n’aura aucun moyen de connaître si la différence des deux prix procède du fait de l’argent ou du fait de la laine, ni pour quelle quantité chacune des deux marchandises concourt à opérer cette différence. Je ne puis pas accorder à M. Ricardo que la distinction présentée par Smith soit vide de sens, et dépourvue d’utilité. Garnier.
↑ Smith s’est formé cette idée des richesses de la Chine d’après le rapport des premiers voyageurs, et particulièrement des jésuites. Des rapports plus récents et plus authentiques nous montrent que la Chine, au lieu d’être un pays riche, est en réalité un pays pauvre et mal cultivé. La population y est excessivement surabondante, et la pauvreté et la misère y règnent à un degré inconnu en Europe, à l’exception de l’Irlande. Mac Culloch.
↑ Ce n’est plus aujourd’hui le cas depuis longtemps. Les exportations de blé d’Écosse en Angleterre dépassent généralement les importations. Mac Culloch.
↑ Chapitre v du livre IV.
↑ Voyez ci-dessus, chap v.
↑ Essai sur la monnaie d’argent, par M. Lowndes, page 68.
↑ Livre IV, chap. v.
↑ Ce sont ceux qu’on nomme country-gentlemen, ceux à qui leur propriété foncière donne le droit de concourir à l’élection des représentants des comtés dans la chambre des communes, et qui de plus n’exercent aucune profession mécanique. Toute personne qui vit noblement, c’est-à-dire sans travail manuel, est désignée en Angleterre par le titre de gentleman.
↑ La taxe foncière ou taxe des terres, land-tax, est un impôt assez semblable à nos anciens vingtièmes, et qui porte sur les revenus fonciers, les maisons et les capitaux de commerce ; ceux placés dans la culture des terres n’y sont pas assujettis. Voyez au reste le liv. V, chap. ii, partie 2e, art. 1er*.
↑ Vouez les traités sur le commerce des blés, 3e partie.
↑ Chap. viii de ce livre.
↑ Voyez ci-dessus, section 2e de ce chapitre.
↑ Solorzano, vol. II.
↑ Le Potose n’a plus cet avantage. Les mines de Guanaxuato, au Mexique, découvertes en 1700, ont été environ deux fois aussi abondantes que celles du Potose. Voyez Humbolt : Essai politique sur la nouvelle Espagne, tom. III, p. 377. Mac Culloch.
↑ Voyez liv. IV, chap. vii, section 2e.
↑ La consommation du thé a considérablement augmenté depuis 1775. La quantité de thé légalement importé pendant les trois ans se terminant en 1783, s’éleva environ à 5 millions et demi de livres poids : mais on estime qu’environ 7 millions et demi de livres poids, en sus de la première quantité, avaient clé introduites en fraude, ce qui élève le total des importations à environ 13 millions de livres poids. Cet excès de contrebande était l’objet des droits oppressifs mis sur le thé ; en 1784, après avoir essayé tous les moyens de réprimer la contrebande, M. Pitt réduisit les droits de 119 à 12 1/2 pour cent, ad valorem. Cette mesure eut le plus grand succès. La contrebande, ayant cessé d’être lucrative, fut immédiatement abandonnée, et la quantité du thé légalement importé s’accrut dans une proportion triple en moins de deux ans. La consommation du thé continua de s’accroitre avec une grande rapidité jusqu’en 1800 ; mais depuis cette époque jusqu’en 1819, elle resta stationnaire, ce qu’il faut attribuer, en grande partie, à l’augmentation des droits pendant cet intervalle ; la manière dont la Compagnie des Indes orientales approvisionnait le marché y entre aussi pour quelque chose. Depuis 1817, où la consommation de la Grande-Bretagne et de l’Irlande s’éleva environ à 24 millions de livres poids, cette consommation alla toujours en augmentant, jusqu’en 1833 où elle s’éleva à 31,829,620 livres poids. En 1834, les importations du commerce privé, ajoutées à l’approvisionnement accumulé par la Compagnie des Indes orientales, firent baisser de beaucoup les prix, et la consommation fit de si grands progrès, qu’en 1838 elle s’élevait à environ 40 millions de livres poids, produisant un revenu de 4 millions de livres sterling ! À la première application du nouveau système, les droits furent fixés à 1 schelling 6 deniers, 2 schellings 2 deniers, et 3 schellings la livre, suivant la qualité ; mais cette variation de droits n’ayant pas été trouvée favorable, fut abandonnée en 1855, où un droit uniforme de 2 schellings 1 denier fut imposé sur tous les thés, sans égard à la qualité. En supposant que le droit présent équivaut à une taxe de 100 pour % ad valorem, ce que l’on croit généralement, il en résulte que le thé annuellement consommé coûte, au peuple de la Grande-Bretagne et de l’Irlande, 8,000,000 livres 200 millions de francs), sans compter les profits des marchands en détail. Mac Culloch.
↑ Chap. viii.
↑ Depuis les dernières années on a cessé d’importer de la monnaie dans l’Est ; et des quantités considérables ont été au contraire importées de l’Inde, de la Chine, en Angleterre. Mac Culloch.
↑ M. de Humboldt pense que sur la somme de 49,500,000 dollars d’or et d’argent qu’il suppose avoir été apportés annuellement d’Amérique en Europe antérieurement aux révolutions d’Amérique, 20,500,000 dollars étaient ensuite importés en Asie ; savoir, 4,000,000 par le levant, 17,500,000 par la route du cap de Bonne-Espérance, et 4,000,000 à travers la Russie par la route de Kiachta et Tobolsk, etc. Il doit y avoir beaucoup de conjecture dans cette estimation ; mais la peine que M. de Humboldt s’est donnée pour l’obtenir fait qu’elle mérite une grande confiance. Voyez Essai politique sur la nouvelle Espagne, tom. IV, p. 278. Mac Culloch.
↑ Post-scriptum du Négociant universel, pages 15 et 16. Ce post-scriptum n’a été imprimé qu’en 1756, trois ans après la publication de l’ouvrage, qui n’a jamais eu de seconde édition. Ainsi il y a eu peu d’exemplaires où se trouve ce post-scriptum, qui contient la correction de quelques erreurs du texte.
(Note de l’auteur.)
↑ Liv. VIII, § 42 des premières éditions ; mais l’auteur a supprimé ce détail dans la seconde publication du même ouvrage ; qu’il a faite en 1780.
↑ Livre IX, § 54 des premières éditions. Dans celle publiée en 1780, cette évaluation se trouve réduite de près de moitié. Voyez cette édition, liv. IX, § 23.
↑ Les savantes recherches de M. de Humboldt ont donné d’importantes informations sur ce sujet. Elles montrent que cette importation d’or et d’argent d’Amérique était beaucoup plus grande à l’époque où la Richesse des nations fut publiée (1770), que ne le supposait le Dr Smith, et que cette importation continua à s’accroître jusqu’au commencement des troubles révolutionnaires. La table suivante contient les résultats des recherches de M. de Humboldt.
Cette somme de 43,500,000 dollars, à 4 schellings 3 deniers le dollar, s’élève à 9,243,730 livres ; M. de Humboldt estime le produit annuel des mines d’Europe et celui des mines de l’Asie Méridionale, environ à 4,000,000 en plus. M. Jacob, auteur des Recherches historiques sur la consommation des métaux précieux, évalue le produit moyen annuel des mines d’Amérique, de 1800 à 1810, à 47,061,000 dollars. Mais les mouvements révolutionnaires qui, commençant en 1810, ont troublé le Pérou, le Mexique et le reste de l’Amérique espagnole, firent bientôt abandonner entièrement quelques-unes des mines les plus productives, et occasionnèrent un déficit extraordinaire dans l’approvisionnement des métaux précieux qu’on tirait auparavant du Nouveau-Monde. M. Jacob évalue le produit moyen annuel des mines d’Amérique, de 1810 à 1829, à 4,036,000 livres ; ce qui n’est pas la moitié de leur produit au commencement du siècle ; et quoiqu’on ait soupçonné cette évaluation d’avoir été faite trop bas, il y a d’excellentes raisons de penser qu’elle n’est pas loin de la vérité (Jacob, II, p. 267). La déconfiture des Compagnies qui se formèrent en Angleterre en 1825 pour l’exploitation des mines américaines, l’instabilité des gouvernements révolutionnaires et l’insécurité continuelle qui a régné jusqu’ici dans toutes les parties du Mexique et des anciennes provinces espagnoles de l’Amérique du Sud, ont empêché que la quantité de métaux précieux fût considérablement augmentée. Dans ces derniers temps il s’est manifesté un accroissement considérable dans la richesse des mines de l’Oural et d’autres mines de Russie ; et depuis 1829 une assez grande quantité a été obtenue par le lavage dans la Caroline du Nord et dans d’autres parties des États-Unis. La quantité annuelle d’or et d’argent fournie par l’Amérique et l’Europe, en y comprenant la Russie Asiatique, peut être raisonnablement évaluée à 5,000,000 ou 5,500,000 livres. Mais pour se faire une idée juste sur ce sujet, il faut se rappeler que sur la somme de 10,243,000 livres qu’on suppose avoir été fournie par les mines d’Amérique, d’Europe et de Sibérie dans la première partie de ce siècle, 25,500,000 dollars au moins, ou 6,000,000 de livres furent exportés dans l’Inde, à la Chine et dans d’autres contrées de l’Orient ; il ne restait ainsi que 4,243,000 livres pour la consommation de l’Amérique et de l’Europe. Depuis quelques années (1838), cette énorme importation d’argent de l’Ouest à l’Est, a entièrement ou presque entièrement cessé ; il en résulte donc, si nous avons raison d’évaluer le produit présent des mines d’Amérique, d’Europe et de Sibérie à 5,000,000 ou 5,500,000 livres, que l’Europe et l’Amérique ont une plus grande quantité d’or et d’argent pour leur propre usage, qu’elles n’en avaient à l’époque où le produit des mines d’Amérique était à son maximum. Mac Culloch.
↑ Calcutta est le principal établissement de la Compagnie des Indes anglaises et le centre du commerce du Bengale ; c’est le marché régulateur de tout l’Indoustan. Garnier.
↑ Cette proportion diffère beaucoup des rapports établis sur des faits mieux observés, et qui fixeraient la proportion de l’or et de l’argent tirés des mines d’Amérique comme 1 : 62 Buchanan.
↑ Si les rapports dans lesquels on tire l’or et l’argent de la mine sont comme 62 : 1, et leur valeur relative comme 16 : 1, il est évident que l’argent a plus de valeur sur le marché que l’or. Buchanan.
↑ Voyez la préface de Rudiman au Recueil des Chartres d’Écosse (Scotiæ Diplomata), par Anderson.
↑ Dans toute l’Europe occidentale, la grande circulation métallique se fait au moyen de l’argent, dont on ne se sert en Angleterre que pour les petits payements. Si les nations européennes renonçaient à leur système, la demande de l’or s’accroîtrait, et celle de l’argent tomberait ; et la valeur relative de ces deux métaux serait dès ce moment modifiée en faveur de l’or. Buchanan.
↑ Ce n’est pas le seul motif.
↑ Cet inconvénient pourrait être compensé par de grands perfectionnements dans l’art d’exploiter les mines. Buchanan.
↑ Il n’y a pas de doute que la valeur l’or et de l’argent n’ait considérablement baissé depuis un demi-siècle. Buchanan.
↑ Il est incontestable que depuis l’année 1773 la valeur de l’or et de l’argent a fortement baissé; et l’énorme accroissement de la circulation en papier, abstraction faite de sa dépréciation, nous présente une explication suffisante de ce fait. C’est principalement pour les besoins des échanges commerciaux que les métaux précieux sont si fortement recherchés. Mais si dans cette fonction ils sont surpassés par le papier, ils seront moins demandes, et naturellement auront moins de valeur. L’invention de la circulation de papier donne lieu aux mêmes effets que la découverte d’une mine nouvelle et très-productive, puisque la circulation ainsi alimentée à peu de frais et dans une proportion indéfinie, laisse une plus grandemasse de métaux précieux pour d’autres usages. Lord Liverpool a constaté qu’en 1774, lors d’une nouvelle émission de monnaies d’or dans ce pays, il avait été apporté à la Monnaie pour la refonte 21 millions de guinées, outre un excédant de 10 millions qui fut maintenu en circulation. Mais il n’est pas probable qu’il existe maintenant dans la circulation plus de 4 à 5 millions : et l’équilibre s’étant étendu sur tout le marché du monde, a dû tendre généralement à réduire la valeur des métaux précieux*. Dans la circulation de la Grande-Bretagne, les monnaies d’or dominent ; mais dans la plupart des autres pays, l’argent est la monnaie principale. Si toutefois la circulation en papier prenait dans les autres parties du monde le développement qu’elle a acquis en Angleterre, une énorme quantité d’argent se trouverait sans emploi, et sa valeur éprouverait naturellement une baisse proportionnée. Le contingent des métaux précieux fourni par les mines de l’Amérique s’est aussi fortement accru dans ces quatre-vingts dernières années. L’Amérique espagnole principalement a vu, dès l’année 1770, sou commerce et sa population faire de rapides progrès : et les travaux de mines y ont été poussés avec une activité toujours croissante. La production a suivi cette proportion ascendante. D’après les rapports consignés dans le ’’Mercurio Peruano,’’ journal périodique publié à Lima, et contenant des vues très-curieuses et très-habiles sur la situation du pays, le monnayage annuel du Mexique, qui est à peu près égal au produit de ses mines en raison de la médiocrité de l’exportation des lingots, se montait, par une moyenne décennale de 1762 à 1773, à 12,503,735 dollars en argent, et 770,742 dollars en or, ensemble 15,074,495 dollars. D’après la moyenne décennale de 1782 à 1793, l’émission se monta à 19,491,509 dollars d’argent et 044,040 dollars d’or, ensemble 20,155,549 : en 1790 il fut en même temps remis à l’Espagne un demi-million de dollars en lingots. En 1795 on atteignit le chiffre de 24,512,942 dollars ; et d’après une moyenne de dix années, entre 1704 et 1805, l’émission annuelle s’éleva à 21,084,787 dollars. Le produit des mines brésiliennes a grandement fléchi depuis les soixante dernières années ; mais, en général, il est évident que toutes les fois que le papier a primé partiellement sur les métaux précieux, comme instrument d’échanges commerciaux, l’approvisionnement moyen a suivi une marche ascendante dans le monde : et ces faits expliquent surabondamment la baisse qui s’est manifestée dans leur valeur.
Cet énorme accroissement des monnaies d’or et d’argent, et la baisse de leur valeur, inséparable du premier phénomène, en provoqueront dans un temps une plus grande consommation, destinée à arrêter graduellement le déclin de leur valeur. Buchanan.
↑ Il a été déjà observé qu’à mesure que l’industrie s’étend et se perfectionne, le produit total du travail de la société devient de plus en plus considérable ; et comme la production crée la plupart des objets plus vite que la consommation ne les détruit, il en résulte une abondance toujours croissante des choses propres aux besoins et commodités de la vie, en sorte que chaque individu, à proportion de la place qu’il occupe dans l’ordre social, reçoit une portion plus ample dans la distribution générale du dividende commun, et se trouve ainsi mieux et plus largement pourvu.
Si nous comparons la somme des consommations faites actuellement dans le cours d’une année par un grand propriétaire, par un riche commerçant, par un bourgeois aisé, par un médecin, par un avocat, par un commis, et ainsi en descendant jusqu’au simple artisan des villes et au petit fermier des campagnes, arec ce qu’était cette consommation il y a cent ou cent cinquante ans, chez des hommes de condition absolument semblable, nous trouverons que la somme a tout au moins triple ou quadruplé ; et comme c’est en argent que nous évaluons toutes choses, cette somme de consommations annuelles sera représentée aujourd’hui par une quantité d’argent trois ou quatre fois plus forte. Ce fait frappe tous les esprits, d’autant plus que les effets de ce progrès étant assez sensibles dans un espace de quarante ou cinquante ans, il est beaucoup de vieillards auxquels il suffit de se rappeler les faits de leur jeunesse pour se convaincre de la vérité de cette observation.
Nous ne remonterons pas, pour le prouver, jusqu’à cette époque où un premier président du parlement de Paris stipulait, dans un bail avec son fermier, que celui-ci amènerait à la ville, à certains jours de l’aunée, une charrette bien garnie de paille fraîche pour voiturer madame la présidente et ses filles ; nous nous arrêterons à ce siècle à jamais célèbre par les chefs-d’œuvre qu’il a produits dans tous les genres de littérature et de beaux-arts, par la politesse et le bon goût dont il nous a laissé tant de modèles, et enfin par la grandeur et la magnificence qui se déployèrent à la cour du prince.
Le comte d’Aubigné, frère de Mme de Maintenon, pouvait passer déjà, en 1678, pour un assez grand seigneur. II fut depuis décoré du cordon bleu, et maria sa fille au fils aîné du maréchal de Noailles, avec une dot de 1500 mille francs de notre monnaie, tant en argent comptant qu’en pierreries. Sa maison, à l’époque où fut écrite la lettre que nous allons citer, sans être très-nombreuse, était composée de dix domestiques, et il y entretenait deux carrosses. Entrons maintenant dans l’intérieur du ménage, et voyons en quoi devait consister la dépense ordinaire, telle que la règle Mme de Maintenon, d’après l’expérience qu’elle en a et ce qu’elle en a appris dans le monde, enfin suivant ce qu’elle ferait pour elle-même, si elle ne vivait pas à la cour. Quinze livres de viande de boucherie suffisent par jour à la table des maîtres et à la nourriture des dix domestiques. C’est sur ces quinze livres de viande bouillie qu’on prend une entrée qui sera, dit-elle, tantôt de fraise de veau, tantôt de langues de mouton, etc. Le rôti sera épargné lorsque monsieur dînera en ville ou lorsque madame ne soupera pas. Une bouteille de vin par jour est plus qu’il ne faut pour la table des maîtres. On ne doit consommer qu’une livre de sucre en quatre jours pour une compote qui formera tout le dessert avec la pyramide éternelle de poires et de pommes dont on renouvelle les feuilles. Il n’y a que deux feux, celui de la cuisine et un seul pour l’appartement. On ne brûle que deux bougies dans toute la maison, et la livre de six doit durer trois jours. Le compte de chandelle est encore plus curieux, t Je mets une livre de chandelle par jour, « qui est de huit ; une dans l’antichambre, une pour les femmes, une pour les cuisines, une pour l’écurie ; je ne vois guère que ces quatre endroits où il en faille ; cependant, comme les jours sont courts, j’en mets huit, et si Aimée est ménagère et qu’elle sache serrer les bouts, cette épargne ira à une livre par semaine*. »
Tel doit être l’état de la dépense de cette maison pour tout le cours du mois et de l’année. Or, si nous revenons au temps présent, il n’est pas un bourgeois un peu aisé qui, dans le seul jour de la semaine où il juge à propos de recevoir ses amis à dîner, ne consomme en provisions de bouche de toute espèce, en vin, en sucreries, en dessert, en bois et en lumières, beaucoup plus que n’en consommaient dans tout le cours de la semaine M. et Mme d’Aubigné, sans compter le café, les vins étrangers et plusieurs autres articles qui ne figurent pas dans leur dépense, et qui sont devenus des objets de consommation habituelle pour toute la classe aisée de la société. Nous n’avons pas de détails aussi circonstanciés sur l’économie des habits et de l’ameublement ; mais il est à présumer qu’elle était mesurée sur la même proportion que la dépense de table**. Cet accroissement général dans la quantité des consommations de chaque individu est ce qui a fait monter le taux pécuniaire des traitements et rétributions dans toutes les professions libérales, et dans tous les emplois et services qui s’élèvent au-dessus du simple travail, parce que ces traitements et rétributions sont la représentation d’une plus grande quantité d’objets consommables. Le simple travail même est, par cette raison seule, payé avec un peu plus d’argent qu’il ne l’était autrefois, parce que l’ouvrier est mieux nourri et mieux velu. Les progrès successifs de l’industrie et du commerce, en offrant les objets consommables en plus grande abondance et à moindre prix, ont invité les particuliers à grossir de plus en plus la somme de leur consommation personnelle ; et par une réaction naturelle, cet effet a continuellement agi comme cause des progrès ultérieurs de l’industrie, parce que les manufactures et le commerce ne peuvent avoir de plus puissant encouragement que le prompt débit de leurs productions et marchandises. Garnier.
↑ Histoire naturelle, liv. X, chap. xxix.
↑ Histoire naturelle, liv. IX, chap. xvii.
↑ Les progrès de l’agriculture et ceux du capital agricole ont été si considérables en Écosse dans ces dernières années, que le vieux système de culture dont parle ici le Dr Smith est généralement abandonné aujourd’hui. Buchanan.
↑ Voyages de Kalm, vol. I, pages 343 et 344.
↑ Le daim est un mets extrêmement recherché en Angleterre, et qu’on se procure à grands frais pour les diners d’apparat ; mais il est probable que le prix n’en sera jamais assez élevé, ni l’usage assez général pour que ce genre de venaison devienne un objet de spéculation agricole
↑ Il est très-difficile d’établir la quotité des prix à cette époque d’une manière exacte, faute de bases régulières : on ne peut calculer que par approximation.
Buchanan.
↑ Le prix du lait peut facilement dépasser cette limite dans une grande ville, parce qu’il est impossible de le faire venir d’une certaine distance, et que les terres environnantes ne sont pas toujours en état d’y suffire. C’est ce qui arrive à Londres : la plupart des terres voisines sont consacrées au pâturage, qui leur fait produire un revenu plus élevé que par la culture du blé. Au delà de cette limite, le lait doit être converti en beurre et en fromage, qui ne procurent pas un profit aussi avantageux. Buchanan.
↑ Ce n’est pas parce qu’il en coûte davantage pour les porter au marché qu’on peut les vendre à plus haut prix ; c’est parce qu’une plus forte demande en élève le prix, indépendamment de ce qu’elles ont primitivement coûté. Buchanan.
↑ Il est inutile de rappeler ici combien les temps sont changés. A. B.
↑ Voyez Mémoires sur la laine, par Smith, vol. I, chap. v, vi et vii, et le vol. II, chap. clxxvi.
↑ II y a de fortes raisons de douter de l’exactitude des prix sur lesquels le Dr Smith raisonne arec confiance, et toutes les comparaisons de ce genre avec le passé ne sont que de pures conjectures. Ainsi, d’après certains documents réunis par Smith, il paraîtrait que le prix du blé est tombé quelquefois à 1 schelling le quarter, pour s’élever à 4 livres sterling et 16 schellings, ou même à plus de 6 livres 8 schellings. De pareilles fluctuations sont évidemment impossibles. Une hausse de prix de 128 pour cent, fût-elle le résultat d’un très-long espace de temps, aurait été la preuve d’une disette aussi funeste que la famine, et l’on n’en trouve la preuve dans aucun document authentique. Buchanan.

↑ Les lois sur la laine dont parle ici Adam Smith ont toutes été abrogées en Angleterre. La laine peut être importée et exportée librement moyennant un denier ou penny par livre, si elle vaut 1 schelling la livre, et 1/2 denier, soit environ 6 centimes par livre, si elle est au-dessous de ce prix. A. B.
↑ En déclarant que les lois qui prohibent l’exportation de la laine ont contribué à en maintenir le prix dans des limites inférieures, le Dr Smith attribue, comme d’ordinaire, de trop grands effets aux mesures artificielles de la législation, quoique ses propres principes suffisent pour démontrer que dans les circonstances actuelles du pays le prix de la laine ne peut avoir été que légèrement modifié par des restrictions de ce genre. Son argument est que le marché national étant encombré de laines par l’effet de la prohibition qui menace l’exportation, elles doivent être violemment maintenues au-dessous de leur prix naturel. Il déclare aussi que la permission d’importer les laines espagnoles en franchise, en ajoutant à la concurrence sur le marché national, a dû fortement seconder cette modicité dans leur prix. Mais si les laines espagnoles ont été importées, n’est-il pas clair que c’est en raison de l’insuffisance de l’approvisionnement national, et que la loi prohibant l’exportation ne pourrait des lors produire aucun mal réel ? Car en effaçant la prohibition, il n’est pas vraisemblable qu’on eût expédié des laines hors d’un pays qui n’en produisait pas une quantité suffisante pour sa propre consommation.
D’ailleurs, l’approvisionnement ne répondant pas à la demande dans le pays, ta marchandise a dû nécessairement, et indépendamment de l’exportation, atteindre un prix très-favorable, sans lequel le marché n’eût pu être pourvu par l’étranger. Buchanan.
↑ Il y a lieu de penser que le Dr Smith a exagéré l’influence des restrictions apportées au commerce des laines. Il est probable qu’elles ont nu avoir quelque effet en faisant baisser le prix de la laine très-longue, employée par les fabriques de tricots et d’étoffes rases, article qui passe pour être de qualité supérieure en Angleterre. Mais le fait que d’énormes quantités de laine d’Allemagne, d’Espagne, d’Australie et d’autres pays, employée dans les principales branches des fabriques de draps, ont été régulièrement importées en Angleterre pendant une longue suite d’années, nous prouve que les prix de la majeure partie des laines anglaises ont été plus élevés, en moyenne, que le prix de la laine du continent. Mac Culloch.
↑ Le stone est un poids usité pour certaines denrées particulières, notamment pour les cuirs.
↑ Environ trois francs.
↑ Les diverses peaux marines font un article considérables des pêches dans les mers du Nord, et notamment dans le commerce du Groënland.
↑ On déclare nuisances tout ce qui est censé délit public et de nature à être poursuivi, par voie de plainte ou accusation, à la requête du ministère public.
↑ Ce sont les marchandises énumérées. Voyez liv. IV, chap vii.
↑ La renommée de grandes richesses et de haute civilisation acquise à la Chine repose sur les exagérations des voyageurs, que des aperçus plus nouveaux et plus profonds ont maintenant corrigées. Les Chinois, en effet, ont fait de grands progrès dans les arts mécaniques ; mais il n’existe aucune raison de croire qu’ils soient plus riches que les nations d’Europe. Les classes laborieuses y éprouvent une grande misère, circonstance qui certainement ne dénote pas de grandes richesses. Buchanan.
↑ Recherche sur la population de différentes villes de France.
↑ Essai sur les monnaies, ou Réflexions sur le rapport entre l’argent et les denrées. Paris, 1746.
↑ Voyez liv. V, chap. ii, partie 2e, art. 4, impôts sur les objets de consommation.
↑ Ce mot florissant est employé, dans tout le cours de cet ouvrage, dans le sens qu’il semble présenter le plus naturellement, quoique le plus communément usité ; il signifie ici un état de croissance et de développement.
↑ Le texte porte grained, ce qui désigne la belle teinture faite avec la drogue nommée kermès végétal, qu’on regardait comme une sorte de graine et qu’on employait pour faire la couleur écarlate avant que la cochenille fût connue en Europe.
↑ La largeur des draps est déterminée par plusieurs statuts, dont le plus ancien date de 1328. Le drap large doit porter doux aunes entre les deux lisières, après avoir été passé à l’eau. L’aune (yard) ayant 33 pouces 9 lignes du pied de roi, les deux répondent à 1 mètre 82 centimètres.
↑ Chap. x, section 1re.
↑ Voyez l’application de ces mots liv. V, chap. ii, art. 4 de la 2e section.
↑ Le docteur Smith n’a fait aucune allusion, ni dans ce chapitre ni ailleurs, à l’industrie du coton. Au temps où il écrivait, elle n’avait qu’un très-faible développement, et personne n’aurait pu prévoir les merveilleux progrès qu’elle a faits depuis. C’est maintenant une industrie d’une importance immense et sans égale, et on peut affirmer avec raison que ses progrès dans la Grande-Bretagne depuis 1770, et le grand développement qu’elle a atteint aujourd’hui, forment sans aucun doute le phénomène le plus extraordinaire que présente l’histoire de l’industrie. M. Baines, de Leeds, a publié un livre intéressant et bien fait sur l’histoire et la situation de l’industrie du coton. Mac Culloch.
↑ Mac Culloch conteste cette assertion d’Adam Smith ; il prétend que toutes les autres classes de la société profitent plus encore que les propriétaires de la baisse qui s’opère dans la valeur des produits manufacturés. Comment accorder cette opinion du commentateur avec le fait bien constaté de l’augmentation croissante de la misère dans les districts industriels de l’Angleterre ? A. B.
↑ Les progrès de l’industrie manufacturière et le bas prix des articles fabriqua n’ont pas plus de tendance à faire hausser la rente des terres que le taux des profils ou du salaire. À vrai dire, ils n’élèvent ni l’une ni l’autre. Lorsque les objets manufacturés deviendront moins chers, la même rente servira à en acheter une plus grande quantité, et la condition du propriétaire se trouvera améliorée. Mais les deux autres espèces de revenus, savoir, le profit et le salaire, ne pourront-ils de même solder une plus grande somme d’objets fabriqués alors qu’ils seront moins chers, et la condition du marchand comme du travailleur ne s’améliorera-t-elle pas en conséquence dans le même rapport que celle du propriétaire? Il ne serait cependant pas exact de dire que la renie, le profit ou les salaires s’accroissent par le seul fait qu’ils peuvent acheter plus d’un objet moins cher, et il est d’autant plus nécessaire d’indiquer des erreurs de mois de ce genre, qu’elles peuvent conduire aux plus graves inexactitudes. Buchanan.
↑ Chap. vi.
↑ Ils tendront naturellement à l’égarer par leur ardent désir d’avoir de larges revenus au moyen de la hausse générale des prix : nous en avons eu un remarquable exemple en 1815. À cette époque où les classes laborieuses et tout le pays, dans un vaste rayon, étaient livrés à la détresse depuis plusieurs années par le haut prix des subsistances, et où il était constant que l’Angleterre n’avait pas depuis longtemps produit le blé nécessaire à la consommation, les propriétaires fonciers usèrent de toute leur influence pour faire prohiber les importations, et imposer ainsi des charges énormes à la société dans un but d’intérêt privé. Buchanan.
↑ Lorsque le docteur Smith écrivit ce paragraphe, il avait oublié la prime à l’exportation du blé et l’acte qui prohibait l’importation du bétail en Angleterre. Ces faits, indépendamment des lois céréales de 1814 et 1815, montrent suffisamment que l’esprit de monopole a été aussi puissant chez les propriétaires que chez l« manufacturiers et les marchands. Mac Culloch.
↑ Chap. viii.
↑ Mac Culloch prétend que c’est le contraire qui est vrai. Selon lui, les profits sont le plus élevés dans les pays qui avancent le plus rapidement ; et s’ils paraissent très-forts dans les pays en décadence, c’est à cause du défaut de sécurité, et parce que l’on confond avec le profit la prime exigée pour la garantie du capital. Voyez ses Principles of Political economy, p. 109, 2e éd.
↑ Smith est ici évidemment égaré par son amour des théories. N’est-il pas clair que lu décadence d’un pays doit nécessairement être suivie de la destruction de son capital ; et le mal ne doit-il pas, dans cette hypothèse, retomber sur ceux qui ont des capitaux en leur possession ? Il est vrai qu’on pourra réaliser de plus grands profits sur le capital subsistant ; mais ce n’est là qu’un de ces avantages particuliers qui surgissent des calamités publiques les plus affreuses.

Il n’est pas facile de voir pourquoi les propriétaires fonciers seraient plus généreux que les autres hommes. Ceux qui envisagent les efforts qu’ils ont toujours faits pour élever le prix du blé, et conséquemment la rente de la terre, en prohibant les importations au milieu de populations accablées par le manque de subsistances, ceux-là, disons-nous, ne leur reconnaîtront certes pas avec empressement une générosité plus que commune. Buchanan.
↑ Cette remarque ne s’applique-t-elle pas avec une égale force aux propriétaires fonciers qui, craignant que le blé, cette marchandise dont ils avaient intérêt à enfler la valeur, ne pût se vendre à un prix assez élevé sur le marché national, obtinrent, par loi, une prime pour ceux qui l’exporteraient ; et qui. malgré la détresse où le pays a toujours été plongé depuis lors parla cherté des subsistances, s’efforcèrent d’arrêter les approvisionnements extérieurs, et cela, dans le seul but de profiter, par le haut prix des grains, des souffrances dont ils accablaient la société ? La vérité est que les différentes classes dans la société sont sujettes à se laisser dominer par leurs intérêts privés, et qu’elles hésitent rarement à les préférer lorsqu’ils se trouvent en opposition avec le bien général*. Buchanan.
↑ L’hectolitre et demi, en bon froment, pèse ne kilogrammes ; l’ancien setier pèse 125 kilogrammes. Cet hectolitre et demi, qui est depuis quelques années la mesure des marchés, ne orne guère que à et demi des boisseaux dont le setier contenait 12.

*. L’opinion émise ici par M. Mac Culloch n’est autre que la fameuse théorie du fermage imaginée par Ricardo, et qui a obtenu plus de succès en Angleterre que sur le continent. Il se peut, en effet, qu’une telle théorie convienne mieux aux habitudes cl peut-être aux préjugés des Anglais sur la propriété ; mais elle nous parait inférieure a celle d’Adam Smith, qui est plus conforme h la nature des choses, et qui explique d’une manière beaucoup plus simple l’origine du fermage. Le fermage n’est, selon nous, que le prix de location d’un instrument privilégié dans les pays d’aristocratie, et d’un accès plus libre dans les pays où règne l’égalité des partages. La liberté absolue du commerce eu ferait encore plus baisser le taux, si elle existait quelque part. A. B.
**. Si ce commentateur a voulu dire que les variations fréquentes auxquelles est assujetti le prix du blé en argent, à cause de la rareté ou surabondance de la denrée, n’affectent pas les autres produits bruts ou manufacturés, il a dit une chose incontestable. Mais quand on considère, en théorie, le prix du blé en argent, on ne s’occupe que de son prix naturel et permanent, et on fait abstraction des circonstances passagères qui déterminent les producteurs ou les consommateurs de blé à foire réciproquement des sacrifices momentanés, les uns en livrant leur denrée au-dessous de son prix naturel pour se débarrasser d’une quantité surabondante, les autres en donnant plus que l’équivalent de ce prix naturel dans la crainte de manquer d’un article de première nécessité. Ces chances de hausse et de baisse se balancent nécessairement, et leurs effets se détruisent les uns par les autres, en sorte que c’est toujours le prix naturel qui reste pour objet d’observation. La science de l’économie politique, qui recherche la nature des choses, n’est point comme la statistique ou comme l’administration, qui ne considèrent que des faits particuliers ou n’opèrent que sur des circonstances données. Elle raisonne d’après des lois générales dont l’action n’est jamais interrompue, et qui dominent également sur toutes les sociétés humaines. L’effet de ces lois peut être retardé ou contrarié par des causes accidentelles, mais il se réalise nécessairement au bout d’une période plus ou moins longue. Ainsi, lorsqu’un produit quelconque est dans une quantité inférieure ou supérieure aux besoins de la consommation, le principe qui détermine la reproduction ne peut manquer de rétablir, tôt ou tard, le niveau entre les quantités respectives de l’offre et de la demande. Lorsque le nombre des ouvriers est moindre ou est plus grand que la demande de travail n’en exige, la loi de la population étend ou resserre la génération suivante, et mesure celle-ci sur la quantité que le travail peut faire subsister.
Par l’effet de ces lois, l’état ordinaire de toute société est d’avoir la plus grande partie de ses membres luttant contre le besoin de se procurer des subsistances, et n’ayant, pour en obtenir, d’autre moyen que l’offre de leur travail. Ainsi, dans l’ordre naturel et permanent des sociétés, la quantité de subsistances suffisantes pour nourrir l’ouvrier et le mettre à même de se reproduire dans la génération suivante, sera le véritable prix ou équivalent du travail. À mesure qu’il y aura plus de subsistances à distribuer, il y aura plus de travail offert ; mais le rapport de valeur entre les subsistances et le travail restera toujours le même.
Si les produits bruts de la terre qui ne se recueillent pas annuellement et ne donnent pas de revenu à la propriété du soi, acquièrent une valeur, c’est parce qu’il existe une assez grande masse de subsistances pour payer le travail qu’exige l’extraction de ces produits. Si les pierres sont tirées de la carrière et transportées à la surface de la terre, si les métaux sont extraits des entrailles de la mine, c’est parce que la demande en est faite par des riches en état de les payer, c’est-à-dire par des hommes qui ont à leur disposition un superflu de subsistances, et qui veulent échanger ce superflu contre le travail des ouvriers employés aux carrières et aux mines. Toutes ces valeurs sont donc la représentation du travail qu’elles ont coûté, et ce travail est lui-même représenté et mesuré par la quantité de subsistances qui a été nécessaire pour l’alimenter. Ce qu’on appelle le prix en argent de ces subsistances, c’est la quantité d’argent que produit la portion de travail alimentée par une certaine mesure de subsistances. Si mille livres pesant de blé valent une livre pesant d’argent lin, c’est que pour rechercher, extraire, affiner et transporter cette livre d’argent, il a fallu une somme égale à ce que mille livres de blé peuvent entretenir. Tous les échanges qui se font entre les hommes sont de deux sortes ; ce sont des produits de travail échangés les uns contre les autres, ou bien ce sont des subsistances échangées soit contre du travail fait, soit contre du travail à faire. Le travail fait et le travail à faire ont naturellement la même valeur, à moins que par des circonstances accidentelles une espèce de travail fait n’ait perdu de son utilité, de cette utilité pour laquelle il a été commandé. Il est donc vrai que toute marchandise, sans exception, a pour mesure de sa valeur le travail qu’elle a coûté pour devenir valeur échangeable, ou, ce qui est la même chose, la quantité de subsistances qui est la mesure de ce travail. Les variations de prix qu’éprouvent un grand nombre de valeurs qui s’écartent de la règle, doivent s’expliquer par des causes accidentelles, étrangères au système général, et qui n’influent que sur ces valeurs particulières. Des pierres à bâtir, si la carrière se trouve dans un lieu où quelqu’un veut construire, acquerront une valeur fort supérieure au travail que nécessite leur extraction ; mais, dans cette valeur, tout ce qui excédera l’équivalent du travail d’extraction sera une prime ou tribut payé au propriétaire du sol pour obtenir de lui la permission d’exploiter la carrière. Ce tribut lui sera payé par le constructeur, et ce dernier ne peut le payer que parce qu’il a à sa disposition ou un superflu de subsistances, ou des produits de travail qui ont eux-mêmes été payés avec des subsistances. En dernière analyse, si l’on attache au mot prix son véritable sens, qui est récompense ou indemnité, on verra que le travail seul, qui est une peine ou une fatigue endurée pour autrui, a droit à un prix, et que ce prix est toujours ta subsistance du travailleur, mesurée sur ce qu’il lui faut pour vivre et se perpétuer.
Mais, dit M. Buchanan, l’ouvrier ne vit pas uniquement de blé. Qu’importe ? cela ne change rien à l’état de la question. Lorsque la société est parvenue à un grand état d’abondance, l’ouvrier reçoit, sous forme d’argent, la quantité de blé nécessaire pour sa nourriture et celle de sa famille ; plus, une portion de blé additionnelle pour payer le travail de celui qui l’habille, de celui qui le loge, de ceux qui lui fournissent de la viande, du vin, du sel, etc., parce que tous ces gens n’ont de blé à leur disposition que celui qu’ils reçoivent en échange de leurs différents travaux.
Le mouvement universel du travail de la société n’est entretenu que par une distribution de subsistances dont cette société dispose. Si vous supposez une population stationnaire qui fasse des progrès continuels en industrie et en richesse, et si vous supposez que, dans cette population, la classe oisive des propriétaires et des rentiers auxquels les propriétaires doivent chaque année une portion fixe de leurs revenus, appliquent à leur consommation personnelle un vingtième des subsistances produites chaque année, et qu’ils distribuent les dix-neuf autres vingtièmes à la classe ouvrière et industrieuse en échange des produits du travail et de l’industrie de cette classe, que! sera le résultat de cette distribution ? Dans le cas où chaque individu travailleur serait réduit à la simple nourriture qui se consomme dans sa famille, la totalité du produit du travail de la société, à la seule déduction des frais et avances indispensables pour le tenir en activité, tournerait exclusivement aux commodités et jouissances de cette classe oisive. Hais les choses ne peuvent pas se passer ainsi. Depuis le plus simple manœuvre employé à la culture, jusques à l’artiste le plus habile ou l’homme à talent le plus distingué, la distribution d’un excédant de subsistances au delà de la simple nourriture se fait par portions inégales et proportionnées au mérite de l’ouvrier, de telle manière qu’un seul individu de cette classe ouvrière reçoit une part de subsistances assez abondante pour pouvoir appliquer à ses commodités et jouissances une quantité considérable du travail des autres ouvriers. Les capitalistes ne sont eux-mêmes qu’une section de cette classe industrieuse, et le profit qui leur est attribué pour le loyer ou l’emploi de leurs capitaux, quoique réglé sur d’autres principes que le salaire, est cependant toujours puisé dans cette source commune qui fournit des indemnités et des récompenses au travail de tout genre et aux services de toute nature. Les échanges réciproques qui se font entre les membres de la classe active et industrieuse des divers produits de leurs travaux et services ne sont autre chose que des subsistances échangées contre le travail et contre ses produits ; et quoique la plupart du temps les subsistances ne s’y montrent point en nature, ce ne sont pas moins elles seules qui règlent et déterminent les conditions de l’échange. À mesure que la classe ouvrière fait des progrès en dextérité et en intelligence, son travail devient plus productif ; une journée de ce travail, plus habilement appliqué, rendra plus de choses utiles, commodes et agréables, que dix journées d’un travail grossier n’en donnaient précédemment. Dans ce sens, la subsistance aura plus de valeur réelle, ou, si l’on veut, plus d’utilité pour celui qui en dispose, mais cette circonstance ne change absolument rien à son prix d’échange ; car ce prix se mesure non sur le degré d’utilité des produits du travail, mais uniquement sur la quantité effective du travail donné. Si vous échangez le produit du genre de travail qui aura acquis le plus de perfectionnement contre le produit d’une antre sorte de travail qui n’aura pu faire aucun progrès, les conditions de l’échange seront toujours réglées par la quantité de travail donné, sans égard à la différence de leurs produits respectifs. Supposez que l’on vienne à découvrir un moyen d’améliorer le travail des mines d’or et d’argent, au point de faire produire à chaque journée de travail du mineur deux fois plus de ces métaux qu’elle n’en produit aujourd’hui, la valeur de l’or et de l’argent baisserait de moitié, attendu que cette valeur se réglerait, comme elle le fait maintenant, par la quantité de travail, et alors la même mesure de blé vaudrait deux fois plus d’argent. Tous les prix en argent, ceux des salaires, ceux des produits bruis ou manufacturés, subiraient la même augmentation, parce que toutes ces choses représentant autant de travail qu’auparavant, ne pourraient être payées que par une quantité d’argent d’une représentation équivalente.
En réduisant donc à son élément le plus simple le jeu si divers et si complique de cette grande circulation dont les innombrables fils se croisent dans tous les sens et se replient continuellement les uns sur les autres, et dont chaque mouvement est opéré par l’intermédiaire de l’argent qui masque à nos yeux la véritable matière circulante ; en nous dégageant de cette illusion qu’entretiennent les habitudes de tous les instants de notre vie, qui nous porte à voir dans l’argent le régulateur de ces valeurs, dont il n’est que la mesure usuelle, nous verrons qu’il n’existe dans la société, considérée comme industrieuse et commerçante, qu’une seule chose à laquelle on mette un prix, dans le sens exact du mot ; cette chose, c’est le travail d’autrui. Voilà ce qui se vend et s’achète sous mille formes variées à l’infini. Une seule monnaie réelle paye ce prix, et cette monnaie, c’est la subsistance. Il ne faut voir dans l’argent que du travail fait, qui a été payé par des subsistances, et qui vaut ce qu’il a été payé. Ce qu’on nomme le prix du blé en argent, est l’expression la plus simple et la plus immédiate de la valeur de l’argent ; c’est son évaluation faite en sa véritable monnaie. C’est pour cela que lorsqu’on veut apprécier l’argent dans les temps anciens, la seule méthode est de l’évaluer en blé. Dire que le prix du blé en argent ne règle pas tous les autres prix en argent, c’est briser le seul lien qui mette en rapport entre elles les diverses valeurs dont se compose la circulation. L’assertion de M. Buchanan tendrait à détruire la première base de la doctrine de l’auteur qu’il a entrepris de commenter. Garnier.
***. Principes d’économie politique, etc., tom II, page 276 de la traduction française.)

%%%%%%%%%%%%%%%%%%%%%%%%%%%%%%%%%%%%%%%%%%%%%%%%%%%%%%%%%%%%%%%%%%%%%%%%%%%%%%%%
%                                                                              %
%                                   Livre II                                   %
%                                                                              %
%%%%%%%%%%%%%%%%%%%%%%%%%%%%%%%%%%%%%%%%%%%%%%%%%%%%%%%%%%%%%%%%%%%%%%%%%%%%%%%%

\part{De la nature des fonds, de leur accumulation et de leur emploi}
\markboth{De la nature des fonds, de leur accumulation et de leur emploi}{}

%%%%%%%%%%%%%%%%%%%%%%%%%%%%%%%%%%%%%%%%%%%%%%%%%%%%%%%%%%%%%%%%%%%%%%%%%%%%%%%%
%                                 Introduction                                 %
%%%%%%%%%%%%%%%%%%%%%%%%%%%%%%%%%%%%%%%%%%%%%%%%%%%%%%%%%%%%%%%%%%%%%%%%%%%%%%%%

\chapter*{\textcolor{coquelicot}{Introduction}}
\markboth{Introduction}{}

Quand la société est encore dans cet état d’enfance où il n’y a aucune division de travail, où il ne se fait presque point d’échanges et où chaque individu pourvoit lui-même à tous ses besoins, il n’est pas nécessaire qu’il existe aucun fonds accumulé ou amassé d’avance pour faire marcher les affaires de la société. Chaque homme cherche, dans sa propre industrie, les moyens de satisfaire aux besoins du moment, à mesure qu’ils se font sentir. Quand la faim le presse, il s’en va chasser dans la forêt ; quand son vêtement est usé, il s’habille avec la peau du premier animal qu’il tue ; et si sa hutte commence à menacer ruine, il la répare, du mieux qu’il peut, avec les branches d’arbre et la terre qui se trouvent sous sa main.
Mais, quand une fois la division du travail est généralement établie, un homme ne peut plus appliquer son travail personnel qu’à une bien petite partie des besoins qui lui surviennent. Il pourvoit à la plus grande partie de ces besoins par les produits du travail d’autrui achetés avec le produit de son travail, ou, ce qui revient au même, avec le prix de ce produit. Or, cet achat ne peut se faire à moins qu’il n’ait eu le temps, non-seulement d’achever tout à fait, mais encore de vendre le produit de son travail. Il faut donc qu’en attendant il existe quelque part un fonds de denrées de différentes espèces, amassé d’avance pour le faire subsister et lui fournir, en outre, la matière et les instruments nécessaires à son ouvrage. Un tisserand ne peut pas vaquer entièrement à sa besogne particulière s’il n’y a quelque part, soit en sa possession, soit en celle d’un tiers. une provision faite par avance, où il trouve de quoi subsister et de quoi se fournir des outils de son métier et de la matière de son ouvrage, jusqu’à ce que sa toile puisse être non-seulement achevée, mais encore vendue. Il est évident qu’il faut que l’accumulation précède le moment où il pourra appliquer son industrie à entreprendre et achever cette besogne.
Puis donc que, dans la nature des choses, l’accumulation d’un capital est un préa­lable nécessaire à la division du travail, le travail ne peut recevoir des subdivisions ultérieures qu’en proportion de l’accumulation progressive des capitaux. À mesure que le travail se subdivise, la quantité de matières qu’un même nombre de personnes peut mettre en œuvre augmente dans une grande proportion ; et comme la tâche de chaque ouvrier se trouve successivement réduite à un plus grand degré de simplicité, il arrive qu’on invente une foule de nouvelles machines pour faciliter et abréger ces tâches. À mesure donc que la division du travail devient plus grande, il faut, pour qu’un même nombre d’ouvriers soit constamment occupé, qu’on accumule d’avance une égale provision de vivres, et une provision de matières et d’outils plus forte que celle qui aurait été nécessaire dans un état de choses moins avancé. Or, le nombre des ouvriers augmente, en général, dans chaque branche d’industrie, en même temps qu’y augmen­te la division du travail, ou plutôt c’est l’augmentation de leur nombre qui les met à portée de se classer et de se subdiviser de cette manière.
De même que le travail ne peut acquérir cette grande extension de puissance pro­duc­tive sans une accumulation préalable de capitaux, de même l’accumulation des capitaux amène naturellement cette extension. La personne qui emploie son capital à faire travailler cherche nécessairement à l’employer de manière à ce qu’il produise la plus grande quantité possible d’ouvrage ; elle tâche donc à la fois d’établir entre ses ouvriers la distribution de travaux la plus convenable, et de les fournir des meilleures machines qu’elle puisse imaginer ou qu’elle soit à même de se procurer. Ses moyens pour réussir dans ces deux objets sont proportionnés, en général, à l’étendue de son capital ou au nombre de gens que ce capital peut tenir occupés. Ainsi, non-seulement la quantité d’industrie augmente dans un pays en raison de l’accroissement du capital qui la met en activité, mais encore, par une suite de cet accroissement, la même quantité d’industrie produit une beaucoup plus grande quantité d’ouvrages.
Tels sont, en général, les effets de l’accroissement des capitaux sur l’industrie et sur la puissance productive.
Dans le livre suivant, j’ai cherché à expliquer la nature des fonds, les effets qui résultent de leur accumulation en capitaux de différentes espèces, et les effets qui résultent des divers emplois de ces capitaux. Ce livre est divise en cinq chapitres.
Dans le premier chapitre, j’ai tâché d’exposer quelles sont les différentes parties ou branches dans lesquelles se divise naturellement le fonds accumulé d’un individu, ainsi que celui d’une grande société.
Dans le second, j’ai traité de la nature et des opérations de l’argent considéré comme une branche particulière du capital général de la société.
Le fonds qu’on a accumulé pour en faire un capital peut être employé par la personne à qui il appartient, ou il peut être prê­té à un tiers ; la manière dont il opère dans l’une et l’autre de ces circonstances est examinée dans les troisième et quatrième chapitres.
Le cinquième et dernier chapitre traite des différents effets que les emplois diffé­rents des capitaux produisent immédiatement, tant sur la quantité d’industrie nationale mise en activité, que sur la quantité du produit annuel des terres et du travail de la société.
 
 
 
↑ Ce mot est employé dans un sens moins étendu que celui que lui a attribué l’usage. Il est ici proprement opposé à ce qu’on entend par biens-fonds ou fonds de terre, et il signifie tout amas quelconque des produits de la terre ou du travail des manufactures. C’est dans ce dernier sens qu’il est pris, quand on dit un fonds de commerce, les fonds publics, etc. Il ne prend le nom de capital que lorsqu’il rapporte à son propriétaire un revenu ou un profit quelconque*. Garnier.
*. Cette distinction entre les fonds et le capital n’est plus admise dans la langue de l’économie politique, et l’on désigne tous le nom général de capital ce que Smith appelle stock et capital. A. B.

%%%%%%%%%%%%%%%%%%%%%%%%%%%%%%%%%%%%%%%%%%%%%%%%%%%%%%%%%%%%%%%%%%%%%%%%%%%%%%%%
%                                  Chapitre 1                                  %
%%%%%%%%%%%%%%%%%%%%%%%%%%%%%%%%%%%%%%%%%%%%%%%%%%%%%%%%%%%%%%%%%%%%%%%%%%%%%%%%

\chapter{Des diverses branches dans lesquelles se divisent les fonds}
\markboth{Des diverses branches dans lesquelles se divisent les fonds}{}

Quand le fonds accumulé qu’un homme possède suffit tout au plus pour le faire subsister pendant quelques jours ou quelques semaines, il est rare qu’il songe à en tirer un revenu. Il le consomme en le ménageant le plus qu’il peut, et il tâche de ga­gner par son travail de quoi le remplacer avant qu’il soit entièrement consommé. Dans ce cas, tout son revenu procède de son travail seulement ; c’est la condition de la majeure partie des ouvriers pauvres dans tous les pays[1].
Mais quand un homme possède un fonds accumulé suffisant pour le faire vivre des mois ou des années, il cherche naturellement à tirer un revenu de la majeure partie de ce fonds, en en réservant seulement pour sa consommation actuelle autant qu’il lui en faut pour le faire subsister jusqu’à ce que son revenu commence à lui rentrer. On peut donc distinguer en deux parties la totalité de ce fonds : celle dont il espère tirer un revenu s’appelle son capital ; l’autre est celle qui fournit immédiatement à sa con­som­mation et qui consiste, ou bien, en premier lieu, dans cette portion de son fonds accumulé qu’il a originairement réservée pour cela ; ou bien en second lieu, dans son revenu, de quelque source qu’il provienne, à mesure qu’il lui rentre successi­vement ; ou bien, en troisième lieu, dans les effets par lui achetés les années précédentes avec l’une ou l’autre de ces choses, et qui ne sont pas encore entièrement consommés, tels qu’un fonds d’habits, d’ustensiles de ménage et autres effets semblables. L’un ou l’autre de ces trois articles, ou tous les trois, composent toujours le fonds que les hom­mes réservent d’ordinaire pour servir immédiatement à leur consommation personnelle. 
Il y a deux manières différentes d’employer un capital pour qu’il rende un revenu ou profit à celui qui l’emploie.
D’abord, on peut l’employer à faire croître des denrées, à les manufacturer ou à les acheter pour les revendre avec profit. Le capital employé de cette manière ne peut rendre à son maître de revenu ou de profit tant qu’il reste en sa possession ou tant qu’il garde la même forme. Les marchandises d’un négociant ne lui donneront point de revenu ou de profit avant qu’il les ait converties en argent, et cet argent ne lui en donnera pas davantage avant qu’il l’ait de nouveau échangé contre des marchandises. Ce capital sort continuellement de ses mains sous une forme pour y rentrer sous une autre, et ce n’est qu’au moyen de cette circulation ou de ces échanges successifs qu’il peut lui rendre quelque profit. Des capitaux de ce genre peuvent donc être très-proprement nommés capitaux circulants.
En second lieu, on peut employer un capital à améliorer des terres ou à acheter des machines utiles et des instruments d’industrie, ou d’autres choses semblables qui puissent donner un revenu ou profit, sans changer de maître ou sans qu’elles aient besoin de circuler davantage ; ces sortes de capitaux peuvent donc très-bien être distingués par le nom de capitaux fixes.
Des professions différentes exigent des proportions très-différentes entre le capital fixe et le capital circulant qu’on y emploie.
Le capital d’un marchand, par exemple, est tout entier en capital circulant. Il n’a pas besoin de machines ou d’instruments d’industrie, à moins qu’on ne regarde comme tels sa boutique ou son magasin.
Un maître artisan ou manufacturier a toujours nécessairement une partie de son capital qui est fixe, celle qui compose les instruments de son métier. Cependant, pour certains artisans, ce n’en est qu’une très-petite partie ; pour d’autres, c’en est une très-grande. Les outils d’un maître tailleur ne consistent qu’en quelques aiguilles ; ceux d’un maître cordonnier sont un peu plus coûteux, mais de bien peu ; ceux du maître tisserand sont beaucoup plus chers que ceux du cordonnier. Tous ces artisans ont la plus grande partie de leur capital qui circule, soit dans les salaires de leurs ouvriers, soit dans le prix de leurs matières, et qui ensuite leur rentre avec profit dans le prix de l’ouvrage.
Il y a d’autres genres de travail qui exigent un capital fixe beaucoup plus consi­dérable. Dans une fabrique de fer en gros, par exemple, le fourneau pour fondre la mine, la forge, les moulins de la fonderie sont des instruments d’industrie qui ne peuvent s’établir qu’à très-grands frais. Dans les travaux des mines de charbon et des mines de toute espèce, les machines nécessaires pour détourner l’eau et pour d’autres opérations sont souvent encore plus dispendieuses.
Cette partie du capital du fermier qu’il emploie aux instruments d’agriculture est un capital fixe ; celle qu’il emploie en salaires et subsistances de ses valets de labour, est un capital circulant. Il tire un profit de l’un en le gardant en sa possession, et de l’autre en s’en dessaisissant. Le prix ou la valeur des bestiaux qu’il emploie à ses travaux est un capital fixe tout comme le prix de ses instruments d’agriculture ; leur nourriture est un capital circulant tout comme celle de ses valets de labour. Il fait un profit sur ses bestiaux de labourage et de charroi en les gardant, et sur leur nourriture en la mettant hors de ses mains. Mais quant au bétail qu’il achète et qu’il engraisse, non pour le faire travailler, mais pour le revendre, le prix et la nourriture de ce bétail sont l’un et l’autre un capital circulant ; car il n’en retire de profit qu’en s’en des­sai­sissant. Dans les pays de pacages, un troupeau de moutons ou de gros bétail, qu’on n’achète ni pour le faire travailler ni pour le revendre, mais pour faire un profit sur la laine, sur le lait et sur le croît du troupeau, est un capital fixe. Le profit de ces bestiaux se fait en les gardant ; leur nourriture est un capital circulant : on en tire profit en le mettant hors de ses mains, et ce capital revient ensuite avec son profit et avec celui du prix total du troupeau, dans le prix de la laine, du lait et du croît. La valeur entière des semences est aussi, à proprement parler, un capital fixe. Bien qu’elles aillent et reviennent sans cesse du champ au grenier, elles ne changent néanmoins jamais de maître, et ainsi on ne peut pas dire proprement qu’elles circulent. Le profit qu’elles donnent au fermier procède de leur multiplication, et non de leur vente.
Pris en masse, le fonds accumulé que possède un pays ou une société est le même que celui de ses habitants ou de ses membres ; il se divise donc naturellement en ces trois mêmes branches, dont chacune remplit une fonction distincte.
La première est cette portion réservée pour servir immédiatement à la con­som­mation, et dont le caractère distinctif est de ne point rapporter de revenu ou de profit[2]. Elle consiste dans ce fonds de vivres, d’ habits, de meubles de ménage, etc., qui ont été achetés par leurs consommateurs, mais qui ne sont pas encore entièrement consom­més. Une partie encore de cette première branche, c’est le fonds total des maisons de pure habitation, existant actuellement dans le pays. Le capital qu’on place en une maison, si elle est destinée à être le logement du propriétaire, cesse dès ce moment de faire fonction de capital ou de rapporter à son maître un revenu. Une maison servent de logement ne contribue en rien, sous ce rapport, au revenu de celui qui l’occupe ; et quoique, sans contredit, elle lui soit extrêmement utile, elle l’est comme ses habits et ses meubles de ménage, qui lui sont aussi très-utiles, mais qui pourtant font une partie de sa dépense et non pas de son revenu[3]. Si la maison est destinée à être louée à quelqu’un, comme elle ne peut rien produire par elle-même, il faut toujours que le locataire tire le loyer qu’il paye, de quelque autre revenu qui lui vient ou de son travail, ou d’un capital, ou d’une terre. Ainsi, quoiqu’une maison puisse donner un revenu à son propriétaire, et par là lui tenir lieu d’un capital, elle ne peut donner aucun revenu au public, ni faire, à l’égard de la société, fonction de capital ; elle ne peut jamais ajouter la plus petite chose au revenu du corps de la nation. Les habits et les meubles meublants rapportent bien aussi quelquefois un revenu de la même manière à certains particuliers, auxquels ils tiennent lieu d’un capital. Dans les pays où les mascarades sont beaucoup en usage, c’est un métier que de louer des habits de masque pour une nuit. Les tapissiers louent fort souvent des ameublements au mois ou à l’année. Les entrepreneurs des convois louent, au jour ou à la semaine, l’attirail qui sert aux funérailles. Beaucoup de gens louent des maisons garnies et tirent un revenu, non-seulement du loyer de la maison, mais encore de celui des meubles. Toutefois, le revenu qu’on retire de toutes les choses de cette espèce provient tou­jours, en dernière analyse, de quelque autre source de revenu. De toutes les parties de son fonds accumulé qu’un individu ou qu’une société réserve pour servir immé­dia­tement à sa consommation, celle qui est placée en maisons est celle qui se consomme le plus lentement ; un fonds de garde-robe peut durer plusieurs années ; un fonds de meubles meublants peut durer un demi-siècle ou un siècle ; mais un fonds de maisons bien bâties et bien entretenues peut en durer plusieurs. En outre, quoique le terme de leur consommation totale soit plus éloigné, elles n’en sont pas moins un fonds destiné à servir immédiatement à la consommation, tout aussi réellement que les habits ou les meubles.
La seconde des trois branches dans lesquelles se divise le fonds général d’une société, est le capital fixe, dont le caractère distinctif est de rapporter un revenu ou profit sans changer de maître. Il consiste principalement dans les quatre articles suivants :
1° Toutes les machines utiles et instruments d’industrie qui facilitent et abrègent le travail ;
2° Tous les bâtiments destinés à un objet utile, et qui sont des moyens de revenu, non-seulement pour le propriétaire qui en retire un loyer en les louant, mais même pour la personne qui les occupe et qui en paye le loyer ; tels que les boutiques, les magasins, les ateliers, les bâtiments d’une ferme, avec toutes leurs dépendances néces­saires, étables, granges, etc. Ces bâtiments sont fort différents des maisons purement d’habitation : ce sont des espèces d’instruments d’industrie, et on peut les considérer sous le même point de vue que ceux-ci.
3° Les améliorations des terres : tout ce qu’on a dépensé d’une manière profitable à les défricher, dessécher, enclore, marner, fumer et mettre dans l’état le plus propre à la culture et au labourage. Une ferme améliorée peut, avec grande raison, être consi­dérée sous le même point de-vue que ces machines utiles qui facilitent et abrègent le travail, et par le moyen desquelles le même capital circulant peut rapporter à son maî­tre un bien plus grand revenu. Une ferme améliorée est aussi avantageuse et beaucoup plus durable qu’aucune de ces machines ; le plus souvent, les seules réparations qu’elle exige, c’est que le fermier applique de la manière la plus profitable le capital qu’il emploie à la faire valoir.
4° Les talents utiles acquis par les habitants ou membres de la société. L’acqui­sition de ces talents coûte toujours une dépense réelle produite par l’entretien de celui qui les acquiert, pendant le temps de son éducation, de son apprentissage ou de ses études, et cette dépense est un capital fixé et réalisé, pour ainsi dire, dans sa personne. Si ces talents composent une partie de sa fortune, ils composent pareillement une partie de la fortune de la société à laquelle il appartient. La dextérité perfectionnée, dans un ouvrier, peut être considérée sous le même point de vue qu’une machine ou un instrument d’industrie qui facilite et abrège le travail, et qui, malgré la dépense qu’il a coûté, restitue cette dépense avec un profit.
La troisième et dernière des trois branches dans lesquelles se divise naturellement le fonds général que possède une société, c’est son capital circulant, dont le caractère distinctif est de ne rapporter de revenu qu’en circulant ou changeant de maî­tre. Il est aussi composé de quatre articles.
1° L’argent[4], par le moyen duquel les trois autres circulent et se distribuent à ceux qui en font usage et consommation.
2° Ce fonds de vivres qui est dans la possession des bouchers, nourrisseurs de bestiaux, fermiers, marchands de blé, brasseurs, etc., et de la vente desquels ils espè­rent tirer un profit.
3° Ce fonds de matières, ou encore tout à fait brutes, ou déjà plus ou moins manufacturées, destinées à l’habillement, à l’ameublement et à la bâtisse, qui ne sont préparées sous aucune de ces trois formes, mais qui sont encore dans les mains des producteurs, des manufacturiers, des merciers, des drapiers, des marchands de bois en gros, des charpentiers, des menuisiers, des maçons, etc.
4° Enfin, l’ouvrage fait et parfait, mais qui est encore dans les mains du marchand ou manufacturier, et qui n’est pas encore débité ou distribué à celui qui doit en user ou le consommer ; tels que ces ouvrages tout faits que nous voyons souvent exposés dans les boutiques du serrurier, du menuisier en meubles, de l’orfèvre, du joaillier, du faïencier, etc.
Ainsi, le capital circulant se compose des vivres, des matières et de l’ouvrage fait de toute espèce, tant qu’ils sont dans les mains de leurs marchands respectifs, et enfin de l’argent qui est nécessaire pour la circulation de ces choses et pour leur distri­bution dans les mains de ceux qui doivent en définitive s’en servir ou les consommer.
De ces quatre articles, il y en a trois, les vivres, les matières et l’ouvrage fait, qui sont régulièrement, soit dans le cours de l’année, soit dans une période plus longue ou plus courte, retirés de ce capital circulant, pour être placés, ou en capital fixe, ou en fonds de consommation.
Tout capital fixe provient originairement d’un capital circulant, et a besoin d’être continuellement entretenu aux dépens d’un capital circulant. Toutes les machines utiles et instruments d’industrie sont, dans le principe, tirés d’un capital circulant, qui fournit les matières dont ils sont fabriqués et la subsistance des ouvriers qui les font. Pour les tenir constamment en bon état, il faut encore recourir à un capital du même genre.
Aucun capital fixe ne peut donner de revenu que par le moyen d’un capital circulant. Les machines et les instruments d’industrie les plus utiles ne produiront rien sans un capital circulant qui leur fournisse la matière qu’ils sont propres à mettre en œuvre, et la subsistance des ouvriers qui les emploient. Quelque améliorée que soit la terre, elle ne rendra pas de revenu sans un capital circulant qui fasse subsister les ouvriers qui la cultivent et ceux qui recueillent son produit.
Les capitaux tant fixes que circulants n’ont pas d’autre but ni d’autre destination que d’entretenir et d’augmenter le fonds de consommation. C’est ce fonds qui nourrit, habille et loge le peuple. Les gens sont riches ou pauvres, selon que le fonds destiné à servir immédiatement à leur consommation se trouve dans le cas d’être approvisionné, avec abondance ou avec parcimonie, par ces deux capitaux.
Puisqu’on retire continuellement une si grande partie du capital circulant pour être versée dans les deux autres branches du fonds général de la société, ce capital a besoin à son tour d’être renouvelé par des approvisionnements continuels, sans quoi il serait bientôt réduit à rien. Ces approvisionnements sont tirés de trois sources princi­pales : le produit de la terre, celui des mines et celui des pêcheries. Ces sources ramè­nent continuellement de nouvelles provisions de vivres et de matières, dont une partie est ensuite convertie en ouvrage fait, et qui remplace ainsi ce qu’on puise continuelle­ment de vivres, de matières et d’ouvrage fait, dans le capital circulant. C’est aussi des mines que l’on tire ce qui est nécessaire pour entretenir et pour augmenter cette partie du capital circulant, qui consiste dans ce qu’on nomme l’argent ; car bien que, dans le cours ordinaire des affaires, cette partie ne soit pas, comme les trois autres, nécessai­rement retirée du capital circulant, pour être placée dans les deux autres branches du fonds général de la société, elle a toutefois le sort de toutes les autres choses, qui est de s’user et de se détruire à la fin et, en outre, elle est sujette à se perdre ou à être envoyée au-dehors et, par conséquent, il faut aussi qu’elle reçoive des remplacements continuels, quoique sans contredit dans une bien moindre proportion.
La terre, les mines et les pêcheries ont toutes besoin, pour être exploitées, de capitaux fixes et circulants, et leur produit remplace avec profit non-seulement ces capitaux, mais tous les autres capitaux de la société. Ainsi, le fermier remplace annu­el­le­ment au manufacturier les vivres que celui-ci a consommés et les matières qu’il a mises en œuvre l’année précédente, et le manufacturier remplace au fermier l’ouvrage fait que celui-ci a usé ou détruit pendant le même temps. C’est là l’échange qui se fait réellement chaque année entre ces deux classes de producteurs, quoiqu’il arrive rare­ment que le produit brut de l’un et le produit manufacturé de l’autre soient troqués directement l’un contre l’autre, parce qu’il ne se trouve guère que le fermier vende son blé et son bétail, son lin et sa laine justement à la même personne chez laquelle il juge à propos d’acheter les habits, les meubles et les outils dont il a besoin. Il vend donc son produit brut pour de l’argent, moyennant lequel il peut acheter partout où bon lui semble le produit manufacturé qui lui est nécessaire. La terre elle-même remplace, au moins en partie, les capitaux qui servent à exploiter les mines et les pêcheries. C’est le produit de la terre qui sert à tirer le poisson des eaux, et c’est avec le produit de la surface de la terre qu’on extrait les minéraux de ses entrailles.
En supposant des terres, des mines et des pêcheries d’une égale fécondité, le produit qu’elles rendront sera en proportion de l’étendue des capitaux qu’on emploiera à leur culture et exploitation, et de la manière plus ou moins convenable dont ces capi­taux seront appliqués. En supposant des capitaux égaux et également bien appli­qués, ce produit sera en proportion de la fécondité naturelle des terres, des mines et des pêcheries.
Dans tous les pays où les personnes et les propriétés sont un peu protégées, tout homme ayant ce qu’on appelle le sens commun, cherchera à employer le fonds accu­mulé qui est à sa disposition, quel qu’il soit, de manière à en retirer, ou une jouissance pour le moment, ou un profit pour l’avenir. S’il l’emploie à se procurer une jouissance actuelle, c’est alors un fonds destiné à servir immédiatement à la consommation. S’il l’emploie à se procurer un profit pour l’avenir, il ne peut obtenir ce profit que de deux manières, ou en gardant ce fonds, ou en s’en dessaisissant. Dans le premier cas, c’est un capital fixe ; dans le second, c’est un capital circulant. Dans un pays qui jouit de quelque sécurité, il faut qu’un homme soit tout à fait hors de son bon sens, pour qu’il n’emploie pas, de l’une ou de l’autre de ces trois manières, tout le fonds accumulé qui est à sa disposition, soit qu’il l’ait en propre, soit qu’il l’ait emprunté d’un tiers.
À la vérité, dans ces malheureuses contrées où les hommes ont à redouter sans cesse les violences de leurs maîtres, il leur arrive souvent d’enfouir ou de cacher une grande partie des fonds accumulés, afin de les avoir en tout temps sous la main pour les emporter avec eux dans quelque asile, au moment où l’un de ces revers auxquels on se voit continuellement exposé, viendra à menacer l’existence. Cette pratique est, dit-on, très-commune en Turquie, dans l’Indostan, et sans doute dans la plupart des autres gouvernements d’Asie. Il paraît qu’elle a été fort en vogue chez nos ancêtres, pendant les désordres du gouvernement féodal. Les trésors trouvés ne fournissaient pas alors une branche peu importante du revenu des plus grands souverains de l’Europe. On comprenait sous ce nom les trésors qu’on trouvait ca­chés en terre, et auxquels personne ne pouvait prouver avoir droit. Cet article for­mait une branche de revenu assez importante pour être toujours réputé appartenir au souverain et non pas à celui qui avait trouvé le trésor, ni au propriétaire de la terre, a moins que celui-ci, par une clause expresse de sa charte, n’eût obtenu la concession de ce droit régalien. La découverte des trésors était assimilée aux mines d’or et d’argent, qui, à moins d’une clause spéciale, n’étaient jamais censées comprises dans la cession générale de la terre, quoique les mines de plomb, de cuivre, d’étain et de charbon y fussent comprises, comme étant de moindre importance[5].
 
 
 
↑ Cette distinction du fonds économique d’une nation (stock) en capital et revenu, est peu satisfaisante, et peutconduire à des conclusions erronées. Le capital d’une nation, dit l’auteur, comprend réellement toutes les parties du produitde l’industrie qui peuvent être directement employées à soutenir l’existence de l’homme ou à faciliter la production. Des portions de ce fonds, employées sans aucune intention de produire un revenu, sont souvent les plus productives. Par exemple, les fonds qu’un manufacturier emploie à sa consommation, et sans lesquels il ne pourrait subsister, sont regardés comme faisant partie du revenu ; et cependant il est évident qu’ils contribuent à augmenter sa richesse, et conséquemment celle du pays, autant qu’une quantité égale de fonds dépensés pour les ouvriers a son service. Il est toujours très-difficile de dire dans quel cas le fonds est productivement employé, et dans quel il ne l’est pas ; et toute définition du capital qui contient la détermination de ce point ne fait qu’embarrasser et obscurcir un sujet d’ailleurs très-simple par lui-même. D’après nos idées il suffit, pour faire considérer un objet comme capital, qu’il puisse concourir directement a soutenir l’existence de l’homme, ou l’aider à s’approprier ou à produire des choses utiles. Il est possible, à la vérité, qu’il ne soit employé à aucun de ces deux objets ; mais la question relative au mode d’emploi d’un objet est tout à fait distincte de la question de savoir s’il est un capital. Mac Culloch.
↑ Voir la note précédente.
↑ Une maison d’habitation est au moins indirectement, sinon directement, une source de revenu. Toute entreprise utile ou industrieuse exige que ceux qu’elle emploie soient logés. Il en résulte donc que le capital employé à bâtir des maisons pour ces personnes est aussi bien employé pour l’avantage public que celui employé à acheter les outils et instruments nécessaires pour accomplir leurs fonctions respectives. Par exemple, la possession d’une maison n’est pas moins nécessaire qu’une navette à l’exercice de sou industrie ; et si l’on dit que la dernière est un instrument productif, parce qu’elle facilite le travail du tisserand, la première, pour la même raison, doit être aussi productive. Mac Culloch.
↑ Ce mot est pris ici dans son sens le plus vulgaire, et désigne toute la monnaie de la corculation.
↑ Ce chapitre, si remarquable par sa netteté, continue une véritable découverte. La distinction des capitaux fixes et des capitaux engagés a exercé une influence immense sur la marche de toutes les industries.

%%%%%%%%%%%%%%%%%%%%%%%%%%%%%%%%%%%%%%%%%%%%%%%%%%%%%%%%%%%%%%%%%%%%%%%%%%%%%%%%
%                                  Chapitre 2                                  %
%%%%%%%%%%%%%%%%%%%%%%%%%%%%%%%%%%%%%%%%%%%%%%%%%%%%%%%%%%%%%%%%%%%%%%%%%%%%%%%%

\chapter{De l’argent considéré comme une branche particulière du fonds général de la société, ou de la dépense qu’exige l’entretien du capital national}
\markboth{De l’argent considéré comme une branche particulière du fonds général de la société, ou de la dépense qu’exige l’entretien du capital national}{}

On a fait voir, dans le premier livre[1], que le prix de la plupart des marchandises se résout en trois parties qui ont concouru à produire la marchandise et à la mettre au marché, et que l’une paye les Salaires du travail, l’autre les profits du capital, et la troisième la rente de la terre ; qu’il y a à la vérité quelques marchandises dont le prix se compose de deux de ces parties seulement, les salaires du travail et les profits du capital, et un très-petit nombre dans lesquelles il consiste ; mais que le entièrement en une seule, les salaires du travail ; prix de toute marchandise quelconque se résout né­ces­sairement en l’une ou l’autre de ces parties, ou en toutes trois, puisque la portion de prix qui ne va ni à la rente ni aux salaires, va de toute nécessité au profit de quelqu’un.
On a observé que puisqu’il en était ainsi pour toute marchandise quelconque prise séparément, il fallait nécessairement qu’il en fût de même pour les marchandises qui composent la totalité du produit de la terre et du travail d’un pays, prises en masse. La somme totale du prix ou de la valeur échangeable de ce produit annuel doit se résou­dre de même en ces trois parties et se distribuer entre les différents habitants du pays, ou comme salaires de leur travail, ou comme profits de leur capital, ou comme rentes de leur terre.
Mais quoique la valeur totale du produit annuel des terres et du travail d’un pays soit ainsi partagée entre les différents habitants et leur constitue un revenu, cependant, de même que dans le revenu d’un domaine particulier nous distinguons le revenu brut et le revenu net, nous pouvons aussi faire une pareille distinction à l’égard du revenu de tous les habitants d’un grand pays.
Le revenu brut d’un domaine particulier comprend généralement tout ce que débourse le fermier ; le revenu net est ce qui reste franc et quitte de toutes charges au propriétaire, après la déduction des frais de régie, des réparations et tous les autres prélèvements nécessaires, ou bien ce qu’il peut, sans nuire à sa fortune, placer dans le fonds qu’il destine à servir immédiatement à sa consommation, c’est-à-dire dépenser pour sa table, son train, les ornements et l’ameublement de sa maison, ses jouissances et amusements personnels. Sa richesse réelle n’est pas en proportion de son revenu brut, mais bien de son revenu net.
Le revenu brut de tous les habitants d’un grand pays comprend la masse totale du produit annuel de leur terre et de leur travail ; leur revenu net est ce qui leur reste franc et quitte, déduction faite de ce qu’il faut pour entretenir premièrement leur capital fixe ; secondement, leur capital circulant, ou bien ce qu’ils peuvent placer, sans empiéter sur leur capital, dans leur fonds de consommation, c’est-à-dire ce qu’ils peu­vent dépenser pour leurs subsistance, commodités et amusements. Leur richesse réelle est aussi en proportion de leur revenu net, et non pas de leur revenu brut.
Il est évident qu’il faut retrancher du revenu net de la société toute la dépense d’entre­tien du capital fixe. Les matières nécessaires pour l’entretien des machines utiles, des instruments d’industrie, bâtiments d’exploitation, etc., pas plus que le pro­duit du travail nécessaire pour donner à ces matières la forme convenable, ne peuvent jamais faire partie de ce revenu net. Le prix de ce travail, à la vérité, peut bien en faire partie, puisque les ouvriers qui y sont employés peuvent placer la valeur entière de leurs salaires dans leur fonds de consommation ; mais la différence consiste en ce que, dans les autres sortes de travail, et le prix et le produit vont l’un et l’autre à ce fonds ; le prix va à celui des ouvriers, et le produit à celui d’autres personnes dont la sub­sis­tance, les commodités et les agréments se trouvent augmentés par le travail de ces ouvriers.
La destination du capital fixe est d’accroître la puissance productive du travail, ou de mettre le même nombre d’ouvriers à portée de faire une beaucoup plus grande quan­tité d’ouvrage. Dans une ferme où tous les bâtiments d’exploitation, où les clôtu­res, les cours d’eau, les communications, etc., sont dans le meilleur ordre possible, le même nombre d’ouvriers et les bestiaux de labour produiront une bien plus grande récolte que dans un terrain tout aussi bon et tout aussi étendu, mais qui ne sera pas pourvu des mêmes avantages. Dans des manufactures, le même nombre d’ouvriers, à l’aide des meilleures machines possibles, fournira une bien plus grande quantité de produits que s’ils avaient des outils moins perfectionnés. Ce qu’on dépense d’une manière judicieuse pour le placer dans un capital fixe quelconque est toujours rem­boursé avec un gros profit, et il ajoute au produit annuel une valeur bien supé­rieure à celle qu’exige l’entretien de ces sortes d’améliorations. Cet entretien cepen­dant emporte nécessairement une portion du produit. Une certaine quantité de matières et le travail d’un certain nombre d’ouvriers qui auraient pu l’un et l’autre être employés immédiatement à augmenter la quantité de vivres, habits et logements, à rendre plus abondantes la subsistance et les commodités de la société, se trouvent par là détournés vers un autre emploi extrêmement avantageux, il est vrai, mais toujours différent de celui-là. C’est sous ce rapport qu’on regarde toujours comme un grand avantage pour une société tous les nouveaux procédés en mécanique, qui mettent un même nombre d’ouvriers en état de faire la même quantité d’ouvrage avec des machines plus simples et moins coûteuses que celles dont on faisait usage précédemment. Il se trouve alors une certaine quantité de matériaux et un certain nombre d’ouvriers qui avaient été employés auparavant à entretenir des machines plus compliquées et plus dispen­dieuses, et qui maintenant peuvent l’être à augmenter la quantité de l’ouvrage pour lequel ces machines ou d’autres ont été faites. Si l’entrepreneur d’une grande manu­facture qui dépense par an 1000 livres à l’entretien de ses machines peut trouver le moyen de réduire cette dépense à 500 livres, il emploiera naturellement les autres 500 livres à acheter une quantité additionnelle de matières pour être mises en œuvre par un nombre additionnel d’ouvriers. Ainsi, la quantité de l’ouvrage particulier pour lequel ces machines ont été faites, et qui constitue toute leur utilité, se trouvera natu­rellement augmentée, ainsi que les commodités et les avantages que cet ouvrage peut procurer à la société.
La dépense d’entretien du capital fixe d’un grand pays peut très-bien se comparer à celle des réparations d’un domaine particulier. La dépense des réparations peut sou­vent être nécessaire pour maintenir le produit du domaine et, par conséquent, pour conserver tant le revenu brut que le revenu net du propriétaire. Cependant, lorsqu’en la dirigeant d’une manière mieux entendue, on peut la diminuer sans donner lieu à aucune diminution de produit, le revenu brut reste tout au moins le même qu’aupa­ravant, et le revenu net est nécessairement augmenté.
Mais, quoique toute la dépense d’entretien du capital fixe se trouve ainsi nécessai­rement retranchée du revenu net de la société, il n’en est pas de même à l’égard de la dépense d’entretien du capital circulant. On a déjà observé que, des quatre articles qui composent ce capital, qui sont l’argent, les vivres, les matières et l’ouvrage fait, les trois derniers en sont régulièrement retirés pour être versés, soit dans le capital fixe de la société, soit dans le fonds de consommation. De ces choses consommables, tout ce qui ne se trouve pas employé à l’entretien du premier de ces deux fonds, va en entier à l’autre, et fait partie du revenu net de la société ; ainsi, l’entretien de ces trois parties du capital circulant ne retranche du revenu net de la société aucune autre portion du produit annuel que celle qui est nécessaire à l’entretien du capital fixe.
À cet égard, le capital circulant d’une société diffère de celui d’un individu. Celui d’un individu ne peut entrer pour la moindre partie dans son revenu net, qui se compose uniquement de ses profits. Mais, encore que le capital circulant de chaque individu fasse une partie de celui de la société dont il est membre, il ne s’ensuit pas que ce capital ne puisse de même entrer pour quelque chose dans le revenu net de la nation. Quoique les marchandises qui composent le fonds de boutique d’un marchand ne puissent nullement être versées dans son fonds de consommation, elles peuvent néanmoins aller à celui d’autres personnes qui, au moyen d’un revenu qu’elles tirent de quelque autre source, sont en état d’en remplacer régulièrement la valeur au mar­chand, ainsi que ses profits, sans qu’il en résulte aucune diminution ni dans le capital du marchand ni dans le leur[2].
L’argent est donc la seule partie du capital circulant d’une société dont l’entretien puisse occasionner quelque diminution dans le revenu net de la nation. 
Le capital fixe et cette partie du capital circulant qui consiste en argent, ont une très-grande ressemblance l’un avec l’autre, sous le rapport de leur influence sur le reve­nu de la société.
Premièrement, de même que les machines et instruments d’industrie, etc., exigent une certaine dépense, d’abord pour les fabriquer et ensuite pour les entretenir, les­quelles dépenses, bien qu’elles fassent partie du revenu brut de la société, sont l’une et l’autre des déductions à faire sur un revenu net, de même le fonds d’argent monnayé qui circule dans un pays exige une certaine dépense, d’abord pour le former, et ensuite pour l’entretenir, lesquelles dépenses sont aussi à déduire l’une et l’autre du revenu net de la société, bien qu’elles fassent partie de son revenu brut. Il se trouve une certaine quantité de matières très-précieuses, l’or et l’argent, et une certaine quantité de travail d’une nature très-industrieuse, lesquelles, au lieu de servir à augmenter le fonds de consommation, à multiplier les subsistances, commodités et agréments des individus, sont employées à entretenir ce grand mais dispendieux instrument de commerce, au moyen duquel les subsistances, commodités et agréments de chaque individu dans la société lui sont régulièrement distribués dans les justes proportions auxquelles il a droit.
Secondement, de même que les machines et instruments d’industrie, etc., qui com­po­sent le capital fixe, soit d’un individu, soit d’une société, ne font partie ni du revenu brut ni du revenu net de l’un ou de l’autre, de même l’argent, au moyen duquel tout le revenu de la société est régulièrement distribué entre ses différents membres, ne fait nullement lui-même partie de ce revenu. La grande roue de la circulation est tout à fait différente des marchandises qu’elle fait circuler. Le revenu de la société se compose uniquement de ces marchandises, et nullement de la roue qui les met en circulation. Quand nous calculons le revenu brut et le revenu net d’une société, nous sommes toujours obligés de retrancher, de la masse totale d’argent et de marchandises qui compose sa circulation annuelle, la valeur entière de l’argent, dont il n’y a pas un seul écu qui puisse jamais faire partie de l’un ni de l’autre de ces revenus.
Il n’y a que l’ambiguïté du langage qui puisse faire paraître cette proposition douteuse ou paradoxale. Bien développée et bien entendue, elle est évidente par elle-même.
Quand nous parlons d’une somme d’argent particulière, quelquefois nous n’enten­dons autre chose que les pièces de métal qui la composent ; quelquefois aussi nous renfermons dans la signification du mot un rapport confus aux choses qu’on peut avoir en échange pour cette somme, ou au pouvoir d’acheter que donne la possession de cet argent. Par exemple, quand nous disons que l’argent qui circule en Angleterre a été évalué à 18 millions sterl., nous voulons exprimer seulement le nombre des pièces de métal que quelques écrivains, d’après leurs calculs ou plutôt leur imagination, ont cru exister dans la circulation du pays. Mais quand nous disons qu’un homme a 50 ou 100 livres de rente, nous voulons ordinairement exprimer, non-seulement le montant des pièces de métal qui lui sont payées annuellement, mais la valeur des choses qu’il peut acheter ou consommer annuellement. Nous entendons communément affirmer quelle est ou doit être sa manière de vivre, ou bien quelle est la quantité et qualité des choses propres aux besoins et commodités de la vie, dont il est maître de se procurer la jouissance.
Lorsque, par une certaine somme d’argent, nous voulons exprimer non-seulement le montant des pièces de métal dont elle est composée, mais que nous entendons encore renfermer dans la signification du mot quelque rapport confus aux choses qu’on peut avoir en échange pour ces pièces, alors la richesse ou le revenu que cette somme indique dans ce cas, est égal seulement à une des deux valeurs qui se trouvent ainsi conjointes, par une sorte d’ambiguïté, dans le même mot, et plus promptement à la dernière qu’à la première, à ce que vaut l’argent, plutôt qu’à l’argent même.
Ainsi, si un particulier a une guinée de pension par semaine, il peut acheter avec, dans le cours d’une semaine, une certaine quantité de choses propres à sa subsistance, ses commodités et agréments. Sa richesse réelle, son revenu réel de la semaine sera grand ou petit, à proportion que sera grande ou petite la quantité de ces choses. Cer­tainement, son revenu de la semaine n’est pas égal à la fois à la guinée et à ce qu’il peut acheter avec, mais seulement à l’une ou l’autre de ces deux valeurs égales, et plus proprement à la dernière qu’à la première ; à ce que vaut la guinée, plutôt qu’à la guinée elle-même.
Si la pension de ce particulier, au lieu de lui être payée en or, lui était payée en un billet d’une guinée à toucher par semaine, à coup sûr ce serait bien moins ce morceau de papier que ce qu’il pourrait acquérir par ce moyen, qui constituerait proprement son revenu. Or, une guinée peut être regardée comme un billet au porteur sur tous les marchands du voisinage, payable en une certaine quantité de choses propres aux besoins et commodités de la vie. Le revenu de celui à qui on la paye consiste, à pro­pre­ment parler, bien moins dans la pièce d’or que dans ce qu’il peut acheter avec ou dans ce qu’il peut avoir en échange. Si l’on ne pouvait l’échanger pour rien, elle serait comme un billet sur un banqueroutier, et n’aurait pas plus de valeur que le moindre chiffon de papier.
De même, quoique tous les différents habitants d’un pays puissent toucher ou touchent en effet le plus souvent en argent leur revenu de la semaine ou de l’année, néanmoins leur richesse réelle à tous, leur véritable revenu de la semaine ou de l’année, pris collectivement, sera toujours grand ou petit, en proportion de la quantité de choses consommables qu’ils peuvent tous acheter avec cet argent. Le revenu d’eux tous, pris collectivement, est évidemment égal, non pas à la fois à l’argent et aux choses consommables, mais seulement à l’une ou à l’autre de ces deux valeurs, et plus proprement à la dernière qu’à la première.
Ainsi, si nous exprimons souvent le revenu d’une personne par les pièces de métal qui lui sont payées annuellement, c’est parce que le montant de ces pièces détermine l’étendue de son pouvoir d’acheter ou la valeur des marchandises qu’elle est en état de consommer annuellement. Nous n’en considérons pas moins son revenu comme consistant dans cette faculté d’acheter ou de consommer, et non pas dans les pièces qui transportent cette faculté.
Mais si cette proposition est assez évidente à l’égard d’un individu, elle l’est enco­re bien plus à l’égard d’une société. Le montant des pièces de métal qui sont payées annuellement à un particulier est souvent précisément égal à son revenu et, sous ce rapport, il est la plus courte et la meilleure expression de la valeur de ce revenu. Mais le montant des pièces de métal qui circulent dans une société ne peut jamais être égal au revenu de tous ses membres. Comme la même guinée qui paye aujourd’hui à un homme sa pension de la semaine peut payer demain celle d’un autre, et après-demain celle d’un troisième, il faut de toute nécessité que le montant des pièces de métal qui circulent annuellement dans un pays soit d’une bien moindre valeur que la totalité des pensions qui se payent annuellement avec. Mais le pouvoir d’acheter, mais les choses qui peuvent être achetées les unes après les autres avec la totalité de ces pensions en argent, à mesure que celles-ci sont payées les unes après les autres, doivent toujours être précisément de la même valeur que toutes ces pensions, comme l’est pareillement le revenu total des personnes à qui ces pensions sont payées. Par conséquent, ce revenu total ne peut consister dans ces pièces de métal dont le montant est si fort inférieur à sa valeur, mais il consiste dans la faculté d’acheter, dans les choses con­som­mables qu’on peut acheter les unes après les autres avec ces pièces, à mesure qu’elles circulent de main en main[3].
Ainsi l’argent, cette grande roue de la circulation, ce grand instrument du com­merce, tel que tous les autres instruments d’industrie, quoiqu’il compose une partie et une partie très-précieuse du capital de la société à laquelle il appartient, n’entre pour rien absolument dans son revenu ; et quoique ce soient les pièces de métal dont il est composé qui, dans le cours de leur circulation annuelle, distribuent tout juste à chacun la portion de revenu qui lui revient, elles ne font nullement elles-mêmes partie de ce revenu.
Troisièmement enfin, cette partie du capital circulant, qui consiste en argent, a encore une autre ressemblance avec les machines, instruments d’industrie, etc., qui com­posent le capital fixe ; c’est que si toute épargne dans les frais de fabrication et d’entretien de ces machines, qui ne diminue pas la puissance productive du travail, est une amélioration dans le revenu net de la société, toute épargne dans la formation et l’entretien de cette partie du capital circulant, qui consiste en argent, est une amélio­ration exactement du même genre.
Il est assez évident (et d’ailleurs on l’a déjà expliqué en partie) que toute épargne dans la dépense d’entretien du capital fixe est une amélioration du revenu net de la société. La totalité du capital de l’entrepreneur d’un ouvrage quelconque est nécessai­re­ment partagée entre son capital fixe et son capital circulant. Tant que son capital total reste le même, plus l’une des deux parts est petite, plus l’autre sera nécessai­re­ment grande. C’est le capital circulant qui fournit les matières et les salaires du travail, et qui met l’industrie en activité. Ainsi, toute épargne dans la dépense d’entretien du capi­tal fixe, qui ne diminue pas dans le travail la puissance productive, doit augmen­ter le fonds qui met l’industrie en activité et, par conséquent, accroître le produit annuel de la terre et du travail, revenu réel de toute société.
La substitution du papier à la place de la monnaie d’or et d’argent est une manière de remplacer un instrument de commerce extrêmement dispendieux, par un autre qui coûte infiniment moins, et qui est quelquefois tout aussi commode. La circulation s’établit ainsi sur une nouvelle roue qui coûte bien moins à la fois à fabriquer et à entretenir que l’ancienne. Mais comment cette opération se fait-elle, et de quelle ma­nière tend-elle à augmenter ou le revenu brut, ou le revenu net de la société ? C’est ce qui n’est pas tout à fait si évident au premier coup d’œil, et ce qui mérite une plus longue explication.
Il y a plusieurs sortes de papier monnaie ; mais les billets circulants des banques et des banquiers sont l’espèce qui est la mieux connue et qui paraît la plus propre à remplir ce but.
Lorsque les gens d’un pays ont assez de confiance dans la fortune, la probité et la sagesse d’un banquier pour le croire toujours en état d’acquitter comptant et à vue ses billets et engagements, en quelque quantité qu’il puisse s’en présenter à la fois, alors ces billets finissent par avoir le même cours que la monnaie d’or et d’argent, en raison de la certitude qu’on a d’en faire de l’argent à tout moment.
Un banquier prête aux personnes de sa connaissance ses propres billets, jusqu’à concurrence, je suppose, de 100 mille livres. Ces billets faisant partout les fonctions de l’argent, les emprunteurs lui en payent le même intérêt que s’il leur eût prêté la même somme en argent. C’est cet intérêt qui est la source de son gain. Quoique sans cesse il y ait quelques-uns de ces billets qui lui reviennent pour le payement, il y en a toujours une partie qui continue de circuler pendant des mois et des années de suite. Ainsi, quoiqu’il ait en général des billets en circulation jusqu’à concurrence de 100 mille livres, cependant souvent 20 mille livres en or et argent se trouvent faire un fonds suffisant pour répondre aux demandes qui peuvent survenir. Par conséquent, au moyen de cette opération, 20 mille. livres en or et argent font absolument la fonction de 100 mille. Les mêmes échanges peuvent se faire, la même quantité de choses con­som­mables peut être mise en circulation et être distribuée aux consommateurs aux­quels elle doit parvenir, par le moyen des billets de ce banquier, montant à 100 mille livres, tout comme cela se serait fait avec la même valeur en monnaie d’or et d’argent. On peut donc, de cette manière, faire une économie de 80 mille livres sur la circula­tion du pays, et si en même temps différentes opérations du même genre venaient à s’établir par plusieurs banques et banquiers différents, la totalité de la circulation pourrait ainsi être servie avec la cinquième partie seulement de l’or et de l’argent qu’elle aurait exigé sans cela.
Supposons, par exemple, que la masse totale d’argent circulant dans un pays, à une certaine époque, se monte à 1 million sterling, somme alors suffisante pour faire circuler la totalité du produit annuel de ses terres et de son travail.
Supposons encore que, quelque temps après, différentes banques et banquiers viennent à émettre des billets au porteur jusques à concurrence d’un million, en con­ser­vant dans leurs différentes caisses 200 mille livres pour répondre aux demandes qui peuvent survenir : il se trouverait donc alors dans la circulation 800 mille livres en or et argent, et un million de billets de banque, ou bien 1800 mille livres, tant en argent que papier. Or, 1 million seulement suffisait auparavant pour faire circuler et distribuer aux consommateurs tout le produit annuel des terres et du travail du pays, et ce produit ne peut pas se trouver augmenté tout d’un coup par ces opérations de banque. Un million suffira donc tout de même après pour le faire circuler. La quantité de marchandises qu’il s’agit de vendre et d’acheter étant la même qu’auparavant, il ne faudra que la même quantité d’argent pour toutes les ventes et tous les achats. Le canal de la circulation, si je puis me permettre cette expression, restera précisément le même qu’auparavant. Un million, d’après notre supposition, suffisait à remplir ce canal. Tout ce qu’on y versera donc au-delà de cette somme ne pourra y prendre son cours, mais sera forcé de déborder[4]. Il se trouve qu’on y a versé 1800 mille livres : donc il y a 800 mille livres qui vont nécessairement déborder, cette somme étant l’excédent de ce que peut employer la circulation du pays. Mais si cette somme ne peut pas trouver à être employée au-dedans, elle est trop précieuse pour qu’on la tienne oisive. On l’enverra donc au-dehors pour y chercher cet emploi profitable qu’elle ne peut trouver au-dedans. Or, le papier ne peut aller hors du pays, parce qu’éloigné des banques qui l’ont émis et du pays où on peut recourir à la loi pour s’en faire payer, il ne serait pas reçu dans les payements ordinaires. L’or et l’argent seront donc envoyés au-dehors jusqu’à concurrence de 800 mille livres, et le canal de la circulation inté­rieure demeurera rempli avec un million en papier, au lieu du million en métal qui le remplissait auparavant[5].
Mais si une aussi forte somme d’or et d’argent est ainsi envoyée au-dehors, il ne faut pas s’imaginer qu’elle y soit envoyée pour rien, et que les propriétaires de cet argent en fassent présent aux nations étrangères. Ils l’échangeront contre des mar­chan­dises étrangères d’une espèce ou d’une autre, destinées à la consommation de quelque autre nation ou à celle de leur propre pays.
S’ils l’emploient à acheter des marchandises dans un pays étranger pour fournir à la consommation d’un autre, ou à faire ce qu’on appelle le commerce de transport[6], tout le profit qu’ils pourront faire sera autant d’ajouté au revenu net de leur propre pays. C’est comme un nouveau fondement créé pour servir de base à un nouveau com­merce, les affaires domestiques se faisant maintenant avec le papier, et l’or et l’argent étant convertis en une matière qui fait le fondement de ce nouveau commerce.
S’ils l’emploient à acheter des marchandises étrangères pour la consommation intérieure, ou bien, en premier lieu, ils achèteront des marchandises de nature à être consommées par des gens oisifs qui ne produisent rien, telles que des vins étrangers, des soieries étrangères, etc., ou bien, en second lieu, ils achèteront un fonds addi­tionnel de matières, d’outils et de vivres, destiné à entretenir et employer un nombre additionnel de ces gens industrieux qui reproduisent, avec un profit, la valeur de leur consommation annuelle.
Employé de la première de ces deux manières, cet argent sert à développer la pro­di­galité ; il augmente la dépense et la consommation sans rien ajouter à la production, ou sans établir un fonds permanent propre à entretenir cette dépense, et sous tous les rapports il tourne au préjudice de la société[7].
Employé de la seconde manière, il agrandit d’autant les bornes de l’industrie ; et quoiqu’il augmente la consommation de la société, il ouvre une source permanente pour entretenir cette consommation, les gens qui consomment reproduisant avec un profit la valeur entière de leur consommation annuelle. Le revenu brut de la société, le produit annuel de ses terres et de son travail s’augmentent de toute la valeur que le travail de ces ouvriers ajoute aux matières sur lesquelles ils s’exercent, et son revenu net s’augmente de ce qui reste de cette valeur, déduction faite de ce qui est nécessaire à l’entretien des outils et instruments de leur industrie.
Il paraît non-seulement probable, mais presque infaillible, que la majeure partie de l’or et de l’argent, chassée au-dehors par les opérations des banques, et employée à l’achat de marchandises étrangères pour la consommation intérieure, est et doit être employée à en acheter de la seconde de ces deux espèces. Quoiqu’il y ait bien quel­ques particuliers capables d’augmenter considérablement leur dépense sans que leur revenu ait augmenté de la moindre chose, cependant nous pouvons être assurés qu’il n’y a pas de classe ou d’ordre de personnes qui soit d’humeur à se conduire ainsi, par­ce que les principes de la prudence ordinaire, s’ils ne dirigent pas toujours la conduite de chaque individu, exercent constamment leur influence sur celle de la majorité d’une classe ou ordre de personnes quelconques. Or, le revenu de gens oisifs, en les consi­dé­rant comme une classe ou ordre de gens, ne reçoit pas la plus légère augmen­tation par ces opérations de banque. Ainsi, ces opérations ne peuvent pas beaucoup contribuer à augmenter en général leur dépense, quoique celle de quelques individus, parmi eux, puisse bien être augmentée et le soit en effet quelquefois. Par conséquent, la demande que les gens oisifs pourrait faire de marchandises étant la même ou à peu près la même qu’auparavant, il est vraisemblable qu’une très-petite partie seulement de l’argent chassé au-dehors par l’effet des banques et employé à l’achat de marchandises étrangères pour la consommation intérieure, se trouvera employée à acheter de celles qui sont à leur usage. La majeure partie de cet argent sera naturellement destinée à fournir de l’emploi à l’industrie, et non pas des jouissances à la fainéantise.
Quand nous cherchons à calculer la quantité d’industrie que peut employer le capital circulant d’une société, il faut toujours n’avoir égard qu’aux trois parties seule­ment de ce capital, qui consistent en vivres, matières et ouvrage fait ; il faut toujours en déduire l’autre, qui consiste en argent et ne sert qu’à faire circuler les trois pre­mières. Pour mettre l’industrie en activité, trois choses sont nécessaires : des matiè­res sur lesquelles on travaille, des outils avec lesquels on travaille, des salaires ou récompenses en vue desquelles on travaille. Or, l’argent n’est ni une matière à travail­ler ni un outil avec lequel on puisse travailler, et quoique pour l’ordinaire ce soit en argent que les salaires se payent à l’ouvrier, cependant le revenu réel de celui-ci, comme celui des autres personnes, ne consiste pas dans l’argent même, mais dans ce que vaut l’argent ; non dans les pièces de métal, mais dans ce qu’on peut acheter avec.
La quantité d’industrie que peut mettre en œuvre un capital doit évidemment être égale au nombre d’ouvriers auxquels il peut fournir des matériaux, des outils et une subsistance convenable à la nature de l’ouvrage. L’argent peut être nécessaire pour acheter les matériaux et les outils, aussi bien que la subsistance des ouvriers ; mais certainement la quantité d’industrie que la masse totale de ce capital peut mettre en activité n’égale pas à la fois et l’argent qui achète, et les matériaux, outils et subsis­tances qui sont achetés avec l’argent ; elle égale seulement l’une ou l’autre de ces deux valeurs, et plus proprement la dernière que la première.
Quand le papier est substitué à la monnaie d’or et d’argent, la quantité de matières, d’outils et de subsistances que peut fournir la masse totale du capital circulant, peut être augmentée de toute la valeur de l’or et de l’argent qu’on avait coutume d’employer pour les acheter. La valeur entière de la grande roue de circulation et de distribution est ajoutée elle-même à la masse des marchandises qui circulaient et se distribuaient par son moyen. C’est en quelque sorte une opération semblable à celle de l’entrepre­neur d’une grande fabrique, qui, par suite de quelque heureuse découverte en mécani­que, réforme ses anciennes machines, et profite de la différence qui existe entre leur prix et celui des nouvelles, pour l’ajouter à son capital circulant à la masse où il puise de quoi fournir à ses ouvriers des matériaux et des salaires.
La proportion dans laquelle la somme d’argent en circulation dans un pays est à la valeur totale du produit annuel qu’elle fait circuler, est peut-être impossible à déter­miner. Différents auteurs l’ont évaluée au cinquième, au dixième, au vingtième et au trentième de cette valeur. Mais quelque petite qu’on suppose la proportion de la som­me d’argent en circulation relativement à la somme du produit annuel, comme il n’y a jamais qu’une portion et souvent qu’une petite portion de ce produit qui soit destinée au soutien de l’industrie, la somme d’argent en circulation doit toujours se trouver très-considérable, relativement à cette portion. Ainsi quand, au moyen de la substitution du papier, l’or et l’argent nécessaires à la circulation se trouvent réduits peut-être à un cinquième de la première somme qui en existait, n’y eût-il seulement que la valeur des quatre autres cinquièmes d’ajoutée au fonds destiné au soutien de l’industrie, ce doit toujours être une addition très-considérable à la quantité de cette industrie et, par conséquent, à la valeur du produit annuel de la terre et du travail.
Il s’est fait en Écosse, depuis vingt-cinq à trente ans, une opération de ce genre, au moyen de nouvelles compagnies de banque qui se sont établies dans presque toutes les villes un peu considérables, et même dans quelques villages. Les effets en ont été précisément ceux que je viens de décrire. Presque toutes les affaires du pays se font avec le papier de ces différentes compagnies de banque, qui sert communément aux achats et aux payements de toute sorte. On ne voit presque point d’argent, si ce n’est pour changer un billet de banque de 20 schellings, et on voit encore bien moins d’or. Mais quoique la conduite de toutes ces différentes compagnies n’ait pas été irrépro­cha­ble et qu’il ait fallu un acte du Parlement pour la régler, néanmoins le commerce du pays en a évidemment retiré de grands avantages. J’ai entendu assurer que le commerce de la ville de Glasgow avait doublé quinze ans environ après que les pre­mières banques y ont été établies, et que le commerce d’Écosse avait plus que quadruplé depuis le, premier établissement des deux banques publiques d’Édimbourg, dont l’une, appelée Banque d’Écosse, fut établie par acte du Parlement en 1695, et l’autre, appelée Banque royale, le fut par une charte du roi en 1727. Le commerce d’Écosse en général, ou celui de la ville de Glasgow en particulier, ont-ils augmenté réellement dans une proportion aussi forte pendant un temps aussi court, c’est ce que je ne prétends pas affirmer. Si l’un ou l’autre a fait un pas aussi rapide, l’effet paraît trop fort pour l’attribuer à l’action seule de cette cause. On ne saurait cependant douter que le commerce et l’industrie n’aient fait en Écosse, depuis cette époque, des progrès très-considérables, et que les banques n’aient beaucoup contribué à ces progrès.
La valeur de la monnaie d’argent qui circulait en Écosse en 1707, avant l’union, et qui fut immédiatement après portée à la banque d’Écosse pour être refrappée, s’élevait à 411,117 livres 10 schellings 9 deniers sterling. On n’a pu avoir aucun état de la monnaie d’or ; mais il paraît, par les anciens comptes de la direction des monnaies d’Écosse, que la valeur de l’or qu’on frappait annuellement excédait de quelque chose celle de l’argent[8]. Il y eut aussi dans cette occasion un assez bon nombre de gens qui, craignant de ne pas être remboursés, ne portèrent pas leur argent à la banque ; et en outre, il y avait dans la circulation un peu de monnaie anglaise qu’on n’était pas tenu d’y porter. On ne peut donc guère évaluer au-dessous d’un million sterling la somme totale d’or et d’argent qui circulait en Écosse avant l’union. Il paraît que cette somme faisait presque toute la circulation du pays ; car, quoique le papier de la banque d’Écos­se, qui n’avait point alors de rivale, fût en assez grande quantité, il paraît ce­pen­dant qu’il ne faisait qu’une petite partie de la circulation totale. Actuellement, la cir­culation totale de l’Écosse ne peut être évaluée à moins de deux millions, dont très-vraisemblablement la valeur d’or et d’argent ne forme pas un demi-million[9]. Mais quoique pendant cette période l’Écosse ait éprouvé une si grande diminution dans la somme de ses espèces circulantes, elle ne paraît en avoir éprouve aucune dans sa ri­chesse et sa prospérité. Au contraire, il y a eu des progrès évidents dans son agricul­ture, ses manufactures et son commerce, dans le produit annuel de ses terres et de son travail.
C’est principalement en escomptant des lettres de change, c’est-à-dire en avançant sur elles de l’argent avant leur échéance, que la plupart des banques et banquiers mettent leurs billets en émission ; et alors ils font, sur la somme qu’ils avancent, la déduction de l’intérêt légal jusqu’à l’échéance de la lettre de change. Le payement de la lettre, quand elle vient à échoir, fait rentrer à la banque le montant de ce qu’elle a avancé, avec le profit net de l’intérêt. Le banquier, qui n’avance ni or ni argent au négo­ciant dont il escompte la lettre de change[10], mais qui lui avance seulement ses billets, a l’avantage de pouvoir étendre ses affaires d’escompte de tout le montant de la valeur des billets qu’il sait, par expérience, avoir communément dans la circulation ; ce qui le met à même de faire le bénéfice net de l’intérêt sur une somme d’autant plus forte.
Le commerce d’Écosse, qui n’est pas à présent fort étendu, l’était encore bien moins quand les deux premières compagnies de banque furent établies ; et ces com­pagnies auraient fait très-peu d’affaires si elles eussent borné leur négoce à l’escompte des lettres de change. Elles imaginèrent donc une autre méthode d’émettre des billets, en accordant ce qu’on nommait des comptes de caisse[11], c’est-à-dire en donnant crédit jusqu’à concurrence d’une certaine somme, de 2 ou 3 mille livres[12], par exemple, à tout particulier en état de présenter deux répondants bien solvables et propriétaires fon­ciers qui voulussent garantir que tout l’argent avancé à ce particulier, dans les limites de la somme pour laquelle était donné le crédit, serait remboursé à la première de­man­de, avec l’intérêt légal. Les crédits de ce genre sont, je crois, d’un usage ordinaire dans les banques et chez les banquiers de toutes les différentes parties du monde ; mais les facilités que les compagnies de banque d’Écosse donnent pour le rembourse­ment sont, autant que je sache, particulières à ces compagnies, et sont peut-être la cause principale tant du grand commerce qu’elles font, que des grands avantages que le pays en a retirés.
Celui qui a un crédit de ce genre sur une de ces compagnies, et qui emprunte, par exemple, 1,000 livres sur ce crédit, peut rembourser la somme petit à petit par 20 ou 30 livres à la fois, la compagnie lui faisant le décompte d’une partie proportionnée à l’intérêt de la somme principale, à partir de la date du payement de chacun de ces acomptes, jusqu’à ce que le total soit ainsi remboursé. Aussi, tous les marchands et presque tous les gens d’affaires trouvent beaucoup d’avantage à ces comptes courants, et sont intéressés par là à soutenir le commerce de ces compagnies, en recevant leurs billets pour argent comptant dans tous les payements, et en engageant tous ceux sur qui ils ont de l’influence à faire de même. En général, c’est avec leurs billets que les banques avancent de l’argent à leurs clients, quand celles-ci leur en demandent. Avec ces billets, les marchands payent aux fabricants leurs marchandises, les fabricants payent aux fermiers leurs matières et subsistances, les fermiers payent aux propriétai­res leurs rentes, ceux-ci payent aux marchands les choses de commodité et de luxe dont ils se fournissent chez eux, et enfin les marchands reportent ces billets aux banques pour balancer leurs comptes courants ou pour rembourser ce qu’ils en ont emprunté, et ainsi presque tous les comptes d’argent se soldent dans le pays avec ces billets : de là le grand commerce de ces compagnies.
Au moyen de ces comptes courants, un marchand peut, sans imprudence, étendre son commerce plus qu’il ne pourrait faire sans cela. En effet, qu’il y ait deux mar­chands, l’un à Londres, l’autre à Édimbourg, qui emploient des capitaux égaux dans la même branche de commerce, le marchand d’Édimbourg pourra sans impru­dence faire un commerce plus étendu et donner de l’emploi à un plus grand nombre de gens que le marchand de Londres. Le marchand de Londres, pour faire face aux demandes qui peuvent lui survenir d’un moment à l’autre, pour le payement des marchandises achetées à crédit, est obligé de garder par devers lui une somme d’argent considérable, ou dans sa caisse ou dans celle de son banquier, qui ne lui en paye point d’intérêt. Supposons que cette somme s’élève à 500 livres, la valeur des marchandises qu’il a en magasin sera toujours de 500 livres moindre qu’elle n’eût été s’il n’avait pas été obligé de garder cette somme sans pouvoir l’employer. Supposons encore qu’en général la totalité de son capital lui rentre une fois par an, ou que les marchandises qui com­posent la valeur de tout son capital soient toutes débitées dans le cours d’une année ; étant forcé de garder une si grosse somme sans emploi, nécessairement dans le cours d’une année il vendra pour 500 livres de moins de marchandises qu’il n’aurait fait sans cela. Ses profits annuels seront nécessairement moindres de tout ce que lui eût valu la vente de 500 livres de plus de marchandises ; et le nombre de gens occupés à préparer et à mettre en état de vente ses marchandises sera aussi nécessairement moindre de toute la quantité qu’un capital de 500 livres aurait pu employer de plus. Le marchand d’Édimbourg, au contraire, ne gardera pas d’argent sans emploi pour faire face à ces demandes du moment. Quand elles lui surviennent, il y fait honneur sur son compte courant avec la banque, et il remplace successivement la somme empruntée avec l’argent ou le papier qui lui rentre de ses ventes journalières. Ainsi, avec un même capital, il peut avoir sans imprudence, dans tous les temps, en magasin, une plus grande quantité de marchandises que le marchand de Londres, et par ce moyen il peut à la fois faire personnellement un plus gros profit, et tenir encore constamment em­ployés un plus grand nombre de travailleurs pour la préparation de ses marchandises ; de là le grand avantage que le pays a retiré de ces sortes d’opérations.
On pourrait croire, en vérité, que la faculté qu’ont les négociants de Londres d’es­comp­ter les lettres de change leur procure le même avantage que les comptes courants aux négociants écossais. Mais il faut songer que les négociants d’Écosse ont, tout comme ceux de Londres, la facilité d’escompter et qu’ils ont, en outre, la commodité des comptes courants[13]. 
La masse totale de papier-monnaie de toute espèce qui peut circuler sans inconvé­nient dans un pays ne peut jamais excéder la valeur de la monnaie d’or et d’argent dont ce papier tient la place, ou qui y circulerait (le commerce étant supposé toujours le même) s’il n’y avait pas de papier-monnaie. Si les billets de 20 schellings, par exemple, sont le plus petit papier-monnaie qui ait cours en Écosse, la somme totale de ce papier qui puisse y circuler sans inconvénient ne peut pas excéder la somme d’or et d’argent qui serait nécessaire pour consommer tous les échanges de la valeur de 20 schellings et au-dessus, qui avaient coutume de se faire annuellement dans le pays. S’il arrivait une fois que le papier en circulation excédât cette somme, comme l’excé­dent ne pourrait ni être envoyé au-dehors ni rester employé dans la circulation intérieure, il reviendrait immédiatement aux banques, pour y être échangé en or ou en argent. Beaucoup de gens s’apercevraient bien vite qu’ils ont plus de ce papier que n’en exigent les affaires qu’ils ont à solder au-dedans et, ne pouvant le placer au-dehors, ils iraient aussitôt en demander aux banques le remboursement. Ce papier surabondant étant une fois converti en argent, ils trouveraient aisément à s’en servir en l’envoyant au-dehors, mais ils ne pourraient rien en faire tant qu’il resterait sous cette forme de papier. Il se ferait donc à l’instant un reflux de papier sur les banques, jusqu’à concurrence de cette surabondance, et même jusqu’à une concurrence encore plus forte, pour peu que le remboursement éprouvât de lenteur ou de difficulté ; l’alar­me qui en résulterait augmenterait nécessairement les demandes de remboursement.
En outre de toutes les dépenses qui lui sont communes avec tous les autres gens de commerce, tels que loyers de bâtiments, salaires de domestiques, commis, teneurs de livres, etc., les dépenses qui sont particulières à une maison de banque consistent principalement en deux articles : 1° la dépense qu’il en coûte pour tenir constamment dans sa caisse, afin de faire face aux demandes éventuelles des porteurs de billets, une grosse somme d’argent dont on perd l’intérêt ; 2° la dépense qu’il en coûte pour remplir la caisse sur-le-champ, à mesure qu’elle se vide en satisfaisant à ces demandes.
Une compagnie de banque qui met en émission plus de papier que n’en peut tenir employé la circulation du pays, et à qui l’excédent de son papier revient sans cesse à remboursement, doit augmenter la quantité d’or et d’argent qu’elle tient constamment en caisse, non-seulement en proportion de ce surcroît d’émission surabondante, mais dans une proportion beaucoup plus forte, parce que ses billets lui reviennent à rem­bour­sement dans une proportion de vitesse beaucoup plus grande que l’excès de leur quantité. Ainsi, cette compagnie doit augmenter le premier article de dépense, non-seulement en proportion de cette extension forcée qu’elle a donnée à ses affaires, mais dans une proportion beaucoup plus forte.
De plus, la caisse de cette compagnie, bien qu’il faille la tenir mieux garnie, se videra néanmoins beaucoup plus vite que si la compagnie eût resserré ses affaires dans des bornes plus raisonnables, et il faudra faire, pour la remplir, des efforts de dépense, non-seulement plus grands en eux mêmes, mais encore plus répétés et plus constants. D’ailleurs, l’argent qui sort continuellement de sa caisse en si grandes quantités ne peut être employé dans la circulation du pays. Il vient prendre la place d’un papier qui excède ce que cette circulation peut contenir ; il excédera donc aussi lui-même ce que cette circulation peut employer. Mais comme cet argent n’est pas fait pour rester oisif, il faut bien que, sous une forme ou sous une autre, on l’envoie au-dehors pour y trouver l’emploi avantageux qu’il ne peut trouver à l’intérieur ; et cette exportation continuelle d’or et d’argent doit augmenter nécessairement pour la banque la difficulté et, par conséquent, la dépense de se procurer de nouvelles espèces pour remplir cette caisse qui se vide avec tant de rapidité. Il faut donc que la compagnie, à mesure qu’elle donne cette extension forcée à son commerce, augmente le second article de dépense encore plus que le premier.
Supposons, en effet, une banque dont tout le papier (porté au maximum de ce que la circulation du pays en peut absorber sans inconvénient) s’élève précisément à 40,000 liv., et qui, pour faire face aux demandes éventuelles, est obligée de garder constamment en caisse un quart de son émission de billets, c’est-à-dire 10,000 liv. en espèces. Que cette banque essaye de porter son émission jusqu’à 44,000 liv., les 4,000 liv. qui sont au-delà de ce que la circulation du pays peut absorber et employer re­vien­dront à la banque presque aussitôt après qu’elles auront été émises. Donc, pour faire face aux demandes qui surviendront, cette banque sera obligée de garder cons­tamment en caisse, non pas seulement 11,000 liv., mais 14,000. Elle ne pourra donc faire aucun bénéfice sur l’intérêt de ces 4,000 liv. d’émission surabondante, et elle aura en pure perte toute la dépense de ramasser continuellement 4,000 liv. en or et en argent, qui sortiront de sa caisse aussi vite qu’on les y aura apportées.
Si chaque compagnie de banque eût toujours bien entendu et bien suivi ses inté­rêts, la circulation n’aurait jamais été surchargée de papier-monnaie ; mais toutes les banques n’ont pas toujours bien vu et bien compris ce que leur intérêt exigeait d’elles, et il est arrivé souvent que le papier a obstrué la circulation.
La banque d’Angleterre, pour avoir émis une trop grande quantité de papier, dont l’excédent lui revenait continuellement à l’échange, a été obligée, pendant plusieurs années de suite, de faire battre de la monnaie d’or jusqu’à concurrence de 800,000 liv. à 1,000,000 dans une seule année, ou en moyenne, jusqu’à environ 860,000 liv. par an. Pour fournir cette immense fabrication, la banque, à cause de l’état usé et dégradé où la monnaie d’or était depuis quelques années, se vit souvent obligée d’acheter, jusqu’au prix de 4 liv. l’once, l’or en lingot, qu’elle émettait bientôt après, sous forme de monnaie, à 3 liv. 17 schellings 10 deniers 1/2 l’once ; ce qui lui faisait une perte de 2 1/2 à 3 p. 100, sur la fabrication d’une somme aussi énorme. Ainsi, quoique la ban­que n’eût point de droit de seigneuriage à payer, et quoique, à proprement parler, la dépense de fabrication fût au frais du gouvernement, cette libéralité du gouvernement ne l’a pas empêchée de faire beaucoup de dépense[14].
Par une suite d’un excès du même genre, les banques d’Écosse se virent toutes obligées d’entretenir constamment à Londres des agents occupés à leur chercher de l’argent qui leur coûtait rarement moins de 1 1/2 ou 2 pour 100. Cet argent était envoyé par la messagerie et assuré par ceux qui se chargeaient du transport, ce qui faisait encore un surcroît de dépense de 3/4 pour 100 ou de 15 schellings par 100 liv. Ces agents ne pouvaient pas toujours suffire à remplir la caisse de leurs commettants aussi promptement qu’elle se vidait. Dans ces cas, les banques n’avaient d’autre ressource que de tirer, sur leurs correspondants à Londres, des lettres de change jus­qu’à concurrence de la somme dont elles avaient besoin. Lorsque ensuite ces corres­pondants tiraient sur la banque pour le payement de cette somme, avec l’intérêt et le droit de commission, quelques-unes de ces banques, dans l’embarras où les avait jetées leur émission excessive, n’avaient pas d’autre moyen de faire honneur à cette traite que de tirer elles-mêmes de secondes lettres de change, ou sur le même, ou sur quelque autre correspondant de Londres, et il se trouvait ainsi que la même somme, ou plutôt des lettres de change pour cette même somme, faisaient quelquefois plus de deux ou trois voyages, la banque débitrice payant toujours l’intérêt de la commission sur toute la somme accumulée. Celles mêmes des banques d’Écosse qui ne se sont pas fait remarquer par une extrême imprudence, ont quelquefois été obligées d’avoir recours à cette ressource ruineuse. 
La monnaie d’or que la banque d’Angleterre ou les banques d’Écosse payaient en échange de cette partie de leur papier qui excédait ce qu’eût pu absorber la circulation du pays, se trouvant elle-même excéder ce que la circulation pouvait contenir, était quelquefois envoyée à l’étranger en espèces, quelquefois fondue et exportée en lin­gots, et quelquefois aussi fondue et revendue à la banque d’Angleterre, au prix énor­me de 4 liv. l’once. On avait bien soin de trier dans la monnaie les pièces les plus neuves et les plus pesantes, et c’étaient celles-là seulement qu’on choisissait pour exporter ou pour fondre. Dans l’intérieur, et tant qu’elles restaient sous forme de monnaie, ces pièces pesantes n’avaient pas plus de valeur que les plus légères ; mais à l’étranger elles avaient plus de valeur, et à l’intérieur lorsqu’elles étaient fondues en lingots. La banque d’Angleterre voyait, à son grand étonnement, que malgré l’immen­se fabrication de monnaie qu’elle faisait annuellement, il y avait chaque année la même disette d’espèces que l’année précédente, et que, malgré la quantité de bonne mon­naie toute neuve qu’elle répandait chaque année, l’état de la monnaie, loin de s’améliorer, ne faisait que se détériorer de plus en plus d’une année à l’autre. Chaque année, elle se trouvait dans la nécessité de faire frapper à peu près la même quantité d’or que celle qu’elle avait fait frapper l’année d’auparavant ; et au moyen de la hausse continuelle du prix des lingots, résultant de la dégradation des espèces courantes, par le frai et les rognures, la dépense de cette énorme fabrication annuelle allait toujours en augmentant de plus en plus. Il faut observer que la banque d’Angleterre, en appro­visionnant d’espèces sa propre caisse, est indirectement obligée d’en approvisionner tout le royaume, où cette caisse le verse continuellement par mille voies différentes. Ainsi, tout ce qu’il fallait d’espèces pour soutenir cette circulation surabondante de papier-monnaie anglais et écossais, tous les vides que cet excès de papier occasion­nait dans la quantité de monnaie d’or et d’argent aux besoins du royaume, c’était à la banque d’Angleterre à y suppléer. Les banques d’Écosse, sans nul doute, payent toutes fort chèrement leur propre défaut de prudence et d’attention ; mais la banque d’Angle­terre payait très-chèrement non-seulement sa propre imprudence, mais encore l’impru­dence beaucoup plus grande de presque toutes les banques d’Écosse.
La cause originaire de cette émission surabondante de papier-monnaie, ce furent les entreprises immodérées de quelques faiseurs de projets dans l’un et dans l’autre des deux royaumes. 
Ce qu’une banque peut avancer, sans inconvénient, à un négociant ou à un entre­pre­neur quelconque, ce n’est ni tout le capital avec lequel il commence, ni même une partie considérable de ce capital, mais c’est seulement cette part de son capital qu’il serait autrement obligé de garder par devers lui, sans emploi et en argent comptant, pour faire face aux demandes accidentelles. Si le papier-monnaie que la banque avan­ce n’excède jamais cette valeur, alors il n’excédera pas la valeur de l’or et de l’argent qui circuleraient nécessairement dans le pays, supposé qu’il n’y eût pas de papier-monnaie ; donc il n’excédera jamais la quantité que la circulation du pays peut aisé­ment absorber et tenir employée.
Quand une banque escompte à un négociant une lettre de change réelle, tirée par un véritable créancier sur un véritable débiteur, et qui est réellement payée à son échéance par ce débiteur, elle ne fait que lui avancer une partie de la valeur qu’il au­rait été sans cela obligé de garder sans emploi et en argent comptant, pour faire face aux demandes du moment. Le payement de la lettre de change, à son échéance, rem­place à la banque la valeur de ce qu’elle a avancé, y compris l’intérêt. La caisse de la banque, en tant qu’elle se borne à faire des affaires avec des personnes de ce genre, ressemble à un bassin dont il sort continuellement un courant d’eau, mais dans lequel il en entre aussi continuellement un autre parfaitement égal en volume à celui qui sort, de manière que, sans exiger d’autre soin ni d’attention, le bassin demeure tou­jours également plein ou à peu près. Pour tenir la caisse d’une telle banque toujours suffisamment remplie, il ne faut que peu ou point de dépense.
Sans excéder les bornes de son commerce, un négociant peut souvent avoir besoin d’une somme d’argent comptant, même sans avoir de lettre de change à escompter. Quand la banque, outre le service de lui escompter des lettres de change, lui fait encore dans ses besoins du moment l’avance de ces sommes sur son compte courant et en reçoit le remboursement petit à petit, à mesure que l’argent rentre à ce négociant par la vente journalière de ses marchandises, ainsi que les compagnies de banque écossaises en donnent la facilité, elle le dispense entièrement de la nécessité de garder par devers lui aucune partie de son capital sans emploi et en argent comptant, destinée à faire face aux demandes qui surviennent d’un instant à l’autre. Quand ces demandes se présentent, il trouve suffisamment de quoi y faire honneur dans la ressource de son compte courant avec la banque. Néanmoins, dans les affaires qu’elle fait avec de tels correspondants, la banque doit observer avec grande attention si, dans le cours d’un terme un peu court, comme de quatre, cinq, six ou huit mois, le montant des rembour­sements qu’ils lui font ordinairement est ou n’est pas absolument égal au montant des avances qu’elle leur fait de son côté. Si dans l’espace de ce court terme, le montant des remboursements que lui font certains de ses correspondants est, la plupart du temps, absolument égal au montant des avances, la banque peut en toute sûreté continuer de faire affaire avec eux. Quoique, dans ce cas, le courant qui sort continuellement du bassin puisse être d’un fort gros volume, celui qui y rentre continuellement doit nécessairement être au moins aussi gros, de manière que, sans exiger plus de soin ni d’attention, il est vraisemblable que la caisse sera toujours également pleine, ou à bien peu de chose près, et qu’il n’y aura presque jamais besoin, pour la remplir, d’une dépense extraordinaire. Si, au contraire, le montant des acomptes que rapportent certains correspondants se trouve être ordinairement fort au-dessous des avances que leur fait la banque, il n’y aurait pas de sûreté pour elle à continuer de faire des affaires avec de tels clients, s’ils persévèrent dans une pareille conduite. Dans ce cas, le cou­rant qui sort continuellement du bassin est nécessairement d’un beaucoup plus gros volume que celui qui y rentre, de manière qu’à moins de quelque grand et continuel effort de dépense pour la tenir pleine, la caisse sera bientôt tout à fait épuisée.
En conséquence, les compagnies de banque écossaises furent pendant longtemps très-attentives à exiger de tous leurs correspondants des remboursements fréquents et réguliers et, quelle que fût la fortune ou le crédit d’une personne, elles ne se souciaient pas de faire affaire avec elle, quand elle ne faisait pas avec la banque ce qu’on appe­lait des opérations fréquentes et régulières. Outre qu’avec cette attention elles s’épar­gnaient presque entièrement toutes dépenses extraordinaires pour tenir leur caisse pleine, elles y trouvaient encore deux autres avantages très-importants.
En premier lieu, cette attention mettait la banque en état de porter un jugement assez certain sur la bonne ou mauvaise situation des affaires de ses débiteurs, sans avoir besoin de chercher d’autres renseignements que ceux qu’elle trouvait dans ses propres livres, les hommes mettant pour l’ordinaire plus ou moins de régularité dans leurs payements selon que l’état de leurs affaires prospère ou décline. Un particulier qui prêtera son argent à une demi-douzaine ou à une douzaine de personnes peut bien faire par lui-même ou par ses agents des observations et des recherches exactes et suivies sur la conduite et la situation de chacun de ses débiteurs ; mais une compagnie de banque, qui prête son argent à peut-être cinq cents personnes différentes, et qui a à donner une attention continuelle à des objets d’une tout autre nature, ne peut guère prendre d’autres informations, sur la conduite et l’état des affaires de la majeure partie de ses débiteurs, que celles qu’elle trouvera dans ses propres livres. C’est vraisembla­blement cet avantage que les compagnies de banque écossaises avaient en vue en exigeant de tous leurs correspondants des remboursements fréquents et réguliers.
En second lieu, par cette précaution, les banques se garantissaient elles-mêmes de la possibilité d’émettre plus de papier-monnaie que n’en pouvait absorber aisément la circulation du pays. Quand elles observaient que, dans un espace de temps modéré, les remboursements d’un de leurs correspondants étaient, la plupart du temps, en balance exacte avec les avances qu’elles lui avaient faites, elles pouvaient être sûres que le papier qu’elles lui avaient avancé n’avait jamais excédé la quantité d’espèces qu’il eût été obligé sans cela de tenir en réserve pour faire face aux demandes du moment et que, par conséquent, le papier-monnaie qu’elles avaient mis en circulation par la voie de ce correspondant n’avait jamais excédé la quantité d’espèces qui aurait circulé dans le pays par la même voie, s’il n’y eût pas eu de papier-monnaie. La fré­quence, la régularité et le montant des acomptes payés par ce négociant étaient une démonstration suffisante que le montant des avances des banques n’avait jamais excé­dé cette partie de son capital qu’il aurait été sans cela obligé de garder chez lui sans emploi et en argent comptant, pour satisfaire aux demandes du moment, c’est-à-dire pour le mettre en état de tenir constamment employé le reste de son capital. Il n’y a que cette partie du capital d’un négociant qui sorte et rentre sans cesse dans ses mains dans de courts espaces de temps, sous la forme de monnaie, soit en espèces, soit en papier. Si les avances des banques eussent ordinairement excédé cette partie de son capital, le montant ordinaire de ses remboursements dans un court espace de temps n’aurait pu balancer le montant des avances à lui faites. Le courant qui serait entré continuellement dans le bassin par le canal particulier de ce négociant n’aurait pas été de volume égal au courant qui en serait sorti continuellement par le même canal. En excédant la quantité d’espèces qu’il aurait été obligé, sans le secours de ces avances, de réserver par devers lui pour faire face aux demandes du moment, les avances des banques auraient bientôt outrepassé la quantité d’espèces qui eût circulé dans le pays (le commerce étant supposé toujours le même) s’il n’y eût pas eu du papier-monnaie ; et elles auraient, par conséquent, outrepassé la quantité que la circulation du pays était en état d’absorber ou de tenir employée sans inconvénient, et l’excédent de ce papier-monnaie aurait immédiatement reflué vers les banques, pour y être échangé contre de l’or et de l’argent. Ce second avantage, quoique tout aussi réel que le premier, ne fut peut-être pas aussi bien senti par toutes les différentes compagnies de banque écossaises.
Quand les négociants accrédités d’un pays, en partie par la facilité d’escompter leurs lettres de change, en partie par celle des comptes courants, peuvent se dispenser de l’obligation de garder par devers eux aucune partie de leur capital sans emploi et en argent comptant pour faire face aux demandes du moment, raisonnablement ils ne doivent pas attendre de secours plus étendu de la part des banques et des banquiers, qui, lorsqu’ils ont été une fois jusque-là, ne sauraient aller plus loin sans compro­mettre leur propre intérêt et leur propre sûreté. Une banque ne peut pas, sans aller contre ses propres intérêts, avancer à un négociant la totalité ni même la plus grande partie du capital circulant avec lequel il fait son commerce, parce que, encore que ce capital rentre et sorte continuellement de ses mains sous forme d’argent, cependant il y a un trop grand intervalle entre l’époque de la totalité des rentrées et celle de la totalité des sorties, et dès lors le montant de ses remboursements ne pourrait balancer le montant des avances qui lui seraient faites dans un espace de temps assez rap­proché pour s’accommoder à ce qu’exige l’intérêt de la banque ; bien moins encore une banque pourrait-elle suffire à lui avancer quelque partie considérable de son capital fixe ; par exemple, du capital qu’un maître de forges emploie à la construction de sa forge, de son fourneau, de ses ateliers et magasins, logements de ses ouvriers, etc., du capital qu’un entrepreneur de mines emploie à construire des ouvrages pour soutenir les terres, à élever des machines pour épuiser les eaux, à faire ouvrir des routes et des communications pour les charrois, etc. ; du capital qu’un cultivateur emploie à défricher, dessécher, enclore, fumer, marner et labourer des terres incultes ; à bâtir des fermes avec toutes leurs dépendances, étables, granges, etc. Les rentrées d’un capital fixe sont presque toujours beaucoup plus lentes que celles d’un capital circulant ; et des dépenses de ce genre, en les supposant même dirigées avec toute l’intelligence et la sagesse possibles, ne rentrent guère à l’entrepreneur avant un intervalle de plusieurs années, terme infiniment trop éloigné pour convenir aux arrangements d’une banque. Des commerçants et des entrepreneurs peuvent bien sans doute très-légitimement faire aller une partie considérable de leurs affaires et entreprises avec des fonds d’em­prunts. Cependant, dans ce cas, il serait de toute justice que leur propre capital fût suffisant pour servir d’assurance, si je puis parler ainsi, au capital de leurs créanciers, ou pour que ces créanciers ne courussent presque aucune chance probable d’essuyer la moindre perte, quand même l’événement de l’entreprise se trouverait extrêmement au-dessous de l’attente des spéculateurs. Encore, même avec cette précaution, de l’argent qu’on emprunte et qu’on n’espère pas pouvoir rendre avant un terme de plusieurs années, ne devrait pas être emprunté à une banque, mais emprunté par obligation sur hypothèque aux individus qui se proposent de vivre du revenu de leur argent sans se donner l’embarras d’employer eux-mêmes le capital, et qui pour cela seront disposés à prêter ce capital à des gens bien solvables, pour un terme de plusieurs années. Il est vrai qu’une banque qui prête son argent sans qu’on ait à faire aucune dépense de pa­pier timbré ni d’honoraires de notaire pour l’obligation et l’hypothèque, et qui reçoit son remboursement avec ces facilités que donnent les compagnies de banque écos­saises, serait sans contredit un créancier fort commode pour de pareils faiseurs de spéculations et d’entreprises ; mais a coup sûr ces faiseurs de spéculations et d’entre­prises seraient, pour une pareille banque, les débiteurs les plus incommodes.
Il y a aujourd’hui plus de vingt-cinq ans que le papier-monnaie mis en émission par les différentes compagnies de banque écossaises a atteint pleinement la mesure de ce que la circulation du pays peut aisément absorber ou tenir employé, et qu’il a mê­me été de quelque chose au-delà de cette mesure. Ces compagnies avaient déjà, depuis un espace de temps aussi long, donné à tous les commerçants et entrepreneurs de l’Écosse des secours aussi étendus qu’il soit possible à des banques et à des ban­quiers d’en donner, sans compromettre leur intérêt personnel ; elles avaient quelque peu dépassé les bornes de leur commerce, et elles s’étaient attiré cette perte, ou au moins cette diminution de profit qui, dans ce genre particulier de commerce, ne man­que jamais d’être la suite du moindre pas qu’on fait au-delà des bornes. Ces commer­çants et entrepreneurs, ayant tiré tant de secours des banques et des banquiers, cher­chèrent à en tirer encore de plus étendus. Ils s’imaginèrent, à ce qu’il semble, que des banques pouvaient étendre leurs crédits à quelque somme que ce fût, selon le besoin qu’on en avait, sans s’exposer à d’autre dépense qu’à celle de quelques rames de papier. Ils se plaignirent des vues étroites et de la pusillanimité des directeurs de ces banques, qui ne savaient pas, disaient-ils, étendre leurs crédits à proportion de l’extension du commerce du pays ; voulant dire sans doute, par l’extension du com­merce, celle de leurs projets au-delà de ce qu’ils étaient en état d’entreprendre avec leurs propres fonds, ou avec ce que leur crédit leur permettait d’emprunter des parti­culiers par la voie ordinaire d’obligation ou d’hypothèque. Il paraît qu’ils s’étaient figuré que l’honneur de la banque l’obligeait à remplir ce déficit, et à leur fournir tout le capital dont ils avaient besoin pour leurs entreprises. Les banques toutefois furent d’une autre opinion, et sur le refus qu’elles firent d’étendre leurs crédits, quelques-uns de ces spéculateurs recoururent à un expédient qui remplit pour un temps leurs vues, à plus grands frais à la vérité, mais d’une manière aussi efficace qu’eût pu le faire l’ex­ten­sion la plus immodérée des crédits de la banque. Cet expédient n’était autre chose que la pratique bien connue de renouveler ses traites[15], c’est-à-dire, de tirer successi­ve­ment des lettres de change l’un sur l’autre, pratique à laquelle ont quelquefois re­cours de malheureux négociants quand ils sont aux bords de la banqueroute. Cette manière de faire de l’argent est connue depuis longtemps en Angleterre, et on dit qu’elle a été portée extrêmement loin pendant le cours de la dernière guerre, où le taux élevé des produits du commerce donnait une grande tentation d’étendre ses affaires au-delà de ses forces. D’Angleterre, cette pratique s’introduisit en Écosse, où, en comparaison du commerce très-borné de ce pays et de la modicité de son capital, elle fut bientôt portée beaucoup plus loin qu’elle n’avait jamais été en Angleterre. 
La pratique de renouveler ses traites est si bien connue de tous les gens d’affaires, qu’on pourra peut-être regarder comme inutile d’en donner l’explication. Mais comme ce livre peut tomber entre les mains de beaucoup de personnes qui ne sont pas dans les affaires, et comme les effets de cette pratique sur le commerce de banque ne sont peut-être pas généralement sentis, même par les gens qui sont dans les affaires, je vais tâcher de l’expliquer aussi clairement qu’il me sera possible.
Les coutumes établies entre marchands, qui prirent nais­sance dans le temps où la jurisprudence barbare de l’Europe ne donnait aucune force à l’exécution des contrats, et qui furent adoptées, pendant le cours des deux derniers siècles, dans la législation de toutes les nations européennes, ont attribué aux lettres de change des privilèges si extraordinaires, que l’on avance bien plus volontiers de l’argent sur ces sortes d’effets que sur toute autre espèce d’obligation, surtout quand les lettres de change sont paya­bles à un court terme, comme deux ou trois mois. Si à l’échéance de la lettre l’accep­teur ne la paye pas à l’instant de la présentation, il est dès lors en état de banqueroute. La lettre est protestée et revient sur le tireur, qui doit l’acquitter sur-le-champ, ou bien il est aussi pareillement réputé en banqueroute. Si avant de venir entre les mains de la personne qui la présente à l’accepteur pour être payée, elle a passé dans les mains d’autres personnes qui en aient successivement avancé la valeur les unes aux autres en argent ou en marchandises et qui, pour témoigner que chacune d’elles, à son tour, a reçu cette valeur, aient toutes à leur tour endossé la lettre, c’est-à-dire écrit leurs noms au dos, chaque endosseur devient, à son tour, garant du montant de la lettre envers le porteur et, faute de payement, est aussi, dès ce moment, réputé en banqueroute. Quoi­qu’il se puisse faire que le tireur, l’accepteur et les endosseurs de la lettre de chaque soient tous d’un crédit douteux, cependant la brièveté du terme de l’échéance donne toujours quelque confiance au porteur. Quand même il serait vraisemblable que toutes ces personnes finiront par faire banqueroute, ce serait grand hasard si dans un temps si court elles allaient toutes faillir. Le logement menace ruine, dit en soi-même un voyageur fatigué, et vraisemblablement il ne durera pas longtemps ; mais il y aurait bien du malheur si on ne pouvait pas risquer d’y passer une nuit.
Supposons que A, négociant à Édimbourg, tire sur B de Londres une lettre de change payable à deux mois de date. Dans la réalité, B de Londres ne doit rien. À d’Édimbourg, mais il consent d’accepter la lettre de change de A, sous la condition qu’avant le terme du payement il pourra tirer sur A d’Édimbourg une autre lettre de change de pareille somme, ensemble l’intérêt et le droit de commission, payables de même à deux mois de date. En conséquence, avant l’expiration des deux premiers mois, B tire cette lettre sur A d’Édimbourg, qui de nouveau, avant l’expiration des seconds deux mois, tire une seconde lettre sur B de Londres, payable pareillement à deux mois de date, et avant l’expiration de ce troisième terme de deux mois, B de Londres tire derechef sur A d’Édimbourg une autre lettre de change payable aussi à deux mois de date. Cette pratique a quelquefois ainsi continué, non-seulement plu­sieurs mois, mais même plusieurs années de suite, la lettre de change revenant toujours sur A d’Édimbourg, chargée de l’intérêt et de la commission accumulée de toutes les lettres précédentes. L’intérêt était de 5 pour 100 par an, et la commission n’était jamais moins du 1/2 pour 100 pour chaque traite. La commission étant répétée plus de six fois par an, tout l’argent qu’a pu faire A par cet expédient lui doit néces­sairement avoir coûté plus de 8 pour 100 par an, et quelquefois bien davantage, soit quand le prix de la commission s’est élevé, soit quand il a été obligé de payer l’intérêt de l’intérêt et de la commission des premières lettres de change. On appela cette manœuvre faire de l’argent par circulation[16].
Dans un pays où les profits ordinaires des capitaux, dans la majeure partie des affaires de commerce, sont censés rouler entre 6 et 10 pour 100, il faudrait une spécu­lation bien extraordinairement heureuse pour que ses rentrées pussent suffire, non-seulement à rembourser les frais énormes auxquels on avait emprunté les fonds pour la faire aller, mais à fournir encore un excédent pour le profit du spéculateur. Cepen­dant beaucoup de projets très-vastes et très-étendus furent entrepris et suivis pendant plusieurs années, sans autres fonds pour les soutenir que ceux qu’on s’était procurés à de si gros frais. Sans doute que les faiseurs de projets, dans leurs beaux rêves, avaient vu ce grand profit le plus clairement du monde. Avec cela je crois qu’ils ont eu bien rarement le bonheur de le rencontrer au moment de leur réveil, soit que ce moment ait tardé jusqu’au terme de leurs projets, soit qu’il ait eu lieu quand ils se sont vus hors d’état de les pousser plus avant[17].
À d’Édimbourg ne manquait pas de faire escompter régulièrement, deux mois avant leur échéance, les lettres de change qu’il, tirait sur B de Londres, auprès de quel­que banquier d’Édimbourg ; et de son côté B de Londres ne manquait pas non plus de faire escompter aussi régulièrement à la banque d’Angleterre, ou chez quelque banquier de Londres, les lettres de change qu’il tirait ensuite su. À d’Édimbourg. Tout ce qui se trouvait avancé sur ces lettres de change circulantes était, à Édimbourg, avan­cé en papier des banques d’Écosse, et à Londres, quand elles étaient escomptées à la banque d’Angleterre, en papier de cette banque. Quoique les lettres sur lesquelles ce papier avait été avancé fussent toutes remboursées à leur tour à mesure de leurs échéances, cependant la valeur qui avait été réellement avancée sur la première lettre de change n’était jamais réellement rentrée à la banque qui l’avait avancée, parce qu’avant l’échéance de chaque lettre il y avait toujours eu une autre lettre de change de tirée pour une somme tant soit peu plus forte que la lettre qui était sur le point d’être payée, et il fallait de toute nécessité, pour le payement de celle-ci, que l’autre lettre de change fût escomptée. Ce payement était donc absolument illusoire. Il ne rentrait de fait dans le bassin de la banque aucun courant qui y remplaçât réellement ce qui s’en était d’abord écoulé par la voie de ces lettres de change circulantes.
Le papier qui avait été émis sur ces lettres circulantes s’éleva, en plusieurs occa­sions, jusqu’à la totalité des fonds sur lesquels roulait quelque entreprise vaste et étendue d’agriculture, de commerce ou de manufacture ; et il ne se bornait pas simple­ment à la seule partie de ces fonds que le faiseur de projets eût été obligé, sans l’aide du papier-monnaie, de garder par-devers lui, en espèces dormantes, pour répondre aux demandes du moment. Par conséquent, la plus grande partie de ce papier se trouvait être en excédent de la valeur des espèces qui eussent circulé dans le pays s’il n’y eût pas eu de papier-monnaie. Il était donc en excédent de ce que la circulation du pays pouvait aisément absorber et tenir employé et, par conséquent, il refluait immé­diatement vers les banques, pour y être échangé contre de l’or et de l’argent qu’il leur fallait trouver où elles pouvaient. C’était un capital que ces faiseurs de projets avaient eu l’art de soutirer très-subtilement des banques, non-seulement sans qu’elles y eussent donné un consentement formel et sans qu’elles en eussent eu connaissance, mais peut-être même encore sans qu’elles pussent avoir, pendant quelque temps, le moindre soupçon qu’elles avaient réellement fait cette avance.
Quand deux particuliers qui ont ainsi à tirer réciproquement des lettres de change successives l’un sur l’autre les font escompter toujours chez le même banquier, il découvre nécessairement bientôt leur manège, et s’aperçoit clairement qu’ils trafiquent avec les fonds qu’il leur avance, et non avec aucun capital qui soit à eux en propre. Mais cette découverte n’est pas tout à fait si aisée à faire quand ils font escompter leurs lettres de change tantôt chez un banquier, tantôt chez un autre, et quand ce ne sont pas les deux mêmes personnes qui tirent constamment et successivement l’une sur l’autre, mais, que leur manœuvre roule entre un grand cercle de faiseurs de projets, qui trouvent réciproquement leur compte à s’aider les uns les autres dans cette méthode de faire de l’argent, et qui s’arrangent entre eux en conséquence pour qu’il soit aussi difficile que possible de distinguer une lettre de change sérieuse ; de recon­naître celle qui est tirée par un vrai créancier sur un vrai débiteur, d’avec celle dont il n’y a véritablement de créancier réel que la banque qui l’a escomptée, et de débiteur réel que le faiseur de projets, qui se sert de l’argent. Lors même qu’un ban­quier venait à découvrir ce manège, il pouvait se faire quelquefois qu’il le découvrît trop tard, et qu’il s’aperçût que, s’étant déjà avancé si loin avec ces gens à projets en escomptant leurs lettres de change, il les réduirait infailliblement à la nécessité de faire banque­route, en refusant tout à coup de leur en escompter davantage, et qu’alors leur ruine pourrait peut-être aussi entraîner la sienne. Dans une position si critique, il se trouvait obligé, pour son intérêt et sa propre sûreté, de leur continuer le crédit pendant quelque temps encore, en tâchant néanmoins de se débarrasser petit à petit, et pour cela en faisant de jour en jour plus de difficultés sur les escomptes, afin de forcer par degrés ces emprunteurs à avoir recours ou à d’autres banquiers, ou à d’autres moyens de faire de l’argent, en sorte qu’il pût se dégager de leurs filets le plus tôt possible. Les difficultés donc que la banque d’Angleterre, que les principaux banquiers de Londres, et même que les banques écossaises les plus prudentes commencèrent à apporter aux escomptes, au bout d’un certain temps et après s’être déjà toutes trop aventurées, non-seulement jetèrent l’alarme parmi les gens à projets, mais même excitèrent leur fureur au dernier point. Leur propre détresse, dont sans contredit la réserve prudente et indispensable des banques fut l’occasion immédiate, ils l’appelèrent détresse natio­nale, et cette détresse nationale, il ne fallait l’attribuer, disaient-ils, qu’à l’ignorance, à la pusillanimité et à la conduite malhonnête des banques qui refusaient de donner des secours assez étendus aux belles entreprises des hommes de génie, à des entreprises faites pour augmenter l’éclat, la prospérité, l’opulence nationale. Le devoir des ban­ques.. à ce qu’ils semblaient s’être imaginé, était de leur prêter pour un aussi long temps et pour d’aussi fortes sommes qu’ils pouvaient désirer d’emprunter. Néanmoins les banques, en refusant ainsi de donner plus de crédit à des gens à qui elles n’en avaient déjà que beaucoup trop accordé, prirent le seul moyen qui leur restât pour sauver ou leur propre crédit, ou le crédit public de leur pays.
Au milieu de cette détresse et de ces clameurs, il s’éleva en Écosse une banque nouvelle, établie exprès pour remédier aux maux dont le pays était menacé[18]. Le des­sein était généreux, mais l’exécution en fut imprudente, et on ne sentit peut-être pas très-bien quelles étaient la nature et les causes des maux auxquels on voulait porter remède. Cette banque fut plus facile pour accorder des comptes courants ou pour escompter des lettres de change, qu’aucune banque ne l’avait jamais été. Quant à ces lettres, il paraît qu’elle ne faisait presque aucune différence entre les lettres de change sérieuses et les lettres circulantes, mais qu’elle les escomptait toutes indistinctement. Cette banque affichait hautement pour principe, d’avancer, sur des sûretés raison­na­bles, la totalité du capital des entreprises dont les rentrées sont les plus lentes et les plus éloignées, telles que celles qui consistent à améliorer des terres. On disait même que l’encouragement de pareilles améliorations était l’intention capitale de l’esprit de patriotisme qui avait dirigé l’institution de cette banque. Cette grande facilité à accor­der des comptes courants et à escompter des lettres de change donna lieu, comme on peut croire, à une immense émission de billets. Mais ces billets étant, pour la plupart, en excédant de ce que la circulation du pays pouvait absorber et tenir employé, ils refluèrent vers la banque, pour y être convertis en or et en argent, tout aussi vite qu’ils étaient émis. Dès l’origine, la caisse de cette banque fut mal fournie. Le capital des actionnaires, réglé par deux souscriptions différentes, devait s’élever à une somme de 160,000 livres ; mais les fonds effectivement versés ne dépassèrent pas 80 pour 100 de cette somme. La souscription devait être payée par les soumissionnaires, en plusieurs payements. Une grande partie de ceux-ci, en faisant leur premier payement, ouvrirent un compte courant avec la banque, et les directeurs, se croyant obligés de traiter leurs propres capitalistes avec la même générosité qu’ils traitaient toutes les autres person­nes, permirent à beaucoup d’entre eux d’emprunter sur leur compte courant ce qu’ils payaient à la banque pour les termes subséquents de leurs soumissions. Ainsi ces payements ne faisaient que mettre dans un des coffres de la banque ce qu’on venait d’ôter d’un autre. Mais quand même les coffres de cette banque auraient été beaucoup mieux fournis, son excessive émission de papier les aurait si promptement vidés, qu’aucun expédient n’eût pu suffire à les tenir assez garnis, si ce n’est l’expédient ruineux de tirer sur Londres et, à l’échéance de la lettre, de la payer avec intérêts et commission, par le moyen d’une autre traite sur la même place. Les coffres de cette banque ayant été aussi peu remplis dès l’origine, on dit qu’elle s’est vue réduite à cette ressource très-peu de mois après qu’elle eut commencé ses opérations. Les propriétés foncières des actionnaires de la banque valaient plusieurs millions, et au moyen de leur signature dans l’acte de société originaire de la banque, ces propriétés se trou­vaient réellement hypothéquées à l’exécution de tous les engagements pris par elle. Le grand crédit que lui donna nécessairement une hypothèque aussi étendue la mit en état, malgré sa conduite trop facile, de tenir encore pendant plus de deux ans. Quand elle fut obligée d’arrêter ses opérations, elle avait pour environ 200,000 livres de ces billets en circulation. Pour soutenir la circulation de ces billets, qui lui revenaient sans cesse aussitôt qu’ils étaient émis, elle avait constamment fait usage de la pratique de tirer des lettres de change sur Londres, dont le nombre et la valeur allèrent toujours en augmentant, et qui s’élevaient, au moment où elle ferma, à plus de 600,000 livres. Ainsi, dans un espace d’un peu plus de deux ans, cette banque avança à différentes personnes au-delà de 800 000 livres à 5 pour 100. Sur les 200,000 livres qui cir­culaient en billets, ces 5 pour 100 peuvent être regardés peut-être comme un gain net, sans autre déduction que les frais d’administration ; mais sur plus de 600,000 livres, pour lesquelles elle avait été sans cesse obligée de tirer des lettres sur Londres, elle avait à payer, en intérêts et en droits de commission, plus de 8 pour 100 et, par con­séquent, elle se trouva en perte de plus de 3 pour 100 sur les trois quarts au moins des affaires qu’elle avait faites.
Les opérations de cette banque paraissent avoir produit les effets directement opposés à ceux que se proposaient les spéculateurs qui l’avaient projetée et établie. Leur intention, à ce qu’il semble, était de soutenir les belles et grandes entreprises (car ils les regardaient comme telles) qu’on avait formées à cette époque, en différents endroits du pays, et en même temps en attirant à eux la totalité des affaires de banque, de supplanter toutes les autres banques d’Écosse, et en particulier celle d’Édimbourg, qui avait excité du mécontentement par les difficultés qu’elle apportait à l’escompte des lettres de change. Cette banque donna sans contredit quelque soulagement mo­men­tané aux spéculateurs, et les mit à même de pousser leurs entreprises environ deux ans encore plus loin qu’ils n’auraient pu faire sans elle. Mais par là elle ne fit que leur donner le moyen de grossir d’autant la masse de leurs dettes, de manière que quand la crise arriva, le poids de ces dettes retomba avec une nouvelle charge sur eux et sur leurs créanciers. Ainsi les opérations de cette banque, loin d’alléger les maux que ces spéculateurs avaient attirés sur eux-mêmes et sur leur pays, ne fit dans la réalité que les aggraver, en en ralentissant l’effet. Il aurait beaucoup mieux valu, pour ces gens-là, pour leurs créanciers et pour leur pays, que la plupart d’entre eux eussent été obligés de s’arrêter deux ans plus tôt qu’ils ne le firent. Cependant le soulagement momentané que la banque offrit à ces mauvais débiteurs en apporta un réel et durable aux autres banques écossaises. Tous ces particuliers, qui travaillaient à l’aide de ces lettres de change circulantes que les autres banques commençaient pour lors à es­comp­ter de si mauvaise grâce, eurent recours à la nouvelle banque, qui les reçut à bras ouverts. Ainsi ces autres banques trouvèrent une issue pour se dégager en assez peu de temps à ce cercle fatal dont elles n’auraient pu guère sortir autrement, à moins de s’exposer à des pertes considérables, et peut-être même aussi de compromettre un peu leur crédit.
Ainsi, à la longue, les opérations de cette banque augmentèrent les véritables embarras du pays auquel elle prétendait porter du secours, et elles tirèrent réellement d’un très-grand embarras les banques rivales qu’elle se flattait de supplanter.
Quand cette banque commença ses opérations, certaines personnes pensaient qu’avec quelque promptitude que ses coffres se vidassent, elle pourrait toujours les remplir aisément en faisant de l’argent sur les sûretés qu’elle s’était fait donner par ceux à qui elle avait avancé son papier. Mais je crois que l’expérience n’a pas tardé à les convaincre qu’une pareille méthode de faire de l’argent était infiniment trop lente pour un tel objet, et que pour tenir pleins des coffres qui avaient été si mal remplis dans l’origine, et qui se vidaient si rapidement, il n’y avait pas d’autre moyen que l’expédient ruineux de tirer les lettres de change sur Londres et, lors de l’échéance, de les payer avec l’intérêt et la commission accumulés, par le moyen d’autres traites sur la même place. Mais, quand même on supposerait que, par cette autre méthode d’em­prun­ter sur des sûretés, la banque eût pu faire de l’argent aussi vite que ses besoins l’exigeaient, il en serait toujours résulté qu’au lieu de donner un profit, chacune de ces opérations aurait été pour elle un article de perte, de sorte qu’à la longue elle se serait nécessairement ruinée en tant que compagnie de commerce, quoique peut-être pas aussi promptement qu’en se servant de la pratique bien plus coûteuse encore des traites renouvelées. En effet, elle ne pouvait toujours rien gagner pour l’intérêt de son papier, puisque ce papier étant en excédent de ce que la circulation du pays pouvait absorber et tenir employé, il lui serait toujours revenu, pour être converti en espèces, aussi vite qu’elle l’aurait émis, tandis que, pour satisfaire au remboursement de ce papier, elle aurait été obligée d’emprunter sans cesse de l’argent. Au contraire, toutes les charges de l’emprunt, la dépense des agents qu’elle aurait entretenus pour chercher des prêteurs, celle de la négociation avec ces prêteurs, le coût des actes et délégations convenables, tous ces frais seraient tombés sur elle, et auraient formé, dans la balance de ses comptes, autant d’articles à porter au compte des pertes. Le projet de remplir la caisse de la banque de cette manière pourrait se comparer à celui d’un homme qui aurait un bassin dont il sortirait continuellement un courant d’eau, sans aucun courant pareil qui s’y déchargeât, mais qui se proposerait de tenir son bassin toujours égale­ment plein, à l’aide d’une quantité de gens occupés à aller sans cesse tirer de l’eau à un puits, à quelques milles de distance.
Enfin, quand même une telle opération eût pu être non-seulement praticable, mais même profitable à la banque en tant que compagnie de commerce, il n’en serait enco­re résulté aucun avantage pour le pays, mais au contraire il en aurait éprouvé une perte très-considérable. Une pareille opération n’aurait certainement augmenté en rien la quantité d’argent à prêter. Elle n’aurait fait autre chose que d’ériger cette banque en une sorte de bureau général de prêt pour tout le pays. Ceux qui auraient eu besoin d’em­­prunter auraient été obligés de s’adresser à cette banque, au lieu de s’adresser dir­ec­tement aux capitalistes prêteurs de la banque. Mais une banque qui prête de l’argent à peut-être cinq cents personnes différentes, dont la plus grande partie ne peut être que très-peu connue des directeurs, n’est vraisemblablement pas dans le cas de choisir plus judicieusement ses débiteurs, que ne le fera un particulier qui prête son argent dans un petit cercle de gens de sa connaissance, et à ceux en qui il voit une conduite sage et économe qui lui donne de justes motifs de confiance. Les débiteurs d’une banque telle que celle dont je viens d’exposer la conduite ne seraient vraisembla­ble­ment, pour la plupart, que des gens à projets chimériques, des tireurs de lettres de change circulantes, n’empruntant d’argent que pour l’employer en entreprises extrava­gantes que probablement ils ne seraient jamais en état de mettre à fin, quelque secours qu’on pût leur donner, et qui, en supposant même qu’elles fussent mises à fin, ne rendraient jamais la dépense qu’elles auraient coûtée, ne fourniraient jamais un fonds capable d’entretenir une aussi grande quantité de travail que celle qu’elles auraient consommée. Au contraire, les débiteurs sages et économes des particuliers seraient vraisemblablement disposés à employer l’argent par eux emprunté à des entreprises prudentes, proportionnées à leurs capitaux, et qui, tout en tenant moins du grand et du merveilleux, auraient offert plus de solidité et plus de bénéfice, qui auraient rendu avec un gros profit tout ce qu’on y aurait versé, et qui ainsi auraient fourni un fonds capable d’entretenir une beaucoup plus grande quantité de travail que celle qu’on aurait employée à les mettre à fin. Par conséquent, le succès d’une telle opération de la part de la banque, sans ajouter la plus petite chose au capital du pays, n’aurait fait qu’en détourner une grande partie, pour la verser dans des projets téméraires et désa­vantageux, au lieu de la laisser aller à des entreprises sages et profitables.
L’opinion du fameux Law était que l’industrie languissait en Écosse, faute d’argent pour la mettre en activité. Il proposa de remédier à ce manque d’argent par l’établisse­ment d’une banque d’une espèce particulière, qui aurait, à ce qu’il paraît, émis du papier jusqu’à concurrence de la valeur de toutes les terres du pays. Il proposa d’abord son projet au parlement d’Écosse, qui ne jugea pas à propos de l’accueillir. Le duc d’Orléans, alors régent de France, l’adopta ensuite avec quelques modifications. L’idée de la possibilité de multiplier le papier-monnaie presque sans bornes fut la véritable base de ce qu’on appela le système du Mississipi, le projet de banque et d’agiotage le plus extravagant peut-être qui ait jamais paru au monde. Les différentes opérations de ce système ont été développées avec tant de clarté et d’étendue, avec tant d’ordre et de sagacité par M. Duverney, dans son Examen des réflexions politiques sur le commer­ce et les finances de M. Dutot[19], que je n’en rendrai ici aucun compte. Les principes qui furent la base de ce système ont été exposés par M. Law lui-même, dans un Discours sur le commerce et sur l’argent qu’il publia en Écosse, quand il y proposa d’abord son projet. Les idées magnifiques mais imaginaires, qu’on trouve dans cet ouvrage et dans quelques autres, écrits dans les mêmes principes, font encore impression sur beau­coup de gens, et ont peut-être contribué en partie à cette fureur de faire la banque, dont on s’est plaint dernièrement en Écosse et ailleurs.
La plus grande banque de circulation de l’Europe, c’est la banque d’Angleterre. Elle a été érigée en corporation[20], en exécution d’un acte du Parlement, par une charte du grand sceau, en date du 27 juillet 1694. À cette époque elle avança au gouverne­ment une somme de 1,200,000 livres moyennant une annuité de 100,000 livres, c’est-à-dire, 96,000 livres d’intérêt annuel, sur le pied de 8 pour 100, et 4,000 livres par an pour frais de régie[21]. Il est à croire que le nouveau gouvernement établi par la révo­lution, avait peu de crédit, pour être obligé d’emprunter à un si haut intérêt.
En 1697, il fut permis à la banque d’augmenter son capital d’un nouveau fonds de 1,100,171 livres 10 schellings. Ainsi son capital entier s’élevait, à cette époque, à 2,201,181 livres 10 schellings. Cette augmentation de fonds fut faite, dit-on, pour soutenir le crédit public. En 1696, les coupons[22] avaient perdu 40, 50 et 60 pour 100, et les billets de banque 20 pour 100[23]. Pendant la grande refonte de l’argent qui se faisait à cette époque, la banque avait jugé à propos de cesser le payement de ses billets, ce qui nécessairement les avait discrédités.
En exécution du statut de la septième année de la reine Anne, ch. 7, la banque versa dans l’Échiquier, à titre d’avance, la somme de 400,000 livres, ce qui fit en tout 1,600,000 livres avancées sur son annuité originaire de 96,000 livres d’intérêt, et 4,000 livres pour frais de régie. Ainsi, en 1708, le crédit du gouvernement était aussi bon que celui des particuliers, puisqu’il pouvait emprunter à l’intérêt de 6 pour 100, taux légal et taux ordinaire de la place à cette époque. En exécution du même acte, la banque annula pour 1,775,027 livres 17 schellings 10 deniers 1/2 de billets de l’Échi­quier moyennant un intérêt de 6 pour 100, et obtint en même temps la permission d’ouvrir des souscriptions pour doubler son capital. Ainsi, en 1708, le capital de la banque s’élevait à 4 402 343 livres sterling, et elle avait avancé au gouvernement la somme de 3 375 027 livres sterling 17 schellings 10 deniers 1,2.
Par un appel de fonds de 15 pour 100, en 1709, il lui fut versé 656,204 livres 1 schelling 9 deniers, et par un autre de 10 pour 100, en 1710, il lui fut versé 501 448 livres 12 schellings 11 deniers ; ainsi, en conséquence de ces deux appels de fonds, le capital de la banque se trouva élevé à 5,559,993 livres 14 schellings 8 deniers.
En exécution du statut de la troisième année de Georges 1er, ch. 8, la banque retira pour 2 millions de billets de l’Échiquier, qui furent éteints. Elle avait donc, à cette époque, avancé au gouvernement 5,375,027 livres 17 schellings 10 deniers 1/2.
En exécution du statut de la huitième année du même règne, ch. 21, la banque acheta une portion des fonds de la Compagnie de la Mer du Sud, s’élevant à 4 millions ; et en 1722, par le fait des souscriptions qu’elle avait reçues pour se mettre en état de faire cette acquisition, son capital se trouva augmenté de 3,400,000 livres. À cette époque donc, la banque avait avancé à l’État 9,375,027 livres 17 schellings 10 deniers 1/2, et son capital ne s’élevait qu’à 8,959,995 livres 14 schellings 8 deniers. Ce fut dans cette occasion que la somme avancée à l’État par la banque, et dont elle recevait un intérêt, commença pour la première fois à excéder son capital ou la somme pour laquelle elle payait un dividende aux propriétaires des fonds de la ban­que, ou bien, en d’autres termes, que la banque commença à avoir un capital sans dividende, outre le capital qui en donnait un. Elle a toujours continué depuis à avoir ainsi un capital sans dividende. En 1746, la banque avait avancé à l’État, en diffé­rentes circonstances, 11,686,800 livres, et son capital à dividende avait été porté, par différents appels de fonds et souscriptions, à 10,780,000 livres. Ces deux sommes sont toujours restées depuis dans le même état. En exécution du statut de la quatrième année de Georges III, ch. 25, la banque a consenti à payer au gouvernement, pour le renouvellement de sa charte, 110,000 livres sans intérêt ni remboursement ; ainsi cette somme n’a rien ajouté à aucune des deux sommes ci-dessus[24]. 
Le dividende de la banque a varié suivant les variations du taux de l’intérêt qu’elle a reçu à différentes époques, pour l’argent qu’elle avait avancé au gouvernement, ainsi que d’après d’autres circonstances. Le taux de cet intérêt a été successivement réduit de 8 à 3 pour 100. Pendant les dernières années, le dividende de la banque a été à 5 1/2 pour 100[25].
La banque d’Angleterre est aussi solide que le gouvernement lui-même. Il faut que tout ce qu’elle a avancé à l’État soit perdu avant que ses créanciers puissent avoir à craindre la moindre perte. Aucune autre compagnie de banque ne peut être établie en Angleterre par acte du Parlement, ou ne peut être composée de plus de six membres. Elle agit, non-seulement comme une banque ordinaire, mais comme une grande ma­chine de gouvernement. Elle reçoit et paye la majeure partie des annuités qui sont dues aux créanciers de l’État[26], elle met en circulation des billets de l’échiquier, et elle avance au gouvernement le montant annuel de la taxe foncière et de la taxe sur la drêche, qui ne sont ordinairement acquittées que quelques années plus tard. Dans ces différentes opérations, ses engagements envers l’État ont pu quelquefois l’obliger à surcharger la circulation de papier-monnaie, sans qu’on puisse imputer aucune faute à ses directeurs. Elle escompte aussi les lettres de change du commerce, et elle a, en plusieurs occasions différentes, soutenu le crédit des principales maisons, non-seulement d’Angleterre, mais de Hambourg et de Hollande. On dit qu’en 1763, elle avança pour cet objet, en une semaine, 1,600,000 livres, dont une grande partie en lingots. je ne prétends cependant pas garantir l’exactitude du fait, quant au temps ni quant à la somme. Dans d’autres circonstances, cette grande compagnie s’est vue réduite à la nécessité de faire ses payements en pièces de six pence[27][28]. 
Si les opérations les plus sages des banques peuvent augmenter l’industrie dans un pays, ce n’est pas qu’elles y augmentent le capital, mais c’est qu’elles rendent active et productive une plus grande partie de ce capital, que celle qui l’aurait été sans elle. Cette portion de capital qu’un marchand est obligé de garder par devers lui, en espèces dormantes, pour faire face aux demandes qui surviennent est un fonds mort qui, tant qu’il reste dans cet état, ne produit rien ni pour lui ni pour le pays. Les opérations d’une banque sage le mettent à portée de convertir ce fonds mort en un fonds actif et productif, en matières propres à exercer le travail, en outils pour le faciliter et l’abréger, et en vivres et subsistances pour le salarier ; en capital enfin qui produira quelque chose pour ce marchand et pour son pays. La monnaie d’or et d’ar­gent qui circule dans ce pays, et par le moyen de laquelle le produit des terres et du travail de ce pays est annuellement mis en circulation et distribué aux consomma­teurs auxquels il appartient, est aussi, tout comme l’argent comptant du négociant, un fonds mort en totalité. C’est une partie très-précieuse du capital du pays, qui n’est point productive[29]. Les opérations d’une banque sage, en substituant du papier à la place d’une grande partie de cet or et de cet argent, donnent le moyen de convertir une grande partie de ce fonds mort en un fonds actif et productif, en un capital qui produira quelque chose au pays[30]. L’or et l’argent qui circulent dans un pays peuvent se comparer précisément à un grand chemin qui, tout en servant à faire circuler et conduire au marché tous les grains et les fourrages du pays, ne produit pourtant pas lui-même ni un seul grain de blé ni un seul brin d’herbe. Les opérations d’une banque sage, en ouvrant de quelque manière, si j’ose me permettre une métaphore aussi hardie, une espèce de grand che­min dans les airs, donnent au pays la facilité de convertir une grande partie de ses grandes routes en bons pâturages et en bonnes terres à blé, et d’augmenter par là, d’une manière très-considérable, le produit annuel de ses terres et de son travail. Il faut pourtant convenir que si le commerce et l’industrie d’un pays peuvent s’élever plus haut à l’aide du papier-monnaie, néanmoins, suspendus ainsi, si j’ose dire, sur ces ailes d’Icare, ils ne sont pas tout à fait aussi assurés dans leur marche que quand ils portent sur le terrain solide de l’or et de l’argent[31]. Outre les accidents auxquels les expose l’impéritie des directeurs de ce papier-monnaie, ils sont encore sujets à en essuyer plusieurs autres dont la prudence ou l’habileté de ces directeurs ne saurait les garantir.
Par exemple, une guerre malheureuse dans laquelle l’ennemi se rendrait maître de la capitale, et par conséquent de ce trésor qui soutenait le crédit du papier-monnaie, occasionnerait de bien plus grands désordres dans un pays où toute la circulation serait établie sur du papier, que dans un pays où la plus grande partie le serait sur l’or et l’argent. L’instrument habituel du commerce ayant perdu sa valeur, on ne pourrait plus faire d’échanges que par troc ou à crédit. Tous les impôts ayant été payés habi­tuellement en papiers, le prince n’aurait plus de fonds pour payer ses troupes ni pour approvisionner ses magasins, et le pays se trouverait dans une situation bien plus désespérée que si la plus grande partie de sa circulation eût consisté en or et en ar­gent. Un prince jaloux de maintenir, dans tous les temps, ses États dans la position où il peut le plus facilement les défendre, doit, sous ce point de vue, les tenir en garde, non-seulement contre cette multiplication excessive de papier-monnaie, qui est funes­te, même aux banques qui l’ont produite, mais même contre ce degré de multiplication qui les met à même de remplir avec du papier la majeure partie de la circulation du pays[32]. 
On peut regarder la circulation d’un pays comme divisée en deux branches différentes : la circulation qui se fait entre commerçants seulement, et la circulation entre les commerçants et les consommateurs. Quoique les mêmes pièces de monnaie, soit papier, soit métal, puissent être employées tantôt dans l’une de ces deux branches de circulation et tantôt dans l’autre, cependant, comme ces deux branches marchent constamment en même temps, chacune d’elles exige un certain fonds de monnaie, d’une espèce ou de l’autre, pour la faire marcher. La valeur des marchandises qui circulent entre les différents commerçants ne peut jamais excéder la valeur de celles qui circulent entre les commerçants et les consommateurs, tout ce qui est acheté par les gens de commerce étant en définitive destiné à être vendu aux consommateurs. La circulation des gens de commerce entre eux, portant sur des ventes en gros, exige en général une somme bien plus grosse pour chaque transaction particulière. Celle entre les commerçants[33] et les consommateurs, au contraire, portant en général sur des ventes en détail, n’exige fort souvent que de très-petites sommes ; 1 schelling, ou même un demi-penny étant quelquefois tout ce qu’il faut. Or, les petites sommes circulent beau­coup plus vite que les grosses. Un schelling change plus souvent de maître qu’une guinée, et un demi-penny plus souvent qu’un schelling. Ainsi, quoique les achats annuels de tous les consommateurs soient au moins égaux en valeur à ceux de tous les gens de commerce, néanmoins ils peuvent, en général, se faire avec une masse de monnaie beaucoup plus petite, les mêmes pièces, au moyen d’une circulation plus rapide, servant d’instruments à beaucoup plus d’achats de la première espèce que de la seconde.
On peut régler le papier-monnaie de manière, ou à le restreindre presque tout à fait à la seule circulation entre les différents commerçants, ou à l’étendre à une grande partie de celle qui a lieu entre les commerçants et les consommateurs. Quand il ne circule pas de billet de banque au-dessous de la valeur de 10 livres, comme à Lon­dres, le papier-monnaie se trouve presque absolument restreint à la circulation entre les commerçants. Quand un billet de banque de 10 livres vient dans les mains d’un consommateur, celui-ci est en général obligé de le changer à la première boutique où il aura occasion d’acheter pour 5 schellings de marchandise, de manière que souvent ce billet revient dans les mains d’un commerçant avant que le consommateur ait dépensé la quarantième partie de la somme. Quand il y a en circulation des billets de très-petites sommes, des billets de 20 schellings, comme en Écosse, le papier-monnaie se répand dans une grande partie de la circulation entre les commerçants et les con­som­mateurs, Il en remplissait encore une bien plus grande partie avant l’acte du Parlement qui a arrêté la circulation des billets de 10 et de 5 schellings. Dans la mon­naie courante de l’Amérique septentrionale, il y avait communément en émission du papier de petites sommes jusqu’à 1 schelling, et il remplissait presque la totalité de la circulation. Il en a circulé même de 6 pence[34] dans quelques monnaies de papier du comté d’York. 
Lorsque les billets de sommes aussi petites sont autorisés dans la circulation et sont d’un usage commun, beaucoup de gens du peuple ont l’envie et la possibilité de se faire banquiers. Tel particulier dont les billets pour 5 livres ou même pour 20 schellings ne seraient reçus de personne, viendra à bout de les passer aisément quand ils seront émis pour une somme aussi petite que 6 pence. Mais les banqueroutes fré­quentes qui doivent arriver dans une classe de banquiers aussi misérables peuvent donner lieu à de grands inconvénients, et quelquefois même causer de très-grandes calamités parmi beaucoup de pauvres gens qui ont reçu de tels billets en payement.
Il serait peut-être mieux que, dans aucun endroit du royaume, on ne laissât mettre en émission aucun billet de banque au-dessous de 5 livres sterling. Alors, vraisembla­ble­ment, par tout le royaume le papier-monnaie se trouverait renfermé dans la circu­la­tion entre les différentes personnes employées au commerce, comme il l’est présente­ment à Londres, où il ne circule pas de billet de banque au-dessous de 10 livres ; 5 livres dans la majeure partie du royaume étant une somme à laquelle on regarde autant et qu’on dépense aussi rarement en une fois, que 10 livres à Londres, vu la profusion et la grande dépense qui règnent dans cette ville, quoique peut-être ces 5 livres, dans le reste du royaume, n’achètent guère pour cela plus de moitié des mar­chandises que 10 livres achètent à Londres[35].
Il faut observer que, partout où le papier-monnaie est à peu près concentré dans la circulation de commerçant à commerçant, comme à Londres, il y a toujours abon­dance d’or et d’argent. Partout où ce papier se répand dans la circulation du commer­çant au consommateur, comme cela est en Écosse et encore plus dans l’Amérique septentrionale, il chasse presque tout à fait l’or et l’argent du pays, presque toutes les affaires du commerce intérieur allant ainsi avec du papier. La suppression des billets de banque de 10 et de 5 schellings en Écosse a un peu diminué la rareté des espèces, et une suppression des billets de 20 schellings l’aurait vraisemblablement diminuée encore plus. On dit que l’or et l’argent sont devenus plus abondants en Amérique depuis la suppression de quelques-uns des papiers circulants de ce pays, et on dit qu’aussi ces métaux y étaient plus abondants avant l’établissement de ces sortes de papiers.
En réglant le papier-monnaie de manière à le concentrer presque en entier dans la circulation de commerçant à commerçant, les banques et les banquiers n’en seront pas moins à portée de prêter à peu près les mêmes secours à l’industrie et au Commerce du pays, qu’ils peuvent le faire quand ce papier remplit presque la totalité de la circu­la­tion. L’argent comptant qu’un homme de commerce est obligé de garder par devers lui pour faire face aux demandes du moment, est entièrement destiné à la circulation qui se fait entre lui et les autres gens de commerce dont il achète des marchandises. Il n’a besoin de garder aucun argent comptant pour la circulation qui se fait entre lui et les consommateurs qui se fournissent chez lui, ceux-ci lui apportant de l’argent comp­tant au lieu de lui en ôter. Ainsi, quand même on ne permettrait d’émettre du papier-monnaie qu’en billets de sommes assez fortes pour restreindre ce papier presque entièrement à la circulation de commerçant à commerçant, cependant les banques et les banquiers, en partie par l’escompte des lettres de change sérieuses, et en partie par des avances sur comptes courants, pourraient toujours être dans le cas de soulager la majeure partie de ces commerçants de l’obligation de garder par devers eux une partie un peu importante de leur capital, en espèces dormantes, pour faire face aux deman­des du moment. Les banques seraient toujours dans le cas de donner aux gens de commerce, de quelque genre qu’ils soient, tous les secours que ceux-ci peuvent rai­son­nablement attendre des banques et des banquiers.
Mais, pourra-t-on dire, empêcher des particuliers de recevoir en payement les billets d’un banquier, de quelque somme qu’ils soient, grande ou petite, quand ils veu­lent bien les accepter, ou bien empêcher un banquier de mettre en émission de pareils billets, quand tous ses voisins consentent à les recevoir, c’est une atteinte manifeste à cette liberté naturelle que la loi a pour objet principal de protéger et non pas d’enfrein­dre. Sans contredit les règlements de ce genre peuvent être regardés, à quelques égards, comme une atteinte à la liberté naturelle ; mais l’exercice de la liberté naturelle de quelques individus, qui pourrait compromettre la sûreté générale de la société, est et doit être restreint par les lois, dans tout gouvernement possible, dans le plus libre comme dans le plus despotique. L’obligation imposée de bâtir des murs mitoyens pour empêcher la communication du feu, est une violation de la liberté naturelle, pré­cisément du même genre que les règlements que nous proposons ici pour le commerce de banque.
Un papier-monnaie qui consiste en billets de banque émis par des gens du crédit le plus solide, payables à la première demande et sans condition quelconque, et payés effectivement toujours comptant à l’instant de la présentation[36], est à tous égards d’une valeur égale à la monnaie d’or et d’argent, puisqu’à tout moment on peut en faire de la monnaie d’or et d’argent. Tout ce qui se vend ou s’achète avec ce papier doit néces­sairement se vendre et s’acheter à aussi bon marché qu’avec de l’or et de l’argent[37]. 
L’augmentation de monnaie résultant du papier, a-t-on dit, en multipliant la quan­tité de monnaie courante et conséquemment, en diminuant la valeur de la masse totale de cette monnaie, augmente nécessairement le prix pécuniaire des marchan­dises. Mais, comme la quantité d’or et d’argent qu’on ôte de la circulation est toujours égale à la quantité de papier qu’on y ajoute, le papier-monnaie n’augmente pas néces­saire­ment la somme de monnaie courante. Depuis le commencement du siècle dernier jusqu’à aujourd’hui, les vivres n’ont jamais été en Écosse à aussi bon marché qu’en 1759, quoiqu’il y eût alors dans le pays plus de papier-monnaie qu’à présent, à cause de la circulation des billets de banque de 10 et de 5 schellings. La proportion du prix des vivres entre l’Écosse et l’Angleterre est aujourd’hui la même qu’elle était avant que les compagnies de banque se fussent si fort multipliées en Écosse. Le blé est presque toujours à tout aussi bon marché en Angleterre qu’en France, quoiqu’il y ait en Angleterre une très-grande quantité de papier-monnaie, et qu’il n’y en ait presque pas en France. En 1751 et 1752, quand M. Hume publia ses Discours politiques, et peu de temps après la grande multiplication du papier-monnaie en Écosse, les vivres avaient haussé dans ce pays d’une manière très-sensible, vraisemblablement à cause des mau­vaises années, et non point à cause de la multiplication du papier-monnaie.
À la vérité, il en serait autrement d’un papier-monnaie qui consisterait en billets dont le payement immédiat dépendrait en partie quelconque, soit de la bonne volonté de ceux qui les auraient émis, soit d’une condition que le porteur n’aurait pas toujours la possibilité de remplir, ou enfin dont le payement ne serait exigible qu’au bout d’un certain nombre d’années, et qui en attendant ne porteraient pas d’intérêt ; sans doute, un pareil papier-monnaie tomberait plus ou moins au-dessous de la valeur de l’or et de l’argent, suivant qu’on supposerait plus ou moins de difficulté ou d’incertitude à en obtenir le payement immédiat, ou suivant le plus ou moins d’éloignement du terme auquel le payement serait exigible[38].
Il y a quelques années que les différentes compagnies de banque d’Écosse étaient dans l’usage d’insérer dans leurs billets de banque ce qu’elles appelaient une clause d’option, par laquelle elles promettaient payer au porteur ou au moment de la présen­tation, ou, à l’option des directeurs, six mois après la présentation, avec l’intérêt légal pendant ledit terme de six mois. Quelquefois, les directeurs de la banque se servaient de cette clause d’option ; quelquefois, quand on leur demandait à échanger en or et en argent une quantité considérable de leurs billets, ils menaçaient les porteurs de se servir contre eux de la clause, à moins qu’ils ne voulussent se contenter d’une partie seulement de la somme demandée. Les billets de ces banques formaient alors la plus grande partie de la monnaie courante d’Écosse, et celle-ci baissa, à cause de l’incerti­tude du payement, au-dessous de la valeur de la monnaie d’or et d’argent. Tant que dura cet abus, qui régna principalement en 1762, 1763 et 1764, le change, qui entre Londres et Carlisle était alors au pair, se trouvait quelquefois en même temps, entre Londres et Dumfries, à 4 pour 100 contre Dumfries, quoique cette ville ne soit pas à trente milles de Carlisle. Mais à Carlisle, les lettres de change se payaient en espèces, tandis qu’à Dumfries elles étaient acquittées en billets des banques d’Écosse, et c’était l’incertitude de pouvoir échanger ces billets contre des espèces qui les avait ainsi fait baisser au-dessous de la monnaie d’or et d’argent. Le même acte du Parlement, qui supprima les billets de banque de 10 et de 5 schellings, supprima aussi cette clause d’option, et rétablit par là le change entre l’Angleterre et l’Écosse à son taux naturel, ou à celui auquel il devait se fixer d’après le cours du commerce et l’état des remises entre ces deux places[39].
Dans les monnaies de papier du comté d’York, le payement d’un aussi petit billet qu’un billet de 6 pence dépendait quelquefois de la condition que le porteur du billet apporterait à la personne qui l’avait émis de quoi changer une guinée ; condition qui pouvait être souvent fort difficile à remplir pour les porteurs de ces billets, et qui a dû faire baisser cette monnaie au-dessous de la monnaie effective. Aussi, un acte du Parlement déclara-t-il toutes ces clauses illégales, et supprima, comme on l’avait fait en Écosse, tout billet payable au porteur, au-dessous de 20 schellings.
Le papier circulant dans l’Amérique septentrionale ne consistait pas en billets de banque payables au porteur à sa demande, mais en papier d’État, dont le payement n’était exigible que plusieurs années après la date de l’émission ; et quoique le gouvernement de la colonie ne payât point d’intérêt aux porteurs de ce papier, il le déclara et le rendit de fait une offre légale de payement pour la valeur entière pour laquelle il avait été créé. Mais en supposant la garantie du gouvernement la meilleure possible, 100 livres payables à quinze ans de terme, par exemple, dans un pays où l’intérêt est à 6 pour 100, ne valent guère plus de 40 livres argent comptant. Ainsi, forcer un créancier à recevoir un pareil effet comme parfait payement pour une dette de 100 livres qui avaient été avancées en argent comptant, c’était une injustice aussi criante peut-être qu’en eût jamais osé risquer tout autre gouvernement ayant quelque prétention à la liberté. Un tel acte d’autorité porte bien les marques évidentes de l’origine que lui donne le sage et honnête docteur Douglas, qui nous assure que ce fut un projet imaginé par des débiteurs de mauvaise foi pour duper leurs créanciers. À la vérité, le gouvernement de Pennsylvanie, à la première émission qu’il fit de papier-monnaie en 1722, prétendit rendre son papier de valeur égale à celle de l’or et de l’argent, en portant des peines contre tous ceux qui feraient quelque différence de prix dans leurs marchandises pour le cas de vente en papier de la colonie, ou de vente en espèces ; règlement tout aussi tyrannique que celui qu’il avait pour objet de soutenir, mais infiniment moins efficace. Une loi positive peut bien faire qu’un schelling soit une offre valable pour le payement d’une dette d’une guinée, parce que cette loi peut enjoindre aux cours de justice de prononcer la décharge du débiteur qui aura fait une telle offre ; mais il n’y a pas de loi positive capable d’obliger un particulier qui vend sa marchandise et qui est maître de la vendre ou de ne la pas vendre, comme il lui plaît, à recevoir, en échange de cette marchandise, un schelling pour l’équivalent d’une guinée. Le change avec la Grande-Bretagne fit bien voir, en dépit de tous les règle­ments de cette espèce, que 100 livres sterling étaient, suivant les circonstances, regar­dées comme valant 130 livres dans certaines colonies, et même la somme énorme de 1,100 livres de la monnaie courante de quelques autres ; cette différence dans la valeur provenant de la différence dans la quantité de papier émis par les diverses colonies, ainsi que de l’époque plus ou moins éloignée et plus ou moins probable de son remboursement définitif.
Il n’y a donc pas de loi plus équitable que l’acte du Parlement, dont on se plaignit si injustement dans les colonies, qui déclara qu’aucun papier qui pourrait être émis par la suite n’aurait la force de monnaie légale dans les payements.
La Pennsylvanie mit toujours plus de modération que nos autres colonies dans l’émission de son papier-monnaie. Aussi dit-on que son papier circulant ne baissa jamais au-dessous de la valeur de la monnaie d’or et d’argent qui avait cours dans la colonie avant la première émission de papier. Avant cette émission, la colonie avait haussé la dénomination de sa monnaie courante, et avait statué, par acte de son assem­blée, que 5 schellings sterling passeraient dans la colonie pour 6 schellings 3 deniers, et ensuite pour 6 schellings 8 deniers. En conséquence, 1 livre courante, argent des colonies, même quand la monnaie y était en or et en argent, était de plus de 30 p. 100 au-dessous de la valeur d’une livre sterling, et quand la monnaie courante y fut convertie en papier, elle ne se trouva guère plus de 30 p. 100 au-dessous de cette même valeur. Le prétexte qui engagea à élever ainsi la dénomination de la monnaie, ce fut de prévenir l’exportation de l’or et de l’argent, en faisant passer dans la circula­tion de la colonie la même quantité de métal pour une somme plus forte que celle pour laquelle cette quantité circulait dans la mère patrie. Il arriva toutefois que le prix de toutes les marchandises venant de la mère patrie augmentât précisément dans la même proportion que les colons avaient élevé la dénomination de leurs monnaies, de manière que leur or et leur argent s’exportent aussi vite qu’auparavant.
Le papier des colonies étant reçu en payement des taxes provinciales pour toute la valeur pour laquelle il avait été créé, il en résulta nécessairement un surcroît de valeur à celle qu’il aurait eue d’après l’éloignement réel ou présumé du terme de son acquitte­ment ou rachat définitif. Ce surcroît de valeur était plus ou moins grand, selon que la quantité de papier en émission se trouvait être plus ou moins au-dessus de ce qui pouvait être employé au payement des taxes de la colonie particulière qui l’avait émis. Dans toutes les colonies, il excéda de beaucoup la quantité qui pouvait être employée de cette manière.
Un prince qui statuerait qu’une certaine portion de l’impôt serait payée en un papier-monnaie d’une certaine espèce, pourrait par là ajouter une valeur à ce papier, quand même le terme de son acquittement et rachat définitif dépendrait entièrement de la volonté du prince. Si la banque qui aurait émis ce papier avait soin d’en tenir toujours la quantité un peu au-dessous de ce qui pourrait aisément être employé de cette manière, la demande en pourrait être telle qu’il gagnât même une prime ou qu’il se vendît au marché quelque chose de plus que la somme numérique pour laquelle il aurait été créé. Il y a quelques personnes qui expliquent d’après ce principe ce qu’on nomme agio dans la banque d’Amsterdam, ou la supériorité qu’a l’argent de banque sur l’espèce courante, quoique cet argent de banque ne puisse, à ce qu’elles préten­dent, être retiré de la banque à la volonté du propriétaire. La plus grande partie des lettres de change étrangères doivent être acquittées en argent de banque, c’est-à-dire en un transfert[40] sur les livres de banque, et les directeurs de la banque, disent-elles, ont soin de tenir la somme totale de l’argent de banque toujours au-dessous de la demande que cette forme de payement occasionne. C’est pour cela, prétendent-elles, que l’ar­gent de banque se vend moyennant une prime, ou qu’il gagne un agio de 4 à 5 pour 100 au-dessus de la même somme nominale en espèces courantes du pays. Cepen­dant, je ferai voir par la suite[41] que cette explication sur la banque d’Amsterdam est en grande partie imaginaire.
Un papier circulant, qui tombe au-dessous de la valeur de la monnaie d’or et d’argent, ne fait pas baisser par là la valeur de l’or et de l’argent, et il ne fait pas que des quantités pareilles de ces métaux viennent pour cela à s’échanger contre une moindre quantité de toute autre marchandise. La proportion entre la valeur de l’or et de l’argent et celle des marchandises de toute autre espèce ne dépend nullement de la nature ou de la quantité de quelque papier-monnaie que ce soit qui circule dans un pays, mais elle dépend toujours de l’état de richesse ou de pauvreté des mines qui se trouvent, à une époque quelconque, approvisionner de ces métaux le grand marché du monde commerçant. Elle dépend de la proportion qui se trouve entre la quantité de travail nécessaire pour faire venir au marché une certaine quantité de ces métaux, et celle qui est nécessaire pour y faire venir une certaine quantité de toute autre espèce de marchandise[42].
En empêchant les banquiers d’émettre aucun billet de banque circulant ou billet au porteur au-dessous d’une certaine somme, et en les assujettissant à l’obligation d’ac­quit­ter ces billets immédiatement et sans aucune espèce de condition, à l’instant de la présentation, on peut après cela, sans craindre de compromettre la sûreté générale, laisser à leur commerce, à tous autres égards, la plus grande liberté possible[43]. La multiplication récente de compagnies de banque, dans toutes les parties des royaumes unis, événement qui a si fort alarmé beaucoup de gens, bien loin de dimi­nuer la sûreté du public, ne fait que l’augmenter. Elle oblige tous ces banquiers à mettre plus de circonspection dans leur conduite ; elle les empêche d’étendre leur émission de billets au-delà de la proportion que comporte l’état de leur caisse, afin de se tenir en garde contre ce reflux de papier que leur suscite malicieusement la rivalité de tant de concurrents toujours prêts à leur nuire ; elle circonscrit la circulation de chaque compagnie particulière dans un cercle plus étroit, et elle restreint leurs billets circulant à un plus petit nombre. En tenant ainsi la circulation divisée en plus de branches différentes, elle fait que la faillite de l’une de ces compagnies, événement qui doit arriver quelquefois dans le cours ordinaire des choses, devient un accident d’une moins dangereuse conséquence pour le public. Cette libre concurrence oblige aussi les banquiers à traiter avec leurs corres­pondants d’une manière plus libérale et plus facile, de peur que leurs rivaux ne les leur enlèvent. En général, dès qu’une branche de commerce ou une division du travail quelconque est avantageuse au public, elle le sera toujours d’autant plus, que la con­currence y sera plus librement et plus généralement établie.
 
 
 
↑ Chapitre vi.
↑ Smith me parait ici confondre le revenu consistant en produits avec le capital. Sou capital circulant ou mobile disparait pendant la production, reparait ensuite, mais ne fait point partie du revenu de la société. C’est le produit sortant de toute ces métamorphoses qui en fait partie.
Sous ce point de vue il n’y a, quoi qu’en dise Smith, aucune différence entre le capital mobile de la société et le capital d’un particulier.
Je sais bien que cette erreur n’entraîne aucune fausse conséquence dans les raisonnements de Smith ; mais elle obscurcit singulièrement son idée.
Note inédite de J.-B. Say.
↑ Toute cette explication est pénible et imparfaite dans Smith, faute par lui d1jwir éclairci la théorie des valeurs. S’il avait dit que le revenu c’est la valeur de ce quoi reçoit en échange des frais de production qu’on fait, il n’aurait pas été obligé de distinguer le revenu net du revenu brut de la société, ce qui est faux. Le revenu de la société est toujours un revenu brut. Le revenu n’est jamais telle matière ou telle autre ; c’est une valeur produite, quelle que soit sa forme.
Note inédite de J.-B. Say.
↑ Cela arrivera ainsi pourvu que la valeur de l’or reste la même qu’auparavant, et que le papier ait la même valeur que l’or. Si la valeur de l’or venait à baisser, ou si a valeur du papier baissait au-dessous de celle de l’or, la quantité de l’un et de l’autre restant dans la circulation serait proportionnellement augmentée.
Mac Culloch.
↑ L’explication que donne ici l’auteur des procédés par lesquels le papier remplace l’or n’est pas tout à fait exacte. Le canal de la circulation, dit-il, est rempli au moyen d’un million, et n’en peut pas contenir davantage ; mais il lui en faudra davantage si la valeur est moindre. Si cette valeur était réduite de moitié, deux millions pourraient circuler avec autant de facilité où naguère un million suffisait. C’est par une réduction proportionnelle de valeur, que la grande inondation de métaux précieux tirés de l’Amérique a pu être contenue dans les anciens canaux de la circulation, et une nouvelle réduction de valeur aurait maintenu dans les mêmes limites une quantité nouvelle de métal. Par une augmentation de papier qui ajoute aux métaux précieux déjà existants dans la circulation, tous les éléments de cette circulation perdront généralement de leur valeur. Dès lors, la portion de ces éléments qui consiste en or et en argent sera envoyée au dehors pour y reprendre son ancienne valeur sur le marché général du monde, et sera remplacée par des émissions de papier. Buchanan.
↑ Voyez l’explication de ce mot, ci-après, chapitre v.
↑ Mac Culloch conteste cette idée de Smith. Selon l’école anglaise, la Dature de la consommation n’importe en rien à la société. Que l’on consomme l’équivalent d’une certaine valeur sous une forme ou sous une autre, cela est indifférent !
↑ Voyez la préface de Rudiman au Receuil des Chartes d’Écosse (Scotiœ Diplomata), par Anderson.
↑ Le comité de la Chambre des communes formé pour présenter un rapport sur les billets au porteur eu Écosse et en Irlande (promissory notes), évalua,en 1825, le montant du papier en circulation dans le royaume d’Écosse, à 3,309,000 livres sterling, dont 2,079,000 livres sterling se composaient de billets au-dessous de cinq livres. Il y a naturellement peu de monnaie d’or en circulation ; on suppose que la valeur réunie des espèces d’or et d’argent ne s’élève pas, en Écosse, au-dessus de 300,000 livres sterling. Il est probable que le montant de la circulation totale de cette partie de l’empire britannique peut être évalué aujourd’hui (1838) à 3,500,000 ou 4,000,000 livres sterling.
Mac Culloch.
↑ De remarquables modifications ont eu lieu depuis la publication de la Richesse des nations dans le nombre, la constitution et l’importance des banques privées. On ne connaît pas exactement le nombre des établissements de ce genre qui existaient en Angleterre et dans le pays de Galles en 1776 ; mais nous pensons qu’il peut être évalué approximativement à 150. En 1792, les banques particulières se montaient à environ 350, sur lesquelles près d’une centaine furent détruites par la révulsion qui s’opéra vers la dernière partie de cette année et le commencement de 1793. En 1800, ces banques excédaient 300, et elles se multiplièrent d’une manière si rapide dans les années suivantes, qu’en 1814 elles avaient atteint le nombre de 940. Il est important à observer que la plupart de ces banques émettaient des billets ; elles approvisionnèrent même la majeure partie de la circulation dans des districts nombreux et étendus par l’importance de leurs émissions, qui formaient une proportion notable de la monnaie du royaume. L’arrêt de 1708, qui prohibait l’admission de plus de six associés dans les établissements émettant des billets payables à vue en Angleterre et dans le pays de Galles, se maintint en vigueur jusqu’en 1826, époque à laquelle, comme on le verra plus loin, il fut rapporté par suite des événements de cette même année et de la précédente.
Il existe une énorme différence dans la manière dont opèrent les banques dans les diverses parties du royaume. Les banques de province en Angleterre accordent un intérêt sur les balances au crédit des comptes qu’elles ont ouverts, et ajoutent, la plupart, une commission à l’intérêt prélevé sur les sommes qu’elles prêtent. Les banquiers particuliers de Londres n’accordent que rarement des intérêts sur les dépôts. Comme ils n’émettent pas de billets en leur nom, ils se trouvent intéressés, non à l’accroissement, mais à la rareté de la circulation. Leurs véritables fonctions consistent à veiller sur les capitaux des autres, et à recevoir ou payer pour eux. Ils ne comptent aucune commission, leurs bénéfices dérivant, comme il a déjà été déterminé, de la balance des comptes de leurs commettants. La plupart des banques écossaises sont à la fois banques de circulation et banques de dépôt. Toutes, indistinctement, reçoivent et payent des intérêts jusque sur des dépôts qui ne dépassent pas 10 livres. Celles d’entre elles qui émettent des billets ne comptent aucune commission au delà du taux normal de l’intérêt sur les sommes qu’elles prêtent ; mais celles qui n’en émettent pas portent quelquefois une commission. En ce moment (1838) les dépôts existants dans les banques Écosse sont présumés surpasser 23,000,000 de livres (653,000,000 de fr.) Mac Culloch.
↑ C’est ce que nous nommons aussi comptes courants.
↑ Aujourd’hui on accorde fréquemment des comptes courants (cash accounts) pour des sommes de 100 à 200 livres sterling seulement.
Mac Culloch.
↑ M. Ricardo a contesté celle proposition. « Des comptes courants, fait-il observer, sont des crédits accordés parles banquiers écossais à leurs clients, en surplus des billets qu’ils escomptent pour eux ; mais comme les banquiers, à mesure qu’ils avancent de l’argent d’un côté et l’envoient en circulation, s’oient le moyen d’en envoyer autant d’un autre côté, on ne voit pas trop en quoi consiste l’avantage. Si un million de papier suffît à la circulation, un million seulement pourra circuler. Il importe peu aux banquiers ou aux marchands que la totalité soit payée en escomptant des billets, ou qu’une partie seulement soit payée ainsi, et que le reste soit émis par le moyen de ces comptes courants. (Principles of political Economy, 1re édit., p. 515.)
L’avantage d’un compte courant ne consiste pas réellement en ce qu’il permet à un banquier d’étendre ses avances à ses clients ; mais il consiste dans l’extrême facilité avec laquelle il permet de faire ces avances, et c’est ce qui a échappé à M. Ricardo…
Il est singulier toutefois que, malgré l’avantage des comptes courants, le moulant des billets des banques d’Écosse, émis par le moyen des comptes courants, soit très-faible, la huitième ou dixième partie seulement des billets émis pour l’escompte des billets.
Voyez le témoignage de H. Gilechrist, dans l’appendice au rapport des Communes sur les moyens de reprendre les payements en espèces en 1819, p. 216 ; voyez aussi le témoignage de M. Kinnear, dans l’appendice au rapport du comité sur let billets d’Écosse en 1826, p. 140. Mac Culloch.
↑ La grande dépense à laquelle la Banque fut réduite à cette époque ne fui pas occasionnée, comme parait le croire Smith, par des émissions imprudentes de billets, mais par le trouble existant dans les bases de la circulation et par le haut prix du lingot qui en était la conséquence. La Banque n’ayant d’autre moyen de se procurer des guinées que l’envoi de lingots à la Monnaie pour les faire frapper, était constamment forcée de battre monnaie pour rembourser ses billets ; et comme les espèces anciennes manquaient généralement de poids, et que le prix du lingot était élevé en proportion, il devint profitable de tirer de la Banque de lourdes guinées neuves en échange de son papier, et de les échanger avec bénéfice contre des billets qui revenaient se convertir en guinées sans cesse revendues au détriment de la Banque. Cet inconvénient a toujours lieu lorsque la monnaie existante est usée par le frai, et qu’il y a profit à échanger le papier contre des espèces nouvellement frappées. Buchanan.
↑ Cette pratique se nomme, dans quelques places de France, faire la navette. On appelle aussi papier de circulation les traîtes factices ainsi obtenues.
↑ M. H. Thomton a montré, dans son Essay on paper credit, p. 34, que le docteur Smith a exagéré les frais qu’un négociant encourt en se procurant de l’argent par circulation. Selon lui, les transactions opérées au moyen de billets artificiels se font à l’avantage des deux parties. Quand A tire sur B il lui paye une commission ; mais lorsque B devient le tireur, c’est A qui reçoit la commission. On peut supposer qu’ils reçoivent autant qu’ils donnent sous forme de commission, et selon M. Thornton, il n’en coûte que l’escompte à payer pour convertir ces billets en argent, non compris les frais de timbre.
Mac Culloch.
↑ La méthode décrite dans le texte n’est ni la plus ordinaire ni la plus dispendieuse qu’employèrent ces gens à projets, pour lever de l’argent par circulation. Il arrivait souvent que A Édimbourg voulait mettre B de Londres en état de payer la première lettre Je change, en tirant, peu de jours avant l’échéance de celle-ci, une seconde lettre à trois mois de date, sur le même B de Londres. Cette lettre étant payable a l’ordre du tireur, A la négociait à Édimbourg, au pair, et avec le montant il achetait des lettres sur Londres, payables à vue à l’ordre de B, à qui il les faisait passer par la poste. Vers la fin de la dernière guerre, le change entre Édimbourg et Londres fut souvent à 3 pour 100 contre Édimbourg, et souvent A était obligé de payer cette prime en achetant ces lettres de change à vue. Cette opération donc, étant répétée quatre fois dans le cours de l’année, et chargée d’une commission d’au moins 1/2 pour 100 à chaque répétition, a dû coûter à A, à cette époque, au moins 14 pour 100 pour l’année. D’autres fois A voulait mettre B en état d’acquitter la première lettre de change, en tirant, peu de jours avant l’échéance de celle-ci, une seconde lettre à deux mois de date, non sur B, mais sur quelque tierce personne, par exemple sur C de Londres. Cette autre lettre de change était faite payable à l’ordre de B, qui, après l’avoir fait accepter par C, l’escomptait auprès de quelque banquier de Londres, et A mettait C en état d’acquitter cette lettre, en tirant, quelques jours avant l’échéance, une troisième lettre, aussi à deux mois de date, tantôt sur son premier correspondant B, tantôt sur quelque quatrième ou cinquième personne, par exemple, sur D ou E. Cette troisième lettre était faite payable à l’ordre de C, lequel, aussitôt qu’elle était acceptée, la faisait escompter de la même manière chez quelques banquiers de Londres. De telles opérations étant répétées au moins six fois dans le cours de l’année, et chargées d’un droit de commission d’au moins 1/2 pour 100 sur chaque renouvellement, y compris l’intérêt légal de 5 pour 100, cette méthode de faire de l’argent, de même que celle qui est décrite dans le texte, doit avoir coûté à A plus de 8 pour 100. Néanmoins, en épargnant le change d’entre Édimbourg et Londres, elle était moins dispendieuse que celle mentionnée dans la première partie de cette note ; mais alors elle exigeait un crédit établi avec plus d’une seule maison de commerce à Londres, avantage que la plupart de ces faiseurs de projets ne pouvaient pas toujours aisément se procurer. Note de l’auteur.
↑ Cette banque commença ses opérations à Ayr, en novembre 1769, sous la raison Douglas, Héron et comp., et elle avait des succursales à Édimbourg et à Dumfries. Elle suspendit ses payements le 25 juin 1772. À cette époque, quoique son capital réalisé ne fût que de 128,000 livres sterling (80 pour 100 de 100,000 livres sterling), elle avait encouru des obligations pour la somme énorme de 800,000 livres sterling, c’est-à-dire 600,000 livres sterling de dettes accumulées à Londres, et 200,000 livres sterling de billets en circulation. La grande fortune privée des sociétaires leur permit à la fin de répondre il toutes les réclamations sur la banque. La perle totale encourue avant la suspension totale de ses opérations s’élevait, dit-on, à 400,000 livres sterling. — Voyez l’ouvrage intitulé : Precipitation and fall of Messrs. Douglas, Heron et comp. ; in-4°, Edimburgh, 1778. Mac Culloch.
↑ Le livre de M. Duverney a été imprimé en France en 1740, en 2 vol. in-12 ; celui de M. Dutot l’a été, pour la première fois, en 1738, aussi en 2 vol. in-12. Ce dernier était une réfutation de l’ouvrage de M. Melon, imprimé en 1736.
↑ Formalité qui a pour objet de donner à une association ou compagnie une existence civile et légalement reconnue : ce qui se faisait en France par des lettres patentes enregistrées dans les cours.
↑ Dans les finances d’Angleterre l’usage est, quand on fonde une dette, de fonder en même temps une somme pour les frais de régie.
↑ Tallies ou tallies : on nomme ainsi ces effets, parce qu’ils consistent en deux morceaux coupés l’un à côté de l’autre, un restant à l’échiquier, l’autre dans la circulation.
↑ Histoire du Revenu public, par James Postlethwaite, p.301.
↑ En 1800, la Banque consentit à avancer au gouvernement 3,000,000 livres sterling pour six ans, sans intérêt ; et, en considération de cette avance, la charte fut prolongée jusqu’après douze mois d’avertissement, à partir du 1er août 1833. En 1807, la Banque consentit à continuer ce prêt de 3,000,000 livres sterling, sans intérêt jusqu’à six mois après la signature d’un traité de paix définitif ; et en 1810, elle prolongea son avance jusqu’à l’expiration de la charte en 1833, à raison de 3 pour 100 d’intérêt. La dette permanente due par l’État à la Banque, de 1800 à 1833, était de 14,686,800 livres sterling ; mais de 1800 à 1810, la Banque ne recevait d’intérêts que sur la somme de 11,680,800 livres sterling. En 1833, la charte fut continuée jusqu’à douze mois après l’avertissement, à partir du 1er août 1855, mais avec la condition importante qu’elle pourrait finir après douze mois d’avertissement, à partir du 1er août 1815. Dans cette dernière circonstance, l’État paya à la Banque le quart de la somme qui lui était due, ou 3,058,250 liv. sterl., ce qui réduit la somme totale due à cet établissement (1838) à 11,018,530 livres sterling, sans compter les sommes de la dette flottante ou d’autres valeurs publiques qu’elle pourrait posséder. En 1782, le capital de la Banque, ou la somme sur laquelle les dividendes sont payés, s’éleva de 10,780,000 livres sterling, chiffre donné par Smith, à 11,042,400 livres sterling, et en 1816 il fut porté à 14,553,000 livres sterling. La somme de 3,038,250 livres sterling payée par l’État à la Banque, en 1833, fut destinée par l’acte de renouvellement de la charte à réduire le capital de la Banque, qui n’est plus, en conséquence de cette opération, que de 10,914,750 livres sterling, depuis le 5 octobre 1834. Mac Culloch.
↑ Voici quels ont été les dividendes payés sur le capital de la Banque, de 1767 jusqu’à nos jours ; de 1767 à 1781, 5 1/2 pour 100 annuellement ; de 1781 à 1788, 6 pour 100 ; de 1788 à 1807,7 pour 100 ; de 1807 à 1823,10 pour 100 ; et de 1823 jusqu’à nos jours (1838), 8 pour 100. Les sommes payées comme dividendes ne comprennent pas celles qui ont été avancées en certaines occasions comme boni. Ces dernières s’élevèrent, depuis 1799, à 3,783,780 livres sterling en sus de l’augmentation du capital de la Banque en 1816, qui s’éleva à 2,910,600 livres sterling. Mac Culloch.
↑ La somme payée par l’État à la banque d’Angleterre, pour l’administration de la dette publique, s’élevait, avant 1833, environ à 270,000 livres sterling par ; mais l’acte de renouvellement de la charte (3 et 4, Guillaume IV, ch. 78) ordonna que la Banque déduirait 120,000 livres sterling chaque année sur les sommes qu’elle était autorisée à prélever en raison de cette administration. Mac Culloch.
↑ Petite monnaie d’argent de la valeur de 50 centimes.
↑ La crise la plus importante dans l’histoire de la circulation de papier de la Grande-Bretagne eut lieu en 1797. En partie par suite des événements résultant de la guerre où nous étions alors engagés, des prêts à l’empereur d’Allemagne, des traites faites sur le trésor par les agents anglais au dehors, et en partie, et principalement peut-être, par suite des larges avances accordées au gouvernement par la banque d’Angleterre, le change devint onéreux en 1795, et, cette année, ainsi que les années suivantes, il fut demandé à la banque des quantités énormes en espèces. Il n’est pas douteux cependant que la dernière crise ne fût entièrement due à des causes politiques. Des bruits d’invasion, et même de descentes qui auraient eu lieu sur les côtes, acquirent une certaine gravité pendant la fin de l’année 1796 et le commencement de 1797. Cette alarme provoqua chez beaucoup de particuliers, mais surtout chez les petits fermiers et les marchands en détail, un vif désir de convertir la plus grande partie possible de leur fortune en espèces. Une foule redoutable se précipita sur la plupart des banques de province ; et la banqueroute de quelques-uns de ces établissements à Newcastle, ainsi qu’en d’autres parties du royaume, imprima une force nouvelle à la première panique. La banque d’Angleterre fut assaillie de tous les points du territoire par des demandes d’argent, et le fonds d’espèces et de lingots renfermés dans ses coffres, qui s’était élevé en mars 1795 à 7,940,000 livres, se trouvait réduit, le samedi 25 février 1797, à 1,272,000 livres, avec la perspective d’une violente irruption pour le lundi suivant. Dans cette douloureuse circonstance, le conseil privé se réunit et décida que les payements en espèces seraient suspendus à la banque jusqu’à ce que le parlement eût pu statuer. À cet effet, un ordre du conseil fut promulgué le dimanche 20 février 1797.
Aussitôt que commença la suspension, les principaux négociants, banquiers et armateurs de Londres signèrent la résolution expresse d’accepter les billets de la banque d’Angleterre, et se portèrent caution des efforts qu’ils tenteraient pour les faire accepter des autres. Cette résolution prise conformément à l’état officiel des affaires de la banque qui fut rendu public, jointe à l’emploi de ses billets dans les payements publics, prévint toute interruption dans leur circulation ; et, grâce à la modération qui présida aux émissions, ils continuèrent pendant trois ans à être parfaitement équivalents à l’or.
La première baisse dans la valeur des billets de banque comparés à l’or commença vers la fin de 1800. Les faibles récoltes de cette année amenèrent une exportation considérable de métaux précieux ; mais au lieu de diminuer leurs émissions comme le leur ordonnaient les vrais principes, et comme ils eussent été obligés de le faire dans le cas où on leur eût imposé l’obligation de payer en argent, les directeurs ajoutèrent encore à la quantité de leurs billets existants, et la conséquence immédiate fut que ceux-ci subirent une dépréciation de 8 pour 100 comparés avec l’or. Mais bientôt après ils reprirent leur valeur ; et de 1803à 1808 inclusivement, ils n’offraient plus qu’un escompte de 2 livres 13 sch. 3 deniers pour 100. En 1809 et 1810 cependant, les directeurs parurent avoir méprisé tous les principes qui avaient jusque-là gouverné leurs émissions. La quantité moyenne de bank-notes en circulation, qui n’avait jamais dépassé 17 millions 1/2, ni été au-dessous de 16 millions 1/2 dans aucune des années de 1802 à 1808 inclusivement, s’éleva en 1809 à 18,927,833 livres, et en 1810 à 22,541,523 livres. Les émissions des banques de province s’accrurent dans un rapport encore plus grand ; et comme il ne se manifesta pas un développement relatif dans les affaires du pays, l’escompte sur les bank-notes s’éleva, de 2 liv. 13 sch. 2 deniers vers le commencementdel809, à 13 livres 9 schellings 6 deniers en 1810. Cette chute extraordinaire dans la valeur du papier comparée à celle de l’or, jointe comme elle le fut à une baisse égale dans le change, excita au plus haut point l’attention, et en février 1810, un comité de la Chambre des communes fut désigné pour rechercher les causes du haut prix des lingots d’or, et de l’état du change. Le comité consulta plusieurs négociants et banquiers, et son rapport, principalement rédigé par M. Francis Norver, renferme une habile réfutation des chiffres et des doctrines posés par ceux qui soutenaient que la baisse du change et le haut prix des lingots devaient être entièrement attribués à nos dépenses au dehors et à l’état spécial de nos relations avec les autres puissances, et ne tenaient nullement aux quantités additionnelles de papier qui étaient venues grossir la circulation. Mais la Chambre des communes refusa de sanctionner le projet par lequel le comité invitait la banque à reprendre ses payements en espèces au bout de deux ans. Aussi en mai 1811, époque à laquelle les guinées emportaient couramment une prime, et où les bank-notes éprouvaient un escompte avoué de plus de 10 pour 100 comparés aux lingots d’or,la Chambre des communes adopta, à une grande majorité, la résolution proposée par M. Vansittart (actuellement lord Bexley), déclarant que les engagements de la banque d’Angleterre avaient été jusqu’alors, et étaient encore en ce moment considérés dans l’opinion publique comme équivalents à la monnaie légale du royaume.
Cette résolution, tellement extraordinaire qu’elle était contraire au simple bon sens, dégagea les directeurs de la banque de toute crainte relativement à l’intervention du Parlement, et les encouragea à accroître le nombre de leurs billets en circulation. Les émissions des banques provinciales s’augmentèrent encore plus rapidement que celles de la banque d’Angleterre. La facilité d’être admis à l’escompte fut telle, que des individus qui pouvaient à peine payer le timbre de leurs billets réussirent très-fréquemment à obtenir de vastes capitaux; et comme ils ne risquaient rien personnellement, ils se livrèrent audacieusement aux spéculations les plus hasardées. M. Wakefield, dont la position lui offrit tant d’occasions de recueillir des renseignements exacts, informa le comité d’agriculture, en 1821, que « jusqu’à l’année 1813, il existait des banques sur presque tous les points du territoire, qui forçaient l’entrée de leur papier dans la circulation au prix d’énormes dépenses pour elles-mêmes, et, en beaucoup de cas, au prix de leur ruine. » Et parmi les diverses réponses qui furent adressées aux enquêtes du conseil d’agriculture en 1816 par les citoyens les plus intelligents des différents districts du pays, il en est à peine une dans laquelle l’émission exagérée des billets de banque ne soit pas particulièrement désignée comme l’une des causes prédominantes de la hausse, sans antécédent encore, qui avait atteint les rentes et les prix.
Le prix du blé s’était élevé à un chiffre extraordinaire pendant les cinq années qui finirent en 1813. Mais partie en raison de la brillante récolte de cette année, partie, et principalement peut-être, par suite de l’ouverture des ports hollandais et du renouvellement des relations avec le continent, les prix fléchirent considérablement vers la fin de l’année 1813 et le commencement de 1814. Et cette baisse ayant produit un manque de confiance, et répandu l’alarme parmi les banques de province et leurs clients, détermina une destruction de papier de province qui n’a pu être égalée que par celle de 182S. En 1814, 1815 et 1816, on ne vit pas moins de 240 banques suspendre leurs payements, et 89 accusations de banqueroute furent lancées contre ces établissements, et cela dans le rapport d’une accusation contre 10 banques 1/2 de province existant en 1813. Les faillites qui s’ouvrirent alors furent les plus désastreuses, car elles atteignaient principalement les classes ouvrières, et dévoraient ainsi en un moment les fruits d’une longue vie de travail et d’économie. Des milliers d’individus, qui avaient en 1812 rêvé l’aisance, se trouvèrent dépourvus de toute véritable propriété, et plongés, comme par enchantement, sans qu’il y eût faute de leur part, dans l’abîme de la pauvreté.
La destruction du papier des banques de province en 1814,1815 et 1816, en réduisant la masse totale mise en circulation, éleva sa valeur, en 1816, à une presque égalité avec l’or. Et cette hausse ayant matériellement facilité un retour aux payements en espèces, on commença à être généralement convaincu de l’opportunité qu’il y aurait à rapporter le décret sur les payements en argent de la banque d’Angleterre. Ceci fut effectué en 1819 par l’acte 59 de George III, chap. 78, communément appelé bill de Peel, parce qu’il avait été proposé et obtenu à la Chambre des communes par sir Robert Peel.
On sera justement étonné que malgré les leçons à puiser dans les banqueroutes de 1793,1814,1815 et 1816 occasionnées d’une manière si funeste par le système des banques de province, il ne fut fait aucun pas en 1819, alors que les restrictions sur les payements en espèces s’effacèrent, pour reconstituer ce système et le fonder sur des bases plus solides. Les nations sont des écoliers lents et rétifs, et il semble qu’une expérience complémentaire était nécessaire pour convaincre le parlement et le peuple d’Angleterre qu’il existait quelque chose de défectueux dans un système qui, dans deux circonstances antérieures, avait inondé le pays de banqueroutes, et qui décerne à tout individu, même pauvre ou sans principes, mais qui se sent porté à être banquier, le droit d’émettre des billets qui serviront comme monnaie dans les transactions habituelles de la société. La crise qui survint en 1823 et 1826 fut le résultat naturel de cet état de choses, et eût pu être prévue par tout individu instruit des principes sur lesquels doivent se baser les opérations des banques, ou de l’histoire précédente de ces banques dans le pays.
Ces événements persuadèrent enfin le parlement et le public de ce dont ils eussent dû être convaincus longtemps avant, c’est-à-dire que le système des banques privées en Angleterre et dans les Galles était au plus haut degré faible et vicieux, et qu’il était impérieusement nécessaire de le réformer et le fortifier. Dans ce dessein, l’acte de 1708 limitant le nombre des associés d’une banque à six, fut rapporté avec le consentement de la banque d’Angleterre. Permission fut accordée pour établir des joint-stock banks, banques à fonds réunis ou par actions, ou des banques composées d’un nombre illimité d’actionnaires, pour l’émission de billets payables sur tous les points du territoire au delà d’un rayon de soixante-cinq milles autour de Londres. On autorisa en même temps l’institution, à Londres, de joint-stock banks pour les dépôts seulement, ou banques destinées à prendre soin de l’argent de leurs commettants. Après les restrictions imposées aux payements en espèces, en 1797, la banque d’Angleterre commença à émettre, pour la première fois, des billets d’une livre, opération dans laquelle elle fut imitée parla plupart des banques de province. La première retira ses billets d’une livre peu après la reprise des paiements en espèces, en 1821 ; mais les billets similaires des banques de province continuèrent à circuler, et formèrent un des principaux canaux par lesquels elles faisaient pénétrer leur papier dans la circulation. En 4820 cependant, l’émission des billets d’une livre fut définitivement prohibée après une certaine époque spécifiée en Angleterre et dans les Galles ; et, depuis 1829, il ne fut plus permis de créer des billets de moins de cinq livres.
La dernière de ces mesures réparatrices, c’est-à-dire la suppression des billets d’une livre, a indubitablement ferme une des voies les plus aisées et les plus sûres dont se servaient les classes inférieures des banques de province pour écouler leur papier, et elle a été sous ce rapport très-avantageuse. Mais un grand nombre d’autres routes leur demeurent ouvertes; et l’exemple de 1792-93, alors qu’il n’existait point de billets au-dessous de cinq livres en circulation, démontre victorieusement que la suppression des billets d’une livre n’offre aucune sécurité contre les sur-émissions, les paniques, contre rien enfin, sinon contre une banqueroute universelle.
Ce fut cependant de la seconde mesure, celle autorisant l’établissement des jointstock banks, qu’on attendait les plus grands avantages. Peut-être serait-ce une exagération que d’affirmer que ces espérances ont été complètement déçues ; mais, si quelques attentes ont été réalisées, elles sont bien peu importantes. Il aurait été, en effet, facile de prédire, à l’origine de cette institution, comme cela eut lieu, du reste, que le seul établissement des joint-stock banks ne fournirait aucun remède contre les maux primitivement inhérents à notre système financier. Une banque avec sept, soixante-dix ou sept cents associés peut n’être pas appelée à plus de crédit qu’une autre banque avec cinq ou six, et peut-être même à moins. La fortune des associés d’une banque privée peut excéder celle des associés d’une vaste banque par actions ; et il est probable que les opérations de la plus petite banque étant conduites par les intéressés eux-mêmes, le seront plus prudemment et plus économiquement que celles d’une grande banque, qui doivent nécessairement être confiées à des agents sur lesquels ne plane qu’un contrôle inefficace. On lie peut concevoir de plus grande erreur que celle qui décide que parce qu’une banque a un plus grand nombre d’associés, elle est plus digne de la confiance publique. Celle-ci doit dépendre de leur richesse et de leur intelligence, mais non de leur nombre : ce serait substituer la masse au mérite. La richesse seule ne peut suffire à mettre en rapport les émissions de papier avec les besoins. Les, joint-stock banks demeurent aussi loin, et, si cela est possible, plus loin même de ce critérium que les banques privées. C’est, en effet, la plus grossière des erreurs et des illusions, que de supposer qu’il est possible de faire disparaître les fluctuations dans la masse et la valeur de la monnaie, par cela seul qu’elle sera fournie par différents agents. Tant qu’un individu ou une réunion d’individus, quelque tarés qu’ils puissent être, jouiront du privilège royal d’émettre du papier sans autorisation ni obstacles, on verra ce papier s’accroître démesurément aux époques de confiance, et disparaître aussitôt que les prix et la confiance s’ébranleront. Si l’on désire que le pays soit à jamais dévoré par une fièvre intermittente et livré tantôt aux accès de sur-excitation, tantôt à un état d’atonie qui en est la suite inévitable, il n’est pas de meilleur moyen à employer que notre système financier actuel. Mais nous pensons que le lecteur se joindra à nous, dans la pensée qu’une fièvre de cette nature est aussi fatale au corps politique qu’au corps physique ; et que si l’on n’opère une cure radicale, elle paralysera et détruira le malade.
Mac Culloch.
↑ C’est une erreur d’affirmer que l’or et l’argent employés comme monnaie ne sont point productifs. Il est évident qu’ils sont, au contraire, excessivement productifs, puisqu’ils facilitent les échanges, et permettent à la division du travail d’être infiniment plus développée qu’elle ne pourrait l’être avec le système d’échange en nature. Mac Culloch.
↑ Les opérations de banque n’ont pas pour effet réel de convertir un capital mort en capital productif. Leur véritable effet consiste tout simplement à substituer des instruments d’échange à bon marché à des instruments plus coûteux. Mac Culloch.
↑ Cette magnifique image de la circulation du papier des banques est l’expression réelle des faits économiques, tels que l’histoire de l’Angleterre et des États-Unis nous les a retracés depuis cinquante ans. Jamais la sagesse des vues de Smith ne s’est révélée avec plus d’éclat que dans ce chapitre où tous les économistes ont puisé les éléments de leurs travaux sur les banques, sans éclipser les siens. A. B.
↑ Dans l’état avancé où se trouve aujourd’hui la civilisation, il y a, dans tout pays ayant un bon gouvernement et une population considérable, si peu de chances de guerre civile ou d’invasion étrangère, qu’en recherchant les moyens propres à assurer la félicité nationale, on ne doit guère tenir compte de ces événements. Adopter une manière d’agir désavantageuse dans tous les temps, excepté ceux de guerre civile ou d’invasion étrangère, uniquement parce qu’elle serait bonne dans ces occasions, serait aussi absurde que de vouloir, en médecine, soumettre continuellement les hommes au régime qui convient pendant une maladie violente. Si les avantages qui résultent de l’usage du papier-monnaie sont tels qu’on en puisse jouir, sans diminution considérable, dans tous les temps, excepté ceux de guerre civile et d’invasion étrangère, l’utilité du papier-monnaie est suffisamment démontrée. Une guerre civile, de même qu’une invasion étrangère, est suivie d’un grand désordre dans la circulation, lorsque le moyen d’échange se compose d’or et d’argent. À de semblables époques, il règne une disposition générale à thésauriser. Une portion considérable du moyen d’échange se trouve retirée de la circulation, et l’on ressent immédiatement tous les maux résultant de la rareté du numéraire : le prix des denrées baisse, la valeur de l’argent hausse ; ceux qui ont du numéraire à vendre et ceux qui ont des dettes à payer éprouvent des pertes, et la misère se répand de tous côtés.
La communauté serait en grande partie préservée des funestes conséquences de la thésaurisation par l’existence d’un papier-monnaie, et beaucoup de motifs nous portent à conclure que celles qui résultent d’une diminution de crédit seraient très-peu à craindre.
Si l’émission du papier avait été faite par un gouvernement digne de la confiance du peuple, une invasion étrangère, qui concentrerait toutes les affections du peuple sur ce gouvernement, ne détruirait pas le crédit de ses billets.
Il ne serait pas de intérêt des envahisseurs de détruire ce crédit, même dans la partie du pays occupée par eux, parce qu’il ne serait pas de leur intérêt de diminuer sa puissance productive.
Personne ne perdrait en définitive, parce que, dans le cas même où la circulation des billets du gouvernement serait arrêtée dans les districts occupés par l’ennemi, ils recouvreraient leur valeur au moment où l’ennemi serait expulsé. J. Mill
↑ Il est aisé de voir que ce mot de commerçants, qu’il a fallu employer faute de terme plus générique, comprend ici généralement tous marchands, manufacturiers, gens de métier, etc. ; en un mot, tous les agents du commerce et de l’industrie.
↑ La moitié d’un schelling.
↑ La banque d’Angleterre a émis pour la première fois des billets de dix livres en 1759, des billets de cinq livres en 1793, et des billets d’une et deux livres en mars 1797. Les billets au-dessous de cinq livres furent pour la plupart retirés de la circulation en 1832. Pendant la crise de 1820, on jugea à propos d’émettre de nouveau pour 1,500,000 livres sterling de billets d’une livre ; mais bientôt une grande quantité de ces billets retournèrent à la Banque. Eu décembre 1835, il ne restait plus que 282,000 livres sterling de billets d’une livre, dont la plupart ont été perdus ou détruits. Mac Culloch.
↑ La monnaie de papier ne peut pas circuler d’une manière sûre sans cette condition. Le remboursement immédiat des espèces est la seule garantie qui existe contre l’excès des émissions. Les directeurs d’une banque, quelque droites que soient leurs intentions, n’ont pas de thermomètre infaillible des besoins de la circulation, et il est hors de doute qu’un papier non remboursable en espèces se multipliera tôt ou tard avec exagération. La Banque d’Angleterre, quelque temps après la suspension de ses payements, conserva une grande mesure à cet égard ; mais bientôt ses billets excédèrent tellement les besoins du commerce, qu’ils perdirent environ 40 pour 100, malgré les lois rendues pour leur conserver leur valeur nominale.
Buchanan.
.
↑
Il semble que le seul moyen de protéger le public contre l’insolvabilité des banquiers soit de les contraindre à donner des garanties pour le payement de leurs billets ; mais cette mesure ne remédierait pas aux fluctuations qui se manifestent dans l’approvisionnement de la monnaie, et qui doivent avoir lieu aussi longtemps que la circulation sera servie par des émetteurs différents. On sait que ces fluctuations produisent le plus grand mal.
Mac Culloch.
Une circulation de papier conversible en espèces à la volonté du porteur ne doit essuyer aucune diminution de valeur par le discrédit ou l’excès, car la sécurité peut toujours être constatée par une demande en payement, et, dans le ras d’une émission surabondante, le papier émis avec excès serait renvoyé à la banque pour y être échangé contre des espèces. Ainsi, un billet d’une livre ou une promesse de payer une livre, ne sera jamais échangée pour moins de sa valeur en espèces, tant que l’obligation première de payer sur demande conservera sa force.
Aux premiers symptômes d’une baisse dans la valeur de ce billet, on s’adressera à la banque pour l’accomplissement de sa promesse ; elle sera poursuivie pour le remboursement en espèces jusqu’à ce que le papier, se trouvant moins abondant, recouvre sa valeur primitive. Si la banque persiste à émettre de nouveau ses billets remboursés, elle s’expose à un écoulement continuel d’espèces pendant tout le temps que l’excès du papier encombrera la circulation.<
Là où une banque n’est pas obligée de payer en espèces, la circulation entière est livrée à la discrétion de ses directeurs. Dans de telles circonstances, il n’existe aucun remède contre une sur-émission, parce que la banque qui aie pouvoir de refuser des espèces, est protégée contre tout retour de ses billets superflus et dépréciés. Mais il se peut qu’une banque n’abuse pas de pouvoirs aussi étendus, et que, réservant son privilège de refus pour les cas d’extrême gêne, elle rembourse ceux de ses billets que la marche naturelle des affaires renvoie à ses bureaux. Dans cette hypothèse, et si aucun soupçon ne plane sur son crédit, son papier n’éprouvera aucune dépréciation. C’est pourquoi la possibilité d’une conversion immédiate en espèces n’est pas absolument essentielle à sa valeur. La conversion offre certainement la meilleure sécurité contre une émission exagérée que l’on peut considérer comme cause évidente de dépréciation ; mais là où le papier est soutenu par de puissants capitaux et n’est pas répandu à l’excès, il conservera toute sa valeur, quoique n’étant convertible en espèces qu’à la discrétion de la banque.
Le service de la circulation en papier se faisant à peu de frais, ce dernier se répand bientôt dans un pays à l’exclusion de la monnaie, qui est, à la longue, encaissée par les banquiers pour satisfaire aux demandes éventuelles de remboursement. Dans la circulation de ce pays (l’Angleterre), la monnaie a été presque entièrement remplacée par le papier, dont la masse parait s’être fortement accrue pendant ce court intervalle de paix qui suivit la guerre d’Amérique. Aussi, vers cette époque, vit-on des banques s’élever sur toutes les parties du territoire.
Les avantages du papier, comme intermédiaire d’échanges, ressortent d’une manière assez claire des remarques précédentes ; mais il faut observer qu’il n’est jamais, pour cette fonction, un agent aussi sûr que la monnaie, car il repose entièrement sur le crédit, dont la chute est nécessairement suivie d’une perturbation générale dans le commerce. C’est principalement par l’escompte des lettres de change que le papier entre en circulation ; et le marchand qui reçoit de l’argent contre ses traites, étendant proportionnellement le cercle de ses opérations, devient graduellement tributaire des banques pour les espèces nécessaires à ses affaires, tandis que les banques à leur tour relèvent de la confiance publique pour la circulation de leurs billets. Le banquier substitue un instrument d’échange de bas prix à un autre plus dispendieux, et, comme ses profits consistent à le prêter aux mêmes conditions, il est naturellement porté à accroitre la circulation de son papier, et, sur la foi de son crédit, à diminuer les réserves destinées aux remboursements.
Tant que la confiance générale domine, il ne peut résulter aucun mal de ce système de crédit si compliqué. Les bank-notes circuleront librement ; les demandes d’espèces seront rares, et les effets des négociants seront immédiatement convertibles en argent. Mais quand la confiance s’ébranle, les banques sont exposées, par le discrédit inévitable de leur papier, à de continuelles demandes en remboursement. Pour éviter cette crise, elles diminuent la circulation de leurs billets ; tandis que le négociant, subitement dépourvu de ses ressources antérieures, ne peut soutenir son crédit et se trouve ainsi réduit aux plus graves embarras. Les demandes d’espèces dans le pays convergent graduellement vers la capitale, les banquiers convertissant en espèces les titres qu’ils possèdent habituellement dans les fonds publics et autres valeurs de l’État, et s’adressant à la banque d’Angleterre pour les fonds nécessaires. Gênée par l’émission croissante de ses espèces, la banque réduit la circulation de son papier ; les transactions de la capitale, jusque-là effectuées avec la plus minime quantité d’argent possible, sont conséquemment altérées, et le désordre ainsi provoqué dans le centre s’étend rapidement jusqu’aux points les plus éloignés de ce vaste système de commerce d’argent. Pour cette solidarité mutuelle que créent le crédit et la confiance, chaque négociant entraine les autres dans son destin : la contagion de la banqueroute se répand, et, dans cette commotion générale, les plus vastes établissements commerciaux peuvent s’écrouler.
Toutes ces fatales conséquences se développèrent dans ce pays pendant l’alarme commerciale de 1792. La rareté de l’urgent et le discrédit du papier occasionnèrent de nombreuses banqueroutes, en même temps qu’une demande générale d’espèces à laquelle la banque d’Angleterre fut impérieusement appelée à satisfaire. Comme cette demande dérivait de l’alarme qui avait discrédité le papier, il était évident que pendant toute la durée de cette panique, la circulation même la plus limitée de ses billets, continuellement émis et continuellement retournés, suffirait à enlever à la banque une masse indéfinie d’espèces ; et c’est pourquoi la marche rationnelle de la banque consistait à accroître l’émission de son papier dans le but de ranimer la confiance du commerce, la seule digue qu’il y eût à opposer aux demandes en remboursement. Mais, au lieu de secourir le négoce, soit par une émission plus forte de son papier, soit par tout autre moyen, la banque refusa d’entendre celles des banques de province qui demandaient assistance. Dans ces circonstances, le parlement intervint pour soulager les négociants qui manquaient d’argent, et offrit de prêter, sur gages valables, des bons de l’échiquier jusqu’à concurrence de 5,000,000 de livres, s’il était nécessaire. Sur cette somme, 3,855,624 livres furent soumissionnées. Après que la plus grande partie des offres eut été retirée ou rejetée, la somme définitivement accordée se monta à 2,903,000 livres, lesquelles furent remboursées, la majorité avant l’échéance, et le reste à des époques déterminées, sans difficulté ou détresse apparente.
Cette salutaire mesure ranima bientôt la confiance publique, qui ne subit aucune nouvelle atteinte jusqu’à l’an 1795, époque à laquelle les directeurs de la banque d’Angleterre ayant, dans le courant de l’année, fait des avances extraordinaires au gouvernement, jugèrent nécessaire de diminuer les sommes allouées aux exigences du commerce. Les transactions de la métropole furent conséquemment dérangées, et la menace d’une invasion, continuellement exprimée par l’ennemi pendant l’année 1796, répandit une panique générale qui fut promptement suivie du discrédit des bank-notes et d’une demande en argent.
Dans le nord de l’Angleterre, plusieurs banques furent dans la nécessité de suspendre leurs payements en espèces, et l’effet de ces faillites s’étendit rapidement à la capitale, de telle sorte que, vers le commencement de l’année 1797, la banque d’Angleterre subit une demande alarmante d’espèces, occasionnée d’abord par les efforts des banques de province en faveur de leur crédit chancelant, et secondement par les retours de ses propres billets frappés de discrédit. Avant cette époque, les directeurs avaient plus d’une fois démontré au chancelier de l’échiquier combien la banque se trouvait embarrassée par la diffusion de ses espèces ; mais alors, sérieusement effrayés, ils lui communiquèrent, le jeudi 21 février, la réduction positive qu’avait subie son fonds, afin qu’il pût plus librement aviser aux mesures que réclamait une crise aussi dangereuse. Pendant le reste de la semaine, les demandes continuèrent \& s’accumuler avec une telle rapidité que, quoique la perte d’espèces eût été ruineuse dans les quatre premiers jours, elle fut surpassée par celle des deux jours suivants, et jusqu’à la dernière heure, les demandes continuèrent i s’accroître dans cette rapide progression. La nécessité de relever la banque altérée par ces ruineux remboursements était devenue manifeste, et le dimanche, les directeurs s’étant réunis par un rendez-vous avec le chancelier de l’Échiquier et les autres ministres, le conseil rendit, séance tenante, un arrêt qui prohibait tous payements ultérieurs en espèces.
Après la suspension des payements en argent par la banque, l’étal de la circulation fut fréquemment exposé au parlement ; et un comité de la chambre des communes, nommé en 1810, pour rechercher les causes du haut prix des lingots, fut d’accord, après un examen laborieux de faits palpables, de l’attribuer à une sur-émission de billets de banque suivie de leur dépréciation immédiate.
Quand la banque d’Angleterre suspendit pour la première fois ses payements en espèces, la loi encouragea si fortement la circulation de ses billets, qu’un débiteur qui les offrait en payement était protégé contre toute contrainte, alors même que son créancier, par le droit commun des contrats, pouvait exiger le payement en guinées, monnaie légale du pays. En 1810, les billets de banque, par les progrès de la dépréciation, commencèrent à s’échanger communément pour moins de leur valeur nominale en espèces. Ainsi 25 et 26 schellings en papier balançaient une guinée ; et quoique le parlement eût à cette même époque décrété que la valeur du papier conserverait le taux primitif d’émission, une loi fut jugée nécessaire pour arrêter les échanges qu’on en faisait ouvertement sous escompte contre l’argent. À cet effet, des peines sévères atteignirent les échanges du papier contre les guinées au taux du marché : et les fermiers qui offraient des billets de banque en payement de leurs rentes étaient en même temps protégés contre la saisie, quoiqu’ils fussent toujours sous le coup de poursuites en justice. En 1811 fut rendu un acte protégeant le débiteur qui offrait un payement en billets de la banque d’Angleterre, contre toutes poursuites ultérieures. Parce même acte, If papier déprécié devint une base légale pour la fixation des dettes existantes, abstraction faite de sa valeur, et tous les créanciers éprouvèrent, en conséquence, des pertes proportionnées aux variations qu’avait subies la monnaie dans laquelle on les remboursait.
Buchanan.
↑ Les billets d’une banque qui a le droit d’en refuser le remboursement en espèces pourraient circuler au pair s’il n’y avait pas d’excès dans les émissions ni de doute sur la solvabilité de la banque.
↑ Sur le taux naturel et le taux apparent du change, et sur l’influence qu’à l’état de la monnaie d’un pays sur le taux apparent du change, voyez le livre IV chap. iii.
↑ Opération qui répond à ce qu’on appelle, dans le commerce de France, virement de parties. Elle consiste à porter sur compte u porteur de la lettre de change une somme égale de cette lettre, et à retrancher cette même somme du compte de celui qui était débiteur de la lettre.
↑ Voyez la digression sur cette banque, livre IV, chap. iii.
↑ Smith ne se tromperait-il point ici ? La valeur d’une marchandise est en raison composée de ce qu’il en coûte pour l’amener au marché, et de la proportion entre cette quantité de marchandise et la demande qui en est faite dans le même marché. Si on émet des billets de banque qui rendent inutiles, comme monnaie, les 2/3 de l’argent qui circule, ces 2/3 se répandront dans tous les marchés et augmenteront la proportion de cette marchandise, comparée avec la demande qui en est faite. Rien dans cette opération ne doit augmenter la demande en même temps que la fourniture. Celle-ci étant plus grande, et la demande étant la même, la valeur de l’argent doit baisser jusqu’à ce que la baisse élève la demande au niveau de la fourniture.
II est vrai que le marché pour l’or et l’argent étant par tout le monde, ce qu’un pays peut jeter de ces métaux dans un si grand marché doit déranger bien peu la proportion. Note inédite de J.-B. Say.
↑ Mais pour rendre effective cette obligation de payement immédiat et sans condition, il est indispensable d’exiger des banques des garanties.

%%%%%%%%%%%%%%%%%%%%%%%%%%%%%%%%%%%%%%%%%%%%%%%%%%%%%%%%%%%%%%%%%%%%%%%%%%%%%%%%
%                                  Chapitre 3                                  %
%%%%%%%%%%%%%%%%%%%%%%%%%%%%%%%%%%%%%%%%%%%%%%%%%%%%%%%%%%%%%%%%%%%%%%%%%%%%%%%%

\chapter{De l’accumulation du capital, ou du travail productif et du travail non productif}
\markboth{De l’accumulation du capital, ou du travail productif et du travail non productif}{}

Il y a une sorte de travail qui ajoute à la valeur de l’objet sur lequel il s’exerce ; il y en a un autre qui n’a pas le même effet. Le premier, produisant une valeur, peut être appelé travail productif ; le dernier, travail non productif[1]. Ainsi, le travail d’un ouvrier de manufacture ajoute en général, à la valeur de la matière sur laquelle travaille cet ouvrier, la valeur de sa subsistance et du profit de son maître. Le travail d’un domestique, au contraire, n’ajoute à la valeur de rien. Quoi­que le premier reçoive des salaires que son maître lui avance, il ne lui coûte, dans le fait, aucune dépense, la valeur de ces salaires se retrouvant en général avec un profit de plus dans l’augmentation de valeur du sujet auquel ce travail a été appliqué. Mais la subsistance consommée par le domestique ne se trouve nulle part. Un particulier s’enrichit à employer une multitude d’ouvriers fabricants ; il s’appauvrit à entretenir une multitude de domestiques. Le travail de ceux-ci a néanmoins sa valeur, et mérite sa récompense aussi bien que celui des autres. Mais le travail de l’ouvrier se fixe et se réalise sur un sujet quelconque, ou sur une chose vénale qui dure au moins quelque temps après que le travail a cessé. C’est, pour ainsi dire, une quantité de travail amas­sé et mis en réserve, pour être employé, s’il est nécessaire, dans quelque autre occa­sion. Cet objet, ou ce qui est la même chose, le prix de cet objet peut ensuite, s’il en est besoin, mettre en activité une quantité de travail égale à celle qui l’a produit origi­nairement. Le travail du domestique, au contraire, ne se fixe ou ne se réalise sur aucun objet, sur aucune chose qu’on puisse vendre ensuite. En général, ses services périssent à l’instant même où il les rend, et ne laissent presque jamais après eux au­cune trace ou aucune valeur qui puisse servir par la suite à procurer une pareille quantité de services.
Le travail de quelques-unes des classes les plus respectables de la société, de même que celui des domestiques, ne produit aucune valeur ; il ne se fixe ni ne se réalise sur aucun objet ou chose qui puisse se vendre, qui subsiste après la cessation du travail et qui puisse servir à procurer par la suite une pareille quantité de travail. Le souverain, par exemple, ainsi que tous les autres magistrats civils et militaires qui servent sous lui, toute l’armée, toute la flotte, sont autant de travailleurs non productifs. Ils sont les serviteurs de l’État, et ils sont entretenus avec une partie du produit annuel de l’industrie d’autrui. Leur service, tout honorable, tout utile, tout né­cessaire qu’il est, ne produit rien avec quoi on puisse ensuite se procurer une pa­reille quantité de services. La protection, la tranquillité, la défense de la chose publique, qui sont le résultat du travail d’une année, ne peuvent servir à acheter la protection, la tranquillité, la défense qu’il faut pour l’année suivante. Quelques-unes des professions les plus graves et les plus importantes, quelques-unes des plus frivoles, doivent être rangées dans cette même classe : les ecclésiastiques, les gens de loi, les médecins et les gens de lettres de toute espèce, ainsi que les comédiens, les farceurs, les musi­ciens, les chanteurs, les danseurs d’Opéra, etc. Le travail de la plus vile de ces profes­sions a sa valeur qui se règle sur les mêmes principes que toute autre sorte de travail ; et la plus noble et la plus utile ne produit par son travail rien avec quoi on puisse ensuite acheter ou faire une pareille quantité de travail. Leur ouvrage à tous, tel que la déclamation de l’acteur, le débit de l’orateur ou les accords du musicien, s’évanouit au moment même qu’il est produit[2]. 
Les travailleurs productifs et les non productifs, et ceux qui ne travaillent pas du tout, sont tous également entretenus par le produit annuel de la terre et du travail du pays. Ce produit, quelque grand qu’il puisse être, ne saurait être infini, et a nécessai­rement ses bornes. Suivant donc que, dans une année, une portion plus ou moins gran­de de ce produit est employée à entretenir des gens non productifs, plus ou moins grande sera la portion qui restera pour les gens productifs, et plus ou moins grand sera, par conséquent, le produit de l’année suivante ; la totalité du produit annuel, à l’exception des productions spontanées de la terre, étant le fruit du travail productif.
Quoique la totalité du produit annuel des terres et du travail d’un pays soit, sans aucun doute, destinée en définitive à fournir à la consommation de ses habitants et à leur procurer un revenu, cependant, à l’instant où il sort de la terre ou des mains des ouvriers productifs, il se divise naturellement en deux parties. L’une d’elles, et c’est souvent la plus forte, est, en premier lieu, destinée à remplacer un capital ou à renou­veler la portion de vivres, de matières ou d’ouvrage fait qui a été retirée d’un capital ; l’autre est destinée à former un revenu, ou au maître de ce capital, comme profit, ou à quelque autre personne comme rente de sa terre. Ainsi, du produit de la terre, une partie remplace le capital du fermier ; l’autre paye son profit et la rente du propriétaire, et forme ainsi un revenu, et au maître de ce capital, comme profit de ses fonds, et à quelque autre personne, comme rente de sa terre. De même, du produit d’une grande manufacture, une partie, et c’est toujours la plus forte, remplace le capital de l’entrepreneur, l’autre paye son profit et forme ainsi un revenu au maître de ce capital[3]. 
Cette partie du produit annuel de la terre et du travail d’un pays qui remplace un capital n’est jamais immédiatement employée à entretenir d’autres salariés que des salariés productifs ; elle ne paye de salaires qu’au travail productif seulement. Celle qui est destinée à former immédiatement un revenu, soit comme profit, soit comme rente, peut indifféremment entretenir des salariés productifs ou des salariés non productifs.
Toute partie de ses fonds qu’un homme emploie comme capital, il s’attend tou­jours qu’elle lui rentrera avec un profit. Il ne l’emploie donc qu’à entretenir des sala­riés productifs ; et, après avoir fait, à son égard, office de capital, cette même partie de fonds forme un revenu à ces travailleurs. Toutes les fois qu’il emploie une partie de ces mêmes fonds à entretenir des salariés non productifs, de quelque espèce que ce soit, dès ce moment cette partie se trouve retirée de son capital et versée dans le fonds réservé pour servir immédiatement à sa consommation.
Les travailleurs non productifs et les gens qui ne travaillent pas du tout sont tous entretenus par un revenu : soit, en premier lieu, par cette partie du produit annuel qui est, dès l’origine, destinée à former un revenu à quelques personnes particulières, ou comme rente de terre, ou comme profit de capital ; soit, en second lieu, par cette autre partie qui, bien qu’elle soit destinée à remplacer un capital et à n’entretenir que des ouvriers productifs, néanmoins, quand elle est une fois venue dans les mains de ceux-ci, pour tout ce qui excède leur subsistance nécessaire, peut être employée indifférem­ment à l’entretien de gens qui produisent ou de gens qui ne produisent pas. Ainsi, le simple ouvrier, si ses salaires sont élevés, peut, tout comme un grand propriétaire ou comme un riche marchand, entretenir un domestique à son service personnel, ou bien il peut aller quelquefois à la comédie ou aux marionnettes, et par là contribuer pour sa part à l’entretien d’une classe de travailleurs non productifs ; ou enfin, il peut payer quelque impôt, et par là concourir à l’entretien d’une autre classe plus honorable et plus utile à la vérité, mais également non productive. Néanmoins, de cette partie du produit de la terre, destinée originairement à remplacer un capital, il n’en passe jamais aucune portion à l’entretien de salariés non productifs, qu’après avoir mis en activité sa mesure complète de travail productif, ou tout ce qu’elle pouvait en mettre en activité, de la manière dont elle pouvait être employée. Il faut que l’ouvrier ait pleine­ment gagné son salaire par de l’ouvrage fait, avant qu’il puisse en dépenser la moindre chose en travail non productif. Ce qu’il dépense ainsi, d’ailleurs, ne peut être, en général, que peu de chose. Ce ne peut être que l’épargne faite sur son revenu, épargne qui n’est jamais bien grande chez les ouvriers productifs. Ils en font pourtant géné­ra­lement quelqu’une, et dans le payement des impôts la modicité de chaque contribution se trouve, en quelque sorte, bien compensée par le nombre des cotes.
La rente de la terre et les profits des capitaux sont donc partout les principales sources où les salariés non productifs puisent leur subsistance. Ce sont les deux sortes de revenu qui donnent à leurs maîtres le plus de matière à faire des épargnes. L’un et l’autre de ces revenus peuvent indifféremment entretenir des salariés productifs et des salariés non productifs ; ils semblent pourtant avoir toujours pour les derniers quelque prédilection. La dépense d’un grand seigneur fait vivre, en général, plus de gens oisifs que de gens laborieux. Quoique le riche commerçant n’emploie son capital qu’à entre­tenir des gens laborieux seulement, néanmoins sa dépense, c’est-à-dire l’emploi de son revenu, nourrit ordinairement des gens de la même espèce que ceux que nourrit le grand seigneur.
Ainsi, ce qui contribue beaucoup à déterminer dans tout pays la proportion entre les gens productifs et les gens non productifs, c’est principalement la proportion qui s’y trouve entre cette partie du produit annuel, qui, au sortir même de la terre ou des mains des ouvriers qui l’ont produite, est destinée à remplacer un capital, et cette autre partie qui est destinée à former un revenu soit comme rente, soit comme profit. Or, cette proportion est très-différente, dans les pays riches, de ce qu’elle est dans les pays pauvres.
Ainsi, de notre temps, chez les nations opulentes de l’Europe, une très-forte partie, et souvent la plus forte du produit de la terre, est destinée à remplacer le capital d’un fermier riche et indépendant ; l’autre, à payer ses profits et la rente du propriétaire. Mais anciennement, sous l’empire du gouvernement féodal, une très-petite portion du produit suffisait à remplacer le capital employé à la culture. Ce capital consistait ordi­nai­rement en quelques chétifs bestiaux entretenus en entier par le produit spontané des terres incultes, et qu’on pourrait, en conséquence, regarder eux-mêmes comme faisant partie de ce produit. En général aussi, ils appartenaient au propriétaire, et celui-ci les avançait aux gens qui faisaient valoir la terre. Tout le reste du produit lui appartenait encore, soit comme rente de sa terre, soit comme profit de son mince capital. Les cultivateurs de la terre, en général, étaient des serfs, dont les personnes et les effets composaient aussi sa propriété. Ceux qui n’étaient pas serfs étaient des tenanciers à volonté[4], et, quoique la rente par eux payée ne fût nominalement guère plus qu’un simple cens, elle n’en égalait pas moins réellement la totalité du produit de la terre. En tout temps, leur seigneur pouvait leur commander du travail pendant la paix, et du service pendant la guerre. Quoiqu’ils vécussent loin de sa maison, ils dépendaient autant de lui que les gens de sa suite, vivant chez lui. Or, sans contredit, celui qui peut disposer du travail et du service de tous ceux qu’une terre fait subsister, a bien la totalité du produit de cette terre. Mais, dans l’état actuel de l’Europe, la part du propriétaire ne va guère au-delà du tiers de la totalité du produit, quelquefois pas au quart. La rente de la terre, néanmoins, a dans le fait triplé et quadruplé depuis ces anciens temps, dans toutes les parties de la campagne qui ont été améliorées ; et ce tiers ou quart du produit annuel est, à ce qu’il paraît, trois ou quatre fois plus grand que n’était auparavant le total. À mesure des progrès que fait l’amélioration, la rente augmente bien relativement à l’étendue de la terre, mais elle diminue dans sa pro­portion avec le produit.
Chez les peuples opulents de l’Europe, on emploie à présent de grands capitaux dans le commerce et les manufactures. Dans l’ancien état de ces pays, le faible et étroit commerce qui s’y faisait, et le petit nombre de fabriques simples et grossières qui y étaient établies, n’exigeaient que de très-minces capitaux. Il fallait pourtant que ces capitaux rendissent de très-gros profits. Nulle part l’intérêt n’était au-dessous de 10 pour 100, et il fallait bien que les profits des fonds pussent suffire à payer un intérêt aussi fort. À présent, dans les pays de l’Europe qui ont fait quelques progrès vers l’opulence, le taux de l’intérêt n’est nulle part plus élevé que 6 pour 100, et dans quelques-uns des plus riches, il est même tombé jusqu’à 4, 3 et 2 pour 100. Si cette partie du revenu des habitants, qui provient de profits, est toujours beaucoup plus grande dans les pays riches que dans les pays pauvres, c’est parce que le capital y est beaucoup plus considérable ; mais les profits y sont en général dans une proportion beaucoup moindre, relativement au capital.
Ainsi cette partie du produit annuel qui, au sortir de la terre ou des mains des ouvriers productifs, est destinée à remplacer un capital, est non-seulement beaucoup plus grande dans les pays riches que dans les pays pauvres, mais encore elle s’y trouve dans une proportion bien plus forte, relativement à la partie destinée immé­diate­ment à former un revenu, soit comme rente, soit comme profit. Le fonds qui est destiné à fournir de la subsistance au travail productif est non-seulement bien plus abondant dans les premiers de ces pays qu’il ne l’est dans les autres, mais il l’est encore dans une grande proportion, relativement au fonds qui, pouvant être employé à entretenir des salariés productifs aussi bien que des salariés non productifs, a néan­moins toujours, en général, plus de prédilection pour les derniers.
La proportion qui se trouve entre ces deux différentes espèces de fonds détermine nécessairement, dans un pays, le caractère général des habitants, quant à leur pen­chant à l’industrie ou à la paresse. Si nous sommes plus portés au travail que nos an­cê­tres, c’est parce qu’à présent le fonds destiné à l’entretien du travail se trouve relati­vement au fonds qui a de la tendance à aller à l’entretien de la classe fainéante, beau­coup plus grand qu’il ne l’était il y a deux ou trois siècles. Nos pères étaient paresseux faute d’avoir de quoi encourager suffisamment l’industrie. Il vaut mieux, dit le pro­verbe, jouer pour rien, que de travailler pour rien. Dans les villes manufacturières et commerçantes, où les classes inférieures du peuple subsistent principalement par des capitaux employés, il est en général laborieux, frugal et économe, comme dans beaucoup de villes d’Angleterre et dans la plupart de celles de la Hollande. Mais dans ces villes qui se soutiennent principalement par la résidence permanente ou tempo­raire d’une cotir, et dans lesquelles les classes inférieures du peuple tirent surtout leur subsistance de dépenses de revenu, il est, en général, paresseux, débauché et pauvre, comme à Rome, Versailles, Compiègne et Fontainebleau. Si vous en exceptez Rouen et Bordeaux, on ne trouve dans toutes les villes de parlement, en France, que peu de commerce et d’industrie, et les classes inférieures du peuple, qui y vivent princi­palement sur la dépense des officiers des cours de justice et de ceux qui viennent y plaider sont, en général, paresseuses et pauvres. Rouen et Bordeaux semblent devoir absolument à leur situation leur grand commerce. Rouen est nécessairement l’entrepôt de presque toutes les marchandises que les pays étrangers ou les provinces maritimes de France fournissent à la consommation immense de Paris. Bordeaux est de même l’entrepôt des vins récoltés le long de la Garonne et des rivières qui se jettent dans ce fleuve, l’un des vignobles les plus riches du monde, et qui paraît produire le vin le plus propre à l’exportation ou le plus conforme au goût des nations étrangères. Des situations aussi avantageuses attirent nécessairement un grand capital par le grand emploi qu’elles lui offrent, et l’emploi de ce capital est la source de l’industrie qui règne dans ces villes. Dans les autres villes de parlement en France, il paraît qu’on n’y emploie guère plus de capital que ce qu’il en faut pour entretenir la consommation du lieu, c’est-à-dire guère plus que le moindre capital possible. On peut dire la même chose de Paris, de Madrid et de Vienne : de ces trois villes, Paris est sans contredit la plus industrieuse ; mais Paris est lui-même le principal marché de toutes ses manu­fac­tures, et sa propre consommation est le grand objet de tout le commerce qui s’y fait[5]. Londres, Lisbonne et Copenhague sont peut-être les trois seules villes de l’Europe qui, étant la résidence permanente d’une cour, puissent en même temps être regardées comme villes commerçantes ou comme villes faisant le commerce, non-seulement pour leur propre consommation, mais encore pour celle des autres villes et des autres pays. Leur situation à toutes trois est extrêmement avantageuse, et est naturellement propre à en faire des entrepôts pour une grande partie des marchandises destinées à la consommation des pays éloignés. Dans une ville où se dépensent de gros revenus, il sera probablement plus difficile d’employer avantageusement un capital en entreprises étrangères à la consommation du lieu, qu’il ne le sera dans une ville où les classes inférieures du peuple vivent uniquement de l’emploi des capitaux de cette espèce. Dans la première de ces villes, la fainéantise qu’y contracte la majeure partie du peuple, en vivant sur des dépenses de revenus, corrompt nécessairement l’industrie de ceux qu’entretiendrait l’emploi d’un capital, et fait qu’il y a moins d’avantages qu’ail­leurs à y employer des fonds. Il y avait à Édimbourg, avant l’union, peu de commerce et d’industrie. Quand le parlement d’Écosse ne s’assembla plus dans cette ville, quand elle cessa d’être la résidence nécessaire de la haute et de la petite noblesse[6] d’Écosse, elle commença à avoir quelque commerce et quelque industrie. Elle continue cepen­dant d’être encore la résidence des principales cours de justice d’Écosse, des chambres de la douane et de l’accise. Il s’y dépense donc encore une masse considérable de revenus ; aussi est-elle fort inférieure en commerce et en industrie à Glasgow, dont les habitants vivent principalement sur des emplois de capitaux. On a remarqué quel­quefois que les habitants d’un gros bourg, après de grands progrès dans l’industrie manufacturière, avaient tourné ensuite à la fainéantise et à la pauvreté, parce que quelque grand seigneur avait établi son séjour dans leur voisinage.
C’est donc la proportion existante entre la somme des capitaux et celle des reve­nus qui détermine partout la proportion dans laquelle se trouveront l’industrie et la fainéantise ; partout où les capitaux l’emportent, c’est l’industrie qui domine ; partout où ce sont les revenus, la fainéantise prévaut. Ainsi, toute augmentation ou diminu­tion dans la masse des capitaux tend naturellement à augmenter ou à diminuer réelle­ment la somme de l’industrie, le nombre des gens productifs et, par conséquent, la valeur échangeable du produit annuel des terres et du travail du pays, la richesse et le revenu réel de tous ses habitants.
Les capitaux augmentent par l’économie ; ils diminuent par la prodigalité et la mauvaise conduite[7].’ 
Tout ce qu’une personne épargne sur son revenu, elle l’ajoute à son capital ; alors, ou elle l’emploie elle-même à entretenir un nombre additionnel de gens productifs, ou elle met quelque autre personne en état de le faire, en lui prêtant ce capital moyennant un intérêt, c’est-à-dire une part dans les profits. De même que le capital d’un individu ne peut s’augmenter que par le fonds que cet individu épargne sur son revenu annuel ou sur ses gains annuels, de même le capital d’une société, lequel n’est autre chose que celui de tous les individus qui la composent, ne peut s’augmenter que par la même voie.
La cause immédiate de l’augmentation du capital, c’est l’économie, et non l’indus­trie. À la vérité, l’industrie fournit la matière des épargnes que fait l’économie ; mais, quelques gains que fasse l’industrie, sans l’économie qui les épargne et les amasse, le capital ne serait jamais plus grand.
L’économie, en augmentant le fonds destiné à l’entretien des salariés productifs, tend à augmenter le nombre de ces salariés, dont le travail ajoute à la valeur du sujet auquel il est appliqué ; elle tend donc à augmenter la valeur échangeable du produit annuel de la terre et du travail du pays ; elle met en activité une quantité additionnelle d’ industrie, qui donne un accroissement de valeur au produit annuel.
Ce qui est annuellement épargné est aussi régulièrement consommé que ce qui est annuellement dépensé, et il l’est aussi presque dans le même temps ; mais il est con­sommé par une autre classe de gens. Cette portion de son revenu qu’un homme riche dépense annuellement, est le plus souvent consommée par des bouches inutiles et par des domestiques, qui ne laissent rien après eux en retour de leur consommation. La portion qu’il épargne annuellement, quand il l’emploie immédiatement en capital pour en tirer un profit, est consommée de même et presque en même temps que l’autre, mais elle l’est par une classe de gens différente, par des ouvriers, des fabricants et arti­sans qui reproduisent avec profit la valeur de leur consommation annuelle. Supposons que le revenu de cet homme riche lui soit payé en argent. S’il l’eût dépensé en entier, tout ce que ce revenu aurait pu acheter en vivres, vêtements et logements, aurait été distribué parmi la première de ces deux classes de gens. S’il en épargne une partie, et que cette partie soit immédiatement employée comme capital, soit par lui-même, soit par quelque autre, alors ce qu’on achètera avec en vivres, vêtements et logement, sera nécessairement réservé pour l’autre classe. La consommation est la même, mais les consommateurs sont différents.
Un homme économe, par ses épargnes annuelles, non-seulement fournit de l’entre­tien à un nombre additionnel de gens productifs pour cette année ou pour la suivante, mais il est comme le fondateur d’un atelier public, et établit en quelque sorte un fonds pour l’entretien à perpétuité d’un même nombre de gens productifs. À la vérité, la destination et l’emploi à perpétuité de ce fonds ne sont pas toujours assurés par une loi expresse, une substitution ou un acte d’amortissement. Néanmoins, un principe très-puissant en garantit l’emploi : c’est l’intérêt direct et évident de chaque individu au­quel pourra appartenir dans la suite quelque partie de ce fonds. Aucune partie n’en pour­ra plus à l’avenir être détournée à un autre emploi qu’à l’entretien des salariés productifs, sans qu’il en résulte une perte évidente pour la personne qui en changerait ainsi la véritable destination.
C’est ce que fait le prodigue. En ne bornant pas sa dépense à son revenu, il entame son capital. Comme un homme qui dissipe à quelque usage profane les revenus d’une fondation pieuse, il paye des salaires à la fainéantise avec ces fonds que la frugalité de nos pères avait pour ainsi dire, consacrés à l’entretien de l’industrie. En diminuant la masse des fonds destinés à employer le travail productif, il diminue nécessairement, autant qu’il est en lui, la somme de ce travail qui ajoute une valeur au sujet auquel il est appliqué et, par conséquent, la valeur du produit annuel de la terre et du travail du pays, la richesse et le revenu réel de ses habitants. Si la prodigalité de quelques-uns n’était pas compensée par la frugalité des autres, tout prodigue, en nourrissant ainsi la paresse avec le pain de l’industrie, tendrait, par sa conduite, à appauvrir son pays.
Quand même toute la dépense du prodigue serait en consommation de marchan­dises faites dans le pays et nullement en marchandises étrangères, ses effets sur les fonds productifs de la société seraient toujours les mêmes. Chaque année, il y aurait une certaine quantité de vivres et d’habits qui auraient dû entretenir les salariés pro­ductifs, et qui auraient été employés à nourrir et vêtir des salariés non productifs. Chaque année, par conséquent, il y aurait quelque diminution dans la valeur qu’aurait eue sans cela le produit annuel de la terre et du travail du pays.
On peut dire, à la vérité, que cette dépense n’étant pas faite en denrées étrangères, et n’occasionnant aucune exportation d’or ni d’argent, il resterait dans le pays la même quantité d’espèces qu’ auparavant ; mais si cette quantité de vivres et d’habits ainsi consommés par des gens non productifs eût été distribuée entre des gens productifs, ceux-ci auraient reproduit, avec un profit en plus, la valeur entière de leur consom­mation. Dans ce cas comme dans l’autre, la même quantité d’argent serait également restée dans le pays, et de plus il y aurait eu une reproduction d’une valeur égale en choses consommables ; il y aurait eu deux valeurs dans ce dernier cas ; dans l’autre, il n’y en aura qu’une.
D’ailleurs, il ne peut pas rester longtemps la même quantité d’argent dans un pays où la valeur du produit annuel va en diminuant. L’argent n’a d’autre fonction que de faire circuler les choses consommables. C’est par son moyen que les vivres, les matiè­res et l’ouvrage fait se vendent et s’achètent, et qu’ils vont se distribuer à leurs con­sommateurs. Ainsi, la quantité d’argent qui peut annuellement être employée dans un pays est nécessairement déterminée par la valeur des choses consommables qui y circulent annuellement. Celles-ci consistent ou en produit immédiat de la terre et du travail du pays même, ou en quelque chose qui a été acheté avec partie de ce produit. Ainsi, leur valeur doit diminuer à mesure que diminue celle de ce produit et, avec leur valeur encore, la quantité d’argent qui peut être employée à les faire circuler. Mais l’argent qui, au moyen de cette diminution annuelle de produit, est annuellement jeté hors de la circulation intérieure ne restera pas inutile pour cela ; l’intérêt de quiconque le possède est qu’il soit employé. Or, n’ayant pas d’emploi au-dedans, il sera envoyé à l’étranger en dépit de toutes les lois et prohibitions, et il sera employé à y acheter des choses consommables qui puissent être de quelque usage dans l’intérieur. Son expor­ta­tion annuelle continuera à ajouter ainsi, pendant quelque temps, à la consommation annuelle du pays, quelque chose au-delà du produit annuel du même pays. Ce qui avait été épargné sur ce produit annuel, dans les jours de prospérité, et employé à acheter de l’or et de l’argent, contribuera pour quelque peu de temps à soutenir la consommation du pays dans les jours d’adversité ; dans ce cas, l’exportation de l’or et de l’argent n’est pas la cause, mais l’effet de la décadence du pays, et cette exportation peut même soulager pendant quelque temps sa misère au moment de sa décadence.
Au contraire, à mesure qu’augmente la valeur du produit annuel d’un pays, la quan­tité d’argent doit naturellement y augmenter aussi. La valeur des choses consom­mables qui doivent circuler annuellement dans la société étant plus grande, il faudra une plus grande somme d’argent pour les faire circuler. Ainsi, une partie de ce surcroît de produit sera naturellement employée à acheter, partout où l’on pourra s’en procurer, la quantité additionnelle d’or et d’argent nécessaire pour faire circuler le reste. L’augmentation de ces métaux sera, dans ce cas, l’effet et non la cause de la prospérité générale. Partout, l’or et l’argent s’achètent de la même manière. Au Pérou comme en Angleterre, le prix qu’on paye pour en avoir représente la nourriture, le vêtement et le logement, en un mot, le revenu et la subsistance de tous ceux dont le travail ou le capital s’emploie à les faire venir de la mine au marché. Le pays qui a de quoi payer ce prix ne sera jamais longtemps sans avoir la quantité de ces métaux dont il a besoin, et jamais aucun pays n’en retiendra longtemps la quantité qui ne lui est pas nécessaire[8].
Ainsi, de quelque manière que nous concevions la richesse et le revenu réel d’un pays, soit que nous les fassions consister, comme le simple bon sens paraît le dicter, dans la valeur du produit annuel de ses terres et de son travail, soit, comme le suppo­sent les préjugés vulgaires, que nous les fassions consister dans la quantité de métaux précieux qui y circulent[9] ; sous l’un ou l’autre de ces points de vue, tout prodigue paraît être un ennemi du repos public, et tout homme économe un bienfaiteur de la société.
Les effets d’une conduite peu sage sont souvent les mêmes que ceux de la prodigalité. Tout projet imprudent et malheureux en agriculture, en mines, en pêche­ries, en commerce ou manufactures, tend de même à diminuer les fonds destinés à l’entretien du travail productif. Quoique dans un projet de cette nature le capital ne soit consommé que par des gens productifs seulement, cependant, comme la manière imprudente dont on les emploie fait qu’ils ne reproduisent point la valeur entière de leur consommation, il résulte toujours quelque diminution dans ce qu’aurait été sans cela la masse des fonds productifs de la société.
Il est rare, à la vérité, que la prodigalité ou la conduite imprudente des individus dans leurs affaires puisse jamais beaucoup influer sur la fortune d’une grande nation, la profusion ou l’imprudence de quelques-uns se trouvant toujours plus que compen­sée par l’économie et la bonne conduite des autres.
Quant à la profusion, le principe qui nous porte à dépenser, c’est la passion pour les jouissances actuelles, passion qui est, à la vérité, quelquefois très-forte et très-difficile à réprimer, mais qui est, en général, passagère et accidentelle. Mais le prin­cipe qui nous porte à épargner, c’est le désir d’améliorer notre sort ; désir qui est en général, à la vérité, calme et sans passion, mais qui naît avec nous et ne nous quitte qu’au tombeau. Dans tout l’intervalle qui sépare ces deux termes de la vie, il n’y a peut-être pas un seul instant où un homme se trouve assez pleinement satisfait de son sort, pour n’y désirer aucun changement ni amélioration quelconque. Or, une augmen­tation de fortune est le moyen par lequel la majeure partie des hommes se propose d’améliorer son sort ; c’est le moyen le plus commun et qui leur vient le premier à la pensée ; et la voie la plus simple et la plus sûre d’augmenter sa fortune, c’est d’épar­gner et d’accumuler, ou régulièrement chaque année, ou dans quelques occasions extraordinaires, une partie de ce qu’on gagne. Ainsi, quoique le principe qui porte à dépenser l’emporte chez presque tous les hommes en certaines occasions, et presque en toutes occasions chez certaines personnes, cependant chez la plupart des hommes, en prenant en somme tout le cours de leur vie, il semble que le principe qui porte à l’économie, non-seulement prévaut à la longue, mais prévaut même avec force.
À l’égard de la conduite des affaires, le nombre des entreprises sages et heureuses est partout beaucoup plus considérable que celui des entreprises imprudentes et malheureuses. Malgré toutes nos plaintes sur la fréquence des banqueroutes, les mal­heu­reux qui tombent dans ce genre d’infortune ne sont qu’en bien petit nombre, com­parés à la masse des personnes engagées dans le commerce et dans les affaires de toute espèce ; ils ne sont peut-être pas plus d’un sur mille. La banqueroute est peut-être la plus grande calamité et la plus forte humiliation à laquelle puisse être exposé un innocent. Aussi, la majeure partie des hommes prennent-ils bien leurs précautions pour l’éviter. À la vérité, il y en a quelques-uns qui ne l’évitent pas, comme il y en a aussi quelques-uns qui ne peuvent venir à bout d’éviter la potence.
Les grandes nations ne s’appauvrissent jamais par la prodigalité et la mauvaise conduite des particuliers, mais quelquefois bien par celles de leur gouvernement. Dans la plupart des pays, la totalité ou la presque totalité du revenu public est employée à entretenir des gens non productifs. Tels sont les gens qui composent une cour nombreuse et brillante, un grand établissement ecclésiastique, de grandes flottes et de grandes armées qui ne produisent rien en temps de paix, et qui, en temps de guerre, ne gagnent rien qui puisse compenser la dépense que coûte leur entretien, même pendant la durée de la guerre. Les gens de cette espèce, ne produisant rien par eux-mêmes, sont tous entretenus par le produit du travail d’autrui. Ainsi, quand ils sont multipliés au-delà du nombre nécessaire, ils peuvent, dans une année, consom­mer une si grande part de ce produit, qu’ils n’en laissent pas assez de reste pour l’entretien des ouvriers productifs, qui devraient le reproduire pour l’année suivante. Le produit de l’année suivante sera donc moindre que celui de la précédente, et si le même désordre allait toujours continuant, le produit de la troisième serait encore moindre que celui de la seconde. Ces hommes non productifs, qui ne devaient être entretenus que sur une partie des épargnes des revenus des particuliers, peuvent quel­quefois consommer une si grande portion de la totalité de ces revenus, et par là forcer tant de gens à entamer leurs capitaux et à prendre sur le fonds destiné à l’entretien du travail productif, que toute la frugalité et la sage conduite des individus ne puissent jamais suffire à compenser les vides et les dommages qu’occasionne, dans le produit annuel, cette dissipation violente et forcée des capitaux.
L’expérience semble pourtant nous faire voir que, dans presque toutes les circons­tances, l’économie et la sage conduite privées suffisent, non-seulement pour com­penser l’effet de la prodigalité et de l’imprudence des particuliers, mais même pour balancer celui des profusions excessives du gouvernement. Cet effort constant, uniforme et jamais interrompu de tout individu pour améliorer son sort ; ce principe, qui est la source primitive de l’opulence publique et nationale, aussi bien que de l’opulence privée, a souvent assez de puissance pour maintenir, en dépit des folies du gouvernement et de toutes les erreurs de l’administration, le progrès naturel des cho­ses vers une meilleure condition. Semblable à ce principe inconnu de vie, que portent avec elles les espèces animales, il rend souvent à la constitution de l’individu la santé et la vigueur, non-seulement malgré la maladie, mais même en dépit des absurdes ordonnances du médecin. 
Pour augmenter la valeur du produit annuel de la terre et du travail dans une nation, il n’y a pas d’autres moyens que d’augmenter, quant au nombre, les ouvriers productifs, ou d’augmenter, quant à la puissance, la faculté productive des ouvriers précédem­ment employés. À l’égard du nombre des ouvriers productifs, il est évident qu’il ne peut jamais beaucoup s’accroître que par suite d’une augmentation des capi­taux ou des fonds destinés à les faire vivre. Quant à la puissance de produire, elle ne peut s’augmenter dans un même nombre d’ouvriers, qu’autant que l’on multiplie ou que l’on perfectionne les machines et instruments qui facilitent et abrègent le travail, ou bien qu’autant que l’on établit une meilleure distribution ou une division mieux entendue dans le travail. Dans l’un et l’autre cas, il faut presque toujours un surcroît de capital. Ce n’est qu’à l’aide d’un surcroît de capital que l’entrepreneur d’un genre d’ouvrage quelconque pourra pourvoir ses ouvriers de meilleures machines ou établir entre eux une division de travail plus avantageuse. Quand l’ouvrage à faire est com­posé de plusieurs parties, pour tenir chaque ouvrier constamment occupé à la tâche particulière, il faut un capital beaucoup plus étendu que lorsque chaque ouvrier est employé indifféremment à toutes les parties de l’ouvrage, à mesure qu’elles sont à faire. Ainsi, lorsque nous comparons l’état d’une nation à deux périodes différentes, et que nous trouvons que le produit annuel de ses terres et de son travail est évidemment plus grand à la dernière de ces deux périodes qu’à la première, que ses terres sont mieux cultivées, ses manufactures plus multipliées et plus florissantes, son commerce plus étendu, nous pouvons être certains que, pendant l’intervalle qui a séparé ces deux périodes, son capital a nécessairement augmenté, et que la bonne conduite de quel­ques personnes y a plus ajouté que la mauvaise conduite des autres ou les folies et les erreurs du gouvernement n’en ont retranché. Or, nous verrons que telle a été la marche de presque toutes les nations, dans les temps où elles ont joui de quelque paix et de quelque tranquillité, même pour celles qui n’ont pas eu le bonheur d’avoir le gou­vernement le plus prudent et le plus économe. À la vérité, pour porter là-dessus un jugement un peu sûr, il faut comparer l’état du pays à des périodes assez éloignées l’une de l’autre. Les progrès s’opèrent si lentement pour l’ordinaire, que dans des pério­des rapprochées, non-seulement l’avancement n’est pas sensible, mais que sou­vent le déclin de quelque branche particulière d’industrie, ou de certaine localité du pays (choses qui peuvent quelquefois arriver dans le temps même où le pays en général est dans une grande prospérité), pourrait faire soupçonner que les richesses et l’industrie générales sont en train de déchoir.
En Angleterre, par exemple, le produit de la terre et du travail est certainement beaucoup plus grand qu’il ne l’était, il y a un peu plus d’un siècle, à la restauration de Charles II. Quoique aujourd’hui il y ait, à ce que je présume, très-peu de gens qui révoquent ce fait en doute, cependant, pendant le cours de cette période-là, il ne s’est guère écoulé cinq années de suite dans lesquelles on n’ait pas publié quelque livre ou quelque pamphlet, écrit même avec assez de talent pour faire impression dans le publie, où l’auteur prétendait démontrer que la richesse de la nation allait rapidement vers son déclin, que le pays se dépeuplait, que l’agriculture était négligée, les manu­fac­tures tombées et le commerce ruiné ; et ces ouvrages n’étaient pas tous des libelles enfantés par l’esprit de parti, cette malheureuse source de tant de productions vénales et mensongères. Beaucoup d’entre eux étaient écrits par des gens fort intelligents et de bonne foi, qui n’écrivaient que ce qu’ils pensaient, et uniquement parce qu’ils le pensaient.
En Angleterre encore, le produit annuel de la terre et du travail était certainement beaucoup plus grand à la restauration que nous ne le pouvons supposer, environ cent ans auparavant, à l’avènement d’Élisabeth. À cette dernière époque encore, il y a tout lieu de présumer que le pays était beaucoup plus avancé en amélioration, qu’il ne l’avait été environ un siècle auparavant, vers la fin des querelles entre les maisons d’York et de Lancastre. Alors même, il était vraisemblablement en meilleure situation qu’il n’avait été à l’époque de la conquête normande, et à celle-ci encore, que durant les désordres de l’heptarchie saxonne[10]. Enfin, à cette dernière période, c’était un pays assurément plus avancé que lors de l’invasion de jules César, où les habitants étaient à peu près ce que sont les sauvages du nord de l’Amérique.
Dans chacune de ces périodes cependant, il y eut non-seulement beaucoup de prodigalité particulière et générale, beaucoup de guerres inutiles et dispendieuses, de grandes quantités du produit annuel détournées de l’entretien des gens productifs, pour en entretenir de non productifs, mais il y eut même quelquefois, dans les désordres des guerres civiles, une destruction et un anéantissement si absolus des capitaux, qu’on peut croire que non-seulement l’accumulation des richesses en a été retardée, comme il n’y a pas à en douter, mais que même le pays en est resté à la fin de cette période, plus pauvre qu’il n’était au commencement. Même dans la plus heureuse et plus brillante de toutes ces périodes, celle qui s’est écoulée depuis la restauration, combien n’est-il pas survenu de troubles et de malheurs qui, si l’on eût pu les prévoir, auraient paru devoir entraîner à leur suite non-seulement l’appauvrisse­ment du pays, mais même sa ruine totale ! L’incendie et la peste de Londres, les deux guerres de Hollande, les troubles de la révolution, la guerre d’Irlande, les quatre guerres si dispendieuses avec la France en 1688, 1701, 1742, 1756, et en outre les deux rébellions de 1715 et 1745. Dans le cours des quatre guerres de France, la nation a contracté plus de 145 millions de liv. sterling de dettes, outre toutes les autres dépenses extraordinaires que ces guerres ont occasionnées annuellement, de manière qu’on ne peut pas compter pour le tout moins de 200 millions de liv. sterling.
Cette immense portion du produit annuel des terres et du travail du pays a été employée, en différentes circonstances, depuis la révolution, à entretenir un nombre extraordinaire de salariés non productifs. Or, si toutes ces guerres n’eussent pas fait prendre cette direction particulière à un aussi énorme capital, la majeure partie en aurait été naturellement consacrée à l’entretien de bras productifs, dont le travail aurait remplacé, avec un profit en plus, la valeur totale de leur consommation. Chaque année, la valeur du produit annuel des terres et du travail du pays en aurait consi­dé­rablement augmenté, et l’augmentation de chaque année aurait contribué à augmenter encore davantage le progrès de l’année suivante. On aurait bâti plus de maisons, on aurait amélioré plus de terres, et celles qui étaient déjà améliorées auraient été mieux cultivées ; il se serait établi un plus grand nombre de manufactures, et celles déjà établies auparavant auraient fait plus de progrès ; enfin, il n’est peut-être pas très-facile d’imaginer jusques à quel degré d’élévation se fussent portés la richesse et le revenu réel du pays[11]. 
Mais, quoique les profusions du gouvernement aient dû, sans contredit, retarder le progrès naturel de l’Angleterre vers l’amélioration et l’opulence, elles n’ont pu néan­moins venir à bout de l’arrêter. Le produit annuel des terres et du travail y est aujourd’hui indubitablement beaucoup plus grand qu’il ne l’était ou à l’époque de la restauration, ou à celle de la révolution. Il faut donc, par conséquent, que le capital qui sert annuellement à cultiver ces terres et à maintenir ce travail soit aussi beaucoup plus grand. Malgré toutes les contributions excessives exigées par le gouvernement, ce capital s’est accru insensiblement et dans le silence par l’économie privée et la sage conduite des particuliers, par cet effort universel, constant et non interrompu de chacun d’eux pour améliorer leur sort individuel. C’est cet effort sans cesse agissant sous la protection de la loi, et que la liberté laisse s’exercer dans tous les sens et com­me il le juge à propos ; c’est lui qui a soutenu les progrès de l’Angleterre vers l’amélio­ration et l’opulence, dans presque tous les moments, par le passé, et qui fera de même pour l’avenir, à ce qu’il faut espérer. Et pourtant, si l’Angleterre n’a jamais eu le bon­heur d’avoir un gouvernement très-économe, l’économie n’a jamais été non plus dans aucun temps la vertu dominante de ses habitants. C’est donc une souveraine inconséquence et une extrême présomption, de la part des princes et des ministres, que de prétendre surveiller l’économie des particuliers et restreindre leur dépense par des lois somptuaires ou par des prohibitions sur l’impor­tation des denrées étrangères de luxe. Ils sont toujours, et sans exception, les plus grands dissipateurs de la société. Qu’ils surveillent seulement leurs propres dépenses, et ils pourront s’en reposer sans crainte sur chaque particulier pour régler la sienne. Si leurs propres dissipations ne viennent pas à bout de ruiner l’État, certes celles des sujets ne le ruineront jamais.
Si l’économie augmente la masse générale des capitaux, et si la prodigalité la diminue, la conduite de ceux qui dépensent tout juste leur revenu, sans rien amasser ni sans entamer leurs fonds, ne l’augmente ni ne la diminue. En outre, il y a certaines manières de dépenser, qui semblent contribuer plus que d’autres à l’accroissement de l’opulence générale.
Le revenu d’un particulier peut se dépenser, ou en choses qui se consomment immédiatement et pour lesquelles la dépense d’un jour ne peut être ni un soulagement ni une augmentation pour celle d’un autre jour, ou bien en choses plus durables et qui, par conséquent, peuvent s’accumuler, et pour lesquelles la dépense de chaque jour peut, au choix du maître, ou alléger la dépense du jour suivant, ou la relever et la rendre plus apparente et plus magnifique. Par exemple, un homme riche peut dépen­ser son revenu à tenir une table abondante et somptueuse, à entretenir un grand nom­bre de domestiques, à avoir une multitude de chiens et de chevaux ; ou bien, en se contentant d’une table frugale et d’un domestique peu nombreux, il peut employer la plus grande partie de son revenu à embellir ses maisons de ville et de campagne, à élever des bâtiments pour son agrément ou sa commodité, à acheter des meubles pour l’usage ou pour la décoration, à faire des collections de livres, de statues, de tableaux. Il peut dépenser ce revenu en choses plus frivoles, en bijoux, en colifichets ingénieux de différentes espèces et, enfin, dans la plus vaine de toutes les frivolités, en une immense garde-robe de magnifiques habits, comme le ministre et le favori d’un grand prince mort depuis peu d’années[12]. Que deux hommes égaux en fortune dépensent chacun leur revenu, l’un de la première de ces deux manières, l’autre de la seconde, la magnificence de celui dont la dépense aurait consisté surtout en choses durables, irait continuellement en augmentant, parce que la dépense de chaque jour contribuerait en quelque chose à rehausser et à agrandir l’effet de la dépense du jour suivant ; la ma­gni­ficence de l’autre, au contraire, ne serait pas plus grande à la fin de sa carrière qu’au commencement. Le premier se trouverait aussi, à la fin, le plus riche des deux. Il se trouverait avoir un fonds de richesses d’une espèce ou d’une autre, qui, sans valoir ce qu’elles auraient coûté, ne laisseraient pas cependant de valoir toujours beaucoup. De la dépense de l’autre, il ne resterait ni indices ni vestiges quelconques, et l’effet de dix ou de vingt ans de profusion serait aussi complètement anéanti que si elles n’eussent jamais eu lieu.
Si l’une de ces deux manières de dépenser est plus favorable que l’autre à l’opulence de l’individu, elle l’est pareillement à celle du pays. Les maisons, les meu­bles, les vêtements du riche, au bout de quelque temps, servent aux classes moyennes ou inférieures du peuple. Celles-ci sont à même de les acheter quand la classe supérieure est lasse de s’en servir ; quand cette manière de dépenser devient générale parmi les gens de haute fortune, la masse du peuple se trouve successivement mieux fournie de tous les genres de commodités. Il n’est pas rare de voir, dans les pays qui ont été longtemps riches, les classes inférieures du peuple en possession de logements et de meubles encore bons et entiers, qui n’auraient jamais été ni construits ni fabriqués pour l’usage de ceux qui les possèdent. Ce qui était autrefois un château de la famille de Seymour est à présent une auberge sur la route de Bath. Le lit de noces de Jacques Ier, roi d’Angleterre, qui lui fut apporté de Danemark par la reine son épouse, comme un présent digne d’être offert à un souverain par un autre souverain, servait d’ornement, il y a quelques années, dans un cabaret à bière de Dumferline. Dans quelques anciennes villes, dont l’état a été longtemps stationnaire ou a été quelque peu en déclinant, vous trouverez quelquefois à peine une seule maison qui ait pu être bâtie pour l’espèce de gens qui l’habitent. Si vous entrez aussi dans ces maisons, vous y trouverez encore fort souvent d’excellents meubles, quoique de forme antique, mais très-bons pour le service, et qui n’ont pas été faits pour ceux qui s’en servent. De superbes palais, de magnifiques maisons de campagne, de grandes biblio­thèques, de riches collections de statues, de tableaux et d’autres curiosités de l’art et de la nature, font souvent l’ornement et la gloire, non-seulement de la localité qui les possède, mais même de tout le pays. Versailles embellit la France et lui fait honneur, comme Stowe et Wilton à l’Angleterre. L’Italie attire encore en quelque sorte les respects du monde par la multitude de monuments qu’elle possède en ce genre, quoi­que l’opulence qui les a fait naître ait bien déchu et que le génie qui les a créés semble tout à fait éteint, peut-être faute de trouver autant d’emploi.
De plus, la dépense qu’on place en choses durables est favorable, non-seulement à l’accumulation des richesses, mais encore à l’économie. Si la personne qui fait cette dépense la portait une fois jusqu’à l’excès, elle peut aisément se réformer sans s’expo­ser aux critiques du public. Mais réduire de beaucoup le nombre de ses domestiques, réformer une table somptueuse pour en tenir une simple et frugale, mettre bas l’équipage après l’avoir eu quelque temps, tous ces changements ne peuvent manquer d’être observés par les voisins, et ils semblent porter avec eux un aveu tacite qu’on s’est précédemment conduit avec peu de sagesse. Aussi, parmi ceux qui ont été une fois assez imprudents pour se laisser emporter trop loin dans ce genre de dépense, y en a-t-il bien peu qui aient par la suite le courage de revenir sur leurs pas avant d’y être contraints par la banqueroute et le désastre complet de leur fortune. Mais qu’une personne se soit une fois laissée aller à de trop fortes dépenses en bâtiments, en meu­bles, en livres ou en tableaux, elle pourra très-bien changer de conduite, sans qu’on en infère pour cela qu’elle ait jamais manqué de prudence. Ce sont des choses dans lesquelles la dépense précédemment faite est une raison pour qu’il soit inutile d’en faire davantage ; et quand une personne s’arrête tout à coup dans ce genre de dépense, rien n’annonce que ce soit pour avoir dépassé les bornes de sa fortune, plutôt que pour avoir satisfait ce genre de fantaisie.
D’un autre côté, la dépense consacrée à des choses durables fait vivre ordinaire­ment une bien plus grande quantité de gens que celle qu’on emploie à tenir la table la plus nombreuse. Sur deux ou trois cents livres pesant de vivres qui seront quelquefois servies dans un grand repas, la moitié peut-être est jetée, et il y en a toujours une grande quantité dont on fait abus ou dégât. Mais si la dépense de ce festin eût servi à faire travailler des maçons, des charpentiers, des tapissiers, des artistes, la même valeur en vivres se serait trouvée distribuée entre un bien plus grand nombre de gens qui les eussent achetés livre par livre, et n’en auraient ni gâté ni laissé perdre une once. D’ailleurs, une dépense ainsi faite entretient des gens productifs ; faite de l’autre manière, elle nourrit des gens inutiles. Par conséquent, l’une augmente la valeur échangeable du produit annuel des terres et du travail du pays, et l’autre ne l’augmente pas.
Il ne faut pourtant pas croire que je veuille dire par là que l’un de ces genres de dépense annonce plus de générosité et de noblesse dans le caractère que l’autre. Quand un homme riche dépense principalement son revenu à tenir grande table, il se trouve qu’il partage la plus grande partie de son revenu avec ses amis et les personnes de sa société ; mais quand il l’emploie à acheter de ces choses durables dont nous avons parlé, il le dépense alors souvent en entier pour sa propre personne, et ne donne rien a qui que ce soit sans recevoir l’équivalent. Par conséquent, cette dernière façon de dépenser, quand elle porte sur des objets de frivolité, sur de petits ornements de parure et d’ameublement, sur des bijoux, des colifichets et autres bagatelles, est sou­vent une indication non-seulement de légèreté dans le caractère, mais même de mesquinerie et d’égoïsme. Tout ce que j’ai prétendu dire, c’est que l’une de ces maniè­res de dépenser, occasionnant toujours quelque accumulation de choses précieuses, étant plus favorable à l’économie privée et, par conséquent, à l’accroissement du capital de la société ; enfin, servant à l’entretien des gens productifs, plutôt que des non productifs, tendait plutôt que l’autre à l’augmentation et aux progrès de la fortune publique.



Les fonds prêtés à intérêt sont toujours regardés par le prêteur comme un capital.
Il s’attend qu’à l’époque convenue ces fonds lui seront rendus et qu’en même temps l’emprunteur lui payera une certaine rente annuelle pour les avoir eus à sa disposition. L’emprunteur peut disposer de ses fonds ou comme d’un capital, ou comme de fonds destinés à servir immédiatement à sa consommation : s’il s’en sert comme d’un capital, il les emploie à faire subsister des ouvriers productifs qui en
 
 
 
↑ Dans ce chapitre, Smith distingue deux espèces de travail ; il qualifie l’un de productif, et l’autre de non productif, et il pense que le premier est plus favorable que l’autre à l’accroissement de la richesse nationale. Cette distinction semble, à quelques égards, peu conciliable avec les principes établis par l’auteur lui-même sur la nature du travail ; les caractères sur lesquels il veut fonder cette distinction ne sont pas assez nettement tracés pour.qu’on puisse faire la séparation qu’il indique ; et enfin, les conséquences qu’il voudrait tirer de cette définition sont susceptibles d’être contestées.
La richesse, a-t-il dit, consiste dans le pouvoir d’appliquer le travail d’autrui, moyennant salaire, à ses propres besoins, commodités et jouissances. Donc tout travail salarié est essentiellement productif d’une chose utile, commode ou agréable pour celui qui le paye, sans quoi celui-ci ne le payerait pas ; ce travail est non moins essentiellement productif d’un salaire pour celui qui l’exécute, sans quoi il ne travaillerait pas. Tout travail salarié (et c’est le seul dont s’occupe l’économie politique) est un service, et l’utilité ou l’agrément que procure ce service, voilà le produit du travail, et il ne peut en avoir d’autre. Quelquefois le travail est directement et immédiatement payé par celui qui en consomme le produit, et c’est ce qui a toujours lieu quand le service du travailleur est rendu, sans nul intermédiaire, à celui qui paye ce service. Plus souvent, le travail est mis en œuvre par un tiers, qui ne se propose nullement d’en consommer le produit, mais qui entend le faire payer par un autre en se réservant un profit pour lui-même. Dans ce second cas, il y a un entrepreneur de travail qui fait l’avance du salaire, avec l’Intention de s’en faire rembourser par celui auquel est définitivement destiné le produit du travail. Ce cas ne peut avoir lieu qu’autant que l’utilité ou l’agrément que procurera le travail résulte de la préparation ou du transport de quelque objet matériel. Hais, dans l’un comme dans l’autre cas, on ne paye le travail qu’en raison du prix qu’on attache à l’utilité ou a l’agrément qu’il procure, et il est indifférent à celui qui veut satisfaire son besoin ou son goût, que cette satisfaction procède ou non d’un objet matériel.
Ce sont cependant ces deux cas qui ont paru à Smith assez distincts en eux-mêmes et par leurs conséquences, pour devoir fonder la distinction qu’il a établie entre le travail productif et le travail non productif. Il appelle travail productif celui qui ajoute une valeur à celle du sujet sur lequel il s’exerce ; tel est, dit-il, en général, le travail des ouvriers de manufacture, qui ajoute à la valeur de la matière celle de son salaire et du profit de son maître. Le travail du domestique, au contraire, quoique également salarié, n’ajoute à la valeur de rien, et la valeur de ce que ce domestique a consommé ne se retrouve nulle part. C’est pour cette raison que ce dernier genre de travail est distingué du premier et est réputé improductif. On s’enrichit, dit Smith, à employer une multitude d’ouvriers fabricants ; on s’appauvrit à entretenir une multitude de domestiques. Cependant il reconnait que le travail des domestiques a sa valeur et mérite son salaire aussi bien que le travail de l’ouvrier ; mais, ajoute-t-il, le travail de l’ouvrier se fixe et se réalise sur un objet ou sur une chose vénale, qui dure au moins quelque temps après que le travail a cessé. Enfin, le caractère qui distingue le travail non productif, c’est de périr à l’instant même où le service est rendu, de ne laisser après soi aucune valeur vénale avec laquelle on puisse acheter un autre service.
Cette définition comprend quelques-unes des professions les plus utiles, et même celles qui sont les plus importantes et les plus respectables dans la société, savoir, les membres du gouvernement, les magistrats, les ecclésiastiques, les militaires, les légistes, les médecins, les savants et les gens de lettres. Elle s’applique aussi à d’autres professions moins utiles et moins élevées, mais qui contribuent beaucoup à l’agrément de la vie, tels que les musiciens, les comédiens, les danseurs, etc.
Toute la distinction, comme on voit, porte sur cette circonstance: c’est que le travail des ouvriers et artisans se réalise sur une valeur vénale, et que le travail des autres personnes placées dans la classe non productive est complètement éteint dès qu’il a été exécuté. Cette différence toutefois n’est pas autre chose que celle qui, par la nature même des choses, existe entre la production et la consommation. Les ouvriers de manufacture travaillent pour un maître qui ne consomme pas, mais qui veut vendre leur produit. Si ce travail n’était pas de nature à former une valeur vénale, s’il ne se fixait pas sur la matière confiée à l’ouvrier, s’il périssait sous la main de celui-ci, il ne pourrait pas être l’objet d’une spéculation de manufacture. Ce travail de l’ouvrier de fabrique n’est pas un des procédés de la production, une des opérations nécessaires pour que la matière devienne un objet propre à la consommation. Quant au travail dont le consommateur reçoit immédiatement le produit, il est tout simple que ce produit périsse à l’instant même où ce travail est exécuté, parce que c’est le propre de toute consommation de détruire, et que la jouissance du consommateur résulte de cette destruction même.
On a dit que les objets immatériels étaient seuls susceptibles d’être accumulés, et de former ainsi un capital servant à accroître la production future. Il est vrai que si l’augmentation des demandes de la consommation encourage l’industrie à déployer de nouveaux efforts et ouvre au travail de nouveaux emplois, alors un surcroit de capital devient nécessaire pour mettre et tenir en activité ce surcroit d’industrie et de travail ; il est également vrai que ce capital ne peut se composer que d’objets matériels accumulés. Mais ce qu’il est essentiel d’observer, c’est que ce ne sont pas indistinctement tous les produits matériels qui peuvent faire fonction de capital, et il y a une grande partie de ces produits matériels qui n’y seraient nullement propres. Les matières premières, les articles nécessaires à la subsistance et au vêtement, tels sont les seuls produits avec lesquels on puisse entretenir des ouvriers. Vainement aurait-on accumulé parle travail réputé productif, des soieries, des rubans, des gazes, des mousselines, des broderies, des dentelles, de la parfumerie, etc., si toutes ces choses ne peuvent trouver des acheteurs qui aient moyen de les payer et de donner en retour les matières premières et les vivres ; une telle accumulation sera totalement inutile pour la production future. De quelque côté qu’on se tourne dans ce cercle de raisonnements, on trouve toujours en face de soi ce principe invariable, c’est que le produit, quel qu’il soit, n’a de valeur qu’autant qu’un consommateur est prêt à se présenter pour le payer par un équivalent, lequel équivalent n’a lui-même de valeur qu’autant qu’il est demandé.
Ce qui importe vraiment à la société, c’est de posséder un capital suffisant pour entretenir la totalité du travail que ses membres peuvent demander et payer ; ce qui lui importe, c’est que ce capital suit épargné et accumulé de manière à s’accroitre graduellement à proportion que les demandes de la consommation provoquent l’activité d’une plus grande quantité de travail, et par conséquent l’emploi de plus de capital ; mais ce qui n’importe nullement à la société, c’est que ce capital soit épargné et accumulé par telles ou telles mains, par celles qui ont concouru directement à le produire, ou par d’autres dans lesquelles il est parvenu par voie d’échange ou en retour de services rendus. Garnier.
↑ Cette distinction trop absolue entre le travail productif et le travail improductif est regardée aujourd’hui comme une capitale erreur. Smith ne songeait qu’au travail matériel, tout en reconnaissant d’ailleurs l’importance des services que la plupart des agents qu’il regarde comme improductifs rendent à la société. Il compare justement un homme qui a fait l’apprentissage d’une industrie difficile et délicate, au prix de beaucoup de temps et de travail, à une machine coûteuse, dont le propriétaire a droit à des profits plus élevés, en conséquence du grand capital qu’elle représente. Le travail qui a donné à l’homme dont parle Smith cette éducation précieuse, a donc été un travail productif, comme celui qui a créé la machine. Tout travail utile est donc un travail productif. La société ne consomme pas seulement des produits matériels; elle a besoin des jouissances de l’intelligence, des nobles plaisirs des arts, de la protection des magistrats, tout aussi bien que de pain et de vêtements. Smith ne l’ignorait pas, et, dans sa fameuse distinction, il a commis plutôt une erreur de mot qu’une erreur de pensée. En rectifiant cette erreur, il faut prendre garde de tomber dans la même faute, et de pousser la rigueur de la démonstration jusqu’à d’insignifiantes subtilités. Parce que le Dr Smith a méconnu le caractère productif de certains travaux, il ne faut pas voir partout des producteurs, et la distinction du fondateur de l’Économie politique, pour être trop absolue, n’en est pas moins vraie en partie. La science ne doit donner le nom de travaux productifs qu’à ceux qui ont pour objet de satisfaire des besoins réels et légitimes, soit matériels, soit immatériels. A. B.
↑ Nous ne voulons pas rappeler à ce sujet les vieilles idées des physiocrates du dix-huitième siècle sur le produit net. Quoique l’auteur ne les partage point, on sent que ces idées ont exercé un moment de l’influence sur son esprit. L’expérience et l’observation en ont fait justice. A. B.
↑ C’est-à-dire, des tenanciers que le propriétaire peut renvoyer à sa volonté.
↑ Le commerce de Paris dépasse aujourd’hui non-seulement l’enceinte de la ville, mais encore la frontière de l’État. On peut évaluer à plus de cent millions de francs les produits que cette capitale exporte dans les départements ou à l’étranger. A. B.
↑ La haute noblesse, nobility, comprend toutes les personnes qualifiées au-dessus du titre de chevalier, tels que ducs, marquis, comtes, vicomtes et barons. La petite noblesse, gentry ou gentility, comprend les chevaliers des différents ordres, baronnets, etc.* Garnier.
↑ La crainte de manquer, l’inquiétude sur l’avenir, le désir de pourvoir d’avance aux chances incertaines et imprévues, sont une de ces dispositions naturelles de l’homme, dont il ne faut pas chercher la cause ailleurs que dans la constitution même de l’individu. Il en résulte que dans quelque condition que l’homme soit placé, il est enclin à ne pas consommer sur-le-champ tout ce dont il pourrait disposer, et qu’il s’arrange de manière à mettre en réserve pour le lendemain une portion quelconque de sa provision du jour. Ainsi tout ce qui est produit n’est pas immédiatement détruit par la consommation, et il reste un excédant dont se un fonds d’accumulation qui va toujours se grossissant de plus en plus. Il n’y a et il ne peut y avoir d’autre cause directe de l’accroissement progressif de la masse totale des objets consommables dans une nation. Quelle que puisse être la faculté productive du travail, quelque abondants que soient les revenus annuels, si tous ces revenus étaient consommés aussi rapidement qu’ils sont produits, la somme des richesses existantes dans la société serait à la fin de l’année ce qu’elle était au commencement. Si le propriétaire foncier consomme dans le cours de l’année la totalité de son revenu, comme il a le droit de le faire et comme il le peut sans s’appauvrir ; si le fabricant ou commerçant consomme tous ses profits aussitôt qu’ils lui sont acquis, et l’ouvrier tous ses salaires à mesure qu’il les reçoit, quelque grand que soit ce revenu, quelque hauts que soient ces profits et ces salaires, la société aura été abondamment pourvue, mais la richesse nationale n’aura pas reçu la plus légère augmentation.
Il s’en faut de beaucoup qu’il en soit ainsi. Le propriétaire même le plus porté à la dépense, l’homme industrieux le plus disposé à se donner toutes ses commodités, mettra toujours de côté une portion de ce qu’il avait droit de consommer. On peut même dire que s’il ne le faisait pas, il n’aurait pas satisfait tous ses besoins, car ce penchant à l’épargne lui est naturel comme tous ses autres besoins. Plus la société est civilisée, plus ce penchant est généralement senti, à quelques exceptions près, dont l’effet est bien plus que compensé par la parcimonie des avares, qui portent leurs épargnes fort au delà de ce que suggère la prévoyance commune, et qui s’exercent à se faire une jouissance des plus dures privations. Les liens de famille, les affections et les devoirs qui en sont la suite, ajoutent beaucoup à ce penchant. C’est alors que l’homme se complait à enrichir un avenir auquel il se sent attaché par les plus doux sentiments de la nature.
C’est sous ce point de vue que Smith a considéré l’épargne faite par les particuliers, et qu’il en a exposé les effets sur l’accroissement de la richesse publique. Mais quelques écrivains récents, en se méprenant complètement sur le sens du mot épargne, ont imputé à l’auteur des idées aussi fausses que contradictoires. Ils ont cru voir dans sa doctrine le précepte de ne guère consommer et de beaucoup produire.
L’épargne faite pour s’enrichir et par des vues d’économie ne doit pas être confondue avec la frugalité ou l’abstinence absolue. Celle-ci opère dans la consommation un vide qu’elle ne remplit par aucune autre demande ; mais l’épargne ne diminue nullement la consommation générale, et, loin d’y porter atteinte, non plus qu’à la production, elle contribue le plus souvent à les encourager et à les accroître l’une et l’autre.
Ainsi, le particulier qui, par des principes moraux ou religieux portés jusque au rigorisme le plus outré, s’impose des privations continues, sans autre but que la satisfaction de remplir la règle qu’il s’est prescrite, opère réellement un vide dans la consommation, parce qu’en se privant il n’a point en vue de se ménager la jouissance de quelque autre produit du travail et de l’industrie. Il n’épargne rien, car il n’a rien à épargner ; il s’abstient sans mettre en réserve. De tels sectaires ne seraient pas des propriétaires vigilants, occupés à défricher et à améliorer; ce ne serait pas parmi eux qu’il faudrait chercher des entrepreneurs actifs, d’habiles commerçants ni des ouvriers laborieux. Ils formeraient dans la population une classe inerte, qui ne prendrait presque aucune part dans le mouvement général du travail et de l’industrie.
Les lois somptuaires agissent de la même manière sur la production, et c’est à cette source même qu’elles attaquent la consommation. Que, dans un pays riche ou en train de le devenir, des règlements d’administration publique interdisent tout à coup l’usage des soieries, des dentelles, des riches tissus, des draperies fines, des bijoux, des carrosses, des festins, des spectacles, etc., non-seulement ces règlements réduiront à l’inactivité tout le travail et l’industrie qui se seraient exercés à produire les articles compris dans la prohibition, non-seulement ils fermeront ces emplois aux capitaux qui se seraient portés dans toutes ces branches, mais de plus ils auront l’effet de détruire, parmi toutes, les personnes que leur goût aurait portées à consommer ces sortes de richesses, le mobile principal qui excite à produire et à accroître ses revenus ; car on ne cherche à s’enrichir que pour jouir de ses richesses.
C’est ainsi qu’opéré la diminution de la dépense des particuliers, quand cette diminution procède de causes qui réagissent sur la production.
L’épargne d’économie et de prévoyance, qui est la seule dont Smith se soit occupé en cet endroit, est d’une tout autre nature, et elle opère d’une manière directement contraire. Celui qui épargne dans la vue d’améliorer sa fortune ne s’impose pas une privation absolue, et s’il s’abstient d’une jouissance, ce n’est que pour s’en ménager une autre qui est plus à sa convenance. Il ne renonce point à la chose qu’il ne veut pas consommer, car il entend bien en consommer toute la valeur. Il ne fait que vendre à un autre le droit de consommer à sa place. Il y a dans le voisinage de Paris des propriétaires de vergers et de jardins qui, dans les années où les fruits sont rares et chers, se privent de manger ceux qu’ils recueillent et les portent à la halle, où ces fruits vont chercher d’autres consommateurs qui consentent à en donner un haut prix. Si la chose épargnée ne trouvait pas un consommateur, le but de l’épargne serait manqué, et il n’y aurait aucun intérêt à épargner. Il n’y a pas absence, mais il y a échange de consommation, et cet échange a, comme tous les autres, l’effet de multiplier les occasions de produire. Garnier.
↑ Voyez livre IV, chap. i.
↑ Ces préjugés sont combattus dans le livre IV, notamment dans le premier chapitre.
↑ Époque de l’histoire d’Angleterre, où ce pays était divisé en sept royaume. Ils furent réunis en une monarchie par Egbert, en 827.
↑
La guerre d’Amérique et les dernières guerres avec la France occasionnèrent une dépense de sang et de trésors qui n’a pas d’égale dans l’histoire du monde.
La somme de la dette non rachetée, consolidée et non consolidée, qui s’élevait environ à 145 millions sterling en 1772, s’élève aujourd’hui (1838) à 785 millions sterling, et en outre des sommes immenses obtenues par les emprunts, le produit brut des taxes levées dans la Grande-Bretagne et en Irlande pendant la dernière guerre dépasse la somme énorme de 1300 millions sterling ! Et cependant la population, les manufactures, l’agriculture, le commerce n’en firent pas moins des progrès plus rapides qu’ils n’avaient fait jusque-là. L’exécution de tant de docks nouveaux, de routes, de canaux, l’infinie variété d’entreprises coûteuses exécutées pendant la durée des hostilités, montrent que les économies de la masse du peuple dépassèrent grandement les dépenses militaires du gouvernement et les dépenses improductives des individus**. Mac Culloch.
↑ Vraisemblablement le comte de Bruhl, ministre et grand chambellan du roi de Pologne. Il laissa à sa mort une garde-robe composée de trois cent soixante-cinq habits tous extrêmement riches.

*. Il n'y a plus que l'Angleterre au monde qui accorde encore une sérieuse attention à toutes ces variétés.
**. Le commentateur explique ces merveilleux phénomènes par la sécurité dont jouit la propriété en Angleterre, par la liberté de l'industrie, la diffusion universelle de l’intelligence, etc… Un économiste américain, Carey, a donné de ces faits une explication qui nous semble plus générale et plus vraie : c'est que les énormes dépenses de la guerre ont profité, en Angleterre, aux classes aisées et n’ont frappé que les classes laborieuses. Mac Culloch nous a appris plus haut que, pendant la guerre avec la France, l'État a payé jusqu'à 10% à ses préteurs ; avec de pareils profits, ils pouvaient largement subventionner l’industrie : la guerre a enrichi les riches et appauvri les pauvres.

%%%%%%%%%%%%%%%%%%%%%%%%%%%%%%%%%%%%%%%%%%%%%%%%%%%%%%%%%%%%%%%%%%%%%%%%%%%%%%%%
%                                  Chapitre 4                                  %
%%%%%%%%%%%%%%%%%%%%%%%%%%%%%%%%%%%%%%%%%%%%%%%%%%%%%%%%%%%%%%%%%%%%%%%%%%%%%%%%

\chapter{Des fonds prêtés à intérêt}
\markboth{Des fonds prêtés à intérêt}{}

Les fonds prêtés à intérêt sont toujours regardés par le prêteur comme un capital.
Il s’attend qu’à l’époque convenue ces fonds lui seront rendus et qu’en même temps l’emprunteur lui payera une certaine rente annuelle pour les avoir eus à sa disposition. L’emprunteur peut disposer de ses fonds ou comme d’un capital, ou comme de fonds destinés à servir immédiatement à sa consommation : s’il s’en sert comme d’un capital, il les emploie à faire subsister des ouvriers productifs qui en reproduisent la valeur avec un profit ; dans ce cas, il peut et rendre le capital et payer l’intérêt, sans aliéner ou sans entamer aucune autre source de revenu ; s’il s’en sert comme de fonds destinés immédiatement à sa consommation, il agit en prodigue et dissipé en subsistances données à la fainéantise ce qui était destiné à l’entretien de l’industrie ; dans ce cas, il ne peut ni rendre le capital ai payer l’intérêt, sans aliéner ou entamer quelque autre source de revenu, telle qu’une propriété ou une rente de terre.
Les fonds prêtés à intérêt sont sans contredit employés, suivant les circonstances, tant de l’une que de l’autre de ces deux manières, mais bien plus fréquemment de la première que de la seconde. Celui qui emprunte pour dépenser sera bientôt ruiné et celui qui lui prête aura lieu, en général, de se repentir de son imprudence ; ainsi, dans tous les cas où il n’est pas question de prêt à usure, il est contre l’intérêt des deux parties d’emprunter, comme de prêter, pour une pareille destination ; et quoique sans doute il y ait des gens à qui il arrive quelquefois de faire l’un et l’autre, toutefois, d’après l’attention que tout homme porte à ses intérêts, nous pouvons être bien sûrs que cela n’arrive pas aussi souvent que nous pourrions nous l’imaginer. Demandez à tout homme riche qui ne sera pas plus imprudent qu’un autre, à qui de ces espèces de gens il a prêté le plus de ses fonds, ou à ceux qu’il jugeait avoir intention d’en faire un emploi profitable, ou à ceux qui étaient dans le cas de les dépenser en pure perte ; à coup sûr il trouvera votre question fort étrange. Ainsi, même parmi les emprunteurs, qui ne forment pas la classe d’hommes où il faille chercher l’économie, le nombre des économes et des laborieux surpasse de beaucoup celui des prodigues et des fainéants.
Les seules gens à qui on prête communément des fonds[1], sans qu’on s’attende qu’il en feront un emploi très-profitable, ce sont les propriétaires ruraux qui empruntent par hypothèque ; encore n’empruntent-ils presque jamais purement en vue de dépenser ; on peut dire que ce qu’ils empruntent est ordinairement dépensé avant qu’ils l’emprun­tent. C’est, en général, pour avoir consommé trop de marchandises qui leur ont été avancées à crédit par des fournisseurs ou des artisans, qu’ils se voient enfin dans la nécessité d’emprunter à intérêt pour s’acquitter. Le capital emprunté remplace les capitaux de ces fournisseurs et de ces artisans, que jamais ces propriétaires n’auraient pu remplacer avec les rentes de leurs domaines ; il n’est pas proprement emprunté pour être dépensé, mais pour remplacer un capital déjà dépensé.
Presque tous les prêts à intérêt sont faits en argent, soit papier, soit espèces ; mais la chose dont vraiment l’emprunteur a besoin, celle que le prêteur lui fournit réelle­ment, ce n’est pas l’argent, c’est la valeur de l’argent ; ce sont les marchandises qu’on peut acheter avec. Si l’emprunteur entend se servir de l’argent comme fonds destiné immédiatement à sa consommation, il n’y a que ces marchandises qui soient de nature à être mises à cet usage ; s’il en a besoin comme d’un capital pour faire aller quelque genre d’industrie, il n’y a encore que ces marchandises qui puissent servir aux gens de travail, comme outils, matières et subsistances pour exécuter leur ouvrage. Par le prêt, le prêteur délègue, pour ainsi dire, à l’emprunteur son droit à une certaine portion du produit annuel de la terre et du travail du pays, pour en user comme il lui plait.
Ce qui détermine donc la quantité de fonds, ou, comme on dit communément, d’argent qui peut être prêtée à intérêt dans un pays, ce n’est pas la valeur de l’argent, papier ou espèces, qui sert d’instrument aux différents prêts qui se font dans le pays, mais c’est la valeur de cette portion du produit annuel qui, au sortir de la terre ou des mains des ouvriers productifs, est non-seulement destinée à remplacer un capital, mais encore un capital que le possesseur ne se soucie pas de prendre la peine d’employer lui-même. Comme ces capitaux sont ordinairement prêtés et remboursés en argent, ils constituent ce qu’on nomme intérêt de l’argent. Cet intérêt est différent, non-seulement de celui que donnent les fonds de terre, mais encore de celui que rendent les entre­prises de commerce et de manufactures, lorsque dans celles-ci les propriétaires des capitaux en font eux-mêmes l’emploi. Cependant, même dans l’intérêt de l’argent, l’argent n’est, pour ainsi dire, que le contrat de délégation qui transporte d’une main dans une autre ces capitaux que les possesseurs ne se soucient pas d’employer eux-mêmes. Ces capitaux peuvent être infiniment plus grands que la somme d’argent qui sert comme d’instrument pour en faire le transport ; les mêmes pièces de monnaie servant successivement pour plusieurs différents prêts, tout comme elles servent pour plusieurs différents achats. Par exemple, A prête à X, 1,000 livres, avec lesquelles X achète immédiatement de B pour la valeur de 1,000 livres de marchandises. B n’ayant pas besoin de cet argent pour lui-même, prête identiquement les mêmes pièces à Y, avec lesquelles Y achète aussitôt de C pour 1,000 livres d’autres marchandises, C de même, et pour la même raison, prête cet argent à Z, qui en achète aussi d’autres marchandises de D. Par ce moyen, les mêmes pièces, soit de métal, soit de papier, peuvent, dans le courant de quelques jours, servir d’instrument à trois différents prêts et à trois différents achats, chacun desquels est de valeur égale au montant total de ces pièces. Ce que les trois capitalistes A, B, C, transportent aux trois emprunteurs X, Y, Z, c’est le pouvoir de faire ces achats ; c’est dans ce pouvoir que consistent la valeur du prêt et son utilité. Le capital prêté par ces trois capitalistes est égal à la valeur des marchandises qu’on peut acheter avec, et il est trois fois plus grand que la valeur de l’argent avec lequel se font les achats. Cependant, ces prêts peuvent être tous parfai­te­ment bien assurés ; les marchandises achetées par les différents débiteurs étant employées de manière à rendre, au terme convenu, une valeur égale en argent ou en papier, avec un profil en plus. Si ces mêmes pièces de monnaie peuvent ainsi servir d’instrument à différents prêts pour trois fois et, par la même raison, pour trente fois leur valeur, elles peuvent pareillement servir autant de fois successivement d’instru­ment de remboursement.
De cette manière, on peut regarder un capital prêté à intérêt comme une délé­ga­tion, faite par le prêteur à l’emprunteur, d’une portion quelconque du produit annuel, sous la condition qu’en retour l’emprunteur lui déléguera annuellement, pendant tout le temps de la durée du prêt, une portion plus petite, appelée l’intérêt et, à l’échéance du prêt, une portion pareille à celle qui a été originairement déléguée ; ce qui s’appelle le remboursement. Quoique l’argent, soit papier, soit espèces, serve en général d’instrument de délégation, tant pour la petite portion que pour la grande, il n’en est pas moins tout à fait distinct de la chose qu’on délègue par son moyen.
À mesure que s’augmente dans un pays cette partie du produit annuel qui, au sortir de la terre ou des mains des ouvriers productifs, est destinée à remplacer un capital, ce qu’on appelle capitaux pécuniaires ou argent à prêter, y grossit en même temps. L’accroissement de ces fonds particuliers dont les possesseurs veulent tirer un bénéfice, sans prendre la peine de les employer eux-mêmes, est une suite naturelle de l’accroissement de la masse générale des capitaux, ou, pour parler autrement à mesu­re que les capitaux se multiplient, la quantité de fonds à prêter à intérêt devient successivement de plus en plus grande. À mesure que la quantité des fonds à prêter à intérêt vient à augmenter, l’intérêt ou le prix qu’il faut payer pour l’usage du capital va nécessairement en diminuant, non-seulement en vertu de ces causes générales qui font que le prix de marché de toutes choses diminue à mesure que la quantité de ces choses augmente, mais encore en vertu d’autres causes particulières à ce cas-ci. À mesure que les capitaux se multi­plient dans un pays, le profit qu’on peut faire en les employant diminue nécessaire­ment ; il devient successivement de plus en plus difficile de trouver dans ce pays une manière profitable d’employer un nouveau capital. En conséquence, il s’élève une concurrence entre les différents capitaux, le possesseur d’un capital faisant tous ses efforts pour s’emparer de l’emploi qui se trouve occupé par un autre. Mais le plus souvent, il ne peut espérer d’obtenir l’emploi de cet autre capital, à moins d’offrir à de meilleures conditions. Il se trouve obligé, non-seulement de vendre la chose sur laquelle il commerce un peu meilleur marché, mais encore, pour trouver occasion de la vendre, il est quelquefois aussi obligé de l’acheter plus cher. Le fonds destiné à l’entretien du travail productif grossissant de jour en jour, la demande qu’on fait de ce travail devient[2] aussi de jour en jour plus grande ; les ouvriers trouvent aisément de l’emploi, mais les possesseurs de capitaux ont de la difficulté à trouver des ouvriers à employer. La concurrence des capitalistes fait hausser les salaires du travail et fait baisser les profits. Or, lorsque le bénéfice qu’on peut retirer de l’usage d’un capital se trouve, pour ainsi dire, rogné à la fois par les deux bouts, il faut bien nécessairement que le prix qu’on peut payer pour l’usage de ce capital diminue en même temps que ce bénéfice. 
MM. Locke, Law et Montesquieu, ainsi que plusieurs autres écrivains, paraissent s’être imaginé que l’augmentation survenue dans la quantité de l’or et de l’argent, conséquence de la découverte des Indes occidentales espagnoles, était la vraie cause qui avait fait baisser le taux de l’intérêt dans la majeure partie de l’Europe. Ces mé­taux, disent-ils, ayant baissé de valeur en eux-mêmes, l’usage d’une portion quelcon­que de ces métaux eut aussi moins de valeur et, par conséquent, le prix qu’il fallait payer pour avoir droit à cet usage dut aussi baisser. Cette idée, qui semble tout à fait plausible au premier coup d’œil, a été si bien approfondie par M. Hume, qu’il est peut-être superflu d’en rien dire. Cependant, un raisonnement très-court et très-simple peut servir encore à faire voir plus clairement l’erreur qui semble avoir fait illusion à ces écrivains.
Il paraît qu’avant la découverte des Indes occidentales espagnoles, le taux ordinaire de l’intérêt dans la majeure partie de l’Europe était à 10 pour 100. Depuis cette époque il est tombé, dans différents pays, à 6, 5, 4 et 3 pour 100. Supposons que dans chaque pays en particulier la valeur de l’argent ait baissé exactement dans la même proportion que le taux de l’intérêt, et que dans le pays, par exemple, où l’intérêt a été réduit de 10 pour 100 à 5, la même quantité d’argent puisse maintenant acheter tout juste en marchandises la moitié de ce qu’elle en aurait acheté auparavant. je ne crois pas que nulle part on trouve cette supposition conforme à la vérité des choses, mais elle est la plus favorable à l’opinion que nous avons à examiner ; cependant, dans cette supposition même, il est absolument impossible que la baisse de la valeur de l’argent ait la moindre tendance à faire baisser le taux de l’intérêt. Si dans ces pays-là 100 livres aujourd’hui n’ont pas plus de valeur que 50 livres n’en avaient alors, nécessairement aussi, 10 livres n’y ont pas aujourd’hui plus de valeur que 5 n’en avaient alors. Quelles que soient les causes qui fassent baisser la valeur du capital, il faut de toute nécessité qu’elles fassent baisser en même temps celle de l’intérêt, et précisément dans la même proportion. La proportion entre la valeur du capital et celle de l’intérêt sera toujours restée la même, si l’on ne change rien au taux de l’intérêt. En changeant le taux, au contraire, la proportion entre ces deux valeurs se trouve néces­sai­re­ment changée. Si aujourd’hui 100 livres ne valent pas plus que 50 livres ne valaient alors, 5 livres aujourd’hui ne vaudront pas plus que ne valaient alors 2 livres 10 sous. Ainsi, en réduisant le taux de l’ intérêt de 10 pour 100 à 5, nous donnons, pour l’usage d’un capital qu’on suppose égal à la moitié de sa première valeur, un intérêt qui ne vaut plus que le quart du premier intérêt.
Toute augmentation survenue dans la quantité de l’argent, tant que la quantité des marchandises qu’il fait circuler reste la même, ne pourrait produire d’autre effet que de diminuer la valeur de ce métal[3]. La valeur nominale de toute espèce de choses serait plus grande, mais leur valeur réelle serait précisément la même qu’auparavant. Elles s’échangeraient contre un plus grand nombre de pièces d’argent qu’auparavant, mais la quantité de travail qu’elles pourraient commander, le nombre de gens qu’elles pour­raient faire subsister et tenir employés, serait toujours précisément le même. Le capi­tal du pays serait toujours le même, encore que, pour en transporter la même portion d’une main à l’autre, il fallût un plus grand nombre de pièces d’argent. Les instruments de la délégation, semblables aux actes d’un notaire diffus dans son style, seraient plus volumineux, mais la chose déléguée serait toujours exactement la même qu’aupara­vant, et ne pourrait toujours produire que le même effet. Le fonds destiné à l’entretien du travail productif étant le même, la demande qu’on ferait de ce travail serait tou­jours la même. Ainsi, son prix ou son salaire, quoique nominalement plus grand, serait le même quant à sa valeur réelle. On le payerait, à la vérité, avec une plus gran­de quantité de pièces d’argent, mais il n’achèterait toujours que la même quantité de choses. Les profits des capitaux seraient toujours les mêmes, réellement et même nominalement, car le salaire du travail se compte ordinairement par la quantité d’ar­gent qu’on paye à l’ouvrier ; ainsi, quand cette quantité augmente, le salaire semble en apparence avoir augmenté, quoiqu’il ne soit pas pour cela quelquefois plus fort qu’auparavant ; au lieu que les profits des capitaux ne se comptent pas par le nombre de pièces d’argent avec lequel on les paye, mais par la proportion qu’il y a entre ces pièces et le capital employé. Ainsi, on dira que, dans tel endroit, le salaire du travail est communément de 5 schellings par semaine, et les profits des capitaux de 10 pour 100. Or, la masse totale des capitaux du pays étant toujours la même qu’auparavant, la concurrence entre les différents capitaux des particuliers dans les mains desquels cette masse est répandue sera aussi la même. Les avantages et désavantages des différents emplois de capitaux seront ce qu’ils étaient auparavant. Par conséquent, le capital et l’intérêt resteront en général, l’un à l’égard de l’autre, dans la même proportion où ils étaient, et dès lors l’intérêt ordinaire de l’argent sera toujours le même ; ce qu’on peut communément donner pour avoir l’usage de l’argent se réglant nécessairement sur ce qu’on peut communément faire de profit en l’empruntant.
Toute augmentation qui surviendrait dans la quantité des marchandises qui circu­lent annuellement dans un pays, tant que la quantité d’argent qui les fait circuler reste la même, produirait, au contraire, plusieurs autres effets importants, outre celui de faire hausser la valeur de l’argent. Le capital du pays, quoiqu’il pût être le même nomi­na­le­ment, serait dans la réalité augmenté. On pourrait bien continuer à en expri­mer la valeur par la même quantité de pièces d’argent ; mais, dans le fait, il comman­de­rait une plus grande quantité de travail. La quantité de travail productif qu’il pourra faire subsister et tenir employé se trouverait augmentée et par consé­quent on demanderait une plus grande quantité de ce travail. Le salaire de ce travail hausserait naturellement en raison de la multiplication des demandes, et malgré cela il pourrait en apparence sembler avoir baissé. Il se pourrait qu’on le payât avec une moindre quantité d’argent, mais cette moindre quantité achèterait plus de marchandises que la plus grande quantité n’eût pu en acheter auparavant. Les profits des capitaux baisse­raient aussi bien en réalité qu’en apparence. La masse générale des capitaux du pays étant augmentée, la concurrence entre les différents capitaux qui la composent aug­men­terait naturellement avec elle. Les possesseurs de ces capitaux particuliers se­raient bien obligés de se contenter d’une plus petite portion dans le produit du travail que mettraient en activité leurs capitaux respectifs. Par ce moyen, l’intérêt de l’argent, qui suit toujours le cours du profit des capitaux, pourrait se trouver extrêmement réduit, encore que la valeur de l’argent, c’est-à-dire la quantité de choses qu’une som­me donnée d’argent pourrait acheter, fût très-augmentée.
Dans certains pays, la loi a prohibé l’intérêt de l’argent ; mais comme partout l’usage de l’argent est bon à quelque chose, partout on payera quelque chose pour se le procurer. L’expérience a fait voir que de telles lois, au lieu de prévenir le mai de l’usure, ne faisaient que l’accroître ; le débiteur étant alors obligé de payer, non-seule­ment pour l’usage de l’argent, mais encore pour le risque que court le créancier en acceptant une indemnité qui est le prix de l’usage de son argent. Le débiteur se trouve obligé, pour ainsi dire, d’assurer son créancier contre les peines de l’usure.
Dans les pays où l’intérêt est permis, la loi en général, pour empêcher les exac­tions de l’usure, fixe le taux le plus élevé qu’on puisse exiger, sans encourir de peine. Ce taux devrait être toujours un peu au-dessus du taux le plus bas de la place ou du prix qui se paye couramment pour l’usage de l’argent, par ceux qui peuvent donner les plus grandes sûretés. Si l’on fixait ce taux légal au-dessous du taux le plus bas de la place, les effets de cette fixation seraient à peu près les mêmes que ceux d’une prohibition absolue de l’intérêt. Le créancier ne voudrait pas prêter pour moins que ne vaut l’usage de son argent, et le débiteur serait obligé de l’indemniser du risque qu’il courrait en acceptant le prix de cet usage dans toute sa valeur. S’il est fixé précisément au taux le plus bas de la place, alors tous ceux qui ne sont pas en état d’offrir les meilleures de toutes les sûretés ne peuvent plus obtenir de crédit auprès des honnêtes gens qui respectent les lois de leur pays, et ils sont obligés d’avoir recours aux usuriers. Dans un pays tel que la Grande-Bretagne, où l’on prête au gouvernement à 3 pour 100, et aux particuliers, sur de bonnes sûretés, à 4 et 4 1/2, le taux légal actuel de l’intérêt à 5 pour 100 est peut-être le plus convenable qu’on puisse fixer.
Il est à observer que si le taux légal doit être un peu au-dessus du taux courant de la place, il ne faut pas qu’il soit non plus trop au-dessus. Si, par exemple, en Angleterre, le taux légal de l’intérêt était fixé à 8 ou 10 pour 100, la plus grande partie de l’argent qui se prêterait serait prêtée à des prodigues ou à des faiseurs de projets, la seule classe de gens qui voulût consentir à payer l’argent aussi cher. Les gens sages qui ne veulent donner pour l’usage de l’argent qu’une partie du profit qu’ils espèrent en retirer, n’iraient pas risquer de se mettre en concurrence avec ceux-là, Ainsi, une grande partie du capital du pays se trouverait, par ce moyen, enlevée aux mains les plus propres à en faire un usage profitable et avantageux, et jetée dans celles qui sont les plus disposées à la dissiper et à l’anéantir. Lorsque, au contraire, le taux légal n’est fixé que très-peu au-dessus du taux courant, les gens sages sont généralement préférés, pour les placements, aux prodigues et aux faiseurs de projets. Le capitaliste peut retirer des premiers un intérêt à peu de chose près aussi élevé que celui qu’il pourrait risquer de demander aux seconds, et son argent se trouve bien plus assuré dans les mains de l’une de ces classes de gens que dans celles de l’autre. Par là, une grande partie du capital du pays se verse dans des mains dont on n’a plus lieu d’es­pérer qu’elles l’emploieront d’une manière avantageuse.
Il n’y a pas de loi qui puisse réduire effectivement le taux ordinaire de l’intérêt au-dessous du taux courant le plus bas, à l’époque où elle est portée. Malgré l’édit de 1766, par lequel le roi de France tâcha de réduire le taux de l’intérêt de 5 pour 100 à 4, on continua toujours de prêter en France à 5 pour 100, et on trouva bien des moyens d’éluder la loi[4]. 
Il est à remarquer que partout le prix courant des terres dépend du taux courant de l’intérêt. Celui qui a un capital dont il désire retirer un revenu sans prendre la peine de l’employer lui-même, délibère s’il en achètera une terre, ou s’il le prêtera à intérêt. La sûreté la plus grande du placement, et puis quelques autres avantages qui accompa­gnent presque partout cette espèce de propriété, le disposeront naturellement à se contenter d’un revenu moindre, en terre, que celui qu’il pourrait se procurer en prêtant son argent à intérêt. Ces avantages suffisent pour compenser une certaine différence dans le revenu, et si la rente de la terre tombait au-dessous de l’intérêt de l’argent plus bas que cette différence, personne ne voudrait acheter des terres ; ce qui réduirait bien­tôt leur prix courant. Au contraire, si les avantages faisaient beaucoup plus que compenser la différence, tout le monde voudrait acheter des terres ; ce qui en relè­verait encore bientôt le prix courant. Quand l’intérêt était à 10 pour 100, les terres se vendaient communément pour le montant de dix à douze années de leur revenu, c’est-à-dire du denier dix au denier douze. À mesure que l’intérêt vint à baisser à 6,5 et 4 pour 100, le prix des terres s’éleva au denier vingt, vingt-cinq et trente. Le taux courant de l’intérêt est plus haut en France qu’en Angleterre, et le prix commun des terres y est plus bas. Elles se vendent communément en Angleterre au denier trente, et en France au denier vingt.
 
 
 
↑ La profession de prêteur d’argent, bien qu’elle n’ait été proscrite que depuis l’établissement du christianisme, et seulement chez les peuples chrétiens, n’a pourtant été populaire à aucune époque et dans aucun pays. Ceux qui sacrifient le présent à l’avenir sont naturellement les objets de l’envie de ceux qui ont sacrifié l’avenir au présent. Les enfants qui ont mangé leur gâteau, sont les ennemis naturels de ceux qui ont conservé le leur. Tant qu’on espère obtenir l’argent dont on a besoin, et quelque temps encore après qu’on l’a obtenu, on regarde celui qui prête comme un ami et un bienfaiteur ; mais bientôt l’argent est dépensé, et arrive l’heure maudite où il faut payer. Le bienfaiteur alors se trouve avoir changé de nature : ce n’est plus qu’un tyran et un oppresseur, car c’est une oppression que de réclamer on argent, tandis qu’il est tout naturel de ne pas rendre celui qu’on doit. Chez les gens irréfléchis, c’est-à-dire dans la grande masse du genre humain, les affections égoïstes conspirent avec les affections sociales, pour attirer toute la faveur sur le dissipateur, et pour refuser toute justice à l’homme économe qui a fourni à ses besoins. Le premier, quel que soit le point de sa carrière auquel il soit parvenu, est toujours assuré de voir l’intérêt public, sous une forme ou sous une autre, s’attacher à sa personne ; tandis que le second, à aucune époque de sa vie, ne doit s’attendre à une pareille faveur. Ceux qui vivent avec un homme sont intéressés à ce que sa dépense soit au moins aussi élevée que sa fortune le comporte, attendu qu’il n’y a point d’espèce de dépense dans laquelle un individu puisse se jeter, dont les avantages ne soient partagés à un degré ou à un autre par tous ceux qui l’entourent. De là cette loi éternelle qui interdit à tout homme, sous peine d’infamie, la faculté de réduire sa dépense au-dessous de sa fortune, en lui laissant toujours celle, d’ailleurs, de la porter au-dessus, tout autant qu’il peut juger à propos de le faire. Or, il peut bien arriver que les moyens que l’on attribue à un individu, par suite de cette loi, soient de beaucoup au-dessus de ceux qu’il possède réellement, mais il n’arrive jamais qu’ils soient au-dessous. Il existe généralement une relation si intime entre l’idée de dépense et celle de mérite, qu’une disposition à dépenser trouve faveur, même aux yeux des gens qui savent que l’individu qui s’y abandonne excède ses propres ressources, et que le premier venu, par suite de cette association d’idées, et sans autre recommandation qu’un penchant à la dissipation , peut facilement acquérir un fonds permanent de considération, au préjudice des individus eux-mêmes aux dépens desquels il a satisfait ses appétits et son orgueil. Le lustre que l’étalage d’une richesse empruntée a jeté sur son caractère, soumet les hommes à son insolence pendant tout le cours de sa prospérité, et lorsque enfin la main de l’adversité vient s’appesantir sur sa tête, le souvenir de la hauteur d’où il est tombé couvre ses injustices du voile de la compassion. La conduite de l’homme économe est toute différente. Son opulence permanente lui attire une partie au moins de l’envie qui s’attache à la splendeur passagère du prodigue ; mais l’usage qu’il eu fait ne lui permet pas de prétendre à la faveur qui attend ce dernier ; c’est que personne ne peut participer à la satisfaction que lui procure sa fortune, satisfaction qui se compose seulement du plaisir de la possession actuelle et de l’espérance de jouir de ses épargnes à quelque époque éloignée
↑ On peut regarder le prix de l’intérêt comme une espèce de niveau au-dessous duquel tout travail, toute culture, toute industrie, tout commerce cessent. C’est comme une mer répandue sur une vaste contrée : les sommets des montagnes s’élèvent au-dessus des eaux et forment des îles fertiles et cultivées. Si cette mer vient à s’écouler, à mesure qu’elle descend, les terrains en pente, puis les plaines et les vallons paraissent et se couvrent de productions de toute espèce. Il suffit que l’eau monte ou s’abaisse d’un pied pour inonder ou pour rendre à la culture des plages immenses. C’est l’abondance des capitaux qui anime toutes les entreprises, et le bas intérêt de l’argent est tout à la fois l’effet et l’indice de l’abondance des capitaux. Turgot.
↑ Le législateur est rarement intervenu dans la fixation du prix des marchandises autres que l’argent, et le peu qu’il ait jamais fait à cet égard se recommande beaucoup plus par la droiture de l’intention que par la rectitude du jugement ou le succès de l’entreprise. Placer de l’argent à intérêt, c’est échanger de l’argent actuel contre de l’argent futur. Il s’agirait de montrer maintenant comment un système universellement considéré comme absurde, en tant qu’appliqué aux échanges en général, pourrait être jugé nécessaire dans le cas de cette espèce particulière d’échange. Il n’existe point de dénomination spéciale de marque d’infamie pour celui qui tire le plus de parti possible de l’usage qu’il concède de toute autre chose que de l’argent, d’une maison, par exemple ; personne n’éprouve de honte de se conduire ainsi, et il n’est pas ordinaire de voir affichée la prétention contraire : comment se fait-il donc qu’un homme qui cherche à faire valoir une somme d’argent de la manière la plus avantageuse, à en tirer 6, 7 ou même 10 pour 100, mérite plutôt, dans ce cas, le nom flétrissant d’usurier, que dans celui où, achetant une maison avec la même somme, il tirerait de ce marché un bénéfice équivalent ? J’avoue que, pour mon compte, c’est ce que je ne saurais comprendre.
Ce que je ne conçois pas davantage, c’est pourquoi le législateur a plutôt limité le taux de l’intérêt quant au maximum qu’au minimum, pourquoi il s’est montré plutôt hostile envers la classe des propriétaires d’argent qu’envers toute autre ; pourquoi il s’est plutôt proposé de les empêcher de faire au delà d’un certain bénéfice que de les empêcher d’en faire un moindre ; pourquoi, en un mot, il n’a pas aussi bien porté des peines contre celui qui offrirait un intérêt moindre que 5 pour 100, que contre celui qui accepterait un intérêt plus élevé. J’abandonne à d’autres le soin de résoudre ces difficultés, car, pour moi, c’est beaucoup plus que je ne saurais faire. Bentham.
↑ Quant à la proposition générale contenue dans ce passage de Smith, si elle est vraie, tant mieux ; mais j’avoue que je ne vois pas pourquoi il en serait ainsi. Il semble que ce soit dans le but de prouver la vérité de cette proposition que le mauvais succès de la tentative dont il est question ici, se trouve mentionné, d’autant plus qu’on n’en donne pas d’autre preuve. Mais en prenant ce fait pour avéré, je ne vois pas comment il serait suffisant pour légitimer une pareille conclusion. La loi qui nous est citée fut éludée, dit-on : mais comment le fut-elle ? comment se prêta-t-elle à l’être ? C’est ce qu’on ne nous dit pas. Cette circonstance put tenir à un vice particulier dans sa rédaction, ou, ce qui revient au même, dans la nature des mesures prises pour la mettre à exécution. Or, dans l’un ni dans l’autre cas, les infractions dont elle fut l’objet ne peuvent servir de base ou de justification à la proposition générale dont il est question. Pour que la vérité de cette proposition fût démontrée par un fait de cette nature, il faudrait prouver que tous les moyens qui étaient convenables pour donner de l’efficacité à la loi dont il s’agit ont été employés, et que, malgré toutes ces précautions, la loi a été encore éludée. Fondée ou non, la proposition qui est avancée ici ne porte pas cependant par elle-même un caractère de vérité assez évident pour être admise sans preuves ; et cependant, sauf le fait ci-dessus cité,qui, comme nous voyons, ne prouve rien, on n’en apporte aucune. Je dirai plus, je ne crois pas que cette proposition soit susceptible d’être prouvée. Pour ma part, en effet, je ne vois pas ce qui pourrait empêcher la loi de réduire le taux de l’intérêt au-dessous du taux ordinaire le plus bas en usage dans les transactions, si ce n’est un tel état de choses, une telle combinaison de circonstances qui devraient apporter des obstacles tout aussi puissants, ou à peu près, à l’efficacité d’une loi dirigée contre un taux d’intérêt plus élevé. Je ne vois de moyen capable d’enlever complètement à la loi son efficacité, que dans la résolution que prendraient tous les sujets d’un État de ne point dénoncer les infractions dont elle serait l’objet ; mais par une résolution de cette nature, le taux d’intérêt le plus élevé peut se trouver tout aussi efficacement protégé que le taux le plus bas. Supposez que leur résolution soit universelle, dans toute la rigueur du mot : la loi devient alors complètement inefficace ; tous les taux d’intérêt demeurent également libres, et, sous ce rapport, les transactions particulières sont exactement ce qu’elles seraient s’il n’existait point de loi sur cette matière. La proposition du docteur Smith, en tant qu’elle limite l’inefficacité de la loi aux taux d’intérêt inférieurs aux plus bas de ceux qui sont en usage dans les transactions particulières, manque d’exactitude. Pour moi, je ne saurais concevoir qu’une pareille résolution ait pu jamais être prise et soutenue, ou puisse l’être jamais, sans une rébellion ouverte contre le gouvernement : or, je ne vois pas que rien de semblable soit arrivé. Quant aux coalitions particulières, elles sont tout aussi capables de protéger contre la loi l’intérêt le plus élevé que l’intérêt le plus bas.
Il faut reconnaître pourtant que le taux d’intérêt le plus bas, dans le cas d’une prohibition légale, doit, selon toute apparence, rencontrer plus fréquemment que tout autre la protection du public. Il y a deux raisons pour cela : d’abord parce que, étant du nombre des taux ordinaires, sa nécessité doit naturellement se faire sentir plus souvent que celle des taux extraordinaires ; et ensuite parce que la défaveur attachée à l’idée d’usure, circonstance capable, à un degré ou à un autre, d’exclure de la protection du public les taux d’intérêts de cette dernière espèce, ne peut pas être supposée s’étendre encore à l’usage du taux dont nous parlons. Un prêteur a certainement moins de raison de s’abstenir de prendre un taux d’intérêt qu’il peut accepter sans infamie, que d’en prendre un qui lui imprimerait cette tache. Or, il n’est pas probable que le public se montre tellement empressé de mettre son imagination et ses sentiments en harmonie avec la volonté de la loi, que, dès qu’elle a parlé, il frappe de réprobation un acte que l’instant d’avant il jugeait innocent.
Que si l’on me demandait comment je suppose que les choses se sont passées dans le cas rapporté par le Dr Smith, jugeant de l’événement d’après les probabilités générales, je dirais que la loi n’était pas rédigée de manière à être complètement à l’abri des violations; que cependant dans beaucoup d’occasions qu’il a été impossible de constater, les citoyens ont dû s’y conformer, soit en s’abstenant absolument de prêter, soit en prêtant au taux réduit par la loi ; que, dans d’autres cas, la loi aura été violée, les prêteurs se fiant à cet égard, en partie aux expédients employés par eux pour l’éluder, et en partie à la bonne foi et à l’honneur des emprunteurs ; je dirais que, par les deux raisons qui ont été exposées plus haut, l’ancien intérêt légal, dans ces derniers cas, aura été, selon toute apparence, plus souvent stipulé que tout autre, et que, par suite de l’usage plus fréquent qui en aura été fait et de son opposition plus directe à la nouvelle loi, il aura dû aussi être plus remarqué, et que voilà sans doute, en point de fait, le fondement de cette proposition générale du Dr Smith, qu’aucune loi ne peut réduire le taux commun de l’intérêt au-dessous du taux le plus bas en usage dans les transactions au moment de sa publication. Bentham.

%%%%%%%%%%%%%%%%%%%%%%%%%%%%%%%%%%%%%%%%%%%%%%%%%%%%%%%%%%%%%%%%%%%%%%%%%%%%%%%%
%                                  Chapitre 5                                  %
%%%%%%%%%%%%%%%%%%%%%%%%%%%%%%%%%%%%%%%%%%%%%%%%%%%%%%%%%%%%%%%%%%%%%%%%%%%%%%%%

\chapter{Des différents emplois des capitaux}
\markboth{Des différents emplois des capitaux}{}

Quoique tous les capitaux soient destinés à l’entretien du travail productif seule­ment, cependant la quantité de ce travail que des capitaux égaux sont capables de mettre en activité, varie extrêmement d’après la nature différente de l’emploi qu’on leur donne, et il y a la même variation dans la valeur que cet emploi ajoute au produit annuel des terres et du travail du pays.
Il y a quatre manières différentes d’employer un capital.
On peut l’employer : 1° à fournir à la société le produit brut qu’il lui faut pour son usage et sa consommation annuelle ; ou bien, 2° à manufacturer et à préparer ce produit brut, pour qu’il puisse immédiatement servir à l’usage et à la consommation de la société ; ou, 3° à transporter, soit le produit brut, soit le produit manufacturé, des endroits où ils abondent à ceux où ils manquent ; ou, 4° enfin, à diviser des portions de l’un et de l’autre de ces produits en parcelles assez petites pour pouvoir s’accom­moder aux besoins journaliers des consommateurs.
C’est de la première manière que sont employés les capitaux de tous ceux qui entreprennent la culture, l’amélioration ou l’exploitation des terres, mines et pêche­ries ; c’est de la seconde que le sont ceux de tous les maîtres manufacturiers et fabri­cants ; c’est de la troisième que le sont ceux de tous les marchands en gros ; et c’est de la quatrième que le sont ceux de tous les marchands en détail. Il est difficile d’imaginer, pour un capital, un genre d’emploi qui ne puisse être classé sous l’une ou l’autre de ces quatre divisions.
Chacun de ces quatre moyens d’employer un capital est essentiellement néces­saire, tant à l’existence ou à l’extension des trois autres genres d’emploi, qu’à la commodité générale de la société.
À moins qu’il n’y ait un capital employé à fournir le produit brut dans un certain degré d’abondance, les manufactures et le commerce d’aucun genre ne pourraient exister.
À moins qu’il n’y ait un capital employé à manufacturer cette partie du produit brut qui exige un certain degré de préparation avant d’être propre à l’usage et à la consommation, cette partie du produit brut ne serait jamais produite, parce qu’il n’y en aurait point de demande ; ou, si elle était produite spontanément, elle n’aurait aucune valeur échangeable et n’ajouterait rien à la richesse de la société.
À moins qu’il n’y ait un capital employé à transporter le produit brut ou manu­facturé des endroits où il est abondant, à ceux où il manque, on ne produirait plus ni de l’un ni de l’autre au-delà de ce qui serait nécessaire pour la consommation locale seulement. Le capital du marchand, en échangeant le superflu d’un pays contre le superflu d’un autre, encourage l’industrie des deux pays et multiplie leurs jouissances.
À moins qu’il n’y ait un capital employé à morceler et à diviser des portions du produit brut ou manufacturé, en parcelles assez petites pour s’accommoder aux deman­des actuelles des consommateurs, chaque personne serait obligée d’acheter les marchandises qu’il lui faut, en plus grande quantité que ne l’exigent ses besoins du moment. Par exemple, s’il n’y avait pas de commerce, tel que celui de boucher, cha­cun serait obligé d’acheter un bœuf entier ou un mouton à la fois. Ce serait, en général, un très-grand inconvénient pour les riches, et un beaucoup plus grand encore pour les pauvres. Si un pauvre artisan était obligé d’acheter à la fois des vivres pour un mois ou pour six, il y aurait une grande partie des fonds qu’il emploie, comme capital, en instruments de son métier ou pour garnir sa boutique, et qui lui rapportent un revenu, qu’il serait force de placer dans la partie de ses fonds réservée pour servir immédiatement à sa consommation, et qui ne lui rapporte aucun revenu. Il n’y a rien de plus commode, pour un homme de cette classe, que de pouvoir acheter sa subsis­tance d’un jour à l’autre ou même d’heure en heure, à mesure qu’il en a besoin. Il se trouve par là en état d’employer presque tous ses fonds comme capital ; il peut, par ce moyen, fournir à ses pratiques pour une plus grande valeur d’ouvrage, et le profit qu’il y fait compense et bien au-delà le surcroît de prix dont les marchandises qu’il achète se trouvent chargées par le profit du détaillant.
Les préventions de certains écrivains politiques contre les petits détaillants et ouvriers en boutique sont tout à fait mal fondées. Tant s’en faut qu’il soit nécessaire d’en restreindre le nombre ou de les gêner par des impositions, qu’au contraire ils ne sauraient jamais se multiplier de manière à nuire au public, bien qu’ils le puissent assez pour se nuire les uns aux autres. La quantité de marchandises, d’épicerie, par exemple, qui peut se vendre dans une ville, est limitée par la demande de cette ville et de ses environs. Ainsi, le capital qu’on peut employer au commerce d’épicerie ne sau­rait excéder ce qu’il faut pour acheter cette quantité. Si ce capital se trouve partagé entre deux différents épiciers, la concurrence fera que chacun d’eux vendra à meilleur marché que si le capital eût été dans les mains d’un seul ; et s’il est divisé entre vingt, la concurrence en sera précisément d’autant plus active, et il y aura aussi d’autant moins de chance qu’ils puissent se concerter entre eux pour hausser le prix de leurs marchandises. La concurrence pourra bien peut-être en ruiner quelqu’un, mais c’est l’affaire des parties intéressées d’y prendre garde, et on peut, en toute sûreté, s’en rapporter là-dessus à leur prudence. Le consommateur ni le producteur ne pourront jamais y perdre ; au contraire, les détaillants seront dans le cas de vendre meilleur mar­ché, et d’acheter en même temps plus cher que si tout le commerce du même genre était accaparé par une ou deux personnes qui pourraient en faire monopole. Il pourra peut-être bien arriver une fois que quelqu’un d’eux trompe quelque chaland trop facile, et lui fasse acheter des choses dont celui-ci n’a pas besoin. Mais c’est là un trop petit inconvénient pour mériter l’attention du gouvernement, et ce ne serait pas un moyen sûr de l’empêcher, que de restreindre le nombre de ces petits marchands ; car pour prendre un exemple dans la classe la plus suspecte, ce n’est pas la multitude des cabarets qui engendre une disposition générale à l’ivrognerie parmi les gens du peuple, mais c’est cette disposition même, produite par d’autres causes, qui fait qu’une multitude de cabarets peut trouver de l’emploi.
Les personnes dont les capitaux sont employés de l’une de ces quatre manières sont elles-mêmes des ouvriers productifs. Leur travail, quand il est convenablement dirigé, se fixe et se réalise dans l’objet ou la chose vénale sur laquelle il est appliqué et, en général, il ajoute au prix de cette chose la valeur au moins de leur subsistance et consommation personnelle. Les profits du fermier, du manufacturier, du marchand, du détaillant, sont tous tirés du prix des marchandises que produisent les deux premiers, et dont trafiquent les deux autres. Cependant des capitaux égaux, selon qu’ils seront employés de l’une ou de l’autre de ces quatre manières différentes, met­tront en activité des quantités très-différentes de travail productif, et augmenteront aussi, dans des proportions très-différentes, la valeur du produit annuel des terres et du travail de la société à laquelle ils appartiennent.
Le capital du détaillant remplace, avec un profit en sus, le capital du marchand dont il achète des marchandises, et met par là ce marchand à portée de continuer son commerce. Ce capital n’emploie pas d’autre ouvrier productif que la personne du détaillant lui-même. C’est dans le profit de celui-ci que consiste toute la valeur que le capital ainsi employé ajoute au produit annuel de la terre et du travail de la société.
Le capital du marchand en gros ou en magasin remplace avec leurs profits les capitaux des fermiers et des manufacturiers dont il achète le produit brut et le produit manufacturé sur lesquels il commerce, et par là il les met les uns et les autres en état de continuer leurs travaux respectifs. C’est principalement par ce service qu’il contribue indirectement à soutenir le travail productif de la société, et à augmenter la valeur du produit annuel de ce travail. Son capital emploie aussi les voituriers et matelots qui transportent ses marchandises d’un lieu dans un autre, et augmente le prix de ces marchandises de la valeur des salaires de ces ouvriers, aussi bien que de celle de ses propres profits. C’est là tout le travail productif que ce capital met immé­diatement en activité, et toute la valeur qu’il ajoute immédiatement au profit annuel. Sous ces deux points de vue, ses opérations sont beaucoup au-dessus de celles du capital du détaillant.
Une partie du capital du maître manufacturier est employée comme capital fixe dans les instruments de son industrie, et remplace, avec un profit en plus, le capital de quelque autre ouvrier dont il les achète. Une partie de son capital circulant est em­ployée à acheter des matières, et remplace, avec leurs profits en sus, les capitaux des fermiers et des entrepreneurs des mines, qui lui vendent ces matières. Mais une grande partie de ce même capital se distribue toujours annuellement, ou dans une période beaucoup plus courte, entre les différents ouvriers qu’emploie le maître. Ce capital ajoute à la valeur des matières celle des salaires des ouvriers et celle des profits du maître sur la totalité du fonds de salaires, de matières et d’instruments de fabrique employés dans l’entreprise. Ainsi, il met en activité une bien plus grande quantité de produit productif, et ajoute une bien plus grande valeur au produit annuel des terres et du travail de la société, que ne ferait un pareil capital entre les mains de quelque marchand en gros que ce fût.
Mais aucun capital, à somme égale, ne met en captivité plus de travail productif que celui du fermier. Ce sont non-seulement ses valets de ferme, mais ses bestiaux de labour et de charroi qui sont autant d’ouvriers productifs. D’ailleurs, dans la culture de la terre, la nature travaille conjointement avec l’homme ; et quoique son travail ne coûte aucune dépense, ce qu’il produit n’en a pas moins sa valeur, aussi bien que ce que produisent les ouvriers les plus chers. Les opérations les plus importantes de l’agriculture semblent moins avoir pour objet d’accroître la fertilité de la nature (quoiqu’elles y parviennent aussi), que de diriger cette fertilité vers la production des plantes les plus utiles à l’homme. Un champ couvert de ronces et de bruyères produit souvent une aussi grande quantité de végétaux que la vigne ou la pièce de blé la mieux cultivée. Le cultivateur qui plante et qui sème excite souvent moins l’active fécondité de la nature, qu’il ne la détermine vers un objet, et après qu’il a terminé tous ses travaux, c’est à elle que la plus grande partie de l’ouvrage reste à faire. Ainsi les hommes et les bestiaux employés aux travaux de la culture, non-seulement comme les ouvriers des manufactures, donnent lieu à la reproduction d’une valeur égale à leur consommation ou au capital qui les emploie, en y joignant de plus les profits des capitalistes, mais ils produisent encore une bien plus grande valeur. Outre le capital du fermier et tous ses profits, ils donnent lieu à la reproduction régulière d’une rente pour le propriétaire. On peut considérer cette rente comme le produit de cette puis­sance de la nature, dont le propriétaire prête l’usage au fermier. Ce produit est plus ou moins grand, selon qu’on suppose à cette puissance plus ou moins d’étendue, ou, en d’autres termes, selon qu’on suppose à la terre plus ou moins de fertilité naturelle ou artificielle. C’est l’œuvre de la nature qui reste après qu’on a fait la déduction ou la balance de tout ce qu’on peut regarder comme l’œuvre de l’homme. Ce reste fait rarement moins du quart, et souvent plus du tiers du produit total. jamais une pareille quantité de travail productif, employé en manufactures, ne peut occasionner une aussi riche reproduction. Dans celles-ci, la nature ne fait rien ; la main de l’homme fait tout, et la reproduction doit toujours être nécessairement en raison de la puissance de l’agent. Ainsi, non-seulement le capital employé à la culture de la terre met en activité une plus grande quantité de travail productif que tout autre capital pareil employé en manufactures, mais encore, à proportion de la quantité de travail productif qu’il emploie, il ajoute une beaucoup plus grande valeur au produit annuel des terres et du travail du pays, à la richesse et au revenu réel de ses habitants. De toutes les manières dont un capital peut être employé, c’est sans comparaison la plus avantageuse à la société[1].
Les capitaux qu’on emploie dans une société à la culture des terres ou au com­merce de détail, restent toujours nécessairement dans le sein de cette société. Leur emploi se fait presque toujours sur un point fixe, la ferme et la boutique du détaillant. En général aussi, quoi qu’il y ait quelques exceptions, ils appartiennent à des membres résidents de la société.
Le capital du marchand en gros, au contraire, semble n’avoir nulle part de rési­dence fixe ou nécessaire ; mais il se promène volontiers de place en place, suivant qu’il peut trouver à acheter meilleur marché ou à vendre plus cher.
Le capital du manufacturier doit sans contredit résider au lieu de l’établissement de la manufacture ; mais le local de cet établissement n’a pas sa place nécessairement déterminée. Il peut être souvent à une grande distance, tant de l’endroit où croissent les matières, que de celui où se consomme l’ouvrage fait. Lyon est fort éloigné, et du lieu qui lui fournit la matière première de ses manufactures, et du lieu où elles se consomment. En Sicile, les gens de bon ton sont habillés d’étoffes de soie fabriquées à l’étranger, et dont la matière première a été produite chez eux. Une partie de la laine d’Espagne est travaillée dans les manufactures d’Angleterre, et une partie du drap qu’elle y produit retourne ensuite en Espagne.
Que le marchand dont le capital exporte le superflu[2] d’un pays, soit naturel de ce pays, soit étranger, c’est une chose fort peu importante. S’il est étranger, le nombre des ouvriers productifs se trouve d’un individu seulement être moindre que s’il eût été naturel du pays, et la valeur du produit annuel moindre de la valeur seulement du profit d’un individu. Les voituriers ou matelots qu’il emploie peuvent toujours être, ou de son propre pays ou du pays dont il s’agit, ou de quelque autre pays indifféremment, de la même manière que s’il eût été lui-même un naturel du pays. Le capital d’un étranger donne une valeur au superflu du produit de votre pays, tout comme le capital d’un de vos compatriotes, en échangeant ce superflu contre une denrée dont il y a demande chez vous. Il remplace tout aussi sûrement le capital de la personne qui produit ce superflu, et il la met tout aussi sûrement en état de continuer ses travaux ; ce qui est le genre principal de service par lequel le capital d’un marchand en gros con­tribue à soutenir le travail productif de la société dont il est membre, et à aug­menter la valeur du produit annuel de cette société.
Il importe beaucoup plus que le capital du manufacturier réside dans le pays. Il met alors nécessairement en activité une plus grande quantité de travail productif, et ajoute une plus grande valeur au produit annuel des terres et du travail de la société. Il peut cependant être fort utile au pays encore qu’il n’y réside pas. Les capitaux des manufacturiers anglais qui mettent en œuvre le chanvre et le lin qui s’importent annuellement des côtes de la mer Baltique, sont sûrement très-utiles aux pays qui produisent ces denrées. Elles sont une partie du produit superflu de ces pays, et si ce superflu n’était pas annuellement échangé contre quelque chose qui y est en demande, il n’aurait plus aucune valeur, et cesserait bientôt d’être produit. Les marchands qui l’exportent remplacent les capitaux des gens qui le produisent, et par là les encou­ragent à continuer cette production, et les manufacturiers anglais remplacent les capitaux de ces marchands. 
Il peut se faire souvent qu’un pays soit, comme le serait un particulier, dans le cas de manquer d’un capital suffisant pour cultiver et améliorer toutes ses terres, manu­facturer et préparer tout leur produit brut, tel que l’exigent l’usage et la consommation, et enfin transporter le superflu des deux produits brut et manufacturé, à des marchés éloignés où on puisse L’échanger contre quelque chose qui soit en demande dans le pays. Il y a beaucoup d’endroits dans la Grande-Bretagne, où les habitants n’ont pas de capitaux suffisants pour cultiver et améliorer leurs terres. La laine des provinces du midi de l’Écosse vient, en grande partie, faire un long voyage par terre dans de fort mauvaises routes, pour être manufacturée dans le comté d’York, faute de capital pour être manufacturée sur les lieux. Il y a en Angleterre plusieurs petites villes de fabri­ques, dont les habitants manquent de capitaux suffisants pour transporter le produit de leur propre industrie à ces marchés éloignés où il trouve des demandes et des consommateurs. Si on y voit quelques marchands, ce ne sont probablement que les agents de marchands plus riches qui résident dans quelques-unes des grandes villes commerçantes.
Quand le capital d’un pays ne peut suffire à remplir en entier ces trois fonctions, plus sera grande la portion qui en sera employée à l’agriculture, et plus sera grande à proportion la quantité de travail productif qu’il mettra en activité dans le pays, plus sera grande pareillement la valeur que son emploi ajoute au produit annuel des terres et du travail de la société. Après l’agriculture, ce sera le capital employé en manu­fac­tures, qui mettra en activité la plus grande quantité de travail productif, et qui ajoutera la plus grande valeur au produit annuel. Le capital employé au commerce d’exporta­tion est celui des trois qui produit le moins d’effet.
Il est vrai que le pays qui n’a pas un capital suffisant pour remplir en entier ces trois fonctions n’est pas encore parvenu au degré d’opulence auquel il semble être naturellement destiné. Cependant, essayer, par des efforts prématurés et avec un capi­tal insuffisant, de les remplir toutes les trois, certainement, pour une société comme pour un individu, ce ne serait pas là la voie la plus courte d’en acquérir un qui fût suffisant. Le capital de tous les individus d’une nation a ses limites comme celui d’un seul de ces individus, et ses opérations ont aussi leurs bornes. Le capital de tous les individus d’une nation se grossit, de la même manière que celui d’un seul individu, de ce qu’ils accumulent sans cesse, et de ce qu’ils y ajoutent par les épargnes faites sur leurs revenus. Il sera donc probablement dans le cas de grossir plus vite que jamais, s’il est employé de manière à fournir le plus gros revenu à tous les habitants du pays, puisque par là il les mettra à même de faire les plus grandes épargnes. Or, le revenu de tous les habitants du pays est nécessairement en raison de la valeur du produit annuel des terres et du travail.
La principale cause des progrès rapides de nos colonies d’Amérique vers la richesse et l’agrandissement, c’est que jusques à présent presque tous leurs capitaux ont été employés à l’agriculture. Elles n’ont point de manufactures, si ce n’est ces fabriques grossières et domestiques qui accompagnent nécessairement les progrès de l’agriculture, et qui sont l’ouvrage des femmes et des enfants dans chaque ménage. La plus grande partie, tant de leur exportation que de leur commerce de cabotage, se fait avec des capitaux de marchands qui résident dans la Grande-Bretagne. Le fonds même et les magasins de marchandises qui se vendent en détail dans quelques pro­vinces, particulièrement dans la Virginie et le Maryland, appartiennent la plupart à des marchands qui résident dans la mère patrie, et c’est un de ces exemples rares d’un commerce de détail fait dans un pays avec des capitaux étrangers. Si, par un projet concerté ou toute autre mesure forcée, les Américains venaient à arrêter l’importation des manufactures d’Europe et, en donnant par là un monopole à ceux de leurs com­patriotes qui fabriqueraient les mêmes espèces d’ouvrages, détourner pour ce genre d’emploi une grande partie de leur capital actuel, ils retarderaient, par cette conduite, les progrès ultérieurs de la valeur de leur produit annuel, bien loin de les accélérer, et ils entraveraient la marche de leur pays vers l’opulence et la grandeur, bien loin de la favoriser.
Ce serait encore bien pis s’ils voulaient se donner de la même manière le mono­pole de tout leur commerce d’exportation.
À la vérité, le cours des prospérités humaines ne paraît guère avoir jamais été d’une durée assez constante pour avoir mis aucun grand peuple dans le cas d’acquérir un capital qui ait pu suffire à remplir ces trois fonctions dans leur entier, à moins peut-être que nous ne voulions ajouter foi aux récits merveilleux qu’on nous fait de la richesse et de la culture de la Chine, de l’ancienne Égypte, et de l’Indostan dans son ancien état ; encore ces trois pays, les plus riches qui aient jamais existé, d’après tous les rapports, sont principalement renommés pour leur supériorité en agriculture et en manufactures. Il ne paraît pas qu’ils aient jamais brillé par le commerce avec l’étran­ger. La superstition des anciens Égyptiens leur inspirait une grande horreur pour la mer ; une superstition à peu près de la même espèce règne chez les Indiens, et les Chinois n’ont jamais porté bien loin leur commerce étranger. La plus grande partie du superflu de ces trois pays paraît avoir été toujours exportée par des étrangers, qui donnaient en échange quelque autre chose pour laquelle il y avait demande dans le pays, souvent de l’or et de l’argent.
C’est ainsi que le même capital dans un pays mettra en activité plus ou moins de travail productif, et ajoutera plus ou moins de valeur au produit annuel des terres et du travail, selon les différentes proportions dans lesquelles on l’emploiera dans l’agri­culture, dans les manufactures ou dans le commerce en gros. Les différentes espèces de commerce en gros, dans lesquelles il y en aura quelque partie d’employée, amène­ront aussi de très-grandes différences dans les effets.
On peut réduire à trois différentes espèces tout commerce en gros, tout achat fait pour revendre en gros : le commerce intérieur, le commerce étranger de consomma­tion et le commerce de transport[3]. Le commerce intérieur se fait en achetant dans un endroit du pays, pour les revendre dans un autre endroit du même pays, les produits de l’industrie nationale. Il comprend à la fois le commerce de cabotage et celui qui se fait par l’intérieur des terres. Le commerce étranger de consommation se fait en ache­tant des marchandises étrangères pour la consommation intérieure. Le commerce de transport se fait en commerçant entre deux pays étrangers, ou en transportant à l’un le superflu de l’autre.
Le capital qui est employé à acheter dans un endroit du même pays, pour le revendre dans l’autre, le produit de l’industrie de ce pays, replace en général, à chaque opération qu’il fait, deux capitaux distincts qui avaient été tous deux employés, soit en agriculture, soit en manufacture, et par là il les met en état de continuer leur fonction. Lorsque ce capital emporte une certaine valeur de marchandises hors de la résidence du marchand, il y rapporte ordinairement en retour une valeur au moins égale en autres marchandises. Quand elles sont les unes et les autres le produit de l’industrie nationale, il remplace alors nécessairement dans chacune de ces opérations deux capitaux distincts, employés l’un et l’autre à faire aller le travail productif, et par là il les met en état de continuer le même service. Le capital qui envoie à Londres des ouvra­ges de fabrique écossaise et rapporte à Édimbourg du blé anglais et des ouvra­ges de fabrique anglaise, remplace nécessairement, dans chacune de ces opérations, deux capitaux appartenant à des sujets de la Grande-Bretagne, et qui ont tous les deux été employés dans l’agriculture ou dans les manufactures de la Grande-Bretagne.
Le capital qui est employé à acheter des marchandises étrangères, pour la con­sommation intérieure, quand l’achat se fait avec le produit de l’industrie nationale, remplace aussi, par chaque opération de ce genre, deux capitaux distincts, mais dont un seulement est employé à soutenir l’industrie nationale. Le capital qui envoie en Portugal des marchandises anglaises et qui rapporte en Angleterre des marchandises portugaises, ne remplace, dans chacune des opérations qu’il fait, qu’un seul capital anglais ; l’autre est un capital portugais. Ainsi, quand même les retours du commerce étranger de consommation seraient aussi prompts que ceux du commerce intérieur, encore le capital employé dans celui-là ne donnerait-il que moitié d’encouragement à l’industrie ou au travail productif du pays.
Mais il est très-rare que les retours du commerce étranger de consommation soient aussi prompts que ceux du commerce intérieur. Les retours du commerce intérieur ont lieu en général avant l’année révolue, et quelquefois trois ou quatre fois dans l’année. Ceux du commerce étranger de consommation rentrent rarement avant la révolution de l’année, et quelquefois pas avant un terme de deux ou trois ans. Ainsi, un capital employé dans le commerce intérieur pourra quelquefois consommer douze opéra­tions, ou sortir et rentrer douze fois avant qu’un capital placé dans le commerce étran­ger de consommation en ait pu consommer une seule. En supposant donc des capitaux égaux, l’un donnera vingt-quatre fois plus que l’autre de soutien et d’encouragement à l’industrie du pays.
Les marchandises étrangères destinées à la consommation intérieure peuvent s’acheter quelquefois, non avec le produit de l’industrie nationale, mais avec quelques autres marchandises étrangères. Néanmoins, il faut toujours que ces dernières aient été achetées, soit immédiatement avec le produit de l’industrie nationale, soit avec quelque autre chose achetée avec ce produit ; car, excepté la voie de la guerre et de la conquête, il n’y a pas d’autre moyen d’acquérir des marchandises étrangères qu’en les échangeant contre quelque chose qu’on a produit chez soi, soit par un échange immédiat, soit après deux échanges différents, ou davantage. Par conséquent, un capital employé à faire, par un tel circuit, le commerce étranger de consommation, pro­duira à tous égards les mêmes effets qu’un capital employé à faire le même genre de commerce par la voie la plus directe, excepté que chaque retour final sera vraisem­blablement beaucoup plus éloigné encore, attendu qu’il dépend lui-même des retours de deux ou trois commerces étrangers distincts. Si l’on achète le chanvre et le lin de Riga avec du tabac de Virginie, qui a été lui-même acheté avec des marchandises de fabrique anglaise, il faut que le marchand attende jusque après les retours de deux commerces étrangers distincts, avant de pouvoir recommencer à employer le même capital en achats d’une pareille quantité de marchandises de fabrique anglaise. Si l’on avait acheté ce tabac de Virginie, non avec des marchandises de fabrique anglaise, mais avec du sucre et du rhum de la Jamaïque, qui auraient été achetés avec celles-ci, il faudrait attendre alors les retours de trois commerces étrangers. S’il arrivait que ces deux ou trois commerces étrangers distincts fussent faits par deux ou trois marchands différents, dont le second achetât la marchandise importée par le premier, et le troi­sième celle importée par le second, pour la réexporter ensuite ; dans ce cas, à la vérité, chacun de ces marchands recevrait plus vite les retours de son propre capital ; mais le retour final de tout le capital employé à consommer l’opération de ce commerce n’en serait toujours pas moins lent. Que le capital employé à parcourir ce circuit de com­merce étranger appartienne à un seul marchand ou à trois, cela ne fait pas la moindre différence quant au pays, quoique cela en puisse faire une quant à chaque marchand particulier. Dans tous les cas, il faudra toujours, pour consommer l’échange d’une certaine valeur de marchandises de fabrique anglaise contre une certaine quantité de lin et de chanvre, employer un capital trois fois plus grand qu’il n’eût été nécessaire, si les marchandises de fabrique d’une part, et le chanvre et le lin de l’autre, eussent été directement échangés ensemble.
Par conséquent, la masse de capital employé à faire ainsi par circuit le commerce étranger de consommation donnera, en général, moins de soutien et d’ encouragement au travail productif du pays, qu’un même capital employé à un commerce du même genre, mais plus direct.
Quelle que soit la nature des marchandises étrangères avec lesquelles on achète à l’étranger des choses destinées à la consommation intérieure, il n’en peut résulter aucune différence essentielle, soit dans la nature de ce commerce, soit dans l’appui et l’encouragement qu’il peut donner au travail productif du pays dans lequel se fait ce commerce. Qu’on les achète, par exemple, avec l’or du Brésil ou avec l’argent du Pérou, il faut toujours que cet or et cet argent ait été acheté, tout comme le tabac de Virginie, avec quelque chose qui soit ou produit par l’industrie du pays, ou acheté avec quelque autre chose produite par elle. Par conséquent, sous le rapport de l’intérêt du travail productif de la société, le commerce étranger de consommation, qui se fait par le moyen de l’or et de l’argent, a tous les avantages et désavantages de tout autre commerce étranger de consommation qui ferait un égal circuit., et il remplacera tout aussi vite ou tout aussi lentement le capital qui sert immédiatement à soutenir ce travail productif. Il paraîtrait même avoir un avantage sur tout autre commerce de ce genre également indirect. Le transport de ces métaux d’un lieu à un autre, vu leur grande valeur, en raison de la petitesse de leur volume, est moins coûteux que celui de presque toute autre espèce de marchandise étrangère de valeur égale. Le fret est beaucoup moindre et l’assurance n’est pas plus forte. Ainsi, par l’intermédiaire de l’or et de l’argent, on pourra souvent acheter une même quantité de marchandises étran­gè­res avec une moindre quantité du produit de l’industrie nationale, qu’on ne le pourrait par l’entremise de toute autre marchandise étrangère. De cette manière, on remplira souvent la demande du pays, plus complètement et à moins de frais que de tout autre. Savoir ensuite si par l’exportation continuelle de ces métaux un commerce de ce genre peut tendre, sous quelque autre rapport, à appauvrir le pays dans lequel il se fait, c’est ce que j’aurai occasion d’examiner fort au long dans la suite[4].
Cette portion du capital d’un pays, qui est employée au commerce de transport, est tout à fait enlevée au soutien du travail productif de ce pays, pour soutenir celui de quelques pays étrangers. Quoique par chacune de ces opérations il remplace deux capitaux distincts, aucun de ces capitaux ne fait partie du capital national. Le capital d’un négociant hollandais qui transporte en Portugal du blé de Pologne, et rapporte en Pologne des fruits et des vins de Portugal, remplace, à chaque opération qu’il fait, deux capitaux, dont aucun n’a servi à soutenir le travail productif de la Hollande, mais dont l’un a soutenu le travail productif de la Pologne, et l’autre celui du Portugal. Il n’y a que les profits qui rentrent régulièrement en Hollande, et ils constituent tout ce que ce commerce ajoute nécessairement au produit annuel des terres et du travail de ce pays. À la vérité, quand le commerce de transport que fait une nation se fait avec des bâtiments et des matelots nationaux, alors, dans le capital employé à ce commerce, la portion qui sert à payer le fret se distribue entre un certain nombre d’ouvriers pro­ductifs de ce pays, et les met en activité. Dans le fait, presque toutes les nations qui ont pris une part considérable au commerce de transport l’ont fait de cette manière. C’est probablement de là que ce commerce lui-même a pris son nom, les gens de ce pays étant réellement les voituriers des autres nations. Avec cela, il ne paraît pas essentiel à la nature de ce commerce qu’il se fasse ainsi. Un négociant hollandais, par exemple, peut employer son capital à des opérations de commerce entre la Pologne et le Portugal, en transportant une partie du superflu de l’un de ces pays à l’autre dans des vaisseaux anglais et non hollandais. Il est à présumer qu’il opère ainsi dans quelques circonstances. C’est pourtant sous ce point de vue qu’on a supposé que le commerce de transport était particulièrement avantageux à un pays tel que la Grande-Bretagne, dont la défense et la sûreté dépendent du nombre de ses matelots et de l’étendue de sa marine. Mais le même capital peut employer tout autant de bâtiments et de matelots, s’il est placé dans le commerce étranger de consommation, ou même dans le commerce intérieur par cabotage, que s’il était placé dans le commerce de transport. Le nombre de vaisseaux et de matelots qu’un capital peut employer ne dépend pas de la nature du commerce dans lequel est placé ce capital, mais il dépend en partie du volume des marchandises en proportion de leur valeur, et en partie de la distance qui se trouve entre les deux ports où elles sont transportées ; et il dépend principalement de la première de ces deux circonstances. Le commerce de charbon, par exemple, qui se fait de Newcastle à Londres, emploie plus de bâtiments et de matelots que tout le commerce de transport de l’Angleterre, quoique ces deux ports ne soient pas très-éloignés l’un de l’autre. Ce ne serait donc pas toujours un moyen assuré d’augmenter la marine d’un pays, que de forcer par des encouragements extraordi­naires les capitaux à se porter dans le commerce de transport en plus grande quantité qu’ils ne s’y porteraient naturellement.
Ainsi, le capital qui sera employé dans le commerce intérieur d’un pays donnera, en général, de l’appui et de l’encouragement à une plus grande quantité de travail productif dans ce pays, et augmentera plus la valeur de son produit annuel, que ne le fera un pareil capital employé au commerce étranger de consommation ; et le capital qui sera employé dans ce dernier genre de commerce aura, sous ces deux rapports, encore un bien plus grand avantage sur le capital employé au commerce de transport. La richesse publique d’un pays et même sa puissance, en tant que la puissance peut dépendre de la richesse, doit toujours être en raison de la valeur de son produit annuel, qui est la source où se puisent, en définitive, tous les impôts. Or, le grand objet que se propose partout l’économie politique, c’est d’augmenter la richesse et la puis­sance du pays. Elle ne doit donc accorder aucune préférence au commerce étran­ger de consommation sur le commerce intérieur, ni au commerce de transport sur aucun des deux autres ; elle ne doit pas encourager l’un de ces commerces plus que l’autre ; elle ne doit pas chercher, par des appâts ou par des contraintes, à attirer dans l’un ou l’autre de ces deux canaux une plus grande quantité du capital du pays, que celle qui s’y jetterait d’elle-même dans le cours naturel des choses.
Toutefois, chacune de ces différentes branches de commerce est non-seulement avantageuse, mais elle est même nécessaire et indispensable, quand elle est naturelle­ment amenée par le cours des choses, sans gêne et sans contrainte.
Quand le produit d’une branche particulière d’industrie excède ce qu’exige la de­man­de du pays, il faut bien qu’on envoie le surplus à l’étranger, pour l’échanger contre quelque chose qui soit demandé dans l’intérieur. Sans cette exportation, une partie du travail productif du pays viendrait à cesser, et la valeur de son produit annuel dimi­nuerait nécessairement. La terre et le travail de la Grande-Bretagne produisent natu­rel­lement plus de blé, de lainages et de quincailleries que n’en exige la demande du marché intérieur. Il faut donc exporter le surplus et l’échanger contre quelque chose dont il y ait demande dans le pays. Ce n’est que par le moyen de cette exportation que ce surplus pourra acquérir une valeur suffisante pour compenser le travail et la dépense qu’il en coûte pour le produire. Le voisinage des côtes de la mer et les bords des rivières navigables ne sont des situations avantageuses pour l’industrie que par la seule raison qu’elles facilitent les moyens d’exporter et d’échanger ces produits superflus contre quelque chose qui est plus demandé dans le pays.
Lorsque les marchandises étrangères qu’on achète ainsi avec le produit superflu de l’industrie nationale excèdent la demande du marché du pays, il faut bien aussi réexporter à l’étranger le superflu de ces marchandises étrangères, et l’échanger contre quelque chose qui soit plus demandé dans le pays. On achète tous les ans, avec une partie du produit superflu de l’industrie de la Grande-Bretagne, environ 96 mille muids[5] de tabac dans la Virginie et le Maryland. Or, la demande de la Grande-Bretagne n’en exige peut-être pas plus de 14 mille. Si les 82 mille restant ne pouvaient être exportés et échangés contre quelque chose de plus demandé dans le pays, l’importation de ce restant cesserait aussitôt, et avec elle le travail productif de tous ceux des habitants de la Grande-Bretagne qui sont maintenant employés à préparer les marchandises avec lesquelles ces 82 mille muids sont achetés tous les ans. Ces marchandises, qui sont une partie du produit des terres et du travail de la Grande-Bretagne, n’ayant pas de marché pour elles au-dedans et étant privées de celui qu’elles avaient au-dehors, cesseraient nécessairement d’être produites. On voit donc qu’en certaines occasions le commerce étranger de consommation le plus détourné sera tout aussi nécessaire que le plus direct pour soutenir le travail productif d’un pays et la valeur de son produit annuel.
Quand la masse des capitaux d’un pays est parvenue à un tel degré d’accroisse­ment qu’elle ne peut être toute employée à fournir à la consommation de ce Pays et à faire valoir son travail productif, alors le superflu de cette masse se décharge naturel­lement dans le commerce de transport, et est employé à rendre les mêmes services à des pays étrangers. Le commerce de transport est bien l’effet et le symptôme naturel d’une grande richesse nationale ; mais il, ne paraît pas qu’il en soit la cause naturelle. Les hommes d’État qui ont cherché à le favoriser par des encouragements particuliers ont pris, à ce qu’il paraît, l’effet de ce symptôme pour la cause. La Hollande, le plus riche pays sans comparaison de tous ceux de l’Europe, en proportion de son territoire et de sa population, fait, par cette raison, la plus grande partie du commerce de transport de l’Europe. L’Angleterre, le second pays peut-être pour la richesse, passe aussi pour faire une très-grande partie de ce commerce ; quoique ce qu’on prend com­mu­nément pour le commerce de transport de l’Angleterre pourrait souvent, au fond, n’être autre chose qu’un commerce étranger de consommation détourné et indirect. Tel est, en grande partie, le commerce qui porte aux différents marchés de l’Europe les marchandises des Indes orientales et occidentales, ainsi que celles de l’Amérique. Ces marchandises sont achetées, en général, ou immédiatement avec le produit de l’industrie de la Grande-Bretagne, ou avec quelque autre chose achetée avec ce pro­duit et, en général, les retours définitifs de ce commerce sont aussi destinés à l’usage et à la consommation de la Grande-Bretagne. Ce qui constitue peut-être les branches principales de ce qui est proprement le commerce des transports de la Grande-Bretagne, c’est celui qui se fait, avec des vaisseaux de cette nation, entre les différents ports de la Méditerranée, et quelque autre commerce du même genre qui se fait, par des marchands anglais, entre les différents ports de l’Inde.
L’étendue du commerce intérieur et du capital qui peut y être employé a néces­sai­rement pour limites la valeur du produit superflu de tous les endroits du pays qui sont éloignés les uns des autres, et qui ont besoin d’échanger ensemble leurs productions respectives. Celle du commerce étranger de consommation a pour limites la valeur du produit superflu de tout le pays et de ce qui peut s’acheter avec ce superflu. Celle du commerce de transport a pour limites la valeur du produit superflu de tous les différents pays du monde. Ainsi, son étendue possible est, en quelque sorte, infinie en comparaison de celle des deux autres, et elle est capable d’absorber les plus grands capitaux.
Le seul motif qui détermine le possesseur d’un capital à l’employer plutôt dans l’agriculture, ou dans les manufactures, ou dans quelque branche particulière de commerce en gros ou en détail, c’est la vue de son propre profit. Il n’entre jamais dans sa pensée de calculer combien chacun de ces différents genres d’emploi mettra de travail productif en activité, on ajoutera de valeur au produit annuel des terres et du travail de son pays. Ainsi, dans le pays où l’agriculture est le plus profitable de tous les emplois, et où la route la plus sûre pour aller à une brillante fortune est de faire valoir et d’améliorer des terres, les capitaux des particuliers seront tout naturellement employés de la manière qui se trouvera en même temps la plus avantageuse à la socié­té en général. Toutefois, il ne paraît pas qu’en aucun endroit de l’Europe les profits de l’agriculture aient aucune supériorité sur ceux des autres emplois de capi­taux. À la vérité, depuis quelques années, il a paru dans tous les coins de l’Europe des spéculateurs qui ont amusé le public par des calculs magnifiques sur les profits à faire dans la culture et l’amélioration des terres. Sans entrer dans aucune discussion parti­culière sur leurs calculs, il ne faut qu’une observation bien simple pour nous montrer la fausseté de leurs résultats. Tous les jours nous voyons les fortunes les plus brillan­tes acquises dans le cours de la vie d’une seule personne, par le moyen du commerce et des manufactures, après avoir commencé souvent par un très-faible capital, et quelquefois même sans aucun capital. Une pareille fortune acquise par l’agri­culture dans le même temps et avec aussi peu de capital est un phénomène dont l’Europe n’offrirait peut-être pas un seul exemple dans tout le cours du siècle. Cepen­dant, dans tous les grands pays de l’Europe, il y a encore beaucoup de bonne terre qui reste inculte, et la majeure partie de celle qui est cultivée est encore bien loin d’être portée au degré d’amélioration dont elle est susceptible. Ainsi, presque partout l’agriculture est en état d’absorber un capital beaucoup plus grand que ce qui y a encore été employé jusqu’à présent.
Quelles sont donc, dans l’histoire politique de l’Europe, les circonstances qui ont donné aux genres d’industrie établis dans les villes un avantage tellement considérable sur celle qui s’exerce dans les campagnes, que des particuliers aient trouvé souvent plus profitable pour eux d’employer leurs capitaux au commerce de transport des pays les plus reculés de l’Asie et de l’Amérique, que de les employer à l’amélioration et à la culture de terres de la plus grande fertilité, et situées dans leur voisinage ? C’est ce que je vais tâcher de développer, avec toute l’étendue possible, dans les deux livres suivants.
 
 
 
↑ On s’étonne qu’un esprit aussi pénétrant que celui de Smith ait pu admettre une proposition aussi erronée que celle-ci : « La nature ne fait rien pour l’homme dans les manufactures. » La puissance de l’eau et du vent qui mettent en mouvement nos machines, supportent nos vaisseaux et les poussent sur la mer, la pression de l’atmosphère et l’élasticité de la vapeur qui nous permettent de construire les plus étonnantes machines, ne sont-ils pas des dons spontanés de la nature ? Mac Culloch.
↑ Ce mot superflu désigne ici tout ce qu’un pays produit d’une marchandise quelconque au delà de ce qu’il en consomme lui-même (en anglais surplus produce).
↑ Plusieurs auteurs, et entre autre Montesquieu, ont donné à ce commerce le nom de commerce d’économie.
↑ Livre IV, chapitre ier
↑ Hogshead, mesure qui a environ 1/7 de moins en capacité que le muid de Paris, de 288 pintes (environ 2 hectolitres et demi).

%%%%%%%%%%%%%%%%%%%%%%%%%%%%%%%%%%%%%%%%%%%%%%%%%%%%%%%%%%%%%%%%%%%%%%%%%%%%%%%%
%                                                                              %
%                                  Livre III                                   %
%                                                                              %
%%%%%%%%%%%%%%%%%%%%%%%%%%%%%%%%%%%%%%%%%%%%%%%%%%%%%%%%%%%%%%%%%%%%%%%%%%%%%%%%

\part{De la marche différente des progrès de l’opulence chez différentes nations}
\markboth{De la marche différente des progrès de l’opulence chez différentes nations}{}

%%%%%%%%%%%%%%%%%%%%%%%%%%%%%%%%%%%%%%%%%%%%%%%%%%%%%%%%%%%%%%%%%%%%%%%%%%%%%%%%
%                                  Chapitre 1                                  %
%%%%%%%%%%%%%%%%%%%%%%%%%%%%%%%%%%%%%%%%%%%%%%%%%%%%%%%%%%%%%%%%%%%%%%%%%%%%%%%%

\chapter{Du cours naturel des progrès de l’opulence}
\markboth{Du cours naturel des progrès de l’opulence}{}

Le grand commerce de toute société civilisée est celui qui s’établit entre les habitants de la ville et ceux de la campagne.
Il consiste dans l’échange du produit brut contre le produit manufacturé, échange qui se fait soit immédiatement, soit par l’intervention de l’argent ou de quelque espèce de papier qui représente l’argent. La campagne fournit à la ville des moyens de subsistance et des matières pour ses manufactures. La ville rembourse ces avances en renvoyant aux habitants de la campagne une partie du produit manufacturé. La ville, dans laquelle il n’y a ni ne peut y avoir aucune reproduction de subsistances, gagne, à proprement parler, toute sa subsistance et ses richesses sur la campagne. Il ne faut pourtant pas s’imaginer pour cela que la ville fasse ce gain aux dépens de la cam­pagne. Les gains sont réciproques pour l’une et pour l’autre et, en ceci, comme en toute autre chose, la division du travail tourne à l’avantage de chacune des différentes personnes employées aux tâches particulières dans lesquelles le travail se subdivise. Les habitants de la campagne achètent de la ville une plus grande quantité de denrées manufacturées avec le produit d’une bien moindre quantité de leur propre travail qu’ils n’auraient été obligés d’en employer s’ils avaient essayé de les préparer eux-mêmes. La ville fournit un marché au surplus du produit de la campagne, c’est-à-dire à ce qui excède la subsistance des cultivateurs, et c’est là que les habitants de la campagne échangent ce surplus contre quelque autre chose qui est en demande chez eux. Plus les habitants de la ville sont nombreux et plus ils ont de revenu, plus est étendu le marché qu’ils fournissent à ceux de la campagne ; et plus ce marché est étendu, plus il est toujours avantageux pour le grand nombre. Le blé qui croît à un mille de la ville s’y vend au même prix que celui qui vient d’une distance de vingt milles. Or, le prix de celui-ci, en général, doit non-seulement payer la dépense de le faire croître et de l’amener au marché, mais rapporter encore au fermier les profits ordinaires de la culture. Ainsi, les propriétaires et cultivateurs qui demeurent dans le voisinage de la ville gagnent, dans le prix de ce qu’ils vendent, outre les profits ordinaires de la cul­ture, toute la valeur du transport du pareil produit qui est apporté d’endroits plus éloignés, et ils épargnent de plus toute la valeur d’un pareil transport sur le prix de ce qu’ils achètent. Comparez la culture des terres situées dans le voisinage d’une ville considérable, avec celle des terres qui en sont à quelque distance, et vous pourrez aisément vous convaincre combien la campagne tire d’avantage de son commerce avec la ville. Parmi toutes les absurdités de cette théorie qu’on a imaginées sur la ba­lan­ce du commerce, on ne s’est jamais avisé de prétendre, ou que la campagne perd dans son commerce avec la ville, ou que la ville perd par son commerce avec la campagne qui la fait subsister.
La subsistance étant, dans la nature des choses, un besoin antérieur à ceux de commo­dité et de luxe, l’industrie qui fournit au premier de ces besoins doit nécessai­rement précéder celle qui s’occupe de satisfaire les autres. Par conséquent, la culture et l’amélioration de la campagne, qui fournit la subsistance, doivent nécessairement être antérieures aux progrès de la ville, qui ne fournit que les choses de luxe et de commodité. C’est seulement le surplus du produit de la campagne, c’est-à-dire l’excé­dant de la subsistance des cultivateurs, qui constitue la subsistance de la ville, la­quel­le, par conséquent, ne peut se peupler qu’autant que ce surplus de produit vient à grossir. À la vérité, il se peut bien que la ville ne tire pas toujours la totalité de ses subsistances de la campagne qui l’avoisine, ni même du territoire auquel elle appar­tient, mais qu’elle les tire de campagnes fort éloignées ; et cette circonstance, sans faire exception à la règle générale, a néanmoins fait varier considérablement, chez diffé­rents peuples et dans différents siècles, la marche des progrès de l’opulence.
Cet ordre de choses, qui est en général établi par la nécessité, quoique certains pays puissent faire exception, se trouve, en tout pays, fortifié par le penchant naturel de l’homme. Si ce penchant naturel n’eût jamais été contrarié par les institutions hu­mai­nes, nulle part les villes ne se seraient accrues au-delà de la population que pou­vait soutenir l’état de culture et d’amélioration du territoire dans lequel elles étaient situées, au moins jusqu’à ce que la totalité de ce territoire eût été pleinement cultivée et améliorée. À égalité de profits, ou à peu de différence près, la plupart des hommes préféreront employer leurs capitaux à la culture et à l’amélioration de la terre, plutôt que de les placer dans des manufactures ou dans le commerce étranger. Une personne qui fait valoir son capital sur une terre l’a bien plus sous les yeux et à son comman­dement, et sa fortune est bien moins exposée aux accidents que celle du commerçant ; celui-ci est souvent obligé de confier la sienne, non-seulement aux vents et aux flots, mais à des éléments encore plus perfides, la folie et l’injustice des hommes, quand il accorde de grands crédits, dans des pays éloignés, à des personnes dont il ne peut guère bien connaître la situation ni le caractère. Au contraire, le capital qu’un proprié­taire a fixé par des améliorations, au sol même de sa terre, paraît être aussi assure que peut le comporter la nature des choses humaines. D’ailleurs, la beauté de la campagne, les plaisirs de la vie champêtre, la tranquillité d’esprit dont on espère y jouir, et l’état d’indépendance qu’elle procure réellement, partout où l’injustice des lois ne vient pas s’y opposer, sont autant de charmes plus ou moins séduisants pour tout le monde ; et comme la destination de l’homme, a son origine, fut de cultiver la terre, il semble conserver dans toutes les périodes de sa vie une prédilection pour cette occupation primitive de son espèce.
À la vérité, la culture de la terre, à moins d’entraîner avec soi beaucoup d’incom­modités et de continuelles interruptions, ne saurait guère se passer de l’aide de quel­ques artisans. Les forgerons, les charpentiers, les faiseurs de charrues et de voitures, les maçons et briquetiers, les tanneurs, les cordonniers et les tailleurs, sont tous gens aux services desquels le fermier a souvent recours. Ces artisans ont aussi, de temps en temps, besoin les uns des autres ; et leur résidence n’étant pas nécessairement attachée, comme celle du fermier, à tel coin de terre plutôt qu’à l’autre, ils s’établissent natu­rel­lement dans le voisinage les uns des autres, et forment ainsi une petite ville ou un village. Le boucher, le brasseur et le boulanger viennent bientôt s’y réunir, avec beau­coup d’autres artisans et de détaillants nécessaires ou utiles pour leurs besoins journa­liers, et qui contribuent encore d’autant à grossir la ville. Les habitants de la ville et ceux de la campagne sont réciproquement les serviteurs les uns des autres. La ville est une foire ou marché continuel où se rendent les habitants de la campagne pour échan­ger leur produit brut contre du produit manufacturé. C’est ce commerce qui fournit aux habitants de la ville et les matières de leur travail, et les moyens de leur subsis­tance. La quantité d’ouvrage fait qu’ils vendent aux habitants de la campagne détermi­ne nécessairement la quantité de matières et de vivres qu’ils achètent. Ainsi, ai leur occupation ni leur subsistance ne peuvent se multiplier en raison de la demande que fait la campagne d’ouvrage fait, et cette demande ne peut elle-même se multiplier qu’en raison de l’extension et de l’amélioration de la culture. Si les institutions hu­maines n’eussent jamais troublé le cours naturel des choses, les progrès des villes en richesses et en population auraient donc, dans toute société politique, marché à la suite et en proportion de la culture et de l’amélioration de la campagne ou du territoire environnant.
Dans nos colonies de l’Amérique septentrionale, où l’on peut encore se procurer des terres à cultiver à des conditions faciles, il ne s’est jusqu’ici établi, dans aucune de leurs villes, de manufactures pour la vente au loin. Dans ce pays, quand un artisan a amassé un peu plus de fonds qu’il ne lui en faut pour faire aller le commerce avec les gens de la campagne voisine, en fournitures de son métier, il ne cherche pas à monter, avec ce capital, une fabrique pour étendre sa vente plus au loin, mais il l’emploie à acheter de la terre inculte et à la mettre en valeur. D’artisan il devient planteur ; ni le haut prix des salaires, ni les moyens que le pays offre aux artisans de se procurer de l’aisance, ne peuvent le décider à travailler pour autrui plutôt que pour lui-même. Il sent qu’un artisan est le serviteur des maîtres qui le font vivre, mais qu’un colon qui cultive sa propre terre, et qui trouve dans le travail de sa famille de quoi satisfaire aux premiers besoins de la vie, est vraiment son maître et vit indépendant du monde entier.
Au contraire, dans les pays où il n’y a pas de terres incultes, ou du moins qu’on puisse se procurer à des conditions faciles, tout artisan qui a amassé plus de fonds qu’il ne saurait en employer dans les affaires qui peuvent se présenter aux environs, cherche à créer des produits propres à être vendus sur un marché plus éloigné. Le forgeron élève une fabrique de fer ; le tisserand se fait manufacturier en toiles ou en laineries. Avec le temps, ces différentes manufactures viennent à se subdiviser par degrés, et par ce moyen elles se perfectionnent de mille manières dont on peut aisé­ment se faire idée, et qu’il est conséquemment inutile d’expliquer davantage. 
Quand on cherche à employer un capital, on préfère naturellement, à égalité de pro­fil ou à peu près, les manufactures au commerce étranger, par la même raison qu’on préfère naturellement l’agriculture aux manufactures ; si le capital du proprié­taire ou du fermier est plus assuré que celui du manufacturier, le capital du manu­facturier, qui est toujours sous ses yeux et à son commandement, est aussi plus assuré que celui d’un marchand qui fait le commerce étranger. À la vérité, dans quelque période[1] que soit une société, il faut toujours que le surplus de ses produits bruts et manufacturés, ou ce qui n’est point en demande chez elle, soit envoyé au-dehors pour y être échangé contre quelque chose dont il y ait demande au-dedans. Mais il importe fort peu pour cela que le capital qui envoie à l’étranger ce produit superflu soit un capital étranger ou un capital national. Si la société n’a pas encore acquis un capital suffisant pour cultiver toutes ses terres et aussi pour manufacturer le plus complète­ment possible tout son produit brut, il y a même pour elle un avantage considérable à ce que son superflu soit exporté par un capital étranger, afin que tout le capital de la société soit réservé pour les emplois les plus utiles. La richesse de l’ancienne Égypte, celle de la Chine et de l’Indostan, suffisent pour démontrer qu’une nation peut parve­nir à un très-haut degré d’opulence, quoique la plus grande partie de son exportation se fasse par des étrangers. Si nos colonies de l’Amérique septentrionale et des Indes occidentales n’avaient eu d’autre capital que celui qui lui appartenait pour exporter le surplus de leurs produits, leurs progrès eussent été bien moins rapides.
Ainsi, suivant le cours naturel des choses, la majeure partie du capital d’une société naissante se dirige d’abord vers l’agriculture, ensuite vers les manufactures, et en dernier lieu vers le commerce étranger. Cet ordre de choses est si naturel, que dans toute société qui a quelque territoire, il a toujours, à ce que je crois, été observé à un certain point. On y a toujours cultivé des terres avant qu’aucunes villes considérables y aient été établies, et on a élevé dans ces villes quelques espèces de fabriques grossières avant qu’on ait pensé sérieusement à faire par soi-même le commerce étranger.
Mais quoique cet ordre naturel de choses ait eu lieu jusqu’à un certain point en toute société possédant un territoire, cependant il a été tout à fait interverti, à beau­coup d’égards, dans tous les États modernes de l’Europe. C’est le commerce étranger de quelques-unes de leurs grandes villes qui a introduit toutes leurs plus belles fabriques ou celles dont les produits sont destinés à être vendus au loin, et ce sont à la fois les manufactures et le commerce étranger qui ont donné naissance aux princi­pales améliorations de la culture des terres. Les mœurs et usages qu’avait introduits chez ces peuples la nature de leur gouvernement originaire, et qu’ils conservèrent encore après que ce gouvernement eut essuyé de grands changements, furent la cause qui les mit dans la nécessité de suivre cette marche rétrograde et contraire à l’ordre naturel.
 
 
↑ La marche progressive d’une société, depuis l’état qu’on nomme sauvage jusqu’à celui d’une grande opulence, peut cire partagée en plusieurs divisions relatives à chacun de ces divers degrés d’avancement, et c’est dans ce sens que doit être pris ici le mot de période, qui désigne une de ces divisions.

%%%%%%%%%%%%%%%%%%%%%%%%%%%%%%%%%%%%%%%%%%%%%%%%%%%%%%%%%%%%%%%%%%%%%%%%%%%%%%%%
%                                  Chapitre 2                                  %
%%%%%%%%%%%%%%%%%%%%%%%%%%%%%%%%%%%%%%%%%%%%%%%%%%%%%%%%%%%%%%%%%%%%%%%%%%%%%%%%

\chapter{Comment l’agriculture fut découragée en Europe après la chute de l’Empire romain}
\markboth{Comment l’agriculture fut découragée en Europe après la chute de l’Empire romain}{}

Lorsque les peuples de la Scythie et de la Germanie envahirent les provinces occidentales de l’empire romain, les désordres qu’entraîna une si grande révolution durèrent pendant plusieurs siècles. Les violences et les rapines que les barbares exerçaient contre les anciens habitants interrompirent le commerce entre la ville et la campagne. On déserta les villes, on laissa les campagnes sans culture, et les provinces occidentales de l’Europe, qui avaient joui, sous le gouvernement des Romains, d’un degré considérable d’opulence, tombèrent dans le dernier état de barbarie et de misère. Dans le cours de ces désordres, les chefs et les principaux capitaines de ces nations barbares acquirent ou usurpèrent pour eux-mêmes la majeure partie des terres de ces provinces. Une grande partie resta inculte ; mais, cultivée ou non, aucune terre ne demeura sans maître. Chaque usurpateur travailla à grossir son lot, et la plus grande partie se trouva réunie dans les mains d’un petit nombre de grands propriétaires.
Cette première réunion de terres incultes par grands lots en un petit nombre de mains fut une grande calamité, mais qui aurait pu n’être que passagère. Elles se seraient bientôt après subdivisées de nouveau ; naturellement, les successions ou les aliénations les auraient réduites en petits lots. Mais la loi de primogéniture s’opposa à ce qu’elles fussent partagées par la voie des successions ; l’introduction des substitu­tions[1] empêcha qu’elles ne fussent morcelées par des aliénations.
Lorsqu’on ne voit dans les propriétés territoriales qu’un moyen de subsistance et de jouissance, comme dans les propriétés mobilières, alors la loi naturelle de succes­sion les partage, de même que celles-ci, entre tous les enfants d’une même famille, entre tous ceux de qui la subsistance et le bien-être étaient censés également chers au père de famille. Aussi cette loi naturelle des successions eut-elle lieu chez les Ro­mains, qui ne firent pas plus de distinction, pour la succession des terres, entre les aînés et les puînés, entre les mâles et les femelles, que nous n’en faisons pour les partages de biens meubles. Mais quand on regarda les terres, non pas comme de sim­ples moyens de subsistance, mais comme des moyens de puissance et de protection, on trouva plus convenable qu’elles descendissent sans partage à un seul. Dans ces temps de désordre, chaque grand propriétaire était une espèce de petit prince ; ses vassaux étaient ses sujets ; il était leur juge et à quelques égards leur législateur pendant la paix, et leur chef pendant la guerre. Il faisait la guerre quand il le jugeait à propos, souvent contre ses voisins, et quelquefois contre son souverain.
La sûreté d’une terre, la protection que le maître pouvait donner à ceux qui y de­meu­raient, dépendaient de son étendue. La diviser, c’eût été la détruire et l’exposer à être de toutes parts ravagée et engloutie par les incursions des voisins. La loi de primo­géniture s’établit ainsi dans la succession des terres, non pas au premier mo­ment, mais dans la suite des temps, par la même raison qui a fait qu’elle s’est généra­lement établie dans les monarchies pour la succession au trône, quoiqu’elle n’ait pas toujours eu lieu au commencement de leur institution. Pour que la puissance et, par conséquent, la sûreté de la monarchie ne soient pas affaiblies par un partage, il faut qu’elle descende tout entière sur la tête des enfants. Pour savoir auquel d’entre eux on accorderait une préférence de si haute importance, il a fallu se déterminer par quelque règle générale qui ne fût pas fondée sur les distinctions si douteuses du mérite person­nel, mais sur quelque différence simple et évidente qui ne pût jamais être matière à contestation. Parmi les enfants d’une même famille, il ne peut y avoir que les diffé­rences de l’âge et du sexe qui ne soient pas susceptibles d’être contestées. Le sexe mâle est, en général, préféré à l’autre, et quand toutes choses sont égales d’ailleurs, l’aîné a toujours le pas sur le puîné ; de là l’origine du droit de primogéniture, et de ce qu’on appelle succession de ligne.
Il arrive souvent que les lois subsistent encore longtemps après qu’ont disparu les circonstances auxquelles elles doivent leur origine, et qui seules pouvaient les rendre raisonnables. Dans l’état actuel de l’Europe, le propriétaire d’un seul acre de terre est aussi parfaitement assuré de sa possession que le propriétaire de cent mille. Cepen­dant, on a encore égard au droit de primogéniture ; et comme c’est, de toutes les insti­tu­tions, la plus propre à soutenir l’orgueil de la distinction des familles, il est vrai­semblable qu’elle doit durer encore plusieurs siècles. Sous tout autre point de vue, rien ne peut être plus contraire aux vrais intérêts d’une nombreuse famille qu’un droit qui, pour enrichir un des enfants, réduit tous les autres à la misère.
Les substitutions sont une conséquence naturelle de la loi de primogéniture. Elles furent imaginées pour conserver une certaine succession de ligne dont la loi de primogéniture fit concevoir la première idée, et pour empêcher qu’aucune partie d’une terre ne fût démembrée de sa consistance primitive et mise hors de la ligne préférée, soit par don, legs ou aliénation, soit par l’inconduite ou la mauvaise fortune de ses pos­sesseurs successifs ; elles étaient tout à fait inconnues chez les Romains. Leurs substitutions et fidéicommis n’avaient aucune ressemblance avec nos substitutions actuelles, quoiqu’il ait plu à quelques jurisconsultes français d’habiller cette institution moderne avec les noms et les formes extérieures de l’ancienne.
Quand les propriétés foncières étaient des espèces de principautés, les substitu­tions pouvaient n’être pas déraisonnables. Semblables à ce que certaines monarchies appellent leurs lois fondamentales, elles pouvaient souvent empêcher que la sûreté de plusieurs milliers de personnes ne fût comprise par le caprice ou les dissipations d’un individu. Mais dans l’état actuel de l’Europe, où les petites propriétés, aussi bien que les plus grandes, tirent toute leur sûreté de la loi, il ne peut y avoir rien de plus absurde. Ces institutions sont fondées sur la plus fausse de toutes les suppositions, la supposition que chaque génération successive n’a pas un droit égal à la terre qu’elle possède et à toutes ses autres possessions, mais que la propriété de la génération actuelle peut être restreinte et réglée d’après la fantaisie de gens morts il y a peut-être cinq cents ans. Cependant, les substitutions sont encore en vigueur dans la majeure partie de l’Europe, et particulièrement dans les pays où la noblesse de naissance est une qualification indispensable pour prétendre aux honneurs civils ou militaires. On regarde donc les substitutions comme nécessaires pour maintenir le droit exclusif de la noblesse aux dignités et aux honneurs de son pays, et cette classe d’hommes ayant déjà usurpé sur le reste de ses concitoyens un privilège inique, de peur que leur pauvreté ne rendît celui-ci ridicule, on a trouvé raisonnable qu’ils y en joignissent un autre. À la vérité, le droit commun de l’Angleterre a en haine, dit-on, la perpétuité des propriétés, et les substitutions y sont aussi plus restreintes que dans toute autre monarchie de l’Europe, quoique l’Angleterre elle-même n’en soit pas entièrement affranchie. En Écosse, il y a plus du cinquième, peut-être plus du tiers des propriétés du pays, qui sont encore actuellement dans les liens d’une substitution rigoureuse[2].
De cette manière, non-seulement de grandes étendues de terres incultes se trouvè­rent réunies dans les mains de quelques familles, mais encore la possibilité que ces terres fussent jamais divisées fut prévenue par toutes les précautions imaginables. Or, il arrive rarement qu’un grand propriétaire soit un grand faiseur d’améliorations. Dans les temps de désordres qui donnèrent naissance à ces institutions barbares, un grand propriétaire n’était occupé que du soin de défendre son territoire ou du désir d’étendre son autorité et sa juridiction sur celui de ses voisins. Il n’avait pas le loisir de penser à cultiver ses terres et à les mettre en valeur. Quand le règne de l’ordre et des lois lui en laissa le loisir, il n’en eut souvent pas le goût, et presque jamais il ne possédait les qualités, qu’exige une telle occupation. La dépense de sa personne et de sa maison absorbant ou même surpassant son revenu, comme cela arrivait le plus souvent, où aurait-il pris un capital pour le destiner à un pareil emploi ? S’il était de caractère à faire des économies, il trouvait en général plus profitable de placer ses épargnes annuelles dans de nouvelles acquisitions, que de les employer à améliorer ses anciens domaines. Pour mettre une terre en valeur avec profit, il faut, comme pour toutes les entreprises de commerce, la plus grande attention sur les plus petits gains et sur les moindres épargnes, ce dont est rarement capable un homme né avec une grande fortune, fût-il même naturellement économe. La situation d’un homme de cette sorte le dispose plutôt à s’occuper de quelque genre de décoration qui flatte sa fantaisie qu’à spéculer sur des profits dont il a si peu besoin. L’élégance de sa parure, de son logement, de son équipage, de ses ameublements, voilà des objets auxquels, dès son enfance, il a été accoutumé à donner ses soins. La pente que de telles habitudes donnent naturellement à ses idées le dirige encore quand il vient à s’occuper d’amé­liorer ses terres ; il embellira peut-être quatre à cinq cents acres autour de sa maison, avec dix fois plus de dépense que la chose ne vaudra après toutes ces améliorations, et il trouve que s’il s’avisait de faire sur la totalité de ses propriétés une amélioration du même genre (et son goût ne le porte guère à en faire d’autres), il serait en banqueroute avant d’avoir achevé la dixième partie d’une telle entreprise. Il y a encore aujourd’hui, dans chacun des royaumes unis, de ces grandes terres qui sont restées, sans interruption, dans la même famille depuis le temps de l’anarchie féodale. Il ne faut que comparer l’état actuel de ces domaines avec les possessions des petits propriétaires des environs pour juger, sans autre argument, combien les propriétés si étendues sont peu favo­rables aux progrès de la culture.
S’il y avait peu d’améliorations à attendre de la part de ces grands propriétaires, il y avait encore bien moins à espérer de ceux qui tenaient la terre sous eux. Dans l’ancien état de l’Europe, tous ceux qui cultivaient les terres étaient tenanciers à volonté. Ils étaient tous ou presque tous esclaves ; mais le genre de leur servitude était plus adouci que celui qui était en usage chez les anciens Grecs et chez les Romains, ou même dans nos colonies des Indes occidentales. Ils étaient censés appartenir plus directement à la terre qu’à leur maître. Aussi, on les vendait avec la terre, mais point séparément d’elle. Ils pouvaient se marier, pourvu qu’ils eussent le consentement de leur maître ; mais ensuite celui-ci ne pouvait pas rompre cette union en vendant l’homme et la femme à des personnes différentes. Si le maître tuait ou mutilait quel­qu’un de ses ’’serfs’’, il était sujet à une peine qui pourtant, en général, était fort légère. Au reste, ils étaient incapables d’acquérir aucune propriété ; tout ce qu’ils avaient était acquis à leur maître, qui pouvait le leur prendre à sa volonté. Toute culture et toute amélioration faite par de tels esclaves était proprement le fait de leur maître ; elle se faisait à ses frais ; les semences, les bestiaux et les instruments de labourage, tout était à lui. Il avait la totalité du profit, ses esclaves ne pouvaient rien gagner que leur subsistance journalière. C’était donc le propriétaire lui-même, dans ce cas, qui tenait sa propre terre et la faisait valoir par les mains de ses serfs. Cette espèce de servitude subsiste encore en Russie, en Pologne, en Hongrie, en Bohême, en Moravie et dans quelques autres parties de l’Allemagne. Ce n’est que dans les provinces de l’ouest et du sud-ouest de l’Europe qu’elle s’est totalement anéantie par degrés.
Mais s’il ne faut pas espérer que de grands propriétaires fassent jamais de grandes améliorations, c’est surtout quand ils emploient le travail de gens qui sont esclaves[3]. L’expérience de tous les temps et de toutes les nations, je crois, s’accorde pour démontrer que l’ouvrage fait par des esclaves, quoiqu’il paraisse ne coûter que les frais de leur subsistance, est, au bout du compte, le plus cher de tous. Celui qui ne peut rien acquérir en propre ne peut avoir d’autre intérêt que de manger le plus possible et de travailler le moins possible. Tout travail au-delà de ce qui suffit pour acheter sa subsistance ne peut lui être arraché que par la contrainte et non par aucune considération de son intérêt personnel. Pline et Columelle ont remarqué l’un et l’autre combien la culture du blé dégénéra dans l’ancienne Italie, combien elle rapporta peu de bénéfice au maître, quand elle fut laissée aux soins des esclaves. Au temps d’Aris­to­te, elle n’allait pas beaucoup mieux dans la Grèce. En parlant de la république imaginaire décrite dans les lois de Platon : « Pour entretenir, dit-il, cinq mille hom­mes oisifs (qui était le nombre de guerriers supposé nécessaire pour la défense de cette république), avec leurs femmes et leurs domestiques, il faudrait un territoire d’une étendue et d’une fertilité sans bornes, comme les plaines de Babylone. »
L’orgueil de l’homme fait qu’il aime à dominer, et que rien ne le mortifie autant que d’être obligé de descendre avec ses inférieurs aux voies de la persuasion. Aussi, toutes les fois que la loi le lui permet, et que la nature de l’ouvrage peut le supporter, il préférera généralement le service des esclaves à celui des hommes libres. Les plantations en sucre et en tabac sont en état de supporter la dépense d’une culture faite par des mains esclaves. Il paraît que la culture du blé ne pourrait aujourd’hui suppor­ter cette dépense. Dans les colonies anglaises dont le blé fait le principal produit, la très-majeure partie se fait par des hommes libres. La résolution prise dernièrement par les quakers de Pennsylvanie, de mettre en liberté tous leurs nègres esclaves, nous prouve assez que le nombre n’en était pas bien grand. S’ils y avaient fait une partie con­sidérable de la propriété, une pareille résolution n’aurait jamais passé. Dans nos colo­nies à sucre, au contraire, tout l’ouvrage se fait par des esclaves, et une très-grande partie du travail se fait de la même manière dans celles à tabac. Les profits d’une sucrerie, dans toutes nos colonies des Indes occidentales, sont, en général, beau­coup plus forts que ceux de toute autre espèce de culture que l’on connaisse en Europe ou en Amérique ; et les profits d’une plantation de tabac, quoique inférieurs à ceux d’une sucrerie, sont, comme on l’a déjà observé, supérieurs à ceux du blé. Les uns et les autres peuvent supporter la dépense d’une culture faite par des mains esclaves, mais les sucreries sont encore plus en état de la supporter que les plantations de tabac. Aussi, dans nos colonies à sucre, le nombre des nègres est-il beaucoup plus grand en proportion de celui des Blancs qu’il ne l’est dans nos colonies à tabac.
Aux cultivateurs serfs des anciens temps succéda par degrés une espèce de fermiers, connus à présent en France sous le nom de métayers. On les nommait en latin coloni partiarii. Il y a si longtemps qu’ils sont hors d’usage en Angleterre, que je ne connais pas à présent le mot anglais qui les désigne. Le propriétaire leur fournissait la semence, les bestiaux et les instruments de labourage : en un mot, tout le capital nécessaire pour pouvoir cultiver la ferme. Le produit se partageait par égales portions entre le propriétaire et le fermier, après qu’on en avait prélevé ce qui était nécessaire à l’entretien de ce capital, qui était rendu au propriétaire quand le fermier quittait la métairie ou en était renvoyé.
Une terre exploitée par de pareils tenanciers est, à bien dire, cultivée aux frais du propriétaire, tout comme celle qu’exploitent des esclaves. Il y a cependant entre ces deux espèces de cultivateurs une différence fort essentielle. Ces tenanciers, étant des hommes libres, sont capables d’acquérir des propriétés ; et ayant une certaine portion du produit de la terre, ils ont un intérêt sensible à ce que la totalité du produit s’élève le plus possible, afin de grossir la portion qui leur revient. Un esclave, au contraire, qui ne peut rien gagner que sa subsistance, ne cherche que sa commodité, et fait pro­duire à la terre le moins possible au-delà de cette subsistance.
Si la tenue en servage vint par degrés à se détruire dans la majeure partie de l’Europe, il est vraisemblable que ce fut en partie à cause de la mauvaise culture qui en résultait, et en partie parce que les serfs, encouragés à cet égard par le souverain, qui était jaloux des grands seigneurs, empiétèrent successivement sur l’autorité de leurs maîtres, jusqu’au point d’avoir rendu à la fin, à ce qu’il semble, cette espèce de servitude tout à fait incommode. Toutefois, le temps et la manière dont s’opéra cette importante révolution, sont deux points des plus obscurs de l’histoire moderne. L’Église de Rome réclame l’honneur d’y avoir beaucoup contribué, et il est constant que, dès le douzième siècle, Alexandre III publia une bulle pour l’affranchissement général des esclaves. Il semble cependant que ce fut plutôt une pieuse exhortation aux fidèles, qu’une loi qui entraînât de leur part une rigoureuse obéissance. La servitude n’en subsista pas moins presque partout pendant encore plusieurs siècles, jusqu’à ce qu’enfin elle fut successivement abolie par l’effet combiné des deux intérêts dont nous avons parlé, celui du propriétaire, d’une part, et celui du souverain, de l’autre. Un serf affranchi auquel on permettait de rester en possession de la terre qu’il cultivait, n’ayant pas de capital en propre, ne pouvait exploiter que par le moyen de celui que le propriétaire lui avançait et, par conséquent, il dut être ce qu’on appelle en France un métayer.
Cependant il ne pouvait pas être de l’intérêt même de cette dernière espèce de cultivateurs de consacrer à des améliorations ultérieures aucune partie du petit capital qu’ils pouvaient épargner sur leur part du produit, parce que le seigneur, sans y rien placer de son côté, aurait également gagné sa moitié dans ce surcroît de produit. La dîme, qui n’est pourtant qu’un dixième du produit, est regardée comme un très-grand obstacle à l’amélioration de la culture ; par conséquent, un impôt qui s’élevait à la moitié devait y mettre une barrière absolue. Ce pouvait bien être l’intérêt du métayer de faire produire à la terre autant qu’elle pouvait rendre, avec le capital fourni par le propriétaire ; mais ce ne pouvait jamais être son intérêt d’y mêler quelque chose du sien propre. En France, où l’on dit qu’il y a cinq parties sur six, dans la totalité du royaume, qui sont encore exploitées par ce genre de cultivateurs, les propriétaires se plaignent que leurs métayers saisissent toutes les occasions d’employer leurs bestiaux de labour à faire des charrois plutôt qu’à la culture, parce que, dans le premier cas, tout le profit qu’ils font est pour eux, et que, dans l’autre, ils le font de moitié avec leur propriétaire. Cette espèce de tenanciers subsiste encore dans quelques endroits de l’Écosse ; on les appelle Tenanciers à l’arc-de-fer[4]. Ces anciens tenanciers anglais, qui, selon le baron Gilbert et le docteur Blackstone, doivent plutôt être regardés comme les baillis[5] du propriétaire, que comme des fermiers proprement dits, étaient vraisemblablement des tenanciers de la même espèce.
À cette espèce de tenanciers succédèrent, quoique lentement et par degrés, les fermiers proprement dits, qui firent valoir la terre avec leur propre capital, en payant au propriétaire une rente fixe. Quand ces fermiers ont un bail pour un certain nombre d’années, ils peuvent quelquefois trouver leur intérêt à placer une partie de leur capital en améliorations nouvelles sur la ferme, parce qu’ils peuvent espérer de regagner cette avance, avec un bon profit, avant l’expiration du bail. Cependant, la possession de ces fermiers fut elle-même pendant longtemps extrêmement précaire, et elle l’est encore dans plusieurs endroits de l’Europe. Ils pouvaient être légalement évincés de leur bail, avant l’expiration du terme, par un nouvel acquéreur, et en Angleterre même, par ce genre d’action simulée qu’on nomme action de commun recouvrement[6]. S’ils étaient expulsés illégalement et violemment par leur maître, ils n’avaient, pour la réparation de cette injure, qu’une action très-imparfaite. Elle ne leur faisait pas toujours obtenir d’être réintégrés dans la possession de la terre, mais on leur accordait seulement des dommages-intérêts, qui ne s’élevaient jamais au niveau de leur perte réelle. En Angleterre même, le pays peut-être de l’Europe où l’on a toujours eu le plus d’égards pour la classe des paysans[7], ce ne fut qu’environ dans la quatorzième année du règne de Henri VII qu’on imagina l’action d’expulsion, par laquelle le tenancier obtient non-seulement des dommages, mais recouvre même la possession, et au moyen de laquelle il n’est pas nécessairement déchu de son droit par la décision incertaine d’une seule assise[8]. Ce genre d’action a même été regardé comme tellement efficace, que, dans la pratique moderne, quand le propriétaire est obligé d’intenter action pour la possession de sa terre, il est rare qu’il fasse usage des actions qui lui appartiennent proprement comme propriétaire, telles que le writ[9] de droit ou le writ d’entrée, mais il poursuit, au nom de son tenancier, par le writ d’expulsion. Ainsi, en Angleterre, la sûreté du fermier est égale à celle du propriétaire. D’ailleurs, en Angleterre, un bail à vie de la valeur de 40 schellings de rente annuelle est réputé franche-tenure[10], et donne au preneur du bail le droit de voter pour l’élection d’un membre du Parlement ; et comme il y a une grande partie de la classe des paysans qui a des franches-tenures de cette espèce, la classe entière se trouve traitée avec égard par les propriétaires, par rapport à la considération politique que ce droit lui donne. Je ne crois pas qu’on trouve en Europe, ailleurs qu’en Angleterre, l’exemple d’un tenancier bâtissant sur une terre dont il n’a point de bail, dans la confiance que l’honneur du propriétaire l’empê­chera de se prévaloir d’une amélioration aussi importante. Ces lois et ces coutumes, si favorables à la classe des paysans, ont peut-être plus contribué à la grandeur actuelle de l’Angleterre, que ces règlements de commerce tant prônés, à les prendre même tous ensemble.
La loi qui assure les baux les plus longs et les maintient contre quelque espèce de successeur que ce soit, est, autant que je puis savoir, particulière à la Grande-Bretagne. Elle fut introduite en Écosse, dès l’année 1449, par une loi de Jacques II[11]. Cependant, les substitutions ont beaucoup nui a l’influence salutaire que cette loi eût pu avoir, les grevés de substitution étant en général incapables de faire des baux pour un long terme d’années, souvent même pour plus d’un an. Un acte du Parlement a dernièrement relâché tant soit peu leurs liens à cet égard, mais il subsiste encore trop de gêne[12]. D’ailleurs, en Écosse, comme aucune tenure à bail ne donne de vote pour élire un membre du parlement, la classe des ’’paysans’’ est, sous ce rapport, moins considérée par les propriétaires qu’elle ne l’est en Angleterre[13]. 
Dans les autres endroits de l’Europe, quoiqu’on ait trouvé convenable d’assurer les tenanciers contre les héritiers et les nouveaux acquéreurs, le terme de leur sûreté resta toujours borné à une période fort courte : en France, par exemple, il fut borné à neuf ans, à compter du commencement du bail. À la vérité, il a été dernièrement étendu, dans ce pays, jusqu’à vingt-sept ans, période encore trop courte pour encourager un fermier à faire les améliorations les plus importantes. Les propriétaires des terres étaient anciennement les législateurs dans chaque coin de l’Europe. Aussi, les lois relatives aux biens-fonds furent toutes calculées sur ce qu’ils supposaient être l’intérêt du propriétaire. Ce fut pour son intérêt qu’on imagina qu’un bail passé par un de ses prédécesseurs ne devait pas l’empêcher, pendant un long terme d’années, de jouir de la pleine valeur de sa terre. L’avarice et l’injustice voient toujours mal, et elles ne prévi­rent pas combien un tel règlement mettrait d’obstacles à l’amélioration de la terre, et par là nuirait, à la longue, au véritable intérêt du propriétaire.
De plus, les fermiers, outre le payement du fermage, étaient censés obligés, envers leur propriétaire, à une multitude de services qui étaient rarement ou spécifiés par le bail, ou déterminés par quelque règle précise, mais qui l’étaient seulement par l’usage et la coutume du manoir ou de la baronnie. Ces services, étant presque entièrement arbitraires, exposaient le fermier à une foule de vexations. En Écosse, le sort de la classe des paysans s’est fort amélioré dans l’espace de quelques années, au moyen de l’abolition de tous les services qui ne seraient pas expressément stipulés par le bail.
Les services publics auxquels les paysans étaient assujettis n’étaient pas moins arbitraires que ces services privés. Les corvées pour la confection et l’entretien des grandes routes, servitude qui subsiste encore, je crois, partout, avec des degrés d’oppression différents dans les différents pays, n’étaient pas la seule qu’ils eussent à supporter. Quand les troupes du roi, quand sa maison ou ses officiers venaient à passer dans quelques campagnes, les paysans étaient tenus de les fournir de chevaux, de voitures et de vivres, au prix que fixait le pourvoyeur. La Grande-Bretagne est, je crois, la seule monarchie de l’Europe où ce dernier genre d’oppression a été totalement aboli. Il subsiste encore en France et en Allemagne[14].
Il n’y avait pas moins d’arbitraire et d’oppression dans les impôts auxquels ils étaient assujettis. Quoique les anciens seigneurs fussent très-peu disposés à donner eux-mêmes à leur souverain des aides en argent, ils lui accordaient facilement la faculté de tailler, comme ils l’appelaient, leur tenancier, et ils n’avaient pas assez de connaissance pour sentir combien leur revenu personnel devait s’en trouver affecté en définitive. La taille, telle qu’elle subsiste encore en France, peut donner l’idée de cette ancienne manière de tailler. C’est un impôt sur les profits présumés du fermier, qui s’évaluent d’après le capital qu’il a sur sa ferme. L’intérêt de celui-ci est donc de paraître en avoir le moins possible et, par conséquent, d’en employer aussi peu que possible à la culture, et point du tout en améliorations. Si un fermier français peut jamais venir à accumuler un capital, la taille équivaut presque à une prohibition d’en faire jamais emploi sur la terre. De plus, cet impôt est réputé déshonorant pour celui qui y est sujet, et est censé le mettre au-dessous du rang, non-seulement d’un gentil­homme, mais même d’un bourgeois ; et tout homme qui afferme les terres d’autrui y devient sujet. Il n’y a pas de gentilhomme ni même de bourgeois possédant un capital, qui veuille se soumettre à cette dégradation. Ainsi, non-seulement cet impôt empêche que le capital qu’on gagne sur la terre ne soit jamais employé à la bonifier, mais même il détourne de cet emploi tout autre capital. Les anciennes dîmes et quinzièmes, si fort en usage autrefois en Angleterre, en tant qu’elles portaient sur la terre, étaient, à ce qu’il semble, des impôts de la même nature que la taille.
On devait s’attendre à bien peu d’améliorations de la part des tenanciers décou­ragés de tant de manières. Cette classe de gens ne peut jamais en faire qu’avec de grands désavantages, quelque liberté et quelque sûreté que la loi puisse lui donner. Le fermier est, à l’égard du propriétaire, ce qu’est un marchand qui commerce avec des fonds d’emprunt, à l’égard de celui qui commerce avec ses propres fonds. Le capital de chacun de ces deux marchands peut bien se grossir ; mais à égalité de prudence dans leur conduite, le capital de l’un grossira toujours beaucoup plus lentement que celui de l’autre, à cause de la grande part de profits qui se trouve emportée par l’intérêt du prêt. De même, à égalité de soins et de prudence, les terres cultivées par un fermier s’amélioreront nécessairement avec plus de lenteur que celles qui sont cultivées par les mains du propriétaire, par rapport à la grosse part du produit qu’emporte le fermage, et que le fermier aurait employée en autant d’améliorations nouvelles, s’il eût été propriétaire. D’ailleurs l’état d’un fermier est, par la nature des choses, au-dessus du propriétaire. Dans la majeure partie de l’Europe, on regarde les paysans comme une classe inférieure même à un bon artisan, et dans toute l’Europe ils sont au-dessous des gros marchands et des maîtres manufacturiers. Il ne peut donc guère arriver qu’un homme, maître d’un capital un peu considérable, aille quitter son état, pour se mettre dans un état inférieur. Par conséquent, même dans l’état actuel de l’Europe, il est probable qu’il n’y aura que très-peu de capital qui aille, des autres professions, à celle de faire valoir des terres comme fermier. Il y en va peut-être plus dans la Grande-Bretagne que dans tout autre pays, quoique là même les grands capitaux qui sont en quelques endroits employés par des fermiers aient été gagnés en général à ce genre de métier, celui de tous peut-être où un capital se gagne le plus lentement. Cependant, après les petits propriétaires, les gros et riches fermiers sont, en tout pays, ceux qui apportent le plus d’améliorations aux terres. C’est ce qu’ils font peut-être plus encore en Angleterre qu’en aucune autre monarchie de l’Europe. Dans les gouvernemens républicains de la Hollande et du canton de Berne, les fermiers, dit-on, ne le cèdent en rien à ceux d’Angleterre.
Mais par-dessus tout, ce qui contribua à décourager la culture et l’amélioration des terres, dans la police administrative de l’Europe, que les terres fussent entre les mains des fermiers ou dans celles d’un propriétaire, ce fut, premièrement, la prohibition générale d’exporter des grains sans une permission spéciale, ce qui paraît avoir été un règlement très-universellement reçu ; et secondement, les entraves qui furent mises au commerce intérieur, non-seulement du blé, mais de presque toutes les autres parties du produit de la ferme, au moyen de ces lois absurdes contre les accapareurs, regrat­tiers et intercepteurs[15], et par les privilèges des foires et marchés. On a déjà observé comment la prohibition de l’exportation des blés, jointe à quelque encouragement donné à l’importation des blés étrangers, arrêta les progrès de la culture dans l’ancien­ne Italie, le pays naturellement le plus fertile de l’Europe, et alors le siège du plus grand empire du monde. Il n’est peut-être pas aisé de s’imaginer jusqu’à quel point de telles entraves sur le commerce intérieur de cette denrée, jointes à la prohibition géné­rale de l’exportation, doivent avoir découragé la culture dans des pays moins fertiles et qui se trouvaient moins favorisés par les circonstances.
 
 
 
↑ Ces sortes de dispositions, dont l’objet est d’établir un ordre de succession différent de l’ordre ordinaire, se nomment en anglais entail, mais en français elles se nomment substitutions, quoique très-différentes de ce que la loi civile appelle de ce nom, comme on le fera observer plus loin.
↑ Le commentateur Mac Culloch annonce ici qu’il ose être en désaccord avec l’opinion de Smith au sujet de la loi de primogéniture, ou de la coutume qui lègue tout ou la plus grande partie d’une propriété appartenant à une famille, au fils aîné. Il est pour le droit d’aînesse, favorable à la grande propriété, et il déduit ses raisons dans un très-long commentaire, où il déclare que l’expérience faite en France de l’égalité des partages est une expérience funeste à l’agriculture. Il conclut des inconvénients de la propriété morcelée à l’immobilisation forcée et légale de la propriété, à la perpétuité des instruments du travail agricole dans un petit nombre de familles. En Angleterre l’opinion de Mac Culloch est encore l’opinion générale, et Adam Smith, partisan de l’égalité des partages, ennemi des substitutions, n’a point fait de conversion. Dans ce pays la science n’a d’autorité qu’autant qu’elle favorise et défend les intérêts prédominants. Nous n’avons donc pas cru devoir citer en entier le commentaire aristocratique de Mac Culloch. A. B.
↑ Dans le moyeu âge, et tant que l’esclavage subsista sous un gouvernement féodal, l’agriculture était partout languissante. Les nobles, propriétaires des terres, avançaient à leurs esclaves le chétif capital qui faisait aller leur culture, et tout le produit de la terre leur appartenait, soit comme rente, soit comme intérêt, soit enfin comme loyer de leurs esclaves. Dans l’état actuel de l’Europe, où la culture des terres se fait par des fermiers indépendants, le propriétaire ne fait aucune avance, il ne reçoit que la rente foncière, et cette rente ne va guère au delà du tiers de la totalité du produit, quelquefois pas au quart. Néanmoins ce tiers ou quart du produit annuel est trois ou quatre fois plus grand que n’était auparavant le total, à cause de l’amélioration des terres et de la culture, suite de l’augmentation des capitaux et de l’industrie, qui eux-mêmes sont une suite de la liberté et de la propriété dont jouit le cultivateur. À mesure des progrès que fait l’amélioration des terres, la rente diminue bien dans sa proportion avec le produit, mais elle augmente relativement à l’étendue de la terre.
↑ Steel-bow tenants. Ce nom vient probablement de la manière dont ils étaient autrefois armés en guerre.
↑ Espèce d’officiers subalternes de justice, comme étaient nos sergents de justice seigneuriale, mais qui de plus faisaient pour le seigneur la collecte des cens, lots, amendes et autres droits, tant fixes que casuels, de la seigneurie.
↑ Action of common recovery, espèce d’action ou île procédure fictive ou concertée pour se faire adjuger par jugement un bien-fonds, et le posséder ainsi libre et purgé de substitutions, reversions et autres droits réels dont il était grevé. C’est ainsi qu’en France, pour purger les hypothèques, on avait imaginé une procédure simulée, qu’on nommait décret volontaire.
↑ Ce mot de paysans, qu’il a fallu employer faute d’autre, désigne ici principalement cette classe qu’on nomme en anglais yeomen, et qui a rang immédiatement après celle de gentlemen, et avant celle des gens de métier ou artisans, tradesmen. Cette classe des yeomen comprend les laboureurs, fermiers, nourrisseurs, etc., et autres ouvriers de la campagne, qui, travaillant manuellement, sont, par cette raison, hors de la classe des gentlemen, mais qui, possédant en pleine propriété, ou au moins a vie, un bien-fonds de 40 schellings de rente au moins, ont droit de concourir à l’élection des représentants de leur comté, d’être nommés jurés, etc.
↑ On entend par ce mol l’ensemble des séances employées, par les juges d’assise et les jurés, à l’examen, instruction et décision d’un procès.
↑ En Angleterre, les actions ne s’intentent qu’en vertu de lettres ou commissions, et chaque action a sa formule particulière. C’est ce qu’on nomme writ, et ce qui ressemble en quelque sorte à nos lettres ou commissions de chancelleries.
↑ Free-hold, c’est-à-dire une possession qui a tous les caractères et les droits d’une pleine propriété.
↑ Voici une copie de l’acte de 1449, chap. xviii, qui a été justement appelé la Grande charte des agriculteurs d’Écosse.
« Item il est ordonné, pour la sûreté et l’avantage du pauvre peuple qui cultive la terre, que ceux et tous autres qui auront pris ou prendront à l’avenir de la terre des mains des seigneurs, et qui auront des termes et baux, dans le cas où les seigneurs vendraient ou aliéneraient cette terre ou terres, ceux-là, les preneurs, garderont leurs baux jusqu’à la fin de leurs termes, en quelque main que la terre passe, pour la même rente qu’ils l’avaient reçue. » Mac Culloch.
↑ Le statut auquel le Dr Smith fait allusion est celui de la dixième année de Georges III, chap. ii. Cet acte permet au possesseur d’un bien grevé de substitution d’accorder des baux pour un nombre quelconque d’années, n’excédant pas trente et un ans, ou pour quatorze ans et une vie existante, ou pour deux vies existantes, pourvu que dans les baux pour deux vies le fermier soit tenu d’exécuter certaines améliorations spécifiées dans l’acte. On permet aussi les baux de quatre-vingt-dix-neuf ans, à condition de bâtir.
Mac Culloch.
↑ « Si le Dr Smith avait dit que la privation de la franchise électorale rendait les paysans d’Écosse moins utiles, au lieu de moins respectables, à leurs propriétaires, il n’y aurait rien à reprendre à son observation. L’acte de la réforme accorde le droit de voter à tout tenancier affermant une terre de 50 livres sterling par an, aussi bien en Écosse qu’en Angleterre, et personne connaissant l’état de l’Écosse avant et depuis la réforme, n’osera dire qu’elle a rendu les fermiers plus respectables aux yeux de leurs propriétaires. Il est certain qu’elle a eu des effets tout contraires ; et, quelles que soient ses conséquences sous d’autres rapports, elle a déjà exercé et continuera d’exercer, il y a tout lieu de le penser, une influence pernicieuse sur les intérêts des fermiers et de l’agriculture. Autrefois les propriétaires d’Écosse s’inquiétaient rarement des opinions politiques de leurs tenanciers, et pourvu qu’ils payassent leur terme et gouvernassent leurs terres conformément aux stipulations de leurs baux, ils pouvaient être du parti politique et religieux qui leur plaisait. Il en est devenu tout autrement depuis. Les propriétaires désireux, comme tout le monde, d’étendre leur influence politique, veulent contrôler et même commander les suffrages de leurs tenanciers, et multiplier sur leurs domaines les électeurs dépendants. Pour obtenir ces résultats, ils n’ont pas scrupule, en beaucoup de cas, d’employer un système d’intimidation et de prendre des mesures vindicatives contre les tenanciers qui ont voté contrairement à leurs vœux. Mais cet inconvénient, quoique le plus sensible aujourd’hui, est encore le moindre des maux qui résulte du nouvel état de choses. Il a déjà conduit, en plusieurs cas, à changer le mode suivant lequel on affermait les terres jusque-là, et il y a bien des raisons de craindre qu’il ne fasse disparaître à la fin le système d’accorder des baux de dix-neuf et vingt ans assurés, système qui a été l’unique cause des merveilleux progrès de l’agriculture en Écosse. Dans un grand nombre de cas, il a aussi amené la subdivision des fermes, dans le seul but de créer des électeurs. Et quoique faite avec de bonnes intentions, l’extension du suffrage aux tenanciers est un des coups les plus funestes qui aient été portés à leur indépendance et à la prospérité de l’agriculture. Les tenanciers, comme tels, sont la dernière classe de citoyens auxquels la franchise électorale devrait être accordée. La plupart d’entre eux sont engagés envers leurs propriétaires et dépendent absolument d’eux ; et le petit nombre de ceux qui sont indépendants ne le sont que parce qu’ils ont acquis de la propriété, et auraient possédé cette franchise, si on l’avait accordée, comme on aurait dû le faire, à ceux-là seulement qui possédaient une certaine quantité de propriété indépendante. Si le meilleur système électoral est celui qui amène au scrutin le plus grand nombre d’électeurs indépendants, et en éloigne le plus grand nombre de ceux qui sont dépendants, l’extension de la franchise aux tenanciers et occupants de terres seigneuriales doit être le système le pire de tous, car de toutes les classes de la société celle-là est la plus dépendante, celle qui est le plus à la merci d’autrui. Mac Culloch.
↑ Il n’est pas besoin de dire que tous ces privilèges féodaux ont été abolis en France par la révolution. A. B.
↑ Ce délit de police, que les lois anglaises nomment forestalling, est distingué de celui qui se nomme engrossing, accaparer. Intercepter les denrées, c’est les attendre sur la route pour les acheter avant qu’elles arrivent au marché.

%%%%%%%%%%%%%%%%%%%%%%%%%%%%%%%%%%%%%%%%%%%%%%%%%%%%%%%%%%%%%%%%%%%%%%%%%%%%%%%%
%                                  Chapitre 3                                  %
%%%%%%%%%%%%%%%%%%%%%%%%%%%%%%%%%%%%%%%%%%%%%%%%%%%%%%%%%%%%%%%%%%%%%%%%%%%%%%%%

\chapter{Comment les villes se formèrent et s’agrandirent après la chute de l’Empire romain}
\markboth{Comment les villes se formèrent et s’agrandirent après la chute de l’Empire romain}{}

artisans et gens de métier qui étaient alors, à ce qu’il semble, de condition servile ou d’une condition qui en approchait beaucoup. Les privilèges que nous voyons, dans les anciennes chartes, accorder aux habitants de quelques-unes des principales villes d’Eu­ro­pe, suffisent pour nous faire voir ce qu’ils étaient avant ces concessions. Des hom­mes auxquels on accorde, comme un privilège, de pouvoir marier leurs filles sans le consentement de leur seigneur, d’avoir pour héritiers à leur mort leurs enfants et non leur seigneur, et de pouvoir disposer de leurs effets par testament, ont dû être tout à fait, ou très-peu s’en faut, dans le même état de servitude que les cultivateurs de la terre dans les campagnes.
Il paraît, en effet, que c’était une très-pauvre et très-basse classe de gens, qui avaient coutume de voyager de place en place et de foire en foire avec leurs marchan­dises, comme nos porteurs de balles d’aujourd’hui. On avait alors, dans tous les diffé­rents pays de l’Europe, la coutume qui se pratique à présent dans plusieurs gouverne­ments tartares de l’Asie, celle de lever des taxes sur les personnes et les effets des voyageurs, quand ils traversaient certains domaines, quand ils passaient sur certains ponts, quand ils portaient leurs marchandises aux foires de place en place, et quand ils y dressaient une loge ou un étai pour les vendre. Ces différentes taxes furent connues en Angleter­re sous les noms de péage, pontonage, lestage[1] et étalage. Quelquefois le roi, et quel­que­fois un grand seigneur qui avait, à ce qu’il semble, droit d’agir ainsi en certaines circonstances, accordait à quelques marchands particuliers, et principale­ment à ceux qui résidaient dans ses domaines, une exemption générale de toutes ces taxes. Ces marchands, quoique à tous autres égards de condition servile ou à peu près ser­vile, étaient, sous ce rapport, appelés francs marchands. En retour, ils payaient or­di­­nai­re­ment à leur protecteur une espèce de capitation[2] annuelle. Dans ces temps-là, la pro­tect­ion ne s’accordait guère que pour une composition assez forte, et on pourrait peut-être regarder cette capitation comme une indemnité de ce que leur exemption des autres taxes pouvait faire perdre à leurs patrons. Il paraît que ces exemptions et ces capitations furent d’abord absolument personnelles, et ne regardaient que quelques particuliers qui jouissaient de ce privilège, ou durant leur vie, ou à la volonté de leurs protecteurs. Dans les extraits fort imparfaits qui ont été publiés du Grand Cadastre[3], à l’article de plusieurs villes d’Angleterre, il est souvent fait mention, tantôt de la taxe que certains bourgeois payaient chacun au roi ou à quelque autre grand seigneur pour cette sorte de protection, et tantôt seulement du montant de toutes ces taxes en somme totale[4].
Mais, quelque servile que puisse avoir été dans l’origine la condition des habitants des villes, il paraît évidemment qu’ils arrivèrent à un état libre et indépendant beau­coup plus tôt que les cultivateurs des campagnes. Ce fut un usage commun de bailler à ferme, pour un certain nombre d’années, moyennant une rente fixe, tantôt au shérif[5] du comte, tantôt à d’autres personnes, cette portion des revenus du roi, provenant de ces capitations, dans une ville particulière. Les bourgeois eux-mêmes eurent souvent assez de crédit pour être admis à affermer les revenus de cette espèce qui se levaient dans leur ville, en se rendant conjointement et solidairement responsables de la tota­lité de la rente[6]. Il était, à ce que je crois, très-conforme à l’ordre pratiqué ordinaire­ment par tous les souverains de l’Europe, dans l’économie de leurs revenus, d’affermer de cette manière. Ils avaient souvent coutume de louer la totalité de leurs terres en masse à tous les tenanciers de ces terres, lesquels devenaient conjointement et séparé­ment responsables pour la totalité de la rente, mais avaient en revanche la permission d’en faire la collecte comme ils jugeaient à propos et de la payer dans l’échiquier du roi par les mains de leur propre bailli[7], et par là étaient entièrement affranchis des in­so­lences des officiers royaux, circonstance qui était alors comptée pour beaucoup.
Au commencement, la ferme de la ville fut vraisemblablement affermée aux bour­geois de la même manière qu’elle l’avait été aux autres fermiers, pour un certain nombre d’années seulement. Cependant, par la suite des temps, il paraît que la pra­tique générale fut de la leur donner à cens, c’est-à-dire pour toujours, moyennant la réser­ve d’une rente fixe qui ne pouvait plus être augmentée. Le payement ayant été ainsi rendu perpétuel, les exemptions qui en étaient l’objet devinrent aussi naturelle­ment perpétuelles. Ces exemptions cessèrent donc d’être personnelles et ne purent plus ensuite être censées appartenir à des individus, comme individus, mais comme bourgeois d’un bourg particulier, qui fut appelé pour cela bourg franc, par la même raison que les individus avaient été nommés francs marchands ou francs bourgeois.
Les bourgeois de la ville à laquelle cette franchise fut accordée eurent aussi, géné­ralement, en même temps les privilèges importants dont nous avons parlé plus haut, c’est-à-dire de pouvoir marier leurs filles hors de l’endroit, de transmettre leur suc­cession à leurs enfants et de disposer de leurs biens par testament. Ce que je ne sais pas, c’est si ces privilèges avaient été habituellement accordés en même temps que la franchise du commerce aux bourgeois individuellement. Je le regarde comme assez probable, quoique je ne puisse en produire aucun témoignage direct ; mais, quoi qu’il en puisse être, les principaux caractères de la servitude et du villenage leur ayant été ainsi ôtés, ils devinrent au moins alors véritablement libres, dans le sens qu’on attache au mot d’hommes libres.
Ce ne fut pas tout ; ils furent en général, dans le même temps, érigés en commu­nautés ou corporations, avec le privilège d’avoir leurs magistrats et leur propre conseil de ville, de faire des statuts pour leur régime intérieur, de construire des murs pour leur propre défense, et de ranger tous leurs habitants sous une espèce de discipline mi­li­taire, en les obligeant de faire le guet ou la garde, c’est-à-dire, suivant l’ancienne signi­fication, de garder et de défendre leurs murs contre toutes les attaques et surpri­ses de nuit comme de jour. En Angleterre, ils furent généralement affranchis de la juridiction du comte et du centenier[8], et toutes les causes qui pouvaient s’élever entre eux, excepté celles qui intéressaient la couronne, étaient laissées à la décision de leurs propres magistrats. Dans d’autres pays, on leur accorda souvent des droits de justice plus considérables et plus étendus[9].
Il était vraisemblablement indispensable d’accorder aux villes auxquelles on avait permis de prendre à ferme leurs propres revenus quelque espèce de juridiction coer­ci­tive pour obliger leurs citoyens au payement de leur contribution. Dans ces temps de troubles, il aurait pu leur être extrêmement incommode d’être réduites à aller chercher justice vers tout autre tribunal. Mais ce qui doit paraître vraiment extraordi­naire, c’est que tous les souverains des différents pays de l’Europe aient ainsi échangé, contre une rente fixe qui n’était plus susceptible d’augmentation, la branche de leurs revenus qui, de toutes, était peut-être la plus susceptible d’augmentation par le cours naturel des choses, sans qu’ils eussent à y mettre ni soins ni dépenses et que, d’ail­leurs, ils aient ainsi, de leur propre volonté, érigé dans le cœur de leurs États des espèces de répu­bliques indépendantes.
Pour expliquer ceci, il faut se rappeler que dans ces temps-là il n’y avait peut-être pas un seul souverain en Europe qui fût en état de protéger, dans toute l’étendue de ses États, la partie la plus faible de ses sujets contre l’oppression des grands seigneurs. Ceux que la loi ne pouvait pas protéger, et qui n’étaient pas assez forts pour se défendre eux-mêmes, furent obligés, ou de recourir à la protection de quelque grand seigneur, et de devenir, pour l’obtenir, ses esclaves ou ses vassaux, ou bien d’entrer dans une ligue de défense mutuelle pour la protection commune. Les habitants des villes et des bourgs, considérés individuellement, n’avaient pas le pouvoir de se dé­fen­dre ; mais en se liguant avec leurs voisins pour une défense mutuelle, ils furent en état de faire une résistance passable. Les seigneurs méprisaient les bourgeois, qu’ils regardaient non-seulement comme une classe fort inférieure, mais comme un ramas d’esclaves émancipés, presque d’une autre espèce qu’eux. La richesse des bourgeois ne manqua pas d’exciter leur colère et leur envie, et ils les pillaient sans pitié et sans remords à toutes les occasions qui s’en présentaient. Naturellement les bourgeois durent haïr et craindre les seigneurs ; le roi les haïssait et les craignait aussi. Quant aux bourgeois, il pouvait bien les mépriser, mais il n’avait pas sujet de les haïr ni de les craindre. Ce fut donc l’intérêt mutuel qui disposa ceux-ci à soutenir le roi, et le roi à les soutenir contre les seigneurs. Ces bourgeois étaient les ennemis de ses ennemis, et son intérêt était d’assurer, autant que possible, leur indépendance à l’égard de ces derniers. En leur accordant des magistrats particuliers, le privilège de faire des statuts pour leur régime intérieur, celui de construire des murs pour leur défense et de ranger tous leurs concitoyens sous une espèce de discipline militaire, il leur donnait contre les barons tous les moyens de sûreté et d’indépendance qu’il était en son pouvoir de leur donner. Sans l’établissement d’un gouvernement régulier de cette espèce, sans une autorité efficace qui pût faire agir tous les habitants d’après un plan ou un système uni­forme, toutes les ligues qu’ils eussent pu volontairement former pour leur défense commune ne leur auraient jamais procuré de sûreté durable, et n’auraient pu les mettre en état de prêter au roi un appui important. En leur accordant la ferme de leur ville, il voulut ôter à ceux dont il cherchait à se faire des amis et, pour ainsi dire, des alliés, tout sujet de crainte et de soupçon qu’il eût aucun dessein de les opprimer par la suite, soit en augmentant la rente de la ferme de leur ville, soit en la donnant à quelque autre fermier.
Les princes qui vécurent le plus mal avec leurs barons sont aussi, à ce qu’ils sem­ble, les plus remarquables par la libéralité de leurs concessions envers les bourgs. Le roi jean d’Angleterre, par exemple, paraît avoir été un des bienfaiteurs les plus généreux envers les villes[10]. Philippe Ier, roi de France, avait perdu toute autorité sur ses barons. Vers la fin de son règne, son fils Louis, connu ensuite sous le nom de Louis le Gros, se consulta, dit le père Daniel, avec les évêques de ses domaines, sur les moyens les plus propres à contenir les violences des grands seigneurs. Leur avis se réduisit à deux propositions. L’une fut d’ériger un nouvel ordre de juridiction, en établissant des magistrats et un conseil de ville dans chaque ville considérable de ses domaines ; et l’autre, de former une nouvelle milice, en rangeant les habitants de ces villes sous le commandement de leurs propres magistrats, pour marcher en toutes les occasions où il s’agirait de prêter assistance au roi. C’est de cette époque, suivant les historiens français, qu’on doit dater en France l’institution des officiers municipaux et des conseils de ville. Ce fut pendant les malheureux règnes des princes de la maison de Souabe que la plupart des villes libres d’Allemagne reçurent les premières con­cessions de leurs privilèges, et que la fameuse ligue hanséatique commença à devenir formidable[11].
La milice des villes, dans ces temps-là, n’était pas, à ce qu’il semble, inférieure à celle des campagnes ; et ayant l’avantage de pouvoir être plus promptement rassem­blée en cas de besoin urgent, il arriva souvent qu’elle eut le dessus dans ses querelles avec les seigneurs du voisinage. Dans les pays tels que l’Italie et la Suisse, dans lesquels, soit par rapport à leur distance du siège principal du gouvernement, soit par rapport à la force résultant de la situation naturelle du pays, ou par quelque autre rai­son, le souverain vint à perdre entièrement son autorité, les villes devinrent généra­le­ment des républiques indépendantes, et subjuguèrent toute la noblesse de leur voi­si­nage, obligeant les nobles à abattre leurs châteaux dans les campagnes, et à vivre dans la ville, comme les autres habitants paisibles. Telle est en résumé l’histoire de la répu­blique de Berne, aussi bien que celle de plusieurs autres villes de Suisse. Si vous en exceptez la ville de Venise, dont l’histoire est tant soit peu différente, c’est l’histoire de toutes les républiques considérables d’Italie, dont il s’éleva et périt un si grand nombre entre la fin du douzième siècle et le commencement du seizième.
Dans les pays tels que la France et l’Angleterre, où, quoique l’autorité du souve­rain fut souvent très-abaissée, elle ne fut pourtant jamais entièrement détruite, les villes n’eurent pas d’occasion de se rendre tout à fait indépendantes. Elles devinrent néanmoins assez considérables pour que le souverain ne fût plus maître d’imposer sur elles, sans leur consentement, aucune taxe au-delà du cens fixe de la ville. On les appela donc aux assemblées des états généraux du royaume, où elles envoyèrent des députés pour se joindre au clergé et à la noblesse, quand il était question, dans les cas urgents, d’accorder au roi des secours extraordinaires. De plus, comme elles étaient, en général, plus disposées à favoriser sa puissance, il paraît que le roi s’est quelque­fois servi de leurs députés pour contre-balancer l’autorité des grands seigneurs dans ces assemblées : de là l’origine de la représentation des communes dans les états généraux de toutes les grandes monarchies de l’Europe. 
C’est ainsi que l’ordre et la bonne administration, et avec eux la liberté et la sûreté des individus, s’établirent dans les villes, dans un temps où les cultivateurs des campagnes étaient toujours exposés à toutes les espèces de violences. Or, les hommes réduits à un tel état et qui se sentent privés de tout moyen de se défendre se contentent naturellement de la simple subsistance, parce que ce qu’ils pourraient gagner de plus ne servirait qu’à tenter la cupidité de leurs injustes oppresseurs. Quand ils sont, au contraire, assurés de jouir des fruits de leur industrie, naturellement ils s’efforcent d’améliorer leur sort et de se procurer, non-seulement les choses nécessaires, mais encore les aisances et les agréments de la vie. Par conséquent, cette industrie qui vise au-delà de l’absolu nécessaire se fixa dans les villes longtemps avant qu’elle pût être communément mise en pratique par les cultivateurs de la campagne. Si quelque petit capital venait à s’accumuler dans les mains d’un pauvre cultivateur écrasé sous le joug de la servitude du villenage, naturellement il devait mettre tous ses soins à le cacher aux yeux de son maître, qui autrement s’en serait emparé comme de sa propriété, et il devait saisir la première occasion de se retirer dans une ville. La loi était alors si favorable aux habitants des villes, et si jalouse de diminuer l’autorité des seigneurs sur l’habitant des campagnes, que s’il pouvait parvenir à se soustraire pendant une année aux poursuites de son seigneur, il était libre pour toujours. Par conséquent, tout capital accumulé dans les mains de la portion laborieuse des habitants de la campagne dut naturellement chercher un refuge dans les villes, comme le seul asile où il pût être assuré dans les mains qui l’avaient acquis.
Il est vrai que les habitants d’une ville doivent toujours, en définitive, tirer de la campagne leur subsistance et tous les moyens et matériaux de leur industrie. Mais ceux d’une ville située, ou proche des côtes de la mer, ou sur les bords d’une rivière navigable, ne sont pas nécessairement bornés à tirer ces choses de la campagne qui les avoisine. Ils ont un champ bien plus vaste, et peuvent les tirer des coins du monde les plus éloignés, soit en les prenant en échange du produit manufacturé de leur propre industrie, soit en faisant l’office de voituriers entre les pays éloignes l’un de l’autre, et échangeant respectivement les produits de ces pays. De cette manière, une ville pourrait s’élever à un grand degré d’opulence et de splendeur, pendant que non-seulement le pays de son voisinage, mais même tous ceux avec lesquels elle trafi­que­rait, seraient dans la pauvreté et le dénuement. Peut-être que chacun de ces pays, pris séparément, ne lui pourrait fournir qu’une extrêmement petite partie de la subsistance qu’elle consomme, ou des emplois qu’elle exerce ; mais tous ces pays, pris collecti­vement, lui pourront fournir une grande quantité de subsistances et une grande variété d’occupations. Dans la sphère étroite du commerce des anciens temps, on remarque encore néanmoins quelques pays qui furent riches et industrieux. Tel fut l’empire grec tant qu’il subsista, et celui des Sarrasins sous le règne des Abassides ; telles furent aussi l’Égypte jusqu’à la conquête des Turcs, quelques parties de la côte de Barbarie, et toutes ces provinces de l’Espagne qui ont été sous le gouvernement des Maures.
Les villes d’Italie paraissent avoir été les premières en Europe qui s’élevèrent, par le commerce, à quelque degré considérable d’opulence. L’Italie est située au centre de ce qui était alors la partie riche et civilisée du monde. D’ailleurs, les croisades, qui ont nécessairement retardé les progrès de la majeure partie de l’Europe, par l’immense dissipation de capitaux et la dépopulation qu’elles entraînèrent, furent extrêmement favorables à l’industrie de quelques villes de l’Italie. Ces grandes armées, qui mar­chaient de toutes parts à la conquête de la Terre-Sainte, donnèrent un encouragement extraordinaire à la marine de Venise, à celle de Gênes, à celle de Pise, quelquefois par le transport des hommes, et toujours par celui des vivres qu’il fallait leur fournir. Ces républiques furent, pour ainsi dire, les commissaires des vivres de ces armées, et la frénésie la plus ruineuse qui jamais ait aveuglé les peuples de l’Europe fut pour elles une sorte d’opulence.
Les habitants des villes commerçantes, en important des pays plus riches des ouvrages raffinés et des objets de luxe d’un grand prix, offrirent un aliment à la vanité des grands propriétaires, qui en achetèrent avec empressement, moyennant de grandes quantités du produit brut de leurs terres. Le commerce d’une grande partie de l’Euro­pe, à cette époque, consistait dans les échanges du produit brut du pays contre le pro­duit manufacturé d’un autre pays plus avancé en industrie. Ainsi, la laine d’Angleterre avait coutume de s’échanger contre les vins de France et les beaux draps de Flandre, de la même manière que le blé de Pologne s’échange aujourd’hui contre les vins et les eaux-de-vie de France, et contre les soieries et les velours de France et d’Italie.
C’est ainsi que le commerce étranger introduisit le goût des objets de manufacture plus recherchés et mieux finis, dans des pays où ce genre de travail n’était pas établi. Mais quand ce goût fut devenu assez général pour donner lieu à une demande consi­dérable, les marchands, pour épargner les frais de transport, tâchèrent naturellement d’établir, dans leur propre pays, des manufactures, dont les produits étaient destinés à être vendus au loin, qui paraissent s’être établies dans la partie occidentale de l’Euro­pe, après la chute de l’empire romain.
Il faut observer qu’un grand pays n’a jamais subsisté ni pu subsister, sans qu’il ait eu chez lui quelque espèce de manufacture ; et quand on dit d’un pays qu’il n’avait point de manufactures, cela doit toujours s’entendre des fabriques d’ouvrages finis et recherchés, ou de ceux qui sont destinés à être vendus au loin. En tout grand pays, les vêtements et ustensiles de ménage de la très-grande partie du peuple sont le produit de l’industrie nationale. C’est même ce qui arrive plus généralement dans ces pays pau­vres dont on dit ordinairement qu’ils n’ont point de manufactures, que dans ces pays riches où on dit qu’elles abondent. Dans ceux-ci vous trouverez, en général, tant dans le vêtement que dans les ustensiles de ménage des dernières classes du peuple, des objets de manufacture étrangère, en beaucoup plus grande quantité, en proportion, que vous n’en trouverez dans les autres.
Ces manufactures d’objets destinés à être vendus au loin paraissent s’être intro­duites en différents pays, de deux manières différentes.
Quelquefois, elles se sont introduites de la manière dont je viens de parler, par l’action violente, pour ainsi dire, des capitaux de quelques marchands et entrepreneurs particuliers qui les avaient établies à l’imitation de manufactures étrangères de la même espèce. Ainsi, ces manufactures durent leur naissance au commerce étranger, et telles ont été, à ce qu’il semble, les anciennes manufactures d’étoffes de soie, de ve­lours et de brocart qui fleurirent à Lucques dans le cours du treizième siècle. Elles furent bannies de cette ville par la tyrannie d’un des héros de Machiavel, Castruccio Castracani. En 1310, neuf cents familles furent chassées de Lucques ; trente et une d’elles se retirèrent à Venise, et offrirent d’y introduire l’industrie de la soie[12]. Leur offre fut acceptée ; on leur accorda plusieurs privilèges, et leur manufacture commen­ça avec trois cents ouvriers. Telles furent encore, à ce qu’il semble, les manufactures de draps fins qui fleurirent anciennement en Flandre, et qui s’introduisirent en Angleterre au commencement du règne d’Élisabeth, et telles sont aujourd’hui les fa­bri­ques d’étoffes de soie de Lyon et celles de Spital-Fields. Les manufactures qui s’introduisent de cette manière travaillent en général sur des matières premières tirées de l’étranger, puisqu’elles sont elles-mêmes une imitation de manufactures étrangères. Lors du premier établissement de la manufacture de Venise, elle tirait toutes ses matières de la Sicile et du Levant. La manufacture de Lucques, qui était plus ancienne, travaillait de même sur des matières premières venant de l’étranger. L’usage de cultiver les mûriers et d’élever les vers à soie ne paraît pas avoir été commun dans les pays du nord de l’Italie avant le seizième siècle. Ces arts ne furent introduits en France que sous le règne de Charles IX. Les manufactures de Flandre travaillaient principalement les laines d’Espagne et d’Angleterre. La laine d’Espagne a été la matière première, non de la première manufacture de lainage éta­blie en Angleterre, mais de la première dont les produits aient été destinés à se vendre au loin. Aujourd’hui, plus de la moitié des matières premières qu’on emploie aux fabriques de Lyon sont des soies étrangères ; à l’époque de l’établissement de ces fabri­ques, on n’en employait pas d’autre, ou presque point d’autre. Il est vraisemblable que, dans les matières premières manufacturées à Spital-Fields, il n’y en aura jamais une seule partie qui soit produite en Angleterre. Ces manufactures étant, en général, le résultat des spéculations d’un petit nombre de particuliers, le lieu de leur établis­se­ment est quelquefois une grande ville maritime, quelquefois une petite ville de l’inté­rieur des terres, suivant qu’il s’est trouvé que les fondateurs ont choisi l’une ou l’autre d’après leur intérêt, leur jugement ou leur caprice.
D’autres fois, des manufactures destinées à des marchés éloignés se sont élevées naturellement et, pour ainsi dire, d’elles-mêmes, par le perfectionnement successif de ces fabriques grossières et domestiques qui s’établissent toujours nécessairement dans tous les temps, même dans les pays les plus pauvres et les moins civilisés. Ces sortes de manufactures travaillent, en général, des matières produites dans le pays, et il paraît qu’elles se sont souvent perfectionnées d’abord dans des localités de l’intérieur des terres, qui, sans être à un très-grand éloignement des côtes de la mer, s’en trou­vaient placées à une distance assez considérable, et quelquefois même privées de tout moyen de transport par eau. Un pays enfoncé dans les terres, naturellement fertile et d’une culture aisée, pro­duira une grande quantité de vivres au-delà de ce qu’exige la subsistance des culti­vateurs, et l’énormité des frais de transports par terre, l’incommodité de la navigation des rivières, peuvent rendre souvent difficile l’exportation de ce surplus de produits. L’abondance y mettra donc les vivres à bon marché, et encourager un grand nombre d’ouvriers à s’établir dans le voisinage, où leur industrie leur permettra de satisfaire aux besoins et aux commodités de la vie, mieux que dans d’autres endroits. Ils tra­vaillent sur place les matières premières que produit le pays, et ils échangent leur ouvrage, ou, ce qui est la même chose, le prix de leur ouvrage contre une plus grande quantité de matières et de vivres. Ils donnent une nouvelle valeur au surplus de ce produit brut, en épargnant la dépense de le voiturer au bord de l’eau ou à quelque marché éloigné, et ils donnent à sa place en échange aux cultivateurs quelque chose qui leur est utile ou agréable, à de meilleures conditions que ceux-ci n’auraient pu se le procurer auparavant. Les cultivateurs trouvent un meilleur prix du surplus de leurs produits, et ils peuvent acheter à meilleur compte les choses commodes qui lui man­quent. Cet arrangement leur donne donc le désir et les moyens d’augmenter encore ce surplus de produit par de nouvelles améliorations et par une culture plus soignée de leurs terres ; et si la fertilité de la terre a donné naissance à la manufacture, à son tour la manufacture, en se développant, réagit sur la terre et augmente encore sa fertilité. Les ouvriers de la fabrique fournissent d’abord le voisinage, et ensuite, à mesure que leur ouvrage se perfectionne, ils fournissent des marchés plus éloignés ; car si le produit brut et même le produit manufacturé de fabrique grossière ne peuvent pas, sans de grandes difficultés, supporter les frais d’un transport par terre un peu long, des ouvrages perfectionnés peuvent les supporter aisément. Ils contiennent sou­vent, sous un très-petit volume, le prix d’une grande quantité de produit brut. Par exemple, une pièce de drap fin, qui ne pèse que quatre-vingts livres, renferme non-seulement le prix de quatre-vingts livres pesant de laine, mais quelquefois de plusieurs milliers pesant de blé employé à la subsistance de tous les différents ouvriers qui l’ont travaillée, et de ceux qui ont mis ces ouvriers en œuvre. Par là, le blé, qu’il eût été si difficile de transporter au loin sous sa première forme, se trouve virtuellement exporté sous la forme de l’ouvrage fait qui en est le résultat, et peut s’envoyer sous cette forme dans les coins du monde les plus reculés.
C’est de cette manière que se sont élevées naturellement et, pour ainsi dire, d’elles-mêmes les manufactures de Leeds, Halifax, Sheffield, Birmingham et Wolverhamp­ton. Ces sortes de manufactures doivent leur naissance à l’agriculture ; leur avance­ment et leur extension sont dans l’histoire de l’Europe moderne un événement posté­rieur aux progrès de celles qui ont dû leur naissance au commerce étranger. L’Angle­terre était connue par ses fabriques de beaux draps de laine d’Espagne, plus d’un siècle avant que les manufactures qui fleurissent aujourd’hui dans les villes que je viens de nommer fussent en état de travailler pour les marchés éloignés. L’avancement et l’extension de ces dernières ne pouvaient avoir lieu qu’en conséquence de l’avan­ce­ment et de l’extension de l’agriculture, qui eux-mêmes sont le dernier et le plus grand effet que puissent produire le commerce étranger et les manufactures auxquelles celui-ci donne immédiatement naissance, comme je vais l’expliquer tout à l’heure.





↑ Ce mot vient d’une sorte de poids nommé lest. La taxe se percevait sur les marchandises qui se vendaient au lest. Les autres termes s’expliquent assez d’eux-mêmes.
↑ Poll-tax, taxe par tête.
↑ Domesday-book, terrier général de toutes les terres d’Angleterre, fait sous Guillaume le Conquérant, et d’après lequel tous les propriétaires remirent leurs terres entre les mains du roi, pour les tenir de lui à titre de fief militaire.
↑ Voyez le Traité historique des Villes et Bourgs, par Brady, page 3, etc.
↑ Le vicomte ou lieutenant du comte.
↑ Voyez le Firma Burgi de Madox, page 18, ainsi que l’histoire de l’échiquier, chap. x, sect. 5, page 223, première édition. (Note de l’auteur)
↑ L’officier chargé de la collecte.
↑ Le centenier était l’officier civil et militaire établi sur dix dizaines, chaque dizaine étant composée de dix familles ou de dix hommes libres. Au-dessus de tous les centeniers d’un comté était le comte, ou le shérif son lieutenant, qui exerçait de même les fonctions civiles et militaires.
↑ Voyez Firma Burgi de Madox ; voyez aussi Pfeffel, sur les événements remarquables arrivés sous Frédéric II et ses successeurs de la maison de Souabe.
↑ Voyez Madox.
↑ Voyez Pfeffel.
↑ Voyez Histoire civile de Venise, par Sandi, part. 2e, vol. I, pages 247 et 256.(Note de l’auteur)

%%%%%%%%%%%%%%%%%%%%%%%%%%%%%%%%%%%%%%%%%%%%%%%%%%%%%%%%%%%%%%%%%%%%%%%%%%%%%%%%
%                                  Chapitre 4                                  %
%%%%%%%%%%%%%%%%%%%%%%%%%%%%%%%%%%%%%%%%%%%%%%%%%%%%%%%%%%%%%%%%%%%%%%%%%%%%%%%%

\chapter{Comment le commerce des villes a contribué à l’amélioration des campagnes}
\markboth{Comment le commerce des villes a contribué à l’amélioration des campagnes}{}

L’accroissement et la richesse des villes commerçantes et manufacturières ont contribué de trois manières différentes à l’amélioration et à la culture des campagnes auxquelles elles appartenaient.
Premièrement, en fournissant un marché vaste et rapproché pour le produit brut du pays, elles ont encouragé sa culture et ont engagé à faire de nouvelles amélio­rations. Cet avantage ne se borna pas même aux campagnes où la ville était située, mais il s’étendit plus ou moins à tous les pays avec lesquels elle faisait quelque com­merce. Elle ouvrait à tous un marché pour quelque partie de leur produit, soit brut, soit manufacturé et, par conséquent, encourageait à un certain point, chez tous, l’industrie et l’avancement. Cependant le pays même où la ville était située dut néces­sai­rement, par rapport à sa proximité, retirer le plus d’avantages de ce marché. Son produit brut se trouvant le moins chargé de frais de transport, les marchands purent en donner aux producteurs un meilleur prix, et néanmoins le fournir aux consommateurs à aussi bon compte que celui des pays les plus éloignes.
Secondement, les richesses que gagnèrent les habitants des villes furent souvent employées à acheter des terres qui se trouvaient à vendre, et dont une grande partie seraient souvent restées incultes. Les marchands sont, en général, jaloux de devenir propriétaires de biens de campagne, et quand ils le sont, ce sont ordinairement ceux qui s’occupent le plus d’améliorer leur propriété. Un marchand est habitué à employer de préférence son argent en projets utiles, tandis qu’un simple propriétaire de biens de campagne est le plus souvent accoutumé à employer le sien en pure dépense. L’un voit journellement son argent sortir de ses mains et y rentrer avec profit ; l’autre s’attend rarement à voir revenir celui qu’il a une fois déboursé. Cette différence d’ha­bi­tude influe naturellement, dans tous les genres d’affaires, sur leur caractère et sur leurs dispositions. Un négociant est communément hardi en entreprises, et le pro­priétaire de biens-fonds est timide. Le premier n’aura pas peur de placer à la fois un gros capital en amélioration sur sa terre, quand il aura la perspective probable qu’elle gagnera en valeur proportionnellement à la dépense. Que l’autre ait un capital, ce qui n’est pas fort ordinaire, il aura peine à se décider à en faire emploi de cette manière. S’il fait tout au plus quelque faible amélioration, ce ne sera pas volontiers avec un capital, mais avec ce qu’il aura épargné sur son revenu annuel. Quiconque a habité quelque temps une ville commerçante située dans un pays où la culture est peu avan­cée, a pu observer souvent combien, dans ce genre d’opérations, les gens de commer­ce sont plus entreprenants que les simples propriétaires de terres. D’ailleurs, les habitudes d’ordre, d’économie et d’attention qu’un commerçant contracte naturelle­ment dans la direction de ses affaires de commerce, le rendent bien plus propre à exécuter avec succès et avec profit des projets d’amélioration de toute espèce.
Troisièmement enfin, le commerce et les manufactures introduisirent par degrés un gouvernement régulier et le bon ordre, et avec eux la liberté et la sûreté indivi­duelle, parmi les habitants de la campagne qui avaient vécu jusqu’alors dans un état de guerre presque continuel avec leurs voisins, et dans une dépendance servile de leurs supérieurs. De tous les effets du commerce et des manufactures, c’est sans com­paraison le plus important, quoiqu’il ait été le moins observé. M. Hume est, autant que je sache, le seul écrivain qui en ait parlé jusqu’ici.
Dans un pays où il n’existe ni commerce étranger ni manufactures importantes, un grand propriétaire ne trouvant pas à échanger la plus grande partie du produit de ses terres qui se trouve excéder la subsistance des cultivateurs, en consomme la totalité chez lui, en une sorte d’hospitalité rustique. Si ce superflu est en état de faire vivre un cent ou un millier de personnes, il n’y a pas d’autre moyen de l’employer, que d’en nourrir un cent ou un millier de personnes. Il est donc en tout temps environné d’une foule de clients et de gens à sa suite, qui, n’ayant aucun équivalent à lui donner en retour de leur subsistance, mais étant entièrement nourris de ses bienfaits, sont à ses ordres, par la même raison qui fait que des soldats sont aux ordres du prince qui les paye. Avant l’extension du commerce et des manufactures en Europe, l’hospitalité qu’exerçaient les grands et les riches, depuis le souverain jusqu’au moindre baron, est au-dessus de tout ce dont nous pourrions aujourd’hui nous faire idée. La salle de Westminster était la salle à manger de Guillaume le Roux, et peut-être souvent n’était-elle pas encore trop grande pour le nombre des convives qu’il y traitait. On a cité comme trait de magnificence de Thomas Becket, qu’il faisait garnir le plancher de sa salle de paille fraîche ou de joncs dans la saison, afin que les chevaliers et les écuyers qui ne pouvaient trouver de sièges ne gâtassent point leurs beaux habits quand ils s’asseyaient à terre pour dîner. On dit que le grand comte de Warwick nourrissait tous les jours dans ses différents châteaux trente mille personnes, et si on a exagéré ce nombre, il faut toujours qu’il ait été très-grand, pour comporter une telle exagération. Il n’y a pas beaucoup d’années qu’en plusieurs endroits des montagnes d’Écosse il s’exerçait une hospitalité du même genre. Il paraît qu’elle est commune à toutes les nations qui connaissent peu le commerce et les manufactures. Le docteur Pocock raconte avoir vu un chef arabe dînant en pleine rue dans une ville où il était venu vendre ses marchandises, et invitant tous les passants, même de simples mendiants, à s’asseoir avec lui et à partager son repas.
Les cultivateurs des terres étaient à tous égards autant dans la dépendance d’un grand propriétaire que les gens même de sa suite. Ceux même d’entre eux qui n’étaient pas dans la condition de vilains étaient des tenanciers à volonté, qui payaient une rente tout à fait disproportionnée à la subsistance que la terre leur fournissait. Il y a quelques années que, dans les montagnes d’Écosse, une couronne[1], une demi-couronne, une brebis, un agneau, étaient une rente ordinaire pour des portions de terre qui nourrissaient toute une famille. Il en est encore de même dans quelques endroits, où cependant l’argent n’achète pas plus de marchandises qu’ailleurs. Mais, dans un pays où il faut que le produit superflu d’un vaste domaine soit consommé sur le domaine même, il sera souvent plus commode pour le propriétaire qu’il y en ait une partie de consommée hors de sa maison, pourvu que ceux qui la consomment soient autant sous sa dépendance que ses domestiques ou les gens de sa suite. Cela lui épargne l’embarras d’une compagnie trop nombreuse ou celui de tenir trop grande maison. Un tenancier à volonté, qui tient autant de terre qu’il lui en faut pour nourrir sa famille, sans en rendre guère plus qu’un simple cens, est autant sous la dépendance du pro­priétaire qu’un domestique ou un suivant quelconque ; il est, tout aussi bien que celui-ci, obligé à une obéissance sans réserve. Ce propriétaire nourrit ses tenanciers dans leurs maisons, tout comme il nourrit ses domestiques et suivants dans la sienne. Les uns et les autres tiennent également leur subsistance de ses bienfaits ; il est le maître de la leur retirer quand il lui plaît.
L’autorité qu’a nécessairement un grand propriétaire, dans cet état de choses, sur ses tenanciers et les gens de sa suite, fut le fondement de la puissance des anciens barons. Ils devinrent nécessairement les juges en temps de paix et les chefs en temps de guerre de tous ceux qui vivaient sur leurs terres. Ils pouvaient maintenir le bon ordre et l’exécution de la loi dans leurs domaines respectifs, parce que chacun d’eux pouvait faire agir contre l’indocilité d’un seul habitant la force réunie de tous les autres. Aucune autre personne n’avait assez d’autorité pour cela. Le roi, en particulier, ne l’avait pas. Dans ces anciens temps, le roi n’était guère autre chose que le plus grand propriétaire du royaume, celui auquel les autres grands propriétaires rendaient certains honneurs, à cause de la nécessité d’une défense commune contre les ennemis communs. Pour contraindre quelqu’un au payement d’une petite dette, dans les terres d’un grand propriétaire, où tous les habitants étaient armés et habitués à se rassembler, il en aurait coûté au roi, s’il avait essayé de le faire de sa propre autorité, autant d’ef­forts que pour étouffer une guerre civile. Il fut donc obligé d’abandonner l’admi­nis­tration de la justice, dans la plus grande partie des campagnes, à ceux qui étaient en état de l’administrer, et par la même raison de laisser le commandement de la milice des campagnes à ceux auxquels elle consentait d’obéir. 
C’est une erreur de croire que ces juridictions territoriales prirent leur origine dans les lois féodales. Non-seulement la justice la plus étendue, tant au civil qu’au criminel, mais même le pouvoir de lever des troupes, de battre monnaie, et même celui de faire des espèces de lois pour le gouvernement de leurs vassaux, furent autant de droits possédés allodialement par les grands propriétaires de terre plusieurs siècles avant que le nom même des lois féodales fût connu en Europe. L’autorité et la juridiction des seigneurs saxons en Angleterre paraissent avoir été tout aussi étendues avant la conquête que le furent après cette époque celles d’aucun seigneur normand. Or, ce n’est que depuis la conquête, à ce qu’on croit, que les lois féodales devinrent le droit commun de l’Angleterre. C’est un fait hors de doute que, longtemps avant l’intro­duction des lois féodales en France, les grands seigneurs y possédaient allodialement l’autorité et la juridiction la plus étendue. Cette autorité et cette multitude de juridic­tions avaient toutes leur source dans l’état où étaient les propriétés, et dans les mœurs et usages que nous venons de décrire. Sans remonter aux époques les plus reculées des monarchies de France et d’Angleterre, nous pourrons trouver dans des temps plus récents que de semblables effets ont toujours été le résultat nécessaire de cette même cause. Il n’y a pas trente ans qu’un gentilhomme du Lochabar, en Écosse, M. Cameron de Lochiel, sans aucune espèce de titre légal quelconque, n’étant pas ce qu’on appelait alors lord de royauté[2], ni même tenant en chef, mais vassal du duc d’Argyle, et moins qu’un simple juge de paix, avait pris néanmoins l’usage d’exercer sur ses gens la juri­diction criminelle la plus absolue. On prétend qu’il exerçait ce pouvoir avec la plus stricte équité, quoique sans nulles formalités de justice, et il est assez vraisemblable que l’état de cette partie de la province, à cette époque, le mit dans la nécessité de s’emparer de cette autorité pour maintenir la tranquillité publique. Ce gentilhomme, dont le revenu n’alla jamais au-delà de 500 livres par an, entraîna avec lui huit cents hommes de sa suite dans la rébellion de 1745. 
Bien loin d’avoir étendu l’autorité des grands seigneurs allodiaux, on doit regarder l’introduction des lois féodales comme une tentative faite pour la réprimer. Elles établirent une subordination réglée, avec une longue chaîne de services et de devoirs, depuis le roi jusqu’au moindre propriétaire. Pendant la minorité du propriétaire, les revenus et l’administration de sa terre tombaient dans les mains de son supérieur immédiat et, par conséquent, ceux des terres de tous les grands propriétaires tom­baient dans les mains du roi, qui était chargé de l’entretien et de l’éducation du pupille, et qui, en sa qualité de gardien, était censé avoir le droit de le marier à sa volonté, pourvu que ce fût d’une manière convenable à son rang. Mais quoique cette institution tendît nécessairement à renforcer l’autorité du roi et à affaiblir celle des grands propriétaires, cependant elle ne pouvait pas remplir assez ces deux objets pour établir l’ordre et un bon gouvernement parmi les habitants des campagnes, parce qu’elle n’apportait pas assez de changement dans l’état des propriétés, ni dans ces mœurs et usages qui étaient la source du désordre. L’autorité du gouvernement continua d’être toujours, comme auparavant, trop faible chez le chef et trop forte chez les membres subalternes, et c’était la force excessive de ces membres qui était cause de la faiblesse du chef. Après l’institution de la subordination féodale, le roi fut aussi hors d’état qu’auparavant de réprimer les violences des grands seigneurs. Ils continuèrent tou­jours de faire la guerre selon leur bon plaisir, presque sans cesse l’un contre l’autre, et très-souvent contre le roi, et les campagnes ouvertes furent toujours, comme aupara­vant, un théâtre de violences, de rapines et de désordres.
Mais ce que les institutions féodales, toutes violentes qu’elles étaient, n’avaient pu effectuer, l’action lente et insensible du commerce étranger et des manufactures le fit graduellement. Ces deux genres d’industrie fournirent peu à peu aux grands proprié­taires des objets d’échange à acquérir avec le produit superflu de leurs terres, objets qu’ils pouvaient consommer eux-mêmes sans en faire part à leurs tenanciers et aux gens de leur suite. Tout pour nous et rien pour les autres, voilà la vile maxime qui paraît avoir été, dans tous les âges, celle des maîtres de l’espèce humaine. Aussi, dès qu’ils purent trouver une manière de consommer par eux-mêmes la valeur totale de leurs revenus, ils ne furent plus disposés à en faire part à personne. Une paire de boucles à diamants, ou quelque autre frivolité tout aussi vaine, fut l’objet pour lequel ils donnèrent la subsistance, ou ce qui est la même chose, le prix de la subsistance d’un millier peut-être de personnes pour toute une année, et avec cette subsistance toute l’influence et l’autorité qu’elle pouvait leur valoir ; mais aussi les boucles étaient pour eux seuls, aucune autre créature humaine n’en partageait la jouissance ; au lieu que, dans l’ancienne manière de dépenser, il fallait au moins faire part à mille per­sonnes d’une dépense qui eût été de même valeur. Pour des hommes tels que ceux qui avaient le choix à faire, cette différence était un motif absolument décisif ; et c’est ainsi que, pour gratifier la plus puérile, la plus vile et la plus sotte de toutes les vanités, ils abandonnèrent par degrés tout ce qu’ils avaient de crédit et de puissance
Dans un pays qui ne fait point de commerce étranger et ne possède aucune manufacture importante, il n’est guère possible à un homme qui a 10,000 liv. sterl. de rente d’employer autrement son revenu qu’à faire subsister un millier peut-être de familles, qui dès lors sont toutes nécessairement à ses ordres. Mais, dans l’état actuel de l’Europe, un homme qui a cette fortune peut dépenser tout son revenu et, en général, il le dépense sans entretenir directement vingt personnes, ou sans avoir à ses ordres plus de dix laquais qui ne valent pas la peine qu’on leur commande. Indirecte­ment peut-être fait-il subsister autant et même beaucoup plus de monde qu’il n’aurait fait par l’ancienne manière de dépenser ; car si la quantité de productions précieuses pour lesquelles il échange son revenu ne forme pas un grand volume, le nombre d’ouvriers employés à les recueillir et à les préparer n’en est pas moins immense. Le prix énorme qu’elles ont vient, en général, des salaires du travail de tous ces ouvriers et des profits de ceux qui les ont mis immédiatement en œuvre. En payant ce prix, il rembourse ces salaires et ces profits, et ainsi il contribue indirectement à faire subsis­ter tous ces ouvriers et ceux qui les mettent en œuvre. Néanmoins, il ne contribue, en général, que pour une très-faible portion à la subsistance de chacun d’eux ; il n’y en a que très-peu auxquels il fournisse même le dixième de toute leur subsistance annuelle ; à plusieurs il n’en fournit pas la centième, et à quelques-uns pas la millième ni même la dix-millième partie. Ainsi, quoiqu’il contribue à la subsistance de tous, ils sont néan­moins tous plus ou moins indépendants de lui, parce qu’en général ils peuvent tous subsister sans lui.
Quand les grands propriétaires fonciers dépensent leur revenu à faire vivre leurs clients, vassaux et tenanciers, chacun d’eux fait vivre en entier tous ses clients, tous ses tenanciers ; mais quand ils dépensent leurs revenus à faire vivre des marchands et des ouvriers, il peut bien se faire que tous ces propriétaires, pris collectivement, fas­sent vivre un aussi grand nombre et peut-être même, à cause du gaspillage qui accom­pa­gne une hospitalité rustique, un bien plus grand nombre de gens qu’auparavant. Néan­moins, pris séparément, chacun de ces propriétaires ne contribue souvent que pour une très-petite part à la subsistance d’un individu quelconque de ce grand nom­bre. Chaque marchand ou ouvrier tire sa subsistance de l’occupation que lui donnent, non pas une seule, mais cent ou mille pratiques différentes. Ainsi, quoiqu’à un certain point il leur ait à toutes ensemble obligation de sa subsistance, il n’est néanmoins dans la dépendance absolue d’aucune d’elles.
La dépense personnelle des grands propriétaires s’étant successivement augmentée par ce moyen, il leur fut impossible de ne pas aussi diminuer successivement le nom­bre des gens de leur suite, jusqu’à finir-par la réformer tout entière. La même cause les amena, aussi par degrés, à congédier toute la partie inutile de leurs tenan­ciers. On étendit les fermes et, malgré les plaintes que firent les cultivateurs sur la dépopulation des terres, ils furent réduits au nombre purement nécessaire pour cultiver, selon l’état imparfait de culture et d’amélioration où étaient les terres dans ce temps-là. Le propriétaire, en écartant ainsi toutes les bouches inutiles, et en exigeant du fermier toute la valeur de la ferme, obtint un plus grand superflu, ou, ce qui est la même chose, le prix d’un plus grand superflu ; et ce prix, les marchands et manufacturiers lui fournirent bientôt les moyens de le dépenser sur sa personne, de la même manière qu’il avait déjà dépensé le reste. La même cause agissant toujours, il chercha à faire monter ses revenus au-dessus de ce que ces terres, dans l’état où était leur culture, pouvaient lui rapporter. Ses fermiers ne purent s’accorder avec lui là-dessus, qu’à la seule condition d’être assurés de leur possession pendant un terme d’années assez long pour avoir le temps de recouvrer, avec profit, tout ce qu’ils pourraient placer sur la terre en améliorations nouvelles. La vanité dépensière du propriétaire le fit souscrire à cette condition, et de là l’origine des longs baux.
Un tenancier, même un tenancier à volonté, qui paye de la terre tout ce qu’elle vaut, n’est pas absolument sous la dépendance du propriétaire. Les gains que ces deux personnes font l’une avec l’autre sont égaux et réciproques, et un pareil tenancier n’ira exposer ni sa vie ni sa fortune au service du propriétaire. Si le tenancier a un bail à long terme, il est alors tout à fait indépendant, et il ne faut pas que son propriétaire s’avise d’en attendre le plus léger service au-delà de ceux qui sont expressément stipulés par le bail, ou auxquels le fermier sait bien être obligé par la loi du pays.
Les tenanciers étant ainsi devenus indépendants, et les clients congédiés, les grands propriétaires ne furent plus en état d’interrompre le cours ordinaire de la justice, ni de troubler la tranquillité publique dans le pays. Après avoir ainsi vendu le droit de leur naissance, non pas comme le fit Ésaü, dans un moment de faim et de nécessité, pour un plat de lentilles, mais dans le délire de l’abondance, pour des colifi­chets et des niaiseries plus propres à amuser des enfants qu’à occuper sérieusement des hommes, ils devinrent aussi peu importants que l’est un bon bourgeois ou un bon artisan d’une ville. Il s’établit dans la campagne une forme d’administration aussi bien réglée que dans la ville, personne n’ayant plus, dans l’une non plus que dans l’autre, le pouvoir de mettre des obstacles à l’action du gouvernement.
je ne puis m’empêcher de faire ici une remarque qui est peut-être hors de mon sujet, c’est qu’il est très-rare de trouver, dans des pays commerçants, de très-anciennes familles qui aient possédé de père en fils, pendant un grand nombre de générations, un domaine considérable. Il n’y a, au contraire, rien de plus commun dans les pays qui ont peu de commerce, tels que le pays de Galles ou les montagnes de l’Écosse. Les histoires arabes sont, à ce qu’il paraît, toutes remplies de généalogies, et il y a une histoire écrite par un khan de Tartares[3], qui a été traduite en plusieurs langues d’Eu­rope, et qui ne contient presque pas autre chose ; preuve que chez ces peuples les anciennes familles sont très-communes. Dans des pays où un homme riche ne peut dépenser son revenu qu’à faire vivre autant de gens qu’il en peut nourrir, il n’est pas dans le cas de se laisser aller trop loin, et il est bien rare que sa bienveillance l’em­porte au point de lui en faire entretenir plus qu’il ne peut. Mais dans les pays où il a occasion de dépenser pour sa personne les revenus les plus considérables, il arrive souvent que sa dépense n’a pas de bornes, parce que souvent sa vanité ou cet amour pour sa personne n’en a aucunes. C’est pourquoi, dans les pays commerçants, il arrive rarement que les richesses demeurent longtemps dans la même famille, en dépit de tous les moyens forcés que prend la loi pour en empêcher la dissipation. Chez les peuples simples, au contraire, cela se voit communément, et sans le secours de la loi ; car, parmi les nations de pasteurs, tels que les Tartares et les Arabes, la nature péris­sable de leurs propriétés rend nécessairement impraticables toutes les lois de cette espèce.
Ainsi, une révolution qui fut si importante pour le bonheur public fut consommée par le concours de deux différentes classes de gens qui étaient bien éloignés de penser au bien général. Le motif des grands propriétaires fut de satisfaire une ridicule vanité. Les marchands et manufacturiers, beaucoup moins ridicules, agirent purement en vue de leur intérêt, et d’après ce principe familier à toute la classe marchande, qu’il ne faut pas négliger un petit profit dès qu’il y a moyen de le réaliser. Pas un d’eux ne sentait ni ne prévoyait la grande révolution que l’extravagance des uns et l’industrie des au­tres amenaient insensiblement à la fin.
C’est ainsi que, dans la majeure partie de l’Europe, le commerce et les manufac­tures des villes, au lieu d’être l’effet de la culture et de l’amélioration des campagnes, en ont été l’occasion et la cause.
Toutefois cet ordre, étant contraire au cours naturel des choses, est nécessairement lent et incertain. Que l’on compare la lenteur des progrès des pays de l’Europe, dont la richesse dépend en grande partie de leur commerce et de leurs manufactures, avec la marche rapide de nos colonies américaines, dont la richesse est toute fondée sur l’agriculture. Dans la majeure partie de l’Europe, il faut au moins, à ce qu’on prétend, cinq cents ans pour doubler le nombre des habitants, tandis que dans plusieurs de nos colonies de l’Amérique septentrionale, il double, dit-on, en vingt ou vingt-cinq ans. En Europe, la loi de primogéniture et toutes celles qui tendent à perpétuer les biens dans les familles empêchent la division des grands domaines, et par là s’opposent à ce que les petits propriétaires se multiplient. Cependant, un petit propriétaire qui connaît tous les recoins de son petit territoire, qui les surveille tous avec cette attention soi­gneuse qu’inspire la propriété, et surtout une petite propriété, et qui, pour cette raison, se plaît non-seulement à la cultiver, mais même à l’embellir, est en général, de tous ceux qui font valoir, celui qui y apporte le plus d’industrie et le plus d’intelli­gence, et aussi celui qui réussit le mieux. D’ailleurs, ces mêmes règlements tiennent hors du marché une si grande quantité de terres, qu’il y a toujours plus de capitaux qui en cherchent qu’il n’y a de terres à vendre, en sorte que celles qu’on vend se vendent toujours à un prix de monopole[4]. La rente ne paye jamais l’intérêt du prix de l’achat et, d’ailleurs, elle est diminuée par des frais de réparations et d’autres charges acciden­telles auxquelles l’intérêt de l’argent n’est pas assujetti. Une acquisition de biens-fonds est, dans toute l’Europe, le moins avantageux de tous les placements pour de petits capitaux. À la vérité, un homme d’une fortune médiocre, qui se retire des affaires, préférera quelquefois placer son petit capital en terres, parce qu’il y trouve plus de sûreté. Souvent aussi un homme de profession, qui tire son revenu d’une autre source, aime à assurer ses épargnes par un pareil placement. Mais un jeune homme qui, au lieu de s’adonner au commerce ou à quelque profession, emploierait un capital de 2 ou 3,000 liv. sterl. à acheter et à faire valoir une petite propriété territoriale, pourrait, à la vérité, espérer de mener une vie fort heureuse et fort indépendante ; mais il faudra qu’il dise adieu pour jamais à tout espoir de grande fortune ou de grande illustration, ce qu’un autre emploi de son capital eût pu lui donner la perspective d’acquérir dans une autre sphère. Il y a aussi telle personne qui, ne pouvant pas aspirer à devenir propriétaire, dédaignera de se faire fermier.
Ainsi, la petite quantité de terres disponibles sur le marché, et le haut prix de celles qui y sont mises, détournent de la culture et de l’amélioration de la terre un grand nombre de capitaux qui, sans cela, auraient pris cette direction. Dans l’Améri­que septentrionale, au contraire, on trouve souvent que 50 ou 60 liv. sterl. sont un fonds suffisant pour commencer une plantation. Là, l’acquisition et l’amendement d’une terre inculte sont l’emploi le plus avantageux pour les plus petits capitaux com­me pour les plus gros, et ils offrent le chemin le plus direct pour arriver à tout ce que le pays peut offrir de fortune et d’honneurs. Ces sortes de terres, à la vérité, s’obtien­nent presque pour rien dans l’Amérique septentrionale, ou du moins à un prix fort au-dessous de ce que vaut le produit naturel ; chose impossible en Europe, et vérita­ble­ment dans tout pays où toutes les terres sont depuis longtemps des propriétés privées. Cependant, si les biens-fonds se partageaient par égales portions entre tous les enfants, alors, à la mort d’un propriétaire, chef d’une famille nombreuse, le bien se trouverait généralement mis en vente. Il viendrait au marché assez de terres pour qu’elles ne fussent plus vendues à un prix de monopole ; la rente nette de la terre se rapprocherait bien davantage de l’intérêt du prix d’achat, et on pourrait employer un petit capital en acquisition de biens-fonds, avec autant de profit que de toute autre manière.
L’Angleterre, par la fertilité naturelle de son sol, la grande étendue de ses côtes, relativement à celle de tout le pays, et par la quantité de rivières navigables qui la traversent et qui donnent à quelques-unes de ses parties les plus enfoncées dans les terres la commodité du transport par eau, est un pays aussi bien disposé peut-être par la nature qu’aucun grand pays de l’Europe, pour être le siège d’un grand commerce étranger, de manufactures destinées aux marchés éloignés, et de tous les autres genres d’industrie qui peuvent en résulter.
De plus, depuis le commencement du règne d’Élisabeth, la législature a mis une attention particulière aux intérêts du commerce et des manufactures et, dans le fait, il n’y a pas de pays en Europe, sans en excepter même la Hollande, dont les lois soient, en somme, plus favorables à cette espèce d’industrie. Aussi, depuis cette période, le commerce et les manufactures ont-ils fait des progrès continuels. La culture et l’amélioration des campagnes ont fait aussi, sans contredit, des progrès successifs ; mais ceux-ci semblent n’avoir fait que suivre lentement et de loin la marche plus rapide du commerce et des manufactures. Vraisemblablement, la majeure partie des terres étaient cultivées avant le règne d’Élisabeth ; il en reste encore une très-grande quantité qui est inculte, et la culture de la très-majeure partie du reste est fort au-des­sous de ce qu’elle pourrait être. Cependant la loi d’Angleterre favorise l’agriculture, soit indirectement en protégeant le commerce, soit même par plusieurs encoura­ge­ments directs. Hors les temps de cherté, l’exportation des grains est non-seulement libre, mais encouragée par une prime. Dans les temps d’une abondance moyenne, l’importation du blé étranger est chargée de droits qui équivalent à une prohibition. L’importation des bestiaux vivants, excepté d’Irlande, est prohibée en tout temps, et ce n’est que récemment qu’elle a été permise de ce dernier pays. Ainsi, les cultiva­teurs des terres ont un privilège de monopole contre leurs concitoyens, pour les deux articles les plus forts et les plus importants du produit de la terre, le pain et la viande de boucherie. Ces encouragements, quoique peut-être au fond absolument illusoires, comme je tâcherai de le faire voir par la suite[5], sont au moins une preuve de la bonne intention qu’a la législature de favoriser l’agriculture. Mais un encouragement qui est d’une bien plus grande importance que tout le reste, c’est qu’en Angleterre la classe des paysans jouit de toute la sûreté, de toute l’indépendance et de toute la considé­ration que lui peut procurer la loi. Ainsi, pour un pays où le droit de primogéniture a lieu, où on paye la dîme, et où la méthode de perpétuer les propriétaires, quoique contraire à l’esprit de la loi, est admise en certains cas, il est impossible de donner à l’agriculture plus d’encouragement que ne lui en donne l’Angleterre ; telle est pourtant, malgré tout cela, l’état de sa culture. Que serait-il donc si la loi n’eût pas donné d’encouragement direct à l’agriculture, outre celui qui procède indirectement des pro­grès du commerce, et si elle eût laissé la classe des paysans dans la condition où on les laisse dans la plupart des pays de l’Europe ? Il y a aujourd’hui plus de deux cents ans d’écoulés depuis le commencement du règne d’Élisabeth, et c’est une période aussi longue que puisse la supporter habituellement le cours des prospérités humaines.
La France paraît avoir eu une partie considérable du commerce étranger, près d’un siècle avant que l’Angleterre fût distinguée comme pays commerçant. La marine de France est importante, suivant les connaissances qu’on pouvait avoir alors, dès avant l’expédition de Charles VIII à Naples. Néanmoins, la culture et l’amélioration sont, en France, généralement au-dessous de ce qu’elles sont en Angleterre. C’est que les lois du pays n’ont jamais donné le même encouragement direct à l’agriculture.
Le commerce étranger de l’Espagne et du Portugal avec les autres nations de l’Europe, quoiqu’il se fasse principalement par des vaisseaux étrangers, est néanmoins considérable. Ces deux pays font le commerce de leurs colonies sur leurs propres bâ­ti­ments, et ce commerce est encore beaucoup plus grand que l’autre, à cause de la richesse et de l’étendue de ces colonies ; mais tout ce commerce n’a jamais introduit, dans aucun de ces deux pays, de manufactures considérables pour la vente au loin, et la majeure partie de l’un et de l’autre reste encore sans culture. Le commerce étranger du Portugal date d’une plus ancienne époque que celui d’aucun autre pays de l’Europe, l’Italie exceptée.
L’Italie est le seul grand pays de l’Europe qui paraisse avoir été cultivé et amélio­ré, dans toutes ses parties, par le moyen du commerce étranger et des manufactures destinées aux marchés éloignés. L’Italie, suivant Guichardin, était, avant l’invasion de Charles VIII, aussi bien cultivée dans les endroits les plus montagneux et les plus stériles que dans les plus unis et les plus fertiles. La situation avantageuse du pays, le grand nombre d’États indépendants qui y subsistaient alors, ne contribuèrent pas peu, vraisemblablement, à cette grande culture. Il n’est pas non plus impossible, malgré cette expression générale d’un des plus judicieux et des plus circonspects de nos histo­riens modernes, que l’Italie ne fût pas alors mieux cultivée que ne l’est aujourd’hui l’Angleterre.
Cependant, le capital acquis à un pays par le commerce et les manufactures n’est toujours pour lui qu’une possession très-précaire et très-incertaine, tant qu’il n’y en a pas quelque partie d’assurée et de réalisée dans la culture et l’amélioration de ses terres. Un marchand, comme on l’a très-bien dit, n’est nécessairement citoyen d’aucun pays en particulier. Il lui est, en grande partie, indifférent en quel lieu il tienne son commerce, et il ne faut que le plus léger dégoût pour qu’il se décide à emporter son capital d’un pays à un autre, et avec lui toute l’industrie que ce capital mettait en acti­vité. On ne peut pas dire qu’aucune partie en appartienne à un pays en particulier, jusqu’à ce que ce capital y ait été répandu, pour ainsi dire, sur la surface de la terre en bâtiments ou en améliorations durables. De toutes ces immenses richesses qu’on dit avoir été possédées par la plupart des villes hanséatiques, il ne reste plus maintenant aucuns vestiges, si ce n’est dans les chroniques obscures des treizième et quatorzième siècles. On ne sait même que très-imparfaitement où quelques-unes d’entre elles fu­rent situées, ou à quelles villes de l’Europe appartiennent les noms latins qui sont donnés à certaines de ces villes. Mais quoique les calamités qui désolèrent l’Italie sur la fin du quinzième siècle et au commencement du seizième aient extrêmement dimi­nué le commerce et les manufactures des grandes villes de la Lombardie et de la Toscane, ces pays n’en sont pas moins encore au nombre des plus peuplés et des mieux cultivés de l’Europe. Les guerres civiles de la Flandre et le gouvernement espa­gnol qui leur succéda chassèrent le grand commerce qui se faisait dans les villes d’Anvers, de Gand et de Bruges. Mais la Flandre continue toujours d’être une des provinces de l’Europe les plus riches, les plus peuplées et les mieux cultivées. Les révolutions ordinaires de la guerre et des gouvernements dessèchent les sour­ces de la richesse qui vient uniquement du commerce. Celle qui procède des progrès plus solides de l’agriculture est d’une nature beaucoup plus durable, et, pour la détrui­re, il ne faut rien moins que ces convulsions violentes causées par un siècle ou deux de déprédations continuelles et d’incursions de peuples guerriers et barbares, telles que celles qui eurent lieu dans la partie occidentale de l’Europe, quelque temps avant et après la chute de l’empire romain.
 
 
 
↑ Pièce de monnaie d’argent qui vaut 5 sterling, ou environ 6 francs.
↑ Lord of Regality. On nommait ainsi des seigneurs qui avaient la prétention de ne pas reconnaître la juridiction royale, et d’exercer de leur chef divers droits régaliens. On peut consulter, sur la nature et l’origine de cette usurpation, la note 23e de l’Introduction à l’Histoire de Charles V, par Robertson.
↑ Histoire généalogique des Tartares, par Abulghazi Rahadur, kan de Khowaresm ; traduite en français. Leyde, 1729 ; 1 vol. in-12.
↑ Les idées que le Dr Smith exprime dans ce passage sont d’une profondeur et d’une générosité que l’on chercherait en vain dans le plus grand nombre de ses disciples et de ses commentateurs. Citoyen d’un pays gouverné par une aristocratie dont le pouvoir et l’existence reposent sur la grande propriété, Adam Smith n’en a pas moins reconnu les avantages et la justice de la division des propriétés foncières, qui donne naissance à la petite propriété. A. B.
↑ Liv. IV, chap. 2, 5 et 8.

%%%%%%%%%%%%%%%%%%%%%%%%%%%%%%%%%%%%%%%%%%%%%%%%%%%%%%%%%%%%%%%%%%%%%%%%%%%%%%%%
%                                                                              %
%                                   Livre IV                                   %
%                                                                              %
%%%%%%%%%%%%%%%%%%%%%%%%%%%%%%%%%%%%%%%%%%%%%%%%%%%%%%%%%%%%%%%%%%%%%%%%%%%%%%%%

\part{Des systèmes d’économie politique}
\markboth{Des systèmes d’économie politique}{}

%%%%%%%%%%%%%%%%%%%%%%%%%%%%%%%%%%%%%%%%%%%%%%%%%%%%%%%%%%%%%%%%%%%%%%%%%%%%%%%%
%                                 Introduction                                 %
%%%%%%%%%%%%%%%%%%%%%%%%%%%%%%%%%%%%%%%%%%%%%%%%%%%%%%%%%%%%%%%%%%%%%%%%%%%%%%%%

\chapter*{\textcolor{coquelicot}{Introduction}}
\markboth{Introduction}{}

L’Économie politique, considérée comme une branche des connaissances du législateur et de l’homme d’État, se propose deux objets distincts : le premier, de procurer au peuple un revenu ou une subsistance abondante, ou, pour mieux dire, de le mettre en état de se procurer lui-même ce revenu ou cette subsistance abondante ; le second objet est de fournir a l’État ou à la communauté un revenu suffisant pour le service public : elle se propose d’enrichir à la fois le peuple et le souverain[1] 
La différence de la marche progressive de l’opulence dans des âges et chez des peuples différents a donné naissance à deux systèmes différents d’économie politique, sur les moyens d’enrichir le peuple. On peut nommer l’un Système mercantile, et l’autre Système de l’Agriculture. Je vais tâcher des les exposer l’un et l’autre avec autant d’étendue et de clarté qu’il me sera possible. Je commencerai par le Système mercantile : c’est le système moderne et le plus connu dans le pays et le siècle où j’écris.
 
 
 
↑ J’aimerais mieux dire que l’objet de l’économie politique est de faire connaître les moyens par lesquels les richesses se forment, se distribuent et se consomment. Le gouvernement n’entre qu’accessoirement dans ce système de choses, soit pour favoriser, soit pour contrarier la production, soit pour prélever une partie des produits. (Note inédite de J. B. Say.)

Dans les vues économiques du docteur Smith, la richesse nationale est toujours trop exclusivement présentée comme le principal objet à étudier. Cependant l’économie politique peut être considérée comme une théorie de gouvernement ayant pour but essentiel le bon ordre et la justice, dont la richesse nationale est une conséquence nécessaire, quoique indirecte.
Buchanan.
L’économie politique est généralement définie aujourd’hui : la science des lois qui règlent la production, la distribution et la consommation des choses qui possèdent une valeur échangeable et qui sont en même temps nécessaires, utiles ou agréable à l’homme*. Mac Culloch.
*. Nous aurions pu multiplier à l’infini les définitions que tous les auteurs d’économie politique se sont crus obligés de sonner de la science, les uns pour étendre son domaine, les autres pour lui imposer des limites. C’est en Allemagne et en France qu’on s’est le plus écarté du véritable terrain aujourd’hui généralement assigné à l’Économie politique : quelques économistes ont voulu en faire la science universelle ; d’autres ont essayé de la restreindre à des proportions exiguës et vulgaires. La lutte qui existe en France entre ces deux opinions extrêmes consiste à savoir si l’Économie politique doit être considérée comme l’exposition de ce qui est, ou comme le programme de ce qui doit être, c’est-à-dire comme une science naturelle, ou comme une science morale. Nous croyons qu’elle participe des deux natures : nous croyons surtout qu’il serait dangereux de la laisser égarer dans le vague où la voudraient pousser des utopistes ardents à la controverse : et c’est pour ce motif que nous applaudissons à l’accord à peu près unanime avec lequel la définition proposée par J. B. Say est adoptée maintenant en Europe. A. B.

%%%%%%%%%%%%%%%%%%%%%%%%%%%%%%%%%%%%%%%%%%%%%%%%%%%%%%%%%%%%%%%%%%%%%%%%%%%%%%%%
%                                  Chapitre 1                                  %
%%%%%%%%%%%%%%%%%%%%%%%%%%%%%%%%%%%%%%%%%%%%%%%%%%%%%%%%%%%%%%%%%%%%%%%%%%%%%%%%

\chapter{Du principe sur lequel se fonde le système mercantile}
\markboth{Du principe sur lequel se fonde le système mercantile}{}

La double fonction que remplit l’argent, et comme instrument de commerce et comme mesure des valeurs, a donné naturellement lieu à cette idée populaire, que l’ar­gent fait la richesse, ou que la richesse consiste dans l’abondance de l’or et de l’argent. L’argent servant d’instrument de commerce, quand nous avons de l’argent, nous pouvons bien plutôt nous procurer toutes les choses dont nous avons besoin, que nous ne pourrions le faire par le moyen de toute autre marchandise. Nous trouvons à tout moment que la grande affaire, c’est d’avoir de l’argent ; quand une fois on en a, les autres achats ne souffrent pas la moindre difficulté. D’un autre côté, l’argent servant de mesure des valeurs, nous évaluons toutes les autres marchandises par la quantité d’argent contre laquelle elles peuvent s’échanger. Nous disons d’un homme riche, qu’il a beaucoup d’argent, et d’un homme pauvre, qu’il n’a pas d’argent. On dit d’un homme économe ou d’un homme qui a grande envie de s’enrichir, qu’il aime l’argent ; et en parlant d’un homme sans soin, libéral ou prodigue, on dit que l’argent ne lui coûte rien. S’enrichir, c’est acquérir de l’argent ; en un mot, dans le langage ordinaire, ri­ches­se et argent sont regardés comme absolument synonymes.
On raisonne de la même manière à l’égard d’un pays. Un pays riche est celui qui abon­de en argent, et le moyen le plus simple d’enrichir le sien, c’est d’y entasser l’or et l’argent. Quelque temps après la découverte de l’Amérique, quand les Espagnols abordaient sur une côte inconnue, leur premier soin était ordinairement de s’informer si on trouvait de l’or et de l’argent dans les environs. Sur la réponse qu’ils recevaient, ils jugeaient si le pays méritait qu’ils y fissent un établissement, ou bien s’il ne valait pas la peine d’être conquis. Le moine Plan-Carpen, qui fut envoyé en ambassade par le roi de France auprès d’un des fils du fameux Gengis-Kan, dit que les Tartares avaient coutume de lui demander s’il y avait grande abondance de bœufs et de mou­tons dans le royaume de France. Cette question avait le même but que celle des Espa­gnols. Ces Tartares voulaient aussi savoir si le pays valait la peine qu’ils en entrepris­sent la conquête. Le bétail est instrument de commerce et une mesure de valeur chez les Tartares, comme chez tous les peuples pasteurs, qui, en général, ne connaissent pas l’usage de l’argent. Ainsi, suivant eux, la richesse consistait en bétail, comme, suivant les Espagnols, elle consistait en or et en argent. De ces deux idées, celle des Tartares approchait peut-être le plus de la vérité.
M. Locke observe qu’il y a une distinction à faire entre l’argent et les autres bien meubles. Tous les autres biens meubles, dit-il, sont d’une nature si périssable, qu’il y a peu de fonds à faire sur la richesse qui consiste dans ce genre de biens et une nation qui en possède, dans une année, une grande abondance, peut sans aucune exportation, mais par sa propre dissipation et son imprudence, en manquer l’année suivante. L’ar­gent, au contraire, est un ami solide qui, tout en voyageant beaucoup de côté et d’au­tre et de main en main, ne court pas risque d’être dissipé ni consommé, pourvu qu’on l’empêche de sortir du pays. Ainsi, suivant lui, l’or et l’argent sont la partie la plus solide et la plus essentielle des richesses mobilières ; et d’après cela il pense que le grand objet de l’économie politique, pour un pays, ce doit être d’y multiplier ces métaux.
D’autres conviennent que si une nation pouvait être supposée exister séparément du reste du monde, il ne serait d’aucune conséquence pour elle qu’il circulât chez elle beaucoup ou peu d’argent. Les choses consommables qui seraient mises en circulation par le moyen de cet argent s’y échangeraient seulement contre un plus grand ou un plus petit nombre de pièces ; la richesse ou la pauvreté du pays (comme ils veulent bien en convenir) dépendrait entièrement de l’abondance ou de la rareté de ces choses consommables. Mais ils sont d’avis qu’il n’en est pas de même à l’égard des pays qui ont des relations avec les nations étrangères, et qui sont obligés de soutenir des guer­res à l’extérieur et d’entretenir des flottes et des armées dans des contrées éloignées. Tout cela ne peut se faire, disent-ils, qu’en envoyant au-dehors de l’argent pour payer ces dépenses, et une nation ne peut pas envoyer beaucoup d’argent hors de chez elle, à moins qu’elle n’en ait beaucoup au-dedans. Ainsi, toute nation qui est dans ce cas doit tâcher, en temps de paix, d’accumuler de l’or et de l’argent, pour avoir, quand le besoin l’exige, de quoi soutenir la guerre avec les étrangers.
Par une suite de ces idées populaires, toutes les différentes nations de l’Europe se sont appliquées, quoique sans beaucoup de succès, à chercher tous les moyens possi­bles d’accumuler l’or et l’argent dans leurs pays respectifs. L’Espagne et le Portugal, possesseurs des principales mines qui fournissent l’Europe de ces métaux, en ont prohibé l’exportation sous les peines les plus graves, ou l’ont assujettie à des droits énormes. Il paraît que la même prohibition a fait anciennement partie de la politique de la plupart des autres nations de l’Europe. On la trouve même là où l’on devrait le moins s’y attendre, dans quelques anciens actes du parlement d’Écosse, qui défendent, sous de fortes peines, de transporter l’or et l’argent hors du royaume. La même politique a eu lieu aussi autrefois en France et en Angleterre.
Quand ces pays furent devenus commerçants, cette prohibition parut, en beaucoup d’occasions, extrêmement incommode aux marchands. Il arrivait souvent que ceux-ci auraient pu acheter plus avantageusement avec de l’or et de l’argent qu’avec toute autre marchandise les denrées étrangères qu’ils voulaient importer dans leur pays ou transporter dans quelque autre pays étranger. Ils réclamèrent donc contre cette prohi­bition, comme nuisible au commerce.
Ils représentèrent d’abord que l’exportation de l’or et de l’argent, faite dans la vue d’acheter des marchandises étrangères, ne diminuait pas toujours la quantité de ces métaux dans le royaume. Qu’au contraire elle pouvait souvent augmenter, parce que si la consommation du pays en denrées étrangères n’augmente pas pour cela, alors ces denrées étrangères importées pourront être réexportées à d’autres pays étrangers, dans lesquels étant vendues avec un gros profit, elles feront rentrer une somme d’argent bien plus forte que celle qui est sortie primitivement pour les acheter. M. Mun com­pare cette opération du commerce étranger à ce qui a lieu dans l’agriculture aux épo­ques des semailles et de la moisson. « Si nous ne considérions, dit-il, l’action du laboureur qu’au moment des semailles seulement, où il répand à terre une si grande quantité de bon blé, il nous semblerait agir en insensé plutôt qu’en cultivateur. Mais si nous songeons en même temps aux travaux de la moisson, qui est le but de ses soins, nous pouvons alors apprécier la valeur de son opération et le grand surcroît d’abondance qui en résulte. »
En second lieu, ils représentèrent que cette prohibition ne pouvait pas prévenir l’exportation de l’or et de l’argent qu’il était toujours facile de faire sortir en fraude, par rapport à la petitesse de volume de ces métaux relativement à leur valeur. Que le seul moyen d’empêcher cette exportation, c’était de porter une attention convenable à ce qu’ils appelaient la balance du commerce. Que quand le pays exportait pour une valeur plus grande que celle de ce qu’il importait, alors il lui était dû une balance par les nations étrangères, laquelle lui était nécessairement payée en or et en argent, et par là augmentait la quantité de ces métaux dans le royaume ; mais que lorsque le pays importait pour une plus grande valeur que celle qu’il exportait, alors il était dû aux nations étrangères une balance contraire qu’il fallait leur payer de la même ma­nière, et qui par là diminuait cette quantité de métaux. Que, dans ce dernier cas, prohiber l’exportation de ces métaux, ce ne serait pas l’empêcher, mais seulement la rendre plus coûteuse en y mettant plus de risques ; que c’était un moyen de rendre le change encore plus défavorable qu’il ne l’aurait été sans cela au pays débiteur de la balance ; le marchand qui achetait une lettre de change sur l’étranger étant obligé de payer alors au banquier qui la lui vendait, non-seulement le risque ordinaire, la peine et les frais du transport de l’argent, mais encore, de plus, le risque extraordinaire résultant de la prohibition. Que plus le change était contre un pays, et plus la balan­ce du commerce devenait aussi nécessairement contre lui, l’argent de ce pays perdant alors nécessairement d’autant de sa valeur, comparativement avec celui du pays auquel la balance était due. Qu’en effet, si le change entre l’Angleterre et la Hollan­de, par exemple, était de 5 pour 100 contre l’Angleterre, il faudrait alors cent cinq onces d’argent en Angleterre pour acheter une lettre de change de cent onces payables en Hollande ; que, par conséquent, cent cinq onces d’argent en Angleterre ne vau­draient que cent onces d’argent en Hollande, et ne pourraient acheter qu’une quan­tité proportionnée de marchandises hollandaises ; tandis qu’au contraire cent onces d’ar­gent en Hollande vaudraient cent cinq onces en Angleterre, et pourraient acheter une quantité proportionnée de marchandises anglaises ; que les marchandises anglaises vendues à la Hollande en seraient vendues d’autant meilleur marché ; et les marchan­dises hollandaises vendues à l’Angleterre le seraient d’autant plus cher, à raison de la différence du change entre les deux nations ; que par ce moyen, d’une part, l’Angle­terre tirerait d’autant moins à soi de l’argent hollandais et que, de l’autre, il irait d’au­tant plus d’argent anglais à la Hollande à proportion du montant de cette différence et que, par conséquent, la balance du commerce en serait nécessairement d’autant plus contraire à l’Angleterre, et nécessiterait l’exportation en Hollande d’une somme plus forte en or et en argent[1]. 
Ces raisonnements étaient en partie justes et en partie sophistiques. Ils étaient justes en tant qu’ils affirmaient que l’exportation de l’or et de l’argent par le commerce pouvait être souvent avantageuse au pays. Ils étaient justes aussi en soutenant qu’au­cu­ne prohibition ne pouvait empêcher l’exportation de ces métaux quand les particu­liers trouvaient quelque bénéfice à les exporter. Mais ils n’étaient que de purs sophismes quand ils supposaient que le soin de conserver ou d’augmenter la quantité de ces métaux appelait plus particulièrement l’attention du gouvernement que ne le fait le soin de conserver ou d’augmenter la quantité de toute autre marchandise utile que la liberté du commerce ne manque jamais de procurer en quantité convenable, sans qu’il soit besoin de la moindre attention de la part du gouvernement. C’était encore un sophisme peut-être que de prétendre que le haut prix du change augmentait nécessairement ce qu’ils appelaient la balance défavorable du commerce, ou qu’il occasionnait une plus forte exportation d’or et d’argent. Ce haut prix du change était, il est vrai, extrêmement désavantageux aux marchands qui avaient quel­que argent à faire remettre en pays étranger ; ils payaient d’autant plus cher les lettres de change que leurs banquiers leur donnaient sur des pays étrangers. Mais encore que le risque procédant de la prohibition pût occasionner aux banquiers quelque dépense extraordinaire, il ne s’ensuivait pas pour cela qu’il dût sortir du pays aucun argent de plus. Cette dépense, en général, se faisait dans le pays même pour payer la fraude qui opérait la sortie de l’argent en contrebande, et elle ne devait guère occasionner l’ex­por­­tation d’un seul écu au-delà de la somme précise pour laquelle on tirait. De plus, le haut prix du change devait naturellement disposer les marchands à faire tous leurs efforts pour balancer le plus près possible leurs importations avec leurs exportations, afin de n’avoir à payer ce haut prix du change que sur la Plus petite somme possible. Enfin, le haut prix du change devait opérer sur le prix des marchandises étrangères comme aurait fait un impôt, c’est-à-dire élever ce prix, et par là diminuer la consom­ma­tion de ces marchandises. Donc il ne devait pas tendre à augmenter, mais au contraire à diminuer ce qu’ils appelaient la balance défavorable du commerce et, par conséquent, l’exportation de l’or et de l’argent.
Néanmoins ces arguments, tels qu’ils étaient, réussirent à convaincre ceux à qui on les adressait, ils étaient présentés par des commerçants à des parlements, à des conseils de princes, à des nobles et à des propriétaires de campagne ; par des gens qui étaient censés entendre parfaitement les affaires de commerce, à des personnes qui se rendaient la justice de penser qu’elles ne connaissaient rien à ces sortes de matières. Que le commerce étranger apportât des richesses dans le pays, c’était ce que l’expé­rience démontrait à ces nobles et à ces propriétaires, tout aussi bien qu’aux commer­çants ; mais comment et de quelle manière cela se faisait-il ? c’est ce que pas un d’eux ne savait bien. Les commerçants savaient parfaitement par quels moyens ce com­merce les enrichissait, c’était leur affaire de le savoir ; mais pour connaître comment et par quels moyens il enrichissait leur pays, c’est ce qui ne les regardait pas du tout ; et ils ne prirent jamais cet objet en considération, si ce n’est quand ils eurent besoin de recourir à la nation pour obtenir quelques changements dans les lois relatives au commerce étranger. Ce fut alors qu’il devint nécessaire de dire quelque chose sur les bons effets de ce commerce, et de faire voir comment son influence bienfaisante se trouvait contrariée par les lois telles qu’elles existaient alors. Les juges auxquels on avait affaire crurent que la question leur avait été présentée dans tout son jour quand on leur eut dit que le commerce étranger apportait de l’argent dans le pays, mais que les lois en question empêchaient qu’il n’en fit entrer autant qu’il aurait fait sans cela ; aussi ces arguments produisirent-ils l’effet qu’on en désirait. La prohibition d’exporter l’or et l’argent fut restreinte, en France et en Angleterre, aux monnaies du pays seule­ment ; l’exportation des lingots et monnaies étrangères fut laissée libre. En Hollande et dans quelques autres pays, la liberté d’exporter fut étendue même aux monnaies du pays. Les gouvernements, débarrassés tout à fait du soin de surveiller l’exportation de l’or et de l’argent, tournèrent toute leur attention vers la balance du commerce, comme sur la seule cause capable d’augmenter ou de diminuer dans le pays la quantité de ces métaux. Ils se délivrèrent d’un soin fort inutile, pour se charger d’un autre beaucoup plus compliqué, beaucoup plus embarrassant et tout aussi inutile. Le titre du livre de Mun, le Trésor de l’Angleterre dans le commerce étranger, devint une maxime fon­da­mentale d’économie politique, non-seulement pour l’Angleterre, mais pour tous les autres pays commerçants. Le commerce intérieur ou domestique, le plus important de tous, celui dans lequel le même capital fournit au pays le plus grand revenu et fait naître le plus d’occupation pour les nationaux, ne fut regardé que comme inférieur au commerce étranger. Ce commerce, disait-on, ne fait entrer ni sortir aucun argent du pays ; il ne peut donc rendre le pays ni plus riche ni plus pauvre, si ce n’est autant seulement que sa prospérité ou sa décadence pourrait avoir une influence indirecte sur l’état du commerce étranger.
Sans contredit, un pays qui n’a pas de mines doit tirer son or et son argent des pays étrangers, tout comme celui qui n’a pas de vignes est obligé de tirer ses vins de l’étranger. Cependant il ne paraît pas nécessaire que le gouvernement s’occupe plus d’un de ces objets qu’il ne s’occupe de l’autre. Un pays qui a de quoi acheter aura toujours tout le vin dont il aura besoin, et un pays qui aura de quoi acheter de l’or et de l’argent ne manquera jamais de ces métaux. On trouve à les acheter, pour leur prix, comme toute autre chose ; et s’ils servent de prix à toutes les autres marchandises, toutes les autres marchandises servent aussi de prix à l’or et à l’argent. Nous nous reposons en toute sûreté sur la liberté du commerce, sans que le gouvernement s’en mêle en aucune façon, pour nous procurer tout le vin dont nous avons besoin ; nous pouvons donc bien nous reposer sur elle, avec autant de confiance, pour nous faire avoir tout l’or et l’argent que nous sommes dans le cas d’acheter ou d’employer, soit pour la circulation de nos denrées, soit pour d’autres usages.
La quantité de chaque marchandise que l’industrie humaine peut produire ou acheter dans un pays s’y règle naturellement sur la demande effective qui s’en fait, ou sur la demande de ceux qui sont disposés à payer, pour l’avoir, toute la rente, tout le travail et tout le profit qu’il faut payer pour la préparer et la mettre au marché[2]. Mais aucune marchandise ne se règle plus aisément ou plus exactement sur cette demande effective que l’or et l’argent, parce que, vu le peu de volume de ces métaux en raison de leur valeur, il n’y a pas de marchandise qui se transporte plus facilement d’un lieu à un autre ; des lieux où ils sont à bas prix, à ceux où ils se vendent plus cher ; des lieux où ils excèdent la demande effective, aux lieux où ils sont au-dessous de cette demande. S’il y avait, par exemple en Angleterre, une demande effective pour une nou­velle quantité d’or, un paquebot pourrait apporter de Lisbonne, ou de toute autre part où l’on pourrait s’en procurer, une charge de cinquante tonneaux d’or, avec lequel on frapperait plus de cinq millions de guinées. Mais s’il y avait une demande effective de grains pour la même valeur, l’importation de ces grains, sur le pied de cinq guinées par tonneau, exigerait un million de tonneaux d’embarquement, ou bien mille bâti­ments du port de mille tonneaux chacun ; la marine d’Angleterre n’y pourrait pas suffire.
Quand la quantité d’or et d’argent importée dans un pays excède la demande effec­tive, toute la vigilance du gouvernement ne saurait en empêcher l’exportation. Toutes les lois sanguinaires de l’Espagne et du Portugal sont impuissantes pour retenir dans ces pays leur or et leur argent. Les importations continuelles du Pérou et du Brésil excèdent la demande effective de l’Espagne et du Portugal, et y font baisser le prix de ces métaux au-dessous de celui des pays voisins. Au contraire, si leur quantité dans un pays se trouve au-dessous de la demande effective, de manière à faire monter leur prix au-dessus de ce qu’il est dans les pays voisins, le gouvernement n’a pas besoin de se mettre en peine pour en faire importer ; il voudrait même empê­cher cette impor­ta­tion, qu’il ne pourrait pas y réussir. Quand les Spartiates eurent gagné de quoi acheter de ces métaux, l’or et l’argent surent bien se faire jour à travers toutes les barrières que les lois de Lycurgue opposaient à leur entrée dans Lacédé­mone. Toute la rigueur du code des douanes[3] ne saurait empêcher l’importation du thé des compagnies des Indes, de Hollande et de Gothembourg, parce que ce thé est un peu à meilleur marché que celui de la compagnie anglaise. Cependant, une livre de thé a environ cent fois au­tant de volume que le prix le plus cher qu’on en paye ordinairement en argent, qui est 16 schellings, et plus de deux mille fois le volume du même prix en or ; par conséquent, elle est tout autant de fois plus difficile à passer en fraude.
C’est en partie à cause de la facilité qu’il y a à transporter l’or et l’argent des endroits où ils abondent à ceux où ils manquent, que le prix de ces métaux n’est pas sujet à des fluctuations continuelles comme celui de la plupart des autres marchan­dises, qui, étant trop volumineuses, ne peuvent pas reprendre aisément leur équilibre quand il arrive que le marché en est dégarni ou en est surchargé. À la vérité, le prix de ces métaux n’est pas absolument exempt de variations ; mais les changements aux­quels il est sujet sont, en général, lents, successifs et uniformes. Par exemple on sup­pose, peut-être sans trop de fondement, qu’en Europe, pendant le cours de ce siècle et du précédent, ils ont été constamment, mais successivement, en baissant de valeur, à cause de l’importation continuelle qui s’en est faite des Indes occidentales espagnoles. Mais, pour produire dans le prix de l’or et de l’argent un changement tellement brus­que qu’il fasse hausser ou baisser à la fois, d’une manière sensible et remarquable, le prix pécuniaire de toutes les autres marchandises, il ne faut pas moins qu’une révolution pareille à celle qu’a causée dans le commerce la découverte de l’Amérique.
Si, malgré tout ceci, l’or et l’argent pouvaient une fois venir à manquer dans un pays qui aurait de quoi en acheter, ce pays trouverait plus d’expédients pour suppléer à ce défaut, qu’à celui de presque toute autre marchandise quelconque. Si les matières premières manquent aux manufactures, il faut que l’industrie s’arrête. Si les vivres viennent à manquer, il faut que le peuple meure de faim. Mais si c’est l’argent qui manque, on pourra y suppléer, quoique d’une manière fort incommode, par des trocs et des échanges en nature. On pourra y suppléer encore, et d’une manière moins incom­mode, en vendant et achetant sur crédit ou sur des comptes courants que les mar­chands balancent respectivement une fois par mois ou une fois par an. Enfin, un papier-monnaie bien réglé pourra en tenir lieu, non-seulement sans inconvénient, mais encore avec de grands avantages. Ainsi, sous tous les rapports, l’attention du gou­ver­ne­ment ne saurait jamais être plus mal employée que quand il s’occupe de surveiller la conservation ou l’augmentation de la quantité d’argent dans le pays.
Cependant, il n’y a rien dont on se plaigne plus communément que de la rareté de l’argent. L’argent, aussi bien que le vin, doit toujours être rare pour ceux qui n’ont ni de quoi acheter ni crédit pour emprunter. Ceux qui auront ou l’un ou l’autre, ne man­queront guère, soit d’argent, soit de vin, quand ils voudront s’en procurer. Cepen­dant ces plaintes sur la rareté de l’argent ne sont pas particulières seulement à d’imprudents dissipateurs ; elles sont quelquefois générales dans toute une ville de commerce et dans les pays environnants. La cause ordinaire en est dans la fureur qu’on a souvent d’entreprendre plus qu’on ne peut accomplir. Les gens les plus économes qui auront fait des spéculations disproportionnées à leurs capitaux, peuvent se trouver dans le cas de n’avoir ni de quoi acheter de l’argent, ni crédit pour en emprunter, tout aussi bien que des prodigues qui auront fait des dépenses disproportionnées à leurs reve­nus. Avant que leurs spéculations soient dans le cas de leur rapporter ce qu’ils y ont mis, tout leur capital a disparu avec leur crédit. Ils courent de tous les côtés pour emprunter de l’argent, et ils n’en peuvent trouver nulle part. Ces plaintes même générales sur la rareté de l’argent ne prouvent pas toujours qu’il ne circule pas dans le pays le nombre habituel de pièces d’or et d’argent, mais seulement que beaucoup de gens manquent de ces pièces, faute d’avoir rien à donner pour en acheter. Quand les profits du commerce viennent à être plus forts qu’à l’ordinaire, l’envie d’entreprendre au-delà de ses forces est une maladie qui gagne les gros commerçants comme les petits. Ce n’est pas qu’ils envoient toujours hors du pays une plus grande quantité d’argent qu’à l’ordinaire, mais ils font, tant au-dedans qu’au-dehors du pays, des achats à crédit pour plus de marchandises que de coutume, et envoient ces marchandises à des marchés éloignés, dans l’espoir que les retours leur rentreront avant les demandes de payement. Les demandes viennent avant que les retours soient arrivés, et ils n’ont rien sous la main qui puisse leur servir, ou à acheter de l’argent, ou à offrir comme sûreté pour en emprunter. Ce n’est pas la rareté de l’or ou de l’argent, mais c’est la difficulté que ces gens-là trouvent à emprunter, et celle que leurs créanciers trouvent à se faire payer, qui font dire à tout le monde que l’argent est rare[4].
Il serait vraiment trop ridicule de s’attacher sérieusement à prouver que la richesse ne consiste pas dans l’argent ou dans la quantité des métaux précieux, mais bien dans les choses qu’achète l’argent et dont il emprunte toute sa valeur, par la faculté qu’il a de les acheter. L’argent, sans contredit, fait toujours partie du capital national ; mais on a déjà fait voir qu’en général il n’en fait qu’une petite partie, et toujours la partie de ce capital qui profite le moins à la société[5].
Si le marchand trouve, en général, plus de facilité à acheter des marchandises avec de l’argent, qu’à acheter de l’argent avec des marchandises, ce n’est pas que la richesse consiste plus essentiellement dans l’argent que dans les marchandises ; c’est parce que l’argent est l’instrument reçu et établi dans le commerce, celui pour lequel toutes choses se donnent sur-le-champ en échange, mais qu’on ne peut pas toujours avoir aussi promptement en échange pour toute autre chose. D’ailleurs, la plupart des marchandises sont plus périssables que l’argent, et leur conservation peut souvent causer au marchand une plus grande perte. De plus, quand il a ses marchandises dans sa boutique, il est plus exposé à ce qu’il survienne des demandes d’argent auxquelles il ne pourra pas faire honneur, que quand il a dans sa caisse le prix de ses marchan­dises. Ajoutez encore à tout cela que son profit se fait plus immédiatement au mo­ment où il vend qu’au moment où il achète, et sous tous ces rapports il est beaucoup plus empressé, en général, de changer ses marchandises pour de l’argent, que son argent pour des marchandises. Mais quoiqu’un marchand, en particulier, puisse quel­quefois, avec une certaine abondance de marchandises en magasin, se trouver ruiné faute de pouvoir s’en défaire à temps, une nation ou un pays ne peut pas avoir un semblable accident à redouter. Souvent tout le capital d’un marchand consiste en marchandises périssables, destinées à faire de l’argent. Mais il n’y a qu’une bien petite partie du produit annuel des terres et du travail, dans un pays, qui puisse jamais être destinée à acheter de l’or et de l’argent des pays voisins. La très-grande partie est destinée à circuler et à se consommer dans le pays même, et encore, du superflu qui s’envoie au-dehors, la plus grande partie, en général, est destinée à acheter à l’étranger d’autres marchandises consommables. Ainsi, quand même on ne pourrait se procurer de l’or et de l’argent avec les marchandises qui sont destinées à en acheter, la nation ne serait pas ruinée pour ce motif[6]. Elle pourrait bien en souffrir quelque dommage et quelques incommodités, et se voir réduite à quelques-unes de ses ressources indis­pensables pour suppléer au défaut d’argent ; néanmoins, le produit annuel de ses terres et de son travail serait toujours le même ou à très-peu de chose près le même qu’à l’ordinaire, parce qu’il y aurait encore le même ou à très-peu de chose près le même capital consommable employé à entretenir ce produit. Et quoique la marchandise n’attire pas à elle l’argent toujours aussi vite que l’argent attire à soi la marchandise, à la longue elle l’attire à elle plus nécessairement encore qu’il ne le fait. La marchandise peut servir à beaucoup d’autres choses qu’à acheter de l’argent, mais l’argent ne peut servir à rien qu’à acheter la marchandise. Ainsi, l’argent court nécessairement après la marchandise, mais la marchandise ne court pas toujours ou ne court pas nécessaire­ment après l’argent. Celui qui achète ne le fait pas toujours dans la vue de revendre ; c’est souvent dans la vue d’user de la chose ou de la consommer ; tandis que celui qui vend le fait toujours en vue de racheter quelque chose. Le premier peut souvent avoir fait toute son affaire, mais l’autre ne peut jamais en avoir fait plus de la moitié. Ce n’est pas pour sa seule possession que les hommes désirent avoir de l’argent, mais c’est pour tout ce qu’ils peuvent acheter avec l’argent.
Les marchandises consommables, dit-on, sont bientôt détruites, tandis que l’or et l’argent sont d’une nature plus durable, et que, sans l’exportation continuelle qu’on en fait, ces métaux pourraient s’accumuler pendant plusieurs siècles de suite, de manière à augmenter incroyablement la richesse réelle d’un pays. En conséquence, on prétend en conclure qu’il ne peut y avoir rien de plus désavantageux pour un pays que le com­merce qui consiste à échanger une marchandise aussi durable contre des marchan­dises périssables. Cependant, nous n’imaginons pas de regarder comme un commerce désavantageux celui qui consiste à échanger la quincaillerie d’Angleterre contre les vins de France, quoique la quincaillerie soit une marchandise très-durable, et que, sans l’exportation continuelle qui s’en fait, elle puisse aussi s’accumuler pendant plusieurs siècles de suite, de manière à augmenter incroyablement les poêlons et les casseroles du pays. Mais s’il saute aux yeux que le nombre de ces ustensiles est, par tous pays, limité à l’usage qu’on en fait et au besoin qu’on en a ; qu’il serait absurde d’avoir plus de poêlons et de casseroles qu’il n’en faut pour faire cuire tout ce qui se consomme habituellement d’aliments dans ce pays ; et que si la quantité des aliments à consommer venait à augmenter, le nombre des poêlons et casseroles augmenterait tout de suite, parce qu’une partie de ce surcroît d’aliments serait employée à acheter de ces vases ou à entretenir un surcroît d’ouvriers dans les fabriques où ils se tra­vail­lent ; il devrait également sauter aux yeux que la quantité d’or ou d’argent est, par tous pays, limitée à l’usage qu’on fait de ces métaux et au besoin qu’on en a ; que leur usage consiste à faire, comme monnaie, circuler des marchandises, et à fournir, comme vais­selle, une espèce de meuble de ménage ; que, par tous pays, la quantité de mon­naie est déterminée par la valeur de la masse de marchandises qu’elle a à faire circu­ler ; que si vous augmentez cette valeur, tout aussitôt une partie de ce surcroît de valeur ira au-dehors chercher à acheter, partout où il pourra se trouver, le surcroît de monnaie qu’exige sa circulation ; qu’à l’égard de la quantité de vaisselle, elle est déterminée par le nombre et la richesse des familles particulières qui sont dans le cas de se donner ce genre de faste ; que si vous augmentez le nombre et la richesse de ces familles, alors très-vraisemblablement une partie de ce surcroît de richesse sera em­ployée à acheter, partout où elle en pourra trouver, un surcroît de vaisselle d’argent ; que prétendre augmenter la richesse d’un pays en y introduisant ou en y retenant une quantité inutile d’or et d’argent, est tout aussi absurde que prétendre augmenter, dans des familles particulières, la bonne chère de leur table, en les obligeant de garder chez elles un nombre inutile d’ustensiles de cuisine. De même que la dépense faite pour acheter ces ustensiles inutiles, loin d’augmenter la quantité ou la qualité des vivres de la famille, ne pourrait se faire sans prendre sur l’une ou sur l’autre, de même l’achat d’une quantité inutile d’or ou d’argent ne peut se faire, dans un pays, sans prendre nécessairement sur la masse de richesse qui nourrit, vêtit et loge le peuple, qui l’entre­tient et qui l’occupe. Il ne faut pas perdre de vue que, l’or et l’argent, sous quelque forme qu’ils soient, sous celle de monnaie ou de vaisselle, ne sont jamais que des ustensiles, tout aussi bien que les ustensiles de cuisine. Augmentez le service qu’ils ont à faire, augmentez la masse des marchandises qui doivent être mises en circula­tion par eux, disposées par eux, préparées par eux, et infailliblement vous verrez qu’ils augmenteront aussi de quantité ; mais si vous voulez essayer d’augmenter leur quantité par des moyens extraordinaires, alors tout aussi infailliblement vous dimi­nuez le nombre des services qu’ils ont à rendre et même leur quantité, la quantité de ces métaux ne pouvant jamais rester au-delà de ce qu’exige le service qu’ils ont à faire. Fussent-ils même déjà accumulés au-delà de cette quantité, leur transport se fait si facilement, ils coûtent tant à garder oisifs et sans emploi, qu’il n’y aura pas de loi capable d’empêcher qu’ils ne soient immédiatement envoyés au-dehors.
Il n’est pas toujours nécessaire d’accumuler de l’or et de l’argent dans un pays pour le mettre en état de soutenir des guerres étrangères, et d’entretenir des flottes et des armées dans les pays éloignés. On entretient des flottes et des armées avec des den­rées consommables, et non avec de l’or et de l’argent. Toute nation qui aura, dans le pro­duit annuel de son industrie domestique, dans le revenu annuel résultant de ses terres, de son travail et de son capital consommable, de quoi acheter dans des pays éloignés ces denrées consommables, pourra bien soutenir des guerres étrangères.
Une nation peut acheter de trois manières différentes la paye et les vivres d’une armée dans un pays éloigné : 1° en envoyant hors de chez elle une partie de l’or et de l’argent qu’elle a accumulés ; ou 2° en exportant une partie du produit de ces manufactures ; ou 3° enfin, en exportant une partie de son produit brut annuel.
Ce qui peut, à proprement parler, former l’approvisionnement d’un pays en or ou en argent, se compose de trois articles : l’argent de la circulation, la vaisselle des particuliers, et l’argent qui aura été amassé par plusieurs années d’économie et gardé dans le trésor du prince.
Il arrive rarement qu’on puisse beaucoup retrancher sur l’argent de la circulation, parce qu’il n’y a guère de superflu dans cet article. La valeur des marchandises qui sont vendues et achetées annuellement dans un pays exige une certaine quantité d’ar­gent pour les faire circuler et les distribuer dans les mains de leurs consomma­teurs, et elle ne peut pas en employer au-delà. Le canal de la circulation absorbe né­ces­sairement la somme d’argent propre à le remplir, et il ne peut en contenir davan­tage. Cependant, en général, on retire bien quelque chose de ce canal, en cas de guer­re étrangère. Le grand nombre de gens qu’on entretient au-dehors fait qu’il y en a moins à entretenir au-dedans ; il y a dès lors moins de denrées à faire circuler au-dedans, et il faut moins d’argent pour opérer cette circulation ; d’ailleurs, dans ces cas-là, on met communément en émission une quantité plus forte qu’à l’ordinaire de papier-monnaie, d’une espèce ou de l’autre, tels que sont en Angleterre les billets de l’Échiquier, les billets de la marine et les billets de banque, et ce papier, prenant la place de l’or et de l’argent de la circulation, fournit les moyens d’envoyer au-dehors une somme plus considérable de ces métaux. Tout ceci néanmoins n’offrirait qu’une bien pauvre ressource pour soutenir une guerre étrangère qui serait dispendieuse et qui durerait plusieurs années.
C’est encore une bien plus pauvre ressource, comme l’expérience l’a toujours fait voir, que de fondre la vaisselle des particuliers. Cet expédient fut employé par les Fran­çais au commencement de la dernière guerre, et le service qu’ils en tirèrent ne compensa pas même la perte de la façon.
Un trésor amassé dans les coffres du prince fournissait, dans les anciens temps, une ressource plus importante et plus durable. Dans ce siècle, si vous en exceptez le roi de Prusse, il ne paraît pas que l’idée d’amasser des trésors entre pour rien dans la politique des princes de l’Europe.
On ne voit pas qu’aucun de ces trois moyens, l’exportation de l’argent circulant, ou de la vaisselle des particuliers, ou du trésor du prince, ait beaucoup contribué à l’entretien des guerres étrangères faites dans ce siècle, les plus dispendieuses peut-être dont l’histoire fasse mention.
La dernière guerre de France coûte à la Grande-Bretagne au-delà de quatre-vingt-dix millions, en comptant non-seulement les soixante-quinze millions de dettes nou­velles qui ont été contractées, mais encore les deux schellings pour livre additionnels à la taxe foncière, et ce qui a été emprunté annuellement du fonds d’amortissement. Plus des deux tiers de cette dépense ont eu lieu dans des pays éloignés, en Allemagne, en Portugal, en Amérique, dans les ports de la Méditerranée, dans les Indes orientales et occidentales[7]. Les rois d’Angleterre n’avaient pas amassé de trésor ; nous n’avons pas entendu dire qu’il y ait eu aucune quantité extraordinaire d’argenterie mise au creuset. Quant à l’or et à l’argent de la circulation, on a pensé qu’ils n’avaient jamais excé­dé dix-huit millions ; néanmoins, d’après la dernière refonte de la monnaie d’or, il est à croire que ce calcul est fort au-dessous de la vérité[8] ; mais supposons, d’après le compte le plus exagéré que je puisse me rappeler en avoir vu ou entendu faire, que l’or et l’argent ensemble soient un objet de trente millions. En partant même de cette supposition, si c’eût été par le moyen de notre argent que nous eussions soutenu la guerre, il faudrait que la masse totale de nos espèces eût été exportée et rapportée au moins deux fois, dans une période d’environ six à sept ans. Si l’on pouvait admettre ce fait, ce serait l’argument le plus décisif pour démontrer toute l’inutilité des soins que prend le gouvernement en veillant à la conservation de l’argent, puisque, dans une telle hypothèse, la totalité de l’argent du royaume en serait sortie et rentrée à deux différentes fois, dans un espace de temps aussi court, sans que qui que ce soit en ait eu le moindre soupçon. Cependant, dans aucun moment de cette période, le canal de la circulation n’a paru plus vide que de coutume. L’argent ne manqua guère à tous ceux qui eurent de quoi le payer. À la vérité, les profits du commerce étranger furent plus forts qu’à l’ordinaire, pendant toute la guerre, mais surtout vers sa fin. Cette circonstance occasionna ce qu’elle occasionne toujours ; tous les commerçants, en géné­ral, entreprirent au-delà de leurs forces, dans tous les ports de la Grande-Bre­tagne ; ce qui fit naître encore ces plaintes ordinaires sur la rareté de l’argent, qui sont toujours une suite de ces entreprises immodérées. Beaucoup de gens manquèrent d’argent faute d’avoir de quoi en acheter, ou faute de crédit pour en emprunter ; et parce que les débiteurs trouvaient de la difficulté à emprunter, les créanciers en trou­vaient à se faire payer. Et pourtant, il y avait, en général, de l’or et de l’argent, moyen­nant leur valeur, pour tous les gens qui étaient en état de la donner.
Il faut donc que les dépenses énormes de la guerre dernière aient été principale­ment défrayées, non par l’exportation de l’or et de l’argent, mais par celle des marchandises anglaises d’une espèce ou d’une autre. Quand le gouvernement ou ses agents traitaient avec un négociant pour une remise à faire dans un pays étranger, ce négociant cherchait naturellement à payer son correspondant étranger sur lequel il avait donné une lettre de change, plutôt par un envoi de marchandises que par un envoi d’or et d’argent. Si les marchandises d’Angleterre n’étaient pas en demande dans ce pays étranger, il tâchait alors de les envoyer dans quelque autre pays étranger dans lequel il pût acheter une lettre de change sur le premier. Le transport des marchandi­ses, quand l’envoi se trouve bien assorti au marché où on les fait passer, est toujours accompagné d’un gros profit, tandis que celui de l’or et de l’argent n’en rend presque jamais aucun. Quand on envoie de ces métaux à l’étranger pour acheter des marchan­dises étrangères, le profit du marchand ne vient pas de l’achat, il vient de la vente des retours ; mais, quand ils vont à l’étranger pour payer une dette, le marchand n’a pas de retour ni, par conséquent, de profit. Naturellement donc il met toute son intelligence à trouver un moyen de payer ses dettes à l’étranger, plutôt par une exportation de mar­chan­dises que par une exportation d’or et d’argent. Aussi l’auteur de l’État présent de la nation[9] remarque-t-il la grande quantité de marchandises anglaises qui ont été exportées pendant le cours de la guerre dernière, sans rapporter aucuns retours[10]. 
Outre les trois articles ci-dessus, il y a encore, dans toutes les grandes nations commerçantes, une grande quantité d’or et d’argent en lingots qui est alternativement importée et exportée pour le service du commerce étranger. Ces lingots circulant parmi les différents peuples commerçants, tout comme la monnaie nationale circule dans chaque pays en particulier, on peut les regarder comme la monnaie de la grande république du commerce. La monnaie nationale reçoit son impulsion et sa direction des marchandises qui circulent dans l’enceinte de chaque pays en particulier ; la mon­naie de la république commerçante, de celles qui circulent entre pays différents. L’une et l’autre de ces monnaies sont employées à faciliter les échanges, l’une entre diffé­rents individus de la même nation, l’autre entre ceux de nations différentes. Une partie de cette monnaie de la grande république commerçante peut avoir été et a probable­ment été employée à soutenir la guerre dernière. Il est naturel de supposer que le moment d’une guerre générale lui imprime un mouvement et une direction différente de celle qu’elle a coutume de suivre dans le temps d’une profonde paix ; qu’elle circule davantage autour du centre de la guerre, et qu’elle y est employée en plus grande quantité pour y acheter, ainsi que dans les pays environnants, la paye et les vivres des différentes armées. Mais quelle que soit la portion de cette monnaie de la république commerçante que la Grande-Bretagne ait employée de cette manière, il faut toujours que cette portion ait été achetée, ou avec des marchandises anglaises, ou avec quelque autre chose achetée avec ces marchandises ; ce qui nous ramène toujours aux mar­chandises, au produit annuel des terres et du travail du pays, comme étant en dernier résultat les ressources qui nous ont mis en état de soutenir la guerre. En effet, il est naturel de supposer que, pour défrayer une dépense annuelle aussi forte, il a fallu un énorme produit annuel. La dépense de 1761, par exemple, a monté à plus de dix-neuf millions. Il n’y a pas d’accumulation qui eût pu supporter une aussi grande profusion ; il n’y a pas de produit annuel, même en or et en argent, capable de la couvrir. Tout l’or et l’argent qui s’importent annuellement en Espagne et en Portugal n’excèdent pas ordinairement, d’après les meilleures informations, six millions sterling ; ce qui, dans certaines années, aurait à peine défrayé quatre mois de la dépense de la dernière guerre.
De toutes les marchandises, les plus propres à être transportées dans des pays éloignés, soit pour y acheter la paye et les vivres d’une armée, soit pour y acheter une partie de cette monnaie de la république commerçante afin de l’employer à acheter cette paye et ces vivres, ce sont, à ce qu’il paraît, les articles manufacturés les mieux travaillés et les mieux finis. Ces produits, contenant une grande valeur sous un petit volume, peuvent dès lors être exportés à de très-grandes distances à peu de frais. Un pays qui produit annuellement par son industrie une grande quantité surabondante de ces sortes d’articles qu’il exporte habituellement en pays étrangers, peut soutenir pen­dant plusieurs années une guerre étrangère très-dispendieuse, sans exporter aucune quantité considérable d’or ou d’argent, sans en avoir même cette quantité à exporter. Dans ce cas, à la vérité, une partie très-considérable du superflu annuelle­ment produit par ses manufactures sera exportée sans rapporter aucuns retours au pays, bien qu’elle en rapporte au marchand, le gouvernement achetant au marchand ses lettres de chan­ge sur les pays étrangers, pour y solder la paye et les vivres de l’armée. Cependant, il peut se faire qu’une partie de ce superflu continue à rapporter des retours au pays. Pendant la guerre, les manufactures seront chargées d’une double demande, et on leur commandera d’abord de l’ouvrage pour être exporté, à l’effet de fournir au payement des lettres de change tirées sur les pays étrangers, et qui ont pour objet de solder la paye et les vivres de l’armée ; et en second lieu, l’ouvrage nécessaire pour acheter les retours ordinaires que le pays a coutume de consommer. Ainsi, au milieu de la guerre étrangère la plus désastreuse, il peut arriver fréquemment que la plupart des manu­factures parviennent à l’état le plus florissant, et qu’au contraire, au retour de la paix, elles viennent à déchoir. Elles peuvent prospérer au milieu de la ruine de leur pays, et commencer à dépérir au retour de sa prospérité. La différence de l’état de plusieurs branches des diverses manufactures d’Angleterre pendant le cour de la dernière guer­re, et de leur état quelque temps après la paix, peut bien servir comme un exemple frappant de ce que nous venons de dire.
Aucune guerre étrangère, ou longue, ou dispendieuse, ne peut facilement se soutenir par l’exportation du produit brut du sol. Il faudrait une trop grande dépense pour en envoyer à l’étranger une quantité qui pût suffire à acheter la paye et les vivres de l’armée. D’ailleurs, il y a peu de pays qui donnent beaucoup plus de produit brut qu’il n’en faut pour la subsistance de leurs habitants. Ainsi, en exporter une grande quantité, ce serait envoyer au-dehors une partie de la subsistance nécessaire du peu­ple. Il n’en est pas ainsi de l’exportation des produits manufacturés. La subsistance des gens employés à ces produits reste dans l’intérieur, et on n’exporte que la surabon­dance de leur travail. M. Hume[11] remarque fréquemment l’impuissance dans laquelle se trouvaient anciennement les rois d’Angleterre de soutenir sans interruption une guerre étrangère un peu longue. Dans ces temps-là, les Anglais n’avaient rien pour acheter dans des pays étrangers la paye et les vivres de leurs armées, si ce n’est le produit brut de leur sol, dont on ne pouvait pas retrancher une grande portion sur la consommation intérieure, ou bien quelque peu d’ouvrages de fabrique de l’espèce la plus grossière, et dont le transport, comme celui du produit brut, eût été trop dispendieux. Cette impuissance ne venait pas du défaut d’argent, mais du défaut de produits mieux travaillés et plus finis. Les transactions du commerce se faisaient en Angleterre, alors tout comme aujourd’hui, avec de l’argent. Il fallait bien que la quantité d’argent en circulation fût proportionnée au nombre et à la valeur des achats et des ventes qui se consommaient habituellement dans ces temps-là, comme aujourd’hui elle l’est aux achats et ventes qui se font ; ou plutôt même, il fallait qu’elle fût à proportion beau­coup plus grande, parce que nous n’avions pas alors le papier qui fait aujourd’hui une grande partie du service de l’or et de l’argent. Chez les peuples qui ont peu de com­merce et de manufactures, le souverain ne peut guère, dans les cas extraor­dinaires, tirer de ses sujets aucun secours considérable, par des raisons que j’expliquerai dans la suite[12]. Aussi est-ce dans ces pays qu’en général il tâche d’amasser un trésor, comme la seule ressource qu’il ait pour de pareilles circonstances. Indépendamment de cette nécessité, il est dans une situation qui le dispose naturellement à l’économie. Dans cet état de simplicité, la dépense même du souverain n’est pas dirigée par cette vanité frivole qui recherche le faste et l’étalage ; mais cette dépense consiste toute en bien­faits à ses vassaux, et en hospitalité envers les gens de sa suite. Or, la bienfai­sance et l’hospitalité ne conduisent guère à faire des folies, tandis que la vanité y mène pres­que toujours. Aussi, chaque chef tartare a-t-il un trésor. On dit que Mazeppa, chef des Cosaques dans l’Ukraine, ce fameux allié de Charles XII, avait d’immenses trésors. Tous les rois francs de la première race avaient des trésors ; quand ils parta­geaient leur royaume entre leurs enfants, ils partageaient aussi le trésor. Il paraît que nos princes saxons et les premiers rois après la conquête avaient un trésor accumulé de la même manière. Le premier acte de chaque nouveau règne était ordinairement de s’em­parer du trésor du roi précédent, comme la mesure la plus essentielle pour s’assurer la succession au trône. Les souverains des pays commerçants et industrieux ne sont pas de même dans la nécessité d’amasser des trésors, parce qu’en général, dans les cas extraordinaires, ils peuvent tirer de leurs sujets des secours extraordinaires. Ils sont aussi moins disposés à accumuler. Naturellement, et peut-être par nécessité, ils suivent les mœurs du temps, et leur dépense vient à se régler aussi sur cet esprit de vanité puérile qui prési­de à celle de tous les autres grands propriétaires de leur royaume. L’étalage frivole de leur cour devient de jour en jour plus brillant, et la dépense qu’entraîne ce vain faste non-seulement empêche qu’ils puissent amasser, mais encore bien souvent elle prend sur des fonds destinés à des dépenses nécessaires. On pourrait appliquer à la cour de plusieurs princes de l’Europe ce que Dercyllidas dit de celle du roi de Perse : qu’il avait vu beaucoup d’éclat, mais peu de force ; un grand nombre de serviteurs, mais peu de soldats.
L’importation de l’or et de l’argent n’est pas le principal bénéfice, et encore bien moins le seul qu’une nation retire de son commerce étranger. Quels que soient les pays entre lesquels s’établit un tel commerce, il procure à chacun de ces pays deux avantages distincts. Il emporte ce superflu du produit de leur terre et de leur travail pour lequel il n’y a pas de demande chez eux, et à la place il rapporte en retour quelque autre chose qui y est demandé. Il donne une valeur à ce qui leur est inutile, en l’échangeant contre quelque autre chose qui peut satisfaire une partie de leurs besoins ou ajouter à leurs jouissances. Par lui, les bornes étroites du marché intérieur n’empêchent plus que la division du travail soit portée au plus haut point de perfec­tion, dans toutes les branches particulières de l’art ou des manufactures. En ouvrant un marché plus étendu pour tout le produit du travail qui excède la consom­mation intérieure, il encourage la société à perfectionner le travail, à en augmenter la puis­sance productive, à en grossir le produit annuel, et à multiplier par là les richesses et le revenu national. Tels sont les grands et importants services que le commerce étranger est sans cesse occupé à rendre, et qu’il rend à tous les différents pays entre lesquels il est établi. Il produit de grands avantages pour tous ces pays, quoique ce­pen­dant le pays de la résidence du marchand en retire encore de plus grands en général que les autres[13], parce que naturellement ce marchand s’occupe davantage de fournir aux besoins de son propre pays et d’en exporter les produits superflus, qu’il ne s’occupe de ceux de tout autre pays. L’importation de l’or et de l’argent dont on peut avoir besoin dans les pays qui n’ont pas de mines, est sans contredit aussi un des articles dont s’occupe le commerce étranger. Cependant, c’est un des moins impor­tants de tous ; un pays qui n’aurait d’autre commerce étranger que celui-là, aurait à peine occasion d’équiper un vaisseau dans tout un siècle.
Ce n’est pas par l’importation de l’or et de l’argent que la découverte de l’Amérique a enrichi l’Europe. L’abondance des mines de l’Amérique a produit ces métaux à meilleur marché. On peut se procurer maintenant un service de vaisselle pour le tiers du blé ou le tiers du travail qu’il aurait coûté au quinzième siècle. Avec la même dépense annuelle en travail et en marchandises, l’Europe peut acheter annuellement environ trois fois plus d’argenterie qu’elle n’en aurait acheté alors. Mais, quand une marchandise vient à se vendre au tiers de ce qu’était son prix ordinaire, non-seulement ceux qui l’achetaient auparavant peuvent en acheter trois fois autant qu’ils en achetaient, mais encore elle se trouve être descendue à la portée d’un beaucoup plus grand nombre d’acheteurs, d’un nombre dix fois, vingt fois peut-être et davantage plus fort que le premier. De manière qu’il y a peut-être actuellement en Europe, non-seule­ment plus de trois fois, mais même plus de vingt ou trente fois autant d’orfèvrerie qu’il y en aurait eu, même dans l’état actuel de son industrie, si la découverte des mines d’Amérique n’eût pas eu lieu. jusque-là, l’Europe a sans doute acquis une véri­table commodité de plus, quoique assurément d’un genre très-futile. Mais aussi le bon marché de l’or et de l’argent rend ces métaux bien moins propres qu’auparavant à remplir les fonctions de monnaie. Pour faire les mêmes achats, il faut nous charger d’une bien plus grande quantité de ces métaux, et il faut porter avec nous dans notre poche 1 schelling, là où une pièce de 4 pence nous eût suffi auparavant. Il serait assez difficile de décider qui l’emporte de ce léger inconvénient ou de cette futile com­modité ; ni l’un ni l’autre n’auraient pu apporter de changement bien important dans l’état de l’Europe, et cependant la découverte de l’Amérique en a produit un de la plus grande importance. En ouvrant à toutes les marchandises de l’Europe un nouveau marché presque inépuisable, elle a donné naissance à de nouvelles divisions de tra­vail, à de nouveaux perfectionnements de l’industrie, qui n’auraient jamais pu avoir lieu dans le cercle étroit où le commerce était anciennement resserré, cercle qui ne leur offrait pas de marché suffisant pour la plus grande partie de leur produit. Le travail se perfectionna, sa puissance productive augmenta, son produit s’accrut dans tous les divers pays de l’Europe, et en même temps s’accrurent avec lui la richesse et le revenu réel des habitants. Les marchandises de l’Europe étaient pour l’Amérique presque autant de nouveautés, et plusieurs de celles de l’Amérique étaient aussi des objets nouveaux pour l’Europe. On commença donc à établir une nouvelle classe d’échanges auxquels on n’avait jamais songé auparavant, et qui naturellement auraient dû être pour le nouveau continent une source de biens aussi féconde pour que l’an­cien. Mais la barbarie et l’injustice des Européens firent d’un événement, qui eût dû être avantageux aux deux mondes, une époque de destruction et de calamité pour plusieurs de ces malheureuses contrées.
La découverte d’un passage aux Indes orientales par le cap de Bonne-Espérance, qui eut lieu presque à la même époque, ouvrit peut-être au commerce étranger un champ plus vaste encore que celle de l’Amérique, malgré le plus grand éloignement de ces pays. Il n’y avait en Amérique que deux nations qui fussent, à quelques égards, supérieures aux sauvages, et elles furent détruites presque aussitôt que découvertes. Le reste était tout à fait sauvage. Mais les empires de la Chine, de l’Indostan, du Japon, ainsi que plusieurs autres dans les Indes orientales, sans avoir des mines plus riches en or et en argent, étaient, sous tous les rapports, beaucoup plus opulents, mieux cultivés et plus avancés dans tous les genres d’arts et de manufactures, que les empires du Mexique ou du Pérou, quand même nous voudrions ajouter foi à ce qui réellement n’en mérite guère, aux récits exagérés des Espagnols qui ont écrit sur l’état de ces empires. Or, des nations riches et civilisées peuvent toujours faire entre elles des échanges pour de bien plus grandes valeurs qu’elles ne peuvent en faire avec des peuples sauvages et barbares. Cependant, jusqu’à présent l’Europe a retiré bien moins d’avantages de son commerce des Indes orientales, que de celui de l’Amérique. Les Portugais s’approprièrent le monopole du commerce des Indes pendant près d’un siècle, et ce ne fut qu’indirectement et par leur canal que les autres nations de l’Europe purent y envoyer ou en recevoir des marchandises. Lorsqu’au commencement du dernier siècle les Hollandais commencèrent à leur arracher une partie de ce mono­pole, ces nouveaux conquérants investirent une compagnie exclusive de tout leur com­mer­ce aux Indes. Cet exemple a été suivi par les Anglais, les Français, les Sué­dois et les Danois, de manière qu’il n’y a pas de grande nation en Europe qui ait encore joui de la liberté du commerce des Indes orientales. Il ne faut pas chercher d’autre raison pour expliquer pourquoi ce commerce n’a jamais été aussi avantageux que celui d’Amérique, qui est toujours demeuré libre à tous les sujets avec leurs propres colonies, dans presque toutes les nations de l’Europe. Les privilèges exclusifs de ces compagnies des Indes, leurs grandes richesses, la faveur et la protection que ces richesses leur ont values auprès de leurs gouvernements respectifs, ont excité contre elles de grandes jalousies[14]. L’envie a souvent représenté leur commerce comme absolument pernicieux, sous le rapport des énormes sommes d’argent qu’il exporte chaque année du pays où il est établi. Les parties intéressées répondaient à cette objection, qu’il se pouvait bien, à la vérité, que leur commerce tendît, par cette conti­nuelle exportation d’argent, à appauvrir l’Europe en général, mais nullement le pays particulier qui faisait ce commerce, parce que, par l’exportation d’une partie des retours aux autres pays de l’Europe, il rentrait annuellement une bien plus grande quantité de ce métal qu’il n’en était sorti. L’objection et la réponse sont fondées l’une et l’autre sur cette idée populaire que j’ai discutée dans ce chapitre ; il est donc inutile d’y revenir davantage. L’exportation annuelle d’argent dans l’Inde fait vraisembla­ble­ment que la vaisselle est un peu plus chère en Europe qu’elle ne le serait sans cela, et que chaque pièce d’argent monnayé sert à acheter une plus grande quantité de travail et de marchandises. Le premier de ces deux effets est un bien petit mal ; l’autre est un bien léger avantage ; l’un et l’autre sont trop peu importants pour mériter en aucune façon l’attention publique. Le commerce de l’Inde, en ouvrant un marché aux mar­chandises de l’Europe, ou ce qui revient à peu près au même, à l’or et à l’argent que ces marchandises achètent, doit tendre nécessairement à augmenter la production annuelle des marchandises de l’Europe et, par conséquent, la richesse et le revenu réel de cette partie du monde. Si jusqu’à présent il a causé si peu d’augmentation dans ce produit annuel, il faut vraisemblablement l’attribuer aux entraves dont on a partout accablé ce commerce[15].
J’ai cru nécessaire, au risque même d’être trop long, d’examiner dans tous ses détails cette idée populaire, que la richesse consiste dans l’argent ou dans l’abondance des métaux précieux. Dans le langage vulgaire, comme je l’ai observé, argent veut souvent dire richesse, et cette ambiguïté d’expression nous a rendu cette idée popu­laire tellement familière, que ceux même qui sont convaincus de sa fausseté sont à tout moment sur le point d’oublier leur principe et, entraînés dans leurs raisonne­ments, à prendre ce préjugé pour une idée reçue et reconnue comme une vérité cer­taine et incontestable. Quelques-uns des meilleurs auteurs anglais qui ont écrit sur le commerce partent d’abord de ce principe, que la richesse d’un pays ne consiste pas uniquement dans son or et son argent, mais qu’elle consiste dans ses terres, ses mai­sons et ses biens consommables de toutes sortes. Néanmoins, dans la suite de leurs discussions, il semble que les terres, les maisons et les biens consommables leur sortent de la mémoire, et la nature de leurs arguments parait souvent supposer qu’ils font consister la richesse dans l’or et dans l’argent, et qu’ils regardent la multiplication de ces métaux comme l’objet capital de l’industrie et du commerce national.
Toutefois, ces deux principes une fois posés, que la richesse consistait dans l’or dans l’or et dans l’argent, et que ces métaux ne pouvaient être apportés dans un pays qui n’a point de mines que par la balance du commerce seulement, ou bien par des exportations qui excédaient en valeur les importations, alors nécessairement ce qui devint l’objet capital de l’Économie politique, ce fut de diminuer autant que possible l’Importation des marchandises étrangères pour la consommation intérieure, et d’augmenter autant que possible l’Exportation des produits de l’industrie nationale. En conséquence, les deux grands ressorts qu’elle mit en œuvre pour enrichir le pays, ce furent les entraves à l’importation et les encouragements pour l’exportation.
Les entraves à l’importation furent de deux sortes :
Premièrement, les entraves à l’importation des marchandises étrangères pour la consommation intérieure, lorsqu’elles étaient de nature à pouvoir être produites dans le pays, et quel que fût le pays d’où elles seraient importées ;
Secondement, les entraves à l’importation de presque toutes les espèces de mar­chan­dises venant des pays avec lesquels on supposait la balance du commerce défa­vorable.
Ces différentes sortes d’entraves consistèrent quelquefois en droits élevés, quel­quefois en des prohibitions absolues.
L’exportation fut encouragée, tantôt par des restitutions[16] de droits, tantôt par des primes[17], tantôt par des traités de commerce avantageux avec des nations étrangères, et tantôt par des établissements de colonies dans des contrées éloignées.
Les restitutions de droits furent accordées en deux occasions différentes ; quand les ouvrages de fabrique nationale étaient assujettis à un droit ou accise[18], on rendit souvent tout ou partie du droit, lors de leur exportation, et quand des marchandises étrangères, sujettes à un droit, étaient importées dans la vue d’être réexportées, alors on rendit quelquefois tout ou partie du droit au moment de la réexportation. 
Les primes furent accordées pour encourager, ou quelque genre de manufacture naissant, ou une espèce d’industrie quelconque qu’on jugeait mériter une faveur particulière.
Par des traités de commerce favorables, on procura chez quelque nation étrangère, aux marchands et aux marchandises de son pays, des privilèges particuliers et d’autres conditions que celles qu’y pouvaient obtenir les marchands des autres pays.
Enfin, par l’établissement des colonies dans des contrées éloignées, on fit obtenir aux marchands et aux marchandises de son pays non-seulement des privilèges parti­culiers, mais souvent même un monopole.
Les deux sortes d’entraves à l’importation qui sont indiquées ci-dessus, ainsi que ces quatre espèces d’encouragements pour l’exportation, constituent les six moyens principaux par lesquels le système du commerce se propose d’augmenter dans le pays la quantité de l’or et de l’argent, en faisant tourner la balance à l’avantage de ce pays.
J’examinerai chacun de ces moyens dans un chapitre particulier, et sans m’occuper davantage de leur prétendue tendance à faire entrer de l’argent dans le pays, je cher­cherai principalement quels sont les effets qu’on peut attendre de chacun d’eux sur le produit annuel de l’industrie nationale. Selon qu’ils tendent à augmenter ou à dimi­nuer la valeur de ce produit annuel, ils doivent tendre évidemment d’autant à augmen­ter ou à diminuer la richesse et le revenu réel du pays.
 
 
 
↑
L’argent a la même valeur intrinsèque dans tous les pays ; mais cette marchandise, comme toutes les autres, acquiert un surcroit de valeur lorsqu’elle est transportée d’un lieu où elle était moins utile dans un lieu où elle le sera davantage. Les blés de Picardie destinés à la consommation de Paris, quand ils sont parvenus aux portes de cette ville, ont acquis une valeur additionnelle par les frais de transport sans lesquels cette denrée n’aurait pas eu toute l’utilité qui en a déterminé la production. Les frais et risques du transport de l’argent sont la seule cause qui fait varier le cours du change, et le résultat des transactions faites entre deux places est la circonstance qui rend ce transport plus ou moins utile.
Quand les dettes et créances respectives du commerce de deux nations qui font des affaires l’une avec l’autre se balancent de telle manière qu’il n’y aura pas nécessité de transporter de l’argent d’un pays dans l’autre pour solder le compte définitif, alors il est indifférent pour un commerçant d’avoir de l’argent dans l’un ou dans l’autre pays : le transport d’espèces n’a point d’utilité, et le change est au pair.
Ce pair du change s’exprime en traduisant simplement d’une langue dans l’autre la valeur nominale d’une quantité déterminée d’argent. Si un poids de 4 onces d’argent au titre ordinaire de la monnaie se nomme en France, quand il est monnayé, 25 francs, et que ce même poids se nomme dans la monnaie d’Angleterre une livre sterling, le change sera au pair entre les deux nations lorsque la livre sterling se fera sur la place de Londres au prix de 35 francs, ou que 25 francs achèteront à la bourse de Paris une lettre de change au moyen de laquelle le porteur «e fera payer en Angleterre une livre sterling.
Mais si, toutes compensations faites, les marchands de Londres sont dans la nécessité de faire transporter de l’argent en France, ils rechercheront les lettres de change sur Paris, puisque ces lettres les mettront en possession d’une somme d’argent toute transportée et leur épargneront les frais et embarras du transport. De l’argent à Paris leur est dans ce cas plus utile que de l’argent à Londres ; il a la valeur additionnelle résultant du transport effectué. Ils achèteront peut-être jusqu’au prix de 21 schellings une lettre de change de 35 francs sur Paris, et par conséquent 35 francs achèteront sur la place de Paris une lettre de change de 31 schellings payables à Londres. Alors le change sera de 5 pour 100 contre l’Angleterre en faveur de la France.
Supposons que la balance de doit et d’avoir étant égale entre ces deux pays, et par conséquent l’argent ayant autant de valeur sur une place que sur l’autre, il se soit en même temps opéré une espèce de révolution dans le langage approprié à la monnaie anglaise, que, par une altération dans les termes, la même expression n’ait plus le même sens et ne représente plus la même chose ; si les Anglais, au lieu d’énoncer, comme par, le passé, par ce mot de une livre sterling, un poids de 4 onces d’argent de notre poids de marc, entendent par ce même mot une promesse ou obligation plus ou moins solide, plus ou moins facile à réaliser, contractée par une association de banquiers de payer au porteur cette livre sterling, alors le commerce des lettres de change entre Paris et Londres ne se réglera plus sur les principes du change ni d’après le plus ou le moins d’utilité du transport des espèces d’un lieu dans l’autre. Ce sera un contrat ou marché d’une tout autre nature ; ce ne sera plus un échange d’argent contre argent, avec addition ou retenue de la somme équivalente aux frais et risques du transport des espèces. Ce contrat-ci est devenu une convention purement aléatoire, dont les conditions dépendent du plus eu moins de probabilité de la réalisation de la promesse, du plus ou moins de confiance qu’inspire le débiteur, du plus ou moins d’espoir de placer la promesse avec facilité et sans perte. Lorsque, par suite d’une émission de papier-monnaie hors de toute mesure, la livre sterling, il y a quelques années, se négociait à Paris au prix de 18 francs 75 centimes, si l’on suppose qu’il y eut balance dans les comptes de commerce respectifs entre les deux pays, alors les risques du non-payement de la promesse ou de la perte à faire pour la réaliser contre argent ou marchandises, étaient évalués à 25 pour 100. Celui qui prenait une lettre de change sur Londres et qui la payait avec des écus français savait que cette lettre de change n’était payable qu’en billets de la banque dépréciés par leur excessive surabondance, et qu’il n’estimait valoir que les trois quarts seulement de leur valeur nominale ou fictive. Quoique les gens de commerce donnassent à cet agio le nom de change, néanmoins le cours auquel se négociaient alors à Paris les traites sur l’Angleterre ne pouvait être regardé comme une indication de l’état du change entre les deux nations. Au temps même où se faisaient ces marchés, il se peut très-bien que, par le résultat des affaires respectives de commerce, le change réel fût en faveur de l’Angleterre, et que le prix de ce change, en élevant de quelque chose en France la valeur du papier-monnaie anglais, ait prévenu une plus forte dépréciation des traites sur l’Angleterre vendues à la bourse de Paris.
Si la circulation monétaire de l’Angleterre, au lieu d’être en papier de banque, était en une monnaie altérée, rognée ou usée de 25 pour 100, en sorte que la livre sterling, au lieu de contenir 4 de nos onces, n’en contint plus que 3, cette livre sterling se vendrait 18 francs 75 centimes sur la place de Paris lorsque le change serait au pair entre les deux pays, parce que 18 francs 75 centimes formeraient alors le même poids d’argent que la livre sterling. Ou ne pourrait pas dire dans ce cas, comme l’a fait M. Ricardo, que le change fût au désavantage de l’Angleterre de 25 pour 100. Supposez que le cours de la livre sterling, pendant cette circulation de mauvaises espèces, fût a 20 francs, il faudra reconnaitre que le change réel est de 5 pour 100 eu faveur de l’Angleterre, puisque 20 francs contiendraient un vingtième d’argent de plus que la livre sterling de cette monnaie rognée.
Mais M. Ricardo, en poursuivant son raisonnement sur cette matière, suppose un concours de circonstances impossible et composé de faits qui s’excluent les uns les autres. 11 suppose qu’il y ait en Angleterre, dans la circulation, plus d’argent que n’en comportent les besoins de cette circulation, et qu’en même temps une loi prohibe d’une manière efficace l’exportation de cet argent superflu. Il pense que, dans un tel état de choses, le prix de toutes les marchandises hausserait dans le pays, et que le change serait, dans la même proportion, défavorable à l’Angleterre.
D’abord, s’il est une maxime évidente en économie politique, c’est assurément celle établie par Smith, que la circulation ne peut retenir une quantité d’argent plus forte que celle qui est nécessaire à son service. Si une loi défendait l’exportation de cet urgent surabondant, et qu’on n’eût aucun moyen d’éluder la défense, les personnes qui posséderaient cette quantité d’argent rejetée par la circulation, comme surcharge inutile, ne voudraient pas pour cela, sans doute, donner, leur argent pour moins que sa valeur, et plutôt que d’y perdre, elles le feraient convertir en ouvrages d’orfèvrerie. Toute importation d’argent du dehors s’arrêterait nécessairement, et attendu que ce métal s’use et se consomme comme toute autre chose, on en reviendrait avec le temps à l’état naturel, et il arriverait tôt ou tard un moment où l’Angleterre n’aurait plus, tant en monnaie qu’en vaisselle, que la quantité d’argent nécessaire à sa consommation dans ces deux genres.
Quoi qu’il en soit, l’hypothèse ne peut avoir aucun rapport avec la question du change. L’état du change dépend de la quantité d’affaires qui se font entre deux pays, mais nullement du prix en argent des marchandises dans l’un ou dans l’autre de ces pays. Admettons avec M. Ricardo que les prix en argent de toutes choses viennent à hausser en Angleterre de 10 pour 100 par une cause quelconque, cette circonstance ne changera rien aux affaires faites avec l’étranger. Si un marchand de Londres est dans l’usage de faire passer à Lisbonne dix pièces de toile pour lesquelles il retire un tonneau de vin de Portugal, il lui importe fort peu de payer ses toiles un dixième de plus en argent si le vin qu’il ramène en Angleterre doit hausser de prix dans la même proportion. Son gain ne diminuera point, et sou compte à solder avec Lisbonne ne donnera pas lieu pour cela à la sortie d’un schelling de plus qu’auparavant. Garnier.
↑ Livre I, chap. vii.
↑ Les droits de douane en Angleterre n’ont lieu qu’à l’entrée et à la sortie du royaume : ils répondent à ce qu’on nommait en France, traites foraines.
↑ La circulation du papier est une des principales causes des excès du commerce, parce qu’elle permet aux marchands de disposer par l’emprunt d’un capital presque sans limites. Buchanan.
↑ Liv. II, chap. ii.
↑ Mais toujours la nation ne dût pas être ruinée, un manque d’espèces temporaire frapperait toujours le commerce d’un coup assez rude pour entraîner des désastres sérieux et irréparables. Buchanan.
↑ D’après un compte soumis au Parlement, il parait que l’argent dépensé pendant cette guerre, sur le continent seul de l’Europe, s’éleva à 20,625,997 livres sterling (plus de 500 millions de francs). Buchanan.
↑ Lord Liverpool estime à plus de 20 millions sterling le nombre de guinées rapportées à la Monnaie pour y être refondues, indépendamment de près de 6 millions qu’on suppose être restés dans la circulation. Buchanan.
↑ Commencé par Miège, et continué par Bolton.
↑ Il est évident que des envois considérables de subsides à l’étranger ne peuvent s’effectuer par une exportation d’argent monnayé ; il est constaté d’ailleurs par des rapports de douanes, qu’une grande partie des dépenses extérieures de ce pays pendant la dernière guerre fut défrayée par l’exportation de marchandises. A partir de l’année 1793, jusqu’en 1797, des traites pour le payement des troupes furent expédiées pour le continent de l’Europe, des subsides considérables furent envoyés à l’empereur d’Autriche et à d’autres princes d’Allemagne. On s’était toujours procuré les fonds nécessaires par des exportations de marchandises et d’espèces. Les envois pour l’Allemagne seule, par exemple, qui pendant la paix avaient été d’environ 1,900,000 livres sterling (47,500,000 fr.), s’élevèrent pendant les années 1795 et 1796, époque où des remises furent envoyées en Autriche, à plus de 8,000,000 livres sterling (200,000,000 fr.). Le prêt accordé à l’empereur en 1793 s’éleva à 4,600,000 livres sterling (115,000,000 fr.), et M. Boyd, qui avait été chargé de la remise de cette somme, rapporte qu’une partie, s’élevant à 1,200,000 livres sterling (30,000,000 fr.), avait été faite en monnaie étrangère et en lingots ; le reste fut effectué par des envois de traites. Il fallait nécessairement varier le mode de la remise selon l’état du change. Des lettres de change furent achetées selon les circonstances, sur Madrid, Cadix, Lisbonne, de préférence à des envois de lingots ou remises directes sur Hambourg.

M. Boyd, dans les explications données au comité secret de la Chambre des lords, en 1797, dit : « C’est en ne demandant à aucun des moyens de remise rien au delà de ce qu’on pouvait raisonnablement en attendre, que nous sommes parvenus à mener à bonne fin une opération aussi importante, sans amener des variations notables dans le cours du change. L’effet naturel d’aussi grandes dépenses extérieures doit être de créer à l’étranger des demandes d’argent très-considérables, et d’exposer par un contre-coup naturel la banque d’Angleterre à de fortes demandes d’espèces, puisque l’exportation en devient alors très-avantageuse. C’est par ces circonstances qu’il sera facile de justifier la suspension par la banque d’Angleterre de ses payements en espèces. Buchanan.
↑ Voyez son Histoire d’Angleterre.
↑ Liv. V, chap. iii.
↑ Smith oublie le principal produit du commerce, c’est-à-dire l’augmentation de valeur donnée à la marchandise indigène en l’exportant, et à la marchandise étrangère en l’important. (Note inédite de J. B. Say.)
↑ Voyez ci-après l’Histoire de la Compagnie des Indes anglaises, liv. V, chap. i, sect. iii.
↑ Le monopole du commerce des Indes accordé à une compagnie privilégiée doit nécessairement empêcher son développement ; mais quand même ces restrictions auraient cessé, la grande distance entre les deux pays rendrait toujours les relations commerciales très-difficiles*. Buchanan.
↑ On restitue, lors de l’exportation de la marchandise, une partie des droits qu’elle a payés, ou dans l’intérieur, ou aux douanes à son entrée dans le royaume.
↑ C’est ce qu’on nomme aussi quelquefois primes d’encouragement. La plupart des auteurs, et notamment ceux de l’Encyclopédie, ont employé le mot de gratification.
↑ L’accise comprend tous les impôts qui se lèvent dans l’intérieur sur les denrées de consommation, tels qu’étaient, en France, les aides, les gabelles, etc.
*. Ce commerce est aujourd’hui libre et la distance diminuée de plus de moitié par l’isthme de Suez. A. B.

%%%%%%%%%%%%%%%%%%%%%%%%%%%%%%%%%%%%%%%%%%%%%%%%%%%%%%%%%%%%%%%%%%%%%%%%%%%%%%%%
%                                  Chapitre 2                                  %
%%%%%%%%%%%%%%%%%%%%%%%%%%%%%%%%%%%%%%%%%%%%%%%%%%%%%%%%%%%%%%%%%%%%%%%%%%%%%%%%

\chapter{Des entraves à l’importation seulement des marchandises étrangères qui sont de nature à être produites par l’industrie}
\markboth{Des entraves à l’importation seulement des marchandises étrangères qui sont de nature à être produites par l’industrie}{}

En gênant, par de forts droits ou par une prohibition absolue, l’importation de ces sortes de marchandises qui peuvent être produites dans le pays, on assure plus ou moins à l’industrie nationale qui s’emploie à les produire, un monopole dans le mar­ché intérieur.
Ainsi, la prohibition d’importer ou du bétail en vie, ou des viandes salées de l’étranger, assure aux nourrisseurs de bestiaux, en Angleterre, le monopole du marché intérieur pour la viande de boucherie. Les droits élevés mis sur l’importation du blé, lesquels, dans les temps d’une abondance moyenne, équivalent à une prohibition, donnent un pareil avantage aux producteurs de cette denrée. La prohibition d’importer des lainages étrangers est également favorable à nos fabricants de lainages. La fabri­que de soieries, quoiqu’elle travaille sur des matières tirées de l’étranger, vient d’obte­nir dernièrement le même avantage[1]. Les manufactures de toiles ne l’ont pas encore obtenu, mais elles font de grands efforts pour y arriver. Beaucoup d’autres classes de fabricants ont obtenu de la même manière, dans la Grande-Bretagne, un monopole complet, ou à peu près, au détriment de leurs compatriotes. La multitude de mar­chandises diverses dont l’importation en Angleterre est prohibée, d’une manière absolue, ou avec des modifications, est fort au-delà de tout ce que pourraient s’ima­giner ceux qui ne sont pas bien au fait des règlements de douanes.
Il n’y a pas de doute que ce monopole dans le marché intérieur ne donne souvent un grand encouragement à l’espèce particulière d’industrie qui en jouit, et que souvent il ne tourne vers ce genre d’emploi une portion du travail et des capitaux du pays, plus grande que celle qui y aurait été employée sans cela. Mais ce qui n’est peut-être pas tout à fait aussi évident, c’est de savoir s’il tend à augmenter l’industrie générale de la société, ou à lui donner la direction la plus avantageuse.
L’industrie générale de la société ne peut jamais aller au-delà de ce que peut en employer le capital de la société. De même que le nombre d’ouvriers que peut occu­per un particulier doit être dans une proportion quelconque avec son capital, de même le nombre de ceux que peuvent aussi constamment tenir occupés tous les mem­bres qui composent une grande société, doit être dans une proportion quelconque avec la masse totale des capitaux de cette société, et ne peut jamais excéder cette pro­por­tion. Il n’y a pas de règlement de commerce qui soit capable d’augmenter l’indus­trie d’un pays au-delà de ce que le capital de ce pays en peut entretenir ; tout ce qu’il peut faire, c’est de faire prendre à une portion de cette industrie une direction autre que celle qu’elle aurait prise sans cela, et il n’est pas certain que cette direction artificielle promette d’être plus avantageuse à la société que celle que l’industrie aurait suivie de son plein gré.
Chaque individu met sans cesse tous ses efforts à chercher, pour tout le capital dont il peut disposer, l’emploi le plus avantageux ; il est bien vrai que c’est son propre bénéfice qu’il a en vue, et non celui de la société ; mais les soins qu’il se donne pour trouver son avantage personnel le conduisent naturellement, ou plutôt nécessaire­ment, à préférer précisément ce genre d’emploi même qui se trouve être le plus avan­tageux à la société[2].
Premièrement, chaque individu tâche d’employer son capital aussi près de lui qu’il le peut et, par conséquent, autant qu’il le peut, il tâche de faire valoir l’industrie natio­nale, pourvu qu’il puisse gagner par là les profits ordinaires que rendent les capitaux, ou guère moins.
Ainsi, à égalité de profits ou à peu près, tout marchand en gros préférera naturelle­ment le commerce intérieur au commerce étranger de consommation, et le commerce étranger de consommation au commerce de transport. Dans le commerce intérieur, il ne perd jamais aussi longtemps son capital de vue que cela lui arrive fréquemment dans le commerce étranger de consommation ; il est bien plus à portée de connaître le caractère des personnes auxquelles il a à se confier, ainsi que l’état de leurs affaires ; et s’il lui arrive d’avoir mal placé sa confiance, il connaît mieux les lois auxquelles il est obligé de recourir. Dans le commerce de transport, le capital du marchand est, pour ainsi dire, partagé entre deux pays étrangers, et il n’y en a aucune partie qui soit dans la nécessité de revenir dans le sien, ni qui soit immédiatement sous ses yeux et à son commandement. Le capital qu’un négociant d’Amsterdam emploie à transporter du blé de Kœnigsberg à Lisbonne, et des fruits et des vins de Lisbonne à Kœnigs­berg, doit, en général, demeurer moitié à Kœnigsberg et moitié à Lisbonne : il n’y en a aucune partie qui ait jamais besoin de venir à Amsterdam. La résidence naturelle de ce négociant devrait être à Kœnigsberg ou à Lisbonne, et il ne peut y avoir que des circonstances particulières qui lui fassent préférer le séjour d’Amsterdam ; en outre, le désagrément qu’il trouve à se voir toujours si éloigné de son capital le détermine, en général, à faire venir à Amsterdam une partie, tant des marchandises de Kœnigsberg destinées pour le marché de Lisbonne, que de celles de Lisbonne qu’il destine pour le marché de Kœnigsberg ; et quoique cette marche l’assujettisse nécessairement à un double embarras de chargement et de déchargement, ainsi qu’au payement de quel­ques droits et à quelques visites de douanes, cependant c’est une charge extraordinaire à laquelle il se résigne volontiers, pour l’avantage seulement d’avoir toujours quelque partie de son capital sous ses yeux et sous sa main ; et c’est ainsi que tout pays qui a une part considérable au commerce de transport devient toujours l’entrepôt ou le mar­ché général des marchandises de tous les différents pays entre lesquels se fait son com­merce. Pour éviter les frais d’un second chargement et déchargement, le mar­chand cherche toujours à vendre, dans le marché intérieur, le plus qu’il peut de mar­chandises de tous ces différents pays ; et ainsi, autant qu’il le peut, il convertit son commerce de transport en commerce étranger de consommation. De même, un mar­chand qui fait le commerce étranger de consommation, et qui rassemble des mar­chandises qu’il destine aux marchés étrangers, se trouvera toujours bien aise, à égalité de profits ou à peu près, d’avoir occasion de vendre autant de ces marchandises qu’il pourra dans le marché intérieur ; il s’épargne d’autant par là les risques et la peine de l’exportation, et ainsi il convertit, autant qu’il est en lui, son commerce étranger de consommation en commerce intérieur. Le marché intérieur est donc, si je puis m’ex­primer ainsi, le centre autour duquel les capitaux des habitants du pays vont tou­jours circulant, et vers lequel ils tendent sans cesse, quoique des causes particulières puis­sent quelquefois les en écarter et les repousser vers des emplois plus éloignés. Or, comme on l’a déjà fait voir[3], un capital employé dans le commerce intérieur met né­ces­sairement en activité une plus grande quantité d’industrie nationale, et fournit de l’occupation et du revenu à un plus grand nombre d’habitants du pays qu’un pareil capital employé au commerce étranger de consommation, et un capital employé dans ce dernier genre de commerce a les mêmes avantages sur un pareil capital placé dans le commerce de transport. Par conséquent, à égalité ou presque égalité de profits, cha­que individu incline naturellement à employer son capital de la manière qui pro­met de donner le plus d’appui à l’industrie nationale, et de fournir de l’occupation et du revenu à un plus grand nombre d’habitants du pays. 
En second lieu, chaque individu qui emploie son capital à faire valoir l’industrie nationale, tâche nécessairement de diriger cette industrie de manière que le produit qu’elle donne ait la plus grande valeur possible.
Le produit de l’industrie est ce qu’elle ajoute au sujet ou à la matière à laquelle elle s’applique. Suivant que la valeur de ce produit sera plus grande ou plus petite, les produits de celui qui met l’industrie en œuvre seront aussi plus grands ou plus petits. Or, ce n’est que dans la vue du profit qu’un homme emploie son capital à faire valoir l’industrie et, par conséquent, il tâchera toujours d’employer son capital à faire valoir le genre d’industrie dont le produit promettra la plus grande valeur, ou dont on pourra espérer le plus d’argent ou d’autres marchandises en échange.
Mais le revenu annuel de toute société est toujours précisément égal à la valeur échangeable de tout le produit annuel de son industrie, ou plutôt c’est précisément la même chose que cette valeur échangeable. Par conséquent, puisque chaque individu tâche, le plus qu’il peut, 1° d’employer son capital à faire valoir l’industrie nationale, et 2° de diriger cette industrie de manière à lui faire produire la plus grande valeur possible, chaque individu travaille nécessairement à rendre aussi grand que possible le revenu annuel de la société. À la vérité, son intention, en général, n’est pas en cela de servir l’intérêt public, et il ne sait même pas jusqu’à quel point il peut être utile à la société. En préférant le succès de l’industrie nationale à celui de l’industrie étrangère, il ne pense qu’à se donner personnellement une plus grande sûreté ; et en dirigeant cette industrie de manière à ce que son produit ait le plus de valeur possible, il ne pense qu’à son propre gain ; en cela, comme dans beaucoup d’autres cas, il est conduit par une main invisible à remplir une fin qui n’entre nullement dans ses intentions ; et ce n’est pas toujours ce qu’il y a de plus mal pour la société, que cette fin n’entre pour rien dans ses intentions. Tout en ne cherchant que son intérêt personnel, il travaille souvent d’une manière bien plus efficace pour l’intérêt de la société, que s’il avait réellement pour but d’y travailler. Je n’ai jamais vu que ceux qui aspiraient, dans leurs entreprises de commerce, à travailler pour le bien général, aient fait beaucoup de bon­nes choses. Il est vrai que cette belle passion n’est pas très-commune parmi les mar­chands, et qu’il ne faudrait pas de longs discours pour les en guérir. 
Quant à la question de savoir quelle est l’espèce d’industrie nationale que son capi­tal peut mettre en œuvre, et de laquelle le produit promet de valoir davantage, il est évident que chaque individu, dans sa position particulière, est beaucoup mieux à mê­me d’en juger qu’aucun homme d’État ou législateur ne pourra le faire pour lui. L’hom­me d’État qui chercherait à diriger les particuliers dans la route qu’ils ont à tenir pour l’emploi de leurs capitaux, non-seulement s’embarrasserait du soin le plus inutile, mais encore il s’arrogerait une autorité qu’il ne serait pas sage de confier, je ne dis pas à un individu, mais à un conseil ou à un sénat, quel qu’il pût être ; autorité qui ne pourrait jamais être plus dangereusement placée que dans les mains de l’homme assez insensé et assez présomptueux pour se croire capable de l’exercer.
Accorder aux produits de l’industrie nationale, dans un art ou dans un genre de manufacture particulier, le monopole du marché intérieur, c’est en quelque sorte diriger les particuliers dans la route qu’ils ont à tenir pour l’emploi de leurs capitaux et, en pareil cas, prescrire une règle de conduite est presque toujours inutile ou nuisi­ble. Si le produit de l’industrie nationale peut être mis au marché à aussi bon compte que celui de l’industrie étrangère, le précepte est inutile ; s’il ne peut pas y être mis à aussi bon compte, le précepte sera, en général, nuisible, La maxime de tout chef de famille prudent est de ne jamais essayer de faire chez soi la chose qui lui coûtera moins à acheter qu’à faire. Le tailleur ne cherche pas à faire ses souliers, mais il les achète du cordonnier ; le cordonnier ne tâche pas de faire ses habits, mais il a recours au tailleur ; le fermier ne s’essaye à faire ai les uns ni les autres, mais il s’adresse à ces deux artisans et les fait travailler. Il n’y en a pas un d’eux tous qui ne voie qu’il y va de son intérêt d’employer son industrie tout entière dans le genre de travail dans lequel il a quelque avantage sur ses voisins, et d’acheter toutes les autres choses dont il peut avoir besoin, avec une partie du produit de cette industrie, ou, ce qui est la même chose, avec le prix d’une partie de ce produit.
Ce qui est prudence dans la conduite de chaque famille en particulier, ne peut guère être folie dans celle d’un grand empire. Si un pays étranger peut nous fournir une marchandise à meilleur marché que nous ne sommes en état de l’établir nous-mê­mes, il vaut bien mieux que nous la lui achetions avec quelque partie du produit de notre propre industrie, employée dans le genre dans lequel nous avons quelque avan­tage. L’industrie générale du pays étant toujours en proportion du capital qui la met en œuvre, elle ne sera pas diminuée pour cela, pas plus que ne l’est celle des artisans dont nous venons de parler ; seulement, ce sera à elle a chercher la manière dont elle peut être employée à son plus grand avantage. Certainement, elle n’est pas employée à son plus grand avantage quand elle est dirigée ainsi vers un objet qu’elle pourrait acheter à meilleur compte qu’elle ne pourra le fabriquer. Certainement, la valeur de son produit annuel est plus ou moins diminuée quand on la détourne de produire des marchandises qui auraient plus de valeur que celle qu’on lui prescrit de produire. D’après la supposition qu’on vient de faire, cette marchandise pourrait s’acheter de l’étranger à meilleur marché qu’on ne pourrait la fabriquer dans le pays ; par consé­quent, on aurait pu l’acheter avec une partie seulement des marchandises, ou ce qui revient au même, avec une partie seulement du prix des marchandises qu’aurait pro­duites l’industrie nationale, à l’aide du même capital, si on l’eût laissée suivre sa pente naturelle. Par conséquent, l’industrie nationale est détournée d’un emploi plus avanta­geux, pour en suivre un qui l’est moins, et la valeur échangeable de son produit an­nuel, au lieu d’être augmentée, suivant l’intention du législateur, doit nécessaire­ment souffrir quelque diminution à chaque règlement de cette espèce.
À la vérité, il peut se faire qu’à l’aide de ces sortes de règlements, un pays acquiè­re un genre particulier de manufacture plutôt qu’il ne l’aurait acquis sans cela, et qu’au bout d’un certain temps ce genre de manufacture se fasse dans le pays à aussi bon marché ou à meilleur marché que chez l’étranger. Mais quoiqu’il puisse ainsi arriver que l’on porte avec succès l’industrie nationale dans un canal particulier, plutôt qu’elle ne s’y serait portée d’elle-même, il ne s’ensuit nullement que la somme totale de l’in­dustrie ou des revenus de la société puisse jamais recevoir aucune augmentation de ces sortes de règlements. L’industrie de la société ne peut augmenter qu’autant que son capital augmente, et ce capital ne peut augmenter qu’à proportion de ce qui peut être épargné peu à peu sur les revenus de la société et, à coup sûr, ce qui diminue son revenu n’augmentera pas son capital plus vite qu’il ne se serait augmenté de lui-même, si l’on eût laissé le capital et l’industrie chercher l’un et l’autre leurs emplois naturels.
Encore que la société ne pût, faute de quelque règlement de cette espèce, acquérir jamais le genre de manufacture en question, il ne s’ensuivrait pas pour cela qu’elle en dût être un seul moment plus pauvre, dans tout le cours de sa carrière ; il pourrait toujours se faire que, dans tous les instants de sa durée, la totalité de son capital et de son industrie eût été employée (quoiqu’à d’autres objets) de la manière qui était, pour le moment, la plus avantageuse. Ses revenus, dans tous ces instants, pourraient avoir été les plus grands que son capital eût été en état de rapporter, et il se pourrait faire que son capital et son revenu eussent toujours été l’un et l’autre en augmentant avec la plus grande rapidité possible.
Les avantages naturels qu’un pays a sur un autre pour la production de certaines marchandises sont quelquefois si grands, que du sentiment unanime de tout le monde, il y aurait de la folie à vouloir lutter contre eux. Au moyen de serres chaudes, de couches, de châssis de verre, on peut faire croître en Écosse de fort bons raisins, dont on peut faire aussi de fort bon vin avec trente fois peut-être autant de dépense qu’il en coûterait pour s’en procurer de tout aussi bon de l’étranger. Or, trouverait-on bien raisonnable un règlement qui prohiberait l’importation de tous les vins étrangers[4], uniquement pour encourager à faire du vin de Bordeaux et du vin de Bourgogne en Écosse ? Mais s’il y a absurdité évidente à vouloir tourner vers un emploi trente fois plus du capital et de l’industrie du pays, qu’il ne faudrait en mettre pour acheter à l’étranger la même quantité de la marchandise qu’on veut avoir, nécessairement la même absurdité existe (et quoique pas tout à fait aussi choquante, néanmoins exac­tement la même) à vouloir tourner vers un emploi de la même sorte un trentième, ou, si l’on veut, un trois-centième de l’un et de l’autre de plus qu’il n’en faut. Il n’importe nullement, à cet égard, que les avantages qu’un pays a sur l’autre soient naturels ou acquis. Tant que l’un des pays aura ces avantages et qu’ils manqueront à l’autre, il sera toujours plus avantageux pour celui-ci d’acheter du premier, que de fabriquer lui-même. L’avantage qu’a un artisan sur son voisin qui exerce un autre métier, n’est qu’un avantage acquis, et cependant tous les deux trouvent plus de bénéfice à acheter l’un de l’autre, que de faire eux-mêmes ce qui ne concerne pas leur aptitude parti­culière.
Les gens qui tirent le plus grand avantage de ce monopole du marché intérieur, ce sont les marchands et les manufacturiers. La prohibition d’importer du bétail étranger ou des viandes salées, ainsi que les gros droits mis sur le blé étranger, lesquels, dans les temps d’abondance moyenne, équivalent à une prohibition, ne sont pas, à beau­coup près, aussi avantageux aux nourrisseurs de bestiaux et aux fermiers de la Grande-Bretagne, que le sont les autres règlements de la même sorte aux marchands et aux manufacturiers. Les ouvrages de manufactures, et principalement ceux du gen­re le plus fini, se transportent bien plus aisément d’un pays à un autre que le bétail ou le blé. Aussi, c’est à porter et à rapporter des articles de manufactures que le com­mer­ce étranger s’emploie principalement. En fait de manufactures, il ne faut qu’un très-petit bénéfice pour mettre les étrangers à même de vendre au-dessous de nos propres ouvriers, même chez nous. Il en faudrait un très-considérable pour les mettre dans le cas d’en faire autant à l’égard du produit brut du sol. Si l’on venait à permettre la libre impor­tation des ouvrages des fabriques étrangères, plusieurs des manufactures de l’inté­rieur en souffriraient vraisemblablement ; peut-être quelques-unes d’elles en se­raient totalement ruinées, et une partie considérable des capitaux et de l’industrie employés aujourd’hui dans nos fabriques serait forcée de chercher un autre emploi. Mais on permettrait la plus libre importation du produit brut du sol que l’agriculture du pays ne ressentirait aucun effet semblable[5].
Si jamais, par exemple, on laissait une pareille liberté à l’importation du bétail étran­ger, il y en aurait si peu d’importé, que le commerce de nourrisseur de bestiaux dans ce pays s’en ressentirait bien peu. Le bétail en vie est peut-être la seule mar­chan­dise dont le transport soit plus coûteux par mer que par terre. Par terre, il se transporte lui-même au marché. Par mer, non-seulement le transport des bestiaux, mais encore celui de la nourriture et de l’eau qu’il faut embarquer avec eux, ne laissent pas que d’en­traîner des frais et beaucoup d’embarras. À la vérité, le trajet si court entre l’Irlan­de et la Grande-Bretagne rend plus facile l’importation du bétail d’Irlande. Mais quand même la libre importation de ce bétail, qui vient d’être permise pour un temps limité seulement, serait rendue perpétuelle, elle ne causerait pas un grand dommage aux nourrisseurs de bestiaux de la Grande-Bretagne. Ces parties de la Grande-Bretagne qui bordent la mer d’Irlande sont toutes des pays d’herbages. Ce ne serait jamais pour leur usage que le bétail d’Irlande pourrait être importé, mais il faudrait le conduire à travers ces pays qui sont fort étendus, avec beaucoup de frais et beaucoup d’embarras, avant qu’il pût arriver à un marché qui lui fût propre. Des bestiaux gras ne pourraient pas faire une aussi longue route ; on ne pourrait donc importer que des bes­tiaux maigres. Or, une pareille importation ne pourrait pas préjudicier à l’intérêt des pays qui nourrissent et engraissent du bétail, et leur serait même plutôt avantageuse, en réduisant le prix du bétail maigre, mais elle toucherait seulement aux intérêts des pays qui font des élèves[6]. Le petit nombre de bestiaux irlandais importés depuis la permission, joint au bon prix auquel le bétail maigre continue encore à se vendre, semblent des preuves convaincantes que la libre importation du bétail d’Irlande n’aurait vraisemblablement jamais aucun effet bien sensible sur le commerce même des pays de la Grande-Bretagne qui font des élèves. À la vérité, on dit qu’en Irlande les gens du peuple se sont quelquefois opposés par la violence à la sortie des bestiaux de leur pays ; mais si les exportateurs avaient trouvé de grands profits à continuer ce commerce, ayant déjà la loi pour eux, ils auraient bien su faire cesser cette opposition populaire.
D’ailleurs, les pays qui font commerce sur l’engrais des bestiaux doivent avoir déjà reçu un très-haut degré d’amélioration, tandis que ceux dont le commerce consiste à faire des élèves sont en général des pays incultes. Le haut prix du bétail maigre, en augmentant la valeur des terres incultes, est comme une sorte de grati­fication contre la culture. Un pays qui serait partout richement cultivé aurait plus d’avantage à importer son bétail maigre de l’étranger, que d’en élever chez soi. Aussi dit-on que c’est la maxime suivie aujourd’hui dans la province de Hollande. Il est vrai que les montagnes d’Écosse, celles du pays de Galles et du Northumberland sont des pays peu susceptibles d’amélioration, et que la nature semble avoir destinés à faire des élèves de bestiaux pour la Grande-Bretagne. La plus grande liberté donnée à l’im­portation du bétail étranger aurait pour tout effet d’empêcher que ces pays qui font des élèves ne prissent avantage de l’accroissement de population du reste du royaume et des progrès de son amélioration, qu’ils ne fissent monter leurs prix à un point exorbitant, et ne levassent ainsi un véritable impôt sur toutes les parties du pays plus améliorées et mieux cultivées.
De même, la plus grande liberté dans l’importation des viandes salées aurait tout aussi peu d’effet sur le commerce des nourrisseurs de bestiaux de la Grande-Bretagne, que celle du bétail en vie. Non-seulement les viandes salées sont une marchandise d’un gros volume, mais, comparées aux viandes fraîches, c’est une marchandise de bien moindre qualité et à la fois plus chère, puisqu’elle coûte plus de travail et de dépense. Elles ne pourraient donc jamais venir en concurrence avec les viandes fraî­ches du pays, mais tout au plus avec ses viandes salées. On pourrait s’en servir à ravitailler des vaisseaux pour des voyages de long cours et pour d’autres usages sem­blables, mais elles ne pourraient jamais faire une partie considérable de la nour­riture du peuple. Ce qui prouve bien par expérience que nos nourrisseurs n’en ont rien à craindre, c’est la petite quantité de viandes salées importées d’Irlande, depuis qu’on en a rendu l’exportation libre. Il ne paraît pas que le prix de la viande de boucherie s’en soit jamais ressenti d’une manière notable.
La liberté même de l’importation du blé étranger ne toucherait que très-peu à l’intérêt des fermiers de la Grande-Bretagne. Le blé est une marchandise d’un bien plus grand encombrement que la viande de boucherie. Une livre de blé est aussi chère à un denier, qu’une livre de viande à quatre. La petite quantité de blé étranger im­porté, même dans les temps de la plus grande cherté, peut bien rassurer nos fermiers contre les suites d’une liberté illimitée d’importation. La quantité moyenne importée, une année dans l’autre, ne monte, suivant l’auteur très-instruit du Traité sur le com­merce des blés, qu’à 23728 quarters de grains de toute espèce, et ne va pas au-delà d’un 571e de la consommation annuelle[7]. Mais, comme la prime sur le blé occa­sionne une plus grande exportation dans les années d’abondance, elle doit par suite occasionner, dans les années de cherté, une importation plus forte que celle qui aurait lieu sans cela. Elle est cause que l’abondance d’une année ne sert plus à balancer la disette d’une autre ; et comme elle augmente nécessairement la quantité moyenne des exportations, il faut bien pareillement qu’elle augmente d’autant la quantité moyenne des importations, l’état étant supposé le même. S’il n’y avait pas de prime, comme on exporterait moins de blé, il est vraisemblable qu’il y en aurait aussi moins d’importé, année commune, qu’il n’y en a à présent. Les marchands de blé, ceux qui font le commerce d’en porter et d’en rapporter entre la Grande-Bretagne et l’étranger, auraient moins d’occupation et pourraient en souffrir beaucoup ; mais les propriétaires de la campagne et les fermiers en souffriraient très-peu. Aussi, c’est chez les mar­chands de blé et non chez les propriétaires ni les fermiers que j’ai remarqué les plus grandes inquiétudes sur le renouvellement et la continuation de la prime. 
Les propriétaires de biens de campagne et les fermiers peuvent se glorifier d’être, de toutes les classes, la moins infectée du misérable esprit de monopole[8]. Vous voyez quelquefois un entrepreneur d’une grande fabrique s’alarmer si une autre fabrique du même genre vient s’établir à vingt milles de la sienne. Le Hollandais[9] entrepreneur de la manufacture de draps d’Abbeville stipula qu’aucune manufacture du même genre ne pourrait s’établir à trente lieues à la ronde de cette ville. Les propriétaires et fer­miers, au contraire, sont, en général, plutôt disposés à favoriser qu’à gêner la culture et l’amélioration des domaines et des fermes de leurs voisins. Ils n’ont pas leurs secrets, comme la plupart des manufacturiers ont les leurs ; mais, en général, s’ils connaissent quelque pratique nouvelle qu’ils aient trouvée avantageuse, ils sont plutôt curieux de la communiquer à leurs voisins et de la propager le plus qu’ils peuvent. Pius quœstus, dit Caton l’ancien, stabilissimusque, minimèque invidiosus ; minimèque malè cogitantes sunt, qui in eo studio occupati sunt. Les propriétaires de campagne et les fermiers, dispersés en différents endroits du pays, ne peuvent se concerter entre eux aussi aisément que les marchands et les manufacturiers, qui, étant réunis dans des villes et accoutumés à cet esprit exclusif de corporation qui règne parmi eux, cherchent naturellement à obtenir contre leurs compatriotes ces mêmes privilèges exclusifs qu’ils ont déjà, en général, contre les habitants de leurs villes respectives ; aussi semblent-ils avoir été les premiers inventeurs de ces entraves à l’importation des marchandises étrangères, qui leur assurent le monopole du marché intérieur. Ce fut vraisemblablement pour les imiter et pour se mettre au niveau de gens qu’ils voyaient toujours disposés à les opprimer, que nos propriétaires de campagne et nos fermiers se sont écartés de la générosité naturelle à leur profession, jusqu’à demander le privi­lège exclusif de fournir de la viande et du blé à leurs compatriotes. Ils ne se donnèrent peut-être pas le temps d’examiner combien ils étaient moins intéressés à gêner la liberté du commerce que ne l’étaient ceux dont ils suivaient l’exemple.
Prohiber, par une disposition perpétuelle, l’importation du blé et du bétail de l’étran­­ger, c’est, à la lettre, statuer que la population et l’industrie du pays n’iront, dans aucun temps, au-delà de ce que peut en faire subsister le produit du sol.
Il paraîtrait cependant qu’il y a deux cas dans lesquels il serait, en général, avan­tageux d’établir quelque charge sur l’industrie étrangère pour encourager l’industrie nationale.
Le premier, c’est quand une espèce particulière d’industrie est nécessaire à la défense du pays. Par exemple, la défense de la Grande-Bretagne dépend beaucoup du nombre de ses vaisseaux et de ses matelots. C’est donc avec raison que l’Acte de navi­gation cherche à donner aux vaisseaux et aux matelots de la Grande-Bretagne le monopole de la navigation de leur pays, par des prohibitions absolues en certains cas, et par de fortes charges, dans d’autres, sur la navigation étrangère. Telles sont les principales dispositions de cet acte :
1° Il est défendu à tous bâtiments dont les propriétaires, les maîtres et les trois quarts de l’équipage ne sont pas sujets de la Grande-Bretagne, de commercer dans les établissements et colonies de la Grande-Bretagne, ou de faire le cabotage sur les côtes de la Grande-Bretagne, sous peine de confiscation du bâtiment et de la cargaison.
2° Une grande quantité de divers articles d’importation du plus grand encom­brement ne peuvent être amenés dans les ports de la Grande-Bretagne que dans des bâtiments tels que ceux permis par l’article ci-dessus, ou dans des bâtiments du pays où sont produites les marchandises importées, et desquels les propriétaires, les maîtres et les trois quarts de l’équipage seraient de ce même pays ; et encore quand c’est dans des bâtiments de cette dernière sorte qu’elles sont importées, elles sont sujettes au double du droit dû par les marchandises étrangères. Si elles sont importées dans des bâtiments de tout autre pays, la peine est de la confiscation du vaisseau et de sa cargaison.
Lorsque cet Acte fut dressé, les Hollandais étaient, comme ils le sont encore aujourd’hui, les grands voituriers de l’Europe ; cette disposition empêcha qu’ils ne fus­sent aussi ceux de la Grande-Bretagne, ou du moins qu’ils n’importassent chez nous les marchandises d’aucun autre pays de l’Europe.
3° Une grande quantité de divers articles d’importation du plus grand encom­brement ne peut être importée, même dans les bâtiments de la Grande-Bretagne, de tout autre pays que de celui qui les produit, et cela sous peine de confiscation du bâtiment et de la cargaison.
Cette clause fut aussi vraisemblablement dirigée contre les Hollandais. La Hollan­de était alors, comme aujourd’hui, le grand entrepôt de toutes les marchandises de l’Europe, et par cette disposition on empêcha que les bâtiments de la Grande-Bretagne n’allassent charger en Hollande les marchandises des autres pays de l’Europe[10].
4° Le poisson salé de toute espèce, les fanons, huiles et graisse de baleine, quand la pêche et la préparation n’en ont pas été faites à bord de bâtiments de la Grande-Bretagne, ne peuvent être importés sans payer un double droit de douane étrangère.
Les Hollandais, qui sont encore les principaux pêcheurs de l’Europe, étaient alors les seuls qui entreprissent de fournir de poisson les pays étrangers. Ce règlement mit une très-forte charge sur l’approvisionnement que la Grande-Bretagne aurait pu tirer d’eux en ce genre.
Lorsque l’Acte de navigation fut passé, quoique l’Angleterre et la Hollande ne fus­sent pas en guerre pour le moment, néanmoins il existait entre les deux nations l’animosité la plus violente. Cette animosité avait commencé sous le gouvernement du long Parlement qui rédigea le premier l’Acte de navigation, et bientôt après elle éclata par les guerres qui eurent lieu avec la Hollande, pendant le protectorat et sous le règne de Charles II. Il n’est donc pas impossible que quelques-unes des dispositions de cet Acte célèbre aient été le fruit de l’animosité nationale. Elles sont néanmoins aussi sages que si elles eussent toutes été dictées par la plus mûre délibération et les intentions les plus raisonnables[11]. La haine nationale avait alors en vue précisément le même but que celui qu’eût pu se proposer la sagesse la plus réfléchie, c’est-à-dire l’affaiblissement de la marine de la Hollande, la seule puissance navale qui fût dans le cas de menacer la sûreté de l’Angleterre.
L’Acte de navigation n’est pas favorable au commerce étranger ou à l’accrois­sement de cette opulence dont ce commerce est la source. L’intérêt d’une nation, dans ses relations commerciales avec les nations étrangères, est le même que celui d’un marchand, relativement aux diverses personnes avec lesquelles il fait des affaires, c’est-à-dire d’acheter au meilleur marché et de vendre le plus cher possible. Mais elle sera bien plus dans le cas d’acheter à bon marché quand, par la liberté de commerce la plus absolue, elle encouragera toutes les nations à lui apporter les marchandises qu’elle peut désirer acheter, et par la même raison elle sera bien plus dans le cas de vendre cher quand ces marchés seront par là remplis du plus grand nombre d’ache­teurs. L’Acte de navigation ne met, à la vérité, aucune charge sur les bâtiments étran­gers qui viennent exporter les produits de l’industrie de la Grande-Bretagne. Même l’ancien droit d’Alien[12], qui avait coutume de se payer sur toutes les mar­chan­dises expor­tées comme sur celles importées, a été, par plusieurs actes subséquents, sup­primé sur la plupart des articles d’exportation. Mais si des prohibitions ou de gros droits empêchent les étrangers de venir vendre, ceux-ci ne sauraient consentir à se présenter toujours pour acheter, parce que, obligés de venir sans cargaison, ils per­draient le fret depuis leur pays jusqu’aux ports de la Grande-Bretagne. Ainsi, en diminuant le nombre des vendeurs, nous diminuons nécessairement celui des ache­teurs, et par là nous sommes d’autant plus exposés, non-seulement à acheter plus cher les marchandises étrangères, mais encore à vendre les nôtres meilleur marché que s’il y avait une parfaite liberté de commerce. Néanmoins, comme la sûreté de l’État est d’une plus grande importance que sa richesse, l’Acte de navigation est peut-être le plus sage de tous les règlements de commerce de l’Angleterre.
Le second cas dans lequel il sera avantageux, en général, de mettre quelque char­ge sur l’industrie étrangère pour encourager l’industrie nationale, c’est quand le produit de celle-ci est chargé lui-même de quelque impôt dans l’intérieur. Dans ce cas, il paraît raisonnable d’établir un pareil impôt sur le produit du même genre, venu de fabrique étrangère. Ceci n’aura pas l’effet de donner à l’industrie nationale le monopole du marché intérieur, ni de porter vers un emploi particulier plus de capital et de travail du pays qu’il ne s’en serait porté naturellement. Tout l’effet qui en résultera, ce sera d’empêcher qu’une partie de ce qui s’y serait porté naturellement n’en soit détourné par l’impôt, pour prendre une direction moins naturelle, et de laisser la concurrence entre l’industrie étrangère et l’industrie nationale, aussi près que possible des conditions où elle se trouvait auparavant. En Angleterre, quand une taxe de ce genre est établie sur quelque produit de l’industrie nationale, il est d’usage en même temps, pour apaiser les clameurs et les doléances des marchands et des manu­facturiers, qui crient qu’ils ne pourront plus soutenir la concurrence dans l’intérieur, d’établir un droit beaucoup plus fort sur l’importation de toutes les marchandises étrangères de même espèce.
Suivant quelques personnes, cette seconde limitation de la liberté du commerce devrait, en certains cas, être étendue beaucoup plus loin qu’aux marchandises étran­gères, précisément de nature à venir en concurrence avec celles qui ont été imposées dans l’intérieur. Quand les choses nécessaires à la vie ont été, dans un pays, assu­jetties à un impôt, il devient à propos, selon ces personnes, d’imposer non-seulement les mêmes choses qui seraient importées des autres pays, mais toute espèce de marchandise étrangère quelconque qui pourrait être dans le cas de faire concurrence à tout autre produit de l’industrie nationale. Ces impôts, dit-on, font renchérir néces­sairement les subsistances, et le prix du travail doit toujours renchérir avec le prix de la subsistance de l’ouvrier. Par conséquent, toute marchandise produite par l’industrie nationale, quoique n’étant pas directement imposée, devient néanmoins plus chère à raison de ces impôts, parce qu’ils élèvent le prix du travail qui la produit. Ces impôts sont donc, ajoute-t-on, réellement équivalents à un impôt sur chaque marchandise produite dans l’intérieur. On en conclut que, pour mettre l’industrie nationale sur le même pied que l’industrie étrangère, il devient indispensable d’établir sur toute marchandise étrangère quelque droit égal au renchérissement qu’éprouvent celles de l’intérieur, avec lesquelles elles pourraient se trouver en concurrence.
Que les impôts sur les choses nécessaires à la vie, tels que, dans la Grande-Bre­tagne, les taxes sur la drêche, la bière, le savon, le sel, le cuir, la chandelle, etc., élè­vent nécessairement le prix du travail et, par conséquent, celui de toute autre mar­chan­dise, c’est ce que j’examinerai dans la suite, quand je viendrai à parler des im­pôts[13]. En supposant toutefois, pour le moment, qu’ils aient cet effet (et ils l’ont indubi­tablement), cependant ce renchérissement général de toutes les marchandises et, par suite, celui du travail, n’est pas la même chose que le renchérissement d’une marchan­dise particulière causé par un droit imposé directement sur elle, et il en diffère sous les deux rapports suivants -
Premièrement, il est toujours aisé de connaître avec la plus grande exactitude de combien une marchandise se trouve renchérie par un droit directement et spéciale­ment imposé sur elle ; mais il serait impossible de déterminer avec quelque précision de combien le renchérissement général du travail pourrait influer sur le prix de chaque différente marchandise produite par le travail. Il y aurait donc impossibilité de proportionner, avec quelque exactitude, l’impôt sur chaque marchandise étrangère au renchérissement de chaque marchandise nationale.
Secondement, les impôts sur les choses nécessaires à la vie ont, sur le sort du Peuple, à peu près le même effet qu’un sol ingrat ou un mauvais climat. Ces impôts renchérissent les denrées de la même manière que si elles coûtaient plus de travail et de dépense qu’à l’ordinaire pour être produites. Comme dans la cherté naturelle qui procède de la pauvreté du sol ou de la dureté du climat, il serait absurde de prétendre diriger les gens sur la route qu’ils ont à prendre pour l’emploi de leurs capitaux et de leur industrie, il ne le serait pas moins de le vouloir faire dans cette cherté artificielle causée par les impôts. Leur laisser assortir, du mieux qu’ils l’entendront, leur industrie à leur situation, et les laisser chercher eux-mêmes les emplois dans lesquels, malgré les circonstances défavorables où ils se trouvent, ils pourront avoir quelque avantage, soit sur le marché intérieur, soit sur le marché étranger, c’est évidemment le parti qui peut, dans l’un comme dans l’autre de ces deux cas, être le plus avantageux pour eux. Mais établir sur eux un nouvel impôt parce qu’ils sont déjà surchargés d’impôts, et par la raison qu’ils payent déjà trop cher les choses nécessaires à la vie, vouloir leur faire payer également plus cher la plupart de tous les autres objets de leur consommation, c’est à coup sûr le moyen le plus étrange qu’on puisse imaginer pour adoucir leur situation.
Ces sortes d’impôts, quand ils sont montés à un certain point, sont une calamité aussi fâcheuse que la stérilité du sol ou l’inclémence des saisons ; et cependant, c’est dans les pays les plus riches et les plus industrieux qu’en général on les trouve établis. Aucun autre pays ne serait en état de supporter une aussi forte maladie. De même qu’il n’y a que les corps les plus vigoureux qui puissent se maintenir en vie et même en santé avec le régime le plus malsain, de même il n’y a que les nations qui sont les plus favorisées dans toute espèce d’industrie par des avantages naturels ou acquis, qui puissent subsister et même prospérer sous le poids de ces sortes d’impôts. La Hollande est le pays de l’Europe où ils se sont le plus multipliés, et qui, par les circonstances particulières où il se trouve, continue toujours à prospérer, non pas à cause de ces impôts, comme on a eu l’extrême absurdité de le supposer, mais en dépit de ces impôts.
S’il y a deux cas dans lesquels il sera, en général, avantageux d’imposer quelque charge sur l’industrie étrangère pour encourager l’industrie nationale, il y en a aussi deux autres dans lesquels il peut y avoir quelquefois lieu à délibérer : dans l’un, jusqu’à quel point il est à propos de laisser libre l’importation de certaines marchan­dises étrangères ; et dans l’autre, jusqu’à quel point et de quelle manière il peut être à propos de rendre la liberté à cette importation, après que cette liberté a été pendant quelque temps interrompue.
Le cas dans lequel il peut y avoir quelquefois lieu à délibérer jusqu’à quel point il serait à propos de laisser subsister la liberté de l’importation de certaines mar­chan­dises étrangères, c’est lorsqu’une nation étrangère gêne, par de forts droits ou par des prohibitions, l’importation de quelqu’un de nos produits manufacturés dans son pays. Dans ce cas, on est naturellement porté à user de représailles, et à imposer les mêmes droits et prohibitions à l’importation de quelques-unes ou de toutes leurs mar­chan­dises chez nous ; aussi est-il rare que les nations manquent de rendre la pa­reille dans ce cas-là. Les Français, en particulier, ont été les premiers à donner l’exem­ple de favoriser leurs propres manufactures, en gênant l’importation des marchandises étran­gères qui auraient pu leur faire concurrence. Ce fut en grande partie ce qui constitua la politique de M. de Colbert, qui, malgré ses grands talents, paraît en cela s’être laissé persuader par les raisonnements sophistiqués des marchands et des manu­factu­riers, toujours ardents à solliciter des monopoles contre leurs compatriotes. Aujour­d’hui, en France, l’opinion des personnes les plus éclairées est que ses opérations en ce genre n’ont pas tourné à l’avantage de sa patrie. Par le tarif de 1667, ce ministre imposa de très-forts droits sur un grand nombre d’articles de manufacture étrangère. Sur son refus de les modérer en faveur de la Hollande, celle-ci, en 1671, prohiba l’importation des vins, des eaux-de-vie et des produits des manufactures de France. Cette querelle de commerce paraît avoir occasionné en partie la guerre de 1672. La paix de Nimègue, en 1678, mit fin à cette guerre, en modérant quelques-uns de ces droits en faveur de la Hollande, laquelle, en conséquence, leva sa prohibition.
Ce fut à peu près vers ce temps que la France et l’Angleterre commencèrent à oppri­mer réciproquement l’industrie l’une de l’autre par de semblables droits et prohi­bitions, dont toutefois la France paraît avoir la première donné l’exemple. L’esprit d’hostilité qui a toujours subsisté depuis entre les deux nations a empêché jusqu’ici que ces entraves n’aient pu être adoucies d’un côté ni de l’autre[14]. En 1697, l’Angleterre prohiba l’importation des dentelles de Flandre. En revanche, le gouvernement de ce Pays, alors sous la domination de l’Espagne, prohiba l’importation des laineries anglaises. En 1700, l’Angleterre leva la prohibition sur l’importation de la dentelle de Flandre, à condition que l’importation de nos laineries en Flandre serait remise sur le même pied qu’auparavant.
Des représailles de ce genre peuvent être d’une bonne politique quand il y a probabilité qu’elles amèneront la révocation des gros droits ou des prohibitions dont on a à se plaindre. L’avantage de recouvrer un grand marché étranger fera, en général, plus que compenser l’inconvénient passager de payer plus cher, pendant un court espace de temps, quelques espèces de marchandises. Quant à juger s’il y a lieu de s’attendre que de telles représailles produiront ce bon effet, c’est une question qui appartient moins peut-être aux connaissances du législateur, dont les décisions doivent être déterminées par des principes généraux et immuables, qu’à l’habileté de cet être insidieux et rusé qu’on appelle vulgairement homme d’État ou politique[15], dont les avis se dirigent sur la marche versatile et momentanée des affaires. Quand il n’y a pas de probabilité que nous puissions parvenir à faire révoquer ces empêchements, c’est, à ce qu’il semble, une mauvaise méthode pour compenser le dommage fait à quelques classes particulières du peuple, que de faire nous-mêmes un autre dommage, tant à ces mêmes classes qu’à presque toutes les autres. Quand nos voisins prohibent quelqu’un de nos objets de manufacture, en général nous prohibons chez nous, non-seulement leurs ouvrages du même genre, ce qui seul ne pourrait pas produire grand effet chez eux, mais quelques autres articles du produit de leur industrie. Cette mesu­re, sans doute, peut donner de l’encouragement à quelques classes particulières d’ou­vriers chez nous, et en frappant d’exclusion quelques-uns de leurs rivaux, elle peut mettre ces ouvriers à même d’élever leurs prix dans le marché intérieur. Mais, toutefois, la classe d’ouvriers qui souffre de la prohibition faite par nos voisins ne tirera pas d’avantages de celles que nous faisons. Au contraire ces ouvriers et presque toutes les autres classes de citoyens se trouveront par là obligés de payer certaines marchandises plus cher qu’auparavant. Ainsi, toute loi de cette espèce impose une véritable taxe sur la totalité du pays, non pas en faveur de cette classe particulière d’ouvriers à qui la prohibition faite par nos voisins a porté dommage, mais en faveur de quelque autre classe.
Le cas dans lequel il peut y avoir quelquefois lieu à délibérer jusqu’à quel point et de quelle manière il serait à propos de rétablir la liberté d’importer des marchandises étrangères, après qu’elle a été interrompue pendant quelque temps, c’est lorsqu’au moyen des gros droits ou prohibitions mises sur toutes les marchandises étrangères qui pourraient venir en concurrence avec elles, certaines manufactures particulières se sont étendues au point d’employer un grand nombre de bras. Dans ce cas, l’humanité peut exiger que la liberté du commerce ne soit rétablie que par des gradations un peu lentes, et avec beaucoup de circonspection et de réserve. Si l’on allait supprimer tout d’un coup ces gros droits et ces prohibitions, il pourrait se faire que le marché intérieur fût inondé aussitôt de marchandises étrangères à plus bas prix, tellement que plusieurs milliers de nos concitoyens se trouvassent tous à la fois privés de leur occupation ordinaire et dépourvus de tout moyen de subsistance. Le désordre qu’un tel événement entraînerait pourrait être très-grand[16]. Il y a pourtant de bonnes raisons pour croire qu’il le serait beaucoup moins qu’on ne se le figure communément, et cela par deux causes :
Premièrement, tous les objets de manufacture dont on exporte ordinairement une partie dans les autres pays de l’Europe sans prime ne se ressentiraient que fort peu de la plus libre importation des marchandises étrangères. Ces objets doivent nécessaire­ment être donnés au-dehors à aussi bon compte que toute autre marchandise étrangère de même sorte et de même qualité et, par conséquent, ils doivent nécessairement se vendre à meilleur marché dans l’intérieur. Ils resteraient donc toujours en possession du marché intérieur, et quand même, par engouement pour la mode, quelque homme à fantaisies viendrait par hasard à préférer la marchandise étrangère, uniquement parce qu’elle est étrangère, à des marchandises de même sorte, de meilleure qualité et à meilleur marché, faites dans le pays, un tel caprice, par la nature même des choses, s’étendrait à si peu de personnes, qu’il ne produirait aucun effet sensible sur l’occu­pation générale du peuple. Or, une grande partie de toutes nos différentes branches de lainages[17], de nos cuirs ouvrés et de nos articles de quincaillerie s’exportent annuelle­ment dans les autres pays de l’Europe, sans aucune prime, et ce sont là les manufac­tures qui emploient le plus grand nombre de bras. Les soieries, peut-être, sont le genre de manufactures qui aurait le plus à souffrir de cette liberté de commerce, et après elles les toiles, quoique celles-ci beaucoup moins que les premières.
Secondement, quoique, dans le cas de ce rétablissement de la liberté du com­merce, un grand nombre de gens dussent se trouver par là tous à la fois jetés hors de leur occupation ordinaire et de leur manière habituelle de subsister, il ne s’ensuivrait nullement pour cela qu’ils fussent, par cet événement, privés d’emploi et de subsistance. Lors de la réduction de l’armée et de la marine, à la fin de la dernière guerre, plus de cent mille soldats et gens de mer, nombre égal à ce qu’emploient les espèces de manufactures les plus étendues, furent tous à la fois déplacés de leur emploi ordinaire ; mais quoiqu’ils en aient eu sans doute à souffrir un peu, ils ne se trouvèrent pas pourtant dénués de toute occupation et de moyens de subsistance. La majeure partie des gens de mer entrèrent successivement au service des vaisseaux marchands, à mesure qu’ils purent en trouver l’occasion, et en même temps eux et les soldats se fondirent dans la masse du peuple, s’adonnèrent à une foule de professions diverses. Un si grand changement dans le sort de plus de cent mille hommes, tous accoutumés au maniement des armes, et plusieurs d’entre eux à la rapine et au pillage, n’entraîna non-seulement aucune convulsion dangereuse, mais même de désordre sensible. À peine s’aperçut-on quelque part que le nombre des vagabonds en eût augmenté ; les salaires mêmes du travail n’en souffrirent de réduction dans aucune profession, autant que j’ai pu le savoir, excepté dans celle de matelot au service du commerce. Mais si nous comparons les habitudes d’un soldat et celles d’un ouvrier de manufacture quelconque, nous trouverons que celles du dernier ne tendent pas autant à le rendre impropre à un nouveau métier, que celles de l’autre à le rendre impropre a toute espèce de travail. L’ouvrier a toujours été accoutumé à n’attendre sa subsistance que de son travail ; le soldat, à l’attendre de sa paye. L’industrie et l’assiduité doivent être familières à l’un ; la fainéantise et la dissipation à l’autre. Or, il est certainement beaucoup plus aisé de changer la direction de l’industrie d’une espèce de travail à une autre, que d’amener la dissipation et la fainéantise à une occupation quelconque. D’ailleurs, comme nous l’avons déjà remarqué[18], la plupart des manufactures ont d’autres branches de travail manufacturier collatérales, qui ont avec elles tant de similitude, qu’un ouvrier peut aisément transporter son industrie de l’une à l’autre. Et puis, la plupart de ces ouvriers ainsi réformés trouvent accidentellement de l’emploi dans les travaux de la campagne. Le capital qui les mettait en œuvre auparavant dans une branche particulière de manufactures restera toujours dans le pays pour y employer un pareil nombre de gens de quelque autre manière. Le capital du pays restant le même, la demande du travail sera pareillement toujours la même ou à très-peu de chose près la même, quoique ce travail puisse se trouver transporté dans des lieux et dans des industries différentes. Il est vrai que les soldats et gens de mer réformés du service du roi sont libres d’exercer toute espèce de métier, en quelque ville ou endroit que ce soit de la Grande-Bretagne et de l’Irlande. Que l’on rende à tous les autres sujets de Sa Majesté, comme on l’a fait aux soldats et gens de mer, cette même liberté naturelle d’exercer telle espèce d’industrie qu’ils jugent à propos d’exercer, c’est-à-dire, qu’on détruise les privilèges exclusifs des corporations, et qu’on révoque le statut d’apprentissage, qui sont autant d’usurpations faites sur la liberté naturelle ; qu’on ajoute à ces suppressions celle de la loi du domicile[19], de manière qu’un pauvre ouvrier, quand il se trouve perdre son emploi dans le métier ou dans le lieu où il était placé, puisse en chercher dans un autre métier ou dans un autre lieu, sans avoir à craindre d’être persécuté ou d’être renvoyé, et alors, ni la société ni les individus n’auront pas plus à souffrir d’un événement qui disperserait quelques classes particulières d’ouvriers de manufacture, qu’ils n’ont à souffrir du licenciement des soldats. Nos manufacturiers sont sans doute des gens fort utiles à leur patrie, mais ils ne peuvent pas l’être plus que ceux qui la défendent au prix de leur sang, et ils ne peuvent pas se plaindre s’ils sont traités de la même manière.
À la vérité, s’attendre que la liberté du commerce puisse jamais être entièrement rendue à la Grande-Bretagne, ce serait une aussi grande folie que de s’attendre à y voir jamais réaliser la république d’Utopie ou celle d’Océana[20]. Non-seulement les préjugés du public, mais, ce qui est encore beaucoup plus impossible à vaincre, l’intérêt privé d’un grand nombre d’individus, y opposent une résistance insurmontable. Si les officiers de l’armée s’avisaient d’opposer à toute réduction dans l’état militaire des efforts aussi bien concertés et aussi soutenus que ceux de nos maîtres manufacturiers contre toute loi tendant à leur donner de nou­veaux rivaux sur le marché national ; si les premiers animaient leurs soldats comme ceux-ci excitent leurs ouvriers pour les porter à des outrages et à des violences contre ceux qui proposent de semblables règlements, il serait aussi dangereux de tenter une réforme dans l’armée, qu’il l’est devenu maintenant d’essayer la plus légère attaque contre le monopole que nos manufacturiers exercent sur nous. Ce monopole a telle­ment grossi quelques-unes de leurs tribus particulières, que, semblables à une immen­se milice toujours sur pied, elles sont devenues redoutables au gouvernement, et dans plusieurs circonstances même elles ont effrayé la législature. Un membre du parle­ment qui appuie toutes les propositions tendant à renforcer ce monopole est sûr, non-seulement d’acquérir la réputation d’un homme entendu dans les affaires du com­merce, mais d’obtenir encore beaucoup de popularité et d’influence chez une classe de gens à qui leur nombre et leur richesse donnent une grande importance. Si, au contraire, il combat ces propositions, et surtout s’il a assez de crédit dans la chambre pour les faire rejeter, ni la probité la mieux reconnue, ni le rang le plus éminent, ni les services publics les plus distingués ne le mettront à l’abri des outrages, des insultes personnelles, des dangers même que susciteront contre lui la rage et la cupidité trompée de ces insolents monopoleurs[21]. 
L’entrepreneur d’une grande manufacture, qui se verrait obligé d’abandonner ses travaux parce que les marchés du pays se trouveraient tout d’un coup ouverts à la libre concurrence des étrangers, souffrirait sans contredit un dommage considérable. Cette partie de son capital qui s’employait habituellement en achats de matières premières et en salaires d’ouvriers trouverait peut-être, sans beaucoup de difficulté, un autre emploi. Mais il ne pourrait pas disposer, sans une perte considérable, de cette autre partie de son capital, qui était fixée dans ses ateliers et dans les divers instruments de son commerce. Une juste considération pour les intérêts de cet entrepreneur exige donc que de tels changements ne soient jamais faits brusquement, mais qu’ils soient amenés à pas lents et successifs, et après avoir été annoncés de loin. S’il était possible que les délibérations de la législature fussent toujours dirigées par de grandes vues d’intérêt général et non par les clameurs importunes de l’intérêt privé, elle devrait, pour cette seule raison peut-être, se garder avec le plus grand soin d’établir jamais aucun nouveau monopole de cette espèce, ni de donner la moindre extension à ceux qui sont déjà établis. Chaque règlement de ce genre introduit dans la constitution de l’État un germe réel de désordre, qu’il est bien difficile de guérir ensuite sans occa­sionner un autre désordre.
J’examinerai dans la suite, quand je traiterai des impôts, jusqu’à quel point il peut être à propos d’imposer des droits sur l’importation des marchandises étrangères, non pas dans la vue d’en empêcher l’introduction dans le pays, mais seulement pour former une branche de revenu au gouvernement. Les droits qui sont imposés dans la vue d’empêcher ou même de diminuer l’importation sont évidemment aussi destructifs du revenu des douanes que de la liberté du commerce.
 
 
 
↑ Chacun sait que les droits sur les soieries ont été abaissés, en Angleterre, à dater de 1826, après une discussion mémorable où le talent de M. Huskisson brilla du plus vif éclat. A. B.
↑ L’expérience a malheureusement démontré que cette assertion d’Adam Smith n’était pas exacte. Nous avons pu nous convaincre, surtout depuis quelques années, que l’intérêt des particuliers n’était pas toujours conforme à celui de l’État. A. B.
↑ Liv. II, chap. v.
↑ C’est sous ce nom que les Anglais désignent les vins de Bordeaux.
↑ Il en résulterait sans doute une baisse dans le prix des céréales et, par conséquent, dans les revenus des propriétés. Les propriétaires de terres en souffriraient, mais la communauté y trouverait de très-grands bénéfices. Toutes les fois qu’il s’agira de soumettre l’importation des céréales à des restrictions, la question sera toujours de savoir si le bien-être général doit être sacrifié aux avantages d’une certaine classe, ou non. Buchanan.
↑ Pays qui commercent sur la multiplication seulement du troupeau, à la différence des pays d’herbages, dont le commerce consiste à engraisser le bétail maigre.
↑ À l’époque où ce calcul fut établi, l’Angleterre exportait beaucoup. L’agriculture rendait beaucoup plus que les populations ne pouvaient consommer, et cet état de choses explique suffisamment pourquoi alors les importations de grains étaient peu considérables. Mais depuis lors les progrès des manufactures, ainsi que l’augmentation du travail ont produit un accroissement de population, à l’entretien de laquelle l’agriculture actuelle, malgré toutes ses améliorations, ne saurait plus suffire ; l’importation des grains a par conséquent augmenté, et se trouve en proportion plus grande relativement à la consommation entière. Pendant les années 1794, 1795 et 1796, la quantité des céréales de tout genre importées s’éleva, d’après les calculs soumis au parlement, à 4,111,325 quarts. On avait en outre, pendant ces trois années, importé 529,122 quintaux de fleur de farine, et de farine ; l’argent payé pour toutes ces fournitures s’éleva, selon l’évaluation établie, à 7,446,012 livres sterling (186,150,300 fr.). Une aussi grande importation devait naturellement produire une baisse dans les prix de tous les moyens de subsistance, et il est de l’intérêt des propriétaires et des fermiers de la prévenir. Mais il serait aussi injuste qu’impolitique d’arrêter l’importation des céréales, et de causer ainsi un préjudice considérable à la communauté, pour favoriser les intérêts des fermiers et des propriétaires. Une tentative de ce genre, faite pendant l’année 1813, fut abandonnée par suite de l’opposition générale et formidable qu’elle avait soulevée. Buchanan.
↑ Les propriétaires de terres et les fermiers veillent à leurs intérêts aussi bien que les autres ; et les efforts faits pendant l’année 1813, dans le parlement, pour faire accepter une loi ayant pour but d’interdire l’importation des céréales, à moins qu’elles ne fussent à un très-haut prix, montre assez qu’ils ne sont hostiles à aucun genre de monopole, quand ils peuvent le faire tourner à leur profit. Buchanan.
↑ Josse Van-Robais, qui obtint en 1665 un privilège exclusif dont sa famille a joui jusqu’à la révolution, malgré les vives réclamations que cette injustice excita à diverses époques, notamment en 1770.
↑ Whale-fins, whale-bones, c’est la baleine qui s’emploie pour corsets, parasols, etc., et qui se vend, ou fendue en lames, ou par fanons entiers. Ceux-ci valent communément le double de la baleine coupée. Il y a des fanons qui pèsent jusqu’à six et sept livres.
↑ Il y a de grands motifs de mettre en doute la sagesse d’une mesure qui porte une atteinte aussi grave à la liberté naturelle du commerce. Le principal but de cet acte est d’assurer le monopole de notre navigation, dans l’intérêt de notre puissance navale. Mais d’autres États ne pourraient-ils pas avoir recours aux mêmes moyens, et étant exclus de la navigation de la Grande-Bretagne, ne pourraient-ils pas à leur tour nous exclure de la navigation sur laquelle leur pouvoir s’étend ? Avec un système de liberté générale, d’autres États prendraient part a la navigation de la Grande-Bretagne, et la Grande-Bretagne participerait à la navigation générale du monde ; et on peut se demander si les chances d’acquérir une grande puissance navale nue seraient pas aussi grandes avec le principe de la liberté de la navigation, qu’avec un système de restriction. D’ailleurs la haine n’est jamais d’accord avec la sagesse. C’est un sentiment à la fois peu digne et contraire à la raison, et les mesures qu’il a fait naître portent le cachet de cet esprit d’aveuglement et de folie dans lequel elles ont été conçues. Buchanan.
↑ Du mot alien, étranger, parce que ce droit était établi sur tout ce qui allait à l’étranger ou en venait.
↑ Liv. V, chap. ii, section ii, art. 4, impôts sur les objets de consommation.
↑
Ces entraves étaient modérées dans le traité de commerce que M. Pitt avait conclu avec la France en 1786. Ce traité fera, par la sagesse et la modération de ses stipulations, l’admiration de la postérité.
Avec la conclusion de ce traité important commence une ère nouvelle dans l’histoire de la France et de l’Angleterre. Pendant plusieurs siècles ces deux pays avaient été rivaux et ennemis, et l’esprit qui avait provoqué leurs guerres désastreuses présidait à la partie commerciale de leur politique à ce point, que quoique en possession de tout ce qu’il fallait pour le développement de leur commerce, les uns dans le domaine des arts et de l’industrie, les autres par les richesses dues à un climat plus heureux et à un sol plus fertile, ils se trouvaient gênés dans l’échange de leurs produits par un système compliqué d’entraves et de droits élevés. L’objet du traité de commerce conclu par M. Pitt en 1786 fut de lever tous ces obstacles, et de faire oublier aux deux nations leurs anciens griefs en liant leurs intérêts par l’exercice d’un commerce réciproque.
Tous les droits élevés qui, antérieurement à ce traité, pesaient dans chacun de ces deux pays sur les produits de l’autre, furent réduits d’un accord commun. Les vins français furent admis aux conditions assurées aux vins du Portugal par le traité de Methuen, et obtinrent par conséquent une diminution d’un tiers.
Les droits sur beaucoup d’autres provenances de France furent considérablement réduits. La bière fut frappée d’un droit de 30 pour 100, et sur plusieurs autres articles le droit fut élevé de 10 à 12 pour 100, conformément à l’article 6 du traité, augmentation qui évidemment ne pouvait pas gêner le commerce des deux pays. Quant aux marchandises qui n’étaient pas expressément mentionnées, il fut convenu qu’elles ne payeraient pas un droit plus élevé que les mêmes marchandises importées par les nations les plus favorisées. Les navires des deux nations furent affranchis des droits de port qu’ils payaient autrefois, et les créanciers eurent la permission de poursuivre leurs débiteurs d’un pays à l’autre. La législation maritime de l’Europe fut modifiée par ce traité en plus d’un point important, de manière que la prohibition n’atteignait que les articles relatifs à une guerre de terre, tandis que la France était libre de fournir les ennemis de la Grande-Bretagne de tout ce qui leur était nécessaire pour une guerre navale. Le droit de recherche fut également abandonné par les deux parties contractantes, et un certificat délivré par les autorités compétentes fut déclaré suffisant pour constater la légalité d’une cargaison.
Les motifs de cette convention furent exposés par M. Pitt, lorsqu’elle fut présentée au parlement, et les sentiments qu’il exprimait donnent à cette mesure un caractère remarquable de modération et de sagesse. Répondant à un argument tiré de la jalousie constante qu’on prétendait nécessaire envers la France, il s’écria : « En se servant de ce mot jalousie, entend-on recommander au pays cette espèce de jalousie qui ne saurait être que l’effet de la folie ou de l’aveuglement, cette espèce de jalousie qui doit le porter à rejeter follement tous les moyens de sa prospérité ou à s’attacher aveuglément aux causes de sa ruine ? Le besoin d’une animosité perpétuelle contre la France est-il si clairement démontré et si pressant, qu’il y faille sacrifier tout l’avantage commercial que nous avons lieu d’attendre des relations pacifiques avec ce pays ? ou des rapports de bonne intelligence entre les deux royaumes sont-ils si attentatoires à l’honneur, que même l’extension de notre commerce n’en pourrait racheter l’opprobre ? » Vers la fin du même discours, il s’exprima en ces termes : « Les querelles entre la France et la Grande-Bretagne ont pendant un trop long espace de temps, non-seulement fatigué ces deux nations puissantes et respectables, mais plus d’une fois compromis la paix de l’Europe et porté le trouble jusqu’aux extrémités de la terre. À les voir agir, on les eût dites résolues à s’entre-détruire.
« J’espère que le temps est venu enfin où elles doivent se conformer à l’ordre de l’Univers et se montrer propres à réaliser les bénéfices d’un commerce amical et d’une bienveillance mutuelle. Si j’envisage le traité au point de vue politique, poursuivit-il, je ne saurais hésiter à combattre cette opinion trop souvent émise, que la France est nécessairement une ennemie irréconciliable de l’Angleterre. Mon esprit réprouve cette doctrine comme monstrueuse et impossible. Il est lâche et puéril d’admettre qu’une nation puisse être l’ennemie irréconciliable d’une autre. C’est démentir l’expérience des peuples et l’histoire de l’humanité. C’est faire la satire de toute société politique, et supposer un levain de malice diabolique dans la nature de l’homme. Ce n’est que lorsque la politique des États repose sur des principes libéraux et éclairés, que les nations peuvent espérer une tranquillité durable. » Avant la conclusion de ce mémorable traité, les relations amicales entre la France et la Grande-Bretagne rencontraient mille difficultés. Les hommes d’Étal de ce pays (l’Angleterre) semblaient croire que parce que Louis XIV a troublé la paix de l’Europe, tous ses successeurs devaient être possédés de la même ambition ; et réglant leur conduite par cette présomption, ils provoquaient l’inimitié dont ils se plaignaient. Par le traité de 1786, une nouvelle politique fut inaugurée. Les deux gouvernements, loin d’admettre la nécessité d’une hostilité éternelle entre les deux peuples, résolurent de faire la tentative d’une union sincère et durable. Ainsi considéré comme mesure politique et comme mesure commerciale , ce traité restera un monument de sagesse d’État et d’intelligence des affaires, sera dans les âges futurs l’entretien des hommes qui réfléchissent, et servira de thème aux éloges de l’historien *. Buchanan.
↑ Il n’y a pas de circonstances qui puissent rattacher ces représailles absurdes à un principe de gouvernement sain et véritable, et si les hommes d’État avaient plus de confiance dans les principes inaltérables de la justice et de la raison, que dans les suggestions de leur propre esprit, ordinairement très-borné, ils feraient à coup sûr infiniment plus pour le bien-être des nations et le bonheur du monde. Buchanan.
↑
Il est permis de croire que les pertes et inconvénients qui suivent toujours la transition d’un système de commerce exclusif à un système libéral, ont été singulièrement exagérés. Les hommes employés dans les quelques branches de l’industrie anglaise qui ne pourraient résister a une concurrence illimitée, ne forment qu’une portion peu considérable de notre population ouvrière. C’est cette fraction de la population qui gagne au maintien du système prohibitif, et qui par conséquent souffrirait de son abolition. — La valeur des marchandises produites annuellement en Angleterre a été évaluée dans les derniers relevés statistiques à peu près à la somme de 125,000,000 livres sterling (3,125,000,000 fr.), y compris les matières premières. Or, les toiles et les soieries sont les deux seules industries auxquelles des relations libres avec les autres pays pourraient sérieusement causer des dommages.
Mais les capitaux engagés dans ces deux industries n’excèdent pas, y compris les matières premières, 17 à 18 millions de livres sterling (425 à 450 millions de fr.). C’est à peu près la septième ou huitième partie de la valeur de toutes nos manufactures. La même proportion existe entre le nombre d’hommes que ces deux industries emploient, et la population de nos manufactures. (Tables statistiques de l’empire britannique, voyez articles toiles et soieries.) D’ailleurs l’importation libre, des toiles et soieries ne ruinerait qu’une très-petite partie de ces manufactures. Aucune branche de l’industrie linière ne souffrirait d’une réduction progressive des droits d’importation sur les toiles. Si les Français excellent dans la fabrication des soieries légères, en revanche nous leur sommes supérieurs, ou au moins égaux dans la fabrication des gants et dans la bonneterie, et les étoffes mêlées, dont la soie forme la base. Nous sommes également leurs rivaux pour l’éclat des couleurs et la durée de la teinture. Il résulte de documents communiqués au comité de la Chambre des communes que citait chose ordinaire que d’assurer à Londres, moyennant une prime de 10 à 15 pour 100, la livraison des soieries françaises. C’est donc moins aux règlements prohibitifs qu’à leur véritable habileté que nos fabricants de soieries doivent le monopole sur le marché dont ils ont joui pendant si longtemps. Mais ce sont précisément des règlements de douanes qui, en les protégeant, les ont rendus indifférents pour toute espèce d’amélioration ; et aujourd’hui même, sous le rapport des machines, nos fabricants se trouvent être inférieurs à ceux de France et d’Allemagne. La sagacité de M. Huskisson s’aperçut bien vite des causes de cette infériorité, et il eut le courage d’entreprendre un changement de système. Ce changement eut lieu en 1826. Les droits sur la soie écrue furent réduits. Ceux sur la soie torse ou organsinée furent diminués d’une manière notable ; en même temps la prohibition contre les soieries étrangères fut abolie et leur entrée accordée moyennant un droit de 30 pour 100 ad valorem. — Le nouveau système fut attaqué avec véhémence. On crut y voir la ruine des manufactures ; mais toutes ces craintes étaient sans fondement. La mesure, au contraire, eut un succès incontestable. Les fabricants, voyant qu’ils ne pouvaient plus compter sur la protection des lois douanières, employèrent toute leur énergie, et appelant à leur aide toutes les ressources de la science et de leur habileté, ils firent faire à cette industrie, pendant les douze années finissant en 1837, plus de progrès qu’elle n’en avait fait dans tout le siècle précédent. Les importations de matières premières et les exportations d’articles fabriqués augmentèrent rapidement ; et maintenant (1838) les capitaux engagés dans cette industrie s’élèvent à la somme énorme de 10,000,000 livres sterling (250,000,000 fr.), et nous exportons des quantités très-considérables de soieries même pour la France. Mac Culloch.
↑ Ce mot de lainages comprend non-seulement tous les articles de draperie et étoffes de laine, comme serges, flanelles, etc., mais encore tous ceux de bonneterie en laine, couvertures, etc.
↑ Tome I, livre I, chap. x.
↑ Voyez sur les privilèges des corporations, sur la loi de l’apprentissage et sur celle du domicile, la seconde section du chap. x du liv. 1.
↑ Une pareille croyance parait aujourd’hui beaucoup moins absurde qu’elle n’a pu l’être en 1775. Depuis 1825 de grands pas ont été faits vers la liberté du commerce, et il n’est pas chimérique d’admettre aujourd’hui qu’un jour elle pourra être entièrement établie. Il faut se rappeler seulement qu’en parlant de liberté du commerce on ne prétend pas dire que les marchandises doivent être exemptes de toute espèce de droits ; mais on désire que le commerce ne soit pas entravé par des prohibitions frappant l’importation ou l’exportation. On ne veut pas que des droits soient établis dans un but de protection pour quelque industrie indigène, ou dans tout autre intérêt que celui des revenus du trésor. Des droits établis dans ce dernier but pourront être onéreux, mais ils ne seront pas une violation du principe de la liberté.
Mac Culloch.
Bien que nous ne puissions pas espérer de voir un système de liberté parfaite s’établir jamais en Angleterre, nous pouvons du moins admettre que la propagation des vrais principes contribuerait i saper par la base ces préjugés absurdes qui jusqu’à présent ont favorisé ce système d’entraves appliqué au commerce. Déjà le livre du docteur Smith a produit une véritable révolution dans l’opinion publique sous ce rapport, et dans les derniers temps la politique commerciale de son pays s’est ressentie de reflet de ses doctrines.
Buchanan.
↑ Les marchands ont maintenant des vues beaucoup plus libérales, plus larges et plus utiles même à leurs véritables intérêts. Comme preuve de ce progrès, il suffira de renvoyer à la pétition signée et présentée à la Chambre des communes, en 1820, par les négociants les plus considérables de Londres. Les avantages d’une concurrence illimitée y sont reconnus de la manière la plus explicite, des vœux y sont émis pour l’abolition des prohibitions et règlements conçus en vue de protéger l’industrie du pays, ainsi que des droits sur l’importation qui auraient d’autres buts que le revenu du Trésor. Une pareille pétition commence une ère nouvelle dans l’histoire du commerce ; elle prouve que les différences de vues, qui séparaient autrefois les théoriciens et les hommes pratiques ont entièrement disparu. Si M. Smith avait pu prévoir que ses principes finiraient par triompher, et que le système mercantile serait condamné par les négociants les plus considérables et les plus éclairés du monde, peut être aurait-il quelque peu tempéré la rigueur de ses observations sur la rapacité mercantile, dans ce paragraphe et dans plusieurs autres *.
Mac Culloch.
*. Il avait été stipulé par le traité de Methuen, que les vins du Portugal payeraient un tiers moins que ceux de France. La réduction d’un tiers opérée sur les droits qui frappaient les vins de France entraînait nécessairement, conformément à la convention conclue avec le Portugal, la réduction d’an tiers sur les droits payés jusqu’alors par les vins du Portugal, condition qui fut mise en vigueur d’un accord commun avec le gouvernement portugais. Buchanan.

%%%%%%%%%%%%%%%%%%%%%%%%%%%%%%%%%%%%%%%%%%%%%%%%%%%%%%%%%%%%%%%%%%%%%%%%%%%%%%%%
%                                  Chapitre 3                                  %
%%%%%%%%%%%%%%%%%%%%%%%%%%%%%%%%%%%%%%%%%%%%%%%%%%%%%%%%%%%%%%%%%%%%%%%%%%%%%%%%

\chapter{Des entraves extraordinaires apportées à l’importation de presque toutes les espèces de marchandises, des pays avec lesquels on suppose la balance du commerce défavorable}
\markboth{Des entraves extraordinaires apportées à l’importation de presque toutes les espèces de marchandises, des pays avec lesquels on suppose la balance du commerce défavorable}{}

Le second expédient au moyen duquel le système mercantile se propose d’aug­men­ter la qualité de l’or et de l’argent consiste à établir des entraves extraordinaires à l’importation de presque toute espèce de marchandises venant des pays avec lesquels on suppose que la balance du commerce est défavorable. Ainsi, dans la Grande-Bretagne, l’importation des linons de Silésie, pour la con­som­mation intérieure, est permise, à la charge de payer certains droits ; mais l’impor­tation des batistes et des linons de France est prohibée[1], excepté pour le port de Lon­dres, où ils sont déposés dans des magasins, à charge d’être réexportés. Il y a de plus forts droits sur les vins de France que sur ceux de Portugal, ou même de tout autre pays. Par ce qu’on appelle l’impôt de 1692[2], il a été établi un droit de 25 pour 100 de la valeur ou du prix au tarif sur toutes les marchandises de France, tandis que les mar­chandises des autres nations ont été, pour la plupart, assujetties à des droits beaucoup plus légers, qui rarement excèdent 5 pour 100. À la vérité, les vins, eaux-de-vie, sels et vinaigres de France ont été exceptés, ces denrées étant assujetties à d’autres droits très-lourds, soit par d’autres lois, soit par des clauses particulières de cette même loi. En 1696, ce premier droit de 25 pour 100 n’ayant pas été jugé un découragement suffisant, on en imposa un second, aussi de 25 pour 100, sur toutes les marchandises françaises, excepté sur les eaux-de-vie ; et en même temps, un nouveau droit de 25 livres par tonneau[3] de vin de France, et un autre de 15 livres par tonneau de vinaigre de France ; les marchandises de France n’ont été omises dans aucun de ces subsides généraux ou droits de 5 pour 100, qui ont été imposés sur toutes ou sur la plus grande partie des marchandises énoncées et détaillées dans le livre des tarifs. Si nous comptons le tiers et les deux tiers de subside comme faisant entre eux un subside entier, il y a eu cinq de ces subsides généraux ; de manière qu’avant le commencement de la guerre actuelle, on peut regarder 75 pour 100 comme le moindre droit auquel fussent assujetties la plupart des marchandises fabriquées ou produites en France. Or, sur la plupart des marchandises, de tels droits sont équivalents à une prohibi­tion. Les Français, de leur côté, ont, à ce que je crois, maltraité tout autant nos den­rées et nos manufactures, quoique je ne sois pas également au fait de toutes les char­ges et gênes qu’ils leur ont imposées. Ces entraves réciproques ont à peu près anéanti tout commerce loyal entre les deux nations, et c’est maintenant par les contrebandiers que se fait principalement l’importation des marchandises anglaises en France, ou des marchandises françaises en Angleterre. Les principes que j’ai examinés dans le chapitre précédent ont leur source dans l’intérêt privé et dans l’esprit de monopole ; ceux que je vais examiner maintenant ont la leur dans les préjugés et dans la haine nationale ; aussi sont-ils, comme on doit bien s’y attendre, beaucoup plus déraisonnables encore ; ils le sont, en partant même des propres principes du système que je combats.
Premièrement, quand même il serait constant que, dans le cas d’une liberté de com­merce entre la France et l’Angleterre, par exemple, la balance dût être en faveur de la France, il ne s’ensuivrait nullement pour cela qu’un tel commerce dût être désa­van­tageux à l’Angleterre, ou que la balance générale de la totalité du commerce anglais dût en être pour cela plus défavorable. Si les vins de France sont meilleurs et moins chers que ceux de Portugal, ou ses toiles meilleures ou moins chères que celles d’Allemagne, il sera plus avantageux à la Grande-Bretagne d’acheter de la France, plutôt que du Portugal et de l’Allemagne, les vins et les toiles qu’elle a besoin de tirer de l’étranger. Quoique par là la valeur de nos importations annuelles de France se trouvât fort augmentée, néanmoins la valeur de la somme totale de nos importations diminuerait de toute la quantité dont les marchandises françaises de même qualité seraient moins chères que celles des deux autres pays ; c’est ce qui arriverait même dans le cas où la totalité des marchandises françaises importées serait pour la con­som­mation de la Grande-Bretagne.
Mais, en second lieu, une grande partie de ces marchandises pourrait être exportée à d’autres pays, où, étant vendue avec profit, elle rapporterait un retour équivalant peut-être au premier achat du total des marchandises françaises importées. Ce qu’on a dit si souvent du commerce des Indes orientales pourrait peut-être avoir lieu pour celui de France ; quoique la plus grande partie des marchandises de l’Inde fussent achetées avec de l’or et de l’argent, la réexportation d’une partie de ces marchandises aux autres pays rapportait plus d’or et d’argent au pays qui faisait ce commerce, que ne lui en avait coûté le premier achat de la totalité. Aujourd’hui, une des branches les plus importantes du commerce de la Hollande consiste dans le transport des mar­chandises de France aux autres pays de l’Europe ; une partie même des vins de France qui se boivent en Grande-Bretagne sont importés en fraude de la Hollande et de la Zélande. S’il y avait liberté de commerce entre la France et l’Angleterre, ou seulement si l’on avait la faculté d’importer les marchandises françaises en payant les mêmes droits qu’on paye sur celles des autres nations de l’Europe, à charge de la restitution du droit lors de la réexportation, l’Angleterre pourrait alors avoir quelque part dans un commerce qui est regardé comme si avantageux à la Hollande.
Troisièmement, nous n’avons aucun indice certain sur lequel nous puissions juger de quel côté penche entre deux pays ce qu’on appelle la balance du commerce, ou lequel des deux exporte pour une plus grande valeur ; les préjugés et la haine natio­nale, excités toujours par l’intérêt particulier des marchands, sont les principes qui dirigent, en général, notre jugement sur toutes les questions relatives à ce sujet. Il y a cependant deux indices qu’on a souvent appelés en témoignage dans ces occasions, les registres des douanes et le cours du change.
Quant aux registres des douanes, je crois qu’il est généralement reconnu aujour­d’hui que c’est un indice fort incertain, à cause de l’inexactitude avec laquelle la plupart des marchandises y sont évaluées. Le cours du change est peut-être un indice tout aussi incertain.
Quand le change entre deux places, telles que Paris et Londres est au pair, c’est un signe, dit-on, que les sommes dues par Londres à Paris sont compensées par celles que Paris doit à Londres ; au contraire, quand on paye à Londres une prime pour avoir une lettre de change sur Paris, c’est signe, dit-on, que les sommes dues par Londres à Paris ne sont pas balancées par celles que Paris doit à Londres, mais que cette dernière place doit solder une balance en argent ; l’exportation de cette somme d’ar­gent offrait quelque risque à courir, de la peine à prendre et des frais à faire, on demande et on accorde une prime comme indemnité. Or, ajoute-t-on, le résultat ou la situation ordinaire des dettes et créances respectives entre ces deux villes doit nécessairement se régler sur le cours ordinaire des affaires qu’elles font l’une avec l’autre. Quand aucune des deux n’importe de chez l’autre pour une plus grande somme qu’elle ne lui exporte, les dettes et créances respectives de chacune d’elles doivent se balancer ; mais quand l’une des deux importe de chez l’autre pour une plus grande somme qu’elle ne lui exporte, la première devient nécessairement débitrice de la dernière d’une plus grande somme que celle-ci n’est débitrice envers elle ; les dettes et créances respectives de chacune ne se balancent plus les unes par les autres, et la place dont les dettes excèdent les créances est obligée d’envoyer de l’argent. Par conséquent, le cours ordinaire du change étant une indication de la situation ordinaire des dettes et créances respectives entre deux places, il doit être pareillement une indication du cours ordinaire de leurs exportations et importations respectives, celles-ci déterminant nécessairement l’état de situation des créances et des dettes.
Mais quand même on accorderait que le cours ordinaire du change pût être une indication suffisante de la situation ordinaire des dettes et créances respectives entre deux places, il ne s’ensuivrait pas de là que la balance du commerce penchât du côté de la place qui aurait en sa faveur l’état de situation ordinaire des dettes et créances. L’état de situation ordinaire des dettes et créances respectives entre deux places ne se règle pas toujours uniquement par le cours ordinaire des affaires de commerce qu’elles font l’une avec l’autre ; mais il se ressent souvent des affaires que peuvent faire l’une ou l’autre d’elles avec plusieurs autres places. Par exemple, si les com­merçants anglais sont dans l’usage de payer, en lettres de change sur la Hollande, les marchandises qu’ils achètent de Hambourg, Dantzick, Riga, etc., l’état de situation ordinaire des dettes et créances respectives entre l’Angleterre et la Hollande ne se réglera pas toujours uniquement sur le cours ordinaire des affaires de commerce faites entre ces deux pays, mais il se ressentira des affaires que l’Angleterre aura traitées avec les trois autres places. L’Angleterre pourrait être obligée d’envoyer chaque année de l’argent en Hollande, quoique ses exportations annuelles en ce pays excédassent de beaucoup la valeur de ce qu’elle en importerait annuellement, et quoique ce qu’on appelle la balance du commerce pût être de beaucoup en faveur de l’Angleterre.
D’ailleurs, de la manière dont le pair du change a été calculé jusqu’ici, le cours ordinaire du change ne peut pas fournir d’indication suffisante pour assurer si la situation ordinaire des dettes et créances respectives est en faveur du pays qui paraît avoir ou qui est supposé avoir en sa faveur le cours ordinaire du change, ou, en d’autres termes, le change réel peut être et est souvent, dans le fait, tellement différent du change tel qu’il est escompté dans le cours public des changes, que la plupart du temps on ne peut rien conclure de certains du cours de ce dernier, relativement à l’état du change véritable.
Quand pour une somme d’argent payée en Angleterre, contenant, d’après le titre de la monnaie, un certain nombre d’onces d’argent fin, vous recevez une lettre de change pour une somme d’argent payable en France, contenant, d’après le titre de la monnaie de France, un pareil nombre d’onces d’argent fin, on dit que le change est au pair entre la France et l’Angleterre. Quand vous payez plus, vous êtes censé donner une prime, et alors on dit que le change est contre l’Angleterre et en faveur de la France ; quand vous payez moins, vous êtes censé gagner une prime, et alors on dit que le change est contre la France et en faveur de l’Angleterre.
Mais, premièrement, nous ne pouvons pas toujours juger de la valeur de la mon­naie courante de différents pays, par le titre et le poids de fabrication. En quelques pays, la monnaie est plus ou moins usée, plus ou moins rognée ou autrement dégra­dée de sa valeur primitive, que dans d’autres. Or, la valeur de la monnaie courante d’un pays, comparée avec celle d’un autre, est en proportion, non pas de la quantité d’argent fin qu’elle devrait contenir, mais bien de celle qu’elle contient en effet pour le moment. Avant la refonte de la monnaie d’argent au temps du roi Guillaume, le change entre l’Angleterre et la Hollande, calculé suivant la méthode ordinaire, d’après le titre et le poids de fabrication de leurs monnaies respectives, était de 25 pour 100 contre l’Angleterre. Mais la valeur de la monnaie courante d’Angleterre, comme nous l’apprend M. Lowndes, était à cette époque de plus de 25 pour 100 au-dessous de sa valeur de fabrication. Par conséquent, le change réel pouvait à cette époque être en faveur de l’Angleterre, encore que le change, tel qu’on le comptait sur la place, fût si fort contre elle ; il pouvait se faire que le nombre d’onces d’argent fin qu’on payait à cette époque en Angleterre pour l’achat d’une lettre de change sur la Hollande, achetât un nombre plus grand d’onces d’argent fin payable dans ce dernier pays, et que celui qui était censé donner la prime la reçût en réalité. Avant la dernière refonte de notre monnaie d’or, la monnaie de France était moins usée que la monnaie anglaise, et était peut-être de 2 ou 3 pour 100 plus près de son poids légal. Par conséquent, si le change au cours de la place n’était pas de plus de 2 ou 3 pour 100 pour la France contre l’Angleterre, alors le change réel pouvait être en notre faveur. Depuis la refonte de la monnaie d’or, le change a été constamment en faveur de l’Angleterre et contre la France.
Secondement, dans quelques pays la dépense du monnayage est défrayée par le gouvernement ; dans d’autres, elle se fait aux frais des particuliers, qui portent leurs lingots à la Monnaie, et le gouvernement tire même quelque revenu du monnayage. En Angleterre, cette dépense est défrayée par le gouvernement, et si vous portez à la Monnaie une livre pesant d’argent au titre, vous en retirez 62 schellings, contenant une pareille livre d’argent au titre. En France, on retient pour le monnayage un droit de 8 pour 100[4], qui non-seulement défraye la dépense de fabrication, mais qui rapporte encore un petit revenu au gouvernement[5]. En Angleterre, comme le monnayage ne coûte rien, la monnaie courante ne peut jamais avoir beaucoup plus de valeur que la quantité de métal qu’elle se trouve contenir pour le moment. En France, comme on paye pour la fabrication, elle ajoute à la valeur de la monnaie, comme la façon ajoute à celle de la vaisselle. Par conséquent, une somme de monnaie française, contenant un poids quelconque d’argent fin, a plus de valeur qu’une somme de monnaie anglaise contenant un pareil poids d’argent fin, et il faut plus de métal ou plus de toute autre marchandise pour acheter la première somme. Ainsi, quand même la monnaie cou­rante de chacun de ces deux pays se trouverait approcher également de son poids de fabrication respectif, une somme de monnaie anglaise ne pourrait guère acheter une somme de monnaie française, contenant le même nombre d’onces d’argent fin, ni, par conséquent, une lettre de change sur France de pareille somme. Si la somme payée en sus pour acheter cette lettre de change n’était tout juste que ce qu’il faut pour com­pen­ser les frais de fabrication de la monnaie française, alors il se pourrait que le change réel fût au pair entre les deux nations, que leurs créances et leurs dettes respectives se balançassent mutuellement les unes par les autres, tandis que le change au cours de la place paraîtrait être considérablement en faveur de la France. Si la somme payée en sus était moindre que l’équivalent de cette compensation, le change réel pourrait être en faveur de l’Angleterre, quoique le cours parût être en faveur de la France.
Troisièmement enfin, sur certaines places, telles que Amsterdam, Hambourg, Venise, etc., les lettres de change étrangères se payent en ce qu’on appelle argent de banque, tandis que sur d’autres places, comme Londres, Lisbonne, Anvers, Livourne, etc., elles se payent en espèces courantes du pays. Ce qui se nomme argent de banque est toujours d’une valeur supérieure à la même somme nominale en espèces cou­ran­tes. À Amsterdam, par exemple, mille florins en banque valent plus de mille flo­rins argent courant d’Amsterdam. La différence entre ces deux espèces de monnaie se nomme agio de la banque, lequel à Amsterdam est, en général, environ de 5 pour 100. Supposez que les espèces courantes de deux pays soient également rapprochées du poids de fabrication de leurs monnaies respectives, et que l’un paye les lettres de change étrangères avec ces espèces courantes, tandis que l’autre les paye en argent de banque, il est évident que le cours du change peut être en faveur du temps qui paye en argent de banque, quoique le change réel soit en faveur de celui qui paye en espèces courantes, par la même raison que le cours du change peut être en faveur du pays qui paye en argent de banque, quoique le change réel qui approche plus de son premier poids, quoique le change réel soit en faveur du pays qui paye en une monnaie infé­ri­eure. Avant la dernière refonte de notre monnaie d’or, le cours du change avec Amsterdam, Hambourg, Venise, et je crois, avec toutes les autres places qui payaient en ce qu’on nomme argent de banque, était, en général, contre Londres. Il ne s’ensuit pas pourtant pour cela que le change réel fût contre nous ; depuis la refonte de notre monnaie d’or, il a été en faveur de Londres, même avec ces places. Le cours était généralement en faveur de Londres avec Lisbonne, Anvers, Livourne, et je crois, à l’exception de la France, avec la plupart des autres pays de l’Europe qui payent en espèces courantes ; et il est assez vraisemblable que le change réel l’était aussi[6].

DIGRESSION

Sur les banques de dépôt, et en particulier sur celle d’Amsterdam.

Les espèces courantes d’un grand État tel que la France ou l’Angleterre, consistent en général presque en entier dans sa propre monnaie. S’il arrive donc, dans un temps, que ces espèces se trouvent usées, rognées ou détériorées de toute autre manière, l’État, par une refonte, parviendra sûrement à rétablir sa monnaie courante. Mais les espèces courantes d’un petit État, tel que Gênes ou Hambourg, ne peuvent guère con­sister entièrement dans sa propre monnaie ; elles se composent nécessairement en grande partie des monnaies de tous les États voisins avec lesquels ses habitants ont une communication continuelle. Ainsi, un tel État, en réformant sa propre monnaie, ne viendrait pas toujours à bout de réformer ses espèces courantes. Si les lettres de change étrangères y sont payées avec ces espèces courantes, l’incertitude de la valeur réelle de la somme qu’on recevra en une chose qui par sa nature est si peu certaine, doit rendre le cours du change toujours très-contraire à un État tel que celui-ci, tous les États étrangers évaluant sa monnaie courante nécessairement même au-dessous de ce qu’elle vaut.
Quand ces petits États ont commencé à porter leur attention sur les intérêts de leur commerce, pour obvier aux désavantages auxquels cette défaveur du change exposait leurs négociants, il leur est arrivé souvent de statuer que les lettres de change étrangères d’une certaine valeur ne seraient pas payées en espèces courantes, mais en un ordre ou transfert sur les livres d’une banque établie sur le crédit de l’État et sous sa protection, cette banque étant toujours tenue de payer en bon argent, exactement conforme au titre et au poids primitif de la monnaie de l’État. Il paraît que c’est dans cette vue qu’ont été originairement établies les banques de Venise, de Gênes, d’Ams­terdam, de Hambourg et de Nuremberg, quoique quelques-unes d’entre elles, par la suite, aient pu servir à d’autres destinations. La monnaie de ces banques étant meil­leure que les espèces courantes du pays, elle produisit nécessairement un agio qui fut plus ou moins élevé, selon que les espèces courantes étaient réputées plus ou moins dégradées au-dessous du poids primitif de leur fabrication. L’agio de la banque de Hambourg, par exemple, qu’on dit être communément de 14 pour 100 environ, est la différence qu’on suppose exister entre la bonne monnaie de l’État au titre et au poids primitif de sa fabrication, et les monnaies courantes usées, rognées et détériorées qui y sont versées par tous les États voisins.
Avant 1609, la grande quantité de monnaie étrangère usée et rognée, que le com­merce immense d’Amsterdam lui apportait de tous les coins de l’Europe, réduisit la valeur de sa monnaie courante à environ 9 pour 100 au-dessous de la valeur de la bonne monnaie neuve sortant de la fabrication ; celle-ci ne paraissait pas plutôt dans le commerce, qu’elle était fondue ou exportée, comme il arrive toujours en pareil cas. Les marchands, qui regorgeaient de monnaie courante, ne pouvaient pas toujours trouver assez de bonne monnaie pour acquitter leurs lettres de change, et la valeur de ces lettres de change devint variable jusqu’à un certain point, en dépit de plusieurs règlements qu’on fit pour l’empêcher.
En vue de porter remède à ces inconvénients, on établit, en 1609, une banque sous la garantie de la ville. Cette banque reçut tant les monnaies étrangères que la monnaie du pays, usée et hors de poids, sur le pied de leur valeur intrinsèque, payable en bonne monnaie au titre et au poids légal, en déduisant seulement ce qui était nécessaire pour payer les frais du monnayage et les autres dépenses indispensables de l’administration. Pour la valeur qui restait après cette légère déduction, elle donnait un crédit sur ses livres. Ce crédit s’appela argent de banque ; et comme il représentait précisément la monnaie suivant son poids primitif de fabrication, il conservait toujours sa même valeur réelle, et il valait mieux intrinsèquement que la monnaie cou­rante. Il fut statué en même temps que toutes les lettres de change tirées sur Amsterdam ou négociées dans cette place, de la valeur de 600 florins et au-delà, seraient payées en argent de banque, ce qui ôta dès lors toute espèce d’incertitude dans la valeur de ces lettres[7]. En conséquence de ce règlement, tout commerçant fut obligé de tenir un compte avec la banque, à l’effet de payer ses lettres de change de l’étranger ; ce qui nécessairement donna lieu à une demande assez considérable d’argent de banque.
Outre sa supériorité intrinsèque sur la monnaie courante et la valeur additionnelle que lui donne nécessairement cette demande, l’argent de banque a encore quelques autres avantages. Il ne craint ni le feu, ni les voleurs, ni les autres accidents ; la ville d’Amsterdam est engagée au payement ; on peut payer avec cet argent par un simple transfert[8], sans avoir la peine de compter et sans courir le risque du transport d’un lieu dans un autre[9]. D’après tous ces divers avantages, il paraît que dès le commencement il a produit un agio, et on croit, en général, que toutes les sommes d’argent déposées originairement dans la banque y ont été laissées, personne ne se souciant de deman­der le payement d’une créance qu’il pouvait vendre sur la place avec bénéfice. En demandant son payement à la banque, le propriétaire d’un crédit sur la banque per­drait ce bénéfice. Un schelling tout neuf sortant de dessous le balancier, n’achètera certainement pas plus de marchandises au marché qu’un de nos vieux schellings ordinaires, tout usés qu’ils sont ; de même, la bonne monnaie de poids qui serait sortie des coffres de la banque pour aller dans ceux d’un particulier, étant une fois mêlée et confondue avec la monnaie courante ordinaire du pays, n’aurait pas eu plus de valeur que cette monnaie courante, de laquelle il n’y aurait plus eu moyen de la distinguer. Tant que cette monnaie restait dans les coffres de la banque, sa supériorité était con­nue et légalement constatée. Mais, une fois versée dans les coffres d’un particulier, il n’était plus guère possible d’en constater la supériorité, à moins de prendre plus de peine que peut-être n’eût valu la différence. D’ailleurs, étant une fois sortie des coffres de la banque, elle perdait encore tous ses autres avantages d’argent de banque, sa sûreté, sa facilité à être transportée sans peine et sans risque, sa faculté de servir au payement des lettres de change étrangères. Pardessus tout cela enfin, on ne pouvait pas la faire sortir de ces coffres, comme on va le voir tout à l’heure, sans payer préalablement quelque chose pour frais de garde.
Ces dépôts d’argent monnayé, ou ce que la banque était obligée de rendre en bon argent monnayé, constituaient le capital originaire de la banque, ou la valeur totale de ce qui était représenté par ce qu’on appelait argent de banque. Aujourd’hui, cela est censé n’en constituer qu’une très-petite partie. Dans la vue de faciliter le commerce des lingots, la banque a adopté, depuis plusieurs années, la pratique de donner crédit sur ses livres, moyennant un dépôt d’or ou d’argent en lingots. Ce crédit est, en général, de 5 pour 100 environ au-dessous du prix pour lequel ces lingots passent à la Monnaie. La banque délivre en même temps ce qu’on nomme un reçu ou un récépissé, portant que « la personne dépositaire ou le porteur du récépissé pourra retirer en une seule fois, dans un terme de six mois, les lingots déposés, en refaisant un transfert, au profit de la banque, d’une quantité d’argent de banque, égale à celle pour laquelle il lui a été donné crédit sur les livres lors du dépôt, et à la charge de payer un quart pour 100 pour la garde si le dépôt a été fait en argent, et un demi pour 100 s’il a été fait en or[10] » ; mais portant aussi déclaration que, « à défaut de ce paye­ment, à l’expiration dudit terme, le dépôt appartiendra à la banque, au prix pour lequel il a été reçu, ou pour lequel il a été accordé crédit par transfert sur les livres ». Ce qui est ainsi payé pour la garde du dépôt peut être regardé comme une sorte de loyer de magasin ; et si ce loyer de magasin est ainsi fixé beaucoup plus haut pour l’or que pour l’argent, on en a donné plusieurs raisons différentes. Le degré de fin de l’or, a-t-on dit, est plus difficile à constater que celui de l’argent. La fraude est plus aisée sur ce métal, et attendu son plus grand prix, elle entraîne plus de perte. D’ailleurs, l’argent étant le métal qui sert de mesure à l’autre, l’État, a-t-on dit, veut encourager à faire des dépôts d’argent plutôt que des dépôts d’or.
Les dépôts de lingots se font le plus communément quand le prix du lingot est de quelque chose plus bas qu’à l’ordinaire, et on les retire quand ce prix vient à hausser. En Hollande, le prix de marché du lingot est, en général, au-dessous du prix qu’en donne la Monnaie, par la même raison qu’il en était ainsi en Angleterre avant la dernière refonte de la monnaie d’or. On dit que la différence va ordinairement de 6 à 16 stivers[11] environ par marc[12] ou par 8 onces d’argent à 11 deniers de fin. Le prix de la banque ou le crédit que la banque donne pour des dépôts d’argent à ce titre (quand le dépôt est fait en monnaies étrangères, dont le degré de fin est bien connu et bien constaté, tels que les dollars du Mexique), est de 22 florins le marc. Le prix, à la Monnaie, est à environ 23 florins, et le prix du marché est depuis 23 florins 6 stivers, jusqu’à 23 florins 16 stivers, ou de 2 à 3 pour 100 au-dessous du prix qu’on en donne à la Monnaie[13]. Les prix du lingot d’or à la banque, à la Monnaie et au marché sont, entre eux trois, en proportion à peu près pareille à celle ci-dessus. En général, une personne peut vendre son récépissé pour la différence de prix entre le prix du marché du lingot et son prix à la Monnaie. Un récépissé pour lingot vaut toujours quelque chose et, par conséquent, il arrive rarement que quelqu’un laisse expirer son récépissé, ou bien laisse tomber ses lingots à la banque au prix qui en a été donné, soit faute de les retirer avant l’expiration des six mois, soit faute d’avoir l’attention de payer le quart ou le demi pour 100, à l’effet d’obtenir un nouveau récépissé pour six autres mois. Ce­pen­dant, quoique cela arrive rarement, cela se voit quelquefois, et plus souvent à l’égard de l’or qu’à l’égard de l’argent, à cause du droit plus fort qui se paye pour la garde du métal le plus précieux.
La personne qui, au moyen d’un dépôt de lingots, obtient un crédit sur la banque et un récépissé, paye ses lettres de change à leur échéance avec son crédit sur la banque et, quant à son récépissé, elle le vend ou elle le garde, selon qu’elle présume que le prix du lingot est dans le cas de baisser ou de hausser. Le récépissé et le crédit sur la banque restent rarement longtemps dans la même main, et il n’est pas besoin qu’ils y restent. La personne qui a un récépissé et qui veut retirer des lingots trouve toujours en abondance des crédits sur la banque ou de l’argent de banque à acheter au prix ordinaire, et la personne qui a de l’argent de banque et qui a besoin de retirer des lingots trouve toujours des récépissés en aussi grande abondance.
Les propriétaires de crédits sur la banque et les porteurs des récépissés forment deux différentes sortes de créanciers à l’égard de la banque.
Le porteur d’un récépissé ne peut retirer le lingot pour lequel ce récépissé a été délivré, sans rétrocéder à la banque une somme, en argent de banque, égale au prix auquel le lingot a été reçu. S’il n’a pas d’argent de banque, il faut qu’il en achète de ceux qui en ont. Le propriétaire d’argent de banque ne peut retirer de lingots, à moins de présenter à la banque des récépissés représentant la valeur des lingots dont il a besoin. S’il n’a pas de récépissé, il faut qu’il en achète de ceux qui en ont. Quand le porteur d’un récépissé achète de l’argent de banque, il achète la faculté de retirer une quantité de lingots qui valent, au prix de la Monnaie, 5 pour 100 au-dessus du prix donné par la banque. Ainsi, l’agio de 5 pour 100 qu’il paye communément pour avoir cet argent de banque, il ne le paye pas pour une valeur imaginaire, mais bien pour une valeur réelle. Quand le propriétaire d’argent de banque achète un récépissé, il achète la faculté de retirer une quantité de lingots, qui valent, au prix de marché, communé­ment de 2 à 3 pour 100 au-dessus du prix qu’on en donne à la Monnaie. Le prix qu’il donne pour ce récépissé est donc également payé pour une valeur réelle. Le prix du récépissé et le prix de l’argent de banque composent ou complètent entre eux la valeur entière ou le prix du lingot.
La banque donne pareillement des récépissés, aussi bien que des crédits, sur des dépôts de monnaie courante du pays ; mais ces récépissés sont souvent sans valeur ou n’ont pas de prix au marché. Par exemple, sur les ducatons, qui passent dans la mon­naie courante pour 3 florins 3 stivers chacun, la banque donne un crédit de 3 florins seulement, ou de 5 pour 100 au-dessous de leur valeur courante. Elle délivre pareillement un récépissé qui autorise le porteur à retirer en une fois, et dans un terme de six mois, le nombre de ducatons déposés, en payant pour la garde le droit d’un quart pour 100. Le plus souvent, ce récépissé n’aura point de valeur au marché. Trois florins, argent de banque, se vendent généralement sur la place 3 florins 3 stivers, valeur totale des ducatons, s’ils étaient retirés de la banque ; et encore il faudrait, avant de pouvoir les retirer, payer le quart pour le droit de garde, ce qui serait en pure perte pour le porteur du récépissé. Cependant, si l’agio de la banque venait une fois à tomber à 3 pour 100, alors ces sortes de récépissés pourraient avoir un prix au mar­ché, et ils pourraient se vendre à un et trois quarts pour 100. Mais l’agio de la banque étant aujourd’hui, en général, à 5 pour 100 environ, on laisse souvent ces sortes de récépissés expirer, ou, comme on dit, tomber à la banque. Les récépissés qui sont délivrés pour des dépôts de ducats d’or, y tombent encore plus fréquemment à cause du droit de dépôt plus fort, ou du demi pour 100 qu’il faut payer pour leur garde avant de pouvoir les retirer. Les 5 pour 100 que gagne la banque quand on laisse ces dépôts de monnaie ou de lingots tomber à la banque, peuvent être considérés comme un loyer de magasin pour la garde à perpétuité de ces dépôts.
L’argent de banque dont les récépissés sont expirés doit faire une somme très-considérable. Il faut y comprendre tout le capital originaire de la banque qui y a été laissé, à ce qu’on suppose généralement, depuis le temps où il a été déposé d’abord, personne ne se souciant ni de renouveler son récépissé ni de reprendre son dépôt, attendu que, par les raisons qui ont été exposées plus haut, on ne pouvait faire ni l’un ni l’autre sans perte. Mais quel que puisse être le montant de cette somme, elle est fort petite, à ce qu’on croit, si on la compare à la masse totale de l’argent de banque. Il y a déjà bien des années que la banque d’Amsterdam est le grand magasin de dépôt de toute l’Europe pour les lingots, et il arrive rarement qu’on laisse les récépissés pour lingots expirer, ou, comme on dit, tomber à la banque. On croit que la très-grande partie de l’argent de banque ou des crédits sur les livres de la banque ont été créés, depuis tout ce temps, par des dépôts de ce genre que les commerçants en lingots font et retirent sans cesse.
On ne peut faire aucune demande à la banque qu’en vertu d’un reçu ou récépissé. La portion bien plus petite d’argent de banque dont les récépissés sont expirés se trouve mêlée et confondue avec la portion, beaucoup plus grande, dont les récépissés sont encore en vigueur ; de manière que, quoiqu’il puisse y avoir une somme très-considérable d’argent de banque pour laquelle il n’y a point de récépissés, cependant il n’y a aucune somme ou portion spécifique d’argent de banque qui ne soit pas sujette à être demandée à tout moment en vertu d’un récépissé. La banque ne peut devoir la même chose à deux personnes à la fois, et le propriétaire d’argent de banque qui n’a pas de récépissé ne peut demander de payement à la banque, à moins qu’il n’en achète un. Dans les temps calmes et ordinaires, il ne doit pas être difficile d’en trouver un à acheter au prix courant de la place, qui, en général, correspond au prix auquel il pourra vendre le lingot ou les espèces que ce récépissé l’autorise à retirer de la banque.
Il n’en serait pas de même au moment d’une calamité publique, d’une invasion, par exemple, telle que celle qui eut lieu de la part des Français en 1672. Les propriétaires d’argent de banque, en pareil cas, étant tous très-pressés de retirer leurs fonds de la banque, pour les avoir eux-mêmes entre leurs mains, la demande de récépissés serait telle qu’elle pourrait en faire monter le prix à un taux exorbitant. Les porteurs de récépissés pourraient faire la loi et spéculer sur des profits excessifs ; ils pourraient, au lieu de 2 ou 3 pour 100, exiger la moitié de l’argent de banque dont il a été donné crédit sur les dépôts pour lesquels les récépissés respectifs ont été délivrés. L’ennemi même, connaissant la constitution de la banque, pourrait faire acheter sur la place tous les récépissés possibles, afin d’empêcher que le trésor ne disparût. Dans des circonstances pareilles, on présume que la banque s’affranchirait de sa règle ordinaire, de ne payer qu’aux porteurs des récépissés. Les porteurs de récépissés qui n’ont pas d’argent de banque, ont dû nécessairement déjà toucher, à 2 ou 3 pour 100 près, toute la valeur des dépôts pour lesquels leurs récépissés respectifs ont été délivrés. En conséquence, dit-on, la banque, en pareil cas, ne se ferait pas scrupule de payer en espèces ou en lingots la totalité des valeurs pour lesquelles seraient crédités sur ses livres les propriétaires d’argent de banque qui n’auraient pu se procurer de récépissés, mais en payant en même temps 2 ou 3 pour 100 aux porteurs de récépissés, qui n’auraient pas d’argent de banque ; ce qui est tout ce qu’on peut présumer leur revenir avec justice dans un pareil état de choses.
Même dans les temps calmes et ordinaires, l’intérêt des porteurs de récépissés est de faire baisser l’agio, afin de pouvoir acheter d’autant meilleur marché l’argent de banque et, conséquemment, le lingot que leurs récépissés les mettraient pour lors en état de retirer de la banque, ou bien afin de pouvoir vendre d’autant plus cher leurs récépissés à ceux qui, n’ayant que de l’argent de banque sans récépissé, voudraient retirer des lingots, le prix du récépissé étant, en général, équivalent à la différence entre le prix de l’argent de banque au cours de la place, et celui des espèces ou lingots pour lesquels le récépissé a été délivré. L’intérêt des propriétaires d’argent de banque, au contraire, est de faire monter l’agio, afin de pouvoir vendre d’autant plus cher leur argent de banque, ou acheter le récépissé d’autant meilleur marché.
Pour empêcher toutes les manœuvres d’agiotage auxquelles ce conflit d’intérêts opposés pouvait quelquefois donner heu, la banque a pris, depuis quelques années, le parti de vendre en tout temps de l’argent de banque pour les espèces courantes, à l’agio de 5 pour 100, et de le racheter aussi, en tout temps, à 4 pour 100 d’agio. D’après cette résolution de la banque, l’agio ne peut jamais ni monter au-dessus de 5 pour 100, ni baisser au-dessous de 4, et la proportion entre le prix de l’argent de ban­que et celui des espèces courantes se maintient en tout temps sur la place, très-près de la véritable proportion de leurs valeurs intrinsèques. Avant que cette résolu­tion eût été prise, habituellement le prix de l’argent de banque, sur la place, tantôt montait jusqu’à 9 pour 100 d’agio, tantôt baissait jusques au pair, suivant que l’in­fluence de l’un ou de l’autre de ces deux intérêts opposés venait à dominer sur la place.
La banque d’Amsterdam fait profession de ne pas prêter la moindre partie des fonds qu’elle a en dépôt, mais de garder dans ses coffres, pour chaque florin dont elle donne crédit sur ses livres, la valeur d’un florin en argent ou lingot. Qu’elle garde dans ses coffres tout l’argent ou lingot dont il y a des récépissés en circulation, que par conséquent on peut lui demander d’un moment à l’autre, et qui dans le fait va et revient sans cesse dans ses coffres, c’est ce dont on ne peut guère douter. Mais qu’elle en fasse de même à l’égard de cette partie de son capital pour laquelle il n’y a que des récépissés, expirés depuis longtemps, qu’on ne peut jamais lui demander dans les temps calmes et ordinaires, et que vraiment on peut s’attendre à y voir rester pour toujours ou au moins aussi longtemps que subsisteront les États des Provinces-Unies, c’est ce qui paraîtra peut-être plus douteux. Cependant on croit à Amsterdam, comme à l’article de foi le mieux établi, que chaque florin qui circule comme argent de banque a son florin correspondant qu’on trouvera en tout temps en or ou argent dans le trésor de la banque ; c’est ce dont la ville est garante. La banque est sous la direction des quatre bourgmestres régnants, qui changent chaque année. À chacune de ces mu­ta­tions, les quatre bourgmestres nouveaux entrant en fonctions visitent le trésor, le vérifient en le comparant avec les livres, le reçoivent sous serment, et le délivrent l’année suivante avec les mêmes solennités et les mêmes formes, aux quatre fonc­tionnaires qui leur succèdent ; et chez cette nation sage et religieuse, les serments sont encore comptés pour quelque chose. Cette rotation continuelle d’administrateurs serait elle seule, à ce qu’il semble, une garantie suffisante contre toute manœuvre qui ne serait pas de nature à être avouée. Au milieu de toutes les révolutions que les diver­ses factions ont fait naître dans le gouvernement d’Amsterdam, en aucun temps on n’a vu le parti dominant accuser ses prédécesseurs d’infidélité dans l’administration de la banque. Aucun chef d’accusation n’eût été plus propre à porter au parti abattu des coups mortels pour son crédit et ses ressources, et s’il y avait eu moyen de soutenir un pareil grief, on peut être bien sûr qu’il aurait été mis en avant. En 1672, quand le roi de France était à Utrecht, la banque d’Amsterdam fit ses payements de manière à ne pas laisser de doute sur la fidélité avec laquelle elle avait respecté ses engagements. Quelques-unes des pièces qui furent alors retirées de ses coffres portaient encore l’empreinte du feu qui les avait attaquées lors de l’incendie arrivé à l’hôtel de ville peu de temps après l’établissement de la banque. Ainsi, ces pièces y étaient restées depuis cette époque.
Une question qui a longtemps exercé la curiosité des oisifs, c’est de savoir quel est le montant du trésor de la banque. On ne peut offrir là-dessus que de pures conjec­tures. En général, on compte qu’il y a environ deux mille personnes qui tiennent des comptes avec la banque, et en accordant que, l’une dans l’autre, elles aient en crédit sur leurs comptes respectifs 1500 livres sterling, ce qui est beaucoup, la totalité de l’argent de banque et, par conséquent, du trésor en caisse, s’élèverait à environ 3 mil­lions sterling, ou à 33 millions de florins. Cette somme est considérable et suffisante pour soutenir une circulation très-étendue ; mais il y a bien loin de là aux idées folles que quelques personnes se sont faites de ce trésor. 
La ville d’Amsterdam tire de la banque un revenu considérable, outre ce qu’on peut appeler le loyer de magasin ou droit de dépôt, dont nous avons parlé. Chaque personne qui ouvre pour la première fois un compte avec la banque paye un droit de 18 florins, et pour chaque nouveau compte, 3 florins 3 stivers ; pour chaque transfert sur les livres, on paye 2 stivers, et si le transfert est pour une somme au-dessous de 300 florins, on paye 6 stivers ; ce qui a eu pour objet d’empêcher que les petites opéra­tions ne devinssent trop multipliées[14]. Une personne qui néglige de régler son compte deux fois par an paye, par forme d’amende, 25 florins. Une personne qui passe à l’or­dre de quelqu’un un transfert pour une somme qui excède le crédit porté à son comp­te, est obligée de payer 3 pour 100 de l’excédant, et par-dessus le marché son ordre est mis au rebut[15]. On pense aussi que la banque fait un gros profit sur la vente des espèces étrangères ou des lingots qu’on lui laisse quelquefois faute de renouveler les récé­pissés, et qu’elle garde toujours jusqu’à ce qu’elle trouve le moment de les vendre avantageusement. Elle fait encore un profit en vendant l’argent de banque à 5 pour 100 d’agio, et le rachetant à 4. Ces divers bénéfices s’élèvent bien au-dessus de ce qui est nécessaire pour payer les gages des employés et défrayer les autres dépenses d’administration. Ce qui se paye seulement pour la garde des lingots sur récépissés monte, par année, à un revenu net de 150 à 200,000 florins. D’ailleurs, l’objet de cette institution a été l’utilité publique, et non le projet d’en tirer aucun revenu. Son but était de soulager le commerce des inconvénients d’un change défavorable ; le revenu qui en résulterait n’entrait pas dans le calcul, et on peut le regarder comme accidentel.
Mais il est bien temps de terminer cette longue digression, dans laquelle je me suis insensiblement laissé entraîner en cherchant à expliquer les raisons pour les­quelles, entre les pays qui payent en ce qu’on appelle argent de banque, et ceux qui payent en espèces courantes, le change paraît généralement être en faveur des pre­miers et contre les autres. Les premiers payent en une espèce de monnaie dont la valeur intrinsèque est toujours la même, et précisément conforme aux titre et poids de fabrication de leurs monnaies respectives ; les autres payent en une espèce de monnaie dont la valeur intrinsèque est dans le cas de varier sans cesse, et se trouve presque toujours plus ou moins au-dessous de ce poids de fabrication.

DEUXIÈME SECTION,
Où l’absurdité des règlements de commerce est démontrée d’après d’autres principes.

Dans la première partie de ce chapitre, j’ai cherché à faire voir combien, d’après les principes mêmes du système mercantile, il est inutile de mettre des entraves extra­or­dinaires à l’importation des marchandises tirées des pays avec lesquels on suppose la balance défavorable.
Mais toute cette doctrine de la balance du commerce, sur laquelle on fonde, non-seulement ces mesures, mais encore presque tous les autres règlements de commerce, est la chose la plus absurde qui soit au monde. Elle suppose que quand deux places commercent l’une avec l’autre, si la balance est égale des deux parts, aucune des deux places ne perd ni ne gagne ; mais que si la balance penche d’un côté à un certain de­gré, l’une de ces places perd, et l’autre gagne en proportion de ce dont la balance s’é­car­te du parfait équilibre. Ces deux suppositions sont également fausses. Un com­merce forcé, que l’on soutient à l’aide de primes et de monopoles, peut bien être et est même pour l’ordinaire désavantageux au pays en faveur duquel on s’est proposé de l’établir, comme je chercherai à le démontrer bientôt[16]. Mais un commerce qui se fait naturellement et régulièrement entre deux places, sans moyens de contrainte, est un commerce toujours avantageux à toutes les deux, quoiqu’il ne le soit pas toujours autant à l’une qu’à l’autre.
Par avantage ou gain, je n’entends pas dire un accroissement dans la quantité de l’or et de l’argent du pays, mais un accroissement dans la valeur échangeable du pro­duit annuel de ses terres et de son travail, ou bien un accroissement dans le revenu de ses habitants.
Si la balance est égale des deux parts, et si le commerce entre ces deux places ne consiste uniquement que dans l’échange respectif de leurs marchandises nationales, alors, dans la plupart des circonstances, non-seulement elles gagneront l’une et l’autre, mais encore elles gagneront toutes deux autant ou presque autant l’une que l’autre ; chacune fournira un marché à l’excédant de produit de l’autre ; chacune servira à remplacer un capital que l’autre aura employé à faire naître cet excédant de produit, capital qui aura été distribué entre une partie des habitants de cette dernière, et qui leur aura fourni un revenu et un moyen de subsister. Ainsi, chacune d’elles aura une partie de ses habitants qui tireront de l’autre leur subsistance et leur revenu. Comme des marchandises qu’on échange ensemble sont d’ailleurs censées de valeur égale, les capitaux employés dans le commerce seront équivalents ou à peu près équivalents des deux parts ; et l’un et l’autre de ces capitaux se trouvant employés dans chacun des deux pays à y faire naître des marchandises nationales, le revenu et la subsistance que la distribution de ces capitaux fournira aux habitants, seront égaux dans chacun de ces pays. À proportion de l’étendue des affaires qu’ils feront l’un avec l’autre, ce qu’ils se fournissent mutuellement de revenus et de subsistances sera plus ou moins con­sidérable. Si ces affaires, par exemple, montaient annuellement à 100,000 livres ou à un million de chaque côté, chacun de ces pays fournirait à l’autre un revenu annuel de 100,000 livres dans le premier cas, ou d’un million dans le second.
Si la nature du commerce de ces deux pays était telle que l’un d’eux n’exportât chez l’autre que des marchandises nationales, tandis que les retours de l’autre seraient composés uniquement de marchandises étrangères, dans ce cas on pourrait regarder la balance comme au pair, puisque ce seraient des marchandises payées en entier avec des marchandises. Dans cette supposition, pourtant, ils gagneraient bien tous les deux, mais ils ne gagneraient pas autant l’un que l’autre ; et le pays qui n’exporterait que des marchandises produites chez lui serait celui qui tirerait le plus grand revenu de ce commerce.
Si, par exemple, l’Angleterre n’importait de France que des marchandises pro­duites dans ce pays, et que, n’ayant pas de son côté de marchandises nationales qui fussent demandées en France, elle payât ses importations annuelles en y envoyant une grande quantité de marchandises étrangères, comme du tabac ou des marchandises des Indes, un tel commerce aurait bien l’avantage de fournir un revenu à quelques habitants de l’un comme de l’autre pays, mais il en fournirait plus à ceux de la France qu’à ceux de l’Angleterre. La totalité du capital français employé annuellement à ce commerce se distribuerait annuellement entre des Français seulement ; mais il n’y aurait de distribué annuellement entre des Anglais que cette seule partie du capital anglais qui aurait été employée à produire les marchandises anglaises avec lesquelles auraient été achetées les marchandises étrangères. La majeure partie de ce capital irait remplacer les capitaux qui auraient été employés en Virginie, à l’Indostan, dans la Chine, et qui auraient donné des revenus et des subsistances aux habitants de ces pays lointains. Ainsi, les capitaux étaient égaux ou a peu près égaux, cet emploi du capital français augmenterait beaucoup plus la masse des revenus du peuple français, que l’emploi du capital anglais n’augmenterait celle des revenus du peuple anglais. Dans ce cas, la France ferait avec l’Angleterre un commerce étranger de consommation direct, tandis que l’Angleterre ferait avec la France un commerce de même nature, mais par circuit. Or, nous avons déjà expliqué fort au long[17] la différence des effets d’un capital employé au commerce étranger de consommation direct, et d’un capital employé dans celui qui se fait indirectement et par circuit[18]
Vraisemblablement on ne trouverait pas d’exemple d’un commerce entre deux pays, consistant uniquement en échanges de marchandises nationales des deux parts, ou bien d’un commerce consistant uniquement en marchandises nationales d’une part, et en marchandises étrangères de l’autre. Presque tous les pays commercent entre eux, partie en marchandises nationales, partie en marchandises étrangères. Cependant, le pays dans les cargaisons duquel les marchandises nationales seront dans la plus forte proportion, et les marchandises étrangères dans la plus faible, sera toujours celui qui gagnera le plus. Si ce n’était pas avec du tabac ou des marchandises de l’Inde que l’Angleterre payât ses importations annuelles de France, mais que ce fût avec de l’or ou de l’argent, alors, dans une telle supposition, la balance serait censée inégale, les marchandises ne se trouvant plus soldées en marchandises, mais en or ou en argent. Néanmoins, dans ce cas ainsi que dans le précédent, ce commerce aurait l’avantage de fournir un revenu aux habitants des deux pays, quoique plus grand à ceux de France qu’à ceux d’Angleterre[19]. Il rapporterait un revenu à l’Angleterre ; le capital qui aurait été employé à produire les marchandises anglaises avec lesquelles cet or et cet argent auraient été achetés, capital qui serait distribué entre quelques habitants de l’Angle­terre et leur aurait fourni un revenu, se trouverait être par là remplacé et mis à même de continuer la même fonction. La masse totale du capital de l’Angleterre ne serait pas plus diminuée par cette exportation d’or et d’argent, que par l’exportation d’une valeur égale en toute autre marchandise. Au contraire, en plusieurs cas, elle en serait augmentée. On n’envoie hors d’un pays que les marchandises pour lesquelles on présume qu’il y a plus de demande au-dehors qu’au-dedans du pays, et dont on attend, par conséquent, des retours qui, à l’intérieur, auront plus de valeur que les mar­chan­dises exportées. Si une cargaison de tabac, valant en Angleterre seulement 100,000 livres, peut acheter, quand elle sera envoyée en France, une cargaison de vin valant en Angleterre 110,000 livres, un pareil échange augmentera de 10,000 livres la masse du capital de l’Angleterre. De même, si une valeur de 100,000 livres en or anglais achète des vins de France qui vaudront en Angleterre 110,000 livres, cet échange augmentera pareillement la masse du capital anglais d’une valeur de 10,000 livres. Si un marchand qui a pour 110,000 livres de vin dans ses caves est plus riche que celui qui n’a que pour 100,000 livres de tabac dans son magasin, il est également plus riche que celui qui n’a que 100,000 livres en or dans ses coffres. Il peut mettre en activité une plus grande quantité d’industrie, et donner de l’emploi et des moyens de subsister, fournir enfin un revenu à un plus grand nombre de personnes qu’aucun des deux autres ne pourrait faire. Or, le capital d’un pays est égal à la somme des capitaux de tous ses divers habitants, et la quantité d’industrie qu’on peut y entretenir annuelle­ment est égale à ce qu’en peuvent entretenir tous ces différents capitaux ensemble. Ainsi, en général, un échange de ce genre doit augmenter à la fois et le capital du pays, et la somme d’industrie qu’on peut y entretenir annuellement. Il vaudrait mieux, à la vérité, pour le profit de l’Angleterre, qu’elle pût acheter les vins de France avec ses quincailleries ou avec ses draps, que de les acheter avec le tabac de Virginie ou avec l’or et l’argent du Brésil et du Pérou. Un commerce étranger de consommation qui est direct est toujours plus avantageux que celui qui se fait par circuit[20]. Mais un commerce étranger de consommation fait par circuit, par l’intermédiaire de l’or et de l’argent, ne parait pas être moins avantageux que tout autre commerce du même genre qui se fait par un égal circuit. Il n’y a pas plus à craindre qu’un pays qui n’a pas de mines vienne à s’épuiser d’or et d’argent par l’exportation annuelle qu’il fait de ses métaux, qu’il n’est à craindre qu’une pareille exportation annuelle de tabac n’épuise de cette plante un pays qui n’en produit pas. Si un pays qui a de quoi acheter du tabac n’a jamais grande peine à s’en procurer, de même celui qui a de quoi acheter de l’or et de l’argent n’attendra pas longtemps après ces métaux, sitôt qu’il voudra en avoir.
C’est dit-on, un commerce à perte que celui qu’un ouvrier fait avec le cabaret ; et le commerce qu’une nation manufacturière ferait naturellement avec un pays vignoble, peut être regardé comme un commerce du même genre. je réponds à cela que le commerce qu’on fait avec le cabaret n’est pas nécessairement un commerce à perte ; il est, par sa nature, tout aussi avantageux que quelque autre commerce que ce soit, quoique peut-être un peu plus sujet à être porté jusqu’à l’abus. Le métier du brasseur, celui même du détaillant de liqueurs fermentées, sont des divisions de travail aussi nécessaires que toute autre. En général, l’ouvrier trouve plus de profit à acheter du brasseur la provision dont il a besoin, que de la faire par lui-même ; et si c’est un ou­vrier pauvre, il trouvera, en général, plus de profit à l’acheter petit à petit du mar­chand en détail, qu’à acheter une provision chez le brasseur. Sans contredit, il peut acheter beaucoup trop chez l’un et chez l’autre, tout comme il peut trop dépenser chez tout autre marchand de son quartier ; chez le boucher, s’il est glouton, ou chez le marchand de drap, s’il aime à briller parmi ses camarades. Néanmoins, il est avanta­geux pour la masse des ouvriers que tous ces genres de négoce soient libres, quoiqu’il soit possible, dans tous, d’abuser de cette liberté, et dans quelques-uns peut-être avec plus de probabilité que dans d’autres. D’ailleurs, quoique des particuliers puissent quelquefois dissiper toute leur fortune par une consommation excessive de liqueurs fermentées, il n’y a pas de risque, à ce qu’il semble, qu’il en puisse arriver autant à une nation. Si dans tout pays il se trouve beaucoup de gens qui dépensent, en liqueurs de ce genre, plus que leur fortune ne le leur permet, il y en a toujours bien davantage qui font sur cet article moins de dépense qu’ils ne pourraient en faire.
C’est aussi une chose à remarquer, si l’on consulte l’expérience, que le bon marché du vin paraît être une cause de sobriété plutôt que d’ivrognerie. Les peuples des pays vignobles sont, en général, les plus sobres de l’Europe, témoin les Espagnols, les Italiens et les habitants des provinces méridionales de France ; rarement les gens sont sujets à faire excès des choses dont ils font un usage journalier. Personne n’affectera, pour se donner un air de magnificence, ou pour bien traiter ses amis, de faire pro­fusion d’une liqueur à bas prix, comme la petite bière ; au contraire, l’ivrognerie est un vice commun dans les pays qui, à cause de la chaleur ou du froid excessif du climat, ne produisent pas de raisins et où, par conséquent, le vin est cher et passe pour une boisson recherchée, comme chez les peuples du Nord, ou chez ceux qui vivent entre les tropiques, tels que les nègres de la côte de Guinée. On m’a dit avoir souvent observé que lorsqu’un régiment français, au sortir de quelque province du nord de la France, où le vin est un peu cher, vient à être envoyé en garnison dans celles du midi où il est à très-bon marché, les soldats sont d’abord assez portés à la débauche par la nouveauté de trouver le vin bon et à bas prix, mais qu’après quelques mois de séjour ils deviennent pour la plupart aussi sobres que le reste des habitants. Si l’on venait à supprimer tout d’un coup tous les droits sur les vins étrangers, ainsi que l’accise sur la drêche, la bière et l’ale, cet événement pourrait de même occasionner dans la Grande-Bretagne un goût général et passager pour l’ivrognerie dans toutes les classes moyen­nes et inférieures, lequel serait vraisemblablement bientôt suivi d’une dispo­si­tion per­ma­nente et presque universelle à la sobriété. Actuellement, l’ivrognerie n’est nulle­ment le défaut des gens du bon ton ou de ceux qui peuvent aisément faire la dépense des boissons les plus chères ; un gentleman ivre d’ale est une chose qui ne se voit presque jamais. D’ailleurs, les restrictions mises en Angleterre au commerce du vin ont eu bien moins pour objet, selon toute apparence, de détourner les gens d’aller, pour ainsi dire, au cabaret, que de les empêcher d’aller là où ils pourraient se procurer le vin le meilleur et à meilleur compte ; ces règlements favorisent le commerce des vins de Portugal et entravent celui des vins de France. Il est vrai qu’on répond à cela que les Portugais sont de meilleurs chalands que les Français pour nos manufactures, et qu’il faut de préférence encourager leur commerce ; puisqu’ils nous donnent leur pratique, dit-on, il est bien juste de leur donner la nôtre. Ainsi, c’est la routine gros­sière et mesquine de la plus basse classe des artisans qu’on érige en maximes politi­ques pour diriger la conduite d’une grande monarchie ; car il n’y a que les artisans de la dernière classe qui se fassent une règle d’employer de préférence leurs pratiques. Un bon fabricant achète ses marchandises sans avoir égard à de petites vues d’intérêt de cette sorte ; il les prend toujours où il les trouve les meilleures et au meilleur compte.
C’est pourtant avec de pareilles maximes qu’on a accoutumé les peuples à croire que leur intérêt consistait à ruiner tous leurs voisins ; chaque nation en est venue à jeter un oeil d’envie sur la prospérité de toutes les nations avec lesquelles elle com­merce, et à regarder tout ce qu’elles gagnent comme une perte pour elle. Le commerce, qui naturellement devait être, pour les nations comme pour les individus, un lien de concorde et d’amitié, est devenu la source la plus féconde des haines et des querelles. Pendant ce siècle et le précédent, l’ambition capricieuse des rois et des ministres n’a pas été plus fatale au repos de l’Europe, que la sotte jalousie des mar­chands et des manufacturiers. L’humeur injuste et violente de ceux qui gouvernent les hommes est un mal d’ancienne date, pour lequel j’ai bien peur que la nature des choses humaines ne comporte pas de remède ; mais quant à cet esprit de monopole, à cette rapacité basse et envieuse des marchands et des manufacturiers, qui ne sont, ni les uns ni les autres, chargés de gouverner les hommes, et qui ne sont nullement faits pour en être chargés, s’il n’y a peut-être pas moyen de corriger ce vice, au moins est-il bien facile d’empêcher qu’il ne puisse troubler la tranquillité de personne, si ce n’est de ceux qui en sont possédés[21].
Il n’y a pas de doute que c’est l’esprit de monopole qui, dans l’origine, a inventé et propagé cette doctrine, et ceux qui la prêchèrent les premiers ne furent certainement pas aussi sots que ceux qui y crurent. En tout pays, l’intérêt de la masse du peuple est toujours et doit être nécessairement d’acheter tout ce dont elle a besoin, près de ceux qui le vendent à meilleur marché. La proposition est d’une évidence si frappante, qu’il paraîtrait ridicule de prendre la peine de la démontrer, et si les arguties intéressées des marchands et des manufacturiers n’étaient pas venues à bout d’embrouiller les idées les plus simples, elle n’aurait jamais été mise en question ; leur intérêt à cet égard est directement opposé à celui de la masse du peuple. Comme l’intérêt des maîtres qui composent un corps de métier consiste à empêcher le reste des habitants d’employer d’autres ouvriers qu’eux, de même l’intérêt des marchands et des manufacturiers de tout pays consiste à s’assurer le monopole du marché intérieur ; de là ces droits extra­ordinaires établis, dans la Grande-Bretagne et dans la plupart des autres pays de l’Europe, sur presque toutes les marchandises importées par des marchands étrangers ; de là ces droits énormes et ces prohibitions sur tous les ouvrages de fabrique étran­gère qui peuvent faire concurrence à ceux de nos manufactures ; de là aussi ces entra­ves extraordinaires mises à l’importation des marchandises de presque toutes les espèces, quand elles viennent des pays avec lesquels on suppose que la balance du commerce est défavorable, c’est-à-dire, de ceux contre lesquels il se trouve que la haine et la jalousie nationales sont le plus violemment animées.
Cependant, si l’opulence d’une nation voisine est une chose dangereuse sous le rapport de la guerre et de la politique, certainement, sous le rapport du commerce, c’est une chose avantageuse. Dans un temps d’hostilité, elle peut mettre nos ennemis en état d’entretenir des flottes et des armées supérieures aux nôtres ; mais quand fleu­ris­sent la paix et le commerce, cette opulence doit aussi les mettre en état d’échanger avec nous pour une plus grande masse de valeurs, de nous fournir un marché plus étendu, soit pour le produit immédiat de notre propre industrie, soit pour tout ce que nous aurons acheté avec ce produit. Si, pour les gens qui vivent de leur industrie, un voisin riche doit être une meilleure pratique qu’un voisin pauvre, il en est de même d’une nation opulente. À la vérité, un homme riche qui se trouve être aussi lui-même un manufacturier, est un voisin fort dangereux pour les personnes qui exercent la même industrie. Malgré cela, tout le reste du voisinage, le plus grand nombre sans comparaison, trouve son profit dans le bon débit que sa dépense lui fournit. Il trouve même son profit à ce qu’il puisse vendre au-dessous du manufacturier moins riche qui exerce la même industrie. Par la même raison, les manufacturiers d’une nation riche peuvent être, sans contredit, des rivaux très-dangereux pour ceux de la nation voisine. Cependant, cette concurrence même tourne au profit de la masse du peuple, qui trouve encore d’ailleurs beaucoup d’avantage au débit abondant que lui ouvre, dans tous les autres genres de travail, la grande dépense d’une telle nation. Les particuliers qui cherchent à faire leur fortune ne s’avisent jamais d’aller se retirer dans les provinces pauvres et reculées, mais ils vont s’établir dans la capitale ou dans quelque grande ville de commerce. Ils savent très-bien que là où il circule peu de richesses, il y a peu à gagner, mais que dans les endroits où il y a beaucoup d’argent en mou­ve­ment, il y a espoir d’en attirer à soi quelque portion. Cette maxime, qui sert de guide au bon sens d’un, de dix, de vingt individus, devrait aussi diriger le jugement d’un, de dix ou de vingt millions d’hommes ; elle devrait également apprendre à toute une nation à voir dans la richesse de ses voisins une occasion et des moyens probables de s’enrichir elle-même. Une nation qui voudrait acquérir de l’opulence par le commerce étranger, a certainement bien plus beau jeu pour y réussir, si ses voisins sont tous des peuples riches, industrieux et commerçants. Une grande nation, entourée de toutes parts de sauvages vagabonds et de peuples encore dans la barbarie et la pauvreté, pourrait, sans contredit, acquérir de grandes richesses par la culture de ses terres et par son commerce intérieur, mais certainement pas par le commerce étranger. Aussi est-ce, à ce qu’il semble, par la culture et par le commerce intérieur que les anciens Égyptiens et les Chinois ont acquis leurs immenses richesses. On dit que les anciens Égyptiens ne faisaient nul cas du commerce étranger ; et quant aux Chinois, on sait avec quel mépris ils le traitent, et qu’à peine daignent-ils lui accorder cette simple protection que les lois ne peuvent refuser nulle part[22]. Les maximes modernes sur le commerce étranger tendent toutes a l’avilissement et à l’anéantissement même de ce commerce, en tant du moins qu’il leur serait possible d’arriver au but qu’elles se proposent, qui est d’appauvrir tous les peuples voisins.
C’est d’après ces maximes que le commerce entre la France et l’Angleterre a été assujetti, dans l’un et l’autre de ces royaumes, à tant d’entraves et de découragements de toute espèce. Cependant, si les deux nations voulaient ne consulter que leurs véritables intérêts, sans écouter la jalousie mercantile et sans se laisser aveugler par l’animosité nationale, le commerce de France pourrait être plus avantageux pour la Grande-Bretagne que celui de tout autre pays, et par la même raison celui de la Grande-Bretagne pour la France. La France est le pays le plus voisin de la Grande-Bretagne. Le commerce entre les côtes méridionales de l’Angleterre et les côtes du nord et nord-ouest de la France pourrait promettre des retours qui, comme dans le commerce intérieur, seraient répétés quatre, cinq ou six fois dans l’espace d’une année. Ainsi, le capital employé dans ce commerce pourrait, dans chacun de ces deux royaumes, entretenir en activité quatre, cinq ou six fois autant d’industries, et fournir de l’occupation et des moyens de subsistance à quatre, cinq ou six fois autant de personnes que le pourrait faire un pareil capital dans la plupart des autres branches du commerce étranger. Entre les parties de la France et de la Grande-Bretagne qui sont les plus éloignées l’une de l’autre, on pourrait s’attendre à des retours au moins répétés une fois par an, et ce commerce même offrirait déjà par là tout au moins autant d’avan­tage que la plupart des autres branches de notre commerce étranger de l’Europe. Il serait au moins trois fois plus avantageux que notre commerce tant vanté avec nos colonies d’Amérique, dans lequel les retours se font rarement en moins de trois ans, et très-souvent pas en moins de quatre ou cinq. En outre, la France est réputée contenir vingt-quatre millions d’habitants. On n’en a jamais compté dans nos colonies de l’Amérique septentrionale plus de trois millions ; et la France est un pays beaucoup plus riche que l’Amérique septentrionale, quoique, à raison de la plus grande inégalité dans la distribution des richesses, le premier de ces pays présente plus de misère et de pauvreté que l’autre. Ainsi, la France pourrait nous ouvrir un marché au moins huit fois plus étendu, et à cause de la supériorité dans la fréquence des retours, vingt-quatre fois plus avantageux que celui que nous ont jamais fourni nos colonies de l’Amérique septentrionale[23]. Le commerce de la Grande-Bretagne serait tout aussi avantageux pour la France, et, en proportion de la richesse, de la population et de la proximité respective des deux pays, il aurait la même supériorité sur celui que fait la France avec ses colonies. Telle est pourtant l’énorme différence qui se trouve entre le commerce que la sagesse de ces deux nations a jugé à propos de décourager, et celui qu’elle a le plus favorisé. 
Mais ces circonstances mêmes, qui auraient rendu si avantageux un commerce libre et ouvert entre ces deux peuples, sont précisément celles qui ont donné nais­sance aux principales entraves qui l’anéantissent. Parce qu’ils sont voisins, ils sont nécessairement ennemis, et sous ce rapport la richesse et la puissance de l’un est d’autant plus redoutable aux yeux de l’autre ; ce qui devrait servir à multiplier les avantages d’une bonne intelligence entre les deux nations ne sert qu’à enflammer la violence de leur animosité mutuelle. Chacune d’elles est riche et industrieuse : les marchands et les manufacturiers de l’une craignent la concurrence de l’activité et de l’habileté de ceux de l’autre. La jalousie mercantile est excitée par l’animosité natio­nale, et ces deux passions s’enflamment réciproquement l’une par l’autre. Des deux côtés, les marchands de ces deux royaumes, avec cette assurance que des hommes passionnés et mus par l’intérêt mettent à soutenir leurs fausses assertions, ont annoncé la ruine infaillible de leur pays, comme conséquence nécessaire de cette balance défavorable que la liberté des transactions avec le pays voisin ne manquerait pas, sui­vant eux, de leur donner.
Il n’y a pas de pays commerçant en Europe dont la ruine prochaine n’ait été souvent prédite par les prétendus docteurs de ce système, d’après l’état défavorable de la balance du commerce. Cependant, malgré toutes les inquiétudes qu’ils ont inspirées sur ce point, malgré tous les vains efforts de presque toutes les nations commerçantes pour tourner cette balance en leur faveur et contre leurs voisins, il ne paraît pas qu’aucune nation de l’Europe ait été le moins du monde appauvrie par ce moyen. Au contraire, à mesure qu’un pays, qu’une ville a ouvert ses ports aux autres nations, au lieu de trouver sa ruine dans cette liberté de commerce, comme on devait le craindre d’après les principes du système, elle y a trouvé une source de richesses ; quoique pourtant, s’il y a en Europe quelques villes qui, à certains égards, méritent le nom de ports libres, il n’y pas de pays auquel on puisse donner absolument ce nom. La Hollande peut-être est celui qui est le plus près d’en avoir le caractère, quoiqu’elle en soit encore extrêmement loin, et il est reconnu que c’est du commerce étranger que la Hollande tire non-seulement toute sa richesse, mais même une grande partie de ce qui lui est indispensable pour subsister.
À la vérité, il y a une autre balance dont j’ai déjà parlé[24], qui est très-différente de la balance du commerce, et qui occasionne, selon qu’elle se trouve être favorable ou défavorable, la prospérité ou la décadence d’une nation. C’est la balance entre le produit annuel et la consommation. Comme on l’a déjà observé, si la valeur échan­geable du produit annuel excède celle de la consommation annuelle, le capital doit nécessairement grossir annuellement en proportion à cet excédent. Dans ce cas, la société vit sur ses revenus, et ce qu’elle en épargne annuellement s’ajoute naturelle­ment à son capital, et s’emploie de manière à faire naître encore un nouveau surcroît dans le produit annuel. Si, au contraire, la valeur échangeable du produit annuel est au-dessous de la consommation annuelle, le capital de la société doit dépérir annuel­le­ment en proportion de ce déficit. Dans ce cas, la société dépense au-delà de ses revenus, et nécessairement entame son capital. Son capital doit donc nécessairement aller en diminuant, et avec lui en même temps la valeur échangeable du produit annuel de l’industrie nationale.
Cette balance de la production et de la consommation diffère totalement de ce qu’on nomme la balance du commerce. Elle pourrait s’appliquer à une nation qui n’aurait point de commerce étranger, mais qui serait entièrement isolée du reste du monde. Elle peut s’appliquer à la totalité des habitants du globe pris en masse, dont la richesse, la population et les progrès dans les arts et l’industrie peuvent aller en croissant par degrés, ou en déclinant de plus en plus.
La balance entre la production et la consommation peut être constamment en fa­veur d’une nation, quoique ce qu’on appelle la balance du commerce soit, en général, contre elle. Il est possible qu’une nation importe pendant un demi-siècle de suite pour une plus grande valeur que celle qu’elle exporte ; l’or et l’argent qu’on lui apporte pendant tout ce temps peut être en totalité immédiatement envoyé au-dehors ; la quan­tité d’argent en circulation chez elle peut aller toujours en diminuant successi­vement, et céder la place à différentes sortes de papier-monnaie ; les dettes même qu’elle contracte envers les autres nations avec lesquelles elle fait ses principales affaires de commerce peuvent aller toujours en grossissant, et cependant, malgré tout cela, pendant la même période, sa richesse réelle, la valeur échangeable du produit annuel de ses terres et de son travail, aller toujours en augmentant dans une proportion beau­coup plus forte. Pour prouver qu’une telle supposition n’est nullement impossible, il suffit de jeter les yeux sur l’état de nos colonies de l’Amérique septentrionale et de leur commerce avec la Grande-Bretagne avant l’époque des derniers troubles[25].
 
 
 
↑ Les linons et batistes français sont maintenant importés, même pour la consommation intérieure, moyennant un droit de 6 schellings pour une pièce de 8 yards de longueur, et six huitièmes de largeur. Cette proportion est maintenue pour les pièces d’un aunage plus considérable. Mac Culloch.
↑ Tous les impôts, tant directs qu’indirects, se nomment taxes en Angleterre ; cependant on emploie quelquefois le mot impôt pour désigner particulièrement un droit sur l’importation d’une denrée étrangère.
↑ Le tonneau contient quatre muids ou huit barils anglais.
↑ L’auteur a été induit en erreur sur ce fait. (Note du traducteur.)
↑ En Angleterre il n’y a pas de droit de monnayage sur l’or ; mais depuis 1816 un droit de 6 pour 100 a été établi sur le monnayage de l’argent. En France le droit ne dépasse pas un tiers pour 100 sur l’or, et 1 1/2 pour 100 sur l’argent.
Mac Culloch.
↑ Pour avoir une exposition plus complète de la théorie et de la pratique du change, on peut consulter la célèbre brochure de M. Blake intitulée : Observations on exchange.
↑ Ce règlement n’était pas en vigueur. De tous les pays du Nord on a tiré sur Amsterdam argent courant, et les lettres ont été payées sans l’entremise de la banque.
↑ Ce transfert ou transport d’argent de banque sur les livres se nomme aussi assignation. Le créancier de la banque cesse de l’être dès qu’il assigne sa partie à une autre personne, et celle-ci est alors couchée sur les livres, comme créancière.
↑ Une loi expresse interdit tout arrêt juridique, direct ou indirect, sur les sommes ou valeurs en banque appartenant à qui que ce soit.
↑ Depuis 1776, le droit a été d’un quart pour 100 sur dépôt de toutes espèces, or ou argent, sauf les ducatons, pour dépôt desquels on a payé demi pour cent seulement. Les lingots, or ou argent, ont payé demi pour 100.
↑ Le stiver est la vingtième partie du florin. Le florin est à peu près égal à 2 fr. 10 cent.
↑ Le marc de Hollande excède de douze grains notre marc de France : leur rapport est 1 et 1/384 à 1.
↑ Voici les prix auxquels la banque d’Amsterdam reçoit aujourd’hui (septembre 1775) les lingots et les différentes monnaies.
ARGENT.

Dollars du Mexique ou piastres, 22 flor. par marc.
Écus de France, 22 flor. par marc.
Monnaie d’argent d’Angleterre, 22 flot. par marc.
Piastres du Mexique, au nouveau coin, 21 flor. 10 stiv. par marc.
Ducatons, 3 flot. pièce.
Rixdallers, 2 flot. 8 stiv. pièce.
Le lingot d’argent, à 11 den. de fin, 21 flor. par marc, et dans cette proportion, jusqu’à un quart ou 3 den. de fin, dont on donne 5 flot.
Lingots fins, 23 flot. par marc.

OR.
Portugaises, 310 flot. par marc.
Guinées, 310 flot. par marc.
Louis d’or neufs, 310 flot. par marc.
Louis d’or vieux, 300 flot. par marc.
Ducats neufs, 4 flot. 10 stiv. 8 penn. par ducat.
Le lingot d’or est reçu à raison du degré de fin comparé à celui des monnaies ci-dessus.
On donne, sur l’or fin, 340 flot. par marc.
En général cependant on donne un peu plus sur une monnaie dont le titre est connu, que sur des lingots d’or ou d’argent dont on ne peut constater le degré de fin que par la fonte et l’essai. (Note de l’auteur.)
↑ In en coûte aussi 6 stiv. pour chaque partie qu’on veut faire écrire en banque passé onze heures du matin.
↑ Mais si le même jour il entre en banque, à son compte, une somme suffisante, l’amende est réduite à un demi pour 100. Au reste, toutes les amendes pour corrections de compte ou retard d’heure sont au profit des pauvres.
↑ Chapitres v et vii, section iii.
↑ Liv. II, chap. v.
↑ Mais, dans le cas supposé, la France aurait un plus grand capital employé à ce commerce ; car elle serait obligée d’employer un capital à la production des marchandises envoyées en Angleterre, tandis que la dernière n’aurait pas de capital employé à la production de celles qu’elle envoie en France, mais seulement i leur transport ; et sur cette portion de son capital elle ferait les mêmes profits que la France*. Mac Culloch.
↑ Il ne donnerait pas plus de revenu à un pays qu’à l’autre. L’exportation de l’or et de l’argent n’affecte pas plus le capital que l’exportation d’une valeur équivalente de toute autre espèce de marchandise, et ne peut par conséquent apporter de plus grand changement à l’industrie du pays. Mac Culloch.
↑ Il est plus avantageux aux consommateurs, parce que moins la distance d’où les marchandises sont amenées est grande, moins les frais de transport sont élevés, et, par conséquent, plus le prix est réduit ; mais, sous d’autres rapports, il est à peu près indifférent que nous trafiquions avec nos voisins les plus rapprochés, ou avec les peuples situés au bout du monde.
Mac Culloch.
↑ Les marchands ont aujourd’hui des vues plus libérales et plus larges, et en même temps plus exactes de leurs véritables intérèts. Nous en trouvons la preuve dans cette fameuse pétition signée par les premiers marchands de Londres, et soumise à la Chambre des communes en 1820. Elle reconnaît de la manière la plus évidente les avantages supérieure d’une concurrence sans entraves, et elle demande le rappel de toutes les prohibitions et règlements qui ont pour objet de protéger l’industrie nationale, et non de créer un revenu à l’État. La présentation d’une pétition semblable signale une ère nouvelle dans l’histoire du commerce, car elle nous fait voir qu’il n’existe plus désormais de désaccord sur ce sujet entre les idées des théoriciens rationnels, et les hommes pratiques les plus intelligents*.
Mac Culloch.
↑ Mac Culloch prétend, nous ne savons trop sur quels documents, que les anciens Égyptiens n’avaient pas de répugnance pour le commerce étranger et la navigation. Il dit la même chose des Chinois actuels. « Ce sont, dit-il, les compagnies privilégiées qui ont un intérêt à nous représenter le commerce avec la Chine comme très-difficile ; mais depuis que ce commerce est librement ouvert à toutes les nations, l’expérience a prouvé que les Chinois n’ont de répugnance ni pour les étrangers, ni pour le commerce, et que, bien que leur gouvernement soit corrompu et ignorant, leurs habitudes et règlements très-différents des nôtres, il n’en est pas moins possible de traiter les affaires à Canton avec autant de facilité, de sécurité, de promptitude, qu’à Londres ou à New-York. » Il faut avoir un optimisme bien robuste, pour trouver que tout est bien, même en Chine. A. B.
↑ Que dirait Adam Smith aujourd’hui ?
↑ Liv. II, chap. iii.
↑ Ce paragraphe a été écrit en 1775. (Note de l’auteur.)

%%%%%%%%%%%%%%%%%%%%%%%%%%%%%%%%%%%%%%%%%%%%%%%%%%%%%%%%%%%%%%%%%%%%%%%%%%%%%%%%
%                                  Chapitre 4                                  %
%%%%%%%%%%%%%%%%%%%%%%%%%%%%%%%%%%%%%%%%%%%%%%%%%%%%%%%%%%%%%%%%%%%%%%%%%%%%%%%%

\chapter{Des drawbacks ou restitution de droits}
\markboth{Des drawbacks ou restitution de droits}{}

Les marchands et les manufacturiers ne se contentent pas de la vente exclusive sur le marché intérieur, mais ils cherchent aussi à étendre le plus loin possible le débit de leurs marchandises. Leur pays n’a pas de juridiction à exercer chez les nations étrangères et, par conséquent, n’a guère de moyens de leur y procurer un monopole. Ils sont donc ordinairement réduits à se contenter de solliciter divers encouragements pour l’exportation.
Parmi ces encouragements, ceux qu’on nomme Drawbacks, ou restitutions de droits, paraissent être les plus raisonnables. En accordant au marchand l’avantage de retirer, lors de l’exportation, ou le tout, ou partie de ce qui est impose comme accise ou taxe intérieure sur l’industrie nationale, on ne peut pas par là donner lieu à l’expor­tation d’une plus grande quantité de marchandises que ce qui en aurait été exporté si la taxe n’eût pas été imposée. Des encouragements de ce genre ne tendent point à tourner vers un emploi particulier une plus forte portion du capital du pays que celle qui s’y serait portée de son plein gré, mais seulement ils tendent à empêcher que cette portion ne soit détournée forcément vers d’autres emplois pas l’effet de l’impôt. Ils ne tendent pas à détruire cet équilibre qui s’établit naturellement entre tous les divers emplois du travail et des capitaux de la société, mais à empêcher que l’impôt ne le détruise. Ils ne tendent pas à intervenir, mais à maintenir ce qu’il est avantageux de maintenir dans presque tous les temps, l’ordre naturel dans lequel le travail se divise et se distribue dans la société.
On peut dire la même chose des drawbacks accordés à la réexportation des mar­chan­dises importées de l’étranger ; ces restitutions équivalent généralement en Angleterre à la plus grande partie du droit d’importation.
Par le second des règlements annexés à l’acte du parlement qui a établi ce qu’on nomme aujourd’hui l’ancien subside, tout marchand, soit anglais, soit étranger, a été autorisé à retirer moitié de ce droit lors de l’exportation ; le marchand anglais, pourvu que l’exportation eût lieu dans un terme de douze mois ; l’étranger, pourvu qu’elle eût lieu dans un terme de neuf. Les vins, les raisins de Corinthe[1] et les soieries furent les seules marchandises qui ne furent pas comprises dans ce règlement, ces marchandises étant déjà favorisées d’ailleurs et traitées plus avantageusement. Les droits établis par cet acte du parlement étaient, à cette époque, les seuls qui fussent imposés sur l’importation des marchandises étrangères. Dans la suite (par le statut de la septième année de Georges Ier, chap. xxi, sect. 10) on étendit à trois ans le terme dans lequel cette restitution de droits et toutes les autres pourraient être réclamées[2].
Les droits qui ont été imposés depuis l’ancien subside sont pour la plupart restituables en totalité lors de l’exportation. Cependant, cette règle générale est sujette à un grand nombre d’exceptions, et la doctrine des restitutions de droits est devenue beaucoup plus compliquée qu’elle ne l’était à l’époque de leur établissement.
Sur l’exportation de certaines marchandises étrangères dont l’importation était présumée devoir excéder considérablement la quantité nécessaire pour la consomma­tion intérieure, on restitua la totalité des droits, sans retenir même la moitié de l’ancien subside. Avant l’insurrection de nos colonies américaines, nous avions le mo­no­pole du tabac de la Virginie et du Maryland ; nous en importions environ quatre-vingt-seize mille muids, et la consommation intérieure, à ce qu’on croyait, n’en excédait pas quatorze mille ; en vue de faciliter la grande exportation nécessaire pour nous débarrasser de cet excédent, on restitua la totalité des droits, pourvu que l’exportation fût faite dans les trois ans.
Nous avons encore à peu près entier le monopole des sucres de nos îles des Indes occidentales. Aussi, dans le cas où les sucres sont exportés dans l’année, la totalité des droits payés à l’importation est restituée ; et s’ils sont exportés dans les trois ans, on restitue tous les droits, excepté la moitié de l’ancien subside, laquelle continue toujours à être retenue à l’exportation de la plupart des marchandises. Quoique l’im­por­tation du sucre excède de beaucoup ce qui est nécessaire pour la consommation intérieure, néanmoins l’excédent est peu de chose, en comparaison de l’excédent ordinaire du tabac.
Il y a certaines marchandises qui ont excité plus particulièrement la jalousie de nos fabricants, et dont l’importation est prohibée pour la consommation intérieure. On peut cependant, moyennant certains droits, les importer en les emmagasinant pour la réexportation ; mais sur cette exportation, on ne restitue aucune partie des droits. Il paraît que nos manufacturiers ne veulent pas que cette importation, toute gênée qu’elle est, reçoive le moindre encouragement, et qu’ils ont peur qu’on ne puisse sous­traire des magasins quelque partie de ces marchandises, qui ferait alors concurrence aux leurs. C’est sous ces conditions seulement que nous pouvons importer les soieries, les batistes et les linons de France, les toiles de coton peintes, imprimées, mouchetées ou teintes, etc.
Nous évitons même d’être les voituriers des marchandises françaises, et nous aimons mieux perdre nous-mêmes le profit du transport, que de laisser faire quelque profit, par notre entremise, à ceux que nous regardons comme nos ennemis. On retient à l’exportation de toutes les marchandises de France, non-seulement la moitié de l’ancien subside, mais encore les seconds 25 pour 100[3].
Pour le quatrième des règlements annexés à l’ancien subside, les restitutions de droits accordées à l’exportation de tous les vins se trouvèrent monter à beaucoup plus de moitié des droits qui, à cette époque, se payaient sur leur importation, et il paraît qu’alors l’intention de la législature avait été de donner au commerce de transport des vins quelque chose de plus que l’encouragement ordinaire. Plusieurs des autres droits qui furent établis à cette époque ou postérieurement l’ancien subside, ce qu’on appelle le droit additionnel, le nouveau subside, le tiers et les deux tiers de subside, l’impôt de 1692, le monnayage[4] sur le vin, furent tous restituables en totalité lors de l’exportation. Toutefois, tous ces droits, à l’exception du droit additionnel et de l’impôt de 1692, étant avancés en argent comptant à l’importation, l’intérêt d’une som­me aussi forte faisait un objet de dépense qui ne permettait pas de pouvoir s’attendre raisonnablement, sur cet article, à aucun commerce de transport un peu avantageux. Ainsi, il n’y a qu’une partie du droit appelé l’impôt sur le vin, qui soit dans le cas de la restitution lors de l’exportation, et elle n’a été accordée pour aucune partie du droit de 25 livres par tonneau de vin de France, ou des droits imposés en 1745, en 1763 et en 1778. La restitution des deux impôts de 5 p. 100, imposés en 1779 et 1781 sur tous les anciens droits de douanes, ayant été accordée pour la totalité à l’exportation de toutes les autres marchandises, la même restitution fut aussi accordée à l’exportation du vin. On a accordé aussi la restitution en totalité du dernier droit qui a été établi particulièrement sur le vin, celui de 1780 ; mais, quand il y a une si grande quantité de droits énormes qu’on retient, il est plus que probable qu’une pareille indulgence ne fera pas exporter un seul tonneau de vin. Ces règlements étaient applicables à tous les lieux où l’exportation était permise par les lois, à l’exception de nos colonies d’Amérique[5].
Le statut de la quinzième année de Charles II, ch. vit, acte qu’on annonce avoir été porté pour l’encouragement du commerce, a donné à la Grande-Bretagne le monopole d’approvisionner les colonies de toutes les marchandises produites ou fabriquées en Europe et, par conséquent, de vin. Dans un pays qui a une aussi grande étendue de côtes que nos colonies de l’Amérique septentrionale et des Indes occidentales, où notre autorité a toujours été si faible et où on a donné aux habitants la faculté de transporter, sur leurs propres vaisseaux, leurs marchandises non énumérées, d’abord à toutes les parties de l’Europe, et ensuite à toutes les parties de l’Europe situées au sud du cap Finistère, il n’est pas vraisemblable que ce monopole puisse jamais être très-respecté ; et probablement en tout temps ils ont bien su trouver le moyen de remporter quelque cargaison des pays où il leur était permis d’en porter une. Cependant, il paraît qu’ils ont trouvé quelque difficulté à importer les vins d’Europe des pays où ils sont produits, et ils ne pouvaient guère les importer de la Grande-Bretagne, où cette denrée était chargée de tant de droits énormes, dont une très-forte partie n’était pas restituée à l’exportation. Le vin de Madère, n’étant pas une marchandise européenne, pouvait être importé directement en Amérique et dans les Indes occidentales, qui les unes et les autres jouirent d’un commerce libre avec l’île de Madère pour toutes leurs marchandises non énumérées. C’est vraisemblablement cette circonstance qui a introduit ce goût général pour les vins de Madère, qui dominait dans toutes nos colonies au commencement de la guerre de 1755, et que nos officiers rapportèrent avec eux dans la mère patrie, où ces vins n’avaient pas été jusque-là fort en vogue. À la conclusion de cette guerre, en 1763 (par le statut de la quatrième année de Georges III, chap. XV, sect. 12), on accorda le drawback de tous les droits, sauf une retenue de 3 livres 10 schellings, en cas d’exportation aux colonies de toute espèce de vins ; les vins de France, au commerce et à la consommation desquels le préjugé national ne voulait accorder aucune sorte d’encouragement, furent exceptés de cette faveur. L’espace de temps qui s’est écoulé entre la concession de cette facilité et l’insurrection de nos colonies d’Amérique, a sans toute été trop court pour qu’il ait pu se faire dans les habitudes de ce pays quelque changement un peu sensible.
Le même acte qui favorisait ainsi les colonies de préférence aux autres pays, en leur accordant ces restitutions sur l’exportation de tous les vins, excepté ceux de France, les favorisait beaucoup moins que les autres pays quant aux restitutions sur l’exportation de toutes les autres marchandises. On restituait la moitié de l’ancien subside à l’exportation de la plupart des marchandises aux autres pays. Mais cet acte portait qu’on ne restituerait aucune partie de ce droit à l’exportation aux colonies de toute marchandise produite ou fabriquée en Europe ou aux Indes orientales, à l’excep­tion des vins, des toiles de coton blanches et des mousselines.
Les drawbacks ont peut-être été accordés, dans le principe, pour encourager le commerce de transport, que l’on supposait plus particulièrement propre à faire entrer de l’or et de l’argent dans le pays, parce que les étrangers payent souvent en argent le fret des vaisseaux. Mais, quoique certainement le commerce de transport ne mérite pas plus d’encouragement qu’un autre, quoique peut-être le motif de l’institution fût extrêmement absurde, toutefois l’institution en elle-même paraît assez raisonnable. Ces restitutions ne peuvent avoir l’effet de jeter forcément dans ce genre de com­merce une plus forte portion du capital de la société que celle qui s’y serait portée d’elle-même, s’il n’y eût pas eu de droits sur l’importation. Elles empêchent seulement que les droits n’en excluent totalement cette portion du capital. Si le commerce de transport ne mérite pas qu’on l’encourage par préférence, il ne doit pas non plus être découragé ; il faut le laisser libre comme tous les autres. Il offre une ressource nécessaire à ces capitaux qui ne peuvent plus trouver d’emploi soit dans l’agriculture ou les manufactures du pays, soit dans le commerce intérieur, soit enfin dans le commerce étranger de consommation.
Le revenu des douanes, au lieu d’en souffrir, trouve son avantage dans ces resti­tutions, au moyen de la retenue faite sur une partie du droit. Si l’on avait voulu retenir la totalité du droit, les marchandises étrangères sur lesquelles on le paye n’auraient guère été exportées, ni conséquemment importées faute de marché ; par conséquent, les droits dont on retient une partie n’auraient jamais été perçus.
Ces raisons paraissent suffisantes pour justifier les drawbacks, et elles les justifie­raient encore quand même on restituerait toujours, lors de l’exportation, la totalité des droits, soit sur les produits d’industrie nationale, soit sur les marchandises étrangères. À la vérité, dans ce cas, le revenu de l’accise en souffrirait un peu, et celui des doua­nes bien davantage ; mais aussi un pareil règlement replacerait plus près de son juste niveau la balance naturelle entre les diverses branches d’industrie et la division et la distribution naturelles du travail, que de pareils droits troublent toujours plus ou moins.
Cependant, ces raisons ne justifient les drawbacks qu’autant qu’ils sont accordés sur les marchandises exportées à des pays tout à fait étrangers et indépendants, et non pas à ceux où nos marchands et manufacturiers jouissent du monopole. Par exemple, une restitution accordée sur l’exportation des marchandises européennes à nos colo­nies d’Amérique n’occasionnera pas toujours une plus forte exportation que celle qui aurait eu lieu sans cette restitution. Au moyen du monopole qu’y exercent nos mar­chands et nos manufacturiers, ils y renverraient souvent peut-être la même quantité de marchandises, quand même on retiendrait la totalité des droits. La restitution est, par conséquent, souvent en pure perte pour le revenu de l’accise et des douanes, sans qu’elle change rien à l’état du commerce, ni qu’elle contribue le moins du monde à lui donner de l’extension.
Mais jusqu’à quel point peut-on justifier ces restitutions sous le rapport d’encoura­gements donnés à l’industrie de nos colonies, ou jusqu’à quel point peut-il être avan­tageux à la mère patrie que nos colonies soient exemptes des impôts que payent tous les autres sujets de l’empire ? C’est ce que j’examinerai par la suite, quand je traiterai des colonies.
Toutefois, on doit toujours entendre que les restitutions ne sont utiles que dans les cas seulement où la marchandise pour l’exportation de laquelle on les accorde est réellement exportée à quelque pays étranger, et qu’elle n’est pas clandestinement réim­portée dans le nôtre. On sait assez que certaines restitutions, et en particulier cel­les sur le tabac, ont été souvent suivies d’abus de ce genre, et qu’elles ont donné naissance à plusieurs fraudes qui font également tort et au revenu public et au commerçant qui travaille loyalement.
 
 
 
↑ Currants. Les Anglais font une grande consommation de ces sortes de raisins qui se tirent des îles loniennes, principalement de celles de Zante et de Céphalonie.
↑ En 1787, pour remédier aux embarras et inconvénients résultant de la multiplicité d’actes séparés relativement aux douanes, M. Pitt introduisit un bill qui avait pour but de les fixer. Diverses consolidations pareilles ont été effectuées à différentes époques. La dernière eut lieu en 1854, époque où les différents droits de douane furent établis, à peu près dans leur état actuel, par l’acte 3 et 4, Guillaume IV, chap. lvi. En conséquence de ces consolidations, les différences auxquelles Smith fait allusion entre les anciens et les nouveaux droits ont totalement disparu. Les drawbacks accordés actuellement, et ils sont en petit nombre, représentent toujours la totalité du droit payé à l’importation. Mac Culloch.
↑ Ces restrictions sont aujourd’hui abolies. Mac Culloch.
↑ On appelle de ce nom certains droits établis pour défrayer les dépenses de monnayage.
↑ Nous observons avec plaisir que les droits différentiels sur les vins de France, peut-être les plus contestables de tous les anciens droits, ont été abolis, et que le même droit de 5 schellings 6 deniers par gallon impérial est aujourd’hui imposé sur tous les vins étrangers importés en Angleterre, quel que soit le lieu de leur origine. Le droit sur le vin du cap de Bonne-Espérance n’est que de 2 schillings 9 deniers par gallon. Cette réduction privilégiée ne s’explique par aucune bonne raison. Mac Culloch.

%%%%%%%%%%%%%%%%%%%%%%%%%%%%%%%%%%%%%%%%%%%%%%%%%%%%%%%%%%%%%%%%%%%%%%%%%%%%%%%%
%                                  Chapitre 5                                  %
%%%%%%%%%%%%%%%%%%%%%%%%%%%%%%%%%%%%%%%%%%%%%%%%%%%%%%%%%%%%%%%%%%%%%%%%%%%%%%%%

\chapter{Des primes}
\markboth{Des primes}{}

Le parlement de la Grande-Bretagne reçoit de fréquentes pétitions tendant à obte­nir des Primes à l’exportation, et ces primes s’accordent quelquefois au produit de certaines branches de l’industrie nationale.
Par ce moyen, dit-on, nos marchands et nos manufacturiers seront en état de ven­dre leurs marchandises, sur les marchés étrangers, à aussi bon ou à meilleur marché que leurs rivaux. Dès lors, il y en aura une plus grande quantité d’exportée et, par conséquent, la balance du commerce en sera d’autant plus en faveur de notre pays. Nous ne pouvons pas accorder à nos ouvriers un monopole sur le marché étranger, comme nous l’avons fait pour le nôtre. Nous ne pouvons pas forcer les étrangers à leur acheter leurs marchandises, comme nous y avons forcé nos concitoyens. Par conséquent, a-t-on dit, le meilleur expédient qui nous reste à employer, c’est de payer les étrangers pour les décider à acheter de nous. Telle est la manière dont le système mercantile se propose d’enrichir tout le pays et de nous remplir à tous les poches d’ar­gent par le moyen de sa merveilleuse balance.
On convient, à la vérité, que les primes ne doivent s’accorder qu’à ces branches d’industrie qui ne sauraient se soutenir sans elles. Mais toute branche de commerce dans laquelle le marchand peut vendre ses marchandises à un prix qui lui remplace, avec le profit ordinaire, tout le capital employé à les préparer et à les mettre au marché, sera en état de se soutenir sans le secours d’une prime. Une telle branche de commerce se trouve évidemment au niveau de toutes les autres qui se soutiennent sans prime et, par conséquent, elle n’en a pas plus besoin qu’elles. Les seules branches de commerce qui aient besoin de gratification, ce sont celles où le marchand est obligé de vendre ses marchandises à un prix qui ne lui remplace pas son capital avec le profit ordinaire, ou bien de les vendre pour moins qu’il ne lui en coûte réellement pour les mettre au marché. La prime se donne en vue de compenser ce déficit, en vue d’encourager le marchand à continuer ou peut-être même à entreprendre un commerce dans lequel la dépense est censée plus forte que les retours, dont chaque opération absorbe une partie du capital qu’on y emploie ; un commerce, enfin de telle nature que, si tous les autres lui ressemblaient, il ne resterait bientôt plus de capital dans le pays.
Il est à observer que les industries qui se soutiennent à l’aide de primes sont les seules qui puissent se maintenir pendant un certain temps entre deux nations, avec cette circonstance que l’une d’elles soit constamment et régulièrement en perte, ou bien vende constamment ses marchandises pour moins qu’il ne lui en coûte réelle­ment à les envoyer à ce marché ; car si la prime ne remboursait pas au marchand ce qu’il perdait sans cela sur le prix de ses marchandises, son intérêt l’obligerait bientôt à employer son capital d’une autre manière, et à chercher quelque autre industrie dans laquelle le prix de ses marchandises pût lui remplacer, avec le profit ordinaire, le capital employé à les mettre au marché. L’effet des primes, comme celui de tous les autres expédients imaginés par le système mercantile, ne peut donc être que de pous­ser par force l’industrie du pays dans un canal beaucoup moins avantageux que celui dans lequel elle serait entrée naturellement de son plein gré.
Un auteur habile et bien instruit, celui des Traités sur le commerce des blés, a fait voir clairement que, depuis le premier établissement de la prime sur l’exportation des blés, le prix du blé exporté, évalué à un prix assez modéré, a excédé celui du blé im­porté, évalué au plus haut, d’une somme beaucoup plus forte que le montant total des primes qui ont été payées pendant la même période de temps[1]. Il trouve, en raisonnant d’après les propres principes du système mercantile, que c’est une preuve évidente que ce commerce forcé est avantageux à la nation, la valeur de l’exportation excédant celle de l’importation d’une somme beaucoup plus forte que toute la dépense extraordinaire faite par l’État pour occasionner cette exportation. Il ne fait pas atten­tion que cette dépense extraordinaire, c’est-à-dire la prime, est la moindre partie de la dépense que l’exportation du blé coûte réellement à la société. Il faut bien mettre aussi en ligne de compte le capital employé par le fermier pour faire croître ce blé. À moins que le prix du blé, quand il est vendu sur les marchés étrangers, en remplace non-seu­le­ment la prime, mais encore ce capital, en y joignant le profit ordinaire des capitaux, la société se trouvera en perte de toute la différence, ou bien la masse du capital national en sera d’autant diminuée. Mais c’est précisément parce qu’on suppose que le prix est insuffisant pour remplir cet objet, qu’on a jugé nécessaire d’accorder une prime.
Le prix moyen du blé, a-t-on dit, a baissé considérablement depuis l’établissement de la prime. Que le prix moyen du blé ait commencé à baisser quelque peu vers la fin du dernier siècle et ait toujours été en baissant pendant le cours des soixante-quatre premières années de celui-ci, c’est un fait que j’ai tâché d’établir[2]. Mais cet événement, en le supposant aussi vrai que je crois qu’il l’est, aura eu lieu malgré la prime, et il n’est pas possible qu’il en soit une conséquence.
Cet événement a eu lieu en France aussi bien qu’en Angleterre, quoique en France non-seulement il n’y ait pas eu de prime, mais que, même jusqu’en 1764, l’exportation eût été absolument prohibée. Cette baisse successive dans le prix moyen du blé ne doit donc vraisemblablement être attribuée, en dernier résultat, ni à l’un ni à l’autre de ces deux règlements opposés, mais à cette hausse graduelle et insensible de la valeur réelle de l’argent, qui s’est manifestée, pendant le cours de ce siècle, sur le marché général de l’Europe, ainsi que j’ai tâché de le démontrer dans le premier livre de cet ouvrage. Il paraît absolument impossible que la prime puisse jamais contribuer à faire baisser le prix des grains[3].
On a déjà observé que, dans les années d’abondance, la prime, en occasionnant une exportation extraordinaire, tient nécessairement le prix du blé, sur le marché intérieur, au-dessus du taux auquel il descendrait naturellement. C’était même là l’objet qu’on se proposait ouvertement par cette institution. Quoique la prime soit sou­vent suspendue pendant les années de cherté, cependant la grande exportation qu’elle occasionne dans les années d’abondance doit avoir souvent pour effet d’em­pêcher plus ou moins que l’abondance d’une année ne soulage la disette d’une autre.
Ainsi, dans les années de cherté, tout aussi bien que dans celles d’abondance, la prime tend de même nécessairement à faire monter le prix vénal du blé plus haut qu’il n’aurait été sans cela sur le marché intérieur.
Je pense bien qu’aucune personne raisonnable ne voudra contester que la prime doit nécessairement avoir cette tendance, l’état de la culture restant le même. Mais il y a beaucoup de gens qui pensent qu’elle tend à encourager la culture des grains, et cela de deux manières différentes : la première, en ouvrant au blé du fermier un marché plus étendu à l’étranger ; ce qui tend, selon eux, à augmenter la demande de blé et, par conséquent, la production de cette denrée ; et la seconde, en assurant au fermier un meil­leur prix que celui qu’il pourrait espérer sans cela, dans l’état actuel de la culture ; ce qui tend, à ce qu’ils supposent, à encourager cette culture. Suivant eux, ce double encou­ragement doit occasionner, dans une longue période d’années, un tel accrois­sement dans la production du blé, que son prix sur le marché intérieur en doit baisser plus que la prime ne pourra le hausser, dans l’état où se trouvera être parvenue la culture à la fin de cette période.
Je réponds à cette objection que, quelque extension que la prime puisse occasion­ner sur le marché étranger, dans une année quelconque, cette extension se fait tou­jours entièrement aux dépens du marché intérieur, attendu que chaque boisseau de blé que la prime fait exporter et qui ne l’aurait pas été sans elle, serait resté sur le marché intérieur, où il aurait augmenté d’autant la consommation et fait baisser le prix de la denrée. Il faut observer que la prime sur le blé, comme toute autre prime pour l’expor­tation, établit sur le peuple deux impôts différents : le premier est l’impôt auquel il faut qu’il contribue pour payer la prime, et le second est l’impôt qui résulte du renché­rissement de prix sur le marché intérieur, impôt qui, pour cette espèce parti­culière de marchandise, se paye par toute la masse du peuple, toute la masse étant nécessaire­ment acheteur de blé. Par conséquent, à l’égard de cette marchandise en particulier, le second impôt est de beaucoup le plus lourd des deux. Supposons en effet que, une année dans l’autre, la prime de 5 schellings à l’exportation du quarter de blé froment élève le prix de cette denrée, sur le marché intérieur, de 6 deniers seulement par bois­seau, ou de 4 schellings par quarter plus haut qu’il n’aurait été sans cela, vu l’état actuel de la récolte ; même dans cette supposition très-modérée, le corps entier du peuple, en outre de sa contribution à la dépense publique qu’entraîne le payement de 5 schellings de prime sur chaque quarter de froment exporté, doit encore payer un autre impôt de 4 schellings sur chaque quarter qu’il consomme lui-même. Or, selon l’auteur des Traités sur le commerce des blés, qui avait de bons renseignements, la proportion moyenne entre la quantité du blé exporté et celle du blé consommé au-dedans est seulement comme 1 est à 31. Par conséquent, par chaque 5 schellings que le peuple paye pour le premier de ces deux impôts, il faut qu’il contribue pour 6 livres 4 schellings au payement du second. Un impôt aussi lourd sur le premier besoin de la vie doit nécessairement, ou retrancher la subsistance même de l’ouvrier pauvre, ou occasionner quelque augmentation dans son salaire en argent, proportionnée à celle du prix en argent de sa subsistance[4]. En tant qu’il agit de la première manière, l’impôt doit diminuer, dans la classe des ouvriers pauvres, les moyens d’élever et de soigner leurs enfants, et il tend d’autant à réduire la population du pays. En tant qu’il agit de l’autre manière, il doit diminuer, dans la classe des maîtres qui font travailler les ouvriers pauvres, les moyens d’en employer un aussi grand nombre qu’ils l’auraient pu faire sans cela, et il tend à réduire d’autant l’industrie du pays. Par conséquent, l’ex­por­tation extraordinaire de blé occasionnée par la prime, non-seulement, dans chaque année en particulier, resserre le marché et la consommation intérieure de tout ce dont elle étend le marché et la consommation chez l’étranger, mais encore, par les entraves qu’elle oppose à la population et à l’industrie du pays, sa tendance, en dernier résultat, est de gêner et de comprimer l’extension graduelle du marché intérieur et, par ce moyen, de diminuer à la longue, bien loin de l’augmenter, la consommation totale et le débit de la denrée.
Cependant, on a encore imaginé que le renchérissement du prix du blé en argent, en rendant cette denrée d’un meilleur rapport pour le fermier, devait nécessairement en encourager la production.
Je réponds que cela pourrait arriver si l’effet de la gratification était de faire monter le prix réel du blé, ou de mettre le fermier en état d’entretenir, avec la même quantité de blé, un plus grand nombre d’ouvriers de la même manière que sont com­mu­nément entretenus les autres ouvriers du voisinage, largement, médiocrement ou petitement. Mais il est évident que ni la prime, ni aucune autre institution humaine ne peut produire un pareil effet. Ce n’est pas sur le prix réel du blé, c’est seulement sur son prix nominal que porte tout l’effet de la prime ; et quoique l’impôt dont cette institution grève toute la masse du peuple soit très-onéreux pour ceux qui le payent, il n’est que d’un très-petit avantage pour ceux qui le reçoivent.
Le véritable effet de la prime est bien moins d’élever la valeur réelle du blé que de dégrader la valeur réelle de l’argent, et de faire en sorte qu’une même somme d’argent s’échange contre de moindres quantités, non-seulement de blé, mais encore de toute autre marchandise que le pays produit ; car le prix pécuniaire du blé règle celui de toutes les autres marchandises produites dans le pays[5].
Il détermine le prix en argent du travail, qui doit toujours nécessairement être tel qu’il mette l’ouvrier en état d’acheter une quantité de blé suffisante pour l’entretient de sa personne et de sa famille, selon que le maître qui le met en œuvre se trouve obligé, par l’état progressif, stationnaire ou décroissant de la société, de lui fournir cet entre­tien abondant, médiocre ou chétif[6].
Il détermine le prix en argent de toutes les autres parties du produit brut de la terre, lequel doit nécessairement, dans toutes les périodes d’avancement de la société, se proportionner avec le prix en argent du blé, quoique la proportion soit différente dans des périodes différentes. Il détermine, par exemple, le prix en argent du foin et du fourrage, de la viande de boucherie, des chevaux et de leur entretien, par consé­quent, des charrois ou de la majeure partie des frais du commerce intérieur par terre[7].
En déterminant le prix en argent de toutes les autres parties du produit brut de la terre, il détermine celui des matières de toutes les manufactures. En déterminant le prix en argent du travail, il détermine celui de la main-d’œuvre et de toutes les applications de l’industrie ; et en déterminant l’un et l’autre de ces prix, il détermine le prix total de l’ouvrage manufacturé[8]. Il faut donc nécessairement que le prix en argent du travail et de toute autre chose qui est le produit de la terre ou du travail, monte ou baisse en proportion du prix en argent du blé.
Par conséquent, encore que la prime puisse avoir l’effet de mettre le fermier à même de vendre son blé 4 sch. le boisseau au lieu de 3 sch. 6 d., et de payer à son propriétaire une rente en argent proportionnée à cette hausse du prix en argent de sa récolte, néanmoins, si, par une suite de cette hausse du prix du blé, 4 sch. ne peuvent acheter plus de marchandises de toute autre espèce du produit du pays que n’en auraient acheté auparavant 3 sch. 6 d., un pareil changement n’aura pas le moins du monde amélioré le sort du fermier ni celui du propriétaire. Le fermier n’en sera pas pour cela en état de cultiver mieux, ni le propriétaire de vivre plus honorablement. Sur les marchandises qu’ils achèteront de l’étranger, ce renchérissement du prix du blé pourra leur donner quelque petit avantage ; sur celles achetées dans le pays, il ne leur en donnera absolument aucun. Or, c’est en marchandises du pays que se fait presque toute la dépense du fermier, et la très-majeure partie même de celle du propriétaire.
Une dégradation dans la valeur de l’argent, qui est l’effet de la fécondité des mines et qui se fait sentir également ou presque également dans la totalité, ou peu s’en faut, du monde commerçant, est de très-peu d’importance pour un pays en particulier. La hausse qui en résulte dans tous les prix en argent, ne rend pas plus riches ceux qui les reçoivent, mais du moins elle ne les rend pas plus pauvres. Un service en argenterie devient réellement à meilleur marché, mais toutes les autres choses restent exacte­ment comme elles étaient auparavant, quant à leur valeur réelle.
Mais cette dégradation dans la valeur de l’argent, qui, étant le résultat ou de la situation particulière d’un pays, ou de ses institutions politiques, n’a lieu que pour ce pays seulement, entraîne des conséquences tout autres, et, bien loin qu’elle tende à rendre personne réellement plus riche, elle tend à rendre chacun réellement plus pauvre. La hausse du prix en argent de toutes les denrées et marchandises, qui, dans ce cas, est une circonstance particulière à ce pays, tend à y décourager plus ou moins toute espèce d’industrie au-dedans, et à mettre les nations étrangères à portée de four­nir presque toutes les diverses sortes de marchandises pour moins d’argent que ne le pourraient faire les ouvriers du pays, et par là de les supplanter non-seulement sur les marchés étrangers, mais même sur leur propre marché intérieur.
Une circonstance qui est particulière à l’Espagne ou au Portugal, c’est d’être, comme propriétaires des mines, les distributeurs de l’or et de l’argent à toute l’Europe et, par conséquent, d’avoir ces métaux chez eux à un peu meilleur marché qu’en tout autre pays d’Europe. La différence cependant ne devrait être que du prix du fret et de l’assurance ; et, vu la haute valeur de ces métaux sous un petit volume, le fret n’est presque rien, et l’assurance pas plus chère que pour toute autre valeur égale. Ainsi, l’Espagne et le Portugal n’auraient que très-peu à souffrir de cette circonstance, si leurs institutions n’en aggravaient encore le désavantage.
L’Espagne, par ses taxes sur ces métaux, et le Portugal, par ses prohibitions sur leur exportation, ont surchargé cette exportation de tous les frais de la contrebande, et ont fait monter la valeur de l’or et de l’argent, dans les autres pays, au-dessus de ce qu’elle est chez eux, de toute la valeur de ces frais. Fermez un courant d’eau par une écluse ; celle-ci une fois remplie, il s’écoulera tout autant d’eau par-dessus les portes que s’il n’y avait point d’écluse. La prohibition d’exporter ne peut pas retenir en Espa­gne et en Portugal plus d’or et d’argent que ces deux pays ne sont en état d’en absorber, plus que ce que le produit de leurs terres et de leur travail leur permet d’en tenir employé en monnaie, en vaisselle, en dorures et en autres ornements d’or et d’argent. Quand ils ont atteint cette quantité, l’écluse est remplie, et tout ce que le courant apporte de plus par la suite doit s’écouler. Aussi, en dépit de toutes les entra­ves, l’exportation annuelle d’or et d’argent de l’Espagne et du Portugal est, d’après tous les rapports, à peu près équivalente à la totalité de ce qui s’y importe annuellement. Cependant, comme l’eau doit nécessairement avoir plus d’élévation en deçà qu’au-delà de l’écluse, de même la quantité d’or et d’argent que ces entraves retiennent en Espagne et en Portugal doit être plus grande, en proportion du produit annuel de leurs terres et de leur travail, qu’elle ne l’est dans les autres pays. Plus la digue sera forte et élevée, plus aussi il y aura de différence dans la hauteur de l’eau en deçà de la digue et au-delà. Plus les taxes seront fortes, plus les peines portées pour assurer la prohibition seront graves, plus la police qui veille à l’exécution de cette loi sera vigilante et rigoureuse, et plus aussi sera grande la différence entre la quantité relative d’or et d’argent par rapport au produit des terres et du travail en Espagne et en Portu­gal, et la quantité relative qu’en ont les autres pays. Aussi, dit-on que cette quantité relative y est extrêmement considérable, et qu’on y voit fréquemment de la vaisselle d’argent en profusion dans des maisons qui n’offrent d’ailleurs rien qui réponde ou qui soit assorti, suivant les usages de tous les autres pays, à ce genre de magnificence. Le bon marché de l’or et de l’argent, ou, ce qui est la même chose, la cherté de toutes les marchandises, qui est une suite nécessaire de cette surabondance des métaux pré­cieux, décourage à la fois l’agriculture et les manufactures en Espagne et en Portugal, et met les nations étrangères à portée de fournir à ces pays beaucoup d’espèces de produits bruts et presque toutes les espèces de produits manufacturés, pour une quantité d’or et d’argent moindre que celle qu’ils dépenseraient pour les faire croître ou les fabriquer chez eux[9]. La taxe et la prohibition opèrent cet effet de deux manièr­es : non-seulement elles abaissent extrêmement la valeur des métaux précieux en Espagne et en Portugal ; mais encore, en y retenant de force une certaine quantité de ces métaux, qui refluerait sans cela dans les autres pays, elles tiennent leur valeur, dans ces autres pays, à un taux un peu plus élevé qu’elle n’y serait sans cela, et leur donnent par là un double avantage dans leur commerce avec l’Espagne et le Portugal. Ouvrez les portes de l’écluse, et tout aussitôt il y aura moins d’eau au-dessus de ces portes ; il y en aura plus au-dessous, et le niveau s’établira bien vite entre ces deux parties du courant. Supprimez la taxe et la prohibition, alors la quantité d’or et d’argent diminuera considérablement en Espagne et en Portugal ; elle augmentera en même temps dans les autres pays, et alors la valeur de ces métaux, leur proportion avec le produit annuel des terres et du travail, prendront partout l’équilibre, ou à peu près. La perte que l’Espagne et le Portugal auraient à essuyer de cette exportation de leur or et de leur argent serait totalement nominale et purement imaginaire. La valeur nominale de leurs marchandises et du produit annuel de leurs terres et de leur travail viendrait à baisser ; elle serait exprimée et représentée par une moindre quantité d’argent qu’auparavant, mais leur valeur réelle serait toujours la même qu’auparavant ; elle suffirait pour entretenir, commander ou employer tout autant de travail qu’elle en employait. La valeur nominale de leurs marchandises venant à tomber, la valeur réelle de ce qui leur resterait de leur or et de leur argent s’en élèverait d’autant, et une moindre quantité de ces métaux remplirait, à l’égard du commerce et de la circulation, tous les services qui en exigeaient auparavant une plus grande quantité. L’or et l’argent qui iraient au-dehors n’iraient pas pour rien, mais rapporteraient, en retour, une valeur égale de marchandises d’une espèce ou d’une autre. Ces marchandises ne seraient pas non plus toutes en objets de luxe ou de pure dépense, destinés à être consommés par ces gens oisifs qui ne produisent rien en retour de leur consomma­tion. Comme cette exportation extraordinaire d’or et d’argent ne saurait augmenter la richesse réelle ni le revenu réel de ces gens oisifs, elle ne saurait non plus apporter une grande augmentation dans leur consommation. Vraisemblablement la plus grande partie de ces marchandises, et pour sûr, au moins une partie, consisterait en matières, outils et vivres pour employer et faire subsister des gens laborieux qui reproduiraient avec profit la valeur entière de leur consommation. Une partie du capital improductif de la société se trouverait ainsi convertie en un capital actif, et on mettrait en activité une plus grande quantité d’industries qu’on n’en entretenait auparavant. Le produit annuel des terres et du travail de ces pays augmenterait sur-le-champ de quelque chose et, au bout de peu d’années, éprouverait vraisemblablement une grande aug­men­tation, leur industrie se trouvant ainsi soulagée d’un des fardeaux les plus accablants sous lesquels elle ait à gémir actuellement[10]. 
La prime accordée à l’exportation du blé produit nécessairement un effet sem­blable à celui de cette politique absurde de l’Espagne et du Portugal. Quel que soit l’état actuel de la culture, la prime rend notre blé un peu plus cher sur le marché intérieur qu’il ne devrait l’être eu égard à cet état de culture, et elle le rend un peu meilleur marché sur les marchés étrangers ; et comme le prix moyen du blé en argent règle plus ou moins celui de toutes les autres marchandises, elle rabaisse considéra­blement la valeur de l’argent dans le premier de ces marchés, et tend à la faire monter un peu dans les autres. Elle met les étrangers, en particulier les Hollandais, à même non-seulement de consommer notre blé à meilleur marché qu’ils ne pourraient le faire sans elle, mais encore de le consommer quelquefois à meilleur marché que nous ne le consommons nous-mêmes dans les mêmes circonstances, comme nous en avons pour garant une excellente autorité, celle de sir Matthieu Decker. Elle empêche nos ouvriers de pouvoir livrer leurs produits pour une aussi petite quantité d’argent qu’ils eussent pu le faire sans cela, et elle met les Hollandais à même de livrer les leurs pour moins d’argent qu’ils n’eussent été en état de le faire. Elle tend à rendre les ouvrages de nos manufactures un peu plus chers sur l’un et l’autre marché, et à rendre les leurs moins chers qu’ils ne l’eussent été sans elle, et par conséquent elle tend doublement à donner à leur industrie de l’avantage sur la nôtre[11].
Comme la prime fait monter sur le marché intérieur, non pas le prix réel, mais simplement le prix nominal de notre blé ; comme elle augmente, non pas la quantité de travail qu’une certaine quantité de blé peut entretenir et mettre en activité, mais simplement la quantité d’argent que cette quantité de blé pourra obtenir en échange, elle décourage nos manufactures, sans rendre le moindre service réel à nos fermiers ni à nos propriétaires ruraux. Elle met bien, à la vérité, un peu plus d’argent dans la poche des uns et des autres, et ce ne serait peut-être pas chose facile à faire entendre à la majeure partie d’entre eux, que ce n’est pas là leur rendre un service très-réel. Mais cependant, si cet argent baisse dans sa valeur, s’il perd du côté de la quantité de travail, de vivres et d’autres marchandises nationales de toute espèce qu’il a la faculté d’acheter, autant qu’il augmente lui-même en quantité, alors le service ne sera guère que nominal et imaginaire.
Il n’y a peut-être dans l’État qu’une seule classe de gens pour qui la prime est ou serait réellement profitable. C’est celle des marchands de blé, de ceux qui exportent et importent les blés. Dans les années d’abondance, la prime a nécessairement occasion­né une plus forte exportation que celle qui aurait eu lieu sans cela ; et en empêchant que l’abondance d’une année ne servît à soulager la disette de l’autre, elle a occasion­né, dans les mauvaises années, une importation plus forte que celle qui eût été néces­saire sans cette institution. Dans les deux cas, la prime a donné plus d’occupa­tion aux marchands de blé, et dans les années de cherté, non-seulement elle les a mis dans le cas d’importer une plus grande quantité, mais encore de vendre à un meilleur prix et, par conséquent, avec de plus gros profits qu’ils n’eussent pu le faire si le pro­duit surabondant d’une année n’eût pas été plus ou moins détourné de venir suppléer au déficit d’une autre. Aussi est-ce dans cette classe de gens que j’ai remarqué la plus grande chaleur pour le renouvellement ou la continuation de la prime[12]. 
Il semble que nos propriétaires ruraux, en imposant à l’importation des blés étrangers de gros droits qui, dans les temps d’une abondance moyenne, équivalent à une prohibition, et en établissant la prime à l’exportation, aient pris exemple sur la conduite de nos manufacturiers. Par l’une de ces mesures, ils se sont assuré le mono­pole du marché intérieur, et par l’autre ils ont essayé d’empêcher que ce marché ne fût en aucun temps surchargé de la marchandise dont ils sont les vendeurs. Par l’une et par l’autre, ils ont cherché à faire hausser la valeur réelle de cette marchandise, de la même manière que nos manufacturiers, à l’aide de pareils moyens, avaient fait haus­ser la valeur réelle de plusieurs différentes sortes de marchandises manufac­turées. Peut-être ils n’ont pas fait attention à la grande et essentielle différence établie par la nature entre le blé et presque toutes les autres sortes de marchandises. Lors­qu’au moyen d’un monopole sur le marché intérieur, ou d’une prime donnée à l’exportation, on met nos fabricants de toiles ou de laines à même de vendre leurs marchandises à un prix un peu meilleur que celui auquel ils les auraient données sans cela, on élève non-seulement le prix nominal, mais le prix réel de leurs marchandises. On les rend équivalentes à plus de travail et à plus de subsistances ; on augmente non-seulement le profit nominal de ces fabricants, mais leur profit réel, leur richesse et leur revenu réel ; on les met à même, ou de vivre plus à l’aise, ou d’employer plus de monde dans leur fabrique[13]. On encourage réellement ces manufactures, et on y pousse une plus grande quantité de l’industrie du pays que celle qui vraisemblablement s’y serait portée d’elle-même. Mais quand, à l’aide de mesures semblables, vous faites hausser le prix nomi­nal du blé ou son prix en argent, vous n’élevez pas sa valeur réelle. Vous n’augmentez pas la richesse réelle, le revenu réel de nos fermiers et de nos propriétaires ruraux. Vous n’encouragez pas la production du blé, parce que vous ne les mettez pas à même de faire subsister plus de monde ou d’employer plus d’ouvriers à cette production. La nature des choses a imprimé au blé une valeur réelle, à laquelle ne peuvent rien chan­ger les révolutions quelconques de son prix en argent. Il n’y a pas de monopole pour la vente au-dedans, pas de prime pour l’exportation, qui aient la puissance de faire hausser cette valeur. La concurrence la plus libre ne saurait non plus la faire baisser. Par tout le monde, en général, cette valeur est égale à la quantité de travail qu’elle peut faire subsister, et dans chaque lieu du monde en particulier elle est égale à la quantité de travail auquel elle peut fournir une subsistance aussi abondante, ou aussi médiocre, ou aussi chétive qu’il est d’usage de la fournir au travail dans cette localité particulière. La toile ni les lainages ne sont pas, parmi les marchandises, le régulateur universel qui mesure et qui détermine, en dernier résultat, la valeur réelle de toute autre marchandise ; c’est le blé qui est ce régulateur[14]. La valeur réelle de toute autre marchandise se règle et se mesure définitivement sur la proportion qui se trouve exister entre son prix moyen en argent et le prix moyen du blé en argent. Au milieu de ces variations, qui arrivent quelquefois d’un siècle à l’autre, dans le prix moyen du blé en argent, la valeur réelle du blé reste immuable ; c’est la valeur réelle de l’argent qui suit le cours de ces variations[15]. 
Les primes à l’exportation pour toute marchandise fabriquée chez nous peuvent être combattues, premièrement, par cette objection générale qu’on peut appliquer à tous les divers expédients du système mercantile, savoir, qu’elles poussent par force quelque partie de l’industrie nationale dans un canal moins avantageux que celui dans lequel elle se serait portée d’elle-même ; et secondement, par cette objection, particu­lière à la prime, qu’elle pousse par force cette portion d’industrie, non-seulement dans un canal qui est moins avantageux, mais même désavantageux pour le moment, puisqu’un commerce qui ne peut marcher qu’à l’aide d’une prime est nécessairement un commerce à perte. Mais la prime pour l’exportation du blé est susceptible encore d’une autre objec­tion, c’est qu’elle ne peut augmenter en rien la production de la denrée dont elle s’est proposé d’encourager la culture. Ainsi, quand nos propriétaires ruraux demandèrent l’établissement de la prime, s’ils agirent à l’imitation de nos marchands et de nos manufacturiers, ils n’agirent pas cependant avec cette parfaite intelligence de leur pro­pre intérêt qui dirige ordinairement la conduite de ces deux autres classes[16] ; ils grevèrent le revenu public d’une dépense énorme ; ils établirent un impôt très-onéreux sur la masse du peuple, mais ils ne parvinrent pas pour cela à augmenter, d’une manière tant soit peu sensible, la valeur réelle de leur marchandise : en rabaissant de quelque chose la valeur réelle de l’argent, ils découragèrent à un certain point l’indus­trie générale du pays et, au lieu d’avancer l’amélioration de leurs terres, qui dépend toujours nécessairement de l’état où se trouve l’industrie générale du pays, ils la retardèrent plus ou moins[17].
On pourrait penser que, pour encourager la production d’une marchandise quel­conque, une prime accordée à la production aurait un effet plus direct qu’une prime accordée à l’exportation ; celle-là d’ailleurs n’établirait d’autre impôt sur le peuple que celui qu’il faudrait payer pour acquitter la dépense publique de la prime. Au lieu de faire monter le prix de la marchandise sur le marché intérieur, elle tendrait à le faire baisser et, par là, au lieu de grever le peuple d’un second impôt, elle pourrait au moins, en partie, offrir une sorte de dédommagement pour ce que lui aurait coûté le premier[18]. Cependant, ce genre de prime n’a été que très-rarement accordé ; les préjugés établis par la doctrine du système mercantile nous ont accoutumés à croire que la richesse nationale procède plus immédiatement de l’exportation que de la production ; celle-là, en conséquence a été bien plus favorisée, comme étant la source la plus immédiate de l’affluence de l’argent dans le pays. On a dit aussi que, d’après l’expé­rience, les primes sur la production avaient été reconnues plus sujettes à la fraude que celles à l’exportation. je ne sais pas jusqu’à quel point cela peut être vrai ; ce qu’il y a de bien avéré, c’est que les primes à l’exportation ont donné lieu à une infinité de fraudes différentes. Mais les marchands et les manufacturiers, les grands inventeurs de tous ces expédients, ne trouveraient pas leur compte à ce que le marché intérieur vînt à être surchargé de l’espèce de marchandise dont ils font commerce, événement qui pourrait quelquefois être la suite d’une prime sur la production. Une prime à l’exportation, en les mettant à même de vendre au-dehors le superflu et de maintenir le prix du reste sur le marché intérieur, est un moyen efficace d’empêcher que cela n’arrive ; aussi, de tous les expédients du système mercantile, est-ce un de ceux qu’ils vantent le plus. J’ai vu les entrepreneurs de certaines manufactures convenir entre eux de donner de leur poche une prime à l’exportation d’une portion déterminée de la marchandise dont ils faisaient commerce ; l’expédient leur réussit si bien, qu’il fit plus que doubler le prix de leurs produits sur le marché intérieur, malgré une augmenta­tion considérable dans la quantité fabriquée. Il faut que la prime à l’exportation du blé ait agi d’une manière prodigieusement différente, si elle a fait baisser le prix en argent de cette denrée[19].
Cependant, dans certaines occasions, on a accordé à la production quelque chose qui ressemblait à une prime. Les primes par tonneau[20], données à la pêche du hareng blanc[21] et à celle de la baleine, pourraient peut-être passer pour des primes de ce genre. On peut croire qu’elles tendent directement à rendre la marchandise moins chère sur le marché intérieur, qu’elle ne l’aurait été sans elles ; mais il faut convenir qu’à d’au­tres égards elles ont les mêmes effets que les primes à l’exportation ; elles font qu’une partie du capital du pays est employée à mettre au marché des marchandises dont le prix ne suffirait pas pour rendre ce qu’elles auraient coûté, plus les profits ordinaires des capitaux[22].
Mais si les primes par tonneau, accordées à ces pêches, ne contribuent pas à enrichir la nation, on pourrait penser peut-être qu’elles tendent à multiplier ses moyens de défense, en augmentant le nombre de ses vaisseaux et de ses matelots. On alléguera que ces sortes de primes atteignent ce but à beaucoup moins de frais que ne le ferait l’entretien, en temps de paix, d’une grande marine militaire toujours sur pied, si je puis me permettre cette expression, comme on fait à l’égard des troupes réglées de terre.
Néanmoins, malgré la faveur que méritent ces allégations, les considérations suivantes me disposent à croire qu’en accordant ces sortes de primes, il y en a une au moins sur laquelle la législature a été grandement induite en erreur.
Premièrement, la prime sur la pêche du hareng, faite par des buyses[23], paraît trop forte.
Depuis le commencement de la pêche de l’hiver de 1771 jusqu’à la fin de l’hiver de 1781, la prime sur la pêche du hareng, par buyses, s’est élevée à 30 sch. par tonneau ; pendant ces onze années, le nombre total des barils de harengs pêchés par les buyses écossaises faisant cette pêche a été à 378,347. Les harengs, tels qu’ils sont quand on les a pêchés et préparés[24] à la mer, se nomment bâtons de mer[25]. Pour en faire ce qu’on nomme des harengs marchands, il faut les regarnir avec une quantité additionnelle de sel en les encaquant une seconde fois, et dans ce cas on compte que trois barils de bâtons de mer font d’ordinaire deux barils de harengs marchands ; ainsi, d’après ce compte, le nombre de barils de harengs marchands pris pendant ces onze années, ne sera plus que de 252,23 1/2. À Pendant ces onze années, les primes par tonneau qui ont été payées pour cette pêche, se sont montées à 155,463 livres 11 sch., ou bien à 8 sch. 2 d. 1/2, par chaque baril de bâtons de mer, et à 12 sch. 3 d. 3/4 par chaque baril de harengs marchands.
Le sel avec lequel on prépare ces harengs est quelquefois du sel d’Écosse et quelquefois du sel étranger ; l’un et l’autre sont livrés aux saleurs du hareng, franc de tout droit d’accise ; ce droit sur le sel d’Écosse est à présent de 1 schelling 6 den. par boisseau, et celui sur le sel étranger de 10 schellings. On suppose qu’un baril de harengs emploie environ un boisseau et un quart de sel étranger, et qu’en sel d’Écosse, il en emploie environ deux boisseaux. Si les harengs sont entrés pour l’exportation, on ne paye aucune partie du droit ; s’ils sont entrés pour la consommation intérieure, qu’on ait employé du sel étranger ou du sel d’Écosse, on ne paye que 1 schelling par baril de harengs ; c’était l’ancien droit d’Écosse sur le boisseau de sel, quantité qu’on avait évaluée au plus bas comme la quantité du sel nécessaire pour la préparation d’un baril de harengs. En Écosse, on ne fait guère usage du sel étranger que pour les salaisons du poisson ; or, du 5 avril 1771 au 5 avril 1782, la quantité de sel étranger importée s’est élevée à 936,974 boisseaux, du poids de quatre-vingt-quatre livres chacun ; la quantité de sel d’Écosse livrée aux saleurs de poissons ne s’est pas élevée à plus de 168,226 boisseaux, du poids de cinquante-six livres seulement ; il semblerait donc que, dans les pêcheries, on fait principalement usage du sel étranger. Il y a, en outre, sur chaque baril de harengs exportés, une prime de 2 schellings 8 deniers ; et plus des deux tiers de harengs pris par les buyses sont pour l’exportation. Additionnez tout cela, et vous trouverez que, pendant ces onze années, chaque baril de harengs pêchés par les buyses et salés en sel d’Écosse, quand il a été exporté, a coûté au gouvernement 17 schellings Il deniers 3/4, et quand il est entré pour la consommation intérieure, lui a coûté 14 schellings 3 deniers 3/4 ; que, pour chaque baril salé avec du sel étranger, le gouvernement a payé, si le baril a été exporté, 1 livre 7 schellings 5 deniers 3/4, et s’il est entré pour la consommation intérieure, 1 livre 3 schellings 9 deniers 3/4 : or, le prix d’un baril de bons harengs marchands varie de 17 et 18 schellings à 24 et 25, environ une guinée le prix moyen[26].
Secondement, la prime pour la pêche du hareng étant une prime par tonneau, elle est proportionnée à la charge du bâtiment, et non pas à la promptitude ou au succès de la pêche ; et j’ai peur qu’il ne soit aussi arrivé souvent que des bâtiments aient mis en mer pour courir, non après le poisson, mais après la prime. En 1759, lorsque la prime était de 50 schellings par tonneau, toute la pêche des buyses d’Écosse n’a rapporté que quatre barils seulement de bâtons de mer ; cette année-là, chaque baril de bâtons de mer coûta au gouvernement, en primes seulement, 113 livres 15 schellings ; ce qui fit, pour chaque baril de harengs marchands, 159 livres 7 schellings 6 deniers.
Troisièmement, la méthode de pêcher, pour laquelle la prime par tonneau a été accordée à la pêche du hareng, c’est à-dire de pêcher par buyses (ou bâtiments pontés, de vingt à vingt-huit tonneaux de port), ne paraît pas aussi bien convenir à la situation de l’Écosse qu’elle convient à celle de la Hollande, dont on a emprunté, à ce qu’il paraît, cette pratique. La Hollande est située à une grande distance des mers où l’on sait que se trouve principalement le hareng et, par conséquent, elle ne peut établir cette pêche qu’à l’aide de bâtiments pontés qui puissent porter assez d’eau et de vivres pour un trajet à des parages assez éloignés. Mais les Hébrides ou îles de l’ouest, les îles de Shetland et les côtes du nord et nord-ouest de l’Écosse, pays dans le voisinage desquels se fait principalement la pêche du hareng, sont partout entrecoupées par des bras de mer qui s’enfoncent considérablement dans les terres, et que, dans le langage du pays, on nomme lacs de mer[27]. C’est dans ces lacs de mer que se rend principale­ment le hareng dans les temps de son passage dans ces mers ; car je crois que le passage de ce poisson, ainsi que de plusieurs autres espèces, n’est pas tout à fait cons­tant et régulier. Ainsi, la pêche par bateau paraît être la manière de pêcher la plus con­ve­nable à la situation particulière de l’Écosse, les pêcheurs portant alors les harengs sur le rivage aussitôt qu’ils sont pris, pour y être salés ou consommés frais. Mais le grand encouragement qu’une prime de 30 schellings par tonneau donne à la pêche par buyses, décourage nécessairement la pêche par bateau, qui, ne jouissant pas d’une pareille faveur, ne peut pas mettre au marché son poisson salé au même compte que le fait la pêche par les buyses. Aussi la pêche par bateau, qui, avant l’établisse­ment de la prime sur la pêche par buyses, était très-considérable, et employait alors, dit-on, un nombre de gens de mer qui n’était pas inférieur à celui que la pêche par buyses emploie aujourd’hui, est à présent presque entièrement tombée. je dois conve­nir cependant que je ne prétends pas pouvoir parler avec précision de l’ancienne étendue de cette pêche, aujourd’hui tombée et abandonnée ; comme on ne payait pas de prime sur les bateaux expédiés pour cette pêche, les officiers des douanes ou des droits sur le sel n’en ont tenu aucun état.
Quatrièmement, dans beaucoup d’endroits d’Écosse, pendant un certain temps de l’année, les harengs font une partie assez considérable de la nourriture des gens du peuple. Une prime qui tendrait à faire baisser leur prix sur le marché intérieur pourrait contribuer de beaucoup au soulagement d’un grand nombre de nos concitoyens les moins aisés. Mais la prime sur la pêche par les buyses ne tend pas à atteindre un but aussi utile ; elle a ruiné la pêche par bateau, qui est, sans comparaison, plus propre à fournir le marché intérieur ; et la prime additionnelle de 2 schellings 8 deniers par baril, lors de l’exportation, fait sortir la plus grande partie, plus des deux tiers, du produit de la pêche faite par les buyses. Il y a trente ou quarante ans, avant l’établisse­ment de la prime donnée aux buyses, le prix ordinaire du baril de harengs, à ce qu’on m’a dit, était de 16 schellings. Il y a dix à quinze ans, avant que la pêche par bateau fût entièrement tombée, le prix était, dit-on, de 17 à 20 schellings le baril. Ces cinq dernières années, il a été, au prix moyen, à 25 schellings le baril ; cependant ce haut prix peut bien avoir été l’effet de la rareté qui s’est fait sentir dans le hareng, sur la côte d’Écosse. je dois faire observer de plus que la caque ou baril qui se vend d’ordinaire avec les harengs, et dont le prix est compris dans tous les prix ci-dessus, est monté environ au double de son ancien prix, c’est-à-dire de 3 schellings à environ 6, depuis le commencement de la guerre d’Amérique. je ferai observer aussi que les rapports que j’ai reçus des prix des anciens temps ne sont pas du tout uniformes ni d’accord entre eux, et un vieillard fort expérimenté et de la plus grande exactitude m’a assuré qu’il y a plus de cinquante ans, le prix ordinaire d’un baril de bons harengs marchands était d’une guinée, prix, qui, selon mon opinion, peut encore aujourd’hui être regardé comme le prix moyen. D’ailleurs, tous les rapports s’accordent, je crois, pour prouver que la prime donnée à la pêche du hareng par les buyses n’a pas fait baisser sur le marché intérieur le prix de cette denrée.
Quand on voit les entrepreneurs de pêcheries, après tant de primes qui leur ont été si libéralement accordées, continuer à vendre leur marchandise au même prix et même à un plus haut prix qu’ils n’avaient coutume de le faire auparavant, on devrait penser que leurs profits doivent être énormes, et il n’est pas sans vraisemblance que quelques particuliers n’en aient fait de tels. Cependant, en général, j’ai tout lieu de croire qu’il en a été tout autrement. L’effet ordinaire de pareilles gratifications est d’encourager des gens hasardeux et téméraires à s’aventurer dans des affaires auxquelles ils n’entendent rien, et ce qu’ils perdent par ignorance ou négligence fait plus que compenser ce que l’extrême libéralité du gouvernement peut leur faire gagner. En 1750, le même acte qui accorda le premier la prime de 30 schellings par tonneau pour l’encouragement de la pêche du hareng (celui de la vingt-troisième année de Georges II, ch. xxiv), érigea une com­pa­gnie par actions, avec un capital de 500,000 livres ; on accorda aux sous­crip­teurs, outre tous les encouragements ci-dessus, c’est-à-dire la prime par ton­neau dont nous parlons ici, celle de 2 schellings 8 deniers par baril à l’exportation, les livraisons de sel national et étranger franc de tous droits et, de plus encore, une prime de 3 livres par année, pendant un espace de quatorze ans, par chaque 100 livres de souscription versées dans les fonds de la société, laquelle annuité leur serait payée par moitié, de six en six mois, par le receveur général des douanes. Outre cette grande compagnie, dont le gouverneur et les directeurs devaient résider à Londres, il fut permis d’établir différentes chambres ou compagnies pour la pêche dans tous les différents ports de départ du royaume, pourvu que leurs souscriptions ne formassent pas au-dessous d’un capital de 10,000 livres pour chacune, qui serait régi à leurs risques et à leurs profits et pertes. La même annuité et les mêmes encouragements de toute espèce furent accor­dés au commerce de ces compagnies inférieures, comme à celui de la grande compa­gnie. La souscription de la grande compagnie fut bientôt remplie, et plusieurs diffé­rentes autres compagnies inférieures s’établirent dans les différents ports de départ du royaume. Malgré tous ces encouragements, presque toutes ces différentes compa­gnies, grandes ou petites, perdirent ou la totalité, ou la majeure partie de leurs capi­taux ; à peine reste-t-il aujourd’hui quelque trace d’une seule d’entre elles, et mainte­nant la pêche du hareng est entièrement ou presque entièrement faite par des spéculateurs particuliers[28].
À la vérité, si quelque fabrique particulière était nécessaire à la défense nationale, il pourrait bien n’être pas très-sage de rester en tout temps dans la dépendance de ses voisins pour l’approvisionnement ; et si une fabrique de ce genre ne pouvait pas se soutenir chez nous sans protection, il serait assez raisonnable que toutes les autres branches d’industrie fussent imposées pour l’encourager. Peut-être pourrait-on justi­fier, d’après ce principe, les primes à l’exportation des voiles de marine et de la poudre de fabrique anglaise.
Mais quoiqu’il y ait très-peu de cas où il soit raisonnable de grever l’industrie générale pour encourager celle de quelque classe particulière de manufacturiers, cependant, dans l’ivresse d’une grande prospérité, quand l’État jouit d’un revenu si grand qu’il ne sait trop qu’en faire, de pareilles primes accordées à des genres de manu­factures qui sont en faveur, sont des dépenses aussi excusables que toute autre dépense inutile à laquelle on pourrait se livrer. Dans les dépenses publiques, comme dans celles des particuliers, de grandes richesses peuvent quelquefois légitimer de grandes profusions. Mais assurément c’est quelque chose de plus qu’une folie ordinaire, que de continuer de pareilles dépenses dans des moments de détresse et d’embarras général.
Quelquefois ce qu’on nomme Prime n’est autre chose qu’une Restitution des droits et, par conséquent, n’est pas susceptible des mêmes objections que la prime propre­ment dite. Par exemple, la prime sur l’exportation du sucre raffiné peut être regardée comme une restitution des droits payés sur les sucres bruns ou moscouades avec lesquels il est fait. La prime à l’exportation des soieries est une sorte de restitution des droits payés à l’importation de la soie écrue, ou simplement filée ; celle sur l’expor­tation de la poudre, une restitution des droits payés à l’importation du soufre ou du salpêtre. Dans la langue des douanes, on n’appelle Restitution (drawback) que ce qui s’accorde à l’exportation des marchandises étant encore sous la même forme où elles ont été importées. On l’appelle Prime dès que la marchandise exportée a subi par la main-d’œuvre une modification qui lui a fait changer de dénomination.
Les prix que donne l’État à des artistes ou à des fabricants qui excellent dans leur profession ne sont pas susceptibles des mêmes objections que les primes.
En encourageant un talent ou une dextérité extraordinaire, ils servent à entretenir l’émulation des ouvriers alors employés dans ces mêmes genres d’occupation, et ils ne sont pas assez considérables pour détourner vers un de ces emplois une plus grande portion du capital du pays que celle qui y aurait été d’elle-même. Ils ne tendent pas à renverser l’équilibre naturel entre les divers emplois, mais à rendre aussi fini et aussi parfait que possible le travail qui se fait dans chacun d’eux. D’ailleurs, la dépense des Prix n’est qu’une bagatelle, celle des primes est énorme ; la seule Prime sur le blé a coûté quelquefois à l’État plus de 300,000 livres dans une seule année.
Quelquefois les Primes sont appelées Prix, tout comme les Drawbacks sont quelquefois appelés Primes. Mais il faut toujours s’attacher à la nature de la chose en elle-même, sans s’embarrasser des termes.

DIGRESSION
Sur le commerce des blés et sur les lois y relatives.[29]

Je ne puis terminer ce chapitre sur les primes, sans observer qu’il n’y a rien de moins mérité que les éloges qu’on a donnés à la loi qui établit la prime à l’exportation des blés, et à tout ce système de règlements qui s’y trouve lié. Pour démontrer la vérité de mon assertion, il ne faut que faire un examen particu­lier de la nature du commerce des blés et des principales lois anglaises relatives à ce commerce. La longueur de la digression sera bien justifiée par la grande importance du sujet.
Le commerce de marchand de blé se compose de quatre branches différentes, qui peuvent bien quelquefois être exercées par la même personne à la fois, mais qui n’en constituent pas moins, par leur nature, quatre commerces distincts et séparés. Ces branches sont :
1° Le commerce du marchand qui trafique sur le blé dans l’intérieur seulement ;
2° Celui du marchand qui importe du blé étranger pour la consommation du pays ;
3° Celui du marchand qui exporte à l’étranger le blé produit dans le pays ;
Et 4° celui du marchand voiturier ou du marchand qui importe du blé étranger, dans la vue de le réexporter ensuite.

§ 1. — Commerce intérieur.

L’intérêt du marchand qui commerce sur les blés dans l’intérieur, et l’intérêt de la masse du peuple, quelque opposés qu’ils puissent paraître au premier coup d’œil, sont pourtant précisément semblables, dans les années mêmes de la plus grande cherté.
L’intérêt de ce marchand est de faire monter le prix de son blé aussi haut que le peut exiger la disette réelle du moment, et ce ne peut jamais être son intérêt de le faire monter plus haut. En faisant monter le prix, il décourage la consommation[30] et met tout le monde, plus ou moins, mais particulièrement les classes inférieures du peuple, dans le cas d’épargner sur cet article et de vivre de privations. Si, en élevant ce prix trop haut, il décourage la consommation au point que la provision de l’année puisse dépasser la consommation de l’année et durer quelque temps après la rentrée de la récolte suivante, il court le risque, non-seulement de perdre une partie considérable de son blé, par des causes naturelles, mais encore de se voir obligé de vendre ce qui lui en reste, pour beaucoup moins qu’il aurait pu en retirer quelques mois auparavant. Si, en ne faisant pas monter le prix assez haut, il décourage si peu la consommation que la provision de l’année soit dans le cas de ne pouvoir atteindre à la consommation de l’année, non-seulement il perd une partie du profit qu’il eût pu faire, mais encore il expose le peuple à souffrir avant la fin de l’année, au lieu de simples rigueurs d’une cherté, les mortelles horreurs d’une famine. C’est l’intérêt du peuple, que sa consom­mation du mois, de la semaine, du jour, soit proportionnée aussi exactement que possible à la provision existante. Or, l’intérêt du marchand qui commerce sur le blé dans l’intérieur est absolument le même[31]. En mesurant au peuple sa provision dans cette proportion, aussi exactement qu’il lui est possible d’en juger, il se met dans le cas de vendre tout son blé au plus haut prix et avec le plus gros profit qu’il puisse faire ; et la connaissance qu’il a de l’état de la récolte, ainsi que du montant de ses ventes du mois, de la semaine, du jour, le met à portée de juger, avec plus ou moins de précision, si réellement le peuple se trouve approvisionné dans cette proportion. Sans se régler sur l’intérêt du peuple, son intérêt personnel le porte nécessairement à traiter le peuple, même dans les années de disette, à peu près de la même manière qu’un prudent maître de vaisseau est quelquefois obligé de traiter son équipage. Quand ce maître prévoit que les vivres sont dans le cas de pouvoir manquer, il dimi­nue la ration de son monde. Quand même il lui arriverait de le faire par excès de précaution et sans une nécessité réelle, encore tous les inconvénients qu’en pourrait souffrir l’équipage ne sont-ils rien en comparaison des dangers, de la misère et de la mort, auxquels une conduite moins prévoyante pourrait quelquefois les exposer. De même, quand on supposerait que, par excès de cupidité, le marchand de blé vînt à faire monter le prix de son blé plus haut que ne l’exige la disette de la saison, une telle conduite, qui garantit efficacement le peuple d’une famine pour la fin de l’année, ne peut causer à ce même peuple que des inconvénients peu considérables en comparai­son des dangers qu’il aurait eus à courir si, dans le commencement de l’année, le marchand eût agi à son égard d’une manière plus généreuse[32]. Le marchand de blé est celui que cet excès d’avarice expose à en souffrir le plus, non-seulement à cause de l’indignation générale qu’elle excite contre lui, mais encore, en supposant qu’il échappe aux suites de cette indignation, à cause de la quantité de blé que sa cupidité lui laisse nécessairement sur les bras à la fin de l’année, et qu’il se verra obligé, si l’année suivante est favorable, de vendre à un prix beaucoup plus bas que celui qu’il aurait pu en retirer sans cela.
S’il était possible, à la vérité, qu’une compagnie de marchands vînt à se rendre maîtresse de la totalité de la récolte d’une grande étendue de pays, alors il pourrait bien être de son intérêt de faire de cette récolte ce qu’on dit que les Hollandais font des épiceries des Moluques, c’est-à-dire d’en jeter ou d’en détruire une partie consi­dérable, pour tenir le reste à haut prix. Mais il est presque impossible, même quand on abuserait pour cela de la force des lois, de venir à bout d’établir à l’égard du blé un monopole aussi étendu ; et toutes les fois que la loi laisse le commerce libre, c’est, de toutes les marchandises, celle qui est le moins sujette à pouvoir être accaparée ou mise en monopole à l’aide de gros capitaux et par des achats faits à l’avance. non-seulement sa valeur excède de beau­coup ce que les capitaux de quelques particuliers seraient jamais en état d’acheter, mais même, en supposant ces capitaux assez forts pour cela, la manière dont cette marchandise est produite rend un pareil achat absolument impraticable. Comme dans tout pays civilisé c’est la marchandise dont la consommation annuelle est la plus forte, aussi y a-t-il annuellement une plus grande quantité d’industries employées à la produire, qu’il n’y en a à produire toute autre marchandise. De plus, au moment où le blé est séparé de la terre, il se divise nécessairement entre un plus grand nombre de propriétaires que toute autre marchandise, et ces propriétaires ne peuvent jamais être rassemblés dans un lieu comme le seraient un nombre de manufacturiers indépen­dants, mais ils sont nécessairement disséminés dans tous les différents coins du pays. Ces premiers propriétaires de blé, ou fournissent immédiatement les consommateurs de leur voisinage, ou fournissent d’autres vendeurs de blé dans l’intérieur, qui fournis­sent ces consommateurs. Par conséquent, les vendeurs de blé dans l’intérieur, y compris le fermier ainsi que le boulanger, sont nécessairement plus nombreux que les vendeurs de toute autre denrée, et la manière dont ils sont dispersés rend absolument chimérique toute possibilité d’une ligue générale entre eux. Ainsi, si, dans une année de disette, quelqu’un d’eux venait à s’apercevoir qu’il eût par-devers lui une plus grande quantité de blé qu’il ne pourrait espérer d’en débiter au prix courant avant la fin de l’année, il ne s’aviserait jamais de chercher à maintenir le prix élevé à son pro­pre détriment et pour le bénéfice seul de ses rivaux et de ses concurrents ; mais, au contraire, il le ferait aussitôt baisser, pour pouvoir se défaire de tout son blé avant la rentrée de la nouvelle récolte. Les mêmes motifs, le même intérêt qui régleraient ainsi la conduite de ce vendeur, régleraient pareillement celle de tout autre, et les oblige­raient tous, en général à vendre leur blé au prix qui, d’après le meilleur jugement qu’ils en pourraient porter, s’accorderait le mieux avec l’état de disette ou l’abondance de la saison.
Quiconque examinera avec attention l’histoire des chertés et des famines qui ont affligé quelques parties de l’Europe, pendant le cours de ce siècle ou des deux précédents, sur plusieurs desquelles nous avons des renseignements fort exacts, trouvera, je crois, qu’une cherté n’est jamais venue d’aucune ligue entre les vendeurs de blé de l’intérieur, ni d’aucune autre cause que d’une rareté réelle du blé[33] occasionnée peut-être quelquefois, et dans quelques lieux particuliers, par les ravages de la guerre, mais dans le plus grand nombre des cas, sans comparaison, par les mauvaises années ; tandis qu’une famine n’est jamais provenue d’autre cause que des mesures violentes du gouvernement et des moyens impropres employés par lui pour tâcher de remédier aux inconvénients de la cherté. 
Dans un pays à blé d’une grande étendue, entre les différentes parties duquel il y a liberté de communication et de commerce, jamais la disette causée par les plus mau­vaises années ne peut être assez grande pour amener une famine ; et la plus misérable récolte, ménagée avec économie et avec frugalité, fera subsister, pendant toute l’an­née, le même nombre de gens qui, dans les années d’abondance moyenne, sont nour­ris plus largement. Les années les plus contraires au blé, ce sont celles d’une excessive sécheresse, ou celles qui sont excessivement pluvieuses. Or, comme le blé croît également sur les terres basses et sur les terres élevées, sur des terres qui sont situées de manière à être trop humides et sur celles qui sont trop exposées à la sécheresse, il s’ensuit que les pluies ou les sécheresses qui sont contraires à certains cantons du pays sont favorables à d’autres, et que si, dans les années de pluie ou de sécheresse, la récolte se trouve, il est vrai, de beaucoup au-dessous de ce qu’elle est dans une année convenablement tempérée, cependant, même dans celles-là, ce qui est perdu dans une des parties du pays se trouve, jusqu’à un certain point, compensé par ce qu’on gagne dans l’autre. Dans les pays à riz, où la récolte exige non-seulement un terrain très-humide, mais où elle a besoin même-, dans une certaine période de sa croissance, d’être ensevelie sous l’eau, les effets d’une sécheresse sont bien plus pernicieux. Toutefois, même dans ces contrées, la sécheresse n’est peut-être jamais assez générale pour y occasion­ner nécessairement une famine, si le gouvernement laisse au commerce sa liberté. La sécheresse qui eut lieu au Bengale, il y a quelques années, aurait vraisemblablement occasionné une très-grande disette. Quelques règlements impropres, quelques entra­ves absurdes mises, par les facteurs de la Compagnie des Indes, au commerce du riz, sont peut-être ce qui a contribué à changer cette disette en une famine.
Quand le gouvernement, pour remédier aux inconvénients d’une cherté, oblige tous les vendeurs de blé à vendre leur marchandise à ce qu’il lui plaît d’appeler un prix raisonnable, alors, ou il les empêche de porter leur blé au marché, ce qui peut quelquefois causer une famine, même dans le commencement de l’année, ou bien, s’ils l’y portent, il met le peuple dans le cas de consommer ce blé si vite, et il encou­rage dès lors tellement la consommation, qu’il doit nécessairement amener une famine avant la fin de l’année. Le commerce de blé sans restriction, sans gênes, sans limites, qui est le préservatif le plus efficace contre les malheurs d’une famine, est aussi le meilleur palliatif des inconvénients d’une disette réelle ; il ne peut y avoir que des adoucissements. Aucun commerce ne mérite mieux la protection la plus entière de la loi, et aucun commerce n’en a autant besoin, parce qu’il n’y en a aucun qui soit aussi exposé à l’animosité populaire.
Dans les années de disette, les classes inférieures du peuple imputent leur détresse à l’avarice du marchand de blé, qui devient l’objet de leur haine et de leur fureur. Aussi, au lieu de faire des profits dans ces occasions, il est souvent en danger d’être totalement ruiné, et d’avoir ses magasins pillés et détruits par leurs violences. C’est cependant dans les années de disette, quand le prix est élevé, que le marchand de blé s’attend à réaliser ses plus grands profits. En général, il a des marchés passés avec des fermiers, pour lui fournir une certaine quantité de blé à un prix fixe, pour un nombre d’années déterminé[34]. Ce prix de contrat s’établit sur ce qu’on suppose le prix modéré et raisonnable, c’est-à-dire le prix ordinaire ou moyen, lequel, avant ces dernières années de disette, était communément environ de 28 schellings le quarter de blé froment, et, pour les autres grains, à proportion. Ainsi, dans les années de disette, le marchand de blé achète une grande partie de son blé au prix ordinaire, et le revend à un prix beaucoup plus élevé. Ce qui démontre pourtant assez clairement que ce profit extraordinaire n’excède pas ce qu’il faut pour porter son commerce au niveau des autres commerces et pour compenser les pertes nombreuses qu’il a à essuyer dans d’autres circonstances, tant à cause de la nature périssable de la marchandise en elle-même, qu’à cause des variations fréquentes et imprévues de son prix, c’est cette seule observation, que les grandes fortunes ne sont pas plus communes dans ce négoce que dans tout autre. Cependant, la haine populaire à laquelle il est en butte dans les années de disette, les seules années où il puisse être très-lucratif, en éloigne tous les gens qui ont de la fortune et quelque considération dans la société. Il est abandonné à une classe inférieure de marchands, et les seules gens à peu près qui soient, dans le marché intérieur, des intermédiaires entre le producteur et le consommateur, sont les meuniers, les boulangers, les fariniers, avec une quantité de malheureux regrattiers.
L’ancienne police de l’Europe, au lieu de chercher à affaiblir cette haine populaire contre un commerce si avantageux au public, paraît, au contraire, l’avoir autorisée et même encouragée.
Par les statuts des cinquième et sixième années d’Édouard VI, chap. xiv, il a été statué que quiconque achèterait du blé ou grain avec intention de le revendre, serait réputé accapareur frauduleux, et serait, pour la première fois, condamné à deux mois de prison et à une amende montant à la valeur du blé ; pour la seconde, à une incarcé­ration de six mois et à une amende du double de cette valeur ; et pour la troisième fois, mis au pilori, condamné à une incarcération aussi longue qu’il plairait au roi, et à la confiscation de tous ses biens meubles et immeubles. La police ancienne de la plupart des autres pays de l’Europe ne valait pas mieux que celle de l’Angleterre.
Il paraît que nos ancêtres s’étaient imaginé que le peuple achèterait son blé à meilleur compte du fermier que du marchand de blé, qui, à ce qu’ils craignaient, exigerait, outre le prix payé par lui au fermier, un profit excessif pour lui-même. Ils tâchèrent même d’empêcher, autant que possible, qu’aucun tiers, quel qu’il fût, pût s’entremettre entre le producteur et le consommateur ; et ce fut là l’objet d’une quantité de gênes qu’ils imposèrent au commerce de ceux qu’ils appelaient blatiers ou voitu­riers de blé. D’abord, personne ne pouvait exercer ce métier qu’en vertu d’une patente qu’il certifiât sa probité et sa bonne foi ; et pour accorder cette patente, il fallait, d’après le statut d’Édouard VI, l’autorité de trois juges de paix. Mais, par la suite, cette formalité même fut jugée une entrave insuffisante et, par un statut d’Élisabeth, le privilège d’accorder la patente fut réservé aux sessions de trimestre[35].
Par là, l’ancienne police de l’Europe cherchait à régler l’agriculture, le grand com­mer­ce des campagnes, sur des maximes tout à fait différentes de celles qu’elle avait adoptées à l’égard des manufactures, le grand commerce des villes. En ne laissant au fermier d’autres acheteurs que les consommateurs ou leurs facteurs immédiats, qui sont les blatiers et les voituriers de blé, elle tendait à l’obliger à exercer lui-même, outre son métier de fermier, celui de marchand ou détaillant de blé. Au contraire, dans presque tous les cas, elle défendait à l’artisan d’exercer le métier de vendeur en boutique, ou de détailler ses propres marchandises. Elle s’imaginait, par l’un de ces règlements, faire le bien général du pays, ou rendre le blé moins cher, sans bien comprendre peut-être comment cela pouvait se faire. Par l’autre, elle avait en vue de faire le bien d’une classe particulière de gens, les marchands en boutique, qui, a ce qu’on supposait, se trouveraient supplantés par l’ouvrier fabricant, lequel vendrait tellement au-dessous de leur prix, si on lui laissait le moins du monde la liberté de détailler, que leur commerce se trouverait totalement anéanti.
Cependant, quand même on eût permis au fabricant de tenir boutique et de vendre ses propres marchandises au détail, il n’eût pas pu vendre pour cela au-dessous du marchand ordinaire ; tout ce qu’il aurait placé de son capital dans sa boutique, il aurait fallu qu’il le retirât de son industrie. Pour porter son commerce total au niveau de tous les autres commerces, de même qu’il lui aurait fallu, sur une partie de ce capital, les profits d’un fabricant, de même il lui aurait fallu sur l’autre les profits d’un marchand en boutique. Si, par exemple, dans le lieu particulier de sa résidence, 10 p. 100 sont le taux du profit des fonds placés, soit dans les manufactures, soit dans le commerce de détail, il faudra, dans ce cas, que chaque pièce de marchandise de sa fabrique qu’il vendra dans sa boutique soit chargée d’un profit de 20 p. 100. Quand il fera passer ces pièces d’ouvrage de son atelier dans sa boutique, il faudra bien qu’il les évalue au prix auquel il les aurait vendues à un débitant ou à un marchand qui les lui aurait achetées en gros. En les évaluant plus bas, il perdrait une partie des profits du capital placé dans sa manufacture. Quand ensuite il les vendra dans sa boutique, à moins de les vendre au même prix que les aurait revendues un détaillant, il perdrait une partie des profits du capital placé dans sa boutique. Ainsi, quoiqu’il paraisse, dans cette suppo­sition, faire un double profit sur la même pièce de marchandise, cependant, comme ces marchandises auront fait successivement partie de deux capitaux distincts, il n’aura toujours fait qu’un seul profit sur la totalité du capital occupé par ces marchan­dises ; et s’il eût fait moins que ce profit, il aurait été en perte, ou il n’aurait pas employé la totalité de son capital d’une manière aussi avantageuse que la plupart de ses voisins. 
Ce qu’on défendait au fabricant, on le prescrivit en quelque sorte au fermier ; on força celui-ci de diviser son capital en deux emplois différents, d’en conserver une partie dans ses greniers et dans ses granges pour fournir d’un moment à l’autre aux besoins du marché, et d’employer l’autre à la culture de ses terres. Mais, de même qu’il n’aurait pas pu sans perte employer la dernière partie de son capital pour moins que les profits ordinaires des fonds placés dans les fermes, de même il n’aurait pas pu davantage employer l’autre pour moins que les profits ordinaires des fonds placés dans le commerce. Que le capital qui fait réellement aller un commerce de marchand de blé appartienne à une personne qu’on appelle fermier, ou à une personne qu’on appelle marchand de blé, il n’en faut pas moins, dans un cas comme dans l’autre, un profit égal qui indemnise le maître de ce capital de l’emploi qu’il en fait ainsi, pour mettre son commerce au niveau de tous les autres emplois, et pour empêcher que son intérêt ne le porte à changer cet emploi pour un autre, dès qu’il en aura la possibilité. Par conséquent, le fermier qu’on obligea ainsi à exercer le métier de marchand de blé ne se trouva pas pour cela en état de vendre son blé à meilleur marché que tout autre marchand de blé n’eût été forcé de le faire, dans le cas d’une libre concurrence.
Celui qui peut employer tout son capital dans un seul genre d’affaires a un avantage de la même espèce que l’ouvrier qui emploie tout son travail à faire une seule et même opération. De même que le dernier y acquiert une dextérité qui le met en état de fournir, avec ses mêmes deux bras, une beaucoup plus grande quantité d’ouvrage, de même l’autre acquiert une méthode tellement facile et prompte dans l’arrangement et la conduite de son commerce, dans l’achat et le débit de sa marchan­dise, qu’avec le même capital il peut mener un bien plus grand nombre d’affaires. Ainsi, de même que l’un peut ordinairement fournir son ouvrage à beaucoup meilleur marché, de même l’autre peut ordinairement livrer ses marchandises à quelque chose de moins que si son attention et son capital étaient partagés entre une grande quantité d’objets divers. La plus grande partie des fabricants ne pourraient suffire à donner leurs propres marchandises au détail à aussi bon marché qu’un actif et vigilant détaillant, dont toute la besogne se borne à les acheter en gros pour les revendre en détail. La plupart des fermiers pourraient encore bien moins suffire à donner leur propre blé au détail ou à fournir les habitants d’une ville, distante de peut-être quatre ou cinq milles du plus grand nombre d’entre eux, à aussi bon compte qu’un actif et vigilant marchand de blé, qui n’a pas autre chose à faire que d’acheter du blé en gros, de l’amasser dans de grands magasins et de le revendre en détail.
La loi qui défendit au fabricant d’exercer ce métier de vendeur en boutique tâcha d’établir forcément cette division dans les emplois des capitaux, plus promptement qu’elle n’aurait eu lieu sans cela. La loi qui obligea le fermier à exercer le métier de marchand de blé tâcha d’empêcher cette division de se faire aussi vite qu’elle se serait faite. L’une et l’autre de ces lois furent des atteintes manifestes à la liberté naturelle et, par conséquent, des injustices ; et elles furent l’une et l’autre aussi impolitiques qu’elles étaient injustes.
C’est l’intérêt de la société que des choses de ce genre ne soient ni précipitées dans leur marche ni gênées dans leur progrès. Celui qui emploie son travail ou son capital à une plus grande diversité d’objets que sa position ne lui en impose la nécessité, ne peut jamais nuire à ses voisins en vendant à meilleur compte qu’eux. Il peut seule­ment se faire tort à soi-même, et c’est en général ce qui lui arrive. L’homme de tous métiers n’est jamais riche, dit le proverbe. Mais la loi devrait toujours s’en reposer sur les gens du soin de leur intérêt personnel, comme étant eux-mêmes en général, dans leur situation locale, plus en état d’en bien juger que ne peut faire le législateur. Néanmoins, la plus pernicieuse, sans comparaison, de ces deux lois, ce fut celle qui força le fermier à faire le métier de marchand de blé.
Elle arrêta non-seulement cette division dans les emplois des capitaux, qui est toujours si avantageuse à la société, mais elle arrêta aussi les progrès de la culture et de l’amélioration des terres. En obligeant le fermier à faire deux métiers au lieu d’un, elle le mit dans la nécessité de partager son capital en deux portions, dont une seulement put être employée à la culture. S’il avait été le maître de vendre toute sa récolte à un marchand de blé à l’instant même que son blé eût été battu, la totalité de son capital serait immédiatement revenue à la terre et aurait été employée à acheter plus de bestiaux et à louer plus de domestiques pour la cultiver mieux et y faire de nouvelles améliorations ; mais, se trouvant obligé de vendre son blé au détail, il fut dans la nécessité de garder dans ses granges et ses greniers une grande partie de son capital pendant toute l’année, et il ne put par conséquent cultiver aussi bien qu’il aurait pu le faire sans cela avec le même capital. Ainsi, cette loi retarda néces­sai­re­ment l’amélioration des terres, et au lieu de rendre le blé moins cher, elle a dû contribuer à le rendre plus rare et dès lors plus cher qu’il n’aurait été sans elle.
Après l’état de fermier, celui de marchand de blé, s’il était convenablement proté­gé et encouragé, est réellement celui dont le travail contribuerait le plus à la pro­duc­tion du blé. Il soutiendrait le métier de fermier de la même manière que le commerce du marchand en gros soutient le métier de manufacturier.
Le marchand en gros, en fournissant au manufacturier le plus prompt débit, en le débarrassant de ses marchandises aussi rapidement que celui-ci peut les fabriquer, et quelque fois même en lui en avançant le prix avant qu’elles soient faites, le met en état de tenir la totalité de son capital, et quelquefois même plus que tout son capital, constamment employée à fabriquer et, par conséquent, il le met en état de fabriquer une bien plus grande quantité de marchandises que s’il était obligé de les débiter lui-même ou à ceux qui les doivent consommer immédiatement, ou même aux détail­lants. De plus, comme le capital d’un marchand en gros est suffisant pour remplacer celui de plusieurs fabricants, cette relation qui s’établit entre lui et eux intéresse le gros capitaliste à en soutenir beaucoup de petits, et à venir à leur aide dans les pertes et les malheurs, qui sans cela pourraient causer leur ruine.
Une relation du même genre, qui s’établirait généralement entre les fermiers et les marchands de blé, produirait des effets également avantageux aux fermiers. Ils se verraient à même de tenir la totalité, et même plus que la totalité de leurs capitaux, constamment employée à la culture. En cas de quelqu’un de ces accidents auxquels leur industrie est plus exposée que toute autre, ils trouveraient dans le riche marchand de blé, leur pratique ordinaire, une personne qui aurait à la fois intérêt à venir à leur secours et les moyens de le faire, et ils ne se verraient pas, comme à présent, totale­ment dépendants de l’indulgence de leur propriétaire ou de la pitié de son intendant. S’il était possible, comme il ne l’est peut-être pas, d’établir tout à la fois universel­lement cette relation, et s’il était possible aussi en même temps de rappeler à leur propre destination la totalité des capitaux de tous les fermiers du royaume, et de les ramener à la culture de la terre en les retirant de tous les autres emplois vers lesquels il peut y en avoir maintenant quelques portions de détournées ; s’il était possible enfin, pour soutenir et pour aider les opérations de cette grande masse de capitaux, d’en former tout d’un coup une autre presque aussi grande, il n’est peut-être pas aisé de se faire une idée de l’importance, de l’étendue et de la rapidité des améliorations que ce seul changement de situation produirait sur toute la surface du pays.
Ainsi, le statut d’Édouard VI, en empêchant, autant qu’il lui a été possible, qu’aucun tiers ne vînt à s’entremettre entre le producteur et le consommateur, a tâché d’ané­antir une profession dont le libre exercice est non-seulement le meilleur palliatif des inconvénients d’une disette, mais encore le plus sûr préservatif contre cette cala­mi­té ; aucune profession ne contribuant plus à la production du blé, après la profes­sion de fermier, que celle de marchand de blé.
La rigueur de cette loi fut ensuite mitigée par plusieurs statuts subséquents, qui permirent successivement d’emmagasiner le blé lorsque le prix du froment n’excé­de­rait pas 20, 24, 32 et 40 sch. le quarter. Enfin, par le statut de la quinzième année de Charles II, chap. vii, il fut déclaré que toutes personnes n’étant point intercepteurs, c’est-à-dire n’achetant pas pour revendre au même marché dans les trois mois, pour­raient librement emmagasiner ou acheter du blé pour le revendre, tant que le prix du froment n’excéderait pas 48 schellings le quarter, et celui des autres grains à pro­portion. Toute la liberté dont ait jamais joui le commerce de marchand de blé dans l’intérieur du royaume dérive de cet acte. Le statut de la douzième année du roi actuel, qui révoque presque toutes les autres anciennes lois contre les accapareurs et inter­cep­teurs, ne révoque point les restrictions portées par cet acte particulier, qui, par conséquent, restent toujours en vigueur[36]. 
Cet acte cependant autorise, jusqu’à un certain point, deux préjugés populaires très-absurdes.
En premier lieu, il suppose que quand le prix du froment est monté jusqu’à 48 sch. le quarter, et celui des autres grains à proportion, tout achat de blé en gros serait dans le cas de nuire au peuple. Or, par ce qui a été dit jusqu’à présent, il paraît assez évident qu’il n’y a aucun prix auquel l’achat du blé en gros par le marchand trafiquant dans l’intérieur du royaume puisse être préjudiciable au peuple ; et d’ailleurs, quoiq­u’on puisse regarder 48 sch. le quarter comme un très-haut prix, cependant, dans les années de disette, c’est un prix qui a souvent lieu immédiatement après la moisson, quand il y a à peine quelque partie de la nouvelle récolte en état d’être vendue, et quand il est impossible, même aux plus crédules, de supposer qu’il y en ait déjà d’acheté en gros, de manière à influer sur l’état des subsistances.
Secondement, cet acte suppose qu’il y a un certain prix auquel le blé est dans le cas d’être intercepté, c’est-à-dire acheté par avance pour être revendu bientôt après sur le même marché, de manière à porter préjudice au peuple. Mais si jamais un mar­chand intercepte du blé qui va à un marché particulier, ou l’achète sur ce marché pour le revendre bientôt après au même marché, ce ne peut être que parce qu’il juge que le marché ne saurait être aussi abondamment fourni pendant tout le cours de l’an­née que dans cette circonstance particulière et que, par conséquent, le prix doit bientôt monter. S’il juge mal à cet égard, et si le prix ne hausse pas, alors non-seulement il perd tout le profit du capital qu’il a employé à cette opération, mais encore une partie même du capital, par la dépense et la perte qu’entraînent toujours l’emmagasinement et la garde du blé. Il se nuit donc à lui-même bien plus essentiellement qu’il ne peut nuire même à ceux en particulier qu’il aura empêchés de se fournir de blé à ce même jour de marché, parce qu’ils peuvent ensuite se fournir, à tout aussi bon compte, à quelque autre jour de marché. S’il se trouve qu’il ait bien jugé, alors, au lieu de nuire à la masse du peuple, il lui aura rendu un service très-important. En faisant sentir aux gens les inconvénients d’une cherté un peu plus tôt qu’ils ne l’auraient sentie sans cela, il empêche qu’ils ne l’éprouvent d’une manière plus dure, comme cela n’eût pas man­qué d’arriver si le bon marché du blé les eût encouragés à consommer plus vite que ne le comporterait la modicité réelle de la provision de l’année. Quand la rareté du blé est réelle, la meilleure chose qu’on puisse faire pour le peuple, c’est de répartir les inconvénients de cette disette, de la manière la plus égale possible, sur tous les différents mois, semaines et jours de l’année. L’intérêt du marchand de blé fait qu’il s’étudie à faire cette répartition le plus exactement qu’il peut ; et comme aucune autre personne que lui ne saurait avoir le même intérêt à le faire, ou les mêmes connais­sances et les mêmes moyens pour le faire avec autant de précision que lui, c’est sur lui qu’il faut s’en reposer pour l’opération la plus importante de son commer­ce, ou bien, en d’autres termes, le commerce de blé, en tant qu’il a pour objet l’appro­visionnement du marché intérieur, doit être laissé parfaitement libre.
On peut comparer ces craintes du peuple contre le monopole des accapareurs et des intercepteurs aux soupçons et aux terreurs populaires qu’inspirait la sorcellerie. Les pauvres misérables accusés de ce dernier crime n’étaient pas plus innocents des malheurs qu’on leur imputait que ceux qui ont été accusés de l’autre. La loi qui a mis fin à toutes poursuites pour cause de sortilège, qui a mis hors du pouvoir d’un homme de satisfaire sa méchanceté en accusant son voisin de ce crime imaginaire, parait avoir guéri de la manière la plus efficace ces terreurs et ces soupçons, en supprimant ce qui en était l’appui et l’encouragement principal. La loi qui rendrait une entière liberté au commerce du blé dans l’intérieur, aurait vraisemblablement autant d’effica­cité pour mettre fin aux craintes du peuple contre les accapareurs et intercepteurs.
Avec toutes ces imperfections, néanmoins, le statut de la quinzième année de Charles II, chap. vii, a peut-être plus contribué qu’aucune autre loi de notre livre des statuts, tant à l’abondance des approvisionnements du marché intérieur, qu’à l’aug­men­tation de la culture du blé. C’est de cette loi que le commerce de blé dans l’inté­rieur a reçu toute la liberté et toute la protection dont il ait jamais joui jusqu’à présent, et ce commerce intérieur contribue bien plus efficacement que celui d’importation ou celui d’exportation, tant à l’abondance des approvisionnements du marché national, qu’à l’encouragement de la culture du blé.
L’auteur des Discours sur le commerce des blés a calculé que la quantité moyenne de grains de toute espèce importés dans la Grande-Bretagne était, à la quantité moyenne de grains de toute espèce qui y étaient consommés, dans une proportion qui n’allait pas au-delà de celle de 1 à 570. Ainsi, pour l’approvisionnement du marché national, l’importance du commerce intérieur des grains doit l’emporter sur celle du commerce d’importation dans le rapport de 570 à 1.
Suivant le même auteur, la quantité moyenne de grains de toute espèce exportés de la Grande-Bretagne n’excède pas la trente-unième partie du produit annuel. Par conséquent, pour encourager la culture du blé en fournissant un marché au produit du pays, l’importance du commerce intérieur doit être à celle du commerce d’exportation dans la proportion de 30 à 1.
Je n’ai pas beaucoup de foi à l’arithmétique politique, et je ne prétends pas garantir l’exactitude de l’un ni de l’autre de ces calculs. je n’en parle que pour faire voir combien, dans l’opinion des personnes qui ont le plus d’expérience et de jugement, le commerce étranger sur le blé est d’une bien moindre conséquence que le commerce intérieur. Le très-bon marché du blé, dans les années qui ont précédé immédiatement l’établissement de la prime, pourrait bien être regardé, avec quelque raison, comme étant en grande partie l’effet de ce statut de Charles 11, qui avait été porté environ vingt-cinq ans auparavant, et qui, par conséquent, avait eu tout le temps de produire son effet.
très-peu de mots suffiront pour expliquer ce que j’ai à dire sur les trois autres branches du commerce des blés.

§ II. — Commerce d’importation.

Le commerce du marchand qui importe du blé étranger pour la consommation intérieure contribue évidemment à approvisionner directement le marché national ; sous ce rapport, il est directement avantageux à la masse du peuple. Il tend, à la véri­té, à faire baisser tant soit peu le prix moyen du blé en argent, mais non pas à dimi­nuer sa valeur réelle ou la quantité de travail qu’il est capable de maintenir.
Si l’importation était libre en tout temps, nos fermiers et nos propriétaires ruraux retireraient vraisemblablement moins d’argent de leur blé, une année dans l’autre, qu’ils ne font à présent que l’importation est par le fait prohibée la plupart du temps ; mais l’argent qu’ils en retireraient aurait plus de valeur, achèterait plus de mar­chandises de toute autre espèce, et emploierait plus de travail. Par conséquent, leur richesse réelle, leur revenu réel, seraient les mêmes qu’à présent, quoique exprimés par une moindre quantité d’argent, et dès lors ils ne se trouveraient ni moins en état de cultiver, ni moins encouragés à le faire, qu’ils ne le sont à présent. Au contraire, comme une hausse dans la valeur de l’argent, procédant d’une baisse dans le prix du blé en argent, fait baisser le prix de toutes les autres marchandises, elle donne à l’industrie du pays où elle a lieu quelque avantage sur tous les marchés étrangers, et tend par là à accroître et à encourager cette industrie. Or, l’étendue du marché national pour le blé doit être en proportion de l’industrie générale du pays où il croît, ou du nombre de ceux qui produisent autre chose et qui, par conséquent, ont d’autres denrées, ou, ce qui revient au même, le prix d’autres valeurs à donner en échange pour le blé. Et le marché national, étant dans tout pays le marché le plus prochain et le plus commode pour du blé, est aussi le plus vaste et le plus important. Par consé­quent, cette hausse dans la valeur réelle de l’argent qui provient de la baisse du prix moyen du blé en argent, tend à agrandir le marché le plus vaste et le plus important pour le blé et, par conséquent, à encourager la production, bien loin de la décourager.
Par le statut de la vingt-deuxième année de Charles II, chapitre xiii, l’importation du blé froment, toutes les fois que, sur le marché national, le prix n’en excéderait pas 53 schellings 4 deniers le quarter, fut assujettie à un droit de 16 schellings le quarter, et à un droit de 8 schellings toutes les fois que le prix n’excéderait pas 4 livres. Il y a plus d’un siècle révolu que le premier de ces deux prix n’a existé, sinon dans les temps d’une très-grande disette, et le dernier, autant que je sache, n’a jamais été atteint. Cependant, à moins que le blé froment ne s’élevât au-dessus de ce dernier prix, l’importation en fut assujettie par ce statut à un très-fort droit, et tant qu’il ne s’éle­vait pas au-dessus du premier de ces prix, elle était soumise à un droit qui équi­valait à une prohibition. L’importation des autres espèces de grains fut restreinte à un certain taux, et par des droits qui, à proportion de la valeur du grain, étaient presque tous aussi élevés[37]. Des lois postérieures ont encore augmenté ces droits.
La stricte observation de ce statut dans des années de disette eût pu vraisembla­ble­ment exposer le peuple à une très-grande misère. Mais, dans de pareilles circons­tances, l’exécution en fut généralement suspendue par des statuts temporaires qui permettaient, pour un temps limité, l’importation des blés étrangers. La nécessité de ces statuts de circonstance est une démonstration suffisante de l’inconvenance du statut général.
Quoique ces entraves mises à l’importation aient précédé l’établissement de la prime, elles ont néanmoins été dictées par le même esprit, par les mêmes maximes qui dictèrent ensuite ce règlement. Quelques nuisibles qu’elles fussent en elles-mê­mes, ces restrictions et quelques autres encore sur l’importation devinrent nécess­aires, en conséquence de l’établissement de la prime. Si, lorsque le froment était au-dessous de 48 schellings le quarter, ou peu au-dessus, il eût été possible d’importer des blés étrangers, ou francs de droits, ou en payant seulement un léger droit, alors on eût pu faire de ces importations, pour réexporter ensuite avec le bénéfice de la prime ; ce qui eût causé une grande perte au revenu public et eût totalement perverti l’institution, dont l’objet était d’étendre le marché pour le produit de l’intérieur, et non pas pour le produit des pays étrangers. 
§ III. — Commerce d’exportation.

Le commerce du marchand qui exporte pour la consommation de l’étranger ne contribue certainement pas d’une manière directe à assurer l’abondance sur le marché national ; néanmoins il le fait indirectement. De quelque source que se tire habituellement cet approvisionnement du marché, que ce soit de la production intérieure ou de l’importation de l’étranger, à moins qu’ha­­bi­tuellement ou cette production intérieure ou cette importation n’excède la con­sommation ordinaire du pays, l’approvisionnement du marché national ne saurait jamais se trouver extrêmement abondant. Or, si le surplus ne peut pas, dans les cir­cons­tances ordinaires, être exporté, les producteurs auront grande attention de ne jamais en produire, et les importateurs de ne jamais en importer plus que ce qu’exige la simple consommation du marché national ; ce marché sera donc très-rarement sura­bondant ; en général, même il se trouvera mal fourni, les gens dont le métier est de l’approvisionner craignant que leur marchandise ne leur reste sur les bras. La prohi­bi­tion de l’exportation limite la culture et l’amélioration des terres du pays à ce qu’exige simplement la consommation des habitants ; la liberté de l’exportation met le pays à même d’étendre sa culture pour approvisionner les étrangers.
Par le statut de la douzième année de Charles II, chap. iv, l’exportation du blé fut permise toutes les fois que le prix du froment n’excéderait pas 40 schellings le quar­ter, et celui des autres grains à proportion. Par un acte de la quinzième année du même prince, cette liberté fut étendue jusqu’au prix qui excéderait, pour le froment, 48 schellings le quarter ; et par un autre de la vingt-deuxième année, elle fut étendue à des prix qui sont tous encore plus élevés ; à la vérité, il y avait à payer au roi un droit de tant par livre sur ces exportations ; mais tous les grains furent évalués si bas dans le livre des tarifs[38], que ce droit n’était que de 1 schelling sur le froment, 4 deniers sur l’avoine, et 6 deniers sur tous les autres grains par chaque quarter. Par l’acte de la pre­mière année de Guillaume et Marie, qui établit la prime, ce petit droit fut tacitement supprimé toutes les fois que le prix du froment n’excéderait pas 48 schellings ; et par le statut des onzième et douzième années de Guillaume III, chap. xxviii, il fut expressément supprimé pour tous les prix au-delà.
Ainsi, le commerce du marchand exportateur fut non-seulement encouragé par une prime, mais encore rendu plus libre que celui du marchand trafiquant dans l’inté­rieur. Par le dernier de ces statuts, le blé pouvait, à tout prix, être acheté en grandes quantités[39] pour l’exportation, mais on ne pouvait l’acheter de cette manière pour le revendre dans l’intérieur, à moins que le prix n’excédât pas 48 schellings le quarter. Néanmoins, comme on l’a déjà fait voir, l’intérêt du marchand qui commerce dans l’intérieur ne saurait jamais être opposé à l’intérêt de la masse du peuple ; mais celui du marchand qui exporte peut y être opposé, et dans le fait l’est quelquefois. Si, dans le temps où son propre pays souffre de la cherté, un pays voisin vient à être affligé d’une famine, ce pourrait être alors son intérêt de porter du blé à ce dernier pays en assez grande quantité pour aggraver de beaucoup dans le sien les inconvénients de la cherté. L’abondance des approvisionnements du marché intérieur n’était pas l’objet direct que se proposaient ces statuts ; mais, sous prétexte d’encourager l’agriculture, leur objet était de faire hausser le prix du blé, en argent, aussi haut que possible, et par là d’occasionner, autant que possible, une cherté constante sur le marché intérieur. Les découragements jetés sur l’importation limitaient l’approvisionnement de ce marché, même dans les temps de grande rareté de la denrée, à la production de l’inté­rieur ; tandis que les encouragements donnés à l’exportation, même quand le prix s’élevait jusqu’à 48 schellings le quarter, ne permettaient pas à ce marché de jouir de la totalité de cette production de l’intérieur, dans des temps même où la disette ne laissait pas que d’être sensible. Ce qui démontre suffisamment la défectuosité du système général des lois de la Grande-Bretagne sur cet objet, ce sont les expédients auxquels elle a été si souvent obligée de recourir, en défendant pour un temps limité l’exportation du blé par des lois de circonstance, et en supprimant aussi temporaire­ment les droits sur l’importation. Si le système eût été bon, elle ne se serait pas vue si fréquemment réduite à la nécessité de s’en écarter.
Si toutes les nations venaient à suivre le noble système de la liberté des expor­tations et des importations, les différents États entre lesquels se partage un grand continent ressembleraient à cet égard aux différentes provinces d’un grand empire. De même que parmi les provinces d’un grand empire, suivant les témoignages réunis de la raison et de l’expérience, la liberté du commerce intérieur est non-seulement le meilleur palliatif des inconvénients d’une cherté, mais encore le plus sûr préservatif contre la famine ; de même la liberté des importations et exportations le serait entre les différents États qui composent un vaste continent. Plus le continent serait vaste, plus la communication entre toutes ses différentes parties serait facile, tant par terre que par eau, et moins alors aucune de ces parties en particulier pourrait jamais se voir exposée à l’une ou à l’autre de ces calamités ; car il serait alors d’autant plus probable que la disette d’un des pays serait soulagée par l’abondance de quelque autre. Mais très-peu de pays ont entièrement adopté ce généreux système ; la liberté du commerce des blés est presque partout plus ou moins restreinte, et dans beaucoup de pays elle est gênée par des règlements tellement absurdes, que souvent ils aggravent les malheurs inévitables d’une cherté, jusqu’à faire naître le terrible fléau de la famine. La demande de blé peut souvent, dans de tels pays, être si grande et si pressante, qu’un petit État de leur voisinage qui se trouverait en même temps éprouver chez soi un certain degré de cherté, ne pourrait se hasarder à les approvisionner sans s’exposer lui-même à cette affreuse calamité. Ainsi, la police très-vicieuse d’un pays peut rendre à un certain point imprudent et dangereux d’établir dans un autre ce qui, sans cela, serait la meilleure police. Néanmoins, la liberté illimitée d’exporter serait beaucoup moins dangereuse dans de grands États, où la production étant beaucoup plus considérable, la quantité de blé qui serait dans le cas d’être exportée, quelle qu’elle fût, pourrait rarement être telle que la totalité de l’approvisionnement pût s’en ressentir. Dans un canton suisse ou dans quelqu’un des petits États de l’Italie, il se peut bien quelquefois qu’il soit nécessaire de restreindre l’exportation du blé ; il ne peut guère l’être jamais dans de grands pays, tels que la France et l’Angleterre. D’ailleurs, empêcher le fermier d’envoyer en tous temps sa marchandise au mar­ché le plus avantageux, c’est évidemment sacrifier les lois ordinaires de la justice à une considération d’utilité publique, à une sorte de raison d’État ; et c’est un acte d’autorité que la puissance législative ne peut exercer que dans le cas de la nécessité la plus urgente, seule circonstance qui puisse le rendre excusable. Si jamais l’exporta­tion du blé devait être défendue, le prix auquel elle pourrait l’être devrait toujours être un prix très-élevé.
Les lois relatives au blé peuvent généralement être comparées aux lois relatives à la religion ; le peuple a un sentiment si fort de son intérêt personnel dans les matières qui touchent à sa subsistance dans cette vie, ou à son bonheur dans une vie future, que le gouvernement est forcé de se plier à ses préjugés et d’établir, pour maintenir la tranquillité publique, un système conforme aux idées populaires. C’est peut-être pour cette raison que, sur l’un ou sur l’autre de ces deux objets capitaux, il est si rare de trouver un système qui soit raisonnable.

§ IV. — Commerce de transport.

Le commerce du marchand voiturier ou de celui qui importe du blé étranger pour réexporter, contribue à assurer l’abondance sur le marché national. À la vérité, ce n’est pas sur ce marché sur le marchand se propose de vendre son blé ; toutefois, il sera généralement disposé à l’y vendre, et même un peu au-dessous de ce qu’il espère en trouver sur le marché étranger, parce qu’il s’épargnera ainsi les dépenses du chargement et du déchargement, celles du fret et de l’assurance. Quand un pays, au moyen du commerce de transport, devient le magasin et l’entrepôt de l’approvisionnement des autres, il ne peut guère arriver que les habitants de ce pays viennent à manquer de blé. Quoique le commerce de transport puisse ainsi contribuer à réduire le prix moyen du blé en argent sur le marché national, néanmoins il ne fera pas baisser par là la valeur réelle du blé, il fera seulement hausser un peu la valeur réelle de l’argent.
Le commerce de transport pour le blé fut, par le fait, interdit dans la Grande-Breta­gne. Dans toutes les circonstances ordinaires, l’importation des blés étrangers était comme prohibée par les droits exorbitants dont elle était chargée, et qui n’étaient pas restituables, pour la plus grande partie du moins lors de l’exportation ; et dans les circonstances extraordinaires, quand une disette obligeait de suspendre ces droits par des lois temporaires, l’exportation était toujours prohibée. Ainsi, par ce système de lois, le commerce de transport se trouva, de fait, interdit dans tous les cas.
Ce système de lois, qui est lié avec l’établissement de la prime, ne parait donc nullement mériter les éloges qui lui ont été prodigués. L’amélioration et la prospérité de la Grande-Bretagne, qu’on a si souvent attribuées à ces lois, peuvent très-aisément s’expliquer par de tout autres causes. Cette assurance que donnent les lois de la Grande-Bretagne à tout individu, de pouvoir compter sur la jouissance des fruits de son propre travail, est seule suffisante pour faire prospérer un pays, en dépit de tous ces règlements de vingt autres lois de commerce qui ne sont pas moins absurdes, et cette sécurité a été portée au plus haut degré par la révolution, presque au même mo­ment où la prime a été établie. L’effort naturel de chaque individu pour améliorer sa condition, quand on laisse à cet effort la faculté de se développer avec liberté et con­fiance, est un principe si puissant, que, seul et sans autre assistance, non-seulement il est capable de conduire la société à la prospérité et à l’opulence, mais qu’il peut enco­re surmonter mille obstacles absurdes dont la sottise des lois humaines vient souvent embarrasser sa marche, encore que l’effet de ces entraves soit toujours plus ou moins d’attenter à sa liberté ou d’atténuer sa confiance. Dans la Grande-Bretagne, l’industrie jouit d’une sécurité parfaite, et quoiqu’elle soit bien éloignée d’avoir une entière liber­té, au moins est-elle aussi libre et plus libre que dans aucun autre pays de l’Europe.
Parce que l’époque de la plus grande prospérité de la Grande-Bretagne et de ses plus grands progrès dans la culture a été postérieure à ce système de lois qui est lié avec l’institution de la prime, il ne faudrait pas, pour cette raison, en faire honneur à ce système de lois. Cette époque a été aussi postérieure à la dette nationale ; or, ce qu’il y a de certain au monde, c’est qu’elle n’a pas été amenée par la dette nationale.
Quoique le système de lois qui est lié avec l’établissement de la prime ait préci­sément la même tendance que les règlements de l’Espagne et du Portugal, celle d’abais­ser un peu la valeur des métaux précieux dans le pays où il est établi, cepen­dant la Grande-Bretagne est certainement un des plus riches pays de l’Europe, tandis que l’Espagne et le Portugal sont peut-être au nombre des plus pauvres. On peut pour­tant se rendre compte de cette différence de situation d’après deux différentes causes : d’abord, la taxe, en Espagne, la prohibition, dans le Portugal, sur l’exportation de l’or et de l’argent, et la police rigoureuse qui maintient l’exécution de ces lois, doivent, dans deux pays très-pauvres, qui importent annuellement entre eux au-delà de 6 millions sterling, opérer non pas plus directement, mais encore plus puissamment la réduction de la valeur de ces métaux, que les lois sur les blés ne peuvent le faire dans la Grande-Bretagne ; secondement, cette mauvaise politique ne se trouve pas, dans ces pays-là, contre-balancée par la liberté et la sécurité générale du peuple ; l’industrie n’y jouit pas d’un libre exercice et n’y est pas animée par la confiance ; enfin, les gouver­nements tant civils qu’ecclésiastiques de ces deux royaumes sont de nature à suffire à eux seuls pour y perpétuer la misère, même quand les règlements de com­merce y seraient aussi sages qu’ils sont pour la plupart absurdes et extravagants.
L’acte de la treizième année du roi actuel paraît avoir établi, sur la législation des blés, un système nouveau, meilleur que l’ancien à bien des égards, mais qui lui est peut-être un peu inférieur sous un rapport.
Par cet acte, les droits énormes mis sur l’importation pour la consommation natio­nale sont supprimés aussitôt que le prix du blé froment de moyenne qualité s’élève jusqu’à 48 schellings le quarter, celui du seigle de moyenne qualité, des pois ou des haricots à 32 schellings, celui de l’orge à 24 schellings, et celui de l’avoine à 16 schellings ; et il établit à leur place un léger droit de 6 deniers seulement sur le quarter de blé froment, et sur celui des autres grains à proportion. Ainsi, à l’égard de toutes ces différentes sortes de grains et spécialement du blé froment, le marché national se trouve ouvert aux secours venant de l’étranger, dans le temps de chertés bien moins grandes que celles où il l’était auparavant.
Par le même acte, l’ancienne prime de 5 schellings sur l’exportation du blé cesse aussitôt que le prix s’élève à 44 schellings le quarter, au lieu de 48 schellings, prix au­quel elle cessait auparavant ; celle de 2 schellings 6 deniers sur l’exportation de l’orge cesse dès que le prix s’élève à 22 schellings au lieu de 24 schellings, prix auquel elle cessait auparavant ; celle de 2 schellings 6 deniers sur l’exportation de la farine d’avoine cesse dès que le prix s’élève à 14 schellings au lieu de 15 schellings, prix auquel elle cessait auparavant ; la prime sur le seigle est -réduite de 3 schellings 6 deniers à 3 schellings seulement, et elle n’a plus lieu dès que le prix est à 28 schellings au lieu de 32 schellings, prix auquel elle cessait auparavant. Si les primes sont une aussi mauvaise institution que j’ai tâché de le prouver, plus tôt elles cessent, plus elles sont faibles et mieux cela vaut.
Le même acte permet, dans les moments même des plus bas prix, l’importation du blé destiné à être réexporté, franche de droits, pourvu qu’en même temps le blé soit serré dans un magasin à deux clefs, dont une au roi, l’autre au marchand qui importe. Cette liberté, il est vrai, ne s’étend qu’à vingt-cinq des différentes ports de la Grande-Bretagne, mais ce sont les principaux ; et dans la plupart des autres, il ne pourrait peut-être guère s’y trouver de magasins convenables pour cet objet.
jusque-là, cette loi paraît évidemment une amélioration faite à l’ancien système.
Mais, par la même loi, on accorde une prime de 2 schellings par quarter pour l’exportation de l’avoine, toutes les fois que le prix n’excède pas 14 schellings. jusqu’à présent, il n’avait pas encore été donné de prime pour l’exportation de ce grain, non plus que pour celle des pois et haricots.
Par la même loi aussi, l’exportation du blé est prohibée dès que le prix s’élève à 44 schellings le quarter, celle du seigle à 28 schellings, celle de l’orge à 22 schellings, et celle de l’avoine à 14 schellings. Ces divers prix semblent tous beaucoup trop bas, et d’ailleurs il paraît qu’il y a une sorte d’inconséquence à prohiber l’exportation précisé­ment aux mêmes prix auxquels on retire la prime donnée pour encourager l’expor­tation. Certainement il aurait fallu, ou supprimer la prime à des prix beaucoup plus bas, ou permettre l’exportation à des prix beaucoup plus hauts.
Sous ce rapport donc, cette loi paraît inférieure à l’ancien système. Cependant, avec toutes ses imperfections, nous pouvons peut-être dire d’elle ce qui a été dit des lois de Solon, que, si elle n’est pas en elle-même la meilleure possible, du moins est-elle la meilleure que pussent comporter les intérêts, les préjugés et les circonstances des temps. Elle pourra peut-être frayer les voies à une meilleure loi dans un temps convenable.
 
 
 
↑ La seconde édition des Discours sur le commerce du blé (Tracts on the corn trade) fut publié en 1766. Depuis cette époque de grands changements ont eu lieu dans le commerce des céréales en Angleterre. Au lieu d’exporter régulièrement comme nous faisions autrefois, nous importons régulièrement depuis cinquante ans. Mais cette importation a beaucoup diminué.
Mac Culloch.
↑ Liv. I, chap. xi, sect. iii, Digression sur les variations de la valeur de l’argent, 3e période.
↑ Ses défenseurs prétendent que non-seulement elle fait baisser le prix des céréales, mais qu’en augmentant le bénéfice du fermier, elle donne en outre de grands encouragements à l’agriculture. Il est vrai qu’ils n’expliquent pas comment ces deux buts opposés peuvent être atteints. Buchanan.
↑ Et pourquoi une augmentation des salaires proportionnée à l’augmentation des prix des denrées de première nécessité ? Si la quantité de ces denrées est diminuée par l’exportation, une augmentation quelconque dans les salaires mettra-t-elle le laboureur en état de consommer la même quantité qu’auparavant ? Et si non, à quoi servira cette augmentation ? Buchanan.
↑ Ceci est une erreur, dit Mac Culloch ; le prix du blé en argent ne règle pas le prix en argent des autres choses.
↑ Mais la quantité de blé étant diminuée dans le pays par suite de l’exportation favorisée par la prime, le prix du travail évalué en argent ne peut certainement pas mettre le laboureur i même de consommer la même quantité qu’auparavant ; il devra se borner évidemment à une portion moins grande ; ceux qui prétendent au contraire que le prix du travail s’élève avec le prix du blé, admettent qu’en dépit de la diminution de la denrée la consommation doive rester la même. Buchanan.
↑ En augmentant le prix de tous les produits du sol, la prime augmente encore davantage les bénéfices du fermier et du cultivateur. Buchanan.
↑ Le prix des céréales ne modifie pas le prix des autres produits bruts du sol. Ainsi il n’a aucun rapport avec le prix des métaux et autres matières, telles que charbon, bois et pierres, et comme il ne détermine pas le prix du travail, il ne peut par conséquent régler le prix des manufactures ; de sorte que la prime, en contribuant à la hausse des céréales, n’est en définitive qu’un bénéfice très-réel pour le fermier. On ne voudra certainement pas insister sur cet argument, pour prouver l’excellence de la mesure. Il est incontestable qu’on encourage l’agriculture en produisant une hausse dans le prix des céréales, mais la question sera toujours de savoir si c’est là une bonne manière d’encourager.Buchanan.
↑ Mac Culloch fait observer que l’effet de ces restrictions n’a pu être aussi grand que le suppose Adam Smith : « Le véritable désavantage et l plus grand, dit-il, des restrictions à l’exportation des métaux précieux consiste à en augmenter la quantité d’une manière factice, et à priver le pays de la valeur des objets qu’il obtiendrait en échange de ce surplus de métal. »
↑ Le docteur Smith exagère certainement ici les inconvénients qui résultent des lois par lesquelles, en Portugal et en Espagne, l’exportation de l’or et de l’argent. est interdite. Ces inconvénients, d’ailleurs, ont maintenant disparu, et les métaux précieux arrivent en Europe par une voie différente. On lit dans le Rapport du comité de la Chambre des Communes, sur le haut prix des lingots : « Si dans le courant de l’année dernière de fortes exportations d’or pour le continent ont eu lieu, d’un autre côté des importations très-considérables de ce métal sont arrivées dans ce pays (l’Angleterre). Ces importations sont venues de l’Amérique du Sud et principalement des Indes Occidentales. Les changements survenus en Espagne et en Portugal, ainsi que les avantages maritimes et commerciaux que nous avons remportés, paraissent avoir fait de cette partie de l’Amérique la voie par laquelle les produits des mines de la Nouvelle-Espagne et du Brésil parviennent aux autres pays. Dans une pareille situation, les importations de lingots et d’argent monnayé nous mettent à même de nous pourvoir de la quantité suffisante, et la rareté de cet article pourrait en conséquence être moins sensible chez nous que sur tout autre marché. Un fait remarquable vient à l’appui de ce que nous avançons. L’argent monnayé du Portugal est maintenant envoyé régulièrement de ce pays-ci aux établissements à coton du Brésil, de Fernambouc et de Maranham ; tandis que des dollars nous arrivent en très-grande quantité de Rio-Janeiro. » Buchanan.
↑ La prime ne tend qu’à abaisser les profits, mais non à élever le prix d’aucune autre denrée excepté le blé. Mac Culloch.
↑ Le marchand de grains peut profiler de cette prime d’une manière indirecte, mais le bénéfice du propriétaire des terres est direct et clair ; et si le docteur Smith avait mieux observé, il aurait vu que ce sont principalement les propriétaires des campagnes qui sont les véritables soutiens de tout genre de prime et de monopole pour favoriser la vente de leurs produits. Buchanan.
↑ L’influence d’une prime d’exportation accordée à des produits manufactures, sur leur prix ou sur le produit du manufacturier, cesse bientôt. L’élévation de prix que la prime occasionne, dans le premier moment, doit attirer infailliblement dans l’industrie favorisée autant de capital de plus qu’il en faut pour répondre à la demande croissante des produits, et en même temps pour réduire les profits du manufacturier et du commerçant au taux commun. Mac Culloch.
↑ Le blé n’est pas une valeur invariable, parce que cette valeur est égale à la quantité de travail qu’elle peut entretenir. La valeur d’une denrée ou la faculté qu’elle a de s’échanger pour acheter du travail ou d’autres marchandises, est une qualité tout à fait différente de son utilité, c’est-à-dire de la propriété qu’elle a de satisfaire nos besoins et nos désirs. L’utilité, quoique élément essentiel de la valeur, n’est pas le principe qui la détermine ; elle dépend uniquement de la facilité ou de la difficulté de la production. Mac Culloch.
↑
Le prix des céréales varie, c’est-à-dire on donne pour la même quantité de blé Bras ou moins d’argent, par suite de variations survenues dans la valeur de l’argent qui sert au payement du prix, ou par suite d’un changement dans la valeur des céréales mêmes. Une augmentation dans le prix des céréales n’implique pas nécessairement une baisse dans la valeur de l’argent ; bien que le docteur Smith regarde un changement dans la valeur réelle des céréales comme absolument impossible. Mais en réfléchissant sur le cas même auquel se rapporte l’argumentation du docteur Smith, c’est-à-dire en admettant que, par suite d’une exportation volontaire ou forcée, le prix îles céréales éprouve une hausse, ne paraît-il pas évident que la hausse dans le prix de cette quantité de céréales, qui reste dans le pays, provient tout simplement d’une augmentation de valeur, par suite de la diminution de la provision ? En affirmant qu’aucune exportation ne pourrait augmenter la valeur des céréales, le docteur Smith prétend-il établir que la prime accordée à l’exportation, tout en diminuant la provision, ne saurait contribuer en même temps à produire une hausse dans la valeur réelle des céréales ? Peut-il nier que la valeur réelle des céréales, ainsi que celle de toutes les marchandises, ne soit augmentée par une diminution de quantité ?
L’assertion que la nature donne aux céréales une valeur inaltérable repose évidemment sur une erreur. Le docteur Smith confond ici l’utilité avec l’échange. Il est vrai qu’un boisseau de froment ne nourrit pas plus de personnes dans nos temps de disette que dans un temps d’abondance ; mais un boisseau de froment pourra être échangé contre une plus grande quantité d’objets de luxe ou de toute autre nature en temps de disette qu’en temps d’abondance ; et le propriétaire de terre qui aura de grandes provisions de grains, sera en définitive plus riche à une époque de disette qu’à une époque d’abondance. Il est donc impossible de soutenir que la prime, en favorisant l’exportation, ne produise pas en même temps une véritable hausse dans les prix. Buchanan.
↑ Ils paraissent au contraire avoir parfaitement compris leurs intérêts. Ils virent qu’en envoyant au dehors une partie de l’approvisionnement, ils obtiendraient de meilleurs prix pour la portion restante, et ils ne s’embarrassèrent guère des considérations raffinées dont M. Smith a embrouillé la question. Maintenant que la production intérieure n’est pas suffisante, et que, par conséquent, le pays est devenu dépendant des envois étrangers, aucune prime ne pourrait effectuer une exportation et produire une hausse dans les prix. Les propriétaires de terre, changeant de vues, ont donc imaginé d’interdire l’importation, comme ils avaient autrefois favorisé l’exportation. Dans tout ceci, ils ont prouvé qu’ils avaient assez l’intelligence de leurs propres affaires ; et on peut seulement regretter que, comme législateurs, ils ne se soient pas montrés assez soucieux du bien-être de la communauté, et qu’ayant été témoins de la misère des pauvres par suite du haut prix des céréales, ils aient persisté dans des mesures dont les effets devaient encore augmenter cette misère. On peut regretter que le désir d’augmenter leurs revenus ait prévalu sur toutes les considérations de justice et d’humanité. Buchanan.
↑ Mac Culloch admet l’inverse de ce paragraphe : • Une prime à l’exportation du blé, dit-il, en élève le prix, et, en forçant de cultiver des terres inférieures, elle élève la rente. (C’est la théorie de Ricardo, qui est l’article de foi fondamental de l’école du commentateur.) Elle produit donc un avantage réel et durable aux propriétaires ; tandis qu’une prime à l’exportation sur des marchandises manufacturées ne donné aux producteurs de ces denrées que des avantages insignifiants et temporaires. »
↑ Une prime accordée à la production ne réduirait pas le prix des céréales ; elle augmenterait seulement les revenus des propriétaires des terres.
Buchanan.
↑ Pour plus de détails sur les effets des primes accordées à la production, voir le chapitre de Ricardo sur ce sujet dans l’ouvrage intitulé : Principles of political economy and taxation. A. B.
↑ Ainsi nommées, parce qu’elles se payent à raison de tant par tonneau du port des bâtiments expédiés pour la pêche.
↑ Pour le distinguer du hareng soret, que les Anglais nomment hareng rouge. Le hareng blanc est notre harreng salé commun. C’est de celui-ci qu’il est question dans tout cet article.
↑ Le produit de la pêche de la baleine rapporte certainement les frais, ainsi que les revenus des capitaux engagés dans ce commerce ; et quand même il n’y aurait pas de prime, ce commerce n’en continuerait pas moins. Buchanan.
↑ Espèce de barque ou bâtiment ponté dont les Hollandais ont les premiers fait usage pour la pêche du hareng ; les buyses hollandaises sont du port de quarante-cinq à soixante tonneaux ; les écossaises, de vingt à vingt-huit.
↑ Cette première opération se fait dans le jour même de la pêche ; elle consiste à fendre le hareng, le vider de ses intestins, le laver dans l’eau fraîche, le saler et l’encaquer. Le baril ne contient alors que six à sept cents harengs ; le baril marchand en contient environ un millier.
↑ Sea-sticks
↑ Voyez les états annexés à la fin du volume.
↑ Sea-loches, du mot écossais lock, qui signifie lac. Voyez Dictionnaire de Johnson.
↑ D’importants changements ont été faits dans les règlements de la pêche du hareng, depuis la publication de la Richesse des Nations. Mac Culloch.
↑ Il y a peu de sujets qui aient donné naissance à une controverse plus vive que la législation des céréales, en France et en Angleterre. Cependant, nous n’avons pas cru devoir reproduire les notes dont les commentateurs anglais d’Adam Smith ont inondé ce chapitre de son ouvrage. La législation anglaise des céréales est une des formes de l’exploitation de la partie laborieuse de la population par la partie oisive. À quoi bon discuter avec les loups au profit des moulons ? — Quant à ce qui concerne la France, nous sommes heureusement loin du temps où Turgot, Necker, l’abbé Galiani, les économistes et leurs adversaires préludaient aux grandes luttes politiques de la Révolution par leurs curieuses discussions sur la liberté du commerce des grains. Tous ces livres sont aujourd’hui oubliés. Il ne reste que le souvenir des nobles efforts de Turgot, et la conviction que le meilleur préservatif de la disette est la plus grande somme de liberté compatible avec la juste rémunération du travail agricole. A. B.
↑ Malheureusement il est éprouvé, par de tristes expériences, que la disette du blé ne fait qu’augmenter, même artificiellement, la demande. A. B.
↑ L’auteur expose ici comment les choses devraient se passer logiquement, mathématiquement, pour ainsi dire ; mais la pratique ne répond pas toujours, ou mieux, elle ne répond presque jamais à cette théorie. A. B.
↑ Adam Smilh tombe ici dans l’optimisme exagéré que l’on est en droit de reprocher à l’un de ses commentateurs, M. Mac Culloch. C’est aller trop loin, que de prétendre que la cupidité du marchand de blé qui fait hausser le prix du blé au-dessus de son taux naturel, en prévision d’une disette, est une chose avantageuse a la masse du peuple. A. B.
↑ Il faut excepter les disettes qui ont précédé et accompagné la Révolution française. A ces deux époques, le fait d’accaparement existait avec toutes les circonstances les plus odieuses qu’on lui impute le plus souvent à tort. A. B.
↑ Cette méthode, si elle a été pratiquée du temps d’Adam Smith, ne l’est plus aujourd’hui, du moins comme méthode ordinaire. Mac Culloch.
↑ Cour formée de la réunion de tous les juges de paix de chaque comté : elle se tient tous les trois mois, et alternativement dans une des villes principales du comté.
↑
Ceci est une erreur. Le statut de 1772 (12 Geo. III, ch. lxxi) abroge les restrictions et pénalités imposées par les statuts plus anciens contre l’achat et la vente du blé et des autres produits naturels. Le préambule de l’acte reconnaît en ces termes les funestes effets de ces restrictions : « Comme il a été prouvé par l’expérience que les restrictions mises au commerce du blé, de la farine, de la viande, du bétail et autres sortes d’aliments, en s’opposant au libre commerce de ces denrées, ont pour effet d’en décourager la production et d’en hausser le prix ; lesquels statuts, s’ils étaient mis en vigueur, causeraient de grands maux aux habitants de la plus grande partie du royaume, et particulièrement aux habitants des cités de Londres et de Westminster, il est dorénavant résolu que, etc…»
Mais par malheur ce statut ne déclarait pas que personne ne pourrait plus être poursuivi devant les tribunaux pour les délits imaginaires d’accapareur, de regrattier, etc. Les auteurs de l’acte croyaient sans doute que le progrès des connaissances et l’esprit du siècle seraient une sécurité suffisante pour la liberté du commerce. Ils se trompaient. En 1795 et 1800, le prix du blé s’éleva à un taux excessif ; et malgré les raisonnements concluants du docteur Smith et d’autres écrivains compétents, malgré la déclaration si explicite du préambule de l’acte cité plus haut, les clameurs contre les marchands de blé furent aussi fortes qu’elles auraient pu l’être dans le siècle des Édouard et des Henri. Les autorités municipales de Londres dénoncèrent les spéculations des marchands de blé… L’un d’eux, nommé Rutley, fat accusé, en 1800, du délit de regrattier, c’est-à-dire pour avoir vendu sur le même marché, le même jour qu’il les avait achetés, trente ’’quarters’’ d’avoine, à la surenchère de deux schellings le quarter Mac Culloch.
↑ Avant le statut de la treizième année du roi actuel, les droits à payer sur l’impor­ta­tion des différentes sortes de grains étaient établis comme il suit :
Grains	Droits	Droits	Droits
Haricots à 28s.	le quart, 19s. 10d. par quart. Ensuite jusqu’à 40s.	16s. 8d.	au-delà 12d.
Orge à 28s.	19s. 10d.	32s. 16s.	12d.
L’importation de la drèche est prohibée par le bill de la taxe annuelle sur la drèche.
Avoine à 16s.	le quart 5s. 10d. et au delà de ce prix,	 	9 1/2d.
Pois à 40s.	16s. 10d. et au delà de ce prix,	 	9 3/4d.
Seigle à 36s.	19s. 10d. jusqu’à 40s.	16s. 8d.	au delà, 12d.
Blé from. à 44s.	21s. 9d. till 53s. 4d.	17s.	au delà, 8s. jusqu’à 4 l. et au delà de ce dernier prix, environ 1s. 4d.
Blé saran. à 32s.	paye 16s. par quarter

Ces différents droits ont été établis en partie par le statut de la vingt-deuxième année de Charles II, à la place de l’ancien subside, et en partie par le nouveau subside, par les tiers et deux tiers de subside, et par le subside de 1747.
↑ Toutes les marchandises sujettes au droit de douane appelé poundage, ou de tant par livre de leur valeur, sont évaluées dans un livre de tarifs pour prévenir l’arbitraire et les contestations dans la perception du droit.
↑ To ingros, acheter des denrées en grandes quantités et en faire des magasins.

%%%%%%%%%%%%%%%%%%%%%%%%%%%%%%%%%%%%%%%%%%%%%%%%%%%%%%%%%%%%%%%%%%%%%%%%%%%%%%%%
%                                  Chapitre 6                                  %
%%%%%%%%%%%%%%%%%%%%%%%%%%%%%%%%%%%%%%%%%%%%%%%%%%%%%%%%%%%%%%%%%%%%%%%%%%%%%%%%

\chapter{Des traités de commerce}
\markboth{Des traités de commerce}{}

Quand une nation s’oblige, par un traité, à permettre chez elle l’entrée de certaines marchandises d’un pays étranger, tandis qu’elle les prohibe venant de tous les autres pays, ou bien à exempter les marchandises d’un pays de droits auxquels elle assujettit celles de tous les autres, le pays ou du moins les marchands et les manufacturiers du pays dont le commerce est ainsi favorisé doivent tirer de grands avantages de ce traité. Ces marchands et manufacturiers jouissent d’une sorte de monopole dans le pays qui les traite avec tant de faveur. Ce pays devient un marché à la fois plus éten­du et plus avantageux pour leurs marchandises ; plus étendu, parce que les marchan­dises des autres nations étant exclues ou assujetties à des droits plus lourds, il absorbe une plus grande quantité de celles qu’ils y portent ; plus avantageux, parce que les marchands du pays favorisé, jouissant dans ce marché d’une espèce de monopole, y vendront souvent leurs marchandises à un prix plus élevé que s’ils étaient exposés à la libre concurrence des autres nations.
Si cependant ces traités peuvent être avantageux aux marchands et aux manu­facturiers du pays favorisé, ils sont nécessairement désavantageux aux habitants du pays qui accorde cette faveur[1]. C’est un monopole qui se trouve ainsi accordé contre eux à une nation étrangère, et il leur faut souvent acheter les marchandises étrangères dont ils ont besoin, plus cher que si la libre concurrence des autres nations était admise. Par conséquent, la partie de son produit avec laquelle cette nation achète des mar­chandises étrangères se trouve vendue à un moindre prix, attendu que, lorsque deux choses s’échangent l’une contre l’autre, le bon marché de l’une est une conséquence néces­saire, ou plutôt est la même chose que la cherté de l’autre. La valeur échan­gea­ble de son produit annuel peut donc éprouver une diminution à chaque traité de cette espèce. Cette diminution cependant ne peut guère aller jusqu’à une perte positive, et elle ne fait qu’affaiblir le gain que cette nation eût pu faire sans cela. Quoiqu’elle vende ses denrées à meilleur marché qu’elle ne les eût vendues sans cette circons­tance, néanmoins elle ne les vendra pas probablement moins qu’elles ne lui coûtent ; elle ne les vendra pas, comme dans le cas des primes, à un prix qui ne saurait rem­placer le capital employé pour les mettre au marché, y compris le profit ordinaire des capitaux. S’il en était autrement, le commerce ne pourrait se soutenir longtemps. Ainsi, la nation qui accorde cette faveur à une autre peut encore gagner à ce com­merce, quoiqu’elle gagne moins que s’il y avait liberté de concurrence.
Cependant il y a des traités de commerce qu’on a supposés avantageux, en partant de principes très-différents de ceux-ci. Un pays commerçant a quelquefois accordé contre lui-même un monopole de ce genre à certaines marchandises d’une nation étrangère, dans l’espérance que, dans la totalité des opérations de commerce qui s’éta­bliraient entre lui et cette nation, il lui vendrait annuellement plus qu’il n’achèterait d’elle, et que dès lors il aurait à recevoir d’elle annuellement une balance en or et en argent.
C’est d’après ce principe que l’on a tant vanté le traité de commerce conclu en 1703 par M. Methuen, entre l’Angleterre et le Portugal. Ce traité ne consiste qu’en trois articles, dont voici la traduction littérale :
« Art. 1er. Sa Majesté le roi de Portugal, tant pour elle que pour les rois ses successeurs, promet de laisser entrer dorénavant et à toujours, en Portugal, les draps et autres ouvrages en laine, de fabrique anglaise, ainsi qu’ils entraient par le passé, avant qu’ils eussent été prohibés par la loi, et ce néanmoins sous la condition suivante :
« Art. 2. C’est que Sa Majesté le roi de la Grande-Bretagne s’oblige, tant pour elle que pour ses successeurs rois, de laisser entrer dorénavant et à toujours, dans la Grande-Bretagne, les vins du cru du Portugal ; de manière que, dans aucun temps, soit qu’il y ait paix ou guerre entre les royaumes de la Grande-Bretagne et de France, il ne pourra être exigé pour ces vins, sous le nom de douane ou de droits, ou à quelque autre titre que ce puisse être, directement ni indirectement, soit qu’ils soient importés en Grande-Bretagne en pipes ou tonneaux, ou en tout autre vase, aucune autre chose de plus que ce qui sera exigé pour pareille quantité ou mesure de vin de France, en déduisant encore ou retranchant un tiers du droit ou entrée. Mais si une fois cette déduction ou soustraction de droits d’entrée, qui doit être faite, comme il est dit ci-dessus, venait à éprouver quelque difficulté ou préjudice en façon quelconque, il sera juste et légitime, pour Sa Majesté le roi de Portugal, de renouveler la prohibition des draps et autres ouvrages en laine, de fabrique anglaise.
« Art. 3. Leurs Excellences les seigneurs plénipotentiaires promettent et garantissent, en leurs noms, que leurs maîtres ci-dessus nommés ratifieront le présent traité, et que les ratifications seront échangées dans le délai de deux mois ».
Par ce traité, la couronne de Portugal s’oblige à laisser entrer les lainages de fabri­que anglaise, sur le même pied qu’elles entraient avant la prohibition, c’est-à-dire de ne pas hausser les droits qui avaient coutume d’être payés avant cette époque. Mais elle ne s’oblige pas à les laisser entrer à de meilleures conditions que les lainages de quelque autre nation, de France ou de Hollande, par exemple. Au contraire, la couronne de la Grande-Bretagne s’oblige à laisser entrer les vins de Portugal pour les deux tiers seulement du droit d’entrée payé pour ceux de France, les vins les plus capables de leur faire concurrence. Jusque-là donc ce traité est évidemment à l’avan­tage du Portugal et au désavantage de la Grande-Bretagne[2].
Il a cependant été vanté comme un chef-d’œuvre de la politique anglaise. Le Portugal reçoit annuellement du Brésil une plus grande quantité d’or que ce qu’il peut en employer dans son commerce intérieur, sous forme de monnaie ou d’orfèvrerie. Le surplus est d’une trop grande valeur pour qu’on le laisse inactivement reposer dans des coffres ; et comme il ne peut trouver dans le pays de marché avantageux, il faut bien, en dépit de toutes les prohibitions, qu’il soit envoyé au-dehors, et échangé pour quelque chose qui trouve dans le pays un débit plus profitable. Une grande portion en vient annuellement à l’Angleterre, soit en retour de marchandises anglaises, soit pour des marchandises d’autres nations européennes, qui reçoivent leurs retours par l’Angleterre. Il a été affirmé à M. Baretti que le paquebot qui arrive chaque semaine apportait, de Lisbonne en Angleterre, une semaine dans l’autre, plus de 50 mille liv. sterling en or. La somme a été probablement exagérée. Elle s’élèverait ainsi à plus de 2 millions 600 mille liv. sterling par an, ce qui est plus que ce que le Brésil n’est réputé fournir. 
Nos marchands étaient, il y a quelques années, mécontents de la couronne de Portugal. On avait enfreint ou révoqué quelques privilèges qui leur avaient été accor­dés, non par traité, mais par pure grâce, à la sollicitation, il est vrai, selon toute apparence, de la couronne de la Grande-Bretagne, et en retour de quelques services de protection et de défense beaucoup plus importants. Ainsi les gens les plus intéressés, pour l’ordinaire, à exalter le commerce du Portugal, étaient alors disposés à le représenter plutôt comme moins avantageux qu’on ne se le figure communément. La majeure partie, disaient-ils, la presque totalité de cette importation d’or annuelle n’était pas pour le compte de la Grande-Bretagne, mais pour celui d’autres nations de l’Europe, les fruits et les vins de Portugal annuellement importés dans la Grande-Bretagne balançant, à peu de chose près, la valeur des marchandises anglaises qu’on y envoyait.
Supposons néanmoins que la totalité soit pour le compte de la Grande-Bretagne, et que l’exportation aille à une somme encore beaucoup plus forte que M. Baretti ne paraît le supposer, ce commerce n’en serait pas pour cela plus avantageux que tout autre dans lequel, pour les exportations de même valeur, nous recevrions en retour une valeur égale de choses consommables.
Il est à présumer qu’il n’y a qu’une très-petite partie de cette importation qui soit employée annuellement comme addition à notre monnaie ou à notre orfèvrerie. Le reste doit nécessairement être renvoyé au-dehors et échangé contre des choses de con­sommation d’une espèce ou d’une autre. Or, si ces choses de consommation étaient achetées directement avec le produit de l’industrie anglaise, ce serait une opération plus avantageuse pour l’Angleterre que de commencer par acheter d’abord avec ce produit l’or du Portugal, pour ensuite, avec cet or, acheter ces mêmes choses de consommation. Un commerce étranger de consommation, par voie directe, est toujours plus avantageux que celui fait par voie détournée, et il faut un bien moindre capital dans le premier cas que dans l’autre, pour rapporter au marché national la même valeur en marchandises étrangères. Par conséquent, il eût été bien plus à l’avantage de l’Angleterre qu’une moindre portion de son industrie eût été employée à produire des marchandises destinées au marché de Portugal, et qu’une plus grande portion en eût été mise à produire les marchandises destinées à ces autres marchés, d’où l’on peut tirer des choses de consommation demandées dans la Grande-Bretagne. De cette manière elle emploierait un bien moindre capital qu’à présent pour se pro­curer à la fois et l’or dont elle a besoin pour son propre usage, et ces mêmes cho­ses de consommation. Il y aurait donc un capital épargné, qu’on pourrait employer à d’autres objets, à mettre en activité un surcroît d’industrie, et à faire naître un plus grand produit annuel.
Quand la Grande-Bretagne serait totalement exclue du commerce de Portugal, elle trouverait très-peu de difficulté à se procurer annuellement toute la provision d’or qui lui est nécessaire, soit pour l’orfèvrerie, soit pour la monnaie, soit pour le commerce étranger. On a de l’or, comme toute autre marchandise, pour sa valeur, pourvu qu’on ait cette valeur à en donner. D’ailleurs, le superflu annuel d’or du Portugal serait toujours envoyé au-dehors, et s’il n’était pas exporté par la Grande-Bretagne, il le serait par quelque autre nation qui serait bien aise de trouver à le revendre pour son prix, tout comme le fait à présent la Grande-Bretagne. Il est vrai qu’en achetant l’or du Portugal, nous l’achetons de la première main, tandis qu’en l’achetant de toute autre nation, si ce n’est de l’Espagne, nous l’achèterions de la seconde main, et nous pour­rions le payer un peu plus cher. Toutefois, cette différence serait sûrement trop peu de chose pour mériter l’attention du gouvernement.
Presque tout notre or, dit-on, vient de Portugal. Avec les autres nations, la balance du commerce, ou est contre nous, ou est de peu de chose en notre faveur. Mais il ne faudrait pas perdre de vue que plus nous importons d’or d’un pays, moins nous devons nécessairement en importer de tous les autres. La demande effective de l’or, comme celle de toute autre marchandise, est, dans tout pays, limitée à une certaine quantité. Si de cette quantité neuf dixièmes sont importés d’un pays, il ne restera qu’un dixième à importer de tous les autres. D’ailleurs, plus nous importerons annuellement, de quel­ques pays, en particulier, de l’or au-delà de ce qu’il nous en faut pour la monnaie et pour l’orfèvrerie, plus nécessairement il faudra que nous en exportions dans d’autres pays ; et plus la balance du commerce, l’objet le plus chimérique de la politique mo­der­ne, paraît nous être favorable avec certaines contrées, plus alors elle doit néces­sairement paraître contre nous avec la plupart des autres.
Ce fut toutefois cette idée ridicule que l’Angleterre ne saurait subsister sans le commerce du Portugal, qui, vers la fin de la guerre dernière, engagea la France et l’Espagne à exiger du roi de Portugal, sans le moindre prétexte d’offense ou de provocation de sa part, qu’il fermât ses ports à tous les vaisseaux de la Grande-Bretagne, et que, pour assurance de cette exclusion, il y reçût des garnisons françaises ou espagnoles. Si le roi de Portugal se fût soumis à ces conditions ignominieuses que lui proposait son beau-frère le roi d’Espagne, l’Angleterre aurait été affranchie d’un inconvénient beaucoup plus fâcheux que la perte du commerce de Portugal : la charge de soutenir un allié extrêmement faible et si mal pourvu de tout pour sa propre défense, que toute la puissance de l’Angleterre, quand même elle aurait été dirigée vers ce seul objet, aurait pu suffire à peine à le défendre encore pendant une campa­gne. La perte du commerce de Portugal aurait, sans contredit, causé un embarras con­si­dérable aux marchands qui auraient été à cette époque engagés dans ce commerce, et qui, pendant un an ou deux peut-être, n’auraient pas pu trouver d’emploi aussi avantageux pour leurs capitaux ; et c’est vraisemblablement en cela seulement qu’au­rait consisté tout le dommage que l’Angleterre aurait eu à souffrir de ce trait remar­quable de politique mercantile.
La grande importation annuelle d’or et d’argent n’est pas destinée aux besoins de l’orfèvrerie ni à ceux des monnaies, mais à ceux du commerce étranger. Un com­merce étranger de consommation par circuit se fait plus avantageusement avec ces métaux qu’avec presque toute autre marchandise. Comme ils sont les instruments universels du commerce, ils sont reçus en retour de toutes marchandises quelconques, plus promptement qu’aucune autre denrée ; et au moyen de la petitesse de leur volume par rapport à leur valeur, ils coûtent moins que presque toute autre espèce de mar­chandise à être transportés et retransportés d’une place à l’autre, et ils perdent moins de leur valeur dans tous ces transports. Ainsi, de toutes les marchandises qu’on achète dans un pays étranger, sans autre objet que de les vendre et de les échanger contre d’autres marchandises dans un autre pays étranger, il n’y en a aucune d’aussi com­mode que l’or et l’argent. L’avantage principal de notre commerce de Portugal, c’est de faciliter tous les différents commerces étrangers de consommation par circuit, qui se font dans la Grande-Bretagne ; et quoique ce ne soit pas là un avantage capital, néanmoins c’en est un considérable.
Il paraît assez évident de soi-même que toute augmentation annuelle qu’on peut raisonnablement supposer dans les ouvrages orfévrerie ou dans ceux des monnaies du royaume, n’exige qu’une très-petite importation annuelle d’or et d’argent ; et quand nous n’aurions pas de commerce direct avec le Portugal, nous pourrions toujours fort aisément nous procurer, dans un endroit ou dans l’autre, cette petite quantité de métal.
Quoique le commerce d’orfèvrerie soit un article très-considérable dans la Grande-Bretagne, la majeure partie des nouveaux ouvrages vendus annuellement est faite avec d’ancienne orfèvrerie fondue, de sorte que l’addition annuelle à la totalité de l’orfèvrerie du royaume ne peut être très-grande, et ne peut exiger qu’une très-faible importation annuelle.
Il en est de même pour les monnaies. Personne n’imagine, je pense, que même la plus grande partie du monnayage actuel, qui, pendant dix années de suite, avant la dernière refonte de la monnaie d’or, s’est élevé à plus de 800,000 livres en or par an, ait été une addition annuelle à la masse de monnaie circulant auparavant dans le royaume. Dans un pays où la dépense du monnayage est défrayée par le gouver­ne­ment, la valeur de la monnaie, même quand elle contient parfaitement son poids légal d’or ou d’argent, ne peut jamais être beaucoup plus grande que celle d’une pareille quantité de ces métaux non monnayés, parce qu’il ne faut que la peine d’aller à la Monnaie, et d’attendre peut-être quelques semaines pour se procurer à la place d’une quantité d’or et d’argent non monnayé, une pareille quantité de ces métaux monnayés. Mais dans tout pays, la plus grande partie de la monnaie courante est presque toujours plus ou moins usée ou dégradée de manière ou d’autre au-dessous de son poids légal ou primitif. Elle l’était dans la Grande-Bretagne, avant la refonte, à un point consi­dérable, l’or étant de plus de 2 pour 100 au-dessous de son poids légal, et l’argent de plus de 8 pour 100. Or, si 44 guinées et demie contenant parfaitement leur poids légal, une livre d’or, ne peuvent acheter que très-peu au-delà d’une livre pesant d’or non monnayé, 44 guinées et demie manquant d’une partie de leur poids ne pouvaient pas acheter une livre d’or, et il fallait ajouter quelque chose pour com­penser le déficit ; par conséquent, le prix courant du lingot d’or au marché, au lieu d’être le même que le prix auquel il était reçu à la Monnaie, c’est-à-dire de 46 livres 14 schellings 6 deniers la livre pesant, était alors d’environ 47 livres 14 schellings, et quelquefois d’environ 48 livres. Cependant, quand la plus grande partie de la monnaie était dans cet état de dégradation, 44 guinées et demie toutes neuves, sortant du balancier, n’auraient pas acheté au marché plus de marchandises que les autres guinées courantes ordinaires, parce que ces guinées neuves, une fois entrées dans la caisse du marchand et confon­dues avec d’autres pièces de monnaie, ne pouvaient plus désormais en être distin­guées, sans qu’il en coûtât pour cela plus de peine que la différence n’aurait valu. Tout comme d’autres guinées, elles ne valaient pas plus de 46 livres 14 schellings 6 deniers la livre pesant : néanmoins, jetées dans le creuset, elles produisaient, sans aucun déchet sensible, une livre pesant d’or au titre, qu’on pouvait vendre en tout temps pour une somme d’environ 47 livres 14 schellings, ou 48 livres en or ou en argent, somme tout aussi bonne pour remplir toutes les fonctions de monnaie que la somme qu’on avait fondue. Il y avait donc un profit évident à fondre la monnaie nouvellement frap­pée, et cela se faisait si promptement, qu’il n’y avait pas de précautions du gouverne­ment capables de l’empêcher. Les opérations de l’hôtel des Monnaies étaient à cet égard à peu près comme la toile de Pénélope ; l’ouvrage fait dans le jour était défait pendant la nuit. L’hôtel des Monnaies était occupé bien moins à faire des additions journalières à la quantité des espèces courantes, qu’à en remplacer sans cesse la partie la meilleure qui était fondue journellement.
Si les particuliers qui portent leur or et leur argent à la Monnaie étaient tenus d’en payer le monnayage, alors il ajouterait à la valeur de ces métaux, tout comme la façon ajoute à celle des ouvrages d’orfèvrerie. L’or et l’argent monnayés auraient plus de valeur que non monnayés. Un droit de seigneuriage qui ne serait pas exorbitant ajou­terait au métal toute la valeur du droit, parce que le gouvernement ayant partout le privilège exclusif de battre monnaie, aucune monnaie ne pourrait se présenter dans le commerce à meilleur marché que le gouvernement ne jugerait à propos de la fournir. À la vérité, si le droit était exorbitant, c’est-à-dire s’il était fort au-dessus de la valeur réelle du travail et des dépenses nécessaires du monnayage, alors les faux-mon­nayeurs, tant au-dedans qu’au-dehors du pays, se trouveraient encouragés, par la grande différence de prix entre le lingot et le métal monnayé, à verser dans le pays une assez grande quantité de monnaie contrefaite, pour pouvoir rabaisser la valeur de la monnaie du gouvernement. Cependant, quoiqu’en France le droit de seigneuriage soit de 8 pour 100, on n’a jamais vu qu’il en fût résulté d’inconvénient sensible de ce genre. Les dangers auxquels est partout exposé un faux-monnayeur s’il demeure dans le pays dont il contrefait la monnaie, et ceux auxquels sont exposés ses agents ou correspondants s’il demeure dans un pays étranger, sont de beaucoup trop grands pour qu’on se décide à les courir, pour l’appât d’un profit de 6 à 7 pour 100.
Le seigneuriage, en France, élève la valeur de la monnaie au-dessus de la pro­portion de la quantité d’or pur qu’elle contient. Ainsi, par l’édit de janvier 1726[3], le prix de l’or fin à 24 carats fut fixé à la Monnaie à 740 livres 9 schellings 1 denier 1/11 tournois le marc de huit onces. La monnaie d’or de France, en tenant compte de ce qu’on passe pour remède d’alloy, contient 21 carats et trois quarts de carat d’or pur, et 2 carats un quart de carat d’alliage. Par conséquent, le marc d’or au titre ne vaut pas plus d’environ 671 livres 10 deniers. Or, en France, ce marc d’or au titre est taillé en 30 louis d’or de 24 livres tournois chacun, ou en 720 livres tournois. Le monnayage augmente donc la valeur d’un marc d’or au titre, de toute la différence qu’il y a entre 671 livres 10 deniers et 720 livres, c’est-à-dire de 48 livres 19 schellings 2 deniers tournois.
Le profit de fondre la monnaie neuve sera, dans la plupart des circonstances, tota­le­ment anéanti, et dans toutes il sera diminué, au moyen d’un droit de seigneuriage. Ce profit procède toujours de la différence entre la quantité de métal que devrait contenir la monnaie courante, et ce qu’elle en contient réellement pour le moment. Si cette différence est moindre que le seigneuriage, il y aura perte au lieu de profit. Si elle est égale au droit de seigneuriage, il n’y aura ni profit ru perte. Si elle est plus grande que le montant du seigneuriage, il y aura, à la vérité, quelque profit, mais moin­dre que s’il n’y eût pas eu de seigneuriage. Si, avant la dernière refonte de notre monnaie d’or, par exemple, il y avait eu sur le monnayage un droit de seigneuriage de 5 pour 100, il y aurait eu une perte de 3 pour 100 à fondre la monnaie d’or. Si le seigneuriage eût été de 2 pour 100, il n’y aurait eu ni profit ni perte. Si le seigneuriage eût été de 1 pour 100, il y aurait eu un profit, mais de 1 pour 100 seulement, au lieu de 2. Ainsi, partout où la monnaie est reçue au compte et non au poids, un droit de seigneuriage est le préservatif le plus efficace pour empêcher que la monnaie ne soit fondue et, par la même raison, qu’elle ne soit exportée. Ce sont ordinairement les pièces les meilleures et les plus pesantes qui sont fondues ou exportées, parce que c’est sur celles-là qu’il y a plus de profit à faire.
La loi pour l’encouragement de la fabrication des monnaies, c’est-à-dire celle qui a affranchi de tous droits cette fabrication, fut d’abord portée sous le règne de Charles II, pour un temps limité, et ensuite, par différentes prorogations, elle fut continuée jusqu’en 1769, époque à laquelle elle fut rendue perpétuelle. La banque d’Angleterre est souvent obligée, pour remplir ses coffres, de porter des lingots à la Monnaie, et vraisemblablement elle s’est imaginé qu’il était plus avantageux pour elle que la fabrication se fit aux frais du gouvernement qu’aux siens. Il est probable que c’est par complaisance pour cette grande compagnie que le gouvernement a consenti à rendre cette loi perpétuelle. Cependant si la coutume de peser l’or venait à se perdre, comme il est à croire qu’elle se perdra à cause de son incommodité ; si la monnaie d’Angle­terre venait à être reçue au compte, comme elle l’était avant la dernière refonte de la monnaie, cette grande compagnie pourrait peut-être trouver que, dans cette occasion, comme en beaucoup d’autres, elle ne s’est pas peu trompée sur ses vrais intérêts.
Avant la dernière refonte, quand la monnaie d’or courante d’Angleterre était de 2 pour 100 au-dessous de son poids légal, comme il n’y avait pas de seigneuriage, elle était de 2 pour 100 au-dessous de la valeur de la quantité de métal au titre qu’elle aurait dû contenir. Ainsi, quand cette grande compagnie achetait du lingot d’or pour le faire monnayer, elle était obligée de le payer 2 pour 100 de plus qu’il ne valait après le monnayage. Mais s’il y avait eu un droit de seigneuriage de 2 pour 100 sur la fabrication, alors la monnaie d’or courante, quoique de 2 pour 100 au-dessous de son poids légal, aurait néanmoins été d’une égale valeur à la quantité de métal au titre qu’elle eût dû contenir ; la valeur de la façon compensant, dans ce cas, la diminution du poids. À la vérité, la banque aurait eu à payer le droit de seigneuriage, lequel étant de 2 pour 100, la perte de la compagnie, sur la totalité de l’opération, aurait été de 2 pour 100, précisément la même qu’elle a été dans le fait, mais elle n’aurait pas été plus grande.
Si le seigneuriage eût été de 5 pour 100, et la monnaie d’or courante de 2 pour 100 seulement au-dessous de son poids de fabrication, dans ce cas la banque aurait gagné 3 pour 100 sur le prix du lingot ; mais, comme elle aurait eu un seigneuriage de 5 pour 100 à payer sur la fabrication, sa perte sur la totalité de l’opération aurait été tout de même précisément de 2 pour 100.
Si la seigneuriage n’eût été que de 1 pour 100, et la monnaie d’or courante de 2 pour 100 au-dessous de son poids légal, dans ce cas la banque n’aurait perdu que 1 pour 100 sur le prix du lingot ; mais, comme elle aurait eu de plus à payer un seigneu­riage de 1 pour 100, sa perte sur la totalité de l’opération aurait été précisément de 2 pour 100, de même que dans tous les autres cas.
S’il y avait un droit modéré de seigneuriage, tandis qu’en même temps la monnaie courante contiendrait pleinement son poids de fabrication, comme elle l’a contenu, à très-peu de chose près, depuis la dernière refonte, alors tout ce que la banque pourrait perdre par le seigneuriage, elle le regagnerait sur le prix du lingot, et tout ce qu’elle pourrait gagner sur le prix du lingot, elle le reperdrait par le seigneuriage. Ainsi, elle ne gagnerait ni ne perdrait sur la totalité de l’opération et, comme dans toutes les hypothèses précédentes, elle se trouverait précisément dans la même situation que s’il n’y eût pas eu de seigneuriage.
Quand l’impôt sur une marchandise est assez modéré pour ne pas encourager la contrebande, le marchand qui commerce sur cette marchandise avance bien l’impôt ; mais, à proprement parler, il ne le paye point, puisqu’il le retire sur le prix de la mar­chan­dise. L’impôt est payé, en fin de compte, par le dernier acheteur ou consom­mateur. Or, l’argent est une marchandise à l’égard de laquelle tout homme est mar­chand ; personne ne l’achète que dans le dessein de la revendre, et pour l’argent, dans les cas ordinaires, il n’y a point de dernier acheteur ou de consommateur. Ainsi, quand l’impôt sur la fabrication de la monnaie est assez modéré pour ne pas encou­rager le faux monnayage, quoique chacun avance l’impôt, personne ne le paye en définitive, parce que chacun le retire dans le surcroît de valeur que ce droit ajoute à la monnaie.
Par conséquent, un droit modéré de seigneuriage n’augmenterait, dans aucun cas, la dépense de la banque ou de tout autre particulier qui porterait un lingot à la Monnaie pour y être monnayé ; et l’exemption de ce droit modéré de seigneuriage n’apporte pas, dans aucun cas, la moindre diminution de dépense. Qu’il y ait ou qu’il n’y ait pas de seigneuriage, si la monnaie courante contient pleinement son poids de fabrication, le monnayage ne coûte rien à personne ; et si elle se trouve être au-des­sous de ce poids, le monnayage doit toujours nécessairement coûter de même la différence entre la quantité de métal qu’elle devrait contenir, et la quantité qu’elle en contient réellement pour le moment.
Ainsi, quand le gouvernement défraye la dépense du monnayage, non-seulement il se charge d’une petite dépense, mais encore il perd un petit revenu que pourrait lui fournir un droit convenablement fixé, et cet acte de générosité nationale ne profite pas le moins du monde à la banque ni à aucun autre particulier.
Mais les directeurs de la banque ne seraient probablement pas très-disposés à consentir à l’imposition d’un droit de seigneuriage, sur la foi d’une spéculation qui ne leur promet pas un gain positif, mais qui prétend seulement leur garantir qu’ils n’en essuieront aucune perte. Dans l’état actuel de la monnaie d’or, et tant qu’elle conti­nuera à être reçue au poids, à coup sûr ils ne gagneraient rien à un pareil changement. Mais si la coutume de peser la monnaie d’or venait jamais à passer d’usage, comme il est à présumer que cela arrivera, et si la monnaie d’or venait jamais à tomber dans le même état de dégradation où elle était avant la dernière refonte, le gain de la banque, ou pour mieux dire l’épargne que lui vaudrait l’imposition d’un seigneuriage, serait alors probablement très-considérable. La banque d’Angleterre est la seule compagnie qui envoie des lingots à la Monnaie pour une valeur importante, et la charge du mon­nayage annuel tombe entièrement ou presque entièrement sur elle. Si ce monnayage annuel n’avait autre chose à faire qu’à réparer les pertes inévitables de la monnaie et le déchet qui provient nécessairement du frai, il ne pourrait guère excéder 50000 livres, ou au plus 100,000 livres. Mais, quand la monnaie est dégradée au-dessous de son poids de fabrication, il faut qu’en outre le monnayage annuel remplisse les vides énormes que font continuellement dans la monnaie courante les opérations du creuset et de l’exportation. C’est pour cette raison que, pendant les dix ou douze années qui ont précédé immédiatement la dernière refonte de la monnaie d’or, le monnayage annuel s’élevait, année moyenne, à plus de 850 000 livres. Mais s’il y eût un droit de seigneuriage de 4 ou 5 pour 100 sur la monnaie d’or, il aurait vraisemblablement, même dans l’état où était alors la monnaie, arrêté d’une manière efficace toute l’activité du creuset et celle de l’exportation. La banque, au lieu de perdre chaque année environ 2 et demi pour 100 sur les lingots qu’elle faisait monnayer jusqu’à concurrence de plus de 850,000 livres, ou d’essuyer une perte annuelle de plus de 21,250 livres, n’aurait pas eu vraisemblablement le dixième de cette perte à supporter.
La somme annuelle accordée par le parlement pour subvenir aux dépenses du monnayage n’est que de 14,000 livres, et la dépense réelle qu’il coûte au gouverne­ment, ou les appointements des employés à la Monnaie, ne s’élèvent pas dans les circonstances ordinaires, à ce qui m’a été assuré, à plus de moitié de cette somme. L’épargne d’une aussi faible dépense, ou même encore le gain d’une autre somme qui ne serait pas beaucoup plus forte, sont des objets qu’on peut croire de trop peu d’importance pour mériter une attention sérieuse de la part du gouvernement. Mais une économie de 18 ou 20,000 livres par an, dans le cas d’un événement qui n’est pas invraisemblable, qui est déjà fréquemment arrivé et qui menace d’arriver encore, est certainement un objet bien digne d’une sérieuse attention, même pour une aussi grande compagnie que la banque d’Angleterre.
Quelques-unes des réflexions et observations précédentes auraient peut-être été plus convenablement placées dans les chapitres du livre 1er, qui traitent de l’origine et de l’usage de la monnaie, et de la différence entre le prix réel des marchandises et leur prix nominal ; mais, comme la loi pour l’encouragement du monnayage prend sa source dans ces préjugés vulgaires nés du système mercantile, j’ai cru plus à propos de les réserver pour ce chapitre. Rien ne pouvait être plus conforme à l’esprit de ce système qu’une espèce de prime donnée à la fabrication de l’argent, la chose même qui, dans son hypothèse, constitue la richesse d’une nation ; aussi est-ce un des mille expédients merveilleux qu’il met en œuvre pour enrichir le pays.
 
 
 
↑
À très-peu d’exceptions près, les traités de commerce conclus jusqu’à présent n’ont jamais eu pour base le principe de l’équité et de la réciprocité. La plupart des négociations commerciales n’ont été entreprises que parce que chacune des parties contractantes avait cru pouvoir remporter un avantage au préjudice de l’autre. Il est superflu d’ajouter que ces prétendus avantages étaient seulement imaginaires, et que quelquefois ils devenaient même préjudiciables aux intérêts auxquels ils semblaient devoir profiter. Quand un pays obtient, par un traité ou d’une autre manière, le privilège exclusif de pourvoir le marché d’un autre pays de certains produits qu’auparavant il n’avait pas l’habitude de lui fournir, il sera forcément obligé, pour rendre ce commerce possible, d’accorder sur son propre marché un monopole semblable. Ainsi, dans le fameux traité de commerce conclu en 1703 par M. Methuen, entre la Grande-Bretagne et le Portugal, le privilège exclusif de pourvoir le marché portugais de toute espèce d’étoffes de laine fut accordé à l’Angleterre ; mais les Portugais n’auraient pas pu donner suite à cette stipulation, et ils n’auraient pas eu de valeurs à nous donner en échange de nos laines, si nous ne leur avions pas accordé le monopole de leurs vins sur le marché anglais. Ce traité avait par conséquent un double inconvénient. Il était préjudiciable aux Portugais, en imposant à leur marché des restrictions pour les laines, et en attirant une très-grande partie de leurs capitaux dans la production des vins ; et il lésait également les intérêts britanniques, en obligeant notre gouvernement à frapper de droits différentiels très-considérables les vins français et autres. Il nous força par conséquent, d’un côté, à payer une boisson, relativement inférieure, à un prix très-élevé ; tandis que, d’un autre côté, il amena les Français et les Espagnols à user de représailles en excluant de leurs marchés plusieurs de nos articles les plus importants.
Le traité de commerce conclu entre la France et l’Angleterre, en 1786, fut un des rares et mémorables exemples qu’offrirent alors ces deux grandes nations, en consentant à établir entre elles des relations commerciales sur un pied de parfaite égalité, et sans stipulation d’avantages particuliers. Il est incontestable qu’en agissant de cette manière elles avaient une intelligence parfaite de leurs véritables intérêts.
La Grande-Bretagne et la France sont près l’une de l’autre ; chacun de ces deux pays possède ce dont l’autre manque. Si celle-ci abonde en produits de tout genre que lui fournit un sol fertile et un climat heureux, celle-là est riche des produits de ses manufactures supérieures et de son industrie commerciale. S’il n’y avait point de restrictions mises sur le commerce entre ces deux pays, ils formeraient mutuellement d’excellents marchés pour leurs produits respectifs. Mais leurs relations commerciales ont été tellement paralysées par leurs jalousies mutuelles, qu’aujourd’hui le commerce avec la Chine se trouve être plus important et avantageux que celui que nous faisons avec nos plus proches et plus riches voisins. L’objet du traité dont il est ici question fut d’établir un système plus amical, de modérer les rigueurs commerciales ; et, en mettant en évidence l’extrême avantage de ces relations, d’enseigner aux deux parties à oublier les anciennes haines, et à s’intéresser à leur bien-être commun.
Il n’est pas nécessaire d’indiquer dans leurs détails les stipulations de ce traité. U guerre qui éclata malheureusement en 1793 interrompit complètement des relations qui avaient à peine commencé ; et depuis le rétablissement de la paix, aucun traité nouveau n’a eu lieu. Mais les maximes qui ont présidé au traité de 1786 pourraient aussi bien maintenant qu’à cette époque trouver une application heureuse ; et les hommes d’État qui trouveraient le moyen de les mettre en pratique une seconde fois, en abolissant et en diminuant les prohibitions et les droits dont le commerce entre les deux pays est frappé, travailleraient certainement à la réalisation du bien-être commun.
Mac Culloch.
↑ Les faits ont démontré assez éloquemment depuis un siècle que le traité de Méthuen n’était pas au désavantage de la Grande-Bretagne A. B.
↑ Voyez le Dictionnaire des Monnaies, par M. Abot de Bazinghen, conseiller en la Cour des Monnaies de Paris, au mot Seigneuriage, tome II, page 589. Note de l’auteur.

Les erreurs de fait dans lesquelles est tombé l’auteur sur l’état des monnaies françaises sont très-aisées à rectifier.

%%%%%%%%%%%%%%%%%%%%%%%%%%%%%%%%%%%%%%%%%%%%%%%%%%%%%%%%%%%%%%%%%%%%%%%%%%%%%%%%
%                                  Chapitre 7                                  %
%%%%%%%%%%%%%%%%%%%%%%%%%%%%%%%%%%%%%%%%%%%%%%%%%%%%%%%%%%%%%%%%%%%%%%%%%%%%%%%%

\chapter{Des colonies}
\markboth{Des colonies}{}

Le premier établissement des différentes colonies européennes dans l’Amérique et dans les Indes occidentales n’a pas eu pour cause un intérêt aussi simple et aussi évident que celui qui donna lieu à l’établissement des anciennes colonies grecques et romaines.
Tous les différents États de l’ancienne Grèce ne possédaient chacun qu’un fort petit territoire, et quand la population de l’un d’eux s’était accrue au-delà de ce que le territoire pouvait aisément faire subsister, on envoyait une partie du peuple chercher une nouvelle patrie dans quelque contrée lointaine ; les nations guerrières dont ils étaient entourés de toutes parts ne permettaient guère à aucun de ces États de pouvoir agrandir beaucoup son territoire autour de soi. Les colonies des Doriens se rendaient principalement en Italie et en Sicile, qui, dans les temps antérieurs à la fondation de Rome, étaient habitées par des peuples entièrement barbares ; celles des Ioniens et des Éoliens, les deux autres grandes tribus des Grecs, se rendaient dans l’Asie Mineure et dans les îles de la mer Égée, dont il paraît que les habitants, à cette époque, étaient absolument au même état que ceux de l’Italie et de la Sicile. Quoique la mère patrie regardât la colonie comme un enfant qui avait droit en tout temps à ses secours et à toutes ses préférences, et qui lui devait en retour beaucoup de recon­naissance et de respect, cependant c’était à ses yeux un enfant émancipé, sur lequel elle ne prétendait réclamer aucune autorité ni juridiction directe. La colonie établis­sait les formes de son gouvernement, portait ses lois, choisissait ses magistrats, et faisait la paix ou la guerre avec ses voisins, comme un État indépendant, sans avoir besoin d’attendre l’approbation ou le consentement de la métropole. Il n’y a rien de plus simple et de plus évident que l’intérêt qui dirigea ces peuples dans chaque établissement de ce genre.
Rome, comme la plupart des autres républiques anciennes, fut fondée originaire­ment sur une loi agraire qui partagea le territoire commun, suivant certaines propor­tions, entre les différents citoyens qui composaient l’État. Le cours des choses humai­nes, les mariages, les successions, les aliénations, dérangèrent nécessairement cette division primitive, et il en arriva fréquemment que des terres qui avaient été destinées à la subsistance de plusieurs familles différentes tombèrent dans la possession d’une seule personne. Pour remédier à ce désordre (car cet état de choses fut regardé com­me un désordre), on porta une loi, qui restreignait à cinq cents jugera, environ trois cent cinquante acres[1] d’Angleterre, la quantité de terre qu’un citoyen pourrait posséder. Cette loi cependant (quoique nous lisions qu’elle a été mise à exécution en deux ou trois circonstances) fut négligée ou éludée, et l’inégalité des fortunes alla toujours croissant. La plus grande partie des citoyens n’avait pas de terres, et d’après les mœurs et les coutumes de ces temps-là, il était difficile à un homme libre de se maintenir sans cela dans l’indépendance. Aujourd’hui, quoiqu’un homme pauvre n’ait pas de terre en propriété, cependant, s’il a un petit capital, il peut affermer la terre d’un autre ou faire quelque petit commerce de détail ; et s’il n’a pas de capital, il peut trouver de l’emploi, ou comme artisan, ou dans les travaux de la campagne. Mais chez les anciens Romains les terres du riche étaient toutes cultivées par des esclaves qui tra­vaillaient sous un inspecteur esclave lui-même ; de manière qu’un homme libre pauvre n’avait guère la chance de trouver de l’emploi, soit comme fermier, soit comme ou­vrier. Toutes les professions du commerce et de l’industrie, même dans le commerce de détail, étaient aussi exercées par les esclaves des riches, pour le compte de leurs maîtres, et ceux-ci avaient trop de puissance et de crédit pour qu’un homme libre pauvre pût espérer de soutenir une pareille concurrence. Les citoyens qui ne possé­daient pas de terres n’avaient donc d’autres moyens de subsistance que les largesses des candidats aux élections annuelles. Lorsque les tribuns avaient envie d’animer le peuple contre les riches et les grands, ils lui rappelaient l’ancien partage des terres, et ils lui représentaient la loi qui limitait cette espèce de propriété privée, comme étant la loi fondamentale de la république. Le peuple prit l’habitude de demander des terres à grands cris, et les riches et les grands étaient bien résolus, comme on peut le penser, à ne lui céder aucune partie des leurs. Pour le contenter donc à un certain point, ils proposèrent fréquemment d’envoyer au-dehors une colonie nouvelle. Mais Rome conquérante n’était pas réduite, même dans ces occasions, à la nécessité d’envoyer ses citoyens chercher fortune par le monde, pour ainsi dire, sans savoir où se placer. Elle leur assignait, en général, des terres dans les provinces conquises de l’Italie, où, se trouvant établis dans l’étendue du domaine de la république, ils ne pouvaient jamais former un État indépendant ; ils n’étaient au plus qu’une espèce de corporation qui avait bien la faculté de porter des règlements pour son propre gouvernement, mais qui était sujette en tout temps à l’inspection, à la juridiction et à la puissance législative de la métropole. L’envoi d’une colonie nouvelle de ce genre non-seulement donnait quelque satisfaction au peuple, mais encore souvent formait une sorte de garnison dans une province nouvellement conquise, dont autrement l’obéissance aurait été fort peu assurée. Ainsi, soit que l’on considère la nature de l’établissement en lui-même, soit que l’on considère les motifs qui l’avaient fait faire, une colonie romaine était tout à fait différente d’une colonie grecque. Aussi les mots qui, dans les langues originaires, désignaient ces différents établissements avaient-ils des significations fort différentes. Le mot latin colonia veut simplement dire plantation ou culture des terres. Le mot grec [mot en grec dans le texte] veut dire au contraire une séparation de demeure, une émigration du pays, un abandon de la maison. Mais quoique les colonies établies par le peuple romain différassent à beaucoup d’égards des colonies grecques, cependant l’intérêt qui le porta à les établir n’était pas moins simple ni moins évident. Ces deux institutions tirèrent l’une et l’autre leur origine, ou d’une nécessité indispensable, ou d’une utilité claire et manifeste[2] 
L’établissement des colonies européennes dans l’Amérique et dans les Indes occidentales n’a pas été un effet de la nécessité ; et quoique l’utilité qui en est résultée ait été très-grande, cependant elle n’est pas tout à fait si claire ni si évidente. Cette utilité ne fut pas sentie lors de leur premier établissement ; elle ne fut le motif ni de cet établissement ni des découvertes qui y donnèrent occasion, et même encore aujour­d’hui, la nature de cette utilité, son étendue et ses bornes ne sont peut-être pas des choses parfaitement bien comprises.
Dans le cours des quatorzième et quinzième siècles, les Vénitiens faisaient un commerce très-avantageux en épiceries et autres denrées des Indes orientales, qu’ils répandaient chez les autres nations de l’Europe. Ils achetaient ces marchandises en Égypte, qui était alors sous la domination des Mamelucks, ennemis des Turcs, com­me l’étaient les Vénitiens, et cette union d’intérêt, aidée de l’argent de Venise, forma une telle liaison, que les Vénitiens eurent presque le monopole de ce commerce.
Les grands profits des Vénitiens excitèrent la cupidité des Portugais. Pendant le cours du quinzième siècle, ceux-ci avaient tâché de trouver par mer une route qui les conduisît aux pays d’où les Maures leur apportaient, à travers le désert, de l’ivoire et de la poudre d’or. Ils découvrirent les îles de Madère, les Canaries, les Açores, les îles du Cap-Vert, la côte de Guinée, celle de Loango, Congo, Angola et Benguela, et enfin le cap de Bonne-Espérance. Ils désiraient depuis longtemps avoir part au com­mer­ce avantageux des Vénitiens, et cette dernière découverte leur ouvrait une pers­pec­tive probable d’en venir à bout. En 1497, Vasco de Gama fit voile du port de Lisbonne avec une flotte de quatre vaisseaux, et après une navigation de onze mois, il toucha la côte de l’Indostan et conduisit ainsi à son terme un cours de découvertes suivi avec une grande constance et presque sans interruption pendant près d’un siècle.
Quelques années avant cet événement, tandis que l’Europe en suspens attendait l’issue des entreprises des Portugais, dont le succès paraissait encore être douteux, un pilote génois formait le dessein encore plus hardi de faire voile aux Indes orientales par l’ouest. La situation de ces pays était très-imparfaitement connue en Europe. Le peu de voyageurs européens qui les avaient vus en avaient exagéré la distance, peut-être parce qu’à des yeux simples et ignorants, ce qui était réellement très-grand, et qu’ils ne pouvaient mesurer, paraissait presque infini, ou peut-être parce qu’en repré­sen­tant à une distance aussi immense de l’Europe les régions par eux visitées, ils croyaient augmenter le merveilleux de leurs aventures. Colomb conclut avec justesse que, plus la route était longue par l’est, moins elle devait l’être par l’ouest. Il proposa donc de prendre cette route, comme étant à la fois la plus courte et la plus sûre, et il eut le bonheur de convaincre Isabelle de Castille de la possibilité du succès. Il partit du port de Palos en août 1492, près de cinq ans avant que la flotte de Vasco de Gama sortît du Portugal ; et, après un voyage de deux ou trois mois, il découvrit d’abord quelques-unes des petites îles Lucayes ou de Bahama, et ensuite la grande île de Saint-Domingue.
Mais les pays découverts par Colomb dans ce voyage ou dans ses voyages postérieurs n’avaient aucune ressemblance avec ceux qu’il avait été chercher. Au lieu de la richesse, de la culture et de la population de la Chine et de l’Indostan, il ne trouva, à Saint-Domingue et dans toutes les autres parties du Nouveau-Monde qu’il put voir, qu’un pays couvert de bois, inculte et habité seulement par quelques tribus de sauvages nus et misérables. Cependant, il ne pouvait aisément se décider à croire que ces pays ne fussent pas les mêmes que ceux décrits par Marco-Polo, le premier Européen qui eût vu les Indes orientales, ou du moins le premier qui en eût laissé quelque description ; et souvent, pour le ramener à l’idée favorite dont il était préoccupé, quoiqu’elle fût démentie par la plus claire évidence, il suffisait de la plus légère similitude, comme celle qui se trouve encore le nom de Cibao, montagne de Saint-Domingue, et le Cipango, mentionné par Marco-Polo. Dans ses lettres à Ferdi­nand et Isabelle, il donnait le nom de Indes aux pays qu’il avait découverts. Il ne faisait aucun doute que ce ne fût l’extrémité de ceux visités par Marco-Polo, et qu’il ne fût déjà peu éloigné du Gange ou des contrées qui avaient été conquises par Alexandre. Même quand il fut enfin convaincu que les pays où il était ne ressemblaient en rien à ceux-là, il continua toujours de se flatter que ces riches contrées n’étaient pas à une grande distance et, en conséquence, dans un autre voyage, il se mit à leur recherche le long de la côte de Terre-Ferme et vers l’isthme de Darien.
Par une suite de cette méprise de Colomb, le nom d’Indes est toujours demeuré depuis à ces malheureuses contrées, et quand à la fin il fut bien clairement démontré que les nouvelles Indes étaient totalement différentes des anciennes, les premières furent appelées Indes occidentales, pour les distinguer des autres qu’on nomma Indes orientales.
Il était néanmoins important pour Colomb que les pays qu’il avait découverts, quels qu’ils fussent, pussent être représentés à la cour d’Espagne comme des pays de très-grande importance ; et à cette époque, ces contrées, pour ce qui constitue la richesse réelle d’un pays, c’est-à-dire dans les productions animales ou végétales du sol, n’offraient rien qui pût justifier une pareille description.
Le plus gros quadrupède vivipare de Saint-Domingue était le cori, espèce d’ani­mal qui tient le milieu entre le rat et le lapin, et que M. de Buffon suppose être le même que l’aperéa du Brésil. Il ne paraît pas que cette espèce ait jamais été très-nombreuse, et on dit qu’elle a été depuis longtemps presque entièrement détruite, ainsi que quelques autres espèces d’animaux encore plus petits, par les chiens et les chats des Espagnols. C’était pourtant, avec un très-gros lézard nommé ivana ou iguane, ce qui constituait la principale nourriture animal qu’offrit le pays.
La nourriture végétale des habitants, quoique fort peu abondante par leur manque d’industrie, n’était pas tout à fait aussi chétive. Elle consistait en blé d’Inde, ignames, patates, bananes, etc., plantes qui étaient alors totalement inconnues en Europe et qui n’y ont jamais été depuis très-estimées, ou dont on a supposé ne pouvoir jamais tirer une substance aussi nourrissante que des espèces ordinaires de grains et de légumes cultivés de temps immémorial dans cette partie du monde.
La plante qui donne le coton offrait, à la vérité, une matière de fabrication très-importante, et c’était sans doute alors pour les Européens la plus précieuse de toutes les productions végétales de ces îles. Mais, quoiqu’à la fin du quinzième siècle les mousselines et autres ouvrages de coton des Indes orientales fussent très-recherchés dans tous les pays de l’Europe, cependant il n’y avait nulle part de manufactures de coton. Ainsi, cette production elle-même ne pouvait alors paraître d’une très-grande importance aux yeux des Européens.
Colomb ne trouvant donc rien, ni dans les végétaux ni dans les animaux des pays de ses nouvelles découvertes, qui pût justifier la peinture très-avantageuse qu’il voulait en faire, tourna son attention du côté des minéraux, et il se flatta d’avoir trouvé, dans la richesse des productions de ce dernier règne, de quoi compenser largement le peu de valeur de celles des deux autres. Les petits morceaux d’or dont les habitants se faisaient une parure, et qu’ils trouvaient fréquemment, à ce qu’il apprit, dans les ruisseaux et les torrents qui tombaient des montagnes, suffirent pour lui persuader que ces montagnes abondaient en mines d’or des plus riches. En consé­quence, il représenta Saint-Domingue comme un pays où l’or était en abon­dance, et dès lors comme une source inépuisable de véritables richesses Pour la couronne et pour le royaume d’Espagne, conformément aux préjugés qui règnent aujourd’hui et qui régnaient déjà à cette époque. Lorsque Colomb, au retour de son premier voyage, fut admis, avec les honneurs d’une espèce de triomphe, en la présence des souverains de Castille et d’Aragon, on porta devant lui, en pompe solen­nelle, les principales productions des pays qu’il avait découverts. Les seules parties de ces productions qui eussent quelque valeur consistaient en de petites lames, bracelets et autres ornements d’or, et en quelques balles de coton. Le reste était des objets de pure curiosité, propres à exciter l’étonnement du peuple : des joncs d’une taille extra­ordinaire, des oiseaux d’un très-beau plumage et des peaux rembourrées du grand alligator et du manati ; le tout précédé par six ou sept des malheureux naturels du pays, dont la figure et la couleur singulières ajoutaient beaucoup à la nouveauté de ce spectacle.
D’après le rapport de Colomb, le conseil de Castille résolut de prendre possession d’un pays dont les habitants étaient évidemment hors d’état de se défendre. Le pieux dessein de le convertir au christianisme sanctifia l’injustice du projet. Mais l’espoir d’y puiser des trésors fut le vrai motif qui décida l’entreprise ; et pour donner le plus grand poids à ce motif, Colomb proposa que la moitié de tout l’or et de tout l’argent qu’on y trouverait appartînt à la couronne. Cette offre fut acceptée par le conseil.
Tant que la totalité ou la plus grande partie de l’or que les premiers chefs de l’entreprise importèrent en Europe ne leur coûta que la peine de piller des sauvages sans défense, cette taxe, quelque lourde qu’elle fût, n’était pas très-difficile à payer ; mais quand les naturels furent une fois dépouillés de tout ce qu’ils en avaient, ce qui fut complètement achevé en six ou huit ans à Saint-Domingue et dans les autres pays de la découverte de Colomb, et quand, pour en trouver davantage, il fut devenu né­ces­saire de fouiller les mines, alors il n’y eut plus aucune possibilité d’acquitter cette taxe. Aussi dit-on que la manière rigoureuse dont on l’exigea fut la première cause de l’abandon total des mines de Saint-Domingue, qui, depuis, n’ont jamais été exploitées. Elle fut donc bientôt réduite à un tiers, ensuite à un cinquième, puis à un dixième, et enfin à un vingtième du produit brut des mines d’or. La taxe sur l’argent continua pendant longtemps à rester au cinquième du produit brut, et ce n’est que dans le courant de ce siècle qu’elle a été réduite au dixième. Mais il ne paraît pas que les premiers entrepreneurs aient pris un grand intérêt à ce dernier métal. Tout ce qui était moins précieux que l’or ne leur semblait pas digne d’attention.
Toutes les autres entreprises des Espagnols dans le Nouveau-Monde, postérieures à celles de Colomb, paraissent avoir eu le même motif. Ce fut cette soif sacrilège de l’or qui porta Oïeda, Nicuessa et Vasco Nuñez de Balboa à l’isthme de Darien, qui porta Cortez au Mexique, Almagro et Pizarre au Chili et au Pérou. Quand ces aventuriers arrivaient sur quelque côte inconnue, leur premier soin était toujours de s’enquérir si on pouvait y trouver de l’or et, d’après les informations qu’ils se procu­raient sur cet article, ils se déterminaient à s’établir dans le pays ou à l’abandonner.
De tous les projets incertains et dispendieux qui mènent à la banqueroute la plupart des gens qui s’y livrent, il n’y en a peut-être aucun de si complètement ruineux que la recherche de nouvelles mines d’or ou d’argent. C’est, à ce qu’il semble, la plus inégale de toutes les loteries du monde, ou celle dans laquelle il y a le moins de proportion entre le gain de ceux qui ont des lots, et la perte de ceux qui tirent des billets blancs ; car, quoique les lots soient en très-petite quantité et les billets blancs très-nombreux, le prix ordinaire du billet est la fortune tout entière d’un homme très-riche. Au lieu de remplacer le capital employé avec les profits ordinaires que rendent les capitaux, les entreprises pour des recherches de mines absorbent communément et profits et capitaux. De tous les projets, ce sont donc ceux auxquels un législateur prudent, jaloux d’augmenter le capital de son pays, évitera de donner des encourage­ments extraordinaires, ou vers lesquels il cherchera le moins à diriger une plus grande partie de ce capital que celle qui s’y porterait d’elle-même. La folle confiance que les hommes ont presque tous dans leur bonne fortune est telle, qu’il y a toujours une trop grande quantité du capital du pays disposée à se porter à ces sortes d’emplois, pour peu qu’il y ait la moindre probabilité de succès.
Mais, quoique les projets de ce genre aient toujours été jugés très-défavora­ble­ment par la saine raison et par l’expérience, la cupidité humaine les a, pour l’ordi­naire, envisagés d’un tout autre œil. La même passion qui a fait adopter à tant de gens l’idée absurde de la pierre philosophale, a suggéré à d’autres la chimère non moins absurde d’immenses mines abondantes en or et en argent. Ils ne considèrent pas que la valeur de ces métaux, dans tous les siècles et dans tous les pays, a procédé princi­palement de leur rareté, et que leur rareté provient de ce que la nature les a déposés en quantités extrêmement petites à la fois dans un même lieu ; de ce qu’elle a presque partout enfermé ces quantités si petites dans les substances les plus dures et les plus intrai­tables et, par conséquent, de ce qu’il faut partout des travaux et des dépenses propor­tionnées à ces difficultés pour pénétrer jusqu’à eux et pour les obtenir. Ils se flattent qu’on pourrait trouver, en plusieurs endroits, des veines de ces métaux, aussi grandes et aussi abondantes que celles qu’on rencontre communément dans les mines de plomb, de cuivre, d’étain ou de fer. Le rêve de sir Walter Raleigh, sur la ville d’or et le pays d’Eldorado, nous fait bien voir que les gens sages eux-mêmes ne sont pas toujours exempts de se laisser entraîner à ces étranges illusions. Plus de cent ans après la mort de ce grand homme, le jésuite Gumila était encore persuadé de l’exis­ten­ce de cette contrée merveilleuse, et il témoignait avec la plus grande chaleur, je puis dire même avec la plus grande franchise, combien il se trouverait heureux de pouvoir porter la lumière de l’Évangile chez un peuple en état de récompenser aussi généreu­sement les pieux travaux des missionnaires.
On ne connaît aujourd’hui, dans les pays des premières découvertes des Espa­gnols, aucunes mines d’or ou d’argent qui soient censées valoir la peine d’être exploitées. Il est vraisemblable que sur les quantités de ces métaux qu’on a dit y avoir été trouvées par ces premiers aventuriers, ainsi que sur la fertilité des mines qui y ont été exploitées immédiatement après la première découverte, il y avait eu de très-grandes exagérations ; toutefois, le compte rendu de tout ce qu’y trouvèrent ces aven­turiers fut suffisant pour enflammer la cupidité de tous leurs compatriotes. Chaque Espagnol qui faisait voile pour l’Amérique s’attendait à rencontrer un Eldorado. La fortune aussi fit à cet égard ce qu’il lui est bien rarement arrivé de faire en d’autres occasions : elle réalisa jusqu’à un certain point les espérances extravagantes de ses adorateurs, et dans la découverte et la conquête du Mexique et du Pérou, dont l’un fut découvert environ trente ans, l’autre environ quarante ans après la première expé­di­tion de Colomb, elle leur offrit ces métaux précieux avec une profusion qui répondait en quelque sorte aux idées qu’ils s’en étaient faites.
Ce fut donc un projet de commerce aux Indes orientales qui donna lieu à la pre­mière découverte des Indes occidentales. Un projet de conquête donna lieu à tous les établissements des Espagnols dans ces contrées nouvellement découvertes. Les motifs qui les portèrent à entreprendre ces conquêtes, ce furent des projets d’ouvrir des mines d’or et d’argent ; et une suite d’événements qu’aucune sagesse humaine n’au­rait pu prévoir rendit ces projets beaucoup plus heureux, dans leur issue, que les entrepreneurs ne pouvaient raisonnablement l’espérer.
Les premiers aventuriers qui, chez toutes les autres nations de l’Europe, tentèrent d’acquérir des établissements en Amérique, y furent entraînés par de semblables chi­mères ; mais tous ne furent pas également fortunés. Il y avait plus d’un siècle que les premiers établissements au Brésil étaient faits, qu’on n’y avait encore découvert aucune mine d’argent, d’or, ni de diamants. Dans les colonies anglaises, françaises, hollandaises et danoises, on n’en a encore découvert aucune, au moins aucune qui soit actuellement censée valoir la peine d’être exploitée. Cependant, les premiers Anglais qui firent un établissement dans l’Amérique septentrionale offrirent au roi, comme un motif pour obtenir leurs patentes, le cinquième de l’or et de l’argent qu’on pourrait y trouver. En conséquence, ce cinquième fut réservé à la couronne dans les patentes accordées à sir Walter Raleigh, aux compagnies de Londres et de Plymouth, au conseil de Plymouth, etc. À l’espoir de trouver des mines d’or et d’argent, ces pre­miers entrepreneurs joignaient encore celui de découvrir un passage au nord, pour aller aux Indes orientales. jusqu’à ce moment, ils n’ont pas été plus heureux dans l’un que dans l’autre.


Section seconde.
Causes de la prospérité des colonies nouvelles.

Une colonie d’hommes civilisés, qui prend possession ou d’un pays désert, ou d’un pays si faiblement peuplé, que les naturels font aisément place aux nouveaux colons, avance plus rapidement qu’aucune autre société humaine vers un état de grandeur et d’opulence.
Ceux qui forment la colonie emportent avec eux des connaissances en agriculture et dans les autres arts utiles, fort supérieures à ce que des peuples sauvages et bar­ba­res pourraient en acquérir par eux-mêmes dans le cours de plusieurs siècles. Ils emportent aussi avec eux l’habitude de la subordination, quelque notion du gouverne­ment établi dans leur pays, du système de lois qui lui sert de base, et d’une admi­nistration fixe et uniforme de la justice. Naturellement donc, ils instituent quelque chose du même genre dans leur nouvel établissement, tandis qu’au contraire, parmi les nations sauvages ou barbares, les progrès naturels du gouvernement et des lois sont encore plus lents que les progrès naturels que font les arts quand une fois ils peuvent fleurir, et quand le gouvernement et les lois sont établis au point de leur assurer une protection suffisante. Chaque colon a plus de terre qu’il ne lui est possible d’en cultiver. Il n’a ni fermages à acquitter, ni presque point d’impôts à payer. Il n’y a pas de propriétaire qui vienne partager le produit de son travail, et la part qu’y prend le souverain n’est ordinairement qu’une bagatelle. Il a tous les motifs possibles d’augmenter, autant qu’il le peut, un produit qui lui appartiendra presque tout entier ; mais la terre qu’il possède est pour l’ordinaire d’une telle étendue, qu’avec toute son industrie et celle des autres bras qu’il peut se procurer, à peine viendra-t-il à bout de lui faire produire le dixième de ce qu’elle est capable de rendre. Il s’applique donc à rassembler de tous les côtés des ouvriers, et à se les assurer par les salaires les plus forts ; mais ces salaires élevés, joints à l’abondance des terres et à leur bon marché, mettent bientôt ces ouvriers en état de le quitter, pour devenir eux-mêmes proprié­taires, et salarier aussi, avec la même libéralité, d’autres ouvriers qui bientôt à leur tour quitteront leurs maîtres pour la même cause, La récompense libérale du travail est un encouragement au mariage. Les enfants sont bien nourris et soignés conve­nablement, dans les années de leur âge le plus tendre, et quand ils sont tous élevés, la valeur de leur travail rembourse fort au-delà la dépense de leur entretien. Arrivés à leur pleine croissance, le haut prix du travail et le bas prix des terres les mettent à même de s’établir de la même manière que leurs pères l’ont fait avant eux.
Dans les autres pays, la rente et les profits s’accroissent aux dépenses des salaires et les réduisent presque à rien, en sorte que les deux classes supérieures écrasent la dernière ; mais dans les colonies nouvelles, les deux premières classes se trouvent obligées, par leur propre intérêt, à traiter la classe inférieure avec plus d’humanité et de générosité, au moins dans les colonies où cette dernière classe n’est pas dans un état d’esclavage. On y peut acquérir presque pour rien des terres incultes que la nature a douées de la plus grande fertilité. L’augmentation du revenu qu’attend de leur amé­lio­ration le propriétaire, qui est toujours l’entrepreneur de la culture, est ce qui constitue son profit, et dans de telles circonstances ce profit est ordinairement très-fort ; mais il ne peut faire ce grand profit sans mettre en œuvre le travail d’autrui pour défricher et cultiver la terre, et la disproportion qui a lieu, pour l’ordinaire, dans les colonies nouvelles, entre la grande étendue de terre à cultiver et le petit nombre d’habitants, est cause qu’il ne peut se procurer ce travail qu’avec difficulté. Il ne dispute donc pas sur le taux des salaires, car il est disposé à employer le travail à tout prix. Les hauts salaires du travail sont un encouragement à la population. La grande quantité de bonnes terres et leur bon marché excitent à faire des améliorations et mettent le propriétaire en état de payer ces hauts salaires. C’est dans cette élévation des salaires que consiste presque tout le prix que coûte la terre, et quoiqu’ils soient très-forts, considérés comme salaires de travail, ils sont toutefois encore très-bas, considérés comme le prix d’une chose qui a tant de valeur. Or, ce qui encourage la culture et la population amène véritablement l’opulence et la prospérité.
Ainsi paraît-il que les progrès de la plupart des anciennes colonies grecques, en agrandissement et en opulence, ont été extrêmement rapides ; plusieurs d’elles, dans le cours d’un siècle ou deux, ont, à ce qu’il semble, rivalisé et même surpassé leur mère patrie. Syracuse et Agrigente en Sicile, Tarente et Locres en Italie, Éphèse et Milet dans l’Asie Mineure, paraissent, d’après tous les témoignages que nous en avons, avoir été au moins les égales de quelque ville que ce soit de l’ancienne Grèce. Quoique fondées postérieurement, cependant nous y trouvons tous les arts de la civilisation, la philosophie, la poésie et l’éloquence, cultivés d’aussi bonne heure, et portés à un aussi haut degré de perfection que dans tout autre endroit de la mère patrie. Il est remarquable que les écoles des deux plus anciens philosophes grecs, celles de Thalès et de Pythagore, ne furent pas fondées dans l’ancienne Grèce, mais que l’une le fut dans une colonie d’Asie, et l’autre dans une colonie d’Italie. Toutes ces colonies s’étaient établies dans des pays habités par des peuples barbares, qui cédèrent bientôt la place aux nouveaux colons. Elles avaient de bonnes terres en abondance, et comme elles étaient entièrement indépendantes de la mère patrie, elles avaient la liberté de diriger leurs affaires de la manière qu’elles jugeaient la plus conforme à leur intérêt.
Il s’en faut bien que l’histoire des colonies romaines soit aussi brillante. Quelques-unes d’elles, à la vérité, telles que Florence, sont parvenues, dans une période de plusieurs siècles, et après la chute de la mère patrie, à former un État considérable ; mais il ne paraît pas que les progrès d’aucune d’elles aient été très-rapides. Elles furent toutes établies dans les provinces conquises, qui le plus souvent avaient été aupa­ra­vant pleinement habitées. La portion de terre assignée à chaque colon fut rarement très-considérable ; et comme la colonie n’était pas indépendante, elle n’eut pas toujours la liberté de conduire ses affaires de la manière qui lui aurait paru le plus à son avantage.
Du côté de l’abondance des bonnes terres, les colonies européennes établies en Amérique et dans les Indes occidentales ressemblent à celles de l’ancienne Grèce, et même l’emportent beaucoup sur elles. Du côté de la dépendance de la métropole, elles ressemblent à celles de l’ancienne Rome ; mais le poids de cette dépendance a été, pour toutes, plus ou moins allégé par leur grand éloignement de l’Europe ; à une telle distance, elles se sont trouvées moins sous les yeux et sous la puissance de leur mère patrie. Quand elles ont cherché à diriger leurs affaires d’après leurs propres vues, elles n’ont eu le plus souvent aucune inspection à subir, parce qu’en Europe on ignorait leur conduite, ou qu’on n’en comprenait pas l’objet ; dans quelques autres circonstances, on les a tout simplement laissées faire ; et comme, attendu l’éloignement, il était difficile de les contraindre, on s’est vu réduit à plier. Le gouvernement d’Espagne lui-même, tout arbitraire et violent qu’il est, a bien été obligé, en maintes occasions, de révoquer ou de modifier les ordres qu’il avait donnés pour le régime de ses colonies, et il a cédé à la crainte d’exciter une insurrection générale ; aussi, les colonies européennes ont-elles fait des progrès très-considérables en amélioration, en population et en richesse.
La couronne d’Espagne, au moyen de la portion qu’elle s’était réservée dans l’or et dans l’argent, a tiré un revenu de ses colonies dès l’époque de leur premier établis­sement ; ce revenu, d’ailleurs, était de nature à exciter la cupidité et à faire naître les plus folles espérances. En conséquence, les colonies espagnoles, dès leur fondation, furent pour leur mère patrie l’objet d’une extrême attention, tandis que celles des autres nations de l’Europe furent, en grande partie, négligées pendant longtemps. Mal­gré cette attention d’une part et cette négligence de l’autre, les premières n’en pros­­pé­rèrent pas mieux pour cela, et les autres n’en allèrent pas plus mal. Propor­tion­nellement à l’étendue des pays dont elles ont jusqu’à un certain point la possession, les colonies espagnoles passent pour les moins peuplées et les moins florissantes de presque toutes les autres colonies européennes ; toutefois les colonies espagnoles elles-mêmes ont fait certainement des progrès très-grands et très-rapides en culture et en population. D’après le rapport d’Ulloa, la ville de Lima, fondée depuis la conquête, paraîtrait avoir contenu, il y a près de trente ans, cinquante mille habitants. Le même auteur nous parle de Quito, qui n’avait été qu’un misérable hameau d’Indiens, comme renfermant de son temps une population égale. Gemeli Carreri, qu’on dit être à la vérité un voyageur supposé, mais qui, dans tout ce qu’il a écrit, paraît avoir suivi d’ex­cel­lentes instructions, représente la ville de Mexico comme contenant cent mille habitants, population qui, malgré toutes les exagérations des écrivains espagnols, est probablement encore plus du quintuple de ce que contenait cette ville au temps de Montezuma. La population de ces villes excède de beaucoup celle de Boston, de New-York et de Philadelphie, les trois plus grandes villes des colonies anglaises[3]. Avant la conquête des Espagnols, il n’y avait ni au Mexique ni au Pérou de bestiaux propres au trait ; le lama était la seule bête de charge qu’eussent ces peuples, et il paraît que pour la force cet animal était fort au-dessous de l’âne ordinaire. La charrue était inconnue aux habitants ; ils ignoraient l’usage du fer. Ils n’avaient pas de monnaie, et n’avaient établi aucune sorte d’instrument de commerce ; leurs échanges se faisaient par troc. Leur principal outil d’agriculture était une espèce de bêche de bois ; des pierres tranchantes leur servaient de couteaux et de haches pour couper ; des arêtes de poisson et des nerfs très-durs de certains animaux, d’aiguilles pour coudre, et c’était en cela que consistaient, à ce qu’il semble, les principaux outils de leurs mé­tiers. Dans cet état de choses, il paraît impossible que l’un ou l’autre de ces empires ait pu être civilisé ni aussi bien cultivé qu’aujourd’hui, où ils sont abondamment pourvus de toutes sortes de bestiaux d’Europe, et où l’usage du fer, de la charrue et de la plu­part de nos arts s’est introduit chez eux ; or, la population d’un pays doit nécessai­rement être en proportion du degré de sa civilisation et de sa culture. Malgré la barbarie avec laquelle on a détruit les naturels du pays après la conquête, vraisembla­blement ces deux grands empires sont aujourd’hui plus peuplés qu’ils ne l’ont jamais été, et le peuple y est certainement d’une nature fort différente ; car je pense que tout le monde conviendra que les créoles espagnols sont, à beaucoup d’égards, supérieurs aux anciens Indiens.
Après les établissements des Espagnols, celui des Portugais au Brésil est le plus ancien de tous ceux des nations européennes en Amérique. Mais, comme il se passa beaucoup de temps après la première découverte sans qu’on y reconnût aucune mine d’or ou d’argent, et que par cette raison il ne rapportait que peu ou point de revenu à la couronne, il fut longtemps en grande partie négligé, et ce fut pendant l’époque où on le traita avec cette indifférence, qu’il prit la forme d’une riche et puissante colonie. Dans le temps où le Portugal était sous la domination de l’Espagne, le Brésil fut attaqué par les Hollandais, qui s’emparèrent de sept des quatorze provinces dont il est composé. Ils se disposaient à se mettre bientôt en possession des sept autres, quand le Portugal recouvra son indépendance par l’élévation de la maison de Bragance au trône. Les Hollandais alors, comme ennemis des Espagnols, ainsi que les Portugais, devinrent amis de ces derniers. Ils consentirent donc à laisser au roi de Portugal la partie du Brésil qu’ils n’avaient pas conquise, et celui-ci convint de leur abandonner celle dont ils étaient en possession, comme un objet qui ne valait pas la peine de se brouiller avec de si bons alliés. Mais le gouvernement hollandais commença bientôt à opprimer les colons portugais, et ceux-ci, au lieu de perdre du temps à se plaindre, prirent les armes contre leurs nouveaux maîtres, et de leur propre détermination, par leur courage seul, de concert il est vrai avec la mère patrie, mais sans aucun secours déclaré de sa part, ils chassèrent les Hollandais du Brésil. Ceux-ci, voyant donc qu’il leur était impossible de garder pour eux aucune portion du pays, aimèrent mieux le voir repasser en entier sous la domination du Portugal. On dit qu’il y a dans cette colonie plus de six cent mille habitants, tant Portugais que descendants des Portugais, créoles, mulâtres et races mêlées de Portugais et de Brésiliens. Aucune colonie en Amérique ne passe pour contenir un aussi grand nombre d’habitants d’origine euro­péenne.
Vers la fin du quinzième siècle, et pendant la plus grande partie du seizième, l’Espagne et le Portugal se trouvèrent être les deux grandes puissances navales de l’Océan ; car, quoique le commerce de Venise s’étendît par toute l’Europe, les flottes de cette répu­blique ne s’étaient guère avancées au-delà de la Méditerranée. Les Espagnols, pour avoir les premiers découvert l’Amérique, la réclamaient tout entière comme leur propriété, et quoiqu’ils n’aient pu empêcher une puissance navale aussi considérable que celle du Portugal de s’établir au Brésil, cependant la terreur qu’inspirait leur nom était alors telle, que la plupart des autres nations de l’Europe n’osaient faire d’établis­sement dans aucune autre partie de ce grand continent. Les Français qui tentèrent de se fixer dans la Floride furent tous mis à mort par les Espagnols. Mais la décadence de la puissance navale de ces derniers, par suite de la déroute ou de la perte de ce qu’ils nommaient leur invincible Armada, qui eut lieu vers la fin du seizième siècle, leur ôta le pouvoir d’arrêter plus longtemps les établissements des autres nations europé­ennes. Ainsi, dans le cours du dix-septième siècle, les Anglais, les Français, les Hollandais, les Danois et les Suédois, c’est-à-dire toutes les grandes nations qui avaient des ports sur l’Océan, essayèrent de faire quelques établissements dans le Nouveau-Monde.
Les Suédois s’établirent à New-jersey, et le nombre de familles suédoises qu’on y trouve encore démontre suffisamment que cette colonie était dans le cas de très-bien prospérer, si elle eût été protégée par la mère patrie. Mais étant abandonnée par la Suède, elle fut bientôt envahie par la colonie hollandaise de New-York, laquelle à son tour, en 1764, tomba au pouvoir des Anglais.
Les petites îles de Saint-Thomas et de Santa-Cruz sont les seuls pays que les Danois aient jamais possédés au Nouveau-Monde. De plus, ces petits établissements ont été mis sous le régime d’une compagnie exclusive, qui seule avait le droit tant d’acheter le produit surabondant des colons, que de leur fournir toutes les marchan­dises étrangères dont ils avaient besoin, et qui non-seulement avait la faculté de les opprimer dans ses achats ainsi que dans ses ventes, mais encore avait le plus grand intérêt à le faire. Le gouvernement d’une compagnie exclusive de marchands est peut-être, pour un pays quelconque, le pire de tous les gouvernements. Cependant, ce fu­nes­te régime ne fut pas encore capable d’arrêter totalement les progrès de ces colonies, quoiqu’il les ait rendus plus lents et plus languissants. Le feu roi de Dane­mark supprima cette compagnie, et depuis ce temps ces colonies ont extrêmement prospéré.
Les établissements des Hollandais dans les Indes occidentales, aussi bien que ceux des Indes orientales, ont été mis, dès leur origine, sous le régime d’une compa­gnie exclusive. En conséquence, les progrès de quelques-uns d’eux, quoique rapides si on les compare aux progrès de presque tout autre pays cultivé et peuplé d’ancienne date, ont néanmoins été languissants et tardifs, en comparaison de ceux de la plupart des colonies nouvelles. La colonie de Surinam, quoique très-considérable, est cepen­dant encore inférieure à la plupart des colonies à sucre des autres nations de l’Europe. La colonie de la Nouvelle-Belgique, qui forme aujourd’hui les deux provinces de New-York et de New-jersey, serait aussi devenue probablement bientôt une colonie importante, même quand elle serait restée sous le gouvernement des Hollandais. La grande quantité et le bon marché de bonnes terres sont des causes si puissantes de prospérité, que même le plus mauvais gouvernement est à peine capable d’arrêter totalement leur activité ; et puis, la distance de la mère patrie eût mis les colons à portée d’échapper plus ou moins, par la contrebande, au monopole dont la compagnie jouissait contre eux. À présent, la compagnie permet à tout vaisseau hollandais de commercer à Surinam, en payant, pour cette permission, 2 et demi pour 100 de la valeur de la cargaison, et elle se réserve seulement le monopole exclusif du com­merce direct d’Afrique en Amérique, qui consiste presque entièrement dans la traite des esclaves. Cette modification des privilèges exclusifs de la compagnie est vrai­semblablement la cause principale du degré de prospérité dont jouit actuellement cette colonie. Curaçao et Saint-Eustache, les deux îles principales qui appartiennent aux Hollandais, sont des ports francs ouverts aux vaisseaux de toutes les nations ; et une telle franchise, au milieu d’autres colonies meilleures, mais dont les ports ne sont ouverts qu’à une seule nation, a été, pour ces deux îles stériles, la grande source de leur prospérité.
La colonie française du Canada a été, pendant la plus grande partie du dernier siècle et une partie de celui-ci, sous le régime d’une compagnie exclusive. Sous une admi­nis­tration aussi nuisible, ses progrès furent naturellement très-lents en compa­rai­son de ceux des autres colonies nouvelles ; mais ils devinrent beaucoup plus rapides lorsque cette colonie fut dissoute, après la chute de ce qu’on appelle l’affaire du Mississippi. Quand les Anglais prirent possession de ce pays, ils y trouvèrent près du double d’habitants de ce que le père Charlevoix y en avait compté vingt à trente ans auparavant. Ce jésuite avait parcouru tout le pays, et il n’avait aucun motif de le représenter moins considérable qu’il ne l’était réellement.
La colonie française de Saint-Domingue fut fondée par des pirates et des flibus­tiers qui y demeurèrent longtemps sans recourir à la protection de la France et même sans reconnaître son autorité ; et quand cette race de bandits eut assez pris le caractère de citoyens pour reconnaître l’autorité de la mère patrie, pendant longtemps encore il fut nécessaire d’exercer cette autorité avec beaucoup de douceur et de circonspection. Durant le cours de cette période, la culture et la population de la colo­nie prirent un accroissement extrêmement rapide. L’oppression même de la compagnie exclusive à laquelle, ainsi que toutes les autres colonies françaises, elle fut assujettie pour quelque temps, put bien sans doute ralentir un peu ses progrès, mais ne fut pas encore capable de les arrêter tout à fait. Le cours de sa prospérité reprit le même essor qu’auparavant, aussitôt qu’elle fut délivrée de cette oppression. Elle est maintenant la plus importante des colonies à sucre des Indes occidentales, et l’on assure que son produit excède celui de toutes les colonies à sucre de l’Angleterre, prises ensemble. Les autres colonies à sucre de la France sont toutes, en général, très-florissantes.
Mais il n’y a pas de colonies dont le progrès ait été plus rapide que celui des colonies anglaises dans l’Amérique septentrionale.
L’abondance de terres fertiles et la liberté de diriger leurs affaires comme elles le jugent à propos, voilà, à ce qu’il semble, les deux grandes sources de prospérité de toutes les colonies nouvelles.
Du côté de la quantité de bonnes terres, les colonies anglaises de l’Amérique septentrionale, quoique sans doute très-abondamment pourvues, sont cependant infé­rieures aux colonies espagnoles et portugaises, et ne sont pas supérieures à quelques-unes de celles possédées par les Français avant la dernière guerre. Mais les insti­tutions politiques des colonies anglaises ont été bien plus favorables à la culture et à l’amélioration de ces bonnes terres, que ne l’ont été les institutions d’aucune des colonies des trois autres nations.
Premièrement, si l’accaparement des terres incultes est un abus qui n’a pu être, à beaucoup près, totalement prévenu dans les colonies anglaises, au moins y a-t-il été plus restreint que dans toute autre colonie. La loi coloniale, qui impose à chaque propriétaire l’obligation de mettre en valeur et de cultiver, dans un temps fixé, une portion déterminée de ses terres, et qui, en cas de défaut de sa part, déclare que ces terres négligées pourront être adjugées à un propriétaire, est une loi qui, sans avoir été peut-être très-rigoureusement exécutée, a néanmoins produit quelque effet.
Secondement, il n’y a pas en Pensylvanie de droit de primogéniture, et les terres se partagent comme des biens meubles, par portions égales, entre tous les enfants. Dans trois des provinces de la Nouvelle-Angleterre, l’aîné a seulement double por­tion, comme dans la loi de Moïse. Ainsi, quoique dans ces provinces il puisse arriver quelquefois qu’une trop grande quantité de terres vienne se réunir dans les mains d’un individu, il est probable que, dans le cours d’une ou deux générations, elle se retrou­vera suffisamment divisée. À la vérité, dans les autres colonies anglaises, le droit de primogéniture a lieu comme dans la loi d’Angleterre. Mais, dans toutes les colonies anglaises, les terres étant toutes tenues à simple cens[4], cette nature de propriété facilite les aliénations, et le concessionnaire d’une grande étendue de terrain trouve son inté­rêt à en aliéner la plus grande partie le plus vite qu’il peut, en se réservant seulement une petite redevance foncière. Dans les colonies espagnoles et portugaises, ce qu’on nomme le droit de majorat (jus majoratus), a lieu dans la succession de tous ces grands domaines auxquels il y a quelques droits honorifiques attachés. Ces domaines passent tout entiers à une seule personne et sont, en effet, substitués et inaliénables. Les colonies françaises, il est vrai, sont régies par la coutume de Paris, qui est beau­coup plus favorable aux puînés que la loi d’Angleterre, dans la succession des immeu­bles. Mais, dans les colonies françaises, si une partie quelconque d’un bien noble ou tenu à titre de foi et hommage est aliénée, elle reste assujettie, pendant un certain temps, à un droit de retrait ou rachat, soit envers l’héritier du seigneur, soit envers l’héritier de la famille, et tous les plus gros domaines du pays sont tenus en fief, ce qui gêne nécessairement les aliénations. Or, dans une colonie nouvelle, une grande propriété inculte sera bien plus promptement divisée par la voie de l’aliénation que par celle de la succession. La quantité et le bon marché des bonnes terres, comme on l’a déjà observé, sont les principales sources de la prospérité rapide des colonies nouvelles. Or, la réunion des terres en grandes propriétés détruit, par le fait, et cette quantité et ce bon marché. D’ailleurs, la réunion des terres incultes en grandes pro­priétés est ce qui s’oppose le plus à leur amélioration. Or, le travail qui est employé à l’amélioration et à la culture des terres est celui qui rend à la société le produit le plus considérable en quantité et en valeur. Le produit du travail, dans ce cas, paye non-seulement ses propres salaires et le profit du capital qui le met en œuvre, mais encore la rente de la terre sur laquelle il s’exerce. Ainsi, le travail des colons anglais étant employé, en plus grande quantité, à l’amélioration et à la culture des terres, est dans le cas de rendre un plus grand produit, et un produit d’une plus grande valeur que le travail de ceux d’aucune des trois autres nations, lequel, par le fait de l’accaparement de la terre, se trouve plus ou moins détourné vers des emplois d’une autre nature.
Troisièmement, il est à présumer, non-seulement que le travail des colons anglais rend un produit plus considérable en quantité et en valeur, mais encore que, vu la modicité des impôts, il leur reste une portion plus grande de ce produit, portion qu’ils peuvent capitaliser et employer à entretenir un nouveau surcroît de travail. Les colons anglais n’ont pas encore payé la moindre contribution pour la défense de la mère patrie ou pour l’entretien de son gouvernement civil. Au contraire, jusqu’à présent les frais de leur propre défense ont été presque entièrement à la charge de la métropole. Or, la dépense qu’exigent l’armée et la marine est, sans aucune proportion, plus forte que celle de l’entretien du gouvernement civil. D’ailleurs, la dépense de leur gouver­ne­ment civil a toujours été très-modique. Elle s’est bornée, en général, à ce qu’il fallait pour payer des salaires convenables au gouverneur, aux juges et à quelques autres officiers de police, et pour entretenir un petit nombre d’ouvrages publics de la pre­mière utilité. La dépense de l’établissement civil de Massachusets, avant le commen­ce­ment des derniers troubles, ne montait pour l’ordinaire qu’à environ 18,000 livres sterling par année ; celle de New-Hampshire et de Rhode-Island, à 3,500 livres pour chacun ; celle de Connecticut, à 4,000 livres ; celle de New-York et de la Pensylvanie, à 4,500 livres pour chacun ; celle de New-jersey, à 1,200 livres ; celle de la Virginie et de la Caroline du Sud, à 8,000 livres pour chacune. La dépense de l’établissement civil de la Nouvelle-Écosse et de la Géorgie est en partie couverte par une concession annuelle du parlement ; mais la Nouvelle-Écosse paye seulement environ 7,000 lires par an pour les dépenses publiques de la colonie, et la Géorgie environ 2,500 livres. En un mot, tous les différents établissements civils de l’Amérique septentrionale, à l’exception de ceux du Maryland et de la Caroline du Nord, dont on n’a pu se procurer aucun état exact, ne coûtaient pas aux habitants, avant le commencement des troubles actuels, au-delà de 64,700 livres par année ; exemple à jamais mémorable du peu de frais qu’exigent trois millions d’hommes pour être, non-seulement gouvernés, mais bien gouvernés. Il est vrai que la partie la plus importante des dépenses d’un gouver­nement, celles de défense et de protection, ont été constamment défrayées par la mère patrie. Et puis, le cérémonial du gouvernement civil dans les colonies, pour la récep­tion d’un gouverneur, pour l’ouverture d’une nouvelle assemblée, etc., quoique rempli avec la décence convenable, n’est accompagné d’aucun étalage ou pompe dispen­dieuse. Leur gouvernement ecclésiastique est réglé sur un plan également économi­que. Les dîmes sont une chose inconnue chez eux, et leur clergé, qui est loin d’être nombreux, est entretenu, ou par de modiques appointements, ou par les contributions volontaires du peuple. Les puissances d’Espagne et de Portugal, au contraire, fournissent à une partie de leur propre entretien par des taxes levées sur leurs colonies. La France, à la vérité, n’a jamais retiré aucun revenu considérable de ses colonies, les impôts qu’elle y lève étant, en général, dépensés pour elles. Mais le gouvernement colonial de ces trois nations est monté sur un pied beaucoup plus dispendieux, et est accompagné d’un cérémonial bien plus coûteux. La réception d’un nouveau vice-roi du Pérou, par exem­ple, a souvent absorbé des sommes énormes. Des cérémonies aussi coûteuses, non-seulement sont une taxe réelle que les colons riches ont à payer dans ces occa­sions particulières, mais elles contribuent encore à introduire parmi eux des habitudes de vanité et de profusion dans toutes les autres circonstances. Ce sont non-seulement des impôts fort onéreux à payer accidentellement, mais c’est une source d’impôts perpétuels du même genre, beaucoup plus onéreux encore, les impôts ruineux du luxe et des folles dépenses des particuliers. D’ailleurs, dans les colonies de ces trois nations, le gouvernement ecclésiastique est extrêmement oppressif. Dans toutes la dîme est établie, et dans les colonies d’Espagne et de Portugal on la lève avec la der­niè­re rigueur. Elles sont, en outre, surchargées d’une foule immense de moines men­diants, pour lesquels l’état de mendicité est une chose non-seulement autorisée, mais même consacrée par la religion ; ce qui établit un impôt excessivement lourd sur la classe pauvre du peuple, à laquelle on a grand soin d’enseigner que c’est un devoir que de faire l’aumône à ces moines, et un très-grand péché de la leur refuser. Par-dessus tout cela encore, dans toutes ces colonies, les plus grosses propriétés sont réunies dans les mains du clergé.
Quatrièmement, pour la manière de disposer de leur produit surabondant ou de ce qui excède leur propre consommation, les colonies anglaises ont été plus favorisées et ont toujours joui d’un marché plus étendu que n’ont fait celles de toutes les autres nations de l’Europe. Chaque nation de l’Europe a cherché plus ou moins à se donner le monopole du commerce de ses colonies, et par cette raison elle a empêché les vais­seaux étrangers de commercer avec elle, et leur a interdit l’importation des marchan­dises d’Europe d’aucune nation étrangère ; mais la manière dont ce monopole a été exercé par les diverses nations a été très-différente.
Quelques nations ont abandonné tout le commerce de leurs colonies à une compa­gnie exclusive, obligeant les colons à lui acheter toutes les marchandises d’Europe dont ils pouvaient avoir besoin, et à lui vendre la totalité de leur produit surabondant. L’intérêt de la compagnie a donc été non-seulement de vendre les unes le plus cher possible, et d’acheter l’autre au plus bas possible, mais encore de n’acheter de celui-ci, même à ce bas prix, que la quantité seulement dont elle pouvait espérer de disposer en Europe à un très-haut prix : son intérêt a été non-seulement de dégrader, dans tous les cas, la valeur du produit surabondant des colons, mais encore, dans la plupart des circonstances, de décourager l’accroissement de cette quantité, et de la tenir au-dessous de son état naturel. De tous les expédients dont on puisse s’aviser pour com­primer les progrès de la croissance naturelle d’une nouvelle colonie, le plus efficace, sans aucun doute, c’est celui d’une compagnie exclusive. C’est cependant là la politique qu’a adoptée la Hollande, quoique dans le cours de ce siècle sa compagnie ait abandonné, à beaucoup d’égards, l’exercice de son privilège exclusif. Ce fut aussi la politique du Danemark jusqu’au règne du feu roi. Accidentellement aussi, ce fut celle de la France, et récemment, depuis 1755, après que cette politique eut été aban­donnée par toutes les autres nations, à cause de son absurdité, elle a été adoptée par le Portugal, au moins à l’égard de deux des principales provinces du Brésil, celles de Fernambouc et de Maragnan.
D’autres nations, sans ériger de compagnie exclusive, ont restreint tout le com­merce de leurs colonies à un seul port de la mère patrie, duquel il n’était permis à aucun vaisseau de mettre à la voile, sinon à une époque déterminée, et de conserve avec plusieurs autres, ou bien, s’il partait seul, qu’en vertu seulement d’une permission spéciale, pour laquelle le plus souvent il fallait payer fort cher. Cette mesure politique ouvrait, à la vérité, le commerce des colonies à tous les natifs de la mère patrie, pourvu qu’ils s’astreignissent à commercer du port indiqué, à l’époque permise et dans les vaisseaux permis. Mais, comme tous les différents marchands qui associèrent leurs capitaux pour expédier ces vaisseaux privilégiés durent trouver leur intérêt à agir de concert, le commerce qui se fit de cette manière fut nécessairement conduit sur les mêmes principes que celui d’une compagnie exclusive ; le profit de ces mar­chands fut presque aussi exorbitant et fondé sur une oppression à peu près pareille ; les colonies furent mal pourvues, et se virent obligées à la fois de vendre à très-bon marché et d’acheter fort cher. Cette politique avait pourtant toujours été suivie par l’Espagne, et elle l’était encore il y a peu d’années ; aussi dit-on que toutes les mar­chandises d’Europe étaient à un prix énorme aux Indes occidentales espagnoles. Ulloa rapporte qu’à Quito une livre de fer se vendait environ de 4 à 6 deniers sterling, et une livre d’acier environ de 6 à 9 : or, c’est principalement pour se procurer les mar­chandises d’Europe que les colonies se défont de leur produit surabondant. Par conséquent, plus elles payent pour les premières, moins elles retirent réellement pour le denier, et la cherté des unes est absolument la même chose, pour elles, que le bas prix de l’autre. Le système qu’a suivi le Portugal à l’égard de toutes ses colonies, excepté celles de Fernambouc et de Maragnan, est, sous ce rapport, le même que suivait anciennement l’Espagne ; et quant à ces deux dernières provinces, le Portugal a adopté des mesures encore bien plus mauvaises.
D’autres nations laissent le commerce de leurs colonies libre à tous leurs sujets, lesquels peuvent le faire de tous les différents ports de la mère patrie, et n’ont besoin d’autre permission que des formalités ordinaires de la douane. Dans ce cas, le nombre et la position des différents commerçants répandus dans toutes les parties du pays les met dans l’impossibilité de former entre eux une ligue générale, et la concurrence suffit pour les empêcher de faire des profits exorbitants. Au moyen d’une politique aussi franche, les colonies sont à même de vendre leurs produits, ainsi que d’acheter les marchandises de l’Europe, à des prix raisonnables. Or, depuis la dissolution de la compagnie de Plymouth, arrivée à une époque où nos colonies n’étaient encore que dans leur enfance, cette politique a toujours été celle de l’Angleterre ; elle a été aussi, en général, celle de la France, et c’est le système qu’a suivi constamment celui-ci depuis la dissolution de ce que nous appelons communément la Compagnie française du Mississipi. Aussi, les profits du commerce que font la France et l’Angleterre avec leurs colonies ne sont-ils pas du tout exorbitants, quoique sans doute un peu plus forts que si la concurrence était libre à toutes les autres nations ; et le prix des marchandises de l’Europe, dans la plupart des colonies de ces deux nations, ne monte pas non plus à un taux excessif.
D’ailleurs, ce n’est qu’à l’égard seulement de certaines marchandises que les colo­nies de la Grande-Bretagne sont bornées au marché de la mère patrie pour l’exporta­tion de leur produit surabondant. Ces marchandises, ayant été détaillées dans l’acte de navigation[5] et dans quelques autres actes subséquents, ont, par cette raison, été nommées marchandises énumérées ; les autres s’appellent non énumérées, et peuvent s’exporter directement aux autres pays, pourvu que ce soit sur des vaisseaux de la Grande-Bretagne ou des colonies, dont les maîtres et les trois quarts de l’équipage soient sujets de l’empire britannique.
Parmi les marchandises non énumérées se trouvent quelques-unes des productions les plus importantes de l’Amérique et des Indes Occidentales les grains de toute espèce, les planches, merrains et bois équarris[6], les viandes salées, le poisson, le sucre et le rhum.
Le grain est naturellement le premier et le principal objet de culture dans les colonies nouvelles. En leur faisant un marché très-étendu pour cette denrée, la loi les encourage à étendre la culture beaucoup au-delà de ce qu’exige la consommation d’un pays faiblement peuplé, et les met à même de préparer ainsi de longue main une ample subsistance pour une population toujours croissante.
Dans un pays tout couvert de forêts où, par conséquent, le bois n’a que peu ou point de valeur, les frais de défrichement du sol sont le principal obstacle à l’amélio­ration de la terre. La loi qui laisse aux colonies un marché très-étendu pour leurs planches, merrains et bois équarris, tend à faciliter la mise en valeur des terres, en élevant le prix d’une denrée qui serait sans cela de peu de valeur, et en mettant des colons dans le cas de tirer profit de ce qui autrement serait un pur objet de dépense.
Dans un pays qui n’est pas même à moitié peuplé ni à moitié cultivé, les bestiaux multiplient naturellement au-delà de la consommation des habitants, et n’ont souvent, par cette raison, que peu ou point de valeur. Or, il est nécessaire, comme on l’a déjà fait voir[7], que le prix du bétail se trouve dans une certaine portion avec celui du blé, avant qu’on puisse mettre en valeur la plus grande partie des terres d’un pays. En laissant un marché très-étendu aux bestiaux américains, sous toutes les formes, morts et vifs, la loi tend à faire monter la valeur d’une denrée dont le haut prix est si essentiel aux progrès de la culture. Néanmoins, les bons effets de cette liberté ont dû être un peu diminués par le statut de la quatrième année de Georges III, chap. xv, qui met les peaux et les cuirs au nombre des marchandises énumérées, et contribue par là à rabaisser la valeur du bétail américain.
L’accroissement de la puissance navale de la Grande-Bretagne et de sa marine, par l’extension de la pêche de nos colonies, est un objet que la législature semble avoir eu presque constamment en vue. Par cette raison, la pêche a eu tout l’encouragement que lui peut donner la liberté, et aussi a-t-elle été très-florissante. La pêche de la Nouvelle-Angleterre en particulier était, avant les derniers troubles, une des plus importantes peut-être qui fût au monde. La pêche de la baleine, qui, en Grande-Bretagne, malgré une prime énorme, est un objet de si peu d’importance, que, suivant l’opinion de beaucoup de gens (opinion que je ne prétends pourtant pas garantir), tout le produit n’excède guère la valeur des primes payées annuellement pour elle, est un objet de commerce extrêmement étendu dans la Nouvelle-Angleterre, sans l’aide d’aucune prime. Le poisson est un des principaux articles du commerce que les Anglais-Américains font avec l’Espagne, le Portugal et la Méditerranée.
Le sucre était, dans l’origine, une marchandise énumérée qui ne pouvait s’exporter qu’à la Grande-Bretagne. Mais, en 1731, sur une réclamation des planteurs, on en permit l’exportation à toutes les parties du monde. Toutefois, les restrictions avec lesquelles cette liberté a été accordée, jointes au haut prix du sucre en Angleterre, l’ont rendue en grande partie sans effet. La Grande-Bretagne et ses colonies conti­nuent toujours d’être presque le seul marché pour tout le sucre que produisent les plantations anglaises. Leur consommation croît si rapidement, que, quoique l’impor­tation du sucre ait extrêmement augmenté depuis vingt ans, en conséquence des progrès toujours continus de la culture à la Jamaïque, aussi bien que dans les îles cédées[8], on assure néanmoins que l’exportation aux pays étrangers n’en est pas pour cela beaucoup plus forte qu’auparavant.
Le rhum est un article très-important du commerce que les Américains font à la côte d’Afrique, d’où ils ramènent en retour des esclaves noirs.
Si le produit surabondant de l’Amérique en grains de toute espèce, en viandes salées et en poisson, eût été compris dans les marchandises énumérées, et énumérées, et qu’il eût été par là contraint de venir en totalité au marché de la Grande-Bretagne, il aurait pu exercer une trop grande influence sur la valeur de notre propre produit. Ce fut vrai­semblablement bien moins par intérêt pour l’Amérique que par la crainte de cette fâcheuse concurrence, que non-seulement ces marchandises importantes ont été affranchies de l’énumération, mais que même, dans l’état ordinaire de la loi, il y a prohibition d’importer en Grande-Bretagne toute espèce de grain, à l’exception du riz, ainsi que les viandes salées.
Dans l’origine, les marchandises non énumérées pouvaient s’exporter dans tous les lieux du monde. Les planches, merrains et bois équarris, ainsi que le riz, compris d’abord dans l’énumération, lorsque par la suite ils en furent affranchis, furent restreints, quant au marché de l’Europe, aux pays situés au sud du cap Finistère. Par le statut de la sixième année de Georges III, chap. lii, toutes les marchandises non énumérées furent assujetties à la même restriction. Les contrées de l’Europe situées au sud du cap Finistère ne sont pas des pays manufacturiers, et notre politique jalouse a peu à craindre que les vaisseaux de nos colonies rapportent de ces pays des ouvrages manufacturés qui puissent nuire au débit des nôtres.
Les marchandises énumérées sont de deux sortes : la première comprend celles qui sont un produit particulier à l’Amérique, ou bien qui ne peuvent être ou, au moins, ne sont pas produites dans la mère patrie. De cette classe sont les mélasses, le café, les noix de cacao, le tabac, le piment, le gingembre, les fanons de baleine, la soie écrue, le coton en laine, le castor et les autres pelleteries d’Amérique, l’indigo, le fustet[9] et autres bois de teinture. La seconde sorte comprend celles qui ne sont pas un produit particulier à l’Amérique, mais qui sont ou peuvent être produites dans la mère patrie, quoique cependant pas en assez grandes quantités pour fournir à la demande, laquelle est remplie principalement par l’étranger. De cette espèce sont les munitions navales, les mâts, vergues et beauprés ; le brai, le goudron et la térébenthine ; le fer en saumon[10] et en barres ; le cuivre brut, les peaux et cuirs non travaillés, la potasse et la perlasse. Les importations les plus abondantes des denrées de la première sorte ne pourraient ni décourager la production, ni nuire à la vente d’aucune partie du produit de la mère patrie. En les bornant à notre seul marché, on espéra que non-seulement nos marchands seraient par là à même de les acheter à meilleur compte dans les colonies et, par conséquent, de les revendre chez nous avec un plus gros profit, mais encore qu’il s’établirait entre nos colonies et les pays étrangers un commerce de transport très-avantageux dont la Grande-Bretagne serait nécessairement le centre ou l’entrepôt, comme étant la pays de l’Europe où ces marchandises seraient importés en premier heu. L’importation des marchandises de la seconde sorte pourrait aussi, à ce qu’on a supposé, être dirigée de manière à ne pas nuire à la vente des marchandises de même espèce produites chez nous, mais bien à la vente de celles importées de l’étranger, parce qu’au moyen de droits bien combinés, on pourrait toujours les rendre un peu plus chères que les nôtres, et néanmoins à bien meilleur marché que celles des étrangers. Ainsi, en restreignant ces marchandises à notre seul marché, on se proposa de décourager, non le produit de la Grande-Bretagne, mais bien celui de quelques pays étrangers avec lesquels on imaginait que la balance du commerce était défavo­ra­ble à la Grande-Bretagne.
La prohibition d’exporter des colonies à tout autre pays que la Grande-Bretagne les mâts, vergues et beauprés, le brai, le goudron et la térébenthine, tendait naturelle­ment à faire baisser dans les colonies le prix du bois de marine et, par conséquent, à augmenter les dépenses du défrichement des terres, le principal obstacle à leur mise en valeur. Mais, vers le commencement de ce siècle, en 1703, la Compagnie suédoise pour le commerce du goudron tâcha de faire hausser le prix de ses marchandises en Angleterre, en en prohibant l’exportation autrement que sur les propres vaisseaux de la Compagnie, au prix par elle fixé, et en telles quantités qu’elle jugerait à propos. Pour riposter à ce tour remarquable de politique mercantile, et se rendre indépen­dante, autant que possible, non-seulement de la Suède, mais de toutes les autres puis­san­ces du Nord, la Grande-Bretagne accorda une prime sur l’importation des muni­tions navales d’Amérique. L’effet de cette prime fut de faire monter en Amérique le prix du bois de marine beaucoup plus que ne pouvait l’abaisser sa limitation au marché de la Grande-Bretagne ; et comme les deux règlements furent portés à la même époque, leur effet réuni tendit plutôt à encourager qu’à décourager le défri­che­ment des terres en Amérique.
Quoique le fer en saumons et en barres ait été mis en nombre des marchandises énumérées, cependant, comme il est exempt, à son importation d’Amérique, des droits considérables auxquels il est assujetti à son importation de tout autre pays, une partie du règlement contribue plus à encourager des établissements de forges en Amérique, que l’autre partie ne contribue à les décourager. Or, il n’y a pas de manu­facture qui occasionne une aussi grande consommation de bois qu’une forge, ou qui puisse contribuer davantage au défrichement d’un pays surchargé de forêts[11]. 
La législature n’a peut-être pas eu en vue ni même compris la tendance de quelques-uns de ces règlements à élever la valeur des différentes sortes de bois en Amérique, et par là à faciliter le défrichement des terres. Si, sous ce rapport, leurs effets bienfaisants ont été accidentels, ils n’en ont pas été pour cela moins réels.
La plus parfaite liberté a été laissée au commerce qui se fait entre les colonies anglaises-américaines et les Indes occidentales, pour les marchandises énumérées, comme pour les autres. Ces colonies sont devenues aujourd’hui si peuplées et si florissantes, que chacune d’elles peut trouver dans quelques-unes des autres un vaste marché pour toutes les diverses parties de son produit. Toutes ces colonies prises ensemble forment un immense marché intérieur pour leurs divers produits respectifs. La libéralité de l’Angleterre envers le commerce de ses colonies s’est bornée princi­palement à leur donner un marché pour leur produit dans son état brut seulement, ou tout au plus dans ce qu’on peut appeler précisément le premier degré de main-d’œuvre. Quant aux ouvrages manufacturés, ou plus avancés, ou plus raffinés, même tirés du produit des colonies, les marchands et manufacturiers de la Grande-Bretagne ont mieux aimé se les réserver ; et ils ont eu assez de crédit sur la législature pour empêcher, tantôt par des droits élevés, tantôt par des prohibitions absolues, l’établissement de ces sortes de fabriques dans les colonies.
Par exemple, tandis que le sucre moscouade des colonies anglaises ne paye à l’importation que 6 schellings 4 deniers le quintal, le sucre blanc paye 1 liv. 1 schelling 1 denier ; et quand il est raffiné double ou simple, en pains, il paye 4 liv. 2 schellings 5 deniers 8 dixièmes. Lorsque ces droits énormes furent établis, la Grande-Bretagne était le seul, et elle est encore aujourd’hui le principal marché sur lequel puisse être exporté le sucre de ses colonies. Ces droits équivalaient donc à une prohi­bition, d’abord de terrer ou raffiner le sucre pour tout marché étranger quelconque, et ensuite d’en terrer ou raffiner pour le marché qui exporte peut-être à lui seul plus des neuf dixièmes du produit total. Aussi, les fabriques pour terrer ou raffiner le sucre, qui ont été très-florissantes dans toutes les colonies françaises, n’ont guère été en activité, dans celles de l’Angleterre, que pour le marché des colonies elles-mêmes. Lorsque la Grenade était entre les mains des Français, il y avait, presque sur chaque plantation, une raffinerie pour terrer au moins le sucre. Depuis que cette île est tombée entre les mains des Anglais, presque tous les travaux de ce genre ont été abandonnés ; et à présent (octobre 1773), il ne reste pas, à ce qu’on m’a assuré, plus de deux ou trois de ces fabriques dans toute l’île. Cependant actuellement, par une indulgence de la douane, le sucre terré ou raffiné, quand il est importé en poudre au lieu de l’être en pains, entre communément comme moscouade.
Tandis que la Grande-Bretagne encourage en Amérique la fabrication du fer en saumons et en barres, en exceptant ces marchandises des droits auxquels elles sont assujetties à leur importation de tout autre pays, elle établit en même temps une prohibition absolue d’élever des forges et fourneaux pour faire l’acier, ni des moulins de fonderie, dans aucune de ses colonies d’Amérique. Elle n’entend pas permettre que ses colons aillent plus loin dans ce genre d’industrie, même pour leur propre consom­mation ; mais elle tient obstinément à ce qu’ils achètent de ses marchands et manufac­turiers toutes les marchandises de cette sorte dont ils peuvent avoir besoin.
Elle prohibe l’exportation d’une province à l’autre, par eau, et même le transport par terre, en chariot ou à dos de cheval, des chapeaux, des laines et lainages du produit de l’Amérique ; règlement qui a l’effet nécessaire d’empêcher l’établissement d’aucune manufacture de ces sortes de marchandises pour la vente au loin, et qui limite l’industrie de ses colons, dans ce genre, aux seuls ouvrages grossiers et de ménage, tels qu’une famille particulière peut les faire pour son usage personnel ou pour celui de quelques-uns de ses voisins dans la même province.
Cependant, empêcher un grand peuple de tirer tout le parti qu’il peut de chacune de ses propres productions, ou d’employer ses capitaux et son industrie de la manière qu’il croit lui être la plus avantageuse, c’est une violation manifeste des droits les plus sacrés des hommes. Tout injustes néanmoins que puissent être ces prohibitions, elles n’ont pas été jusqu’à présent très-nuisibles aux colonies ; la terre y est toujours à si bon marché et le travail, par conséquent, y est si cher, que les colons peuvent importer de la mère patrie presque tous les ouvrages de fabrique les plus raffinés et les plus avancés en main-d’œuvre, à meilleur compte qu’ils ne pourraient les établir par eux-mêmes. Ainsi, quand même ils n’auraient pas éprouvé de prohibition relativement à l’établissement de ces sortes de fabriques, vraisemblablement encore, dans l’état actuel de leurs progrès et de leur culture, ils en auraient été détournés par pure consi­dé­ration pour leur intérêt personnel. Peut-être, dans l’état actuel où est l’amélioration de leur société, ces prohibitions, sans enchaîner leur industrie ou sans la repousser d’aucun emploi où elle se fût portée d’elle-même, n’agissent-elles seule­ment que comme des marques injustes et odieuses de servitude, imprimées sur eux sans nulle espèce de raison et simplement par suite de l’absurde jalousie des marchands et manu­facturiers de la mère patrie ; dans un état d’amélioration plus avancé, elles pourraient être réellement oppressives et insupportables.
Si la Grande-Bretagne borne à son seul marché quelques-unes des productions les plus importantes des colonies, aussi, en compensation, elle donne à quelques-unes de ces productions un avantage sur ce marché, tantôt en imposant des droits plus forts sur les productions pareilles qui sont importées d’autres pays, et tantôt en accordant des primes à leur importation des colonies. C’est de la première de ces deux manières qu’elle donne sur son marché un avantage au sucre, au tabac et au fer de ses colonies ; et c’est de la seconde manière qu’elle favorise leur soie écrue, leur lin et leur chanvre, leur indigo, leurs munitions navales et leurs bois de construction. Cette seconde manière d’encourager les productions de la colonie par des primes à l’importation est, autant que j’ai pu m’en assurer, particulière à la Grande-Bretagne ; la première ne l’est pas. Le Portugal ne s’est pas contenté d’imposer des droits plus élevés sur l’impor­tation du tabac de tout autre pays, mais il l’a prohibée sous les peines les plus rigoureuses.
Pour ce qui regarde l’importation des marchandises d’Europe, l’Angleterre a pa­reil­le­ment agi avec ses colonies d’une manière beaucoup plus généreuse que n’a fait toute autre nation.
La Grande-Bretagne accorde, sur les droits payés à l’importation des marchan­dises étrangères, la restitution d’une partie, presque toujours de la moitié, en général d’une plus forte portion, et quelquefois même de la totalité du droit, lorsque ces mêmes marchandises sont exportées en quelque pays étranger. Il n’était pas difficile de prévoir qu’aucun pays étranger indépendant ne les recevrait, si elles lui arrivaient chargées des droits énormes auxquels presque toutes les marchandises étrangères sont assujetties à leur importation dans la Grande-Bretagne. Par conséquent, à moins qu’une partie de ces droits ne fût rendue lors de l’exportation, c’en était fait du com­merce de transport, commerce si favorisé par le système mercantile.
Mais nos colonies ne sont nullement des pays étrangers indépendants, et la Grande-Bretagne s’étant emparée du droit exclusif de leur fournir toutes les mar­chandises d’Europe, elle eût pu les obliger, comme les autres nations ont fait à l’égard de leurs colonies, à recevoir ces marchandises, avec la charge de tous les droits qu’elles payent dans la mère, patrie. Au contraire, jusqu’en 1763 on accorda, sur l’exportation de la plupart des marchandises étrangères à nos colonies, les mêmes drawbacks que sur l’exportation à tout pays étranger indépendant. À la vérité, en 1763, par le statut de la quatrième année de Georges III, chap. xv, on rabattit beau­coup de cette indulgence, et il fut statué : « Qu’aucune partie du droit dit l’ancien subsi­de ne serait restituée pour toutes marchandises d’origine, production ou fabrique de l’Europe ou des Indes orientales, qui seraient exportées de ce royaume à quelque colonie de la Grande-Bretagne ou plantation américaine, à l’exception des vins, des toiles de coton blanches et des mousselines. » Avant cette loi, plusieurs espèces de marchandises étrangères auraient pu être achetées à meilleur marché dans nos colo­nies que dans la mère patrie, et quelques-unes peuvent l’être encore.
Il faut observer que ce sont les marchands qui font le commerce avec les colonies, dont les avis ont principalement contribué à la création des règlements relatifs à ce commerce. Il ne faut donc pas s’étonner si, dans la plupart de ces règlements, on a eu plus d’égard à leur intérêt qu’à celui des colonies ou à celui de la mère patrie. En donnant à ces marchands le privilège exclusif de fournir aux colonie ! ; toutes les marchandises d’Europe dont elles ont besoin, et d’acheter, dans le produit superflu des colonies, tout ce qui n’est pas de nature à nuire à quelqu’un des trafics qu’ils font chez eux, l’intérêt des colonies a été sacrifié à l’intérêt de ces marchands. Quand on a accordé, sur la réexportation de la plupart des marchandises d’Europe et des Indes aux colonies, les mêmes restitutions de droit que sur la réexportation de ces marchandises dans tout autre pays étranger indépendant, en cela c’est l’intérêt de la mère patrie qui lui a été sacrifié, mais suivant les idées que le système mercantile se forme de cet intérêt. Ce furent les marchands qui eurent intérêt à payer le moins possible les marchandises étrangères qu’ils envoyaient aux colonies et, par conséquent, à retirer le plus possible des droits par eux avancés lors de l’importation de ces marchandises dans la Grande-Bretagne. Ils se trouvèrent par là à même de vendre dans les colonies, ou la même quantité de marchandises avec un plus gros profit, ou bien une plus grande quantité de marchandises avec le même profit et, par conséquent, de gagner quelque chose d’une façon ou de l’autre. C’était également l’intérêt des colonies de se procurer toutes ces marchandises au meilleur compte et dans la plus grande abon­dance possible ; mais cela pouvait n’être pas toujours l’intérêt de la mère patrie. Elle pouvait souvent en souffrir pour son revenu, en rendant ainsi une grande partie des droits qui avaient été perçus à l’importation de ces marchandises, et en souffrir pour ses manufactures dont les produits étaient supplantés sur le marché de la colonie, à cause de la facilité des conditions auxquelles, au moyen de ces restitutions de droits, on pouvait y porter les produits des fabriques étrangères. On croit communément que les drawbacks sur la réexportation des toiles d’Allemagne aux colonies d’Amérique ont retardé les progrès des manufactures de toiles dans la Grande-Bretagne.
Mais quoique la politique de la Grande-Bretagne, à l’égard du commerce de ses colonies, ait été dictée par le même esprit mercantile que celle des autres nations, toutefois elle a été au total moins étroite et moins oppressive que celle d’aucune autre nation.
Quant à la faculté de diriger leurs affaires comme ils le jugent à propos, les colons anglais jouissent d’une entière liberté sur tous les points, à l’exception de leur commerce étranger. Leur liberté est égale, à tous égards, à celle de leurs concitoyens de la mère patrie, et elle est garantie de la même manière par une assemblée de représentants du peuple, qui prétend au droit exclusif d’établir des impôts pour le soutien du gouvernement colonial. L’autorité de cette assemblée tient en respect le pouvoir exécutif, et le dernier colon, le plus suspect même, tant qu’il obéit à la loi, n’a pas la moindre chose à craindre du ressentiment du gouverneur ou de celui de tout autre officier civil ou militaire de la province. Si les assemblées coloniales, de même que la Chambre des communes en Angleterre, ne sont pas toujours une représentation très-légale du peuple, cependant elles approchent de plus près qu’elle de ce caractère ; et comme le pouvoir exécutif ou n’a pas de moyens de les corrompre, ou n’est pas dans la nécessité de le faire, à cause de l’appui que lui donne la mère patrie, elles sont peut-être, en général, plus sous l’influence de l’opinion et de la volonté de leurs com­mettants. Les conseils qui, dans les législatures coloniales, répondent à la Chambre des pairs dans la Grande-Bretagne, ne sont pas composés d’une noblesse héréditaire. En certaines colonies, comme dans trois des gouvernements de la Nouvelle-Angleterre, ces conseils ne sont pas nommés par le roi, mais ils sont élus par les représentants du peuple. Dans aucune des colonies anglaises, il n’y a de noblesse héréditaire. Dans toutes, à la vérité, comme dans tout autre pays libre, un citoyen issu d’une ancienne famille de la colonie est, à égalité de mérite et de fortune, plus considéré qu’un parvenu ; mais son privilège se borne à être plus considéré, et il n’en a aucun qui puisse être importun à ses voisins. Avant le commencement des troubles actuels, les assemblées coloniales avaient non-seulement la puissance législative, mais même une partie du pouvoir exécutif. Dans les provinces de Connecticut et de Rhode-Island, elles élisaient le gouverneur. Dans les autres colonies, elles nommaient les officiers de finances qui levaient les taxes établies par ces assemblées respectives, devant lesquelles ces officiers étaient immédiatement responsables. Il y a donc plus d’égalité parmi les colons anglais que parmi les habitants de la mère patrie. Leurs mœurs sont plus républicaines, et leurs gouvernements, particulièrement ceux de trois des provinces de la Nouvelle-Angleterre[12], ont aussi jusqu’à présent été plus répu­blicains.
Au contraire, la forme absolue du gouvernement qui domine en Espagne, en Portugal et en France, s’étend à leurs colonies, et les pouvoirs arbitraires que ces sor­tes de gouvernements délèguent, en général, à tous les agents subalternes, s’exercent naturellement avec plus de violence dans des pays qui se trouvent placés à une aussi grande distance. Dans tous les gouvernements absolus, il y a plus de liberté dans la capitale que dans tout autre endroit de l’empire. Le souverain, personnellement, ne peut jamais avoir d’intérêt ou de penchant à intervertir l’ordre de la justice ou à opprimer la masse du peuple. Dans la capitale, sa présence tient plus ou moins en respect tous ses officiers subalternes, qui, dans des provinces plus éloignées de lui, où les plaintes du peuple sont moins à portée de frapper ses oreilles, peuvent se livrer avec beaucoup plus d’assurance aux excès de leur esprit tyrannique. Or, les colonies européennes de l’Amérique sont à une distance bien plus grande de leur capitale, que les provinces les plus reculées des plus vastes empires qui aient jamais été connus au monde jusqu’à présent. Le gouvernement des colonies anglaises est peut-être le seul, depuis l’origine des siècles, qui ait donné à des provinces aussi éloignées une sécurité parfaite. Toutefois, l’administration des colonies françaises a été conduite avec plus de modération et de douceur que celle des colonies espagnoles et portugaises. Cette supériorité dans la conduite de l’administration est conforme, à la fois, au caractère de la nation française et à ce qui forme le caractère d’une nation, c’est-à-dire à son gouvernement. Or, le gouvernement de France, bien qu’en comparaison de celui de la Grande-Bretagne il puisse passer pour violent et arbitraire, est néanmoins un gou­vernement légal et libre, si on le compare à ceux d’Espagne et de Portugal. 
C’est principalement dans les progrès des colonies de l’Amérique septentrionale que se font remarquer les avantages du système politique de l’Angleterre. Le progrès des îles à sucre de la France a été au moins égal, peut-être même supérieur à celui de la plupart des îles à sucre de l’Angleterre, et celles-ci cependant jouissent d’un gouvernement libre, de même nature à peu près que celui qui existe dans les colonies anglaises de l’Amérique septentrionale. Mais on n’a pas, dans les îles à sucre de la France, découragé la raffinerie de leurs produits, comme on l’a fait dans celles de l’Angleterre ; et ce qui est encore d’une bien plus grande importance, la nature du gouvernement des îles françaises y amène naturellement un meilleur régime à l’égard des nègres esclaves.
Dans toutes les colonies européennes, la culture de la canne à sucre se fait par des esclaves noirs. On suppose que la constitution des hommes nés dans le climat tempéré de l’Europe ne pourrait pas supporter la fatigue de remuer la terre sous le ciel brûlant des Indes occidentales ; et la culture de la canne à sucre, telle qu’elle est dirigée à présent, est tout entière un travail de main, quoique, dans l’opinion de beau­coup de monde, on pourrait y introduire, avec de grands avantages, l’usage de la charrue. Or, de même que le profit et le succès d’une culture qui se fait au moyen de bestiaux dépend extrêmement de l’attention qu’on a de les bien traiter et de les bien soigner, de même, le produit et le succès d’une culture qui se fait au moyen d’esclaves doit dépendre également de l’attention qu’on apporte à bien les traiter et à les bien soigner ; et du côté des bons traitements envers leurs esclaves, c’est une chose, je crois, généralement reconnue, que les planteurs français l’emportent sur les Anglais. La loi, en tant qu’elle peut donner à l’esclave quelque faible protection contre la violence du maître, sera mieux exécutée dans une colonie où le gouvernement est en grande partie arbitraire, que dans une autre où il est totalement libre. Dans un pays où est établie la malheureuse loi de J’esclavage, quand le magistrat veut protéger l’esclave, il s’immisce jusqu’à un certain point dans le régime de la propriété privée du maître ; et dans un pays libre, où le maître est peut-être un membre de l’assemblée coloniale ou un électeur des membres de cette assemblée, il n’osera le faire qu’avec la plus grande réserve et la plus grande circonspection. La considération et les égards auxquels il est tenu envers le maître rendent plus difficile pour lui la protection de l’esclave. Mais dans un pays où le gouvernement est en grande partie arbitraire, où il est ordinaire que le magistrat intervienne dans le régime même des propriétés parti­culières des individus, et leur envoie peut-être une lettre de cachet s’ils ne se conduisent pas, à cet égard, selon son bon plaisir, il est bien plus aisé pour lui de donner à l’esclave quelque protection, et naturellement la simple humanité le dispose à le faire. La protection du magistrat rend l’esclave moins méprisable aux yeux de son maître, et engage celui-ci à garder un peu plus de mesure dans sa conduite envers l’autre, et à le traiter avec plus de douceur. Les bons traitements rendent l’esclave non-seulement plus fidèle, mais plus intelligent et, par conséquent, plus utile ; sous ce double rapport il se rapproche davantage de la condition d’un domestique libre, et il peut devenir susceptible de quelque degré de probité et d’attachement aux intérêts de son maître, vertus qu’on rencontre souvent chez les domestiques libres, mais qu’on ne doit jamais s’attendre à trouver chez un esclave, quand il est traité comme le sont communément les esclaves dans les pays où le maître est tout à fait libre et indépendant.
L’histoire de tous les temps et de tous les peuples viendra, je crois, à l’appui de cette vérité, que le sort d’un esclave est moins dur dans les gouvernements arbitraires que dans les gouvernements libres[13] Dans l’histoire romaine, la première fois que nous voyons le magistrat interposer son autorité pour protéger l’esclave contre les violences du maître, c’est sous les empereurs. Lorsque Védius Pollion, en présence d’Auguste, ordonna qu’un de ses esclaves qui avait commis quelque légère faute fût coupé par morceaux et jeté dans un vivier pour servir de pâture à ses poissons, l’em­pe­reur, indigné, lui commanda d’affranchir immédiatement, non-seulement cet escla­ve, mais tous les autres qui lui appartenaient. Sous la république, aucun magistrat n’eût eu assez d’autorité pour protéger l’esclave, encore bien moins pour punir le maître.
Il est à remarquer que le capital qui a servi à améliorer les colonies à sucre de la France, et en particulier la grande colonie de Saint-Domingue, est provenu, presque en totalité, de la culture et de l’amélioration successive de ces colonies. Il a été presque en entier le produit du sol et de l’industrie des colons, ou, ce qui revient au même, le prix de ce produit graduellement accumulé par une sage économie, et employé à faire naître toujours un nouveau surcroît de produit. Mais le capital qui a servi à cultiver et à améliorer les colonies à sucre de l’Angleterre a été en grande partie envoyé d’Angleterre, et ne peut nullement être regardé comme le produit seul du territoire et de l’industrie des colons. La prospérité des colonies à sucre de l’Angleterre a été, en grande partie, l’effet des immenses richesses de l’Angleterre, dont une partie, débordant pour ainsi dire de ce pays, a reflué dans les colonies ; mais la prospérité des colonies à sucre de la France est entièrement l’œuvre de la bonne conduite des colons, qui doit, par conséquent, l’avoir emporté de quelque chose sur celle des colons anglais ; et cette supériorité de bonne conduite s’est par-dessus tout fait remarquer dans leur manière de traiter les esclaves.
Tel est, en raccourci, le tableau général de la politique suivie par les différentes nations de l’Europe, relativement à leurs colonies.
La politique de l’Europe n’a donc pas trop lieu de se glorifier, soit de l’établisse­ment primitif des colonies de l’Amérique, soit de leur prospérité ultérieure, en ce qui regarde le gouvernement intérieur qu’elle leur a donné.
L’extravagance et l’injustice sont, à ce qu’il semble, les principes qui ont conçu et dirigé le premier projet de l’établissement de ces colonies ; l’extravagance qui faisait courir après des mines d’or et d’argent, et l’injustice qui faisait convoiter la possession d’un pays dont les innocents et simples habitants, bien loin d’avoir fait aucun mal aux Européens, les avaient accueillis avec tous les témoignages possibles de bonté et d’hospitalité, quand ils avaient paru pour la première fois dans cette partie du monde.
À la vérité, les aventuriers qui ont formé quelques-uns des derniers établissements ont joint au projet chimérique de découvrir des mines d’or et d’argent d’autres motifs plus raisonnables et plus louables ; mais ces motifs mêmes font encore très-peu d’honneur à la politique de l’Europe.
Les puritains anglais, opprimés dans leur patrie, s’enfuirent en Amérique pour y trouver la liberté, et ils y établirent les quatre gouvernements de la Nouvelle-Angle­terre. Les catholiques anglais, traités avec encore bien plus d’injustice, fondèrent celui de Maryland ; les quakers, celui de Pensylvanie. Les juifs portugais, persécutés par l’Inquisition, dépouillés de leur fortune et bannis au Brésil, introduisirent, par leur exemple, quelque espèce d’ordre et d’industrie parmi les brigands déportés et les prostituées dont la colonie avait été peuplée originairement, et ils leur enseignèrent la culture de la canne à sucre. Dans toutes ces différentes circonstances, ce ne fut pas par leur sagesse et leur politique, mais bien par leurs désordres et leurs injustices que les gouvernements de l’Europe contribuèrent à la population et à la culture de l’Amérique.
Les divers gouvernements de l’Europe ne peuvent pas plus prétendre au mérite d’avoir donné naissance à quelques-uns des plus importants de ces établissements, qu’à celui d’en avoir conçu le dessein.
La conquête du Mexique ne fut pas un projet imaginé par le conseil d’Espagne, mais par un gouverneur de Cuba ; et ce projet fut mis à exécution par le génie hardi et entreprenant de l’aventurier qui en fut chargé, en dépit de tout ce que put faire pour le traverser ce même gouverneur, qui se repentit bientôt d’avoir confié cette entreprise à un pareil homme. Les conquérants du Chili et du Pérou, et de presque tous les autres établissements espagnols sur le continent américain, n’emportèrent avec eux d’autre encouragement de la part du gouvernement, qu’une permission générale de faire des établissements et des conquêtes au nom du roi d’Espagne. Les hasards de toutes ces entreprises étaient aux risques et aux frais personnels de ces aventuriers ; à peine le gouvernement d’Espagne contribua-t-il pour la moindre chose à aucune des dépenses. Celui d’Angleterre n’a pas fait plus de frais pour la création des établissements qui forment aujourd’hui quelques-unes de ses plus importantes colonies de l’Amérique septentrionale.
Quand ces établissements furent formés et quand ils furent devenus assez considérables pour attirer l’attention de la mère patrie, les premiers règlements qu’elle fit à leur égard eurent toujours pour objet de s’assurer le monopole de leur commerce, de resserrer leur marché, d’agrandir le sien à leurs dépens et, par conséquent, de décourager et de ralentir le cours de leur prospérité, bien loin de l’exciter et de l’accélérer. Les diverses manières dont a été exercé ce monopole sont ce qui constitue une des différences les plus essentielles entre les systèmes politiques suivis par les différentes nations de l’Europe, à l’égard de leurs colonies. Tout ce qu’on peut dire du meilleur de ces systèmes, celui de l’Angleterre, c’est qu’il est seulement un peu moins mesquin et moins oppressif qu’aucun de ceux des autres nations. 
De quelle manière la politique de l’Europe a-t-elle donc contribué soit au premier établissement, soit à la grandeur actuelle des colonies de l’Amérique ? D’une seule manière, et celle-là n’a pas laissé d’y contribuer beaucoup. Magna virum mater ! Elle a élevé, elle a formé les hommes qui ont été capables de mettre à fin de si grandes choses, de poser les fondements d’un aussi grand empire, et il n’y a pas d’autre partie du monde dont les institutions politiques soient en état de former de pareils hommes, ou du moins en aient jamais formé de pareils jusqu’à présent. Les colonies doivent à la politique de l’Europe l’éducation de leurs actifs et entreprenants fondateurs, et les grandes vues qui les ont dirigés ; et pour ce qui regarde leur gouvernement intérieur, c’est presque là tout ce que lui doivent quelques-unes des plus puissantes et des plus considérables.

Section troisième.
Des avantages qu’a retirés l’Europe de la découverte de l’Amérique, et de celle d’un passage aux Indes par le cap de Bonne-Espérance.

On a vu quels sont les avantages que les colonies de l’Amérique ont retirés de la politique de l’Europe.
Quels sont maintenant ceux que l’Europe a retirés de la découverte de l’Amérique et des colonies qui s’y sont formées ?
Ces avantages peuvent se diviser en deux classes premièrement, les avantages généraux que l’Europe, considérée comme un seul vaste pays, a retirés de ces grands événements ; et secondement, les avantages particuliers que chaque pays à colonies a retirés des colonies particulières qui lui appartiennent, en conséquence de l’autorité et de la domination qu’il exerce sur elles.
Les avantages généraux que l’Europe, considérée comme un seul grand pays, a retirés de la découverte de l’Amérique et de sa formation en colonies, consistent, en premier lieu, dans une augmentation de jouissances, et en second lieu, dans un ac­crois­sement d’industrie.
Le produit superflu de l’Amérique importé en Europe fournit aux habitants de ce vaste continent une multitude de marchandises diverses qu’ils n’auraient jamais possédées sans cela, les unes pour l’utilité et la commodité, d’autres pour l’agrément et le plaisir, d’autres enfin pour la décoration et l’ornement, et par là il contribue à aug­menter leurs jouissances. 
On conviendra sans peine que la découverte de l’Amérique et sa formation en colonies ont contribué à augmenter l’industrie, 1° de tous les pays qui commercent directement avec elle, tels que l’Espagne, le Portugal, la France et l’Angleterre ; et 2° de tous ceux qui, sans y faire de commerce directe, y envoient, par l’intermédiaire d’autres pays, des marchandises de leur propre produit, tels que la Flandre autri­chienne et quelques provinces d’Allemagne, qui y font passer une quantité consi­dérable de toiles et d’autres marchandises par l’entremise des nations qui y commercent directement. Tous ces pays ont gagné évidemment un marché plus étendu pour l’excédent de leurs produits et, par conséquent ont dû être encouragés à en augmenter la quantité.
Mais ce qui n’est peut-être pas aussi évident, c’est que ces grands événements aient dû pareillement contribuer à encourager l’industrie de pays qui peut-être n’ont jamais envoyé en Amérique un seul article de leurs produits, tels que la Hongrie et la Pologne. C’est cependant ce dont il n’est pas possible de douter. On consomme en Hongrie et en Pologne une certaine partie du produit de l’Amérique ; et il y a dans ces pays une demande quelconque pour le sucre, le chocolat et le tabac de cette nouvelle partie du monde. Or, ces marchandises, il faut les acheter, ou avec quelque chose qui soit le produit de l’industrie de la Hongrie et de la Pologne, ou avec quelque chose qui ait été acheté avec une partie de ce produit. Ces marchandises américaines sont de nouvelles valeurs, de nouveaux équivalents survenus en Hongrie et en Pologne, pour y être échangés contre l’excédent du produit de ces pays. Transportées dans ces contrées, elles y créent un nouveau marché, un marché plus étendu pour cet excédent de produit. Elles en font hausser la valeur, et contribuent par là à encourager l’aug­men­tation. Quand même aucune partie de ce produit ne serait jamais portée en Amérique, il peut en être porté à d’autres nations qui l’achètent avec une partie de la portion qu’elles ont dans l’excédent de produit de l’Amérique, et ainsi ces nations trouveront un débit au moyen de la circulation du commerce nouveau que l’excédent de produit de l’Amérique a primitivement mis en activité.
Ces grands événements peuvent même avoir contribué à augmenter les jouis­sances et à accroître l’industrie de pays qui non-seulement n’ont jamais envoyé aucune marchandise en Amérique, mais même n’en ont jamais reçu aucune de cette contrée. Ces contrées-là même peuvent avoir reçu en plus grande abondance les marchandises de quelque nation dont l’excédent de produit aura été augmenté par le commerce de l’Amérique. Cette plus grande abondance, ayant nécessairement ajouté à leurs jouissances, a été pour eux un motif d’accroître leur industrie. Il leur a été présenté un plus grand nombre de nouveaux équivalents, d’une espèce ou d’une autre, pour être changés contre l’excédent de produit de cette industrie. Il a été créé un marché plus étendu pour ce produit surabondant, de manière à en faire hausser la valeur, et par là à en encourager l’augmentation. Cette masse de marchandises qui est jetée annuelle­ment dans la sphère immense du commerce de l’Europe, et qui, par l’effet de ses diverses révolutions, est distribuée annuellement entre toutes les différentes nations comprises dans cette sphère, a dû être augmentée de tout l’excédent de produit de l’Amérique. Il y a donc lieu de croire que chacune de ces nations a recueilli une plus grande part dans cette masse ainsi grossie, que ses jouissances ont augmenté et que son industrie a acquis de nouvelles forces.
Le commerce exclusif des métropoles tend à diminuer à la fois les jouissances et l’industrie de tous ces pays en général, et de l’Amérique en particulier, ou au moins il tend à les tenir au-dessous du degré auquel elles s’élèveraient sans cela. C’est un poids mort qui pèse sur l’action d’un des principaux ressorts dont une grande partie des affaires humaines reçoit son impulsion. En rendant le produit des colonies plus cher dans tous les autres pays, il en rend la consommation moindre, et par là il affaiblit l’industrie des colonies, et il retranche à la fois et des jouissances et de l’industrie de tous les autres pays ; ceux-ci se donnant moins de jouissances quand il faut les payer plus cher, et en même temps produisant moins quand leur produit leur rapporte moins. En rendant le produit de tous les autres pays plus cher dans les colonies, il affaiblit de la même manière l’industrie de tous ces autres pays, et il retranche de même aux colonies et de leurs jouissances et de leur industrie. C’est une entrave qui, pour le bénéfice prétendu de quelques pays particuliers, restreint les plaisirs et comprime l’industrie de tous les autres pays, mais encore plus des colonies que de tout autre. Il ne fait qu’exclure tous les autres pays, autant qu’il est possible, d’un marché particulier ; mais il confine les colonies, autant qu’il est possible, à un marché particulier ; et il y a une extrême différence d’être exclu d’un marché particulier quand on a tous les autres ouverts, ou d’être confiné sur un marché particulier quand les autres vous sont fermés. Néanmoins, c’est l’excédent de produit des colonies qui est toujours la source primitive de ce surcroît de jouissances et d’industrie qui revient à l’Europe de la découverte de l’Amérique et de sa formation en colonies, et le commerce exclusif des métropoles tend seulement à rendre cette source beaucoup moins abondante qu’elle n’aurait été sans cela.
Les avantages particuliers que chaque pays à colonies retire des colonies qui lui appartiennent sont de deux différentes espèces ; premièrement, les avantages généraux que tout État retire des provinces soumises à sa domination ; secondement, les avantages spéciaux qu’on suppose résulter de provinces d’une nature aussi particulière que les colonies européennes de l’Amérique.
Les avantages généraux que retire un État des provinces sujettes à sa domination consistent, en premier lieu, dans la force militaire qu’elles fournissent pour sa défense et, en second lieu, dans le revenu qu’elles donnent pour le soutien de son gouverne­ment civil. Les colonies romaines fournissaient, dans l’occasion, l’une et l’autre. Les colonies grecques fournissaient quelquefois une force militaire, mais rarement aucun revenu ; rarement elles se reconnaissaient comme soumises à la domination de la métropole ; elles étaient, en général, ses alliées pendant la guerre, mais très-rarement ses sujettes en temps de paix.
Les colonies européennes de l’Amérique n’ont encore fourni aucune force mili­taire pour la défense de la métropole ; leur force militaire n’a pas encore été suffisante pour leur défense propre ; et dans les guerres différentes dans lesquelles leur mère patrie a été engagée, il lui a fallu, en général, distraire une patrie très-considérable de ses forces militaires pour défendre ses colonies. Ainsi, sous ce rapport, toutes les colonies de l’Europe, sans exception, ont été, pour leurs métropoles respectives, une cause d’affaiblissement plutôt que de force.
Les seules colonies de l’Espagne et du Portugal ont contribué, par un revenu, à la défense de leur mère patrie ou au soutien de son gouvernement civil. Les impôts qui ont été levés sur celles des autres nations européennes, sur celles de l’Angleterre en particulier, ont rarement égalé la dépense qu’on a faite pour elles, et n’ont jamais été suffisants pour défrayer celle qu’elles ont occasionnée en temps de guerre ; ainsi, ces colonies ont été pour leurs métropoles respectives une source de dépense et non de revenu.
Les avantages que ces colonies ont pu procurer à leurs métropoles respective consistent donc uniquement dans ces avantages spéciaux qu’on suppose résulter de la nature particulière de ces possessions ; et la seule source de tous ces avantages spéciaux, c’est, à ce qu’on assure généralement, le commerce exclusif.
En vertu de ce droit exclusif, toute cette partie du produit surabondant des colo­nies anglaises, par exemple, qui consiste en ce qu’on appelle marchandises énumé­rées, ne peut être envoyée à aucun autre pays que l’Angleterre ; il faut que ce soit d’elle que les autres pays l’achètent ensuite. Ce produit doit donc nécessairement être à meilleur marché en Angleterre qu’il ne peut l’être dans tout autre pays, et il doit contribuer à augmenter les jouissances de l’Angleterre plus que celles de tout autre pays ; il doit de même aussi contribuer davantage à encourager son industrie. L’Angle­terre doit tirer un meilleur prix de toutes les parties de l’excédant de son propre produit qu’elle échange contre ces marchandises énumérées, que les autres pays ne peuvent en tirer de celles du leur, qu’elles échangeraient contre ces mêmes mar­chandises. Par exemple, les ouvrages des fabriques anglaises achèteront une plus grande quantité de sucre et de tabac des colonies anglaises, que de pareils ouvrages des fabriques des autres pays ne pourraient en acheter. Ainsi, en tant que les ouvrages des fabriques anglaises et ceux des fabriques des autres pays peuvent être dans le cas de s’échanger contre le sucre et le tabac des colonies anglaises, cette supériorité de prix donne aux premières de ces fabriques plus d’encouragement que les autres ne peuvent en recevoir de la même source. Par conséquent, comme le commerce exclusif des colonies diminue à la fois et les jouissances et l’industrie des pays qui sont exclus de ce commerce, ou qu’au moins il tient ces jouissances et cette industrie au-dessous du degré auquel elles s’élèveraient sans cela, ce commerce donne, aux pays qui en sont en possession, un avantage d’autant plus manifeste sur les autres pays.
Cependant, on trouvera peut-être que cet avantage devrait plutôt passer pour ce qu’on peut appeler un avantage relatif que pour un avantage absolu, et que la supériorité qu’il donne au pays qui en jouit consiste moins à faire monter l’industrie et le produit de ce pays au-dessus de ce qu’ils seraient naturellement, dans le cas où le commerce serait libre, qu’elle ne consiste à rabaisser l’industrie et le produit des autres pays au-dessous de ce qu’ils seraient sans cette restriction.
Par exemple, le tabac du Maryland et de la Virginie, au moyen du monopole dont jouit l’Angleterre sur cette denrée, revient certainement à meilleur marché à l’Angle­terre qu’il ne peut revenir à la France, à qui l’Angleterre en vend ordinairement une partie considérable. Mais si la France et tous les autres pays de l’Europe eussent eu, dans tous les temps, la faculté de commercer librement au Maryland et à la Virginie, le tabac de ces colonies aurait pu, pendant cette période, se trouver revenir à meilleur compte qu’il ne revient actuellement, non-seulement pour tous ces autres pays, mais aussi pour l’Angleterre elle-même. Au moyen d’un marché qui eut été si fort étendu au-delà de celui dont il a joui jusqu’à présent, le produit du tabac aurait pu tellement s’accroître, et probablement même se serait tellement accru pendant cette période, qu’il aurait réduit les profits d’une plantation de tabac à leur niveau naturel avec ceux d’une terre à blé, au-dessus desquels ils sont encore, à ce que l’on croit ; durant cette période, le prix du tabac eût pu tomber, et vraisemblablement serait tombé un peu plus bas qu’il n’est à présent. Une pareille quantité de marchandises, soit d’Angleterre, soit de ces autres pays, aurait acheté, dans le Maryland et dans la Virginie, plus de tabac qu’elle ne peut en acheter aujourd’hui, et ainsi elle y aurait été vendue à un prix d’autant meilleur. Par conséquent, si l’abondance et le bon marché de cette plante ajoutent quelque chose aux jouissances et à l’industrie de l’Angleterre ou de tout autre pays, ce sont deux effets qu’ils auraient vraisemblablement produits à un degré un peu plus considérable qu’ils ne font aujourd’hui, si la liberté du commerce eût eu lieu. À la vérité, dans cette supposition, l’Angleterre n’aurait pas eu d’avantage sur les autres pays ; elle aurait bien acheté le tabac de ses colonies un peu meilleur marché qu’elle ne l’achète et, par conséquent, aurait vendu quelques-unes de ses propres marchan­dises un peu plus cher qu’elle ne fait à présent ; mais elle n’aurait pas pu pour cela acheter l’un meilleur marché, ni vendre les autres plus cher que ne l’eût fait tout autre pays ; elle aurait peut-être gagné un avantage absolu, mais bien certainement elle aurait perdu un avantage relatif.
Cependant, en vue de se donner cet avantage relatif dans le commerce des colo­nies, en vue d’exécuter un projet de pure malice et de pure jalousie, celui d’exclure, autant que possible, toutes les autres nations de la participation à ce commerce, l’Angleterre a, selon toute apparence, non-seulement sacrifié une partie de l’avantage absolu qu’elle devait retirer, en commun avec toutes les autres nations, de ce commerce particulier, mais encore elle s’est assujettie, dans presque toutes les autres branches de commerce, à un désavantage absolu, et en même temps à un désavantage relatif.
Lorsque, par l’acte de navigation, l’Angleterre s’est emparée du monopole du com­merce des colonies, les capitaux étrangers, qui avaient été auparavant employés dans ce commerce, en ont été nécessairement retirés. Le capital anglais, qui n’avait soutenu jusque-là qu’une partie de ce commerce, fut alors obligé d’en soutenir la totalité. Le capital qui jusque-là n’avait fourni aux colonies que partie seulement des marchan­dises qu’elles recevaient d’Europe, forma alors la totalité du capital employé à leur amener tout ce qu’elles pouvaient tirer d’Europe. Or, ce capital ne pouvait leur fournir la totalité de ce qu’elles demandaient de marchandises, et celles qu’il leur amenait leur étaient nécessairement vendues fort cher. Le capital qui n’avait acheté auparavant qu’une partie seulement du produit surabondant des colonies, composa alors tout le capital destiné à acheter la totalité de ce produit. Mais il ne pouvait pas acheter cette totalité à l’ancien prix, ni même à beaucoup près et, par conséquent, tout ce qu’il en achetait était acheté nécessairement à très-bas prix. Or, dans un emploi de capital, où le marchand vendait fort cher et achetait à très-bon marché, les profits ont dû être nécessairement très-forts, et bien au-dessus du niveau ordinaire des profits dans les autres branches de commerce. Cette supériorité des profits du commerce colonial ne pouvait manquer d’attirer, de toutes les autres branches de commerce, une partie du capital qui leur avait été consacré jusque-là. Mais si cette révolution dans la direction du capital national a dû nécessairement augmenter successivement la concurrence des capitaux dans le commerce des colonies, elle a dû, par la même raison, diminuer successivement cette concurrence dans les autres branches de commerce ; si elle a dû faire baisser par degrés les profits de ce commerce, elle a dû, par la même raison, faire hausser par degrés les profits des autres, jusqu’à ce que le niveau fût rétabli dans les profits de tous, niveau différent, il est vrai, du premier, et un peu plus élevé que celui qui existait entre eux auparavant[14]. Ce double effet d’attirer les capitaux de tous les autres genres de commerce, et de faire monter en même temps, dans tous, le taux du profit un peu plus haut qu’il n’aurait été sans cela, a été non-seulement produit par le monopole, au moment où celui-ci a été établi, mais a continué d’être toujours produit par lui depuis[15]. 
Premièrement, ce monopole n’a pas cessé d’attirer continuellement le capital de tous les autres genres de commerce, pour le porter dans le commerce des colonies.
Quoique l’opulence de la Grande-Bretagne ait extrêmement augmenté depuis l’établissement de l’acte de navigation, elle n’a certainement pas augmenté dans la mê­me proportion que celle des colonies. Or, le commerce étranger d’un pays augmente naturellement dans la même proportion que son opulence ; l’excédent de son produit augmente dans la proportion qu’augmente son produit total, et la Grande-Bretagne s’étant emparée pour son propre compte de tout ce qu’on peut appeler le commerce étranger des colonies, sans que son capital ait augmenté à proportion de l’extension de ce commerce, elle n’aurait pu le soutenir si elle n’eût pas sans cesse retiré des autres branches de son commerce quelque partie du capital qui leur avait été destiné jusqu’alors, et si elle n’eût pas aussi sans cesse éloigné de ces mêmes branches de trafic une quantité encore bien plus grande de capital qui sans cela s’y serait portée. Aussi, depuis l’établissement de l’acte de navigation, le commerce avec les colonies a-t-il été continuellement en s’étendant de plus en plus, tandis que plusieurs autres branches de commerce étranger, et en particulier celui avec les autres parties de l’Europe, a été continuellement en dépérissant. Les produits de nos manufactures destinés à être vendus à l’étranger, au lieu de s’adapter, comme avant l’acte de navigation, au marché de l’Europe qui nous avoisine, ou au marché plus éloigné que nous offrent les pays situés aux bords de la Méditerranée, se sont appropriés, pour la plupart, aux besoins et aux demandes du marché des colonies, qui est infiniment plus éloigné ; du marché où ces manufactures jouissent du monopole, plutôt que de celui où elles peuvent trouver une foule de concurrents. Ces causes du dépérissement des autres branches de notre commerce étranger, que sir Matthieu Decker et d’autres écrivains ont été chercher dans l’excès des taxes, dans le mode vicieux de l’impôt, dans le haut prix du travail, dans l’accroissement du luxe, etc., on peut les trouver toutes dans la croissance monstrueuse de notre commerce des colonies[16]. Comme le capital de la Grande-Bretagne, quoique extrêmement considérable, n’est pourtant pas infini, et comme ce capital, quoique grandement augmenté depuis l’acte de navi­gation, n’a cependant pas augmenté dans la même proportion que notre commerce des colonies, il n’aurait jamais été possible de soutenir ce commerce sans enlever aux autres branches quelque portion de capital, ni, par conséquent, sans y occasionner quelque dépérissement.
Il faut observer que l’Angleterre était déjà un grand pays commerçant ; que la masse de ses capitaux engagés dans le négoce était déjà très-considérable, et suscep­tible de grossir encore de jour en jour, non-seulement avant que l’acte de navigation eût établi le monopole du commerce des colonies, mais avant même que ce com­merce eût acquis une grande importance. Pendant la guerre de Hollande, sous le gou­vernement de Cromwell, la marine anglaise était supérieure à celle de la Hollande ; et dans la guerre qui éclata au commencement du règne de Charles II, elle était au moins égale, peut-être supérieure aux marines réunies de la France et de la Hollande. Cette supériorité paraîtrait à peine plus grande aujourd’hui, du moins si la marine de Hollande était maintenant proportionnée au commerce actuel de cette république, comme elle l’était alors. Or, dans aucune de ces guerres, ce ne pouvait être à l’acte de navigation qu’elle dut cette grande puissance maritime. Pendant la première, le projet de cet acte venait à peine d’être formé, et quoique, avant les premières hostilités de la seconde, il eût déjà reçu force de loi, cependant aucune de ses dispositions n’avait encore eu le temps de pouvoir produire quelque effet considérable, et bien moins que toutes les autres, celles qui établissaient le commerce exclusif avec les colonies. Les colonies et leur commerce avaient alors fort peu d’importance, en comparaison de celle qu’ils ont aujourd’hui. L’île de la Jamaïque était un désert malsain, fort peu habité et encore moins cultivé. New-York et New-jersey étaient en la possession de la Hollande ; la moitié de Saint-Christophe était aux mains des Français. L’île d’Antigoa, les deux Carolines, la Pensylvanie, la Géorgie et la Nouvelle-Écosse n’étaient pas encore cultivées. La Virginie, le Maryland et la Nouvelle-Angleterre étaient mis en culture ; mais, quoique ces colonies fussent très-florissantes, il n’y avait peut-être pas alors une seule personne en Europe ou en Amérique qui prévît ou qui même soup­çon­nât le progrès rapide qu’elles ont fait depuis en richesse, en population et en industrie. En un mot, à cette époque, la Barbade était la seule colonie anglaise de quelque importance, dont la situation eût quelque ressemblance avec celle où elle est aujour­d’hui. Le commerce des colonies, dont l’Angleterre n’avait encore qu’une partie, même quelque temps encore après l’acte de navigation (car cet acte ne fut exécuté très-strictement que plusieurs années après sa promulgation) ; ce commerce, dis-je, ne pouvait pas, à cette époque, être la cause du grand commerce de l’Angleterre ni de cette grande force navale qui était soutenue par ce commerce. Le commerce qui soutenait alors l’étendue de sa puissance maritime, c’était celui d’Europe et des pays situés autour de la Méditerranée. Or, la part qu’a maintenant l’Angleterre dans ce commerce ne pourrait pas soutenir de pareilles forces navales. Si le commerce des colonies, qui croissait alors, eût été laissé libre à toutes les nations, quelle qu’eût été la part qui en serait échue à la Grande-Bretagne (et il est probable que cette part aurait été très-importante), elle aurait été tout entière en surcroît de ce grand commerce dont l’Angleterre était déjà en possession. Mais, par l’effet du monopole, l’accroissement du commerce des colonies a bien moins été, pour le commerce général de la Grande-Bretagne, la cause d’une addition à ce qu’il était auparavant, que celle d’un change­ment total de direction.
Secondement, ce monopole a contribué nécessairement à maintenir, dans toutes les autres branches du commerce de la Grande-Bretagne, le taux du profit à un degré plus élevé que celui où il se serait tenu naturellement si le commerce avec les colo­nies anglaises eût été laissé libre à toutes, les nations.
Si le monopole du commerce des colonies a nécessairement entraîné vers ce commerce une plus grande partie du capital de la Grande-Bretagne que celle qui s’y serait portée d’elle-même, d’un autre côté, en en expulsant tous les capitaux étrangers, il a nécessairement réduit la quantité totale de capital employé dans ce commerce, au-dessous de ce qu’elle aurait été naturellement dans le cas où le commerce aurait été fibre. Or, en diminuant la concurrence des capitaux dans cette branche de commerce, il y a nécessairement fait hausser le taux du profit. En diminuant aussi la concurrence des capitaux anglais dans toutes les autres branches du commerce, il a nécessairement fait hausser le taux du profit, en Angleterre, dans toutes ces autres branches. Quel qu’ait pu être, à une époque quelconque depuis l’établissement de l’acte de navigation, l’état ou l’étendue de la masse des capitaux de la Grande-Bretagne engagés dans le commerce, nécessairement le monopole du commerce des colonies, tant que cette masse est restée la même, doit avoir élevé le taux du profit en Angleterre plus haut qu’il n’aurait été sans cela dans cette branche de commerce et dans toutes les autres. Si le taux ordinaire du profit en Angleterre a considérablement baissé depuis l’établissement de l’acte de navigation, comme assurément cela est arrivé, il aurait été forcé de tomber encore plus bas si le monopole établi par cet acte n’eût pas contribué à le tenir élevé.
Or, tout ce qui fait monter dans un pays le taux ordinaire du profit plus haut qu’il n’aurait été naturellement, assujettit nécessairement ce pays et à un désavantage absolu et à un désavantage relatif dans toutes les autres branches de commerce dont il n’a pas le monopole.
Il assujettit ce pays à un désavantage absolu, attendu que, dans toutes ces autres branches de commerce, ses marchands ne peuvent retirer ce plus gros profit sans vendre à la fois et les marchandises des pays étrangers qu’ils importent dans le leur, et les marchandises de leur propre pays qu’ils exportent à l’étranger, plus cher qu’ils ne les eussent vendues sans cette circonstance. Il faut que leur propre pays à la fois vende plus cher et achète plus cher qu’il n’aurait fait ; il faut à la fois qu’il achète moins et vende moins ; il faut, enfin, qu’il jouisse moins et qu’il produise moins.
Il assujettit ce pays à un désavantage relatif, attendu que, dans toutes ces autres branches de commerce, les autres pays, qui ne sont pas assujettis au même désavan­tage absolu, se trouvent par là placés, vis-à-vis de ce pays, ou plus au-dessus, ou moins au-dessous de lui qu’ils n’y auraient été. Il les met en état à la fois de jouir plus et de produire plus relativement à la proportion dans laquelle ce pays jouit et produit. Il rend leur supériorité plus grande à son égard, ou leur infériorité moindre qu’elle n’eût été. En faisant monter le prix du produit de ce pays au-dessus de ce qu’il eût été, il met les marchands des autres pays à même de vendre à meilleur compte que ce pays ne peut le faire sur les marchés étrangers, et par là de le supplanter et de l’ex­clure dans presque toutes les branches de commerce dont celui-ci n’a pas le monopole.
On entend souvent nos marchands se plaindre de l’élévation des salaires du travail indigène, comme de la cause qui empêche les produits de leurs fabriques de se sou­tenir sur les marchés étrangers ; mais on ne les entend jamais parler des hauts profits du capital. Ils se plaignent du gain excessif des autres, mais ils ne disent rien du leur. Cependant, les hauts profits du capital en Angleterre peuvent contribuer, dans beaucoup de circonstances, autant que l’élévation des salaires payés au travail, et dans quelques circonstances peut-être contribuer davantage à faire hausser le prix des produits des fabriques anglaises[17].
C’est ainsi qu’on peut dire avec raison que le capital de la Grande-Bretagne a été retiré et en partie exclu de la plupart des différentes branches de commerce dont elle n’a pas le monopole, particulièrement du commerce de l’Europe et de celui des pays situés autour de la Méditerranée. 
Il a été en partie retiré de ces branches de commerce par l’attraction qu’a exercée sur lui la supériorité du profit dans notre commerce des colonies, supériorité résultant de l’accroissement continuel de ce commerce, et de l’insuffisance continuelle du capital qui l’avait soutenu une année, à pouvoir le soutenir l’année suivante.
Il a été en partie exclu de ces branches de commerce par l’avantage que le taux élevé des profits qui a lieu en Angleterre donne aux autres pays dans toutes les différentes branches de commerce dont la Grande-Bretagne n’a pas le monopole[18].
Comme le monopole du commerce des colonies a retiré de ces autres branches de commerce une partie du capital anglais qui y aurait sans cela été employé, de même il y a poussé forcément beaucoup de capitaux étrangers, qui n’y seraient jamais entrés s’ils n’avaient pas été chassés du commerce des colonies. Dans ces autres branches de commerce, il a diminué la concurrence des capitaux anglais, et par là il a fait monter le taux du profit du négociant anglais plus haut qu’il n’aurait pu atteindre. Au con­traire, il a augmenté la concurrence des capitaux étrangers, et par là il a abaissé le taux du profit du négociant étranger au-dessous de ce qu’il aurait été. Il a donc dû nécessairement à la fois, de ces deux manières, assujettir la Grande-Bretagne à un désavantage relatif dans toutes ses autres branches de commerce.
Mais peut-être, va-t-on dire, le commerce des colonies est plus avantageux que tout autre à la Grande-Bretagne et, en forçant d’entrer dans ce commerce une plus forte portion du capital de la Grande-Bretagne que celle qui s’y serait portée sans cela, le monopole a tourné ce capital vers un emploi plus avantageux à la nation que tout autre emploi qu’il eût pu trouver.
La manière la plus avantageuse dont un capital puisse être employé pour le pays auquel il appartient, c’est celle qui y entretient la plus grande quantité de travail productif, et qui ajoute le plus au produit annuel de la terre et du travail de ce pays. Or, nous avons fait voir, dans le second livre, que la quantité de travail productif que peut entretenir un capital employé dans le commerce étranger de consommation est exactement en proportion de la fréquence de ses retours. Un capital de 1,000 livres, par exemple, employé dans un commerce étranger de consommation dont les retours se font régulièrement une fois par an, peut tenir constamment en activité, dans le pays auquel il appartient, une quantité de travail productif égale à ce que 1,000 livres peuvent y en faire subsister pour un an[19]. Si les retours se font deux ou trois fois dans l’année, il peut tenir constamment en activité une quantité de travail productif égale à ce que 2 ou 3,000 livres peuvent y en faire subsister pour un an. Par cette raison un commerce étranger de consommation qui se fait avec un pays voisin est, en général, plus avantageux qu’un autre qui se fait dans un pays éloigné ; et par la même raison, un commerce étranger de consommation qui se fait par voie directe est, en général, comme on l’a fait voir pareillement dans le second livre, plus avantageux que celui qui se fait par circuit.
Or, le monopole du commerce des colonies, autant qu’il a pu influer sur l’emploi du capital de la Grande-Bretagne, a, dans toutes les circonstances, détourné forcé­ment une partie de ce capital d’un commerce étranger de consommation fait avec un pays voisin, pour la porter vers un pareil commerce avec un pays plus éloigné ; et dans beaucoup de circonstances, il l’a détournée d’un commerce étranger de consom­mation fait par voie directe, pour la porter vers un autre fait par circuit.
Premièrement, le monopole du commerce des colonies a, dans toutes les circon­stances, enlevé quelque portion du capital de la Grande-Bretagne à un commerce étranger de consommation fait avec un pays voisin, pour la porter vers un pareil commerce fait avec un pays plus éloigné.
Il a, dans toutes les circonstances, enlevé quelque portion de ce capital au commerce avec l’Europe et avec les pays environnant la Méditerranée, pour la porter au commerce avec les contrées bien plus reculées de l’Amérique et des Indes occidentales, commerce dont les retours sont nécessairement moins fréquents, non-seulement par rapport au grand éloignement, mais encore par rapport à la situation particulière où se trouvent les affaires de ces contrées. De nouvelles colonies, comme on l’a déjà observé, sont toujours dépourvues de capitaux ; la masse de leurs capitaux est toujours fort au-dessous de ce qu’elles pourraient employer avec beaucoup d’avan­tage et de profit dans l’amélioration et la culture de leurs terres ; elles ont donc constamment chez elles une demande de capitaux pour plus que ce qu’elles en pos­sèdent en propre, et, pour suppléer au déficit de la masse de leurs propres capitaux, elles tâchent d’emprunter, autant qu’elles le peuvent, de la mère patrie, envers laquelle, par ce moyen, elles sont toujours endettées. La manière la plus ordinaire dont les colons contractent ces dettes, ce n’est pas en empruntant par obligation aux riches capitalistes de la métropole, quoiqu’ils le fassent aussi quelquefois, mais c’est en traînant leurs payements en longueur avec leurs correspondants qui leur expédient des marchandises d’Europe, aussi longtemps que ces correspondants veulent bien le leur laisser faire. Leurs retours annuels très-souvent ne montent pas à plus d’un tiers de ce qu’ils doivent, quelquefois moins ; par conséquent, la totalité du capital que leur avancent leurs correspondants ne rentre guère dans la Grande-Bretagne avant trois ans, et quelquefois pas avant quatre ou cinq. Or, un capital anglais de 1,000 livres, par exemple, qui ne rentre en Angleterre qu’une fois dans un espace de cinq ans, ne peut tenir constamment en activité qu’un cinquième seulement de l’industrie anglaise qu’il aurait pu entretenir s’il fût rentré en totalité dans le cours d’une année et, au lieu de tenir en activité la quantité d’industrie que 1,000 livres pourraient entretenir pendant une année, il n’y tient constamment, employée que celle seulement que peuvent entretenir pendant une année 200 livres. Le planteur, sans contredit, par le haut prix auquel il paye les marchandises d’Europe, par l’intérêt qu’il paye sur les lettres de change qu’il donne à de longues échéances, et par le droit de commission pour le renouvellement de celles qu’il donne à de plus courts termes, bonifie à son corres­pondant, et probablement fait plus que lui bonifier toute la perte que celui-ci pourrait essuyer de ce délai ; mais, s’il peut dédommager son correspondant de sa perte, il ne peut dédommager de même la Grande-Bretagne de celle qu’elle éprouve. Dans un commerce dont les retours sont très-lents, le profit du marchand peut être aussi grand et même plus grand que dans un autre où ils sont très-fréquents et très-rapprochés ; mais l’avantage du pays où réside ce marchand, la quantité du travail productif qui peut y être constamment en activité, le produit annuel des terres et du travail, en doivent toujours nécessairement beaucoup souffrir[20]. Or, je pense que quiconque a la moindre expérience dans ces différentes branches de commerce, m’accordera sans peine que les retours d’un commerce en Amérique, et encore plus ceux d’un com­merce aux Indes occidentales, sont, en général, non-seulement plus lents que ceux d’un commerce à quelque endroit de l’Europe, et même aux pays circonvoisins de la Méditerranée, mais encore plus irréguliers et plus incertains.
Secondement, le monopole du commerce des colonies a, dans beaucoup de cir­cons­tances, enlevé une certaine portion du capital de la Grande-Bretagne à un com­merce étranger de consommation fait par voie directe, pour la forcer d’entrer dans un autre fait par circuit.
Parmi les marchandises énumérées qui ne peuvent être envoyées à aucun autre marché qu’à celui de la Grande-Bretagne, il y en a plusieurs dont la quantité excède de beaucoup la consommation de la Grande-Bretagne, et dont il faut, par conséquent, qu’une partie soit exportée à d’autres pays ; or, c’est ce qui ne peut se faire sans entraîner quelque partie du capital de la Grande-Bretagne dans un commerce étranger de consommation par circuit. Par exemple, le Maryland et la Virginie envoient an­nuel­lement à la Grande-Bretagne au-delà de quatre-vingt-seize mille muids de tabac, et la consommation de la Grande-Bretagne n’excède pas, à ce qu’on dit, qua­torze mille muids ; il y en a donc plus de quatre-vingt-deux mille qu’il faut exporter dans d’autres pays, en France, en Hollande et aux contrées situées autour de la mer Baltique et de la Méditerranée. Or, cette portion du capital de la Grande-Bretagne qui porte ces quatre-vingt-deux mille muids à la Grande-Bretagne, qui de là les réexporte à ces autres pays, et qui rapporte de ces autres pays dans la Grande-Bretagne ou d’autres marchandises, ou de l’argent en retour, est employée dans un commerce étranger de consommation par circuit, et elle est forcément entraînée à cet emploi par la nécessité qu’il y a de disposer de cet énorme excédent. Pour supputer en combien d’années la totalité de ce capital pourra vraisemblablement être rentrée dans la Grande-Bretagne, il faudrait ajouter à la lenteur des retours de l’Amérique celle des retours de ces autres pays. Si, dans le commerce étranger de consommation qui se fait par voie directe avec l’Amérique, il arrive souvent que la totalité du capital employé ne rentre pas en moins de trois ou quatre ans, il y a lieu de présumer que la totalité du capital employé dans ce commerce ainsi détourné ne rentrera pas en moins de quatre ou cinq. Si le premier ne peut tenir constamment en activité qu’un tiers ou qu’un quart seulement du travail national que pourrait entretenir un capital dont la rentrée aurait lieu une fois par an, l’autre ne pourra tenir constamment employé qu’un quart ou un cinquième de ce travail. Des négociants de quelques-uns de nos ports accordent ordinairement un crédit aux correspondants étrangers auxquels ils exportent leur tabac ; à la vérité, au port de Londres, il se vend communément argent comptant ; la règle est : Pesez et payez. Par conséquent, au port de Londres, les retours définitifs de la totalité du circuit de ce commerce se trouvent être plus tardifs que les retours d’Amérique de la quantité de temps seulement pendant laquelle les marchandises peuvent rester dans le magasin sans être vendues, temps qui ne laisse pas cependant d’être quelquefois assez long. Mais si les colonies n’eussent pas été confinées au marché de la Grande-Bretagne pour la vente de leur tabac, il n’en serait probablement venu chez nous que très-peu au-delà de ce qui est nécessaire à notre propre consom­mation. Les marchandises que la Grande-Bretagne achète à présent, pour sa consommation, avec cet énorme excédent de tabac qu’elle exporte à d’autres pays, elle les aurait probablement, dans ce cas, achetées immédiatement avec le produit de son industrie ou avec quelque partie du produit de ses manufactures ; ce produit, ces ouvrages de manufactures, au lieu d’être, comme à présent, presque entièrement assortis aux demandes d’un seul grand marché, auraient été vraisemblablement appropriés à un grand nombre de marchés plus petits, au lieu d’un immense commer­ce étranger de consommation par circuit, la Grande-Bretagne aurait probablement entretenu un grand nombre de petits commerces étrangers du même genre par voie directe. À cause de la fréquence des retours, une partie seulement, et vraisembla­blement une petite partie, peut-être pas plus d’un tiers ou d’un quart du capital sur lequel roule aujourd’hui cet immense commerce par circuit, aurait été suffisante pour faire aller tous ces petits commerces directs, aurait tenu constamment en activité une égale quantité d’industrie anglaise, et aurait fourni le même aliment au produit annuel des terres et du travail de la Grande-Bretagne. Tous les objets utiles de ce commerce se trouvant ainsi remplis par un capital beaucoup moindre, il y aurait eu une grosse portion de capital épargnée, qu’on eût pu appliquer à d’autres objets, à l’amélioration des terres de la Grande-Bretagne, à l’accroissement de ses manufactures et à l’exten­sion de son commerce, qui eût pu servir au moins à venir en concurrence avec les autres capitaux anglais employés dans tous ces divers genres d’affaires, à réduire dans tous ces emplois le taux du profit, et par là à donner à la Grande-Bretagne, dans ces mêmes emplois, une plus grande supériorité sur tous les autres pays que celle dont elle jouit maintenant.
Le monopole du commerce des colonies a de plus enlevé au commerce étranger de consommation une certaine portion du capital de la Grande-Bretagne, pour la forcer d’entrer dans le commerce de transport et, par conséquent, il a enlevé à l’in­dustrie de la Grande-Bretagne le soutien qu’elle en recevait, pour le faire servir uniquement à soutenir en partie celle des colonies et en partie celle de quelque autre pays.
Par exemple, les marchandises qui s’achètent annuellement avec cet énorme excédent de tabac, ces quatre-vingt-deux mille muids annuellement réexportés de la Grande-Bretagne, ne sont pas toutes consommées dans la Grande-Bretagne. Partie de ces marchandises, les toiles d’Allemagne et de Hollande, par exemple, sont renvoyées aux colonies pour leur consommation particulière. Or, cette portion du capital de la Grande-Bretagne qui achète le tabac avec lequel ensuite on achète ces toiles, est néces­sairement retirée à l’industrie de la Grande-Bretagne, pour aller servir unique­ment à soutenir en partie celle des colonies, et en partie celle des pays qui payent ce tabac avec le produit de leur industrie.
D’un autre côté, le commerce des colonies, en entraînant dans ce commerce une portion beaucoup plus forte du capital de la Grande-Bretagne que celle qui s’y serait naturellement portée, paraît avoir entièrement rompu cet équilibre qui se serait établi sans cela entre toutes les diverses branches de l’industrie britannique. Au lieu de s’assortir à la convenance d’un grand nombre de petits marchés, l’industrie de la Grande-Bretagne s’est principalement adaptée aux besoins d’un grand marché seule­ment. Son commerce, au lieu de parcourir un grand nombre de petitscanaux, a pris son cours principal dans un grand canal unique. Or, il en est résulté que le système total de son industrie et de son commerce en est moins solidement assuré qu’il ne l’eût été de l’autre manière ; que la santé de son corps politique en est moins ferme et moins robuste. La Grande-Bretagne, dans son état actuel, ressemble à l’un de ces corps malsains dans lesquels quelqu’une des parties vitales a pris une croissance mons­trueuse, et qui sont, par cette raison, sujets à plusieurs maladies dangereuses aux­quelles ne sont guère exposés ceux dont toutes les parties se trouvent mieux propor­tionnées. Le plus léger engorgement dans cet énorme vaisseau sanguin qui, à force d’art, s’est grossi chez nous fort au-delà de ses dimensions naturelles, et au travers duquel circule, d’une manière forcée, une portion excessive de l’industrie et du commerce national, menacerait tout le corps politique des plus funestes maladies. Aussi jamais l’armada des Espagnols ni les bruits d’une invasion française n’ont-ils frappé le peuple anglais de plus de terreur que ne l’a fait la crainte d’une rupture avec les colonies. C’est cette terreur, bien ou mal fondée, qui a fait de la révocation de l’acte du timbre une mesure populaire, au moins parmi les gens de commerce. L’ima­gination de la plupart d’entre eux s’est habituée à regarder une exclusion totale du marché des colonies, ne dût-elle être que de quelques années, comme un signe certain de ruine complète pour eux ; nos marchands y ont vu leur commerce totalement arrêté, nos manufacturiers y ont vu leurs fabriques absolument perdues, et nos ouvriers se sont crus à la veille de manquer tout à fait de travail et de ressources. Une rupture avec quelques-uns de nos voisins du continent, quoique dans le cas d’en­traîner aussi une cessation ou une interruption dans les emplois de quelques individus dans toutes ces différentes classes, est pourtant une chose qu’on envisage sans cette émotion générale. Le sang dont la circulation se trouve arrêtée dans quelqu’un des petits vaisseaux se dégorge facilement dans les plus grands, sans occasionner de crise dangereuse ; mais s’il se trouve arrêté dans un des grands vaisseaux, alors les convulsions, l’apoplexie, la mort, sont les conséquences promptes et inévitables d’un pareil accident. Qu’il survienne seulement quelque léger empêchement ou quelque interruption d’emploi dans un de ces genres de manufacture qui se sont étendus d’une manière démesurée, et qui, à force de primes ou de monopoles sur les marchés national et colonial, sont arrivés artificiellement à un degré d’accroissement contre nature, il n’en faut pas davantage pour occasionner de nombreux désordres, des sédi­tions alarmantes pour le gouvernement, et capables même de troubler la liberté des délibérations de la législature. À quelle confusion, à quels désordres ne serions-nous pas exposés infailliblement, disait-on, si une aussi grande portion de nos principaux manufacturiers venait tout d’un coup à manquer totalement d’emploi ?
Le seul expédient, à ce qu’il semble, pour faire sortir la Grande-Bretagne d’un état aussi critique, ce serait un relâchement modéré et successif des lois qui lui donnent le monopole exclusif du commerce colonial, jusqu’à ce que ce commerce fût en grande partie rendu libre. C’est le seul expédient qui puisse la mettre à même ou la forcer, s’il le faut, de retirer de cet emploi, monstrueusement surchargé, quelque portion de son capital pour la diriger, quoique avec moins de profit, vers d’autres emplois et qui, en diminuant par degrés une branche de son industrie et en augmentant de même toutes les autres, puisse insensiblement rétablir entre toutes les différentes branches cette juste proportion, cet équilibre naturel et salutaire qu’amène nécessairement la parfaite liberté, et que la parfaite liberté peut seule maintenir. Ouvrir tout d’un coup à toutes les nations le commerce des colonies pourrait non-seulement donner lieu à quelques inconvénients passagers, mais causer même un dommage durable et important à la plupart de ceux qui y ont à présent leur industrie ou leurs capitaux engagés. Une cessa­tion subite d’emploi, seulement pour les vaisseaux qui importent les quatre-vingt-deux mille muids de tabac qui excèdent la consommation de la Grande-Bretagne, pourrait occasionner des pertes très-sensibles. Tels sont les malheureux effets de tous les règlements du système mercantile ! non-seulement ils font naître des maux très-dangereux dans l’état du corps politique, mais encore ces maux sont tels qu’il est souvent difficile de les guérir sans occasionner, pour un temps au moins, des maux encore plus grands. Comment donc le commerce des colonies devait-il être successivement ouvert ? Quelles sont les barrières qu’il faut abattre les premières, et quelles sont celles qu’il ne faut faire tomber qu’après toutes les autres ? Ou enfin, par quels moyens et par quelles gradations rétablir le système de la justice et de la parfaite liberté ? C’est ce que nous devons laisser à décider à la sagesse des hommes d’État et des législateurs futurs[21]. 
Cinq événements différents, qui n’ont pas été prévus et auxquels on ne pensait pas, ont concouru très-heureusement à empêcher la Grande-Bretagne de ressentir d’une manière aussi sensible qu’on s’y était généralement attendu l’exclusion totale qu’elle éprouve aujourd’hui, depuis plus d’un an (depuis le 1er décembre 1774), d’une branche très-importante du commerce des colonies, celui des douze Provinces-Unies de l’Amérique septentrionale. Premièrement, ces colonies, en se préparant à l’accord fait entre elles de ne plus importer, ont épuisé complètement la Grande-Bretagne de toutes les marchandises qui étaient à leur convenance ; secondement, la demande extraordinaire de la flotte espagnole a épuisé cette année l’Allemagne et le Nord d’un grand nombre de marchandises, et en particulier des toiles qui avaient coutume de faire concurrence, même sur le marché britannique, aux manufactures de la Grande-Bretagne ; troisièmement, la paix entre la Russie et les Turcs a occasionné une demande extraordinaire sur le marché de la Turquie, qui avait été extrêmement mal pourvu dans le temps de la détresse du pays et pendant qu’une flotte russe croisait dans l’Archipel ; quatrièmement, la demande d’ouvrages de manufacture anglaise pour le nord de l’Europe a été, depuis quelque temps, toujours en augmentant d’année en année ; et cinquièmement, le dernier partage de la Pologne et la pacification qui en a été la suite, en ouvrant le marché de ce grand pays, ont ajouté, cette année, à la demande toujours croissante du Nord, une demande extraordinaire de ce côté-là.
Ces événements, à l’exception du quatrième, sont tous, de leur nature, accidentels et passagers, et si malheureusement l’exclusion d’une branche aussi importante du commerce des colonies venait à durer plus longtemps, elle pourrait occasionner encore quelque surcroît d’embarras et de dommage. Mais néanmoins, comme cette gêne sera survenue par degrés, on la sentira moins durement que si elle fût survenue tout d’un coup, et en même temps l’industrie et le capital du pays pourront trouver un nouvel emploi et prendre une nouvelle direction, de manière à empêcher que le mal ne devienne jamais très-considérable. 
Ainsi, toutes les fois que le monopole du commerce des colonies a entraîné dans ce commerce une plus forte portion du capital de la Grande-Bretagne que celle qui s’y serait portée sans lui, il a toujours déplacé ce capital d’un commerce étranger de consommation avec un pays voisin, pour le jeter dans un pareil commerce avec un pays plus éloigné ; souvent encore, il l’a éloigné d’un commerce étranger de consom­ma­tion par voie directe, pour le jeter dans un pareil commerce fait par circuit ; et enfin, quelques autres fois, il l’a enlevé à toute espèce de commerce étranger de consommation pour le faire entrer dans un commerce de transport. Par conséquent, dans toutes ces circonstances il a détourné cette portion du capital d’une direction dans laquelle elle aurait entretenu une plus grande quantité de travail productif, pour la pousser dans une autre où elle ne peut en entretenir qu’une quantité beaucoup moindre. En outre, en obligeant une si grande portion du commerce et de l’industrie de la Grande-Bretagne à s’assortir uniquement aux convenances d’un marché particu­lier, il a rendu l’ensemble de cette industrie et de ce commerce plus précaire, et moins solidement assuré que si tout leur produit eût été assorti aux besoins et aux demandes d’un plus grand nombre de marchés divers.
Gardons-nous bien cependant de confondre les effets du commerce des colonies avec les effets du monopole de ce commerce. Les premiers sont nécessairement, et, dans tous les cas, bienfaisants ; les autres sont nécessairement et, dans tous les cas nuisibles ; mais les premiers sont tellement bienfaisants, que le commerce des colo­nies, quoique assujetti à un monopole, et malgré tous les effets nuisibles de ce mono­pole, est encore, au total, avantageux et grandement avantageux, quoiqu’il le soit beaucoup moins qu’il ne l’aurait été sans cela.
L’effet du commerce des colonies, dans son état fibre et naturel, c’est d’ouvrir un marché vaste, quoique lointain, pour ces parties du produit de l’industrie anglaise qui peuvent excéder la demande des marchés plus prochains, du marché national, de celui de l’Europe et de celui des pays situés autour de la Méditerranée. Dans son état libre et naturel, le commerce des colonies, sans enlever à ces marchés aucune partie du produit qui leur avait toujours été envoyé, encourage la Grande-Bretagne à augmenter continuellement son excédent de produit, parce qu’il lui présente continuellement de nouveaux équivalents en échange. Dans son état libre et naturel, le commerce des colonies tend à augmenter dans la Grande-Bretagne la quantité du travail productif, mais sans changer en rien la direction de celui qui y était déjà en activité auparavant. Dans l’état libre et naturel du commerce des colonies, la concurrence de toutes les autres nations empêcherait que, sur le nouveau marché ou dans les nouveaux emplois de l’industrie, le taux du profit ne vînt à s’élever au-dessus du niveau commun. Le nouveau marché, sans rien enlever à l’ancien, créerait, pour ainsi dire, un nouveau produit pour son propre approvisionnement ; et ce nouveau produit constituerait un nouveau capital pour faire marcher les nouveaux emplois, qui de même n’auraient pas besoin de rien ôter aux anciens.
Le monopole du commerce des colonies, au contraire, en excluant la concurrence des autres nations, et en faisant hausser ainsi le taux du profit, tant sur le nouveau marché que dans les nouveaux emplois, enlève le produit à l’ancien marché, et le capital aux anciens emplois. Le but que se propose ouvertement le monopole, c’est d’augmenter notre part dans le commerce des colonies au-delà de ce qu’elle serait sans lui. Si notre part dans ce commerce ne devait pas être plus forte avec le mono­pole qu’elle ne l’eût été sans lui, il n’y aurait pas eu de motif pour l’établir. Or, tout ce qui entraîne dans une branche de commerce dont les retours sont plus tardifs et plus éloignés que ceux de la plupart des autres branches une plus forte portion du capital d’un pays que celle qui s’y serait portée d’elle-même, fait nécessairement que la somme totale de travail productif annuellement tenue en activité dans ce pays, que la masse totale du produit annuel des terres et du travail de ce pays, seront moindres qu’elles n’eussent été sans cela. Il retient le revenu des habitants de ce pays au-des­sous du point auquel il s’élèverait naturellement, et diminue par là en eux la faculté d’accumuler. non-seulement il empêche en tout temps que leur capital n’entretienne une aussi grande quantité de travail productif qu’il en ferait subsister, mais il empêche encore que ce capital ne vienne à grossir aussi vite qu’il le pourrait, et par là n’arrive au point d’entretenir une quantité de travail productif encore plus grande.
Néanmoins, les bons effets qui résultent naturellement du commerce des colonies font plus que contre-balancer, pour la Grande-Bretagne, les mauvais effets du mono­pole ; de manière qu’en prenant tous ces effets ensemble, ceux du monopole ainsi que les autres, ce commerce, même tel qu’il se fait à présent, est une circons­tance non-seulement avantageuse, mais encore grandement avantageuse. Le nouveau marché et les nouveaux emplois que le commerce des colonies a ouverts sont d’une beaucoup plus grande étendue que ne l’était cette portion de l’ancien marché et des anciens emplois qui s’est perdue par l’effet du monopole. Le nouveau produit et le nouveau capital qui ont été créés, pour ainsi dire, par le commerce des colonies, entretiennent dans la Grande-Bretagne une plus grande quantité de travail productif que celle qui s’est trouvée paralysée par l’effet de l’absence des capitaux enlevés à ces autres commerces dont les retours sont plus fréquents. Mais si le commerce des colonies, même tel qu’il se pratique aujourd’hui, est avantageux à la Grande-Bretagne, ce n’est assurément pas grâce au monopole, mais c’est malgré le monopole.
Si les colonies ouvrent à l’Europe un nouveau marché, c’est bien moins à son produit brut qu’au produit de ses manufactures. L’agriculture est proprement l’indus­trie des colonies nouvelles, industrie que le bon marché de la terre rend plus avantageuse que toute autre. Aussi abondent-elles en produit brut et, au lieu d’en im­por­ter des autres pays, elles en ont, en général, un immense excédent à exporter. Dans les colonies nouvelles, l’agriculture enlève des bras à tous les autres emplois, ou les détourne de toute autre profession. Il y a peu de bras qu’on puisse réserver pour la fabrication des objets nécessaires ; il n’y en a pas pour celle des objets de luxe. Les colons trouvent mieux leur compte à acheter des autres pays les objets fabriqués de l’un et de l’autre genre, qu’à les fabriquer eux-mêmes. C’est principalement en encou­rageant les manufactures de l’Europe, que le commerce des colonies encourage indi­rectement son agriculture. Les ouvriers des manufactures d’Europe, auxquels ce commerce fournit de l’emploi, forment un nouveau marché pour le produit de la terre, et c’est ainsi qu’un commerce avec l’Amérique se trouve donner en Europe une extension prodigieuse au plus avantageux de tous les marchés, c’est-à-dire au débit intérieur du blé et du bétail, du pain et de la viande de boucherie.
Mais, pour se convaincre que le monopole du commerce avec des colonies bien peuplées et florissantes ne suffit pas seul pour établir ou même pour soutenir des manufactures dans un pays, il ne faut que jeter les yeux sur l’Espagne et le Portugal. L’Espagne et le Portugal étaient des pays à manufactures avant qu’ils eussent aucune colonie considérable ; ils ont l’un et l’autre cessé de l’être depuis qu’ils ont les colonies les plus riches et les plus fertiles du monde. 
En Espagne et en Portugal, les mauvais effets du monopole, aggravés par d’autres causes, ont peut-être, à peu de chose près, fait plus que contre-balancer les bons effets naturels du commerce des colonies ; ces causes, à ce qu’il semble, sont des monopoles de différentes sortes : la dégradation de la valeur de l’or et de l’argent au-dessous de ce qu’est cette valeur dans la plupart des autres pays ; l’exclusion des marchés étran­gers causée par des impôts déraisonnables sur l’exportation, et le rétrécissement du marché intérieur par des impôts encore plus absurdes sur le transport des mar­chandises d’un lieu du royaume à l’autre ; mais, par-dessus toutes choses, c’est cette administration irrégulière et partiale de la justice, qui protège souvent le débiteur riche et puissant contre les poursuites du créancier lésé, ce qui détourne la partie indus­trieuse de la nation de préparer des marchandises pour la consommation de ces grands si hautains auxquels elle n’oserait refuser de vendre à crédit, et dont il serait ensuite si difficile de se faire payer.
En Angleterre, au contraire, les bons effets naturels du commerce des colonies, aidés de plusieurs autres causes, ont surmonté en grande partie les mauvais effets du monopole. Ces causes, à ce qu’il semble, sont la liberté générale du commerce, qui, malgré quelques entraves, est au moins égale et peut-être supérieure à ce qu’elle est dans tout autre pays, la liberté d’exporter, franches de droits, presque toutes les espèces de marchandises qui sont le produit de l’industrie nationale à presque tous les pays étrangers, et ce qui est peut-être d’une plus grande importance encore, la liberté illimitée de les transporter d’un endroit de notre pays à l’autre, sans être obligé de rendre compte à aucun bureau public, sans avoir à essuyer des questions ou des examens d’aucune espèce ; mais, par-dessus tout, c’est cette administration égale et impartiale de la justice qui rend les droits du dernier des sujets de la Grande-Bretagne respectables aux yeux du plus élevé en dignité et qui, par l’assurance qu’elle donne à chacun de jouir du fruit de son travail, répand sur tous les genres quelconques d’industrie le plus grand et le plus puissant de tous les encouragements.
Néanmoins, si le commerce des colonies a favorisé, comme certainement il l’a fait, les manufactures de la Grande-Bretagne, ce n’est pas à l’aide du monopole, mais c’est malgré le monopole. L’effet du monopole n’a pas été d’augmenter la quantité, mais de changer la forme et la qualité d’une partie des ouvrages de manufactures de la Grande-Bretagne, et d’approprier à un marché dont les retours sont éloignés et tardifs ce qui eût été approprié à un marché dont les retours sont fréquents et rapprochés. Par conséquent, son effet a été de déplacer une partie du capital de la Grande-Bretagne d’un emploi dans lequel ce capital aurait entretenu une plus grande quantité d’indus­trie manufacturière, pour le porter dans une autre où il entretient une moindre quan­tité ; et ainsi il a diminué la masse totale d’industrie manufacturière en activité dans la Grande-Bretagne, au lieu de l’augmenter.
Comme tous les autres expédients misérables et nuisibles de ce système mercan­tile que je combats, le monopole du commerce des colonies opprime l’industrie de tous les autres pays, et principalement celle des colonies, sans ajouter le moins du monde à celle du pays en faveur duquel il a été établi, mais au contraire en la dimi­nuant.
Quelle que puisse être, à une époque quelconque, l’étendue du capital de ce pays, le monopole empêche que ce capital n’entretienne une aussi grande quantité de travail productif qu’il ferait naturellement, et qu’il ne fournisse aux habitants vivant de leur industrie un aussi grand revenu que celui qu’il pourrait leur fournir. Or, comme le capital ne peut s’accroître que de l’épargne des revenus, si le monopole l’empêche de produire un aussi grand revenu que celui qu’il aurait pu donner naturellement, il l’empêche nécessairement d’augmenter aussi vite qu’il aurait pu le faire et, par conséquent, d’entretenir une quantité encore plus grande de travail productif, et de produire un revenu encore plus grand aux habitants de ce pays vivant de leur travail. Ainsi, une des grandes sources primitives du revenu, les salaires du travail, devient nécessairement, par l’effet du monopole, moins abondante, dans tous les temps, qu’elle ne l’aurait été.
En faisant hausser le taux des profits mercantiles, le monopole met obstacle à l’amélioration des terres. Le profit de cette amélioration dépend de la différence entre ce que la terre produit actuellement et ce qu’on pourrait lui faire produire au moyen de l’application d’un certain capital. Si cette différence offre un plus gros profit que celui qu’on pourrait retirer d’un pareil capital dans quelque emploi de commerce, alors l’amélioration des terres enlèvera les capitaux à toutes les opérations de commerce. Si le profit est moindre, les entreprises de commerce enlèveront les capi­taux à l’amélioration des terres. Ainsi, tout ce qui fait hausser le taux des profits du commerce doit ou affaiblir la supériorité du profit de l’amélioration des terres, ou augmenter son infériorité, et dans un cas, doit empêcher les capitaux de se porter vers cette amélioration ; dans l’autre, il doit lui enlever les capitaux qui y sont consacrés. Or, en décourageant l’amélioration des terres, le monopole retarde nécessairement l’accroissement naturel d’une autre grande source primitive de revenu, la rente de la terre. D’un autre côté, en faisant hausser le taux des profits, le monopole contribue nécessairement à tenir le taux courant de l’intérêt plus élevé qu’il n’aurait été. Or, le prix capital de la terre relativement à la rente qu’elle rapporte, c’est-à-dire le denier auquel elle se vend, ou le nombre d’années de revenu qu’on paye communément pour acquérir le fonds, baisse nécessairement à mesure que le taux de l’intérêt monte, et monte à mesure que le taux de l’intérêt baisse. Par conséquent, le monopole nuit de deux manières aux intérêts du propriétaire de terre, en retardant l’accroissement naturel, premièrement de sa rente, et secondement du prix relatif qu’il retirerait de sa terre, c’est-à-dire en retardant l’accroissement de la proportion entre la valeur du fonds et celle du revenu qu’il rapporte.
À la vérité, le monopole élève le taux des profits mercantiles, et augmente par ce moyen le gain de nos marchands. Mais, comme il nuit à L’accroissement naturel des capitaux, il tend plutôt à diminuer qu’à augmenter la masse totale du revenu que recueillent les habitants du pays, comme profits de capitaux, un petit profit sur un gros capital donnant un plus grand revenu que ne fait un gros profit sur un petit capital. Le monopole fait hausser le taux du profit, mais il empêche que la somme totale des profits ne monte aussi haut qu’elle aurait fait sans lui.
Toutes les sources primitives de revenu, les salaires du travail, la rente de la terre et les profits des capitaux deviennent donc, par l’effet du monopole, beaucoup moins abondantes qu’elles ne l’auraient été sans lui. Pour favoriser les petits intérêts d’une petite classe d’hommes dans un seul pays, il blesse les intérêts de toutes les autres classes dans ce pays-là, et ceux de tous les hommes dans tous les autres pays.
Si le monopole est devenu ou peut devenir profitable à une classe particulière d’hommes, c’est uniquement par l’effet qu’il a de faire monter le taux ordinaire du profit. Mais, outre tous les mauvais effets que nous avons déjà dit résulter nécessai­rement contre le pays, en général, du taux élevé du profit, il y en a un plus fatal peut-être que tous les autres pris ensemble, et qui se trouve inséparablement lié avec lui, si nous en jugeons par l’expérience. Le taux élevé du profit semble avoir partout l’effet de détruire cet esprit d’économie qui est naturel à l’état de commerçant dans d’autres circonstances. Quand les profits sont élevés, il semble que cette vertu sévère soit deve­nue inutile, et qu’un luxe dispendieux convienne mieux à l’abondance dans laquelle on nage. Or, les propriétaires des grands capitaux de commerce sont néces­sairement les chefs et les directeurs de tout ce qui compose l’industrie d’un pays, et leur exemple a une bien plus grande influence que celui de toute autre classe sur la totalité des habitants vivant de leur travail. Si le maître est économe et rangé, il y a beaucoup à parier que l’ouvrier le sera aussi ; mais s’il est sans ordre et sans conduite, le compagnon, habitué à modeler son ouvrage sur le dessin que lui prescrit son maître, modèlera aussi son genre de vie sur l’exemple que celui-ci lui met sous les yeux. Ainsi, la disposition à l’épargne est enlevée à tous ceux qui y ont naturellement le plus de penchant ; et le fonds destiné à entretenir le travail productif ne reçoit point d’augmentation par les revenus de ceux qui devraient naturellement l’augmenter le plus. Le capital du pays fond successivement au lieu de grossir, et la quantité de travail productif qui y est entretenue devient moindre de jour en jour. Les profits énormes des négociants de Cadix et de Lisbonne ont-ils augmenté le capital de l’Espagne et du Portugal[22] ? Ont-ils été de quelque secours à la pauvreté de ces deux misérables pays ? En ont-ils animé l’industrie ? La dépense des gens de commerce est montée sur un si haut ton dans ces deux villes commerçantes, que ces profits exorbitants, bien loin d’ajouter au capital général du pays, semblent avoir à peine suffi à entretenir le fonds des capitaux qui les ont produits. Les capitaux étrangers pénètrent de plus en plus journellement, comme des intrus, pour ainsi dire, dans le commerce de Cadix et de Lisbonne. C’est pour chasser ces capitaux étrangers d’un commerce à l’entretien duquel leur propre capital devient de jour en jour moins en état de suffire, que les Espagnols et les Portugais tâchent, à tout moment, de resserrer de plus en plus les liens si durs de leur absurde monopole. Que l’on compare les mœurs du commerce à Cadix et à Lisbonne avec celles qu’il nous montre à Ams­terdam, et on sentira combien les profits exorbitants ou modéré affectent différem­ment le caractère et la conduite des commerçants. Les négociants de Londres, il est vrai, ne sont pas encore devenus, en général, d’aussi magnifiques seigneurs que ceux de Cadix et de Lisbonne, mais ils ne sont pas non plus, en général, des bourgeois ran­gés et économes, comme les négociants d’Amsterdam. Cependant, plusieurs d’entre eux passent pour être de beaucoup plus riches que la plupart des premiers, et pas tout à fait aussi riches que beaucoup de ces derniers. Mais le taux de leur profit est d’ordinaire bien plus bas que celui des premiers, et de beaucoup plus élevé que celui des autres. Ce qui vient vite s’en va de même, dit le proverbe ; et c’est bien moins sur le moyen réel qu’on a de dépenser, que sur la facilité avec laquelle on voit venir l’argent, qu’on règle partout, à ce qu’il semble, le ton de sa dépense.
C’est ainsi que l’unique avantage que le monopole procure à une classe unique de personnes est, de mille manières différentes, nuisible à l’intérêt général du pays.
Aller fonder un vaste empire dans la vue seulement de créer un peuple d’acheteurs et de chalands, semble, au premier coup d’œil, un projet qui ne pourrait convenir qu’à une nation de boutiquiers. C’est cependant un projet qui accommoderait extrêmement mal une nation toute composée de gens de boutique, mais qui convient parfaitement bien à une nation dont le gouvernement, est sous l’influence des boutiquiers. Il faut des hommes d’État de cette espèce, et de cette espèce seulement, pour être capable de s’imaginer qu’ils trouveront de l’avantage à employer le sang et les trésors de leurs concitoyens à fonder et à soutenir un pareil empire. Allez dire à un marchand tenant boutique : Faites pour moi l’acquisition d’un bon domaine, et moi j’achèterai toujours mes habits à votre boutique, quand je devrais même les payer un peu plus cher que chez les autres ; vous ne lui trouverez pas un grand empressement à accueillir votre proposition. Mais si quelque autre personne consentait à acheter un pareil domaine pour vous, le marchand serait fort aise qu’on imaginât de vous imposer la condition d’acheter tous vos habits à sa boutique. L’Angleterre a acheté un vaste domaine dans un pays éloigné, pour quelques-uns de ses sujets qui ne se trouvaient pas commo­dément chez elle. Le prix n’en a pas été, à la vérité, bien cher, et au lieu de payer ce fonds au denier 30 du produit, qui est à présent le prix courant des terres, elle n’a eu guère autre chose à donner que la dépense des différents équipements des vaisseaux qui ont fait la première découverte, qui ont reconnu la côte, et qui ont pris une possession fictive du pays. La terre était bonne et fort étendue, et les cultivateurs, ayant en abondance de bons terrains à faire valoir, et étant restés un certain temps les maîtres de vendre leur produit partout où il leur plaisait, sont devenus, dans l’espace de trente ou quarante ans à peu près (entre 1620 et 1660), si nombreux et si prospères, que les gens de boutique et autres industriels et commerçants de l’Angleterre ont conçu l’envie de s’assurer le monopole de leur pratique. Ainsi, quoiqu’ils ne préten­dis­sent pas avoir rien payé ou pour l’acquisition primitive du fonds, ou pour les dépenses postérieures de l’amélioration, ils n’en ont pas moins présenté au parlement leur pétition, tendant à ce que les cultivateurs de l’Amérique fussent à l’avenir bornés à leur seule boutique, d’abord pour y acheter toutes les marchandises d’Europe dont ils auraient besoin, et secondement pour y vendre toutes les différentes parties de leur produit que ces marchands jugeraient à propos d’acheter ; car ils ne pensaient pas qu’il leur convînt d’acheter toutes les espèces de produits de ce pays. Il y en avait certaines qui, importées en Angleterre, auraient pu faire concurrence à quelqu’un des trafics qu’ils y faisaient eux-mêmes. Aussi, quant à ces espèces particulières, ils ont consenti volontiers que les colons les vendissent où ils pourraient ; le plus loin était le meilleur ; et pour cette raison ils ont proposé que ce marché fût borné aux pays situés au sud du cap Finistère. Ces propositions, vraiment dignes de boutiquiers, ont passé en loi par une clause insérée dans le fameux Acte de navigation.
Jusqu’à présent, le soutien de ce monopole a été le principal, ou, pour mieux dire, peut-être le seul but et le seul objet de l’empire que la Grande-Bretagne s’est attribué sur ces colonies. C’est dans le commerce exclusif, à ce qu’on suppose, que consiste le grand avantage de provinces qui jamais encore n’ont fourni ni revenu ni force mili­taire pour le soutien du gouvernement civil ou pour la défense de la mère patrie. Le monopole est le signe principal de leur dépendance, et il est le seul fruit qu’on ait recueilli jusqu’ici de cette dépendance. Dans le fait, toute la dépense que la Grande-Bretagne a pu faire jusqu’à ce moment pour maintenir cette dépendance a été consa­crée au soutien de ce monopole. Avant le commencement des troubles actuels, la dépense de l’établissement ordinaire des colonies pendant la paix consistait dans la solde de vingt régiments d’in­fan­terie, dans les frais d’artillerie, de munitions et de provisions extraordinaires qu’exigeait leur entretien, et dans les frais d’une force navale très-considérable, cons­tamment sur pied, pour garder les côtes immenses de l’Amérique septentrionale et celle de nos îles des Indes occidentales contre les navires de contrebande des autres nations. La dépense totale de cet établissement pendant la paix était à la charge du revenu de la Grande-Bretagne, et pendant cette époque ce n’a été encore que la moindre partie de ce qu’a coûté à la métropole sa domination sur les colonies. Si nous voulons avoir une idée du total de ces dépenses, il faut ajouter à la dépense annuelle de cet établissement l’intérêt des sommes que la Grande-Bretagne a employées, en plusieurs occasions, pour leur défense, par suite de l’habitude qu’elle avait prise de considérer ses colonies comme des provinces sujettes de son empire. Il faut y ajouter en particulier la dépense totale de la dernière guerre, et une grande partie de celle de la guerre précédente. La dernière guerre fut absolument une querelle de colonies, et c’est avec raison qu’on doit porter au compte des colonies toutes les dépenses qu’elle a pu entraîner, en quelque partie du monde que ses dépenses aient été faites, en Allemagne ou aux Indes orientales. Elles forment un objet de plus de 90 millions sterling, en comprenant non-seulement la nouvelle dette qui a été contractée, mais les deux schellings pour livre additionnels à la taxe foncière et les sommes qu’on a em­prun­tées chaque année sur le fonds d’amortissement. La guerre d’Espagne, com­mencée en 1739, était principalement une querelle de colonies. Son premier objet était d’empêcher la visite (recherche) des navires de la colonie, qui faisaient un commerce interlope avec le continent espagnol. Toute cette dépense n’est dans le fait qu’une prime accordée pour soutenir un monopole. On supposait qu’elle avait pour but d’encourager les manufactures de la Grande-Bretagne et d’étendre son commerce ; mais son effet réel a été de faire hausser les taux des profits du commerce, et de mettre nos marchands à même de reporter dans une branche de commerce, dont les retours sont plus lents et plus éloignés que ceux de la majeure partie des autres trafics, une plus forte portion de leur capital qu’ils n’auraient fait sans cela ; deux effets tels, qu’on eût peut-être mieux fait de donner la prime pour les prévenir, si une prime avait pu le faire.
Ainsi, avec le système actuel d’administration adopté par la Grande-Bretagne pour ses colonies, l’empire qu’elle s’attribue sur elles n’est pour elle qu’une source de pertes et de désavantages[23]. 
Proposer que la Grande-Bretagne abandonne volontairement toute autorité sur ses colonies, qu’elle les laisse élire leurs magistrats, se donner des lois et faire la paix et la guerre comme elles le jugeront à propos, ce serait proposer une mesure qui n’a jamais été et ne sera jamais adoptée par aucune nation du monde. jamais nation n’a abandonné volontairement l’empire d’une province, quelque embarras qu’elle pût trouver à la gouverner, et quelque faible revenu que rapportât cette province proportionnellement aux dépenses qu’elle entraînait.
Si de tels sacrifices sont bien souvent conformes aux intérêts d’une nation, ils sont toujours mortifiants pour son orgueil, et ce qui est peut-être encore d’une plus grande conséquence, ils sont toujours contraires à l’intérêt privé de la partie qui gouverne, laquelle se verrait par là enlever la disposition de plusieurs places honorables et lucra­tives, de plusieurs occasions d’acquérir de la richesse et des distinctions, avantages que ne manque guère d’offrir la possession des provinces les plus turbulentes et les plus onéreuses pour le corps de la nation. À peine si le plus visionnaire de tous les enthousiastes serait capable de proposer une pareille mesure avec quelque espérance sérieuse de la voir jamais adopter. Si pourtant elle était adoptée, non-seulement la Grande-Bretagne se trouverait immédiatement affranchie de toute la charge annuelle de l’entretien des colonies, mais elle pourrait encore faire avec elles un traité de commerce fondé sur des bases propres à lui assurer de la manière la plus solide un commerce libre, moins lucratif pour les marchands, mais plus avantageux au corps du peuple, que le monopole dont elle jouit à présent. En se séparant ainsi de bonne amitié, l’affection naturelle des colonies pour leur mère patrie, ce sentiment que nos dernières divisions ont peut-être presque entièrement éteint, reprendrait bien vite sa force. Il les disposerait non-seulement à respecter, pendant une suite de siècles, ce traité de commerce conclu avec nous au moment de la séparation, mais encore à nous favoriser dans les guerres aussi bien que dans le commerce et, au lieu de sujets turbulents et factieux, à devenir nos alliés les plus fidèles, les plus généreux et les plus affectionnés. On verrait revivre entre la Grande-Bretagne et ses colonies cette même espèce d’affection paternelle d’un côté et de respect filial de l’autre, qui avait coutume de régner entre celles de l’ancienne Grèce et la métropole dont elles étaient descendues[24].
Pour qu’une province devienne avantageuse à l’empire auquel elle appartient, il faut qu’elle fournisse en temps de paix à l’État un revenu qui suffise non-seulement à défrayer la dépense totale de son propre établissement pendant la paix, mais encore à contribuer au soutien du gouvernement général de l’empire. Chaque province contribue nécessairement, plus ou moins, à augmenter la dépense de ce gouvernement général. Ainsi, si une province particulière ne contribue pas, pour sa portion, à défrayer cette dépense, alors il Lut que la charge retombe inégalement sur quelque autre partie de l’empire. Par une raison semblable aussi, le revenu extraordinaire que chaque province fournit à l’État en temps de guerre doit être, avec le revenu extra­ordinaire de la totalité de l’empire, dans la même proportion que le revenu ordinaire qu’elle a à fournir en temps de paix. Or, on n’aura pas de peine à convenir que ai le revenu ordinaire ni le revenu extraordinaire que la Grande-Bretagne retire de ses colonies ne sont dans cette proportion avec le revenu total de l’empire britannique. Il est vrai qu’on a prétendu que le monopole, en augmentant les revenus privés des particuliers de la Grande-Bretagne, et les mettant par là en état de payer de plus forts impôts, compense le déficit dans le revenu public des colonies. Mais j’ai tâché de faire voir que ce monopole, quoiqu’il soit un impôt très-onéreux sur les colonies, et quoiqu’il puisse augmenter le revenu d’une classe particulière d’individus de la Grande-Bretagne, diminue toutefois, au lieu de l’augmenter, le revenu de la masse du peuple et, par conséquent, retranche, bien loin d’y ajouter, aux moyens que peut avoir le peuple de payer des impôts. Et puis, les hommes dont le monopole augmente les revenus constituent une classe particulière qu’il est absolument impossible d’imposer au-delà de la proportion des autres classes, et qu’il est à la fois extrêmement impo­litique de vouloir imposer au-delà de cette proportion, comme je tâcherai de le faire voir dans le livre suivant[25]. Il n’y a donc aucune ressource particulière à tirer de cette classe. 
Les colonies peuvent être imposées ou par leurs propres assemblées, ou par le parlement de la Grande-Bretagne.
Il ne paraît pas très-probable qu’on puisse jamais amener les assemblées colo­niales à lever sur leurs commettants un revenu public qui suffise, non-seulement à entretenir en tout temps l’établissement civil et militaire des colonies, mais à payer encore leur juste proportion dans la dépense du gouvernement général de l’empire britannique. Bien que le parlement d’Angleterre soit immédiatement placé sous les yeux du souverain, il s’est encore passé beaucoup de temps avant qu’on en ait pu venir à le rendre assez docile ou assez libéral dans les subsides à l’égard du gouvernement pour soutenir les établissements civils et militaires de son propre pays comme il convient qu’ils le soient. Pour manier le parlement d’Angleterre lui-même jusqu’au point de l’amener là, il n’y a pas eu d’autre moyen que de distribuer entre les membres de ce corps une grande partie des places provenant de ces établissements civils et militaires, ou de laisser ces places à leur disposition. Mais quant aux assemblées coloniales, quand même le souverain aurait les mêmes moyens de s’y ménager cette influence permanente, la distance où elles sont de ses yeux, leur nombre, leur situa­tion dispersée et la variété de leurs constitutions lui rendraient cette tâche extrême­ment difficile ; et d’ailleurs, ces moyens n’existent pas. Il serait impossible de distri­buer entre tous les membres les plus influents de toutes les assemblées coloniales une part dans les places ou dans la disposition des places dépendant du gouvernement général de l’empire britannique, assez importante pour les engager à sacrifier leur popularité chez eux et à charger leurs commettants de contributions pour le soutien de ce gouvernement général, dont presque tous les émoluments se partagent entre des gens qui leur sont tout à fait étrangers. D’un autre côté, l’ignorance inévitable où serait l’administration sur l’importance relative de chacun des différents membres de ces différentes assemblées la mettrait dans le cas de les choquer très-souvent, et de commettre perpétuellement des bévues dans les mesures qu’elle tenterait pour les diriger de cette manière ; ce qui paraît rendre un pareil plan de conduite totalement impraticable à leur égard.
D’ailleurs, les assemblées coloniales ne peuvent être en état de juger ce qu’exigent la défense et le soutien de tout l’empire. Ce n’est pas à elles qu’est confié le soin de cette défense et de ce soutien. Ce n’est pas là leur fonction, et elles n’ont aucune voie constante et légale de se procurer à cet égard les informations nécessaires. L’assem­blée d’une province, comme la fabrique d’une paroisse, peut juger très-convenable­ment de ce qui est relatif aux affaires de son district particulier, mais elle ne peut pas avoir de moyens pour juger de ce qui est relatif à celles de l’ensemble de l’empire. Elle ne peut pas même bien juger de la proportion de sa propre province avec la totalité de l’empire, ou bien du degré relatif de richesse et d’importance de cette pro­vince par rapport aux autres, puisque ces autres provinces ne sont pas sous l’ins­pec­tion et la surintendance de l’assemblée provinciale. Pour juger de ce qui est nécessaire à la défense et au soutien de l’ensemble de l’empire, et dans quelle propor­tion chaque partie du tout doit contribuer, il faut absolument l’œil de cette assemblée qui a l’inspection et la surintendance des affaires de tout l’empire. On a proposé, en conséquence, de taxer les colonies par réquisition, le parlement de la Grande-Bretagne déterminant la somme que chaque colonie aurait à payer, et l’assemblée provinciale faisant la répartition et la levée de cette somme de la manière qui conviendrait le mieux à la situation particulière de la province. De cette manière, la chose qui intéresserait l’ensemble de l’empire serait déterminée par l’assemblée qui a l’inspection et la surintendance des affaires de tout l’empire, tandis que les conve­nan­ces locales et les intérêts particuliers de chaque colonie se trouveraient toujours réglés par sa propre assemblée. Quoique, dans ce cas, les colonies n’eussent pas de représentants dans le parlement britannique, cependant, si nous en jugeons par l’expé­rience, il n’y a pas de probabilité que la réquisition parlementaire fût déraison­nable. Dans aucune occasion, le parlement d’Angleterre n’a montré la moindre dispo­sition à surcharger les parties de l’empire qui ne sont pas représentées dans le parlement. Les îles de jersey et de Guernesey, qui n’ont aucun moyen de résister à l’autorité du parlement, sont taxées plus modérément qu’aucun endroit de la Grande-Bretagne. Lorsque le parlement a essayé d’imposer les colonies, il n’a jusqu’à présent jamais exigé d’elles rien qui approchât même de la juste proportion de ce qui était payé par les habitants de la mère patrie. D’ailleurs, si la contribution des colonies était telle qu’elle dût monter ou baisser à proportion que viendrait à monter ou baisser la taxe foncière, le parlement ne pourrait les taxer sans taxer en même temps ses propres com­­mettants et, dans ce cas-là, les colonies pourraient se regarder comme virtuelle­ment représentées dans le parlement. 
Il ne manque pas d’exemples d’empires dans lesquels toutes les différentes pro­vinces ne sont pas taxées, si je puis m’exprimer ainsi, en une seule masse, mais où le souverain, ayant déterminé la somme que doit payer chacune des différentes pro­vin­ces, en fait l’assiette et la perception dans quelques-unes suivant le mode qu’il juge convenable, tandis que, dans d’autres, il laisse faire l’assiette et la perception de leur contingent d’après la détermination des états respectifs de chacune d’elles.
Dans certaines provinces de France, non-seulement le roi impose telles sommes qu’il juge à propos, mais encore il en fait l’assiette et la perception de la manière qu’il lui plaît d’adopter. Dans d’autres provinces, il demande une certaine somme, mais il laisse aux états de chacune de ces provinces à asseoir et à lever cette somme comme ils le jugent convenable. Dans le plan proposé de taxer par réquisition, le parlement de la Grande-Bretagne se trouverait à peu près dans la même situation, à l’égard des assemblées coloniales, que celle du roi de France à l’égard des états de ces provinces qui jouissent encore du privilège d’avoir leurs États particuliers, et qui sont les provinces de France qui passent pour être le mieux gouvernées.
Mais si, dans ce projet, les colonies n’ont aucun motif raisonnable de craindre que leur part des charges publiques excède jamais la juste proportion de ce qu’en suppor­tent leurs compatriotes européens, la Grande-Bretagne pourrait avoir, elle, des motifs fondés de craindre que cette part n’atteignît jamais à la hauteur de cette juste propor­tion. Le parlement de la Grande-Bretagne n’a pas sur les colonies une autorité établie de longue main, telle que celle qu’a le roi de France sur ses provinces, qui ont conser­vé le privilège d’avoir leurs états particuliers. Si les assemblées coloniales n’étaient pas très-favorablement disposées (et à moins qu’elles ne soient maniées avec beau­coup plus d’adresse qu’on n’y en a mis jusqu’à présent, il est très-probable qu’elles ne le seraient pas), elles trouveraient toujours mille prétextes pour rejeter ou pour éluder les réquisitions les plus raisonnables du parlement. Qu’une guerre avec la France, je suppose, vienne à éclater, il faut lever immédiatement 10 millions pour défendre le siège de l’empire. Il faut emprunter cette somme sur le crédit de quelque fonds parle­men­taire destiné au payement des intérêts. Le parlement propose de créer une partie de ce fonds par un impôt à lever dans la Grande-Bretagne, et une partie par une réqui­si­tion aux différentes assemblées coloniales de l’Amérique et des Indes occidentales. Or, je le demande, se presserait-on beaucoup d’avancer son argent sur le crédit d’un fonds qui dépendrait en partie des bonnes dispositions de ces assemblées, toutes extrêmement éloignées du siège de la guerre, et quelquefois peut-être ne se regardant pas comme fort intéressées aux résultats de cette guerre ? Vraisemblablement on n’avancerait guère sur un tel fonds plus d’argent que la somme présumée devoir être produite par l’impôt à lever dans la Grande-Bretagne. Tout le poids de la dette contractée pour raison de la guerre tomberait ainsi, comme il a toujours fait jusqu’à présent, sur la Grande-Bretagne, sur une partie de l’empire, et non sur la totalité de l’empire. La Grande-Bretagne est peut-être le seul État, depuis que le monde existe, qui, à mesure qu’il a agrandi son domaine, ait seulement ajouté à ses dépenses sans augmenter une seule fois ses ressources. Les autres États, en général, se sont déchargés sur leurs provinces sujettes et subordonnées de la partie la plus considérable des dépenses de la souveraineté. Jusqu’à présent, la Grande-Bretagne a souffert que ses provinces sujettes et subordonnées se déchargeassent sur elle de presque toute cette dépense. Pour mettre la Grande-Bretagne sur un pied d’éga­­lité avec ses colonies, que la loi a supposées jusqu’ici provinces sujettes et su­bor­don­nées, il paraît nécessaire, dans le projet de les imposer par réquisition parle­men­taire, que le parlement ait quelques moyens de donner un effet sûr et prompt à ses réquisitions, dans le cas où les assemblées coloniales chercheraient à les rejeter ou à les éluder. Or, quels sont ces moyens ? C’est ce qu’on n’a pas encore dit jusqu’à présent, et c’est ce qu’il n’est pas trop aisé d’imaginer.
En même temps, si le parlement de la Grande-Bretagne venait jamais à être en plei­ne possession du droit d’imposer les colonies, indépendamment même du consen­tement de leurs propres assemblées, dès ce moment l’importance de ces assemblées serait détruite, et avec elle celle de tous les hommes influents de l’Amérique anglaise. Les hommes désirent avoir part au maniement des affaires publiques, principalement pour l’importance que cela leur donne. C’est du plus ou moins de pouvoir que la plupart des meneurs (les aristocrates naturels du pays) ont de conserver ou de défen­dre leur importance respective, que dépendent la stabilité et la durée de toute consti­tution libre. C’est dans les attaques que ces meneurs sont continuellement occu­pés à livrer à l’importance l’un de l’autre, et dans la défense de leur propre importance, que consiste tout le jeu des factions et de l’ambition domestique. Les meneurs de l’Amé­rique, comme ceux de tous les autres pays, désirent conserver leur importance per­sonnelle. Ils sentent ou au moins ils s’imaginent que si leurs assemblées, qu’ils se plaisent à décorer du nom de parlements, et à regarder comme égales en autorité au parlement de la Grande-Bretagne, allaient être dégradées au point de devenir les officiers exécutifs et les humbles ministres de ce parlement, ils perdraient eux-mêmes à peu près toute leur importance personnelle. Aussi ont-ils rejeté la proposition d’être imposés par réquisition parlementaire, et comme tous les autres hommes ambitieux qui ont de l’élévation et de l’énergie, ils ont tiré l’épée pour maintenir leur importance.
Vers l’époque du déclin de la république romaine, les alliés de Rome, qui avaient porté la plus grande partie du fardeau de la défense de l’État et de l’agrandissement de l’empire, demandèrent à être admis à tous les privilèges de citoyens romains. Le refus qu’ils essuyèrent fit éclater la guerre sociale. Pendant le cours de cette guerre, Rome accorda le droit de citoyen à la plupart d’entre eux, un à un, et à mesure qu’ils se détachaient de la confédération générale. Le parlement d’Angleterre insiste pour taxer les colonies ; elles se refusent à l’être par un parlement où elles ne sont pas repré­sen­tées. Si la Grande-Bretagne consentait à accorder à chaque colonie qui se détacherait de la confédération générale un nombre de représentants proportionné à sa portion contributive dans le revenu public de l’empire (cette colonie étant alors soumise aux mêmes impôts et, par compensation, admise à la même liberté de commerce que ses co-sujets d’Europe), avec la condition que le nombre de ses représentants augmen­terait à mesure que la proportion de sa contribution viendrait à augmenter par la suite, alors on offrirait par ce moyen aux hommes influents de chaque colonie une nouvelle route pour aller à l’importance, un objet d’ambition nouveau et plus éblouissant. Au lieu de perdre leur temps à courir après les petits avantages de ce qu’on peut appeler le jeu mesquin d’une faction coloniale, ils pourraient alors, d’après cette bonne opi­nion que les hommes ont naturellement de leur mérite et de leur bonheur, se flatter de l’espoir de gagner quelque lot brillant à cette grande loterie d’État que forment les institutions politiques de la Grande-Bretagne. À moins qu’on n’emploie cette méthode (et il paraît difficile d’en imaginer de plus simple), ou enfin quelque autre qui puisse conserver aux meneurs de l’Amérique leur importance et contenter leur ambition, il n’y a guère de vraisemblance qu’ils veuillent jamais se soumettre à nous de bonne grâce ; et nous ne devons jamais perdre de vue que le sang, que chaque goutte de sang qu’il faudra répandre pour les y contraindre, sera toujours ou le sang de nos conci­toyens, ou le sang de ceux que nous désirons avoir pour tels. Ils voient bien mal, ceux qui se flattent que dans l’état où en sont venues les choses il sera facile de conquérir nos colonies par la force seule. Les hommes qui dirigent aujourd’hui les résolutions de ce qu’ils appellent leur congrès continental se sentent, dans ce moment, un degré d’importance que ne se croient peut-être pas les sujets de l’Europe les plus hauts en dignité. De marchands, d’artisans, de procureurs, les voilà devenus hommes d’État et législateurs ; les voilà employés à fonder une nouvelle constitution pour un vaste empire qu’ils croient destiné à devenir, et qui en vérité paraît bien être fait pour devenir un des plus grands empires et des plus formidables qui aient jamais été au monde. Cinq cents différentes personnes peut-être, qui agissent immédiatement sous les ordres du congrès continental, et cinq cent mille autres qui agissent sous les ordres de ces cinq cents, tous sentent également leur importance personnelle augmentée. Presque chaque individu du parti dominant en Amérique remplit à présent, dans son imagination, un poste supérieur non-seulement à tout ce qu’il a pu être auparavant, mais même à tout ce qu’il avait jamais pu s’attendre à devenir ; et à moins que quelque nouvel objet d’ambition ne vienne s’offrir à lui ou à ceux qui le mènent, pour peu qu’il ait le cœur d’un homme, il mourra à la défense de ce poste.
C’est une observation du président Hénault que nous recherchons aujourd’hui avec curiosité et que nous lisons avec intérêt une foule de petits faits de l’histoire de la Ligue, qui alors ne faisaient peut-être pas une grande nouvelle dans le monde. Mais alors, dit-il, chacun se croyait un personnage important, et les mémoires sans nombre qui nous ont été transmis de ces temps-là ont, pour la plupart, été écrits par des gens qui aimaient à conserver soigneusement et à relever les moindres faits, parce qu’ils se flattaient d’avoir joué un grand rôle dans ces événements. On sait quelle résistance opiniâtre fit la ville de Paris dans cette occasion, et quelle horrible famine elle sup­por­ta plutôt que de se soumettre au meilleur des rois de France, au roi qui, par la suite, fut le plus chéri. La plus grande partie des citoyens, ou ceux qui en gouver­naient la plus grande partie, se battaient pour maintenir leur importance personnelle, dont ils prévoyaient bien le terme au moment où l’ancien gouvernement viendrait à être rétabli. À moins que l’on n’amène nos colonies à consentir à une union, il est très-probable qu’elles se défendront contre la meilleure des mères patries avec autant d’opiniâtreté que s’est défendu Paris contre un des meilleurs rois.
La représentation était une idée inconnue dans les temps anciens. Quand les gens d’un État étaient admis au droit de citoyen dans un autre, ils n’avaient pas d’autre manière d’exercer ce droit que de venir en corps voter et délibérer avec le peuple de cet autre État. L’admission de la plus grande partie des habitants de l’Italie aux privi­lèges de citoyen romain amena la ruine totale de la république. Il ne fut plus possible de distinguer celui qui était citoyen romain de celui qui ne l’était pas. Une tribu ne pouvait plus reconnaître ses membres. Un ramas de populace de toute espèce s’intro­duisit dans les assemblées nationales ; il lui fut aisé d’en chasser les véritables citoyens et de décider des affaires, comme s’il eût composé lui-même la république. Mais quand l’Amérique aurait à nous envoyer cinquante ou soixante nouveaux représentants au parlement, l’huissier de la Chambre des communes n’aurait pas pour cela plus de peine à distinguer un membre de la chambre d’avec quelqu’un qui ne le serait pas. Ainsi, quoique la constitution de la république romaine ait dû nécessai­re­ment trouver sa ruine dans l’union de Rome avec les États d’Italie, ses alliés, il n’y a pas pour cela la moindre probabilité que la constitution britannique ait quelque échec à redouter de l’union de la Grande-Bretagne avec ses colonies. Cette union, au con­trai­re, serait le complément de la constitution, qui, sans cela, paraîtra toujours impar­faite. L’assemblée qui délibère et prononce sur les affaires de chaque partie de l’empire devrait certainement, pour être convenablement éclairée, avoir des représen­tants de chacune de ces parties. je ne prétends pourtant pas dire que cette union soit une chose très-facile à réaliser, ou que l’exécution ne présente pas des difficultés et de grandes difficultés. Toutefois, je n’en ai entendu citer aucune qui paraisse insurmon­table. Les principales ne viennent pas peut-être de la nature des choses, mais des opinions et des préjugés qui dominent tant de ce côté-ci que de l’autre de l’océan Atlantique.
De ce côté, nous avons peur que le grand nombre de représentants que donnerait l’Amérique ne vînt à détruire l’équilibre de la constitution, en ajoutant trop ou à l’in­fluence de la couronne sur l’un des côtés de la balance, ou à la force de la démo­cra­tie sur l’autre. Mais si le nombre des représentants de l’Amérique était propor­tionné au produit des contributions en Amérique, alors le nombre des gens à ménager et à se concilier augmenterait précisément dans la même proportion que les moyens de le faire ; et d’un autre côté, les moyens pour gagner des suffrages augmenteraient en proportion du nombre des nouveaux votants qu’on serait obligé de se concilier. La partie monarchique et la partie démocratique de la constitution resteraient donc, à l’égard l’une de l’autre, après l’union, précisément au même degré de force relative où elles étaient auparavant.
Les gens de l’autre côté de la mer Atlantique ont peur que leur distance du siège du gouvernement ne les expose à une foule d’oppressions ; mais leurs représentants dans le parlement, qui dès le principe ne laisseraient pas d’être fort nombreux, se­raient bien en état de les protéger contre toute entreprise de ce genre. La distance ne pourrait pas affaiblir beaucoup la dépendance des représentants à l’égard de leurs commettants, et les premiers sentiraient toujours bien que c’est à la bonne volonté des autres qu’ils sont redevables de l’honneur de siéger au parlement et de tous les avantages qui en résultent. Il serait donc de l’intérêt des représentants d’entretenir cette bonne volonté, en se servant de tout le poids que leur donnerait le caractère de membres de la législature, pour faire réprimer toute vexation commise dans ces lieux reculés de l’empire par quelque officier civil ou militaire. D’ailleurs, les habitants de l’Amérique se flatteraient, et ce ne serait pas non plus sans quelque apparence de raison, que la distance où se trouve aujourd’hui l’Amérique du siège du gouvernement pourrait bien ne pas être d’une très-longue durée. Les progrès de ces contrées en industrie, en richesse et en population ont été tels jusqu’à présent, que, dans le cours peut-être d’un peu plus d’un siècle, le produit des contributions d’Amérique pourrait excéder celui des contributions de la Grande-Bretagne. Naturellement alors, le siège de l’empire se transporterait dans la partie qui contribuerait le plus à la défense générale et au soutien de l’État.
La découverte de l’Amérique et celle d’un passage aux Indes orientales par le cap de Bonne-Espérance sont les deux événements les plus remarquables et les plus importants dont fassent mention les annales du genre humain ; ils ont déjà produit de bien grands effets. Mais dans le court espace de deux à trois siècles qui s’est écoulé depuis que ces découvertes ont été faites, il est impossible qu’on aperçoive encore toute l’étendue des conséquences qu’elles doivent amener à leur suite. Aucune sagesse humaine ne peut prévoir quels bienfaits ou quelles infortunes ces deux grands événements préparent aux hommes dans la suite des temps. Par l’union qu’ils ont établie en quelque sorte entre les deux extrémités du monde, par les moyens qu’ils leur ont donnés de pourvoir mutuellement aux besoins l’une de l’autre, d’augmenter réciproquement leurs jouissances et d’encourager de part et d’autre leur industrie, il paraîtrait que leur tendance générale doit être bienfaisante. Il est vrai que, pour les naturels des Indes orientales et occidentales, les avantages com­mer­ciaux qui peuvent avoir été le fruit de ces découvertes ont été perdus et noyés dans un océan de calamités qu’elles ont entraînées après elles. Toutefois, ces cala­mités semblent avoir été plutôt un effet accidentel que le résultat naturel de ces grands événements. À l’époque particulière où furent faites ces découvertes, la supé­riorité de forces se trouva être si grande du côté des Européens, qu’ils se virent en état de commettre impunément toutes sortes d’injustices dans ces contrées reculées. Peut-être que dans la suite des temps les naturels de ces contrées deviendront plus forts ou ceux de l’Europe plus faibles, de sorte que les habitants de toutes les différentes parties du monde arriveraient à cette égalité de forces et de courage qui, par la crainte réciproque qu’elle inspire, peut seule contenir l’injustice des nations indépendantes, et leur faire sentir une sorte de respect des droits les unes des autres. Or, il n’y a rien qui paraisse plus propre à établir une telle égalité de forces que cette communication mutuelle des connaissances et des moyens de perfection de tous les genres, qui est la suite naturelle ou plutôt nécessaire d’un vaste et immense commerce de tous les pays du monde avec tous les pays du monde.
En même temps aussi, un des principaux effets de ces découvertes a été d’élever le système mercantile à un degré de splendeur et de gloire auquel il ne serait jamais arrivé sans elles. L’objet de ce système est d’enrichir une grande nation, plutôt par le commerce et les manufactures que par la culture et l’amélioration des terres, plutôt par l’industrie des villes que par celle des campagnes. Or, par une conséquence de ces découvertes, les villes commerçantes de l’Europe, au lieu d’être les manufacturiers et les voituriers seulement d’une très-petite partie du monde (cette partie de l’Europe qui est baignée par l’océan Atlantique, et les pays voisins des mers Baltique et Méditerra­née), sont devenues maintenant les manufacturiers des cultivateurs nombreux et florissants de l’Amérique ; elles sont devenues les voituriers et, à quelques égards aussi, les manufacturiers de presque toutes les différentes nations de l’Asie, de l’Afri­que et de l’Amérique. Deux mondes nouveaux ont été ouverts à leur industrie, chacun desquels est beaucoup plus vaste et plus étendu que l’ancien, et dont un lui offre un marché qui s’agrandit encore tous les jours de plus en plus.
Les pays qui possèdent les colonies de l’Amérique et qui commercent directement avec les Indes orientales jouissent, à la vérité, de tout l’appareil et de la splendeur de ce vaste commerce. Néanmoins d’autres pays, en dépit de toutes les barrières jalouses qu’on a élevées dans le dessein de les en exclure, jouissent bien souvent d’une part plus grande dans ses avantages réels. Les colonies de l’Espagne et du Portugal, par exemple, donnent plus d’encouragement réel à l’industrie de quelques autres pays, qu’elles n’en donnent à celle de l’Espagne et du Portugal. Pour le seul article des toi­les, on dit (mais je ne prétends pas garantir la quantité) que la consommation de ces colonies s’élève à plus de trois millions sterling par an. Or, cette énorme consom­mation est presque en entier fournie par la France, la Flandre, la Hollande et l’Allemagne. L’Espagne et le Portugal n’en fournissent qu’une très-petite partie. Le capital employé à pourvoir les colonies de cette grande quantité de toile se distribue annuellement parmi les habitants de ces contrées, et leur forme un revenu. Les profits seuls de ce capital se dépensent en Espagne et en Portugal, où ils servent à soutenir le faste et la prodigalité des marchands de Cadix et de Lisbonne.
Les mesures mêmes et les règlements par lesquels une nation tâche de s’assurer le commerce exclusif de ses colonies sont souvent plus nuisibles aux pays en faveur desquels on a voulu les établir, qu’ils ne le sont à ceux contre lesquels ils sont dirigés. Le poids de l’oppression injuste dont on veut accabler l’industrie des autres pays retombe, pour ainsi dire, sur la tête des oppresseurs, et écrase leur propre industrie plus que celle des autres pays. Par exemple, au moyen de ces règlements, il faut que le marchand de Hambourg envoie à Londres la toile qu’il destine pour le marché de l’Amérique, et il faut qu’il rapporte aussi de Londres le tabac qu’il destine pour le marché de l’Allemagne, parce qu’il n’a pas la liberté d’envoyer la toile directement en Amérique, ni d’en rapporter directement le tabac. Cette gêne l’oblige vraisemblable­ment à vendre la première un peu meilleur marché, et d’acheter l’autre un peu plus cher qu’il n’aurait fait sans cela, et ses profits s’en trouvent probablement affaiblis de quelque chose. Néanmoins, dans ce commerce entre Hambourg et Londres, il reçoit certainement des retours de son capital beaucoup plus promptement qu’il n’aurait jamais pu les recevoir dans le commerce direct avec l’Amérique, quand même on supposerait, ce qui n’est certainement pas, que les payements d’Amérique se fissent aussi ponctuellement que ceux de Londres. Par conséquent, dans le genre de com­merce auquel le marchand de Hambourg se trouve restreint par ces règlements, son capital peut tenir constamment en activité une beaucoup plus grande quantité d’in­dustrie en Allemagne, qu’il ne l’aurait sans doute pu faire dans le genre de commerce dont ce marchand se trouve exclu. Ainsi, quoique le premier de ces genres d’emploi soit peut-être pour lui moins lucratif que n’eût été ]’autre, il ne peut pas être moins avantageux pour son pays. Il en est tout autrement à l’égard de l’emploi dans lequel le monopole entraîne naturellement, pour ainsi dire, le capital du marchand de Londres. Il se peut bien que cet emploi soit plus lucratif pour lui que la plupart des autres sortes d’emploi ; mais, par rapport à la lenteur des retours, cet emploi ne saurait être plus avantageux que les autres à son pays.
Ainsi, en dépit de tous les injustes efforts de chaque nation de l’Europe pour se réserver à elle seule la totalité des avantages du commerce de ses colonies, aucune encore n’a pu réussir à se réserver exclusivement autre chose que la charge de mainte­nir en temps de paix et de défendre en temps de guerre la puissance oppressive qu’elle s’est arrogée sur elles. Pour les inconvénients résultant de la possession de ses colo­nies, chaque nation se les est pleinement réservés tout entiers ; quant aux avantages qui sont le fruit de leur commerce, elle a été obligée de les partager avec plusieurs autres nations.
Sans doute, au premier coup d’œil, le monopole du vaste commerce d’Amérique semble naturellement une acquisition de la plus haute valeur. À des yeux troublés par les chimères d’une folle ambition, il se présente, au milieu de la mêlée confuse des luttes opposées que se livrent la guerre et la politique, comme un objet éblouissant, digne prix de la victoire. C’est cependant le grand éclat de l’objet, l’immense étendue du commerce, qui est la qualité même pour laquelle le monopole est nuisible ; c’est elle qui est cause qu’un emploi, par sa nature moins avantageux au pays que la plupart des autres emplois, absorbe une bien plus grande portion du capital national que celle qui s’y serait portée sans cela.
On a fait voir, dans le livre second, que le capital commercial d’un pays cherche naturellement et prend de lui-même, pour ainsi dire, l’emploi le plus avantageux au pays. S’il est employé à faire le commerce de transport, alors le pays auquel appar­tient ce capital devient l’entrepôt général des marchandises de tous les pays dont il transporte ainsi les produits. Or, le propriétaire de ce capital cherche nécessairement à se défaire chez lui de la plus grande partie possible de ses marchandises. Il s’épargne par là la peine, les risques et les frais de l’exportation, et par cette raison il les vendra volontiers chez lui, non-seulement à un bien moindre prix, mais même quand il devrait en retirer un peu moins de profit que ce qu’il eût pu en espérer en les envoyant au-dehors. Il tâche donc naturellement de convertir, autant qu’il peut, son commerce de transport en commerce étranger de consommation. Si encore son capital se trouve employé dans le commerce étranger de consommation, il sera bien aise, par la même raison, de trouver à se défaire chez lui de la plus grande partie possible des marchan­dises nationales qu’il amasse en vue de les exporter, et par là il tâche de convertir, autant qu’il peut, son commerce étranger de consommation en commerce intérieur. Ce capital commercial de chaque pays recherche ainsi naturellement l’emploi le plus rapproché, et se retire de lui-même du plus éloigné ; naturellement, il se porte à l’emploi où les retours sont fréquents, et quitte celui où ils sont distants et tardifs ; natu­rellement, il est attiré vers l’emploi par lequel il peut entretenir le plus de travail productif, dans le pays auquel il appartient ou dans lequel réside son possesseur, et il est repoussé de l’emploi qui ne lui permet par d’en entretenir autant. Ainsi, de lui-même il cherche l’emploi qui, dans les circonstances ordinaires, est le plus avanta­geux à ce pays, et il fuit celui qui, dans les circonstances ordinaires, est le moins avantageux à ce pays.
Mais s’il arrive que, dans quelques-uns de ces emplois éloignés qui, dans les circonstances ordinaires, sont les moins avantageux pour le pays, le profit vienne à s’élever un peu au-dessus de ce qu’il faut pour contre-balancer la préférence que l’on est porté naturellement à donner aux emplois les plus rapprochés, cette supériorité de profit enlèvera le capital à ces emplois plus rapprochés, jusqu’à ce que les profits de tous les emplois reviennent entre eux à leur juste niveau. Cependant, cette supériorité dans le profit est une preuve que, dans l’état actuel où se trouve la société, ces emplois éloignés sont un peu moins fournis de capitaux, à proportion, que ne le sont les autres emplois, et que le capital national n’est pas réparti de la manière le plus convenable entre tous les différents emplois existant dans le pays. C’est une preuve qu’il y a quelque chose qui s’achète à meilleur marché, ou se vend plus cher qu’il ne devrait se faire, et que quelque classe particulière de citoyens est plus ou moins opprimée, soit en payant plus, soit en gagnant moins, que ne le comporte cette égalité qui devrait avoir lieu entre toutes les différentes classes, et qui s’y établit naturelle­ment. Quoique dans un emploi éloigné le même capital ne puisse jamais entretenir la même quantité de travail productif qu’il le ferait dans un emploi plus rapproché, cependant un emploi éloigné peut être tout aussi nécessaire au bien-être de la société qu’un emploi rapproché, attendu que les marchandises qui font l’objet du trafic de cet emploi éloigné peuvent être nécessaires pour faire marcher plusieurs des emplois les plus rapprochés. Mais si les profits de ceux qui trafiquent sur ces sortes de marchan­dises sont au-dessus de leur juste niveau, ces marchandises seront alors vendues plus cher qu’elles ne devraient l’être, ou un peu au-dessus de leur prix naturel, et tous ceux qui se trouvent engagés dans les emplois plus rapprochés auront plus ou moins à souffrir de ce haut prix. Leur intérêt exige donc, dans ce cas, qu’on retire quelques capitaux de ces emplois plus rapprochés, pour les porter dans cet emploi éloigné, afin de réduire à leur juste niveau les -profits de celui-ci, et de faire redescendre à leur prix naturel les marchandises sur lesquelles roule cet emploi. Dans cette circonstance extraordinaire, l’intérêt public veut qu’on retire quelque capital de ces emplois qui, dans les circonstances ordinaires, sont le plus avantageux à la société, pour le porter dans un emploi qui est moins avantageux pour elle dans les circonstances ordinaires. Et, dans cette circonstance extraordinaire, l’intérêt et le penchant naturel des individus se trouvent d’accord avec l’intérêt général, aussi exactement que dans toutes les autres circonstances ordinaires ; ils portent les capitalistes à retirer leurs capitaux de l’emploi le plus rapproché, pour les porter vers le plus éloigné.
C’est ainsi que les intérêts privés et les passions des individus les portent naturel­le­ment à diriger leurs capitaux vers les emplois qui, dans les circonstances ordinaires, sont les plus avantageux à la société. Mais si, par une suite de cette préférence natu­relle, ils venaient à diriger vers ces emplois une trop grande quantité de capital, alors la baisse des profits qui se ferait sentir dans ceux-ci, et la hausse qui aurait lieu dans tous les autres, les amèneraient sur-le-champ à réformer cette distribution vi­cieuse. Ainsi, sans aucune intervention de la loi, les intérêts privés et les passions des hom­mes les amènent à diviser et à répartir le capital d’une société entre tous les différents emplois qui y sont ouverts pour lui, dans la proportion qui approche le plus possible de celle que demande l’intérêt général de la société[26].
Toutes les différentes mesures et les règlements du système mercantile dérangent nécessairement plus ou moins cette distribution naturelle du capital, la plus avanta­geuse de toutes. Mais les règlements relatifs au commerce de l’Amérique et des Indes orientales la dérangent peut-être plus que tout autre, parce que le commerce avec ces deux vastes continents absorbe une plus grande quantité de capital que deux autres branches de commerce quelconque n’en pourraient absorber. Néanmoins, les règle­ments qui opèrent ce dérangement à l’égard de ces deux différentes branches de com­merce ne sont pas absolument de même nature. Le monopole est bien le grand ressort de ces règlements, dans l’une de ces branches comme dans l’autre ; mais ce sont deux sortes de monopoles différents. C’est toujours le monopole, d’une espèce ou d’une autre, qui est, à ce qu’il semble, le ressort unique employé par le système mercantile.
Dans le commerce de l’Amérique, chaque nation tâche de s’emparer toute seule, autant qu’il lui est possible, de tout le marché de ses colonies, en excluant ouverte­ment les autres nations de tout commerce direct avec elles. Pendant le cours de la plus grande partie du seizième siècle, les Portugais tâchèrent de soumettre à un pareil régime le commerce des Indes orientales, en vertu du droit exclusif de naviguer dans les mers de l’Inde, auquel ils prétendaient pour en avoir trouvé la route les premiers. Les Hollandais continuent encore à exclure toutes les autres nations européennes de tout commerce direct avec leurs îles à épices. Les monopoles de cette sorte sont évidemment établis contre toutes les autres nations de l’Europe, qui, par là, se voient non-seulement exclues d’un commerce dans lequel elles pourraient trouver de l’avan­tage à placer une partie de leurs capitaux, mais sont encore obligées d’acheter les marchandises sur lesquelles roule ce commerce, un peu plus cher que si elles avaient la faculté de les importer directement des pays qui les produisent.
Mais depuis la décadence de la puissance de Portugal, aucune nation de l’Europe n’a prétendu au droit exclusif de naviguer dans les mers des Indes, et les ports prin­cipaux de ces mers sont maintenant ouverts aux vaisseaux de toutes les nations euro­péennes. Cependant le commerce des Indes, excepté en Portugal et depuis quelques années en France, a été soumis, dans chaque pays de l’Europe, au régime d’une com­pa­gnie exclusive[27]. Les monopoles de ce genre sont proprement établis contre la nation même qui les institue. La majeure partie de cette nation se trouve par là non-seule­ment exclue d’un commerce vers lequel elle pourrait trouver l’avantage à diriger une partie de ses capitaux, mais encore obligée d’acheter les marchandises sur lesquelles porte ce commerce, un peu plus cher que s’il était ouvert et libre à tous les citoyens. Depuis l’établissement de la Compagnie des Indes anglaises, par exemple, les autres habitants de l’Angleterre, outre ce qu’ils ont eu à souffrir de l’exclusion de ce com­mer­ce, ont encore été obligés de payer dans le prix des marchandises de l’Inde qu’ils ont consommées, non-seulement tous les profits extraordinaires que la Compagnie peut avoir faits sur ces marchandises en conséquence de son monopole, mais encore tout le dégât et les pertes extraordinaires qu’ont nécessairement entraînés les abus et les malversations inséparables de l’administration des affaires d’une aussi grande compagnie. L’absurdité de cette seconde espèce de monopole est donc beaucoup plus évidente encore que l’absurdité de la première. 
Ces deux sortes de monopoles dérangent plus ou moins la distribution naturelle du capital de la société, mais ils ne la dérangent pas toujours de la même manière.
Les monopoles de la première sorte attirent toujours dans le commerce pour lequel ils sont établis une plus forte portion du capital de la société que celle qui s’y serait portée naturellement d’elle-même.
Les monopoles de la seconde sorte peuvent quelquefois attirer le capital dans le commerce particulier pour lequel ils sont établis, et quelquefois ils peuvent l’en repousser, selon la différence des circonstances. Dans les pays pauvres, ils attirent naturellement vers ce commerce plus de capital qu’il ne s’en serait porté sans cela. Dans les pays riches, ils repoussent naturellement de ce commerce une bonne partie du capital qui s’y rendrait sans eux.
De pauvres pays, tels que la Suède et le Danemark par exemple, n’auraient proba­blement jamais équipé un seul vaisseau pour les Indes orientales, si le commerce n’eût pas été mis sous le régime d’une compagnie exclusive. L’établissement d’une telle compagnie encourage nécessairement les entreprises maritimes. Le monopole des entreprises de ce commerce les garantit de tous concurrents sur le marché intérieur, et pour les marchés étrangers, ils ont la même chance que les commerçants des autres nations ; ce monopole leur présente la certitude d’un très-gros profit sur une quantité assez considérable, et la chance d’un profit assez considérable sur une très-grande quantité de marchandises. Sans un encouragement extraordinaire comme celui-là, les pauvres commerçants de ces pauvres pays n’auraient vraisemblablement jamais songé à hasarder leurs petits capitaux dans une spéculation aussi incertaine et aussi éloignée qu’aurait dû leur paraître naturellement le commerce des Indes orientales.
Au contraire, un pays riche comme la Hollande, dans le cas de la liberté de com­merce aux Indes orientales, y aurait probablement envoyé un plus grand nombre de vaisseaux qu’il ne le fait actuellement. Le capital limité de la Compagnie des Indes hollandaises repousse vraisemblablement de ce commerce un grand nombre de capi­taux de commerce qui s’y seraient portés sans cela. Le capital commercial de la Hollande est tellement abondant, qu’il déborde continuellement, pour ainsi dire, et va chercher un écoulement tantôt dans les fonds publics des nations étrangères, tantôt dans des prêts particuliers à des marchands et à des armateurs des pays étrangers, tantôt dans des commerces étrangers de consommation du plus long circuit, tantôt dans le commerce de transport. Tous les emplois rapprochés se trouvant complète­ment remplis, tous les capitaux qui peuvent s’y placer avec quelque profit un peu passable y étant déjà entrés, nécessairement le capital de la Hollande reflue vers des emplois plus éloignés. Si le commerce aux Indes orientales était totalement libre, il absorberait probablement la plus grande partie de ce capital surabondant. Les Indes orientales ouvrent à la fois aux manufactures de l’Europe et aux métaux précieux de l’Amérique, ainsi qu’à plusieurs autres de ses productions, un marché plus vaste et plus étendu que l’Europe et l’Amérique tout ensemble.
Tout dérangement dans la distribution naturelle du capital est nécessairement nuisible à la société dans laquelle il a lieu, soit qu’il arrive parce qu’une partie du capital est repoussée d’un commerce particulier où elle se serait rendue sans cela, soit qu’il arrive parce qu’une partie du capital est attirée dans un commerce particulier où elle ne serait pas entrée. S’il est vrai que, sans compagnie exclusive, le commerce de la Hollande aux Indes orientales serait plus grand qu’il n’est actuellement, alors ce pays doit souffrir une perte considérable par l’exclusion d’une partie de son capital de l’emploi qui lui convient le mieux. Et de même, s’il est vrai que, sans compagnie exclu­sive, le commerce de la Suède et du Danemark aux Indes orientales serait moindre que ce qu’il est actuellement, ou, ce qui est peut-être plus probable, n’exis­terait pas du tout, dès lors ces deux derniers pays doivent pareillement souffrir une perte considérable de ce qu’une partie de leur capital se trouve ainsi entraînée dans un emploi qui est plus ou moins mal assorti à leur situation particulière. Il vaudrait mieux peut-être pour eux, dans leur situation actuelle, acheter des autres nations les marchandises de l’Inde, quand même ils devraient les payer un peu plus cher, que d’aller porter une si grande portion de leur petit capital dans un commerce d’une distance si considérable, dont les retours sont si excessivement tardifs, et dans lequel ce capital ne peut entretenir qu’une faible quantité de travail productif dans leur pays où ils en ont tant besoin, où il y a si peu de chose de fait et tant à faire.
Ainsi, quand même un pays serait hors d’état de faire, sans l’aide d’une compagnie exclusive, aucun commerce direct aux Indes orientales, il ne s’ensuivrait pas pour cela qu’il fallût y établir une compagnie de cette espèce, mais seulement qu’un tel pays, dans cette situation, ne devrait pas faire de commerce direct aux Indes orientales. Pour se convaincre que ces sortes de compagnies ne sont pas, en général, nécessaires pour soutenir un commerce aux Indes orientales, il suffit de l’expérience qu’en ont faite les Portugais, qui, sans aucune compagnie exclusive, ont joui de ce commerce presque tout entier pendant plus d’un siècle de suite.
Il ne pourrait guère se faire, a-t-on dit, qu’un commerçant particulier possédât un capital suffisant pour entretenir, dans les différents ports des Indes orientales, des agents et des facteurs, à l’effet d’y commander et faire préparer à l’avance des mar­chandises pour les vaisseaux qu’il aurait occasion d’y faire passer ; et cependant, à moins qu’il ne fût en état de faire ces avances, la difficulté de trouver une cargaison toute prête pourrait mettre très-souvent ses vaisseaux dans le cas de perdre la saison favorable pour le retour, et la dépense d’un retard aussi long consommerait non-seu­le­ment tout le profit de l’armement, mais entraînerait encore fréquemment une perte énorme. Mais si un tel argument pouvait prouver quelque chose, il prouverait qu’aucune branche de commerce ne pourrait se soutenir sans compagnie exclusive, ce qui se trouve démenti par l’expérience de toutes les nations. Il n’y a pas de grande branche de commerce dans laquelle le capital d’un commerçant particulier suffise pour faire marcher toutes les branches subordonnées qui doivent être mises en activité pour que la branche principale puisse marcher. Mais, quand une nation est mûre pour quelque grande branche de commerce, il s’y trouve des commerçants qui dirigent naturellement leurs capitaux vers la branche principale, et d’autres qui diri­gent les leurs vers les branches accessoires et subordonnées ; et quoique, par ce moyen, toutes les branches différentes de ce commerce se trouvent marcher à la fois, cependant il n’arrive presque jamais qu’elles roulent toutes sur le capital d’un com­mer­çant particulier. Ainsi, si une nation est mûre pour le commerce des Indes orien­tales, une certaine portion de son capital se subdivisera naturellement entre tou­tes les branches différentes de ce commerce. Quelques-uns de ses négociants trouve­ront leur intérêt à établir leur résidence dans l’Inde, et à employer leurs capitaux en contractant et disposant des marchandises pour les vaisseaux que pour­ront y envoyer les autres négociants de ce pays résidant en Europe. Les établissements qu’ont obte­nus dans les Indes orientales différentes nations de l’Europe, étant ôtés aux compa­gnies exclusives auxquelles ils appartiennent aujourd’hui et mis immédiatement sous la protection du souverain, rendraient cette résidence sûre et commode, au moins pour les commer­çants des nations particulières auxquelles appartiennent ces établis­se­ments. Mais si, à une époque quelconque, il venait à se trouver que cette portion du capital d’un pays, qui d’elle-même tendait ou inclinait, pour ainsi dire, vers le commerce des Indes orientales, ne fût pas suffisante pour faire marcher toutes ces branches différentes qui le composent, ce serait une preuve qu’à ce moment-là ce pays n’était pas mûr pour ce commerce, et qu’il vaudrait mieux pour lui, pendant quelque temps, acheter des autres nations de l’Europe, même à un plus haut prix, les marchandises de l’Inde dont il a besoin, que de les importer lui-même directement des Indes orientales. Ce qu’il pour­rait perdre par le haut prix auquel il achèterait ces marchandises ne pourrait guère équivaloir à la perte qu’il aurait à essuyer en détournant une forte partie de son capital de quelques autres emplois plus nécessaires, ou plus utiles, ou mieux assortis à sa situation et à ses circonstances particulières, que ne le serait un commerce direct aux Indes orientales.
Quoique les Européens possèdent, tant sur la côte d’Afrique que dans les Indes orientales, une quantité d’établissements considérables, ils n’ont cependant encore fon­dé ni dans l’une ni dans l’autre de ces contrées d’aussi nombreuses et d’aussi florissantes colonies que celles des îles et du continent de l’Amérique. Cependant l’Afrique, aussi bien que plusieurs des pays compris sous le nom général d’Indes orientales, sont habités par des nations barbares. Mais ces peuples n’étaient pas, à beaucoup près, aussi faibles ni aussi dépourvus de moyens de défense que les mal­heureux Américains, et ils étaient, d’ailleurs, bien plus nombreux propor­tion­nellement à la fertilité naturelle du sol. Les nations les plus barbares de l’Afrique ou des Indes orientales étaient dans l’état pastoral ; les Hottentots mêmes étaient un peuple pasteur. Mais les naturels de tous les pays de l’Amérique, à l’exception du Mexique et du Pérou, n’étaient que des chasseurs, et il y a une différence immense entre le nombre de pasteurs et celui de chasseurs que peut faire subsister une même étendue de territoire également fertile. Ainsi, dans l’Afrique et dans les Indes orientales, il était plus difficile de déplacer les naturels et d’étendre les colonies européennes sur la plus grande partie des terres des habitants originaires. En outre, comme on l’a déjà observé, le régime et l’esprit des compagnies exclu­sives ne sont pas favorables à l’avancement des nouvelles colonies, et ils ont été probablement la cause principale du peu de progrès qu’elles ont fait dans les Indes orientales. Les Portugais ont soutenu leur commerce avec l’Afrique et les Indes orientales, sans aucune compagnie exclusive ; aussi, quoique leurs établissements du Congo, d’Angola et de Benguela sur la côte d’Afrique, et de Goa dans les Indes orientales, soient extrêmement opprimés sous le poids de la superstition et de tous les genres de mauvais gouvernement, cependant ils ont encore quelque ombre de ressem­blance avec les colonies de l’Amérique, et sont habités en partie par des Portugais qui y sont établis depuis plusieurs générations. Les établissements hollandais au cap de Bonne-Espérance et à Batavia sont à présent les colonies les plus considérables fondées par les Européens soit en Afrique, soit aux Indes orientales, et ces établis­sements se trouvent situés l’un et l’autre d’une manière singulièrement heureuse. Le cap de Bonne-Espérance était habité par une sorte de peuple presque aussi barbare et tout aussi peu capable de se défendre que les naturels de l’Amérique. Ce cap est d’ailleurs, pour ainsi dire, un lieu de repos qui coupe en deux moitiés la route de l’Europe aux Indes orientales, et auquel presque tout vaisseau européen fait quelque relâche, tant en allant qu’en revenant. L’approvisionnement de ces vaisseaux en den­rées fraîches de toute espèce, en fruits et quelquefois en vin, ouvre seul à l’excédent de produit des colons un marché très-étendu. Batavia occupe entre les principaux établissements des Indes orientales la même position que le cap de Bonne-Espérance entre l’Europe et tout point quelconque des Indes ; il est situé sur la route la plus fréquentée de l’Indostan à la Chine et au japon, et à peu près à moitié chemin de cette route. Presque tous les vaisseaux qui naviguent entre l’Europe et la Chine relâchent aussi à Batavia ; il est par-dessus tout cela le centre et le rendez-vous principal de ce qu’on nomme le commerce du pays même des Indes, non-seulement de cette partie de ce commerce que font les Européens, mais de celle que font les naturels de l’Inde, et l’on voit fréquemment dans son port des vaisseaux montés par des marchands de la Chine et du japon, par des habitants de Tonquin, de Malacca, de la Cochinchine et de l’île des Célèbes. Des situations aussi avantageuses ont mis ces deux colonies en état de surmonter tous les obstacles que le génie oppressif d’une compagnie exclusive leur a dû souvent faire rencontrer dans le cours de leur avancement. Cette situation a mis Batavia à même de surmonter en outre le désavantage du climat le plus malsain peut-être qui soit au monde.
Quoique les Compagnies anglaise et hollandaise n’aient pas fondé de colonies con­si­dérables aux Indes orientales, à l’exception des deux dont je viens de parler, elles y ont fait cependant des conquêtes importantes. Mais si l’esprit qui dirige natu­rel­le­ment une compagnie exclusive s’est jamais bien fait voir, c’est surtout dans la manière dont celles-ci gouvernent l’une et l’autre leurs nouveaux sujets. Dans les îles à épices, les Hollandais brûlent de ces denrées tout ce qu’en produit une année fertile au-delà de ce qu’ils peuvent espérer en débiter en Europe avec un profit qui leur paraisse suffisant. Dans les îles où ils n’ont pas d’établissement, ils donnent une prime à ceux qui arrachent les boutons et les feuilles nouvelles de girofliers et de musca­diers qui y croissent naturellement, et que cette politique barbare a maintenant, dit-on, presque entièrement détruits. Dans les îles mêmes où ils ont des établissements, ils ont extrêmement réduit, à ce qu’on dit, le nombre de ces arbres. Ils ont peur que si le produit même de leurs propres îles était beaucoup plus abondant que ce qu’il faut à leur marché, les naturels du pays ne puissent trouver moyen d’en faire passer quelque partie aux autres nations, et le meilleur moyen, à ce qu’ils s’imaginent, d’assurer leur monopole sur ces denrées, c’est de prendre bien garde qu’il n’en croisse plus que ce qu’ils portent eux-mêmes au marché. Par différentes mesures oppressives, ils ont réduit la population de plusieurs des Moluques au nombre d’hommes seulement suffisant pour fournir des provisions fraîches et les choses de première nécessité aux garnisons presque nulles qu’ils y tiennent, et à ceux de leurs vaisseaux qui viennent de temps en temps y prendre leur cargaison d’épices. Cependant, sous le gouverne­ment même des Portugais, ces îles étaient, dit-on, passablement peuplées. La compa­gnie anglaise n’a pas encore eu le temps d’établir dans le Bengale un système aussi complètement destructeur. Toutefois, le plan suivi par l’administration de cette compagnie a eu exactement la même tendance. On m’a assuré qu’on y avait vu assez communément le chef, c’est-à-dire le premier commis d’une factorerie, donner ordre à un paysan de passer la charrue sur un riche champ de pavots, et d’y semer du riz ou quelque autre grain. Le prétexte dont il se servait était l’intention de prévenir une disette de subsistances ; mais la véritable raison, c’était de laisser à ce chef la facilité de vendre à un meilleur prix une grande quantité d’opium dont il se trouvait chargé pour le moment. Dans d’autres occasions, l’ordre a été donné en sens inverse, et il a fallu passer la charrue sur un champ de riz ou d’autre grain pour faire place à une plantation de pavots, quand le chef prévoyait la possibilité de faire quelque profit extraordinaire sur l’opium. En maintes circonstances, les facteurs de la compagnie ont tâché d’établir pour leur propre compte le monopole de quelques-unes des plus impor­tantes branches, non-seulement du commerce étranger, mais même du commerce intérieur du pays. Si on les eût laissés faire, il est certain qu’ils auraient essayé, dans un temps ou dans l’autre, de restreindre la production des articles particuliers dont ils avaient ainsi usurpé le monopole, de manière à la réduire non-seulement à ce qu’ils auraient pu acheter eux-mêmes, mais même à ce qu’ils auraient pu espérer vendre avec un profit qui leur eût semblé raisonnable. Avec de pareils moyens, il ne fallait pas plus d’un siècle ou deux pour que la politique de la Compagnie anglaise se fût probablement montrée, par ses effets, tout aussi complètement destructive que celle de la compagnie hollandaise.
Il n’y a cependant rien qui soit plus directement contraire au véritable intérêt d’une Compagnie de ce genre, considérée comme souverain des pays qu’elle a conquis, que ce régime destructeur. Dans presque tous les pays, le revenu du souverain est tiré de celui du peuple. Ainsi, plus le revenu du peuple sera considérable, plus le produit annuel de ses terres et de son travail sera abondant, et plus alors il sera en état d’en rendre au souverain. L’intérêt de celui-ci est donc d’augmenter le plus possible ce produit annuel. Mais si c’est l’intérêt de tout souverain, c’est plus particulièrement encore celui d’un souverain qui, comme ceux du Bengale, tire principalement son revenu d’une redevance foncière. Cette redevance doit nécessairement être propor­tion­née à la quantité et à la valeur du produit ; or, l’une et l’autre doivent dépendre aussi nécessairement de l’étendue du marché. La quantité du produit se proportion­nera toujours, avec plus ou moins d’exactitude, à la consommation de ceux qui sont en état de le payer, et le prix qu’ils en payeront sera toujours en raison de l’activité de la concurrence. Il est donc de l’intérêt d’un tel souverain d’ouvrir au produit de son pays le marché le plus étendu, de laisser au commerce la plus entière liberté, pour augmenter le plus possible le nombre et la concurrence des acheteurs et, à cet effet, d’abolir non-seulement tous les monopoles, mais de supprimer encore toutes les barrières qui pourraient gêner ou le transport du produit national d’un endroit du pays à l’autre, ou son exportation aux pays étrangers, ou enfin l’importation des marchan­dises de toute espèce contre lesquelles il pourrait s’échanger. Une telle conduite mettra de plus en plus ce souverain dans le cas de voir augmenter et la quantité et la valeur de ce produit, et celles, par conséquent, de la part qui lui en appartient, c’est-à-dire de voir augmenter son propre revenu.
Mais il semble qu’il soit tout à fait hors du pouvoir d’une compagnie de mar­chands de se considérer comme souverain, même après qu’elle l’est devenue. Trafi­quer ou acheter pour revendre est toujours ce que ces gens-là regardent comme leur affaire principale et, par une étrange absurdité, ils ne considèrent le caractère de souverain que comme accessoire de celui de marchand, comme quelque chose de subordonné à ce dernier titre, et qui doit leur servir seulement comme un moyen d’acheter à plus bas prix dans l’Inde, et par là de revendre avec un plus gros profit. Dans cette vue, ils mettent tous leurs soins à écarter du marché des pays soumis à leur gouvernement le plus de concurrents possible et, conséquemment, à réduire quelque partie au moins de l’excédent de produit de ces pays à la quantité purement suffisante pour remplir ce qu’eux-mêmes jugent à propos d’en demander, c’est-à-dire à la quan­tité qu’ils espèrent pouvoir débiter en Europe avec un profit qui leur paraisse raisonnable. Leurs habitudes mercantiles les entraînent ainsi par une pente presque irrésistible, quoique peut-être insensible, à préférer le plus souvent les petits profits passagers du monopoleur au riche et permanent revenu du souverain, et elles les conduiront infailliblement, par degrés, à traiter les pays soumis à leur gouvernement à peu près comme les Hollandais traitent les Moluques. L’intérêt de la Compagnie des Indes considérée comme souverain, c’est que les marchandises européennes qui sont apportées dans les États soumis à sa domination y soient vendues au meilleur marché possible, et que les marchandises indiennes qu’on tire de ces mêmes États y rendent le plus haut prix possible ou s’y vendent le plus cher possible. Mais, considérée comme compagnie de marchands, son intérêt est entièrement opposé. Comme souverain, son avantage est précisément le même que celui des pays qu’elle gouverne ; comme compagnie marchande, il se trouve directement contraire à celui-ci.
Mais si l’esprit d’un pareil gouvernement, même pour ce qui a rapport à sa direction en Europe, se trouve ainsi essentiellement vicieux et peut-être irrémédiable, celui de son administration dans l’Inde l’est encore davantage. Cette administration est nécessairement composée d’un conseil de marchands, profession sans doute extrê­mement recommandable, mais qui, dans aucun pays du monde, ne porte avec soi le caractère imposant qui inspire naturellement du respect au peuple, et qui commande une soumission volontaire sans qu’il soit besoin de recourir à la contrainte. Un conseil ainsi composé ne peut obtenir d’obéissance qu’au moyen des forces militaires qui l’entourent et, par conséquent, son gouvernement est nécessairement militaire et despotique. Toutefois, le véritable état de ces administrateurs, c’est l’état de marchands[28]. Leur principale affaire, c’est de vendre pour le compte de leurs maîtres les marchandises d’Europe qui leur sont commises, et d’acheter en retour des marchandises indiennes pour le marché de l’Europe ; c’est donc de vendre les unes aussi cher, et d’acheter les autres à aussi bon marché que possible et, par conséquent, d’exclure, autant qu’ils le peuvent, toute espèce de rivaux du marché particulier où ils tiennent leur boutique. Ainsi, l’esprit de l’administration, en ce qui concerne le commerce de la Compagnie, est le même que l’esprit de la direction ; il tend à subordonner le gouvernement aux intérêts du monopole et, par conséquent, à étouffer la croissance naturelle de quel­ques parties au moins de l’excédent de produits du pays, et à les réduire à la quantité purement nécessaire pour remplir la demande qu’en fait la Compagnie.
D’un autre côté, tous les membres de l’administration commercent plus ou moins pour leur propre compte, et c’est en vain qu’on voudrait le leur défendre. Il serait trop absurde de s’attendre que les commis d’une immense maison de commerce à quatre mille lieues de distance, et sur lesquels, par conséquent, il est presque impossible d’avoir les yeux, iront, sur un simple ordre de leurs maîtres, renoncer tout d’un coup à faire aucune espèce d’affaires pour leur compte, abandonner pour jamais toute perspective de faire fortune, quand ils en ont les moyens sous la main, et se contenter des modiques salaires que ces maîtres leur abandonnent, salaires qui, tout modiques qu’ils sont, ne sont guère susceptibles d’augmentation, puisqu’ils sont ordinairement aussi forts que le peuvent supporter les profits réels de la Compagnie. Dans de pareil­les circonstances, une défense aux agents de la Compagnie de commercer pour leur compte ne pourrait guère produire d’autre effet que de mettre les agents supérieurs à même d’opprimer, sous prétexte d’exécuter cette défense, ceux des agents inférieurs qui auraient eu le malheur de leur déplaire. Les agents tâchent naturellement d’établir, en faveur de leur commerce particulier, le même monopole que celui du commerce public de la Compagnie. Si on les laisse faire à leur fantaisie, ils établiront ce mono­pole directement et ouvertement, en défendant tout uniment à qui que ce soit de commercer sur les articles qu’ils auront choisis pour l’objet de leur trafic, et c’est peut-être là la meilleure manière et la moins oppressive de l’établir. Mais s’il existe un ordre venu d’Europe qui leur défend d’en user ainsi, alors ils n’en chercheront pas moins à s’assurer un monopole du même genre, mais secrètement et indirectement, par des voies bien plus oppressives pour le pays. Ils emploieront toute l’autorité du gouvernement, ils abuseront de l’administration de la justice pour vexer et pour perdre les personnes qui s’aviseront de leur faire concurrence dans quelque branche de commerce qu’ils aient jugé à propos d’adopter, et qu’ils exerceront à l’aide de courtiers cachés ou au moins non avoués publiquement. Mais le commerce particulier des agents s’étendra naturellement à un bien plus grand nombre d’articles divers, que le commerce public de la Compagnie. Le commerce public de la Compagnie ne s’étend pas au-delà du commerce avec l’Europe, et ne peut embrasser qu’une partie seulement du commerce étranger du pays, tandis que le commerce particulier des agents peut s’étendre à toutes les branches différentes, tant du commerce intérieur du pays que de son commerce étranger. Le monopole de la Compagnie ne peut tendre à rien de plus qu’à étouffer la croissance naturelle de cette partie du produit qui serait exportée en Europe en cas de liberté du commerce. Le monopole des agents tend à étouffer la croissance naturelle de toute espèce de produit sur laquelle il leur plaira de trafiquer, de celle destinée pour la consommation du pays aussi bien que de celle qui est destinée pour l’exportation et, par conséquent, tend à dégrader la culture générale du pays et à diminuer la population ; il tend à réduire toutes les espèces de produc­tions, même celles nécessaires aux besoins de la vie (s’il plaît aux agents de la Com­pa­gnie de trafiquer sur ces articles), aux quantités seulement que ces agents peuvent suffire à acheter, avec la perspective de les revendre au profit qui leur convient.
De plus, par la nature même de leur position, les agents doivent être plus portés à soutenir, avec rigueur et avec dureté, leurs intérêts personnels contre l’intérêt du pays qu’ils gouvernent, que leurs maîtres n’y seraient disposés pour soutenir les leurs. C’est à ces maîtres qu’appartient le pays, et ceux-ci ne peuvent s’empêcher d’avoir quelque ménagement pour la chose qui leur appartient. Mais le pays n’appartient pas aux agents. Le véritable intérêt de leurs maîtres, si ceux-ci étaient bien en état de l’enten­dre, est le même que celui du pays[29], et s’ils l’oppriment, ce ne peut être jamais que par ignorance et par suite de leurs misérables préjugés mercantiles. Mais l’intérêt réel des agents n’est nullement le même que celui du pays et, à quelque point qu’ils vins­sent à s’éclairer, il n’en résulterait pas pour cela nécessairement un terme à leurs oppressions. Aussi, les règlements qui ont été envoyés d’Europe, quoiqu’ils fussent souvent mauvais, annonçaient ordinairement de bonnes intentions ; mais dans ceux qui ont été faits par les agents dans l’Inde, on a pu remarquer quelquefois plus d’intel­ligence et peut-être des intentions moins bonnes. C’est un gouvernement d’une espèce bien singulière, qu’un gouvernement dans lequel chaque membre de l’administration ne songe qu’à quitter le pays au plus vite et, par conséquent, à se débarrasser du gouvernement le plut tôt qu’il peut, et verrait avec une parfaite indifférence la contrée tout entière engloutie par un tremblement de terre le lendemain du jour où il l’aurait quittée, emportant avec soi toute sa fortune.
Dans tout ce que je viens de dire, néanmoins, je n’entends pas jeter la moindre impression défavorable sur l’honnêteté des facteurs de la Compagnie des Indes en général, et bien moins encore sur celle de qui que ce soit en particulier. C’est le systè­me de gouvernement, c’est la position dans laquelle ils se trouvent placés que j’entends blâmer, et non pas le personnel de ceux qui ont eu à agir dans cette position et dans ce gouvernement. Ils ont agi selon la pente naturelle de leur situation parti­culière, et ceux qui ont déclamé le plus haut contre eux n’auraient probablement pas mieux fait à leur place. En matière de guerre et de négociation, les conseils de Madras et de Calcutta se sont conduits, dans plusieurs occasions, avec une sagesse et une fermeté mesurées qui auraient fait honneur au sénat romain dans les plus beaux jours de la république. Cependant, les membres de ces conseils avaient été élevés dans des professions fort étrangères à la guerre et à la politique[30]. Mais leur situation toute seule, sans le secours que donnent l’instruction, l’expérience et l’exemple, semble avoir for­mé en eux tout d’un coup les grandes qualités qu’elle exigeait, et leur avoir donné, comme par inspiration, des talents et des vertus qu’ils ne se flattaient guère de posséder. Si donc, dans quelques circonstances, cette situation les a excités à des actes de magnanimité qu’on n’était pas trop en droit d’attendre de leur part, il ne faut pas s’étonner que, dans d’autres circonstances, elle les ait poussés à des exploits d’une nature un peu différente.
De telles Compagnies exclusives sont donc un mal public, sous tous les rapports ; c’est un abus toujours plus ou moins incommode aux pays dans lesquels elles sont établies, et un fléau destructeur pour les pays qui ont le malheur de tomber sous leur gouvernement.
 
 
 
↑ Cette loi ne s’appliquait qu’au territoire conquis, et non aux anciens patrimoines. A. B.
↑
La constitution des anciennes colonies comporte des considérations qui ont excité beaucoup d’intérêt et donné lieu à de nombreuses investigations. Une Dissertation de Bougainville, qui remporta le prix décerné par l’Académie des inscriptions au meilleur essai sur cette question, fut publiée en 1745.
Le professeur Barron de Saint-André, dans un écrit anonyme intitulé History of the colonisation of the free states of antiquities, s’efforça de prouver que les anciens exerçaient sur leurs colonies la même espèce de contrôle que les modernes exerçaient ordinairement sur les leurs. Le traité de Barron fut réfuté par le docteur Lymonds, de Cambridge, qui publia des Remarks à ce sujet en 1778, et par sir William Mereddith, dans les Historical remarks on the taxation of free states, publiées en 1784. Beyne a écrit quelques dissertations savantes sur ce sujet dans ses Opuscula academica. Mais le meilleur ouvrage sur les colonies des anciens est sans doute celui de Sainte-Croix ; De l’état et du sort des anciennes colonies, publié en 1778. La Verona illustrata du savant marquis Maffei contient un excellent exposé du système de colonisation des Romains. Raoul-Rochette, dans son volumineux ouvrage sur les Colonies grecques, a recherché leur histoire jusque dans les plus petits détails ; mais il manque de la connaissance des principes, et ses vues générales ne sont pas suffisamment approfondies.
Malgré les nombreux ouvrages publiés pendant le dernier demi-siècle, un bon traité sur cet important sujet est encore à désirer. L’History of the British west Indian colonies de Bryan Edward est bien écrite ; maie il exagère leur importance, et il a une forte propension pour les propriétaires. La Colonial policy de lord Breugham fut publiée en 1803. Elle contient des renseignements auxquels on ne peut pas toujours se fier, sur les systèmes coloniaux des différentes nations européennes ; mais, sous les autres rapports, cet ouvrage n’a aucune valeur. L’auteur exagère l’importance des colonies beaucoup plus encore qu’Edward ; il défend ou atténue les oppressions restrictives si fréquemment imposées sur leur commerce, et qui ont été aussi funestes aux métropoles qu’aux colonies. Il prétend, pour justifier ces restrictions, « que les intérêts des commerçants, dans l’emploi de leurs capitaux, ne sont nullement les mêmes, dans tous les cas, que les intérêts de la communauté à laquelle ils appartiennent… » (T. I, p. 254.) II n’est pas nécessaire de parler des ouvrages plus récents sur la colonisation*. Mac Culloch.
↑ La population relative des différentes contrées et villes d’Amérique s’est beaucoup modifiée depuis la publication de la Richesse des nations. A. B.
↑ Ce qu’on nomme en Angleterre libre soccage est une sorte de tenure suivant laquelle le seigneur n’a droit à autre chose qu’à une redevance fixé et annuelle en argent, ce qui ressemble à nos censives, si ce n’est que les droits seigneuriaux, en cas de mutation par vente ou aliénation, ont été abolis en Angleterre par un statut de Charles II.
↑ L’origine des lois de navigation de l’Angleterre remonte au règne de Richard II, ou peut-être à une époque encore plus reculée. Mais comme il serait difficile de rendre compte de tous les changements et variations survenus dans un temps aussi éloigné, nous nous bornerons à constater que les deux principes essentiels des lois de navigation ont été posés d’une manière explicite sous le règne de Henri VII ; l’importation de certaines marchandises fut alors interdite, à moins qu’elles ne fussent portées par des navires anglais et n’ayant à bord que des marins anglais. Au commencement du règne d’Élisabeth (S. Élis., ch. v), les navires étrangers furent exclus des pêcheries et du commerce de cabotage. Le parlement républicain donna une grande extension aux lois de navigation par l’acte de 1650, qui interdit aux vaisseaux de toutes les nations étrangères de faire le commerce avec les colonies de l’Amérique, sans en avoir préalablement obtenu l’autorisation. Ces différents actes se rapportaient plutôt au commerce entre les différents ports et colonies de l’empire qu’aux relations commerciales et étrangères. Mais l’année suivante (9 oct. 1651), le parlement républicain publia le célèbre acte de navigation. Cet acte avait un double but ; il devait, d’un côté, donner de plus grands développements à notre navigation et frapper un coup décisif sur la puissance maritime des Hollandais, qui avaient alors le monopole du commerce de transport, et contre lesquels différentes circonstances avaient fait naître, en Angleterre, une grande aigreur. L’acte dont il est ici question établit, que ni produits ni marchandises provenant de l’Asie, de l’Afrique ou de l’Amérique ne pourraient être importés en Angleterre, en Irlande ou en aucune de leurs colonies, que sur des navires appartenant à des sujets anglais, commandés par des Anglais, et dont les équipages se composeraient en grande partie de marins anglais. Après avoir ainsi assuré aux armateurs anglais le commerce d’importation de l’Asie, de l’Afrique et de l’Amérique, cet acte leur garantit en outre, autant que cela était possible, le commerce d’importation de l’Europe. A cet effet, il fut expressément dit, que les produits provenant de n’importe quel pays de l’Europe ne pourraient être importés en Angleterre que sur des navires anglais, ou sur des vaisseaux qui seraient la propriété réelle de la nation et du pays d’où ces produits seraient exportés.

Cette dernière mesure est entièrement dirigée contre les Hollandais, qui avaient très-peu de produits indigènes à exporter, et dont les navires étaient principalement employés à transporter- les produits des autres pays aux marchés étrangers. Telles étaient les principales dispositions de ce fameux acte. Elles furent maintenues par le gouvernement royal qui suivit le protectorat de Cromwell, et forment la base de l’acte xii (Charles II, chap. xviii). Elles sont restées jusqu’aux temps modernes la loi d’après laquelle nos relations commerciales avec les pays étrangers ont été réglées, et qu’on a pompeusement appelée la Charte maritime de l’Angleterre.
En supposant que tout ce qui a été dit par les apologistes de cet acte fût parfaitement vrai ; en admettant que l’acte de navigation, au moment où il fut conçu, était bien réellement le résultat d’une pensée politique profonde, il ne s’ensuivrait pas encore qu’il dût être maintenu de notre temps. Les institutions humaines ne sont pas fondées pour l’éternité. Elles doivent toujours s’adapter aux circonstances ainsi qu’aux besoins de la société. Mais la situation de la Grande-Bretagne et des autres pays de l’Europe a complètement changé depuis 1650. La grandeur commerciale et les richesses tant enviées des Hollandais ont disparu ; nous n’avons plus rien à craindre de leur inimitié ; et ce serait un véritable anachronisme que de conserver aujourd’hui quelques-unes de ces haines ou préventions qui ont donné naissance à cette mesure. Londres est aujourd’hui ce que Tut autrefois Amsterdam, le grand entrepôt du monde commercial, universi orbit terrarum emporium. Et la véritable question est maintenant de savoir, non point quels sont les meilleurs moyens pour arriver à la grandeur maritime, mais quels sont les meilleurs moyens pour nous conserver la supériorité incontestée que nous avons déjà atteinte. La réponse à cette question ne présente pas de grandes difficultés. La navigation et la puissance maritime sont les effets, non les causes du commerce. Si ce dernier augmente, l’agrandissement de la puissance navale s’ensuivra naturellement. Plus le commerce entre les différents pays s’étend, plus l’augmentation des marins et des navires deviendra nécessaire.
Il serait difficile, par conséquent, de mettre en doute la sagesse des modifications opérées dans les lois de navigation, en partie par les bills introduits par M. (aujourd’hui lord) Wallace en 1824, par M. Huskisson en 1825, et en partie par l’adoption du système dit de réciprocité. Sous le régime des lois existantes (6, George IV, chap. cix), une égalité parfaite règle les relations commerciales entre la Grande-Bretagne et celles des contrées de l’Europe qui se trouvent en bons rapports avec elle. Les souvenirs de nos anciennes haines et de notre jalousie de la prospérité de quelques-uns de nos voisins n’existent plus, et c’est une législation uniforme qui règle notre commerce avec le continent. Cette uniformité de législation, en ouvrant une plus grande carrière aux opérations mercantiles, et en donnant au commerce avec les plus riches de nos voisins une plus grande importance, éloigne beaucoup d’embarras et de difficultés, en même temps qu’elle diminue ridée qu’on s’était faite, non sans quelque raison, sur le continent, que les principes essentiels de notre système de commerce étaient conçus dans des vues exclusives et égoïstes.
Il résulte des observations précédentes, que les lois de navigation, à part le préjudice qu’elles causaient au commerce du pays, étaient en outre impuissantes à atteindre leur véritable but, c’est-à-dire à produire l’emploi d’un plus grand nombre de vaisseaux. Mais, en supposant même que, par rapport à ce dernier objet, elles aient eu un plein et entier succès, et qu’elles n’aient point eu de suites fâcheuses pour la prospérité de notre commerce extérieur, leur véritable utilité n’en serait pas pour cela démontrée.
On a toujours regardé comme un axiome en fait de politique maritime que, pour avoir des forces navales puissantes, il faut absolument avoir une marine marchande considérable qui puisse fournir des matelots ; et A. Smith se prononce en faveur des lois de navigation, principalement en vue de l’accroissement qu’elles procureraient à la marine marchande du pays, accroissement qu’il regarde comme indispensable pour le développement de notre marine de guerre, et par conséquent pour la sécurité et la défense du pays. Mais il serait facile de démontrer que celle opinion ne repose sur aucun fondement solide. Mac Culloch.
↑ En général, tous les bois propres à la menuiserie, au charronnage, à la tonnellerie et à la charpente, en exceptant ceux propres aux mâtures, etc., dont il est question plus bas.
↑ Liv. I, chap. ii.
↑ Les îles cédées par la paix de Paris, de 1763, sont proprement la Grenade et les Grenadins ; mais les îles de Saint-Vincent, de la Dominique et de Tabago, qui ont été laissées à l’Angleterre par la même paix, ne l’ont pas été à titre de cession. Néanmoins l’auteur comprend ici toutes ces îles sous le nom d’îles cédées.
↑ Bois jaune propre à la teinture, et qu’on tire principalement de Tabago.
↑ C’est le fer non forgé, qui se nomme aussi fonte.
↑ Après la denrée dont l’ouvrier se nourrit, celle dont le bon marché contribue davantage à donner de l’activité au travail et à augmenter l’aisance générale du peuple, c’est le fer. il y a peu de substances dont la consommation soit aussi étendue et dont les services soient à la fois plus utiles et plus variés. Le fer fournit des instruments à presque tous les arts et métiers, depuis le soc de la charrue jusqu’au ciseau du sculpteur et à la lime de l’horloger ; il sert à mettre en œuvre les autres matières, le bois, la pierre, les métaux et le fer lui-même ; il taille le diamant et les pierres précieuses ; il entre, comme partie essentielle, dans la construction des édifices et des vaisseaux, ainsi que dans la fabrication de presque tous les meubles solides et durables. Il sert à fournir une quantité d’ustensiles de ménage, et n’est pas moins nécessaire dans le foyer et dans la cuisine du pauvre que dans ceux du riche ; enfin, il contribue puissamment à la défense du pays en temps de guerre.

Mais le fer, tel que la nature nous le donne, se trouve combiné avec certaines substances minérales qui, d’après les proportions dans lesquelles elles existent, le rendent ou plus doux, ou plus cassant, et, par cette raison, plus ou moins propre à des usages particuliers ; et comme ces qualités du fer varient en différentes contrées, il en résulte que certains pays possèdent en quantité surabondante la qualité du fer recherchée pour quelques genres de travaux, tandis que d’autres pays, ou ne la trouvent point dans leur sol dans une quantité égale à leurs besoins, ou du moins ne peuvent la produire chez eux qu’avec de très-grands frais.
La France, avant la révolution de 1789, consommait, année commune, à ce que l’on croit, environ 180 millions de livres ou 1,800 mille quintaux de fer, acier et fonte moulée, dans laquelle somme on peut compter l’acier pour 8 millions de livres et la fonte moulée pour 16. Sur ces 180 millions de livres pesant de fer consommé ou travaillé en France, on pense que près des quatre cinquièmes étaient de fabrication française, et qu’un peu plus d’un cinquième (trois à quatre mille quintaux du poids de marc) était importé de la Suède, de l’Espagne et de l’Allemagne, mais principalement du premier de ces trois pays.
Quelques personnes ont porté beaucoup plus haut le montant de la fabrication française, parce qu’elles ont établi leur calcul sur le nombre des hauts-fourneaux existant dans le royaume, et qu’elles les ont supposés tous roulant sans interruption, c’est-à-dire neuf à dix mois dans l’année, tandis qu’il en est plusieurs qui ne sont allumés que tous les deux ans et même tous les trois ans, et d’autres qui ne le sont que cinq et six mois par année.
Une partie du fer travaillé en France, soit qu’il provint du sol, soit qu’il eût été acquis par l’importation, était exportée aux colonies françaises de l’Amérique.
Le fer de Suède acquittait à l’entrée dans le royaume de France un droit de dix sous par quintal du poids de marc, et ce modique droit subsistait ainsi depuis plus de cent ans, sans exciter aucune plainte de la part de nos maîtres de forges, dont l’exploitation étant partout réglée sur les besoins habituels de la consommation, tant pour la quantité que pour la qualité des fers, ne recevait aucune atteinte de l’importation étrangère, également mesurée sur ces besoins. Mais ce qui excitait alors avec justice de vives réclamations de la part de ces fabricants, c’était la taxe perçue dans l’intérieur sur les produits de leur fabrication, sous le nom de droit de la marque des fers, taxe dont la perception était accompagnée de visites gênantes et même de mesures vexatoires.
L’assemblée qui exerça la première le pouvoir législatif en France, depuis cette révolution, supprima le droit de la marque des fers ; et pour que le fisc reçût une indemnité de cette suppression, elle doubla le droit d’entrée sur les fers étrangers, et le porta, par une loi de 1794, à un franc par quintal, poids de marc.
Mais les événements qui suivirent amenèrent bientôt après une grande secousse dans l’état de la fabrication et de la consommation des fers en France. La guerre qui éclata entre ce pays et les puissances du nord et de l’est de l’Europe, ainsi que l’interruption de tout commerce maritime, empêchèrent l’importation ordinaire des fers de la Suède et de l’Espagne. Quoique les mêmes causes arrêtassent aussi l’exportation du fer fabriqué en France que recevaient les colonies, cependant les besoins immenses d’une guerre dans laquelle on armait presque tout individu en âge de porter les armes, joints au défaut d’ordre et d’économie qui est inséparable d’un état de confusion et de turbulence, élevèrent la demande pour le fer à un tel point, qu’il s’établit de toutes parts des forges nouvelles, et jusque dans les cantons voisins des côtes de la mer, dans lesquels on n’aurait jamais songé à en établir dans d’autres circonstances, puisque le transport d’une telle marchandise étant très-dispendieux, des forges ainsi situées n’auraient pu soutenir la concurrence des produits étrangers qui étaient transportés par mer. La fabrication du fer fut, en France, fort au-dessus de ce qu’elle avait jamais été. On lit dans quelques écrivains très-judicieux, mais qui n’ont point indiqué la source où ils avaient puisé cette information**, que la fabrication du fer, en France, s’éleva à 225 millions de kilogrammes ; ce qui serait presque trois fois la fabrication antérieure à 1789, et ce qui semble trop peu vraisemblable pour qu’on ne soit pas tenté de soupçonner ce calcul d’une forte exagération.
Quand, après les jours de tumulte et d’anarchie, le gouvernement eut pris, en France, une forme régulière et que les relations commerciales commencèrent à se renouer, la plupart de ces forges, qui ne devaient leur naissance qu’à des circonstances tout à fait extraordinaires, ne pouvaient plus lutter contre les fers étrangers. Ceux-ci étaient le produit naturel du développement de l’industrie dans des pays couverts de forêts sans utilité ; mais la fabrication française était, à certains égards, une véritable superfétation hors de toute proportion avec la valeur des autres produits du pays. Le gouvernement crut donc, en 1806, devoir accorder comme encouragement aux forges françaises, un doublement du droit d’entrée sur les fers étrangers, ce qui quadrupla le droit qui avait été perçu jusqu’à l’époque de 1790.
Mais, en 1806, les fabricants français trouvaient dans l’anéantissement presque total du commerce maritime une garantie suffisante contre l’introduction des fers du Nord, ceux dont ils avaient le plus à redouter la concurrence. Aussi ces fabricants eurent-ils, dans le fait, le monopole absolu de la fabrication des fers pour tous les besoins de la consommation de leur pays ; et comme l’état permanent de guerre rendait continuelles les demandes du gouvernement pour ce genre de fourniture, que ce consommateur est le moins économe de tous et le moins difficile sur les conditions de ses achats, le prix du fer monta à une hauteur excessive, au grand détriment de tous les arts mécaniques et de tous les consommateurs privés.
La cherté extrême du fer ne fut pas le seul dommage qui résulta de cet état forcé et contraire au cours naturel des choses. Une autre espèce de denrée de consommation générale et de première nécessité fut entraînée dans le renchérissement des fers et obligée de suivre le mouvement qu’ils avaient pris, parce qu’elle est en quelque sorte la matière première dont ce métal est fabriqué. Le bois entre pour plus des deux tiers dans les frais de la fabrication du fer, et forme, avec la subsistance des ouvriers et employés de la forge, la totalité des valeurs qui se consomment dans ce genre d’exploitation ; le minerai, la casline et autres substances minérales qui entrent dans le fourneau avec le charbon, n’ayant presque d’autre valeur que la dépense faite pour le transport. Ainsi, l’opération du maître de forges consiste principalement à acheter des bois qu’il convertit en charbon, puis en fonte et en fer, et c’est surtout le prix auquel il achète le bois qui détermine le prix auquel il peut livrer sa marchandise au commerce. La rareté du bois est la cause qui force à éteindre les fourneaux. Au commencement du dix-huitième siècle, on comptait en Angleterre trois cents hauts-fourneaux en activité ; ils se trouvaient réduits, vers le milieu du même siècle, au nombre de cinquante-neuf seulement, à cause du manque de bois ; mais l’industrie anglaise vint à bout de relever cette branche importante de manufacture, en substituant au bois un combustible minéral dans la fabrication de la fonte qui n’était pas destinée à être convertie en fer, et en adoptant plusieurs procédés ingénieux tendant à économiser la consommation du bois dans les diverses manipulations qui avaient pour objet de donner au fer en barres ces formes carrées ou cylindriques, plates ou laminées, qu’exigent les services différents auxquels il doit être employé.
L’épuisement de bois a mis également les maîtres de forges de la Russie dans la nécessité d’abandonner leurs exploitations, mais sans que ce genre d’industrie en éprouvât aucune diminution, de nouvelles forges s’étant élevées à mesure dans d’autres parties de l’empire où le bois se trouvait surabondant.
Le bois propre au service des forges, et qui est désigné dans le commerce sous le nom de charbonnette, forme, en France, environ un quart, quant à sa valeur, du produit total de nos bois, et fournit annuellement, à ce que l’on croit, deux millions huit cent mille cordes. La corde de ce bois est de quatre-vingts pieds cubes. D’après ce qu’on lit dans le Journal des Mines***, on brûle dans les fourneaux du département du Cher deux cent quarante pieds cubes de bois pour obtenir un quintal métrique de fer, et, d’après les observations faites par le baron de Dietrich****, on brûle, en Alsace, deux cent trente pieds cubes de bois pour fabriquer la même quantité de cent kilogrammes de fer. En prenant pour moyenne deux cent quarante pieds cubes de bois, il s’ensuivrait que, pour obtenir les quatre-vingt-huit millions de kilogrammes de fer auxquels on évalue la consommation de la France actuelle, année commune, il faudrait consommer deux millions six cent quarante mille cordes. Il ne resterait donc pour le charbon des villes, celui des autres manufactures et des cuisines, que cent soixante mille cordes. Mais Paris seul brûle, par an, quatre millions de pieds cubes de charbon, qui sont le produit de cent soixante mille cordes de bois de charbonnette. Comme l’emploi de la houille dans las procédés de la fonte est si peu répandu qu’il ne mérite pas d’entrer en compte, on doit inférer des calculs ci-dessus que le travail des forges, poussé au delà de sa mesure naturelle, a entraîné nécessairement une cherté permanente et un renchérissement dans le charbon du commerce, ainsi que dans les bois destinés au même chauffage, au même service de la boulangerie et à quelques autres usages semblables.
D’après les recherches les plus étendues et les plus exactes, on s’est assuré qu’en 1814 la consommation du charbon dans les forges françaises était de cinq cents à cinq cent vingt-cinq parties pondérables pour cent parties de fer, et c’est sur cette proportion que les maîtres de forges ont établi le prix de leurs fers au taux de 37 à 30 fr. les cinquante kilogrammes.
Mais si l’on compare cette consommation des forges françaises à celle qui a lieu dans les autres forges de l’Europe, on se convaincra aisément combien le travail du fer en France, à cette époque, était éloigné du point d’amélioration auquel il lui était facile de parvenir.
Dans le travail ordinaire du fer, la consommation de charbon se divise en deux parties ou en deux degrés différents de main-d’œuvre, qui sont : 1° la conversion du minerai en fonte ; 2° l’affinage de la fonte ou sa conversion en fer.
Dans la première de ces opérations, la consommation varie de soixante-six à cinq cent quarante-neuf parties pondérables pour cent parties ; et, sur plus de cent hauts-fourneaux de différents pays, la moyenne s’est trouvée être de cent soixante-deux parties*****. La consommation de fonte et de charbon, pour obtenir du fer, est de cent dix à cent cinquante parties pondérables de fonte, et de cent dix à deux cent soixante parties de charbon pour obtenir cent parties de fer******.
En général, la consommation moyenne de charbon pour obtenir cent parties pondérables de fer par la méthode des hauts-fourneaux et de l’affinage de la fonte, varie entre deux cent quatre-vingt-cinq et quatre cent quarante parties******* ; en sorte que la plus forte consommation, dans ces différents pays, est encore de 13 à 20 pour 100 plus faible que celle qui a été observée dans nos forges. Il ne faudrait cependant pas se figurer que cette différence dans la proportion du charbon consommé procédai de la nature du minerai qu’on traite en France : on voit, dans l’ouvrage déjà cité, que cette différence doit être uniquement attribuée au mode de procéder dans les deux opérations, et qu’il suffirait d’adopter de meilleures méthodes pour obtenir une économie de deux cinquièmes dans la quantité du charbon brûlé.
Dégagés de toute concurrence étrangère, les maîtres de forges de France n’avaient guère d’intérêt à économiser la consommation de charbon, et ils préféraient suivre les vieilles routines plutôt que de travailler à introduire les méthodes nouvelles. Comme ils étaient assurés de se faire rembourser de leurs avances, et qu’ils pouvaient élever le prix de leur marchandise en raison de leurs frais de fabrication, ils ne songèrent qu’à se procurer les bois qui leur étaient nécessaires. Aussi la corde de charbonnette qui, dans quelques cantons du royaume, ne se vendait guère, en 1790, que 50 à 40 sous, tripla de valeur dans ces mêmes endroits, et même elle se paya jusqu’à 8 et 9 fr. dans quelques districts de la Normandie où l’on s’était avisé d’élever des fourneaux pour la fabrication de la fonte moulée.
Tel était l’état des choses au moment où la paix générale fut rétablie en Europe, et quand les relations commerciales purent reprendre, entre la France et les pays du Nord, sur le pied où elles étaient avant 1789. Alors les maîtres de forges virent avec effroi l’introduction des fers de la Suède et de la Russie, qui pouvaient entrer dans les ports de France au prix de 10 à 12 fr. les cinquante kilogrammes, c’est-à-dire à moins de moitié du prix qu’avaient chez nous les fers d’une qualité à peu près pareille. Ils se liguèrent donc entre eux et réclamèrent en corps contre cette invasion des fers étrangers, en invoquant la protection due par le gouvernement à l’industrie nationale, et, suivant le langage toujours usité en pareil cas, ils ne manquèrent pas de prédire les plus sinistres conséquences et de montrer la France comme sur le point de manquer de fer même pour sa défense et pour ses premiers besoins, ou du moins réduite à subir le joug de l’étranger pour se procurer un des articles les plus importants de notre consommation.
Les clameurs de l’intérêt privé n’ont pas de peine à étouffer la voix toujours calme de la raison et de la justice. Quelques autres intérêts particuliers qui se trouvaient menacés par la prohibition des fers étrangers ne parvinrent pas à se faire entendre, parce que la masse de ces intéressés était peu nombreuse. Les maîtres de forges, au contraire, se fortifièrent encore de l’alliance des propriétaires de bois, qu’ils entraînèrent dans leur ligue en leur persuadant que l’introduction des fers étrangers, anéantissant toutes les forges en France, laisserait une grande partie de bois sans aucune espèce de valeur. Enfin une loi fut rendue en décembre 1814, dont la première disposition fut de prohiber l’entrée des fontes en gueuse dont lé poids serait au-dessous de quatre cents kilogrammes , ce qui a pour but de réserver aux seuls maîtres de forges français la fourniture du lest des vaisseaux de la marine royale, au grand préjudice de l’administration, qui eût pu se procurer ce lest à très-bon marché, et qui se trouve forcée de le faire venir de loin et de supporter , outre l’élévation du prix d’achat, les frais d’un transport très-dispendieux.
Par les autres dispositions de cette loi, les fers en barres ou fers de commerce sont chargés d’un droit d’entrée de 13 fr. par cent kilogrammes, ce qui est quinze fois le droit que payaient les fers de Suède avant 1791. Le droit fut porté à 25 fr. pour les petits fers, et à 40 fr. pour le fer de platinerie, connu sous le nom de tôle. Enfin l’acier fut grevé du droit énorme de 45 fr. par quintal métrique, ce qui peut être considéré comme une prohibition absolue de l’acier d’Allemagne, qui, jusqu’alors, avait été employé avec grand avantage pour la fabrication des ressorts de voiture et pour quelques autres usages. Heureusement, sur ce point, l’activité de notre industrie, si puissamment secondée par les recherches et les études de nos savants, est venue à bout de se passer du secours des aciers étrangers. Mais la cherté générale du fer de commerce, maintenue par la loi de 1814, est une calamité contre laquelle ne peuvent rien les efforts de l’industrie ni les découvertes de la science. Cette calamité affecte presque tous les arts et métiers, et pèse particulièrement sur l’agriculture. On croit que, dans le cours de l’année, chaque charrue qui travaille donne lieu à une consommation de cinquante livres pesant de fer que le laboureur pouvait aisément se procurer, avant 1790, pour une somme de 7 livres 10 sous au plus, et qui, maintenant, lui coûte au moins trois fois cette somme. Ainsi, s’il y a, comme on le suppose, neuf cent vingt mille charrues mouvantes en France, le renchérissement seul du fer grève l’agriculture d’un nouvel impôt de 14 millions. On assure que la construction d’un vaisseau de premier rang coûte aujourd’hui trois cinquièmes de plus qu’auparavant, par le seul effet de l’élévation survenue dans le prix du fer qui doit entrer dans cette construction.
On ne saurait donc trop se hâter de révoquer une loi aussi désastreuse, pour revenir à cet ancien état de choses qu’une épreuve de plus d’un siècle a dû faire suffisamment apprécier. La concurrence des fers étrangers ramènera parmi nos maîtres de forges une émulation et une activité dont ils ont depuis longtemps perdu l’habitude, et les forcera à sortir de cette ornière dans laquelle leur industrie reste immobile. On se rassurera contre cette crainte chimérique de l’anéantissement des forges en France, quand on observera qu’il existe dans ce royaume plusieurs districts très-étendus, abondants en minerai, et dont les bois ne peuvent être employés à aucun autre service qu’à alimenter les fourneaux. Il faut songer aussi que le fer étranger ne peut pénétrer fort avant dans l’intérieur des terres, parce que cette marchandise n’ayant qu’une valeur très-faible relativement à son poids, renchérit au double de son premier prix lorsqu’elle est transportée par terre à une distance tant soit peu considérable ; qu’en conséquence, dès que notre industrie en ce genre d’exploitation se sera élevée au niveau de celle des autres pays de l’Europe, il est extrêmement probable que les consommateurs français trouveront presque toujours de l’avantage à se fournir de fer de fabrication nationale, et que l’importation des fers de l’étranger n’aura jamais intérêt \& dépasser les limites dans lesquelles elle se tenait autrefois circonscrite, et qui ne comprenait pas plus d’un cinquième de notre consommation totale.
Garnier.
↑ Les quatre gouvernements qui composaient la Nouvelle-Angleterre avant la révolution d’Angleterre étaient Massachusets, Connecticut, New-Hampshire et Rhode-Island. Le premier avait eu, en 1684, sa charte révoquée et tous ses priviléges supprimés par Charles II.
↑ Ce qui se passe aujourd’hui aux États-Unis démontre avec la dernière évidence la vérité de cette observation d’Adam Smith. A. B.
.
↑ L’opinion développée ici par A. Smith, relativement à l’augmentation des profils, par suite du monopole, dans le commerce des colonies, n’est qu’une conséquence de sa théorie, qui fait dépendre le taux des profits de la quantité du capital et de l’étendue du champ ouvert à son emploi. En fait, cependant, le taux des profits dépend plutôt de la fécondité d’une entreprise industrielle que de l’espace ouvert à ses opérations. Les profits ne sont autre chose que des valeurs nouvelles créées par remploi des capitaux et du travail dans les entreprises industrielles, qui restent, quand le capital et la valeur du travail ainsi employés ont été mis de côté. Il est évident, par conséquent, que l’étendue des opérations n’est pour rien dans cette production. Et si A. Smith pense qu’en étendant les opérations le monopole du commerce des colonies élève en même temps le taux des profits, il est évident qu’il lui attribue des effets qu’il n’a pas réellement. Mac Culloch.
↑ La politique que la Grande-Bretagne et les autres nations ont suivie relativement à leurs colonies a été traitée par A. Smith d’une manière tellement complète, qu’il serait inutile de rien ajouter sur cette matière, si ce n’est quelques mots sur l’influence que, selon A. Smith, le monopole exerce sur l’élévation du taux des profits. Il ne sera pas difficile de démontrer que ceci est une erreur. Le taux des profits ne dépend pas de l’étendue du champ ouvert à l’emploi d’un capital, mais de la production de l’industrie dans laquelle un capital est engagé. Les profits ne sont autre chose qu’un excédant de valeur, résultat de l’emploi d’un capital et du travail, qui reste après la déduction du capital et des salaires du travail. Il est, par conséquent, évident que la seule étendue du champ ouvert à l’emploi du capital, quelque grande qu’elle puisse être, ne saurait produire un pareil résultat. Supposons, pour rendre ceci plus clair par un exemple, que, par un décret de la Providence, un million d’arpents de terre soit ajouté à la Grande-Bretagne, l’influence de cette augmentation du sol sur le taux des profits dépendrait alors entièrement de la fertilité de ces nouveaux arpents. S’ils n’étaient pas plus productifs que les terres pauvres que nous cultivons maintenant, 500 ou 1 ,000 millions livres sterling pourraient être mis dans cette nouvelle culture, sans que pour cela le taux des profits éprouvât une augmentation. Si les fermiers des mauvaises terres qui sont cultivées maintenant gagnent dix quarts ou 10 livres sterling sur l’emploi d’un capital déterminé, ils retireront évidemment le même profit d’un capital égal engagé dans la culture des terres de la qualité de celles dont nous venons de parler. Mais si les terres ainsi ajoutées rapportaient plus que les terres de dernière qualité actuellement cultivées, le taux des profits s’élèverait, non point à cause de l’accroissement de l’espace ouvert à l’emploi des capitaux , mais parce que la production serait devenue plus considérable. Car, au lieu d’un rapport de dix quarts ou de dix livres sterling rendu par les mauvaises terres mises en culture maintenant, il y en aurait un de douze ou de quinze quarts, ou de douze ou de quinze livres sterling.

Mais on a dit que le monopole du commerce des colonies avait précisément ce double effet ; que d’un côté il étendait le champ des opérations, et que d’un autre côté il les rendait plus productives. Et voici comment, selon la théorie d’Adam Smith, il arrive presque toujours que, par suite de l’ouverture de nouvelles voies dans le commerce extérieur, les premiers marchands qui en profitent réalisent des bénéfices plus gros qu’à l’ordinaire. Ces bénéfices considérables engagent d’autres capitalistes à retirer leurs fonds d’emplois moins lucratifs ; de telle sorte que la quantité de marchandises sur le marché intérieur en diminue. Mais comme la demande reste toujours la même, il s’ensuit nécessairement une hausse dans les prix, et par conséquent une augmentation des bénéfices. Ce système a déjà été réfuté par M. Ricardo. Une certaine partie du revenu national est dépensée en marchandises étrangères. Quand, par suite du monopole, ou de toute autre manière, des voies nouvelles s’ouvrent au commerce, trois cas se présentent : ou la quantité du revenu national dépensée en marchandises étrangères restera la même, ou elle s’augmentera, ou elle diminuera.
Dans le premier cas, c’est-à-dire quand la quantité dépensée reste la même, il est clair que les demandes de produits indigènes resteront également les mêmes ; il n’y aura donc pas de changement du tout.
Dans le second cas, c’est-à-dire en supposant que la quantité du revenu national dépensée en articles étrangers devienne plus considérable, il est évident que les demandes de produits indigènes diminueront en proportion de cette augmentation ; une portion des capitaux et du travail, employée jusqu’à présent dans la production des articles destinés au marché indigène, sera ainsi forcée de chercher un nouvel emploi dans la fabrication des marchandises destinées à être expédiées au dehors en échange des envois étrangers, devenus plus considérables. Ainsi, chaque augmentation de demandes de produits étrangers, amenant forcément avec elle les moyens de se la procurer, sans qu’on ait besoin de recourir à une augmentation du capital national, il en résultera évidemment que les prix, et par conséquent le taux des profits, n’en éprouveront aucune hausse.
Il ne nous reste que le troisième et dernier cas à examiner. En supposant que, par suite du bas prix des produits étrangers, une portion moins grande du revenu national suffise pour se les procurer, il est évident que le capital nécessaire pour la fabrication des marchandises à donner en échange pourra être moins considérable ; il y aura ainsi des capitaux disponibles qui, par conséquent, chercheront à s’employer dans la production des marchandises destinées au marché intérieur. C’est le marché intérieur qui profitera ainsi de cette portion des capitaux qu’on n’aura plus besoin d’affecter aux achats extérieurs. Dans chacune de ces suppositions, que le capital destiné à l’achat des denrées étrangères reste le même, qu’il augmente ou qu’il diminue, jamais la découverte, ou formation de nouvelles voies pour la concurrence, ne pourra avoir d’influence sur le taux des profits.
Il est vrai que, si par le moyen du commerce extérieur nous pouvons obtenir des grains ou d’autres articles nécessaires aux cultivateurs, à un plus bas prix que par la production à l’intérieur, le taux des salaires baissera, et il pourra y avoir une hausse dans le taux des profits. Mais ce résultat ne sera dû en aucune façon au monopole ; on le devra à l’importation libre et illimitée de la part des étrangers aussi bien que des colons.
Il est inutile d’ajouter que les principes développés par A. Smith ont été pleinement confirmés par les conséquences de la guerre avec l’Amérique. Notre commerce avec les États-Unis, à partir de l’époque de leur indépendance, a toujours suivi le mouvement de leur développement progressif, et aujourd’hui il est aussi considérable qu’alors, que nous avions un gouverneur dans chaque État. Nous avons donc tous les avantages d’un commerce actif, sans les charges que nous imposaient le gouvernement et la défense d’établissements aussi éloignés et étendus.
L’explication donnée par A. Smith sur les causes du développement rapide et de la prospérité des colonies, fondées dans des situations avantageuses, bien que combattue par Sismondi et d’autres, parait d’accord avec la théorie et l’histoire. Quand une colonie est fondée dans un lieu inhabité ou peu peuplé, chaque colon occupe une assez grande étendue des meilleures terres ; il n’a ni rente ni impôts à payer ; et, comme sa provision d’articles manufacturés lui arrive, soit de la métropole, soit d’un autre pays, il pourra appliquer toute son énergie à l’agriculture, qui, dans de pareilles circonstances, est très-productive. Les demandes de travail dans ces colonies sont très-grandes ; car le taux élevé des salaires, ainsi que le bas prix des terres, font du laboureur un propriétaire qui bientôt, à son tour, peut employer d’autres laboureurs.
De cette manière, la population et le bien-être augmentent d’une manière extraordinaire ; et il y a des exemples, ainsi que cela est arrivé aux États-Unis, que, pendant un laps de temps très-considérable, ils se soient accrus du double tous les vingt ou vingt-cinq ans. — Mais, tout en établissant que la facilité de tirer des richesses d’un sol fertile et inoccupé soit la principale cause du développement rapide des nouvelles colonies, on ne prétend pas dire que ce soit la cause unique. Une position favorable aux entreprises commerciales et une grande supériorité dans la navigation peuvent procurer aune colonie une très-grande prospérité, sans même qu’il y ait une étendue de territoire très-grande, et plus rapidement même que s’il n’y avait eu qu’un vaste territoire à exploiter. C’est ainsi que les colonies grecques, auxquelles A. Smith fait allusion, se sont rapidement étendues. Les plus célèbres d’entre elles, telles que Syracuse et Agrigente en Sicile, Tarente et Locri en Italie, Éphèse et Milet dans l’Asie Mineure, étaient les entrepôts les plus riches de l’ancien monde. Toutes ces villes étaient des ports de mer ; elles étaient fondées dans des situations favorables aux entreprises commerciales, et devaient leur grandeur et leurs richesses surtout au commerce et à la navigation. Mais comme leurs territoires étaient très-limités, soit par suite des difficultés qu’elles éprouvaient à se soumettre les populations indigènes, soit par suite du voisinage de colonies fondées pardes États rivaux, leur puissance n’était pas basée sur des fondements larges et solides ; de sorte que la chute des métropoles entraînait presque toujours l’anéantissement des colonies. — Les colonies fondées dans les temps modernes ont été placées dans des circonstances tout à fait différentes. D’abord, ou les pays dans lesquels elles furent établies étaient peu habités et presque déserts, ou ils étaient occupés par une race faible et incapable de résister aux envahissements des colons. Ces colonies occupaient donc de très-vastes territoires, et avaient en général plutôt un caractère agricole que commercial. Cette circonstance, en les rendant plus fortes, une fois les difficultés du premier établissement vaincues, n’a aucunement empêché le développement de leur prospérité ; tout au contraire, les plus florissantes des colonies anciennes ne sauraient se comparer, sous le rapport de la puissance et de la grandeur, aux États-Unis ; et si les colonies espagnoles et portugaises se sont développées plus lentement, il ne faut pas en attribuer la cause à la trop grande étendue de leurs territoires, mais à la mauvaise politique de la métropole vis-à-vis d’elles et aux restrictions vexatoires imposées au commerce avec les étrangers. Mac Culloch.
↑ Adam Smith aurait dû donner les preuves de cette assertion. L’Essay de sir Matthieu Decker, qu’il cite, est un ouvrage ingénieux et estimable ; mais on est forcé d’admettre néanmoins que la décadence du commerce étranger, dont il essaye d’assigner les causes, n’a en fait aucune réalité. Toutes les branches de notre commerce étranger n’ont fait que se développer progressivement pendant le dernier siècle. Mac Culloch.
↑ Une hausse dans les profits occasionne une hausse dans le prix de certains produits ; mais elle occasionne en même temps une baisse dans le prix de certains autres ; de sorte qu’on peut dire, généralement parlant, que son effet doit être nul*. Mac Culloch.
↑ Il est étrange que Smith ait avancé que le monopole du commerce des colonies nous a exclus de quelque branche productive du commerce européen, lorsque, à l’exception peut-être du commerce avec la France, nos relations avec les autres contrées étaient beaucoup plus grandes qu’elles n’avaient jamais été auparavant. Mac Culloch.
↑ C’est par le taux du profit net rendu par le capital, que l’on doit déterminer la nature avantageuse des divers emplois dans lesquels il est engagé ; et s’il est remplacé trois ou quatre fois l’année lorsqu’on l’emploie dans le commerce intérieur, et une fois seulement lorsqu’il est employé dans le commerce étranger, Ce seul retour sera égal au total des autres. Mac Culloch.
↑ Cette simple phrase répond en partie à toutes les observations précédentes de Mac Culloch ; et les faits industriels qui se passent aujourd’hui en Angleterre démontrent assez le danger et le désavantage qu’il y a pour un peuple à employer presque tous ses capitaux dans le commerce étranger. A. B
.
↑
On peut remarquer qu’Adam Smith est généralement disposé à exagérer les effets des mesures artificielles adoptées par les législateurs en faveur du développement du commerce, et son raisonnement sur le monopole colonial de l’Angleterre confirme cette observation. Il parait être de l’avis que les relations commerciales entre la Grande-Bretagne et l’Amérique étaient principalement dues à l’influence du monopole ; il prétend qu’avant l’établissement du monopole les capitaux de la Grande-Bretagne étaient profondément engagés dans le commerce avec l’Europe, et qu’ils ne s’en étaient détournés que pour suivre le commerce infiniment plus lucratif des colonies américaines, résultat de l’établissement du monopole.
D’après cette allégation, on pourrait naturellement conclure qu’après la révolution de l’Amérique et l’abolition des anciennes restrictions commerciales, le commerce du monde aurait repris son cours naturel ; que ceux qui jusque-là avaient été exclus du commerce américain en auraient pris leur part, et que le commerce de l’Angleterre, n’étant plus favorisé par. le monopole, aurait baissé jusqu’à son niveau primitif. Ceci en effet aurait eu lieu, si l’Angleterre avait dû à ce monopole les avantages que lui attribue Adam Smith.
Mais c’est précisément le contraire qui a eu lieu. Le commerce entre la Grande-Bretagne et l’Amérique, loin de baisser par suite de l’abolition du monopole, a au contraire doublé, et sa part dans le commerce général est aussi considérable que jamais. En 1772, la valeur des exportations de l’Angleterre pour l’Amérique du Nord et les Indes Occidentales s’éleva à 5,155,734 liv. sterling (128,903,350 fr.) ; et avant l’interruption du commerce entre les deux pays par suite de leurs querelles sur les droits des neutres, l’exportation de l’Angleterre pour l’Amérique seule s’était élevée à 42,000,000 l. st. (300,000,000 fr.) ; et comme aucune loi n’existait qui pût avoir produit cet effet, ont est obligé de l’attribuer simplement à l’échange de produits qui provient de la position respective de ces deux nations, dont l’une, trouvant ses principales richesses dans son agriculture, demande à l’industrie étrangère sa provision d’objets fabriqués, et dont l’autre, abondant surtout en capitaux et en industrie, achète des nations étrangères ses matières premières. C’est cette dépendance mutuelle, et non le monopole, qui forme le lien qui attache ces deux grands pays l’un à l’autre. La richesse et l’industrie, dont l’Amérique manque, ne se trouvent qu’en Angleterre ; et, d’un autre côté, c’est en Amérique seulement que l’Angleterre saurait trouver ce qui lui est nécessaire pour alimenter ses immenses manufactures. Tous les effets qu’Adam Smith attribue au monopole ont donc également lieu quand le commerce est libre ; et on a trouvé en effet qu’à l’époque de l’interruption des relations commerciales entre les deux pays par suite des mesures dirigées par le cabinet anglais contre le commerce français, nos fabricants n’avaient pas moins besoin qu’auparavant du marché américain pour l’écoulement de nos produits. Il résulte des documents communiqués au Bureau du commerce avant le rappel de ces édits, que la stagnation de nos manufactures, le désœuvrement et la misère de nos ouvriers étaient en grande partie occasionnés par la perte du marché américain ; et on pensait généralement que le libre accès de ce marché nous aurait singulièrement soulagés de la proscription générale à laquelle notre commerce se trouvait alors en butte en Europe. Buchanan.
↑ Est-il vrai que les profits nets réalisés par les marchands de Cadix et de Lisbonne auxquels Smith fait allusion, aient été réellement plus forts que ceux réalisés par les marchands de Londres ? Dans le cas où ils l’auraient été, les vicieuses institutions de l’Espagne auraient empêché les marchands d’accumuler et d’employer leur excédant comme le faisaient les marchands anglais. M. Culloch.
↑ Les premiers mouvements de l’insurrection américaine venaient à peine d’éclater, que le coup d’œil sûr et pénétrant d’Adam Smith avait prévu l’issue de la lutte qui allait s’engager, et en avait découvert toutes les conséquences. Frappés d’un aveuglement général, le peuple anglais et les conseils chargés de les diriger regardaient l’indépendance des colonies comme la ruine totale du commerce et de la prospérité de l’Angleterre, et ils se précipitaient dans une guerre qu’ils croyaient inévitable, mais dont le succès ne leur semblait pas douteux. Ne calculant que l’inégalité apparente des forces, ils traitaient avec hauteur un ennemi qu’ils dédaignaient, ou plutôt, suivant eux, un rebelle qu’il fallait châtier sévèrement pour le faire rentrer dans le devoir. Mais le temps, dont la marche, quoique tardive, est toujours déterminée par la liaison nécessaire des effets avec leurs causes, nous a appris que le philosophe qui ne partageait alors ni les craintes de sa nation sur les suites de cette indépendance, ni la confiance qu’elle avait dans ses armes, avait seul raison contre tous. Si l’on eût suivi les conseils que lui dictait sa sage prévoyance, et que l’on eût consenti une union franche et loyale entre ces deux branches d’une même famille, en accordant aux Américains une représentation au parlement proportionnée au contingent de leurs contributions, qui peut dire à quel degré de richesse et de grandeur ne se serait pas élevée la puissance anglaise par la combinaison des moyens qu’offrent la nature et la situation des deux pays, embrassant ainsi les deux mondes, dominant sur toutes les mers, et maîtresse d’un territoire presque sans bornes qui eût pu donner une assiette réelle à sa dette nationale, et rendre productive et industrieuse cette partie oisive et turbulente de sa population qui devient tous les jours plus menaçante pour la tranquillité publique ? Garnier.
↑
L’émancipation d’une colonie de la domination de la métropole parait être la conséquence naturelle de son développement progressif; et toutes les tentatives faites pour la tenir dans l’obéissance, et pour resserrer des liens virtuellement rompus par suite de la différence profonde des intérêts, ne feront qu’accélérer une séparation devenue inévitable. La Grande-Bretagne et ses colonies, avant leur séparation, n’avaient aucun intérêt commun qui les unit ; et le droit d’impôt, que l’Angleterre s’était arrogé, lui aurait seulement donné la faculté de tirer un revenu d’un pays pour des objets dont celui-ci ne se serait soucié en aucune façon. Si l’Amérique avait consenti à cette imposition projetée, une influence étrangère aurait dominé dans ses conseils ; elle aurait été exploitée pour servir des vues étrangères, et elle aurait été exposée à la dégradation et à l’esclavage. Sous quelque prétexte spécieux qu’on cherchât à déguiser ce plan, l’Angleterre au fond ne voulait que faire payer un tribut à l’Amérique ; elle voulait lui faire porter une partie des charges qui pesaient sur la métropole, pour des objets qu’elle croyait essentiels à sa sécurité et à son bien-être, mais qui n’avaient aucune importance réelle pour l’Amérique.
La tendance de l’Angleterre à se mêler des affaires de l’Europe est constatée par son histoire ; et l’Amérique, exempte, par sa situation même, des dangers réels ou imaginaires qui menacent la Grande-Bretagne, aurait été enveloppée dans toutes ses querelles ; elle aurait eu à supporter des taxes pour des guerres dans lesquelles elle n’aurait eu aucun intérêt ; ses ressources auraient servi à une politique étrangère ; et ç’auraient été les besoins de l’Angleterre, et non les siens propres, qui auraient déterminé la mesure de ses contributions.
Et pourquoi l’Amérique aurait-elle renoncé au droit de s’imposer elle-même ? Pourquoi une grande nation, ayant l’intelligence de sa politique intérieure et extérieure, irait-elle demander la distribution de ses impôts à un pays étranger ? Une imposition venant de l’Angleterre, n’importe sous quelle forme, aurait été un coup mortel pour la liberté américaine, et c’est avec raison que M. Burke adressa aux partisans du système de taxation pour l’Amérique, ces paroles : « Quelle sera à l’avenir la liberté dont jouiront les Américains, et de quelle espèce d’esclavage resteront-ils exempts, si dans leur propriété et leur industrie vous les frappez par les lois que vous imposez au commerce, et si en même temps vous en faites une espèce de bêtes de somme pour les taxes que vous jugerez convenable d’établir, sans leur laisser la moindre part dans ces règlements ?… »
« S’ils portent, continua-t-il, le fardeau du monopole illimité (unlimited monopoly), leur ferez-vous également porter le fardeau des revenus publics ?… L’Anglais de l’Amérique sentira que c’est là l’esclavage, et il ne trouvera pas de compensation ni pour ses sentiments, ni pour sa raison, dans la considération que c’est un esclavage légal. » (Discours sur l’Impôt américain, œuv. de Burke, IIe vol., p. 435.)
Il était évident que les Américains ne pouvaient pas se soumettre à cette injustice ; il était également clair que l’Angleterre ne pouvait pas se relâcher de ses prétentions, sans reconnaître implicitement l’indépendance de l’Amérique. C’est seulement par le droit d’impôt que la métropole pouvait espérer de tirer quelque avantage de sa souveraineté sur les colonies. L’Amérique, en fournissant un revenu à la Grande-Bretagne, aurait été une dépendance utile de sa puissance ; mais sans ce revenu, sa soumission paraissait complètement inutile aux partisans de l’Angleterre. Mais il ne s’ensuit pas que, parce que l’union entre l’Amérique et l’Angleterre était sans utilité, elle dût devenir utile à la Grande-Bretagne aux dépens de l’Amérique. Lorsque la Grande-Bretagne eut perdu, par rapport a l’Amérique, toute espèce d’influence légitime et naturelle, il fallait en conclure, non point qu’elle dût ressaisir une autorité illégitime et usurpée, mais qu’elle eût à renoncer à des prétentions désormais sans réalité.
C’est de cette manière seulement qu’on aurait pu fonder un établissement durable ; et il est à regretter que les gouvernants de ce pays-ci, voyant que leur empire transatlantique devenait un simple hochet de la vanité nationale, et qu’aucun revenu ne pourrait en être retiré par l’Angleterre, n’aient pus contribué à fonder un pareil établissement. Si leur politique avait suivi cette direction, la paix et la conciliation en seraient résultées, et l’histoire du pays n’offrirait pas le spectacle d’une guerre sanglante ; et les petits-fils n’auraient pas eu a payer les querelles et les folies de leurs aïeux. Buchanan.
↑ Liv. V, chap. ii.
↑ Les capitaux sont toujours attirés vers les entreprises qu’on considère comme devant, à circonstances égales, rapporter le plus ; et on peut dire que des affaires, quelle que soit d’ailleurs leur différence, offrent au public les mêmes avantages quand elles rapportent des bénéfices égaux.
Mac Culloch.
↑ Ceci est inexact. Le commerce des Indes n’a jamais été ouvert à tous les Portugais. À l’exception d’une très-courte époque pendant laquelle il était livré à deux compagnies privilégiées, ce commerce a toujours été exploité par un monopole royal, à l’aide d’un certain nombre de navires particulièrement autorisés à ce trafic, à peu près comme l’a été le commerce des Espagnols avec leurs colonies de l’Amérique du Sud. Le commerce intérieur des Indes a été en très-grande partie cédé par le roi à des particuliers. Les gouverneurs et autres personnages haut placés vendaient des permissions de trafiquer à des individus qui n’avaient d’autre mérite que de pouvoir les leur acheter. Mac Culloch.
↑
Dans tous les temps, à remonter jusqu’à ceux de la plus haute antiquité, le commerce de l’Inde, qui comprend celui de la Chine, a été, par la variété et l’attrait particulier des productions dont il se compose, l’objet de l’ambition de tous les antres peuples du monde. Ce que la magnificence a pu étaler de plus éblouissant, ce que le luxe des jouissances a pu imaginer de plus exquis et de plus recherché, a toujours été fourni au reste de la terre par cette contrée privilégiée. À mesure que la civilisation et le raffinement se sont étendus parmi les nations, cette passion universelle pour les produits de l’Orient a pris encore plus d’énergie et a trouvé un nouvel aliment dans des objets jusqu’alors inconnus. Le thé, qui parait avoir été de toute ancienneté la boisson favorite des Chinois, apporté pour la première fois en Europe, il y a moins de cent quarante ans, forme aujourd’hui, à lui seul, dans le commerce du monde, une valeur presque égale à tous les produits réunis des mines précieuses du Mexique et du Pérou, et il est vraisemblable que la production de cette feuille est encore bien au-dessous de ce que la consommation doit lui demander un jour. Ce seul article établit entre la Chine et l’Europe un lien qu’aucune révolution humaine ne saurait rompre, et que chacun des peuples qui y touchent a un égal intérêt à maintenir.
Pour les nations de l’Europe, la route la plus directe et la plus naturelle de ce commerce, c’est celle de Suez et de la mer Rouge, et c’est celle qui a été pratiquée dans les temps les plus anciens. Dans ces âges, que l’histoire ne nous laisse apercevoir qu’à travers d’épaisses ténèbres, les Arabes allaient, à ce qu’il semble, chercher les denrées de l’Inde en côtoyant le golfe qui les sépare de cette contrée, et ils les revendaient comme productions de leur propre pays. Les Phéniciens, en mettant à profit le préjugé superstitieux qui éloignait les Égyptiens de toute entreprise maritime, s’emparèrent de ce riche commerce et l’enlevèrent aux Arabes. Alexandre, rétablissant l’Égypte dans ses droits naturels, y fonda cette ville célèbre qui fut, pendant dix-huit siècles consécutifs, le centre où venaient se rendre la plus grande partie des immenses richesses de l’Orient, destinées à la consommation des régions occidentales.
La prévention qui a longtemps existé contre la navigation de la mer Rouge est maintenant démentie par les rapports des voyageurs et par des observations plus exactes. James Bruce explique comment le commerce des anciens, en suivant cette route, se trouvait secondé par les vents périodiques qui soufflent dans des directions favorables , soit dans le golfe Arabique, soit dans l’Océan Indien. Le travail qui a été entrepris d’une carte de la mer Rouge, ainsi que de la description des courants qui y régnent, confirme les conjectures de ce voyageur.
Ce ne sont pas des obstacles naturels qui ont intercepté cette antique route de l’Inde ; elle a été abandonnée par une suite de ces révolutions purement humaines, mille fois plus désastreuses que la fureur des éléments. La dispersion de l’empire romain par les Barbares, et l’invasion de l’Egypte par les Mahométans, sont les événements qui ont forcé le commerce de l’Inde à quitter sa route naturelle pour celle de Constantinople, par le golfe Persique et par la mer Noire, en suivant la terre jusqu’à Trébizonde. C’est principalement par cette voie que les Vénitiens, les Génois, les Pisans et les Lombards fournirent les marchés de l’Europe des productions de l’Orient. Les Génois surtout formèrent à Gaffa un établissement qui découragea tous leurs concurrents. Ce fut alors que les Vénitiens, se voyant supplantés par leurs rivaux, se retournèrent vers l’Egypte, et, profilant des troubles intérieurs de ce pays, qui s’était détaché de l’empire des califes, traitèrent avec le sultan, et cherchèrent même à s’assurer un monopole qui finit par indisposer contre eux l’Europe entière, et par donner naissance à la ligue de Cambrai. Peu après la prise de Constantinople par Mahomet II, les Génois furent chassés de la Crimée, elles richesses indiennes ne parvinrent plus en Europe que par les rives de la mer Rouge, à travers tous les obstacles et les vexations que suscitaient l’insatiable rapacité des Arabes et l’inquiète jalousie des Mamelucks. Tel était l’état des choses à la fin du quinzième siècle, lorsqu’un Portugais osa doubler le cap de Bonne-Espérance, et s’ouvrit un nouveau passage dans l’Océan Indien. Cet événement, dont on a tant vanté l’importance, a dû tous ses effets bien moins à la découverte en elle-même qu’aux circonstances dans lesquelles elle a été faite. Elle eut lieu dans un temps où les deux communications pratiquées jusqu’alors (celle de l’Egypte et celle de Constantinople) se trouvaient livrées à des barbares étrangers à toute idée de commerce ; dans un temps où les progrès de l’industrie, de la navigation et de tous les arts de la civilisation suivaient en Europe une marche rapide ; dans un temps enfin où les mines de l’Amérique allaient bientôt offrir, avec une abondance jusqu’alors inconnue aux hommes, ces métaux précieux qui sont la principale marchandise qu’on puisse porter aux Indes. C’est la réunion de toutes ces circonstances, et non pas la découverte du nouveau passage, qui a amené l’ère nouvelle du commerce ; ces circonstances ne pouvaient pas manquer leur effet ; et si Vasco de Game n’eût pas doublé le cap de Bonne-Espérance, infailliblement, un peu plus tôt ou un peu plus tard, les autres communications eussent été forcées.
Mais la puissance maritime était alors entre les mains de peuples qui ne possédaient point de ports sur la Méditerranée , et qui, sans le passage du Cap, n’auraient eu aucun espoir de prendre jamais une part directe dans le commerce des Indes. Les Portugais, les Hollandais, les Anglais ont dû chercher à exalter cette découverte et à détourner l’attention des autres peuples de toute tentative vers une autre route ; ils ont tellement redouté de telles entreprises, qu’Albuquerque, le chef des premiers aventuriers portugais, avait conçu, dit-on, le projet de tailler les rives du Nil et de détourner dans la mer Rouge le cours de ce fleuve, afin d’enlever à la Basse-Égypte la source de sa fécondité, changer en un désert inhabitable cette fertile contrée, et porter ainsi une affreuse solitude pour barrière entre la Méditerranée et le golfe Arabique. Cet abominable stratagème qui, pour la conservation d’un monopole, se proposait de retrancher à jamais une portion de l’espèce humaine en diminuant la terre habitable , a été recommandé tout récemment par un écrivain anglais à la Compagnie des Indes, comme une dernière ressource* ; mais il n’a pas été nécessaire de recourir à ces extrémités, et les nations voisines de la Méditerranée, satisfaites d’obtenir quelques établissements dans l’Inde ou de se procurer indirectement les produits de cette contrée, n’ont pas même paru songer à s’y frayer un passage ; et quand elles en auraient conçu le projet, leurs divisions politiques auraient vraisemblablement empêché le concert nécessaire à son exécution. C’est donc une opinion qui s’est généralement établie depuis trois cents ans, qu’il ne faut pas songer à arriver aux Indes autrement qu’en traversant l’Océan Atlantique.
Cependant, dans les dernières années du dix-huitième siècle, un de ces hommes que la destinée semble avoir fait naître pour presser la marche des événements et déterminer les grandes révolutions, porta sur l’Égypte l’œil perçant de son génie, et prévit la crise qui se prépare depuis longtemps dans le commerce des nations, Il lut, dans les infaillibles décrets de la nature, que les arts et les sciences de l’Europe devaient un jour s’étendre sur une des contrées les plus fertiles et les plus heureusement situées du monde, et que le joug des Barbares opprime depuis douze siècles ; il vit que le commerce de l’Orient était dévolu de droit à la colonie européenne qui pourrait parvenir à s’établir en Égypte. Si ce grand projet eût pu s’accomplir, on ne peut pas douter qu’il n’eût amené des résultats de la plus haute importance.
Des causes qui agissent insensiblement depuis longtemps préparent cette grande révolution, dont l’effet sera de déplacer de leur rang les principales puissances de l’Europe. Un commerce susceptible d’une extension presque indéfinie, se trouve tout à fait concentré dans les mains de quelques insulaires avec l’autre extrémité du globe, dont les orgueilleuses prétentions révoltent tous les autres peuples. La Russie, destinée par son étendue, sa situation, son immense navigation intérieure, à monter au plus haut degré de puissance, et qui, en moins d’un siècle, a pu franchir un si prodigieux intervalle, enveloppe peu à peu la Turquie d’Europe, et s’avançant de tout son poids sur ce rival expirant, est impatiente de s’assurer la libre navigation de la Méditerranée par la mer Noire et le Bosphore. L’Autriche est enfin venue à bout d’occuper Venise, que son ambition convoitait depuis longtemps. Ces deux puissances se trouvent réunies d’intérêts avec la France, l’Espagne et les États d’Italie, pour que les richesses de l’Inde se versent en Europe par la Méditerranée. Une population de plus de cent millions d’Européens doit tourner de ce côté ses regards et ses efforts. L’intérêt général, non pas seulement pour l’Europe, mais pour l’Inde elle-même’, veut que les productions indiennes parviennent à l’Europe par la voie la plus directe ; que l’Égypte enfin soit le grand marché où l’Orient et l’Occident viennent faire l’échange des produits respectifs de leur sol et de leur industrie. Il est dans la justice que chaque nation prenne dans le commerce la part plus ou moins avantageuse que lui assigne sa situation naturelle ; et comme, en définitive, la justice est toujours l’intérêt de tous, les nations même les plus séparées de la Méditerranée y trouveront encore leur avantage. Ce grand golfe, peu agité par les tempêtes, ne présente pas ces chances périlleuses qui rendent les transports si dispendieux sur la plus orageuse des mers. Les produits qui sont la matière de ce commerce arriveront plus promptement, plus sûrement et avec moins de frais au marché des échanges et aux marchés de la consommation ; les retours seront plus fréquents, et par conséquent l’industrie sera plus sollicitée et la reproduction plus rapide. Il y aura économie de temps, de travail et de dépense dans chacune des opérations de ce commerce, et dès lors on recueillera, à égalité de frais , plus d’objets consommables. L’Inde, rendue à ses droits naturels, à la jouissance de tous les bienfaits que la nature a voulu prodiguer à son heureux climat, pourra déployer en liberté les ressources inépuisables de son sol et de l’infatigable patience de ses industrieux habitants. Il en résultera pour l’espèce humaine tout entière plus de sources de richesses, plus de moyens de jouir, plus d’occasions de travail, plus d’encouragements à la population. On objectera peut-être que dans les principes mêmes de cette justice universelle qui doit régler les droits des nations, ce vaste marché dont la nature a placé le siège en Égypte, devrait être tenu par les habitants du pays favorisé, et que c’est à eux qu’il appartient de recueillir les avantages de l’heureuse situation dans laquelle le ciel les a fait naître. Mais pour pouvoir jouir de ces avantages, il est d’autres conditions à remplir qui sont hors de leur pouvoir, et ce serait anéantir ce marché que de l’abandonner aux indigènes. Il est évident qu’il n’y a qu’une colonie européenne qui puisse le faire valoir et le rendre profitable au reste du monde. Ainsi que tous les autres arts, le commerce acquiert avec la civilisation des moyens d’activité et de perfectionnement, et les peuples grossiers restent, sous ce rapport comme sous tous les autres, dans un état d’infériorité qu’aucun avantage local ne saurait balancer. Ce n’est pas seulement dans les moyens de navigation que se fait remarquer la supériorité des nations européennes ; c’est surtout dans une multitude de procédés qui augmentent dans une proportion incalculable la puissance du commerce. Le change, qui épargne les frais et les risques du transport des espèces ; les assurances, qui réduisent à des calculs positifs les chances les plus hasardeuses ; le crédit, qui multiplie les capitaux ; l’ordre, de la comptabilité, la tenue des livres, la garantie des transactions, et une foule d’autres méthodes que la haute civilisation de l’Europe a introduites et perfectionnées, ont donné à ses opérations de commerce une marche si assurée, si active et si régulière, que tout autre peuple dépourvu des mêmes secours ne saurait se livrer aux mêmes entreprises. D’ailleurs, les achats dans l’Inde, qui ne se font guère qu’en argent, exigent une grande abondance de ce métal, et nécessitent par conséquent l’emploi des mesures propres à se le procurer à meilleur compte.
On peut donc prédire avec assurance que dans un temps plus ou moins prochain, et qui ne saurait être fort éloigné, le commerce des Indes avec les nations occidentales, dégagé des liens et du joug qui l’opprime, reprendra sa marche et sa liberté primitives, et que l’Égypte, alliant à tous ses avantages natifs, les arts et l’industrie européenne, fixera à jamais cet important marché qui, ayant enfin trouvé son assiette naturelle, n’aura plus de nouvelles révolutions à éprouver. Le court irrésistible des choses les pousse toujours tôt ou tard dans la voie la plus conforme aux dispositions de la nature, c’est-à-dire la plus avantageuse peur tous, et les combinaisons privées, en opposition à l’intérêt général, ne peuvent jamais avoir qu’une consistance précaire et des succès passagers. Garnier.
↑ L’intérêt d’un propriétaire d’action dans les fonds de la Compagnie des Indes n’est pourtant nullement le même que celui du pays dans le gouvernement duquel il a de l’influence par son droit de suffrage. (Voy. liv. V, chap. 1er, sect. 3e) (Note de l’auteur.)
↑
Gouvernement, revenu et commerce de la Compagnie des Indes. — En 1784, avant la publication de la quatrième édition de l’ouvrage d’Adam Smith, tes affaires de la Compagnie des Indes Orientales étaient tombées dans un désordre profond ; de tous côtés on somma les ministres de présenter quelques projets de réforme. C’est pour se conformer au vœu de l’opinion publique que M. Fox présenta le fameux bill sur les Indes, qui avait pour but d’abolir les cours des directeurs et des propriétaires, comme ne remplissant aucun but sérieux dans l’administration, et de charger du gouvernement sept commissaires nommés par le Parlement. Cette proposition produisit une fermentation extraordinaire. La coalition entre lord North et M. Fox avait rendu le ministère excessivement impopulaire, et on profita de cette circonstance pour provoquer contre cette mesure une opposition violente. La Compagnie des Indes fit un appel au public ; elle se plaignit de la violation de ses droits garantis, bien qu’il fût manifeste aux yeux de tout le monde qu’à cause de son incapacité reconnue de donner suite aux stipulations de sa charte, renouvelée en 1781, il appartenait au Parlement de pourvoir aux difficultés par une nouvelle législation.
L’opposition représenta la mesure qui attribuait i la législature la nomination des commissaires comme un empiétement sur la prérogative de la couronne; elle accusa le ministre de vouloir étendre sou influence outre mesure, en se saisissant de ce nouveau patronage. Le bill passa à la Chambre des Communes ; mais le mécontentement général et l’opposition avouée du roi le firent rejeter à la Chambre des Lords. Cet événement entraîna la chute du ministère de la coalition. Une nouvelle administration se forma, ayant à sa tète M. Pitt ; et le Parlement ayant été dissous quelque temps après, le nouveau ministère obtint une majorité considérable dans les deux Chambres. Assuré ainsi du concours du Parlement, M. Pitt présenta son bill sur le gouvernement des Indes, qui fut adopté par tous les pouvoirs. Ce bill créa un bureau de contrôle composé de six membres du conseil privé, qui avait pour mission « de surveiller et de contrôler tous les actes et toutes les opérations qui pouvaient en aucune manière avoir rapport au gouvernement civil ou militaire, ou aux revenus des territoires et possessions de la Compagnie des Indes Orientales. »
Tout ce qui, dans les communications avec les Indes, avait trait à un des objets mentionnés, devait lui être soumis ; les directeurs étaient tenus de déférer à ses ordres et de n’obéir qu’à ses instructions. Une commission secrète de trois directeurs fut formée, avec laquelle le bureau du contrôle pouvait traiter des affaires qu’il ne jugerait pas à propos de soumettre à la cour des directeurs. Les personnes revenant des Indes étaient obligées, sous des peines sévères, de déclarer l’état de leur fortune, et un tribunal fut installé pour le jugement de ceux qui étaient accusés de concussion dans l’administration des Indes ; il était composé de trois juges pris dans chacune des trois grandes cours : la cour du banc du roi (court of king’s bench), la cour des plaids communs (court of common pleas) et la cour de l’échiquier (court of exchequer) ; puis de cinq membres de la Chambre des Lords et de sept membres de la Chambre des Communes ; ces derniers étaient élus au sort au commencement de chaque session. La haute administration des affaires commerciales restait entre les mains des directeurs.
Depuis la création de ce bureau, la direction des affaires a toujours appartenu à son président) qui est de fait le secrétaire d’Étal pour les Indes. Il est évident que dans ce système d’administration le succès des entreprises doit dépendre du caractère et de la capacité du président. En partageant ainsi, du moins en apparence, la responsabilité entre le bureau du contrôle et la cour des directeurs, on s’exposait naturellement à ce qu’aucun de ces deux corps ne mit dans la répression des abus, ainsi que dans l’amélioration de l’administration, l’énergie qu’il aurait déployée si l’autre n’avait point existé.
Le monopole du commerce de la Grande-Bretagne avec les pays à l’est du cap de Bonne-Espérance continuait de rester entre les mains de la Compagnie des Indes, sauf quelques légères modifications introduites depuis l’an 1793 jusqu’en 1813, époque où le commerce des Indes fut ouvert au public. La Compagnie ne devait rester eu possession du monopole du commerce avec la Chine que jusqu’en 1834.
L’acte de l’année précédente (3 et 4 de Guill. IV, ch. lxxxv), tout en prorogeant la charte jusqu’en 1854, abolit non-seulement ce monopole, mais il ôte également à la Compagnie le caractère originaire d’une association commerciale qu’elle avait gardé jusqu’alors. A partir de ce moment, les fonctions de la Compagnie deviennent essentiellement politiques. Elle continue, d’après le projet développé dans l’acte de M. Pilt, de gouverner les Indes jusqu’au 30 avril 1854, avec l’assistance et sous la surveillance du bureau du contrôle. Les biens meubles et immeubles appartenant à la Compagnie le 22 avril 1834 sont transportés à la couronne ; la gestion au nom de la couronne est réservée à la Compagnie. Cesdits biens demeurent passibles de toutes les obligations déjà existantes, ou qui pourront être contractées par la suite par les autorités compétentes. Les dettes et engagements de la Compagnie restent à la charge de l’Inde. Les dividendes, fixés à 10 1|2 pour 100, seront payables en Angleterre et pris sur les revenus de l’Inde. L’établissement d’un fonds de sécurité est destiné à ce service. Après le mois d’avril 1874, les dividendes pourront être rachetés par le Parlement, à raison de 200 livres sterling pour 100 livres sterling d’actions. Dans le cas où, en 1854, l’administration de l’Inde cesserait d’appartenir à la Compagnie, elle pourra exiger le rachat des dividendes trois ans après en avoir lait la notification.
Voici les principales bases de la constitution de la Compagnie, telle qu’elle existe maintenant :
1° Les actions de la Compagnie forment un capital de 6,000,000 livres sterling (150,000,000 francs). Tout le monde, nationaux et étrangers, hommes et femmes, corporations politiques ou autres********, peuvent en acquérir sans limites. Depuis 1793, les dividendes ont été de 10 1|2 pour 100 ; ils restent ainsi fixés par le dernier acte.
2° Les possesseurs d’actions peuvent, dans les assemblées générales, proposer de nouvelles lois. Ils peuvent, en outre, exercer toute espèce de recherche, d’investigation et de contrôle sur les affaires de la Compagnie ; mais le pouvoir exécutif, en faveur de la plus prompte expédition des affaires, est entièrement réservé à la cour des directeurs. Une assemblée générale doit avoir lieu dans les mois de mars, juin, septembre et décembre de chaque année. Nul ne peut assister à l’assemblée générale s’il n’est possesseur d’actions d’une valeur de 500 livres sterling (12,500 francs). Nul ne peut voter sur les déterminations à prendre si, dans les douze mois précédents, il n’a pas constamment été en possession d’actions d’une valeur de 4,000 livres sterling (25,000 francs), à moins qu’il ne vienne d’acquérir ce nombre d’actions soit par un héritage, soit par un mariage.
Les possesseurs d’actions de la valeur de 4,000 livres sterling ont une voix. 3,000 livres sterling d’actions donnent droit à 2 voix ; 6,000 livres sterling à 3 voix ; 46,000 livres sterling et au-dessus à 4 voix.
En 1825, il y avait sur les registres de la Compagnie 2,003 actionnaires, dont 1,494 avaient droit à une voix, 392 à 2 voix, 69 à 3 voix, et 48 a 4 voix. Neuf actionnaires justifiant chacun d’une possession de 4,000 livres sterling d’actions peuvent, dans une occasion particulière, par une demande écrite adressée à la cour des directeurs, demander une assemblée générale ; les directeurs sont tenus de donner suite à une pareille demande dans un délai de dix jours, sinon les propriétaires peuvent convoquer eux-mêmes une pareille assemblée par une note affichée à la Bourse. Dans toutes ces assemblées, les questions sont décidées par la majorité des voix ; en cas d’égalité des votes, la question sera décidée par un vote tiré au sort par le trésorier. Neuf actionnaires peuvent, par écrit, demander le scrutin sur une question, scrutin qui toutefois n’aura lieu que vingt-quatre heures après la clôture de l’assemblée générale.
3° La cour des directeurs se compose de vingt-quatre membres choisis parmi les actionnaires ; chacun d’eux devra avoir pour 2,000 livres sterling d’actions, et ne pourra, après son élection, rester en fonctions que s’il a gardé ses actions. Six d’entre les directeurs sont choisis, le second mercredi du mois d’avril de chaque année, {tour entrer en fonctions pendant les quatre années suivantes, eu remplacement de six autres dont les fonctions expirent. Après un intervalle de 42 mois, ceux que leur tour aura fait sortir pourront être réélus pour les quatre années suivantes. Autrefois, quiconque avait été au service de la Compagnie, soit civil, soit militaire, ne pouvait être élu directeur, s’il n’avait passé au moins deux ans en Angleterre après avoir quitté le service ; cette condition n’existe plus aujourd’hui, et maintenant tous les serviteurs civils ou militaires de la Compagnie des Indes, en supposant qu’ils aient les qualités requises pour être éligibles, sont aptes à être élus, immédiatement après leur retour en Angleterre, excepté dans le cas où leurs comptes avec la Compagnie ne seraient pas réglés ; alors ils ne peuvent être élus que deux ans après leur retour, à moins que leurs comptes ne soient réglés auparavant. (3 et 4, Guill. IV, ch. lxxxv, sect. 28.) Les directeurs choisissent parmi eux un président et un président-délégué (deputy-chairroan). Ils sont obligés, par un statut additionnel, de se réunir au moins une fois par semaine ; mais ils se réunissent encore plus souvent quand l’occasion l’exige.
Treize membres sont nécessaires pour se constituer en cour. Les déterminations sont prises à la majorité des voix. Dans les cas douteux, un avis tiré au sort par le trésorier décide la question. Dans toutes les questions importantes, la cour se prononce par un vote au scrutin. Les officiers de la Compagnie, à l’intérieur et à l’extérieur, reçoivent leurs commissions immédiatement de la cour ; ils lui rendent compte de la manière dont ils ont exécuté les ordres qui leur ont été donnés. Les affaires sont toutefois réglées de manière à ce que chaque membre de la cour ait son influence particulière et directe.
4° Les pouvoirs principaux de la cour des directeurs résident dans le comité secret qui forme une espèce de cabinet ou conseil privé. Toutes les communications d’une nature confidentielle et délicate, entre le bureau du contrôle et la Compagnie, sont soumises, en première instance du moins, aux délibérations du comité, et les ordres du bureau, dans les affaires politiques, peuvent être directement transmis aux Indes, sans avoir été préalablement vus par les autres directeurs. Le comité secret reçoit ses pouvoirs de la cour des directeurs ; ses membres prêtent serment de garder le secret. Mac Culloch.
*. Un professeur d’économie politique à l’université d’Oxford, M. Merivale, a fait, pendant les trois dernières années, de la Colonisation et des colonies le principal objet de ses cours. Ces leçons viennent de paraître en deux volumes. A. B.
.
**. M. de HumboMt : Essai politique sur la Nouvelle-Espagne, tome IV, pag. 246 de l'édil. in-8°.
***. Tome XXVI, page 321.
****. Description des gilet de minerai, tomes III et IV.
*****. Sidérotechnie, tome I, pages 208 à 210, et tome II, pages 38 et 51.
******. idem, tome III, page 75.
*******. Idem, idem, pages 48.
********. Sont exceptés le gouverneur et la corporation de la banque d’Angleterre.

%%%%%%%%%%%%%%%%%%%%%%%%%%%%%%%%%%%%%%%%%%%%%%%%%%%%%%%%%%%%%%%%%%%%%%%%%%%%%%%%
%                                  Chapitre 8                                  %
%%%%%%%%%%%%%%%%%%%%%%%%%%%%%%%%%%%%%%%%%%%%%%%%%%%%%%%%%%%%%%%%%%%%%%%%%%%%%%%%

\chapter{Conclusion du système mercantile}
\markboth{Conclusion du système mercantile}{}

Quoique l’encouragement de l’exportation et le découragement de l’importation soient les deux grandes mesures par lesquelles le système mercantile se propose d’enri­chir le pays, cependant, à l’égard de certaines marchandises en particulier, il pa­raît suivre un plan tout opposé : il décourage l’exportation et encourage l’importation. Toutefois, à ce qu’il prétend,l’objet qu’il se propose en dernier résultat est toujours le même : c’est d’enrichir le pays par une Balance de commerce avantageuse. Il dé­cou­rage l’exportation de matières premières de manufactures et des instruments de métier, afin de donner à nos ouvriers un avantage sur ceux des autres nations et de les mettre en état de supplanter ceux-ci sur tous les marchés étrangers ; et en restreignant ainsi l’exportation d’un petit nombre de marchandises de peu de valeur, il espère donner lieu à une exportation bien plus forte et d’une bien plus grande valeur dans les autres genres de produits. Il encourage l’importation des matières premières de manufactures, afin que nous puissions être dans le cas de travailler nous-mêmes à meilleur marché, et afin d’empêcher par ce moyen une importation plus forte et plus chère de marchandises manufacturées. Je ne sache pas qu’il existe d’encouragements donnés à l’importation des instru­ments d’industrie, au moins je n’en trouve aucun dans notre Recueil de statuts. Quand les manufactures ont atteint un certain degré de développement, la fabrication des instruments d’industrie devient elle-même l’objet d’un grand nombre de fabriques très-importantes. Donner un encouragement particulier à l’importation de ces instruments, c’eût été faire un trop grand tort à ces fabriques. Aussi cette importation, au lieu d’être encouragée, a été souvent prohibée. Un acte de la troisième année d’Édouard IV a prohi­bé l’importation des cardes à carder la laine, à moins qu’elles ne fussent impor­tées d’Irlande ou qu’elles ne fissent partie d’une prise ou d’un naufrage. Cette prohi­bition a été renouvelée par le statut de la trente-neuvième année d’Élisabeth, et des lois postérieures l’ont continuée et rendue perpétuelle[1].
L’importation des matières premières de manufactures a été encouragée tantôt par une exemption des droits auxquels les autres marchandises sont assujetties et tantôt par des primes.
L’importation de la Laine non ouvrée de plusieurs pays différents, celle du coton en laine de tous les pays, celle du lin non sérancé, celle de la plus grande partie des drogues propres à la teinture, celle de la plupart des cuirs non apprêtés d’Irlande ou des colonies anglaises, des peaux de veau marin de la pêche anglaise du Groenland, celle du fer en saumons ou en barres des colonies anglaises, aussi bien que celle de plusieurs autres matières premières de manufactures, ont été encouragées par une exemption de tous droits, pourvu qu’elles fussent déclarées au bureau des douanes dans les formes prescrites. L’intérêt particulier de nos marchands et manufacturiers a peut-être arraché à la législature ces exemptions, tout comme il a fait de la plupart de nos autres règlements de commerce. Elles sont néanmoins parfaitement justes et raisonnables, et si l’on pouvait, sans nuire aux besoins de l’État, les étendre à toutes les autres matières de manufactures, certainement le public ne pourrait qu’y gagner[2].
Néanmoins, l’avidité de nos gros manufacturiers a, dans certains cas, étendu ces exemptions beaucoup au-delà de ce qu’on peut justement regarder comme pure ma­tière première de leur manufacture. Par le statut de la vingt-quatrième année de Georges II, chap. xlvi, un léger droit d’un denier par livre seulement avait été établi sur l’importation des fils écrus ou roux de l’étranger, au lieu de droits beaucoup plus forts auxquels ils étaient assujettis auparavant, savoir : de 6 deniers par livre sur le fil de voiture, d’une schelling par livre sur les fils de France et de Hollande, et de 2 livres 13 sous 4 deniers par quintal sur le fil de Russie. Mais nos manufacturiers ne furent pas longtemps satisfaits de cette réduction. Par le statut de la vingt-neuvième année du même roi, chapitre xv (la même loi qui accorde une prime à l’exportation des toiles d’Angleterre et d’Irlande dont le prix n’excéderait pas 18 pence l’aune), on supprima même ce faible droit sur l’importation des fils écrus. Cependant, dans les différentes opérations nécessaires à la préparation du fil de tisserand, il y a beaucoup plus de travail employé que dans les opérations à faire ensuite pour mettre ce fil en œuvre de toile. Sans parler du travail de ceux qui font croître le lin et de ceux qui le sérancent, il faut au moins trois ou quatre fileuses pour tenir un tisserand cons­tamment occupé, et dans la totalité du travail nécessaire à la fabrication de la toile, les quatre cinquièmes tout au moins sont pour la préparation du fil. Mais c’est que notre filature se fait par de pauvres gens, ordinairement par des femmes qui vivent disper­sées dans les divers endroits du pays et qui n’ont ni appui ni protection. Ce n’est pas sur la vente de l’ouvrage de celles-ci, mais c’est sur la vente de l’ouvrage complet sortant des mains des tisserands que nos gros maîtres manufacturiers font leurs profits. Comme c’est leur intérêt de vendre l’ouvrage fait le plus cher qu’ils peuvent, c’est pareillement leur intérêt d’en acheter la matière première au meilleur marché possible. En surprenant à la législature des primes pour l’exportation de leurs toiles, de forts droits sur l’importation de toutes les toiles étrangères et une prohibition absolue de la consommation de quelques espèces de toiles françaises dans l’intérieur, ils ont cherché à vendre leur propres marchandises aussi cher que possible. En en­cou­rageant l’importation du fil étranger pour toiles et en le faisant venir ainsi en con­currence avec celui que filent nos ouvriers, ils cherchent à acheter au meilleur mar­ché possible l’ouvrage des pauvres qui vivent de ce métier. Ils ne sont pas moins attentifs à tenir à bas prix les salaires de leurs tisserands que ceux des pauvres fi­leuses ; et s’ils cherchent tant à hausser le prix de l’ouvrage fait ou à faire baisser celui de la matière première, ce n’est nullement pour le profit de l’ouvrier. L’indus­trie qu’encourage principalement notre système mercantile, c’est celle sur laquelle porte le bénéfice des gens riches et puissants. Celle qui alimente les profits du faible et de l’indigent est presque toujours négligée ou opprimée[3].
La prime pour l’exportation de la toile, ainsi que l’exemption de droits sur l’impor­ta­tion du fil étranger, qui n’avaient été accordées que pour quinze ans, mais qui ont été continuées par deux prolongations différentes, expirent à la fin de la session du parlement, immédiatement après le 24 juin 1786.
L’encouragement donné à l’importation des matières premières de manufactures par des Primes a été borne principalement à celles qui s’importent de nos colonies d’Amérique.
Les premières primes de ce genre furent celles accordées vers le commencement de ce siècle, sur l’importation des munitions navales d’Amérique. Sous cette dénomi­na­tion, on comprit le bois propre aux mâts, vergues et beauprés, le chanvre, la poix, le goudron et la térébenthine. Cependant, la prime de 1 livre par tonneau sur le bois de mâture, et celle de 6 livres par tonneau sur le chanvre, furent étendues à celles de ces denrées que l’on importerait d’Écosse en Angleterre. Ces deux primes restèrent sans variations sur le même pied jusqu’à leur expiration, qui arriva, pour celle sur le chanvre, le 1er janvier 1741, et pour celle sur le bois de mâture, à la fin de la session parlementaire, immédiatement après le 24 juin 1781.
Les primes à l’importation de la poix, du goudron et de la térébenthine ont subi diverses modifications pendant leur durée. Dans le principe, celle sur le goudron était de 4 liv. par tonneau, celle sur la poix était la même, et celle sur la térébenthine de 3 liv. par tonneau. La prime de 4 liv. par tonneau pour le goudron a été par la suite restreinte à celui qui serait préparé d’une certaine manière, et celle pour tout autre goudron, bon, loyal et marchand, a été réduite à 2 liv. 4 sch. par tonneau. La prime sur la poix a été aussi modérée à 1 liv., et celle sur la térébenthine à 1 liv. 10 sch. par tonneau.
La seconde prime à l’importation des matières premières de manufactures, en suivant l’ordre de date, fut celle accordée par le statut de la vingt-unième année de Georges II, chapitre xxx, sur l’importation de l’indigo des colonies anglaises. Lorsque l’indigo de nos colonies ne s’élevait qu’aux trois quarts du prix du meilleur indigo de France, il avait droit, par cet acte, à une prime de 6 den. par liv. Cette prime qui a été accordée, comme la plupart des autres, pour un temps limité seulement, fut continuée par différentes prolongations, mais elle fut réduite à 4 den. par livre. On l’a laissée expirer à la fin de la session parlementaire, terminée immédiatement après le 25 mars 1781.
La troisième prime de ce genre fut celle accordée à l’importation du chanvre ou du lin non sérancé des colonies anglaises par le statut de la quatrième année de Georges III, chapitre xxvi, dans le temps même où nous commencions tantôt à nous quereller avec nos colonies d’Amérique, tantôt à vouloir nous les attacher par des grâces. Cette prime fut accordée pour vingt et un ans, du 24 juin 1764 au 24 juin 1785. Pour les premières sept années, elle devait être de 8 liv. par tonneau ; pour les sept secondes, de 6 liv., et pour les sept dernières, de 4 liv. On ne l’étendit pas à l’Écosse, dont le climat n’est pas très-propre à cette production, quoiqu’il y croisse quelquefois du chanvre en petite quantité et de qualité inférieure. Une pareille prime à l’importation du lin d’Écosse en Angleterre aurait été un trop grand découragement pour les produits analogues du midi de la Grande-Bretagne.
La quatrième prime de ce genre fut celle accordée par le statut de la cinquième année de Georges III, chap. xlv, à l’importation du bois d’Amérique ; elle fut accordée pour neuf années, du 1er janvier 1766 au 1er janvier 1775 ; elle devait être, pendant les trois premières années, sur le pied de 1 liv. par chaque cent vingt bonnes tiges de sapin, et de 12 sch. par chaque charge de cinquante pieds cubes et des autres bois carrés. Pour les secondes trois années, elle était pour les sapins de 15 sch. et pour l’autre bois carré, de 8 sch. ; pour le troisième et dernier terme de trois ans, de 10 sch. sur les sapins et de 5 sur les autres bois carrés.
La cinquième prime de ce genre fut celle accordée par le statut de la neuvième année de Georges III, chap. xxxviii, sur l’importation des soies écrues des plan­ta­tions anglaises. Elle fut accordée pour vingt et un ans, à compter du 1er janvier 1770 jusqu’au 1er janvier 1791 ; elle fut établie, pour les premières années, sur le pied de 25 pour 100 de la valeur, de 20 pour 100 pendant les secondes sept années, et de 15 pour 100 pendant les sept dernières. L’éducation des vers et la préparation de la soie exigent tant de travail manuel, et ce travail est si cher en Amérique, qu’on ne s’atten­dait guère, m’a-t-on dit, que cette prime, tout élevée qu’elle était, dût produire aucun effet considérable.
La sixième prime de ce genre fut celle accordée par le statut de la onzième année de Georges III, chapitre l, à l’importation des fonds et douves pour pipes, muids et barils, importés des plantations anglaises. Elle fut établie pour neuf années, du 1er janvier 1772 au 1er janvier 1781. Pour les trois premières années, elle était sur le pied de 6 livres pour une quantité déterminée de ces marchandises, de 4 livres pendant les secondes trois années, et de 2 livres pendant les trois dernières.
La septième et dernière prime de ce genre fut celle accordée par le statut de la dix-neuvième année de Georges III, chap. xxxvii, à l’importation du chanvre d’Irlande ; elle a été accordée de la même manière que celle sur l’importation du chanvre et du lin non sérancé d’Amérique, pour vingt et un ans à compter du 24 juin 1779 jusqu’au 24 juin 1800[4]. Ce terme est pareillement divisé en trois périodes de sept années chacune de ces périodes le taux de la prime pour l’Irlande est le même que le taux de celle pour l’Amérique ; cependant elle ne s’étend pas, comme celle de l’Amérique, à l’importation du lin non sérancé ; elle aurait trop découragé la culture de cette plante en Angleterre. Quand cette dernière prime fut accordée, les législatures de la Grande-Bretagne et de l’Irlande n’étaient pas beaucoup mieux l’une avec l’autre que n’avaient été auparavant celles de la Grande-Bretagne et de l’Amérique. Mais il faut espérer que cette faveur accordée à l’Irlande l’aura été sous de meilleurs auspices que celles accordées à l’Amérique.
Les mêmes marchandises sur lesquelles nous avons accordé ainsi des primes à leur importation d’Amérique, ont été assujetties à des droits considérables à leur importation de tout autre pays. On regardait l’intérêt de nos colonies d’Amérique comme étant le même que celui de la métropole ; leur richesse était censée la nôtre ; tout ce que nous leur envoyions d’argent nous revenait, disait-on, par la balance du commerce, et quelques dépenses que nous fissions pour elles, nous n’en pouvions jamais devenir d’un sou plus pauvres. Les colonies étaient, à tous égards, notre propre chose ; ces dépenses étaient donc toutes faites pour bonifier une propriété qui était la nôtre, et elles tournaient à l’emploi et au profit de gens qui ne faisaient avec nous qu’une même nation. je pense qu’il n’est pas besoin d’en dire davantage à présent pour montrer toute l’absurdité d’un système qu’une funeste expérience n’a que trop fait juger. Si réellement les colonies américaines avaient été une partie de la Grande-Bretagne, ces primes auraient pu être regardées comme des encouragements à la production, et elles auraient été sujettes à toutes les objections qui s’élèvent contre ces sortes de primes, mais à ces objections-là seulement.
L’exportation des matières premières de manufactures est découragée tantôt par des prohibitions absolues, tantôt par des droits élevés.
Nos manufacturiers en lainages ont mieux réussi qu’aucune autre classe d’indus­triels à persuader à la législature que la prospérité de la nation dépendait du succès et de l’étendue de leur branche particulière d’industrie. non-seulement ils ont obtenu un monopole contre les consommateurs par une prohibition absolue d’importer des étof­fes de laine, de quelque pays étranger que ce soit, mais ils se sont fait donner encore un autre monopole contre les fermiers qui élèvent des moutons et contre les producteurs de laine par une semblable prohibition sur l’exportation du bétail vivant et sur celle de la laine. On s’est souvent plaint avec justice de la rigueur des peines portées pour assurer le revenu de l’État, comme établissant des châtiments sévères pour des actions que l’on avait toujours regardées comme innocentes avant les statuts qui les ont déclarées criminelles. Mais je puis l’affirmer hardiment : les plus cruelles de nos lois fiscales sont douces et modérées en comparaison de quelques-unes de celles que les clameurs de nos marchands et de nos manufacturiers ont arrachées à la législature pour le soutien de leurs injustes et absurdes monopoles. On peut dire de ces lois ce que l’on a dit de celles de Dracon, qu’elles ont toutes été écrites avec du sang.
Par le statut de la huitième année d’Élisabeth, chapitre III, quiconque exporte des brebis, agneaux ou béliers, doit pour la première fois avoir tous ses biens confisqués à perpétuité, subir un emprisonnement d’un an, et au bout de ce temps avoir la main gauche coupée, à un jour de marché, dans une ville où elle restera clouée ; en cas de récidive, il est jugé coupable de félonie[5], et en conséquence puni de mort. Il semble que l’objet de cette loi a été d’empêcher que la race de nos brebis ne se propageât dans les pays étrangers. Par des actes des treizième et quatorzième années de Charles II, l’exportation de la laine fut réputée crime de félonie, et le délinquant sujet aux peines et confiscations attachées à ce crime.
Il faut supposer, pour l’honneur de la nation, que ni l’un ni l’autre de ces statuts n’a jamais été mis à exécution. Cependant le premier, autant que je sache, n’a jamais été expressément révoqué, et le jurisconsulte Hawkins paraît le regarder comme étant en vigueur. Mais il est censé peut-être révoqué indirectement par le statut de la douziè­me année de Charles II, chap. xxxii, section 3, qui, sans abolir formellement les peines portées par les anciens statuts, établit une nouvelle peine, savoir : celle d’une amende de 20 schellings pour chaque brebis exportée ou qu’on aurait essayé d’expor­ter et, en outre, la confiscation tant des brebis que de tout ce que le propriétaire peut posséder dans le vaisseau. Le second a été expressément révoqué par les actes des septième et huitième années de Guillaume III, chap. xxviii, section 4, ainsi conçus : « Attendu que les statuts des treizième et quatorzième années du roi Charles Il contre l’exportation de la laine, entre autres dispositions y mentionnées, portent que cette exportation sera réputée félonie, la rigueur de laquelle peine a empêché de faire des poursuites efficaces contre les prévenus de ces délits, il est définitivement statué par ces présentes que ledit acte, en ce qui concerne la peine de félonie, contre lesdits délits, demeure, à cet égard seulement, nul et révoqué. »
Mais les peines établies par ce statut moins rigoureux, ou bien celles portées par d’anciens statuts qu’il ne révoque point, sont encore bien assez rigoureuses. Outre la confiscation des marchandises, le délinquant encourt une amende de 3 schellings par chaque livre pesant de laine exportée ou qu’il aurait essayé d’exporter, ce qui en est environ quatre ou cinq fois la valeur. Tout marchand ou autre personne convaincue de ce délit est déchue du droit de répéter aucune dette ou compte de ses facteurs ou de qui que ce soit. Quelle que puisse être la fortune du délinquant, qu’il soit ou non en état de supporter d’aussi fortes amendes, l’intention de la loi est de le ruiner com­plètement. Mais comme la morale du peuple n’est pas encore aussi corrompue que celle des auteurs d’un pareil statut, je n’ai jamais entendu dire qu’aucun débiteur se soit prévalu de cette clause. Si la personne convaincue du délit n’est pas en état de satisfaire à ces peines dans les trois mois du jugement, elle est déportée pour sept ans, et si elle revient avant l’expiration de ce terme, elle est dans le cas des peines de la félonie, sans bénéfice de clergie[6]. Le propriétaire du vaisseau, s’il a eu connaissance du délit, est puni par la confiscation de son intérêt dans le bâtiment et les apparaux. Le maître de l’équipage et les matelots qui ont participé à la contravention encourent la confiscation de tous leurs biens meubles et trois mois de prison. Par un statut subséquent, la prison du maître est portée à six mois.
Dans la vue d’empêcher l’exportation, tout le commerce intérieur de la laine est soumis aux gênes les plus dures et les plus oppressives. On ne peut l’emballer en boîte, baril, caisse, coffre ou autre chose quelconque, mais seulement la renfermer sous une enveloppe de cuir ou de toile d’emballage, sur laquelle il faut marquer en dehors les mots laine ou fil de laine, en grosses lettres, d’au moins trois pouces de long, sous peine de confiscation de la marchandise et de l’enveloppe, et d’une amende, contre le propriétaire ou l’emballeur, de 3 schellings par chaque livre pesant. On ne peut la charger sur un cheval ou sur un chariot, ni la transporter par terre plus près que cinq milles des côtes, si ce n’est entre le soleil levant et le soleil couchant, à peine de confiscation de la marchandise, des chevaux et de la voiture. La centénerie[7] voisine, joignant les côtes, hors de laquelle ou à travers laquelle la laine a été voiturée ou exportée, doit une amende de 20 livres, si la laine en contravention est d’une valeur moindre de 10 livres ; et si la valeur est plus forte, l’amende est alors du triple de cette valeur, outre le triple des frais, le tout payable dans l’année. L’exécution se fait contre deux des habitants, que les sessions sont tenues de faire rembourser par une taxe assise sur tous les autres habitants, comme dans le cas de vol. Si quelqu’un s’avise de composer avec la centénerie pour une peine moindre que celle-ci, il est puni de cinq ans de prison, et toute autre personne est autorisée à faire la poursuite. Ces règlements sont en vigueur dans toute l’étendue du royaume.
Mais dans les comtés de Kent et de Sussex en particulier, les gênes sont encore plus incommodes. Tout propriétaire de bêtes à laines, dans les dix milles des côtes de la mer, doit fournir à l’officier de la douane le plus voisin, trois jours après la tonte, un état par écrit du nombre de ses toisons et du lieu où elles sont placées ; et avant d’en déplacer la moindre partie, il faut qu’il donne une pareille déclaration du nombre et du poids des toisons qu’il veut faire enlever, du nom et demeure de la personne à qui elles sont vendues, et du heu auquel il entend les faire transporter. Personne à distance de quinze milles de la mer, dans ces comtés, ne peut acheter de laine, sans préalablement s’obliger envers le roi qu’aucune partie de cette laine ne sera vendue par lui à autre personne demeurant dans les quinze milles du voisinage de la mer. Si l’on trouve dans ces comtés quelqu’un transportant de la laine du côté de la mer sans avoir contracté l’engagement dont je viens de parler et sans avoir donné caution, la laine est confisquée et le délinquant mis de plus à l’amende de 3 schellings par livre pesant. Si quelqu’un tient de la laine en dedans de cette distance de quinze milles de la mer sans avoir rempli les formalités ci-dessus, elle est saisie et confisquée, et si quelqu’un vient à la réclamer après la saisie, il faut qu’il donne caution à l’Échiquier pour le payement du triple des frais, outre les autres peines en cas qu’il succombât au procès.
Quand le commerce intérieur est assujetti à de pareilles entraves, on doit bien présumer qu’il n’a pas été laissé une grande liberté au commerce de côte en côte. Tout propriétaire qui transporte ou fait transporter de la laine à quelque port ou endroit de la côte, pour être de là transportée par mer à un autre port ou endroit de la côte, doit d’abord en faire faire la déclaration au port d’où il entend en faire faire le départ, avec désignation des poids, marques et nombre des ballots, avant qu’elle puisse être portée dans les cinq milles du voisinage de ce même port, sous peine de confiscation de la laine, ainsi que des chevaux, chariots et autres voitures, et encore sous toutes les pei­nes et amendes portées par les autres lois subsistantes contre l’exportation de la laine. Cette loi cependant (première année de Guillaume III, chap. xxxii) a l’extrême indulgence de déclarer : « Que cette disposition n’empêche pas que toute personne puisse transporter sa laine chez soi du lieu où se fait la tonte, quand même ce serait en dedans des cinq milles du bord de la mer, pourvu que dans les dix jours après la tonte, et avant de déplacer sa laine, elle fournisse au plus proche officier des douanes une déclaration, certifiée et signée d’elle, du véritable nombre des toisons, et du local où elles sont déposées, et pourvu encore qu’elle ne déplace pas ladite laine sans donner au même officier une déclaration aussi signée, portant qu’elle a intention de faire ce déplacement, laquelle déclaration sera donnée trois jours d’avance. » Il faut donner caution que la laine à transporter par mer le long des côtes sera débarquée au port particulier pour lequel elle a été déclarée, et si l’on en débarque la moindre partie hors la présence d’un officier, non-seulement il y a peine de confiscation de la laine, comme pour toutes les autres marchandises, mais il y a encore la peine ordinaire de l’amende additionnelle de 3 schellings par livre pesant de laine[8].
Lorsque nos fabricants en laine ont sollicité des règlements aussi extraordinaires et aussi oppressifs, pour justifier leurs démarches ils ont soutenu avec assurance que la laine d’Angleterre était d’une qualité particulière, supérieure à celle de tous les autres pays ; que la laine des autres pays ne pourrait même être travaillée de manière à faire aucun ouvrage passable, sans quelque mélange de celle-là ; que sans cette laine on ne saurait fabriquer de drap fin ; que par conséquent, si l’on parvenait à en empêcher totalement l’exportation, l’Angleterre s’assurerait le monopole de presque tout le commerce de draps du monde entier, et qu’ainsi n’ayant point de rivaux, et vendant dès lors au prix qu’elle voudrait, elle arriverait en peu de temps à un degré incroyable d’opulence, au moyen de la balance du commerce la plus avantageuse possible. Cette doctrine, comme beaucoup d’autres qui sont soutenues par un grand nombre de gens avec le ton de la plus haute confiance, fut crue sur parole, et l’est encore aujourd’hui par un bien plus grand nombre, par presque tous ceux qui ne sont pas très-au fait du commerce de lainerie, ou qui n’ont pas fait là-dessus de recherches particulières. Il est néanmoins si complètement faux que la laine d’Angleterre soit nécessaire, à un degré quelconque, à la fabrication des draps fins, que même elle est tout à fait impropre à ce genre d’ouvrage. Les draps fins sont faits en entier de laine d’Espagne. La laine d’Angleterre ne peut même être mélangée avec la laine d’Espa­gne, pour entrer dans la composition de ces sortes de draps, sans en dégrader et en altérer, à un certain point, la belle qualité.
Dans une précédente partie de cet ouvrage, on a démontré que l’effet de ces rè­gle­ments avait été de rabaisser le prix de la laine d’Angleterre, non-seulement au-dessous de ce qu’il serait naturellement dans le moment actuel, mais même beaucoup au-dessous de ce qu’il était au temps d’Edouard III. On dit que lorsque la laine d’Écosse, en conséquence de l’union des deux royaumes, vint à être assujettie à ces règlements, son prix baissa environ de moitié. L’auteur très-exact et très-intelligent des Mémoires sur les laines, M. John Smith, observe que le prix de la meilleure laine anglaise, en Angleterre, est généralement au-dessous de ce que la laine d’une qualité très-infé­rieure se vend communément au marché d’Amsterdam. Le but avoué de ces règle­ments était de rabaisser le prix de cette denrée au-dessous de ce qu’on peut appeler son prix naturel et son juste prix, et il paraît qu’il n’y a pas à douter qu’ils n’aient produit l’effet qu’on s’en promettait.
On pourrait croire peut-être que cet avilissement du prix de la laine, décourageant la production de cette denrée, a dû diminuer de beaucoup son produit annuel, et en réduire la quantité, sinon au-dessous de ce qu’elle était anciennement, au moins au-dessous de ce qu’elle serait vraisemblablement, dans l’état actuel des choses, si, par un marché libre et absolument ouvert, on eût laissé la denrée s’élever à son prix naturel, et à ce qu’on peut nommer réellement son juste prix. je suis cependant porté à croire que si la quantité du produit annuel a pu se ressentir quelque peu de ces règlements, elle ne s’en est pas trouvée beaucoup diminuée. La production de la laine n’est pas le principal objet que se propose, dans l’emploi de son industrie et de son capital, le fermier qui élève des moutons. Il n’attend pas tant son profit du prix de la toison que de celui du corps de la bête, et le prix moyen ou ordinaire de ce dernier article doit même le plus souvent lui bonifier tout le déficit qu’il peut y avoir dans le prix moyen ou ordinaire de la toison. Dans la première partie de cet ouvrage, on a observé que « tous règlements, quels qu’ils soient, qui tendent à abaisser le prix, soit de la laine, soit de la peau crue, au-dessous de ce qu’il serait naturellement, doivent nécessairement, dans un pays cultivé et amélioré, avoir quelque tendance à faire hausser le prix de la viande de boucherie. Il faut que le prix du bétail qu’on nourrit sur une terre améliorée et cultivée, soit gros, soit menu bétail, suffise à payer la rente et le profit que le propriétaire et le fermier sont en droit d’attendre d’une terre améliorée et cultivée ; sans cela, ceux-ci cesseraient bientôt d’en nourrir. Ainsi, toute partie de ce prix qui ne se trouve pas payée par la laine et la peau, il faut que le corps la paye. Moins on paye pour l’un de ces articles, plus il faut payer pour l’autre. Pourvu que le propriétaire et le fermier trouvent tout leur prix, il leur importe peu comment il est réparti sur les différentes parties de la bête. Ainsi, comme propriétaires et comme fermiers, dans tout pays cultivé et amélioré, ils ne peuvent guère être lésés par de tels règlements, quoiqu’ils puissent en souffrir, comme consommateurs, par la hausse du prix des vivres ». Si ce raisonnement est juste, l’avilissement du prix de la laine n’est donc pas dans le cas d’occasionner, dans un pays cultivé et amélioré, une diminution de quantité dans le produit annuel de la denrée, à moins seulement qu’en faisant hausser le prix de la chair du mouton, il ne puisse en diminuer la demande et, par conséquent, la production de cette sorte de viande. Cependant son effet, même sous ce rapport, ne saurait être, selon toute apparence, bien considérable. 
Mais si l’effet de cet avilissement du prix peut bien n’avoir pas été très-consi­dérable sur la quantité du produit annuel, son effet sur la qualité, pourrait-on croire, a dû être nécessairement fort important. Il serait naturel de présumer qu’à mesure de l’avilissement du prix, la qualité de laine anglaise a dû, dans la même proportion à peu près, se dégrader, sinon au-dessous de ce qu’elle était dans l’ancien temps, au moins au-dessous de ce qu’elle eût été naturellement dans l’état actuel de la culture et de l’amélioration. Comme la qualité dépend de la nourriture des brebis, de la nature de leur pâturage et de la manière plus ou moins propre et soignée dont elles sont te­nues pendant tout le temps que la toison met à pousser et à croître, il est assez naturel de penser que l’attention qu’on apportera à toutes ces choses sera toujours en propor­tion de la récompense qu’on pourra espérer de la toison, pour se payer du travail et de la dépense que cette attention aura exiges. Néanmoins, il arrive que la bonté de la toison dépend en grande partie de la santé, de la taille et de la force de l’animal ; la même attention qu’il faut apporter pour bonifier le corps de la brebis est, à un certain point, suffisante pour bonifier la toison. Malgré la baisse que le prix a souffert, on assure que la laine anglaise a considérablement gagné en qualité, même pendant le cours de ce siècle. Peut-être aurait-elle encore plus gagné à cet égard, si le prix eût été meilleur ; mais si l’avilissement du prix a pu nuire à cette bonification, il ne l’a certainement pas empêché totalement.
Les mesures violentes établies par ces règlements n’ont donc pas, à ce qu’il semble, autant influé qu’on aurait pu s’y attendre, tant sur la quantité du produit annuel de la laine, que sur sa qualité, quoique je sois d’avis qu’elles ont dû vraisem­bla­­blement influer beaucoup plus sur la dernière que sur l’autre ; et si, au total, l’intérêt des producteurs de cette denrée a dû en souffrir, c’est toujours de beaucoup moins qu’on aurait pu le penser.
Toutefois, ces considérations ne sauraient justifier la prohibition absolue de l’ex­por­tation de la laine, mais elles suffiront pour justifier pleinement l’imposition d’une forte taxe sur cette exportation.
Blesser les intérêts d’une classe de citoyens, quelque légèrement que ce puisse être, sans autre objet que de favoriser ceux de quelque autre classe, c’est une chose évidemment contraire à cette justice, à cette égalité de protection que le souverain doit indistinctement à ses sujets de toutes les classes. Or, certainement la prohibition nuit jusqu’à un certain point à l’intérêt des producteurs, uniquement pour favoriser celui des manufacturiers.
Toute classe de citoyens est obligée de contribuer aux dépenses du souverain ou de la république. Une taxe de 5 ou même de 10 schellings sur l’exportation de chaque balle de vingt-huit livres de laine produirait au souverain un revenu fort considérable. Elle nuirait un peu moins à l’intérêt des producteurs que ne le fait la prohibition, parce que vraisemblablement elle ne ferait pas baisser tout à fait d’autant le prix de la laine ; elle donnerait au manufacturier un avantage suffisant, parce qu’en supposant qu’il ne pût pas acheter sa laine précisément à aussi bon marché qu’il le fait avec la prohibi­tion, au moins il l’achèterait toujours 5 ou 10 schellings à meilleur marché que ne pourrait l’acheter tout autre manufacturier étranger, sans compter encore l’épargne du fret et de l’assurance que celui-ci serait obligé de payer. Il n’est guère possible d’ima­giner d’impôt capable de produire un revenu considérable au souverain, et qui en même temps entraîne aussi peu d’inconvénients pour personne.
La prohibition, en dépit de toutes les peines dont on l’a entourée, n’empêche pas encore l’exportation de la laine. On sait parfaitement qu’il s’en exporte une quantité considérable ; la grande différence entre le prix qu’elle a sur notre marché et celui qu’elle a sur les marchés étrangers, offre un tel appât à la contrebande, que toutes les rigueurs de la loi ne peuvent la contenir. Cette exportation illégale n’est avantageuse à personne qu’au contrebandier. Une exportation légale, soumise à un impôt, tournerait à l’avantage de tous les sujets de l’État, en fournissant un revenu au souverain et en épargnant par là l’établissement de quelques autres impôts peut-être plus onéreux et plus incommodes.
L’exportation de cette terre qu’on nomme terre à foulon, et qu’on croit nécessaire pour préparer et dégraisser des ouvrages de lainerie, a été prohibée à peu près sous les mêmes peines que l’exportation de la laine. La terre a pipe même, quoique reconnue pour être différente de la terre grasse employée par les foulons a été, à cause de la ressemblance, comprise dans la prohibition et sous les mêmes peines, de peur que la terre à foulon ne fût quelquefois exportée comme terre à pipe[9]. 
Par le statut des treizième et quatorzième années de Charles II, chap. vii, on prohiba l’exportation non-seulement des peaux crues, mais encore du cuir tanné, à moins qu’il ne fût sous forme de bottes, souliers ou pantoufles ; et la loi donna ainsi à nos bottiers et à nos cordonniers un monopole, non-seulement contre nos nourrisseurs de bestiaux, mais encore contre nos tanneurs. Par des statuts subséquents, nos tan­neurs sont venus à bout de se faire affranchir de ce monopole moyennant un léger droit de 1 schelling seulement sur le quintal de cuir tanné, poids de 112 livres. Ils ont pareillement obtenu la restitution des deux tiers des droits d’accise établis sur leur marchandise, lorsqu’elle est exportée, même sans avoir subi d’autre nouvelle main-d’œuvre. Quant aux ouvrages de manufacture en cuir, ils peuvent s’exporter francs de tous droits, et celui qui exporte obtient, en outre, la restitution de la totalité des droits d’accise[10]. Mais nos nourrisseurs de bestiaux sont toujours restés sous le joug de ce monopole. Ces nourrisseurs, qui vivent séparés l’un de l’autre et dispersés dans les différentes parties du pays, ne peuvent pas, sans de grandes difficultés, se concerter entre eux dans la vue, ou d’obtenir des monopoles contre leurs concitoyens, ou de s’affranchir de ceux que d’autres ont pu obtenir contre eux ; mais c’est ce que peuvent faire aisément les manufacturiers, qui se trouvent rassemblés en nombreuses corpo­rations dans toutes les grandes villes. Il n’y a pas jusqu’aux cornes de bestiaux dont l’exportation ne soit prohibée[11], et les deux chétives professions de tourneur en corne et de faiseur de peignes jouissent à cet égard d’un monopole contre les nourrisseurs de bestiaux.
Les entraves mises par des prohibitions ou par des impôts à l’exportation des marchandises qui ne sont travaillées encore qu’en partie et non complètement manu­fac­turées, ne sont pas une chose particulière aux ouvrages de manufacture en cuir. Tant qu’il reste quelque main-d’œuvre à donner pour rendre une marchandise propre à servir immédiatement à l’usage ou à la consommation, nos manufacturiers soutien­nent que c’est à eux qu’appartient le droit de le faire. L’exportation de la laine filée et de l’estame est prohibée comme celle de la laine et sous les mêmes peines[12]. Les draps en blanc même sont assujettis à un droit à l’exportation, et nos teinturiers ont, à cet égard, obtenu un monopole contre nos drapiers. Vraisemblablement nos drapiers auraient bien été en état de s’en défendre, mais il se trouve que la plupart de nos principaux drapiers sont eux-mêmes aussi teinturiers. On a prohibé l’exportation des boîtes à montres, des boîtes à pendules, et des cadrans de montres et de pendules. Nos horlogers ont eu peur, à ce qu’il semble, que ces objets ouvrés ne vinssent à renchérir pour eux par la concurrence des acheteurs étrangers[13].
Par d’anciens statuts d’Édouard III, de Henri VII et d’Édouard VI, l’exportation de tous métaux fut prohibée. On n’excepta que le plomb et l’étain, vraisemblablement à cause de la grande abondance de ces métaux, dont l’exportation constituait alors une par­tie considérable du commerce du royaume. Pour encourager le commerce d’ex­ploi­tation des mines, le statut de la cinquième année de Guillaume et Marie, chap. xvii, exempta de cette prohibition le fer, le cuivre et le mundick[14], extrait de minerai an­glais. L’exportation de toutes sortes de cuivres rouges en barres, étrangers aussi bien qu’anglais, fut ensuite permise par le statut des neuvième et dixième années de Guillaume III, chapitre xxvi. L’exportation du cuivre jaune non manufacturé, de ce qu’on appelle métal de canon, métal de cloche et métal de batterie de cuisine, reste encore sous la prohibition[15]. Les ouvrages de manufacture en cuivre de toute espèce peuvent s’exporter francs de droits.
L’exportation des matières premières de manufacture, qui ne sont pas sous une prohibition absolue, est assujettie le plus souvent à des droits considérables.
Par le statut de la huitième année de Georges 1er, chap. xv, on affranchit de tous droits l’exportation de toutes les marchandises du cru ou des fabriques de la Grande-Bretagne, sur lesquelles il avait été établi quelques droits par les anciens statuts. Ce­pen­dant on excepta les marchandises suivantes : l’alun, le plomb, la mine de plomb, l’étain, le cuir tanné, la couperose, les charbons, les cardes à carder la laine, les étof­fes de laines en blanc, la calamine, les peaux de toute espèce, la colle-forte, le poil ou laine de lapin, le poil de lièvre, les crins de toute espèce, les chevaux et la litharge de plomb. À l’exception des chevaux, toutes ces marchandises sont, ou des matières premières de manufacture, ou des ouvrages incomplets de main-d’œuvre qu’on peut considérer comme matière première pour d’autres manufactures, ou enfin ce sont des instruments d’industrie. Ce statut les laisse assujetties à tous les anciens droits qui peuvent avoir été établis sur elles par l’ancien subside[16] et 1 pour 100 en sus[17].
Par le même statut, un grand nombre de drogues étrangères propres à la teinture sont exemptées de tous droits à l’importation. Chacune d’elles cependant est ensuite assujettie à un certain droit, très-peu lourd à la vérité, à l’exportation. Il paraît que nos teinturiers, tout en trouvant leur intérêt à encourager l’importation de ces drogues par une exemption de tous droits, ont aussi imaginé qu’il était pareillement de leur intérêt d’en décourager l’exportation par quelque petit droit. Pourtant, il est extrêmement probable que la cupidité qui a suggéré ce beau trait d’habileté mercantile a manqué son but. Elle a averti nécessairement ceux qui importent de mettre plus d’attention qu’ils n’en auraient peut-être mis sans cela, à ce que leur importation n’excédât point ce qui était nécessaire pour les besoins du marché intérieur. Vraisemblablement, ce marché a dû en être par là moins bien approvisionné en tout temps ; ces marchandises ont dû vraisemblablement y être en tout temps un peu plus chères qu’elles ne l’eussent été si l’exportation eût été rendue aussi libre que l’importation.
Par ce dernier statut, la gomme du Sénégal, ou gomme arabique, étant comprise dans la liste des drogues pour la teinture, pouvait être importée franche de droits. Ces gommes, à la vérité, étaient assujetties à un léger droit de pondage montant à 3 deniers par quintal sur leur réexportation. La France jouissait à cette époque d’un commerce exclusif dans le pays le plus productif de ces sortes de drogues, celui qui est dans le voisinage du Sénégal, et l’on ne pouvait pas aisément fournir le marché de la Grande-Bretagne par une importation immédiate du lieu où croissent ces gommes. En conséquence, par le statut de la vingt-cinquième année de Georges II, contre les dispositions générales de l’acte de navigation, on permit l’importation, de tous les endroits de l’Europe, de la gomme du Sénégal. Cependant, comme la loi ne cherchait pas à encourager ce genre de commerce si contraire aux principes généraux de la poli­tique commerciale de l’Angleterre, elle imposa un droit de 10 schellings par quintal sur cette importation, et aucune partie de ce droit n’était restituable lors de l’exportation. Les succès de la guerre commencée en 1755 donnèrent à la Grande-Bretagne, dans ces pays, le même commerce exclusif que celui dont la France avait joui auparavant. Dès que la paix fut faite, nos manufacturiers tâchèrent de tourner cet avantage à leur profit personnel et de s’assurer un monopole, tant contre les produc­teurs de cette denrée que contre ceux qui l’importent. En conséquence, par le statut de la cinquième année de Georges III, chap. xxxvii, l’exportation de la gomme du Sénégal, des pays de la domination de Sa Majesté en Afrique, fut bornée à la Grande-Bretagne et soumise à toutes les mêmes restrictions, règlements, peines et confisca­tions que celles des marchandises énumérées des colonies anglaises d’Amérique et des Indes occidentales. À la vérité, l’importation de cette drogue fut assujettie à un léger droit de 6 deniers par quintal ; mais sa réexportation fut chargé d’un droit énor­me de 1 livre 10 schellings par quintal[18]. L’intention de nos manufacturiers était que tout le produit de ces pays pût être importé en Angleterre, et dans la vue de se mettre à même de l’acheter au prix qui leur conviendrait, ils voulurent qu’on n’en pût réex­porter la moindre partie, sinon avec des frais capables de décourager cette expor­tation. Mais, dans cette occupation comme en beaucoup d’autres, leur avidité a manqué son but. Ce droit énorme offrit un tel appât à la contrebande, qu’il y eut de grandes quantités de cette denrée exportées en fraude, vraisemblablement dans tous les pays manufacturiers de l’Europe, mais en particulier en Hollande, et non-seule­ment de la Grande-Bretagne, mais même de l’Afrique. En conséquence, le statut de la quatorzième année de Georges III, chap. x, modéra ce droit sur l’exportation à 5 schellings par quintal.
Dans le Livre des tarifs, selon l’évaluation duquel se percevait l’ancien subside, les peaux de castor étaient évaluées à 6 schellings 8 deniers la pièce, et les différents subsides et impôts qui, avant 1722, avaient été établis sur leur importation, s’élevaient au cinquième de l’évaluation du tarif ou à 16 deniers sur chaque peau ; tous ces droits étaient rendus en cas d’exportation, excepté moitié de l’ancien subside, laquelle montait seulement à 2 deniers. Ce droit sur l’importation d’une matière première de manufacture aussi importante a été jugé trop fort, et en 1722 l’évaluation du tarif fut modérée à 2 schellings 6 deniers, et de celui-ci moitié seulement fut restituable lors de l’exportation. Les mêmes succès militaires mirent sous la domination de la Grande-Bretagne le pays le plus productif en castors, et les peaux de castor étant com­prises dans les marchandises énumérées, leur exportation de l’Amérique fut, par con­sé­quent, bornée au marché de la Grande-Bretagne. Nos manufacturiers ne tardè­rent pas à s’apercevoir de l’avantage qu’ils pouvaient tirer de cette circonstance, et en 1764 le droit sur l’importation des peaux de castor fut réduit à un denier, mais le droit sur l’exportation fut porté à 7 deniers par chaque peau, sans aucune restitution du droit payé à l’importation. Par la même loi, on établit un droit de 18 deniers par livre sur l’exportation du poil de castor, sans rien changer au droit sur l’importation de cette marchandise, fixé alors sur le pied d’environ 4 à 5 deniers par livre, quand l’impor­tation était faite par des sujets et par des bâtiments de la Grande-Bretagne[19].
Les charbons de terre peuvent être regardés comme matière première de manu­facture et comme instrument d’industrie ; aussi a-t-on chargé leur exportation de droits très-forts, s’élevant actuellement (1783) à plus de 5 schellings le tonneau, ou à plus de 15 schellings le chaldron[20], mesure de New-Castle ; ce qui, le plus souvent, est plus que la valeur primitive de la denrée à la fosse à charbon, ou même au port de mer où se fait l’exportation[21].
Toutefois, l’exportation des instruments d’industrie proprement dits est ordinai­rement empêchée, non par des droits élevés, mais par des prohibitions absolues. Ainsi, par le statut des septième et huitième années de Guillaume III, chap. xx, sect. 8, l’exportation des métiers ou machines à faire des bas ou des gants est prohibée, non-seulement sous peine de confiscation des métiers ou machines ainsi exportées ou qu’on a essayé d’exporter, mais encore sous peine d’une amende de 40 livres, dont la moitié pour le roi, et l’autre pour celui qui dénoncera ou fera la poursuite du délit. De même, par le statut de la quatorzième année de Georges III, chap. lxxi, l’exportation aux pays étrangers de tous ustensiles applicables à la fabrication des ouvrages en coton, en toile, en laine ou en soie est prohibée, non-seulement sous peine de confis­cation de ces ustensiles, mais encore sous peine d’amende de 200 livres contre l’auteur du délit, et de pareille amende de 200 livres contre le maître du vaisseau qui, en connaissance de cause, aura laissé charger ces outils sur son bord[22].
Lorsqu’on voit des peines aussi rigoureuses portées contre l’exportation des instru­ments inanimés, on peut bien s’attendre que l’instrument vivant, l’ouvrier, ne conser­vera pas la liberté de s’en aller. Aussi, par le statut de la cinquième année de Georges Ier, chap. xxvii, toute personne convaincue d’avoir engagé un homme de métier ou ouvrier de manufactures de la Grande-Bretagne à aller pratiquer son métier ou l’ensei­gner dans quelque pays étranger, est, pour la première fois, sujette à une amende qui ne pourra excéder 100 livres et à trois mois de prison, prolongés jusqu’au payement de l’amende, et pour la seconde fois, à une amende laissée à la discrétion des juges, et à une prison de douze mois, qui sera encore prolongée jusqu’après le payement de l’amende. Par le statut de la vingt-troisième année de Georges II, chap. xiii, cette pei­ne est augmentée et portée pour la première fois à 500 livres par chaque ouvrier qu’on aura ainsi embauché, et à douze mois de prison, prolongés jusqu’à ce que l’amende soit acquittée, et pour la seconde fois, à 1,000 livres d’amende, et deux ans de prison, prolongés aussi jusqu’après le payement de l’amende.
Par le premier de ces statuts, s’il y a preuve que quelqu’un ait tenté de débaucher ainsi un ouvrier, ou qu’un ouvrier ait contracté l’engagement ou seulement promis de passer en un pays étranger pour l’objet ci-dessus expliqué, cet ouvrier peut être obligé de donner caution, à la discrétion de la cour, qu’il ne passera pas la mer, et il peut être détenu en prison jusqu’à ce qu’il ait fourni cette caution[23].
Si un ouvrier a passé la mer et exerce ou enseigne son métier dans quelque pays étranger, et que, sur l’avertissement qui lui est donné par quelqu’un des ministres de Sa Majesté ou consuls à l’étranger, ou par un des secrétaires d’État alors en fonction, il ne rentre pas dans le royaume dans les six mois de l’avertissement reçu, pour s’y fixer à l’avenir et y résider continuellement, il est dès lors incapable de recevoir aucun legs à lui fait dans le royaume, d’être administrateur ou exécuteur testamentaire, et de pouvoir acquérir aucune terre par succession, donation ou achat. Tous ses biens, meubles et immeubles, sont aussi confisqués au profit de la couronne ; il est réputé étranger à tous égards et mis hors de la protection du roi.
Je pense qu’il n’est pas besoin de faire observer combien de tels règlements sont contraires à cette liberté civile si vantée, et dont nous nous montrons si jaloux, liberté qu’on sacrifie ouvertement dans ce cas au misérable intérêt de nos marchands et de nos manufacturiers[24].
Le motif si louable qui a dicté tous ces règlements, c’est d’étendre le progrès de nos manufactures, non pas en les perfectionnant en elles-mêmes, mais en affaiblissant celles de tous nos voisins, et en anéantissant autant que possible la concurrence fâcheuse de rivaux si odieux et si incommodes. Nos maîtres manufacturiers trouvent qu’il est juste de leur accorder ainsi le monopole du travail et de l’industrie de tous leurs concitoyens. Si en bornant, dans certains métiers, le nombre d’apprentis qu’on peut y tenir à la fois, et en établissant dans tous la nécessité d’un long apprentissage, ils cherchent tous de leur côté à resserrer dans le plus petit nombre d’individus possi­ble les connaissances nécessaires à leurs métiers respectifs, ils ne veulent pas pourtant que la moindre partie de ce petit nombre puisse aller au-dehors instruire les étrangers.
La consommation est l’unique but, l’unique terme de toute production, et l’on ne devrait jamais s’occuper de l’intérêt du producteur, qu’autant qu’il le faut seulement pour favoriser l’intérêt du consommateur. Cette maxime est si évidente par elle-même, qu’il y aurait de l’absurdité à vouloir la démontrer. Mais, dans le système que je combats, l’intérêt du consommateur est à peu près constamment sacrifié à celui du producteur, et ce système semble envisager la production et non la consommation, comme le seul but, comme le dernier terme de toute industrie et de tout commerce.
Dans les entraves mises à l’importation de toutes marchandises étrangères qui pourraient venir en concurrence avec celles de notre sol ou de nos manufactures, on a évidemment sacrifié l’intérêt du consommateur national à celui du producteur. C’est uniquement pour le bénéfice de ce dernier, que l’autre est obligé de payer le renché­rissement qu’un tel monopole ne manque presque jamais d’occasionner dans le prix des marchandises.
C’est uniquement pour le bénéfice du producteur qu’on a accordé des primes à l’exportation de quelques-unes de nos productions. Il faut que le consommateur natio­nal paye premièrement l’impôt qui sert à acquitter la dépense publique de la prime, et secondement l’impôt, encore bien plus fort, résultant nécessairement du renchéris­sement de la denrée sur le marché intérieur.
Au moyen du fameux traité de commerce avec le Portugal, le consommateur est détourné, par des droits énormes, d’acheter d’un pays voisin une denrée que notre climat ne peut produire, mais qu’il se trouve forcé d’acheter d’un pays éloigné, quoi­qu’il soit bien reconnu que la denrée du pays éloigné est de moins bonne qualité que celle du pays voisin. Le consommateur national est obligé de se soumettre à cet inconvénient, uniquement pour que le producteur ait la faculté d’importer quelques-unes de ses productions dans ce pays éloigné à des conditions plus avantageuses qu’il n’eût pu l’espérer sans cela. Il faut de plus que le consommateur paye en entier le renchérissement que le prix de ces mêmes productions pourra éprouver sur le marché national au moyen de cette exportation forcée.
Mais c’est dans le système de lois adopté pour le régime de nos colonies d’Amé­rique et des Indes occidentales, qu’on voit l’intérêt du consommateur national sacrifié à celui du producteur, à un excès porté encore bien plus loin que dans tous nos autres règlements de commerce. On a fondé un grand empire dans la seule vue de former à nos différents producteurs une nation de chalands, une nation qui fût forcée de venir acheter à leurs différentes boutiques toutes les marchandises qu’ils pourraient lui fournir. Pour ce petit surhaussement de prix qu’un tel monopole devait procurer à nos producteurs, les consommateurs nationaux se sont trouvés chargés de toute la dépen­se qu’entraînent l’entretien et la défense de cet empire. C’est dans cette vue, et dans cette seule vue, que les deux dernières guerres ont englouti plus de 200 millions, et qu’on a contracté une nouvelle dette de plus de 170 millions, outre tout ce qui a été dépensé pour le même objet dans les guerres précédentes. L’intérêt seul de cette dette excède, non-seulement tout le profit extraordinaire qu’on pourrait jamais supposer provenir du monopole du commerce des colonies, mais encore toute la valeur de ce commerce, ou la valeur totale, année commune, des marchandises exportées annuel­le­ment aux colonies.
Il n’est pas bien difficile de décider quels ont été les inventeurs et les constructeurs de tout ce système ; ce ne sont pas à coup sûr les consommateurs, dont l’intérêt a été totalement mis de côté, mais bien les producteurs, à l’intérêt desquels on a porté une attention si soigneuse et si recherchée ; et dans cette dernière classe, les principaux architectes du système ont été, sans comparaison, nos marchands et nos manufac­tu­riers. Dans les règlements mercantiles dont il a été question dans ce chapitre, l’intérêt de nos manufacturiers est celui dont on s’est le plus particulièrement occupé, et ici c’est encore moins l’intérêt des consommateurs qu’on lui a sacrifié, que celui de quelques autres classes de producteurs.
 
 
 
↑ Cette restriction n’existe plus.
↑ L’absurdité et l’injustice paraissent être les principaux éléments dont se compose le système mercantile, réprouvé d’ailleurs aujourd’hui par tout le monde, excepté par ceux qui, en politique, admirent tout ce qui est vieux, et sans autre raison que parce que c’est vieux. C’est une chose étrange que, dans un pays où tout le inonde condamne les restrictions commerciales, on fasse si peu pour arriver à leur abolition complète. On est d’accord que le système entier est un tissu de préjuges et d’absurdités ; que les restrictions qu’il impose sont contre les intérêts de la communauté ; que dans la plupart des cas elles sont préjudiciables aux intérêts même de ceux en faveur desquels elles ont été établies ; et pourtant elles existent toujours. Il ne serait peut-être pas convenable de les abolir d’un coup ; car, ayant existé depuis longtemps, elles ont imprimé au commerce et aux capitaux du pays une direction artificielle, et tout changement brusque pourrait occasionner de graves désordres. Mais puisque cet état fâcheux est maintenant connu de tous, pourquoi des mesures ne sont-elles pas adoptées pour arriver aune réforme graduelle ? Pourquoi ne rentre-t-on pas dans la voie régulière et l’ordre naturel desquels les violences d’un système artificiel nous ont jusqu’à présent éloignés ? La raison en est fort simple. Les hommes d’État sont rarement les promoteurs zélés des réformes. Us savent parfaitement que tout projet de réforme sera combattu par les partis et les préjugée. Ce sont là deux ennemis qu’ils osent rarement défier. Ils préfèrent tolérer des abus qui existent depuis longtemps et auxquels on s’est habitué, plutôt que de s’aventurer dans des réformes qui, bien qu’approuvées par le bon sens, deviendraient pour eux une source d’attaques de la part de leurs adversaires politiques. M. Pitt, au commencement de son administration, proposa et fit accepter plusieurs mesures commerciales libérales ; mais son projet d’établir la liberté du commerce entre l’Angleterre et l’Irlande rencontra une opposition politique et commerciale tellement vive, qu’il fût obligé de le modifier dans plusieurs de ses dispositions ; et, après l’avoir ainsi fait accepter par son pays, il dut à la fin l’abandonner entièrement, par suite du refus de la part du Parlement irlandais de lui donner son approbation. Depuis les réformas commerciales de M. Pitt, aucun essai n’a été fait pour délivrer le commerce d«s liens qui l’entravaient ; et il reste encore aujourd’hui sous le joug des restrictions absurdes que lui ont imposées les statuts des Édouard et des Henri, à une époque où on croyait que le commerce ne pouvait subsister sans que les règlements du pouvoir législatif lui vinssent en aide. Une révision entière du système commercial de l’Angleterre, afin de parvenir à des réformes indispensables, est devenue plus nécessaire que jamais ; et l’introduction de changements aussi importants dans notre politique intérieure donnerait, sans aucun doute, à ceux qui les auraient effectués, des titres incontestables à l’estime et à la confiance de leur pays ; elle prouverait qu’ils avaient réellement à cœur le bien public, et qu’ils n’avaient reculé devant aucun obstacle pour en poursuivre la réalisation*. Buchanan.
↑ Quelles franchise et quelle honnêteté dans ce langage ! Adam Smith sympathisait évidemment avec les classes laborieuses, et c’est bien à tort qu’on affecte de le confondre avec quelques économistes sans entrailles qui règnent dans son pays. A. B.
↑ Ces primes ont naturellement cessé après la déclaration de l’indépendance des États-Unis. Des droits leur ont été généralement substitués, non point pour entraver le commerce avec l’Amérique, mais pour augmenter les revenus du trésor. Le commerce entre l’Angleterre et les États-Unis, par le consentement mutuel de leurs gouvernements respectifs, a toujours eu lieu sur un pied de parfaite liberté. Jamais des restrictions ou droits prohibitifs n’ont été imposés de part ou d’autre ; chacune des parties comprenant parfaitement les avantages d’un commerce actif entre les deux pays, et désirant en conséquence plutôt l’encourager que l’entraver. Buchanan.
↑ C’est-à-dire d’un crime qui emporte la peine capitale.
↑ Privilége des clercs de décliner toute juridiction. (Voy. liv. V, chap. i, sect. 3, art. 3.)
↑ Division partielle d’un comté, laquelle comprend une dizainerie.
↑ Les lois relatives à l’exportation et à l’importation de la laine, dont s’occupe ici Adam Smith, ont toutes été rapportées. Par l’acte de 1825 (6, George IV, ch. cxi), la laine valant un schelling la livre peut être librement exportée, en payant un droit d’un demi-penny par livre. Et si elle vaut plus d’un schelling la livre, elle peut être exportée moyennant un droit d’un penny la livre. La laine importée est chargée d’un droit d’un demi-penny la livre si elle vaut moins d’un schelling, et d’un droit de trois pence par livre si elle dépasse cette valeur.
Mac Culloch.
↑ Cette restriction n’existe plus. La terre à foulon et la terre à pipe peuvent être actuellement exportées, moyennant un droit d’un demi pour 100 ad valorem Mac Culloch.
↑ Par l’acte 6, George IV, ch. cxi, un droit de demi pour 100 ad valorem est imposé à l’exportation de toutes les espèces d’ouvrages en cuir. Mac Culloch.
↑ Cette prohibition n’existe plus depuis longtemps.
↑ Cette prohibition est rapportée ; la laine en fil et l’estame peuvent être exportées avec un droit d’un penny par livre. Mac Culloch.
↑ Cette prohibition à l’exportation des boites de montre et de pendules continue toujours. Mac Culloch.
↑ Cette substance métallique qu’on extrait d’une pyrite qui se trouve en abondance dans quelques mines d’étain.
↑ Cette prohibition est abolie. N. C.
↑ Voy. liv. V, ch. ii.
↑ Les droits sur ces articles ont été ou entièrement abolis ou considérablement modifiés par des actes récents. Mac Culloch.
↑ Le droit sur la gomme du Sénégal importé dans la Grande-Bretagne pour la consommation intérieure est actuellement (1838) de 6 schellings le quintal. Si elle est importée et entreposée pour la réexportation, elle est franche de droits. Mac Culloch.
↑ Le droit actuel (1838) à l’importation du poil de castor écru est d’un schelling 7 pence par livre, et s’il est rasé et peigné, de 4 schellings 9 pence par livre.
Mac Culloch.
↑ Mesure usitée pour le charbon de terre seulement, et qui contient trente-six boisseaux combles.
↑ Par un acte récent, le charbon peut être exporté franc de droits par navires anglais, et par navires étrangers avec un droit de 4 schellings par tonneau*.
Mac Culloch.
↑ Pour connaître les restrictions imposées à l’exportation des machines, voyez les actes 3 et 4, Guillaume IV, ch. lii.
↑ L’ouvrier, comme le fait justement remarquer Adam Smith, a son travail pour unique patrimoine ; l’empêcher de tirer de son travail le plus grand avantage possible serait un acte de pouvoir inexcusable. Le but de tous ces règlements est d’anéantir l’industrie des autres nations, afin de gagner le marché du monde à l’industrie indigène. Un projet pareil, qui ne peut être inspiré que dans des vues de la plus basse rivalité mercantile, est aussi absurde qu’impuissant. L’importation et l’exportation de certains produits peuvent en effet être empêchées par certaines lois particulières ; mais, qui est-ce qui pourrait lier la faculté d’invention et le génie de la société ? Les résistances d’un ou de plusieurs pays peuvent-elles arrêter les progrès du monde ? et quand, par l’accroissement général de la prospérité, une société a besoin d’une plus grande provision de produits fins, et qu’elle peut offrir des valeurs en échange, les lois d’un seul État empêcheront-elles d’autres États de lui procurer ce qu’elle désire ? Est-ce que d’ailleurs une politique qui, pour s’assurer les avantages mesquins du monopole, voudrait étouffer toute prospérité naissante, n’est pas basse et méprisable ? Buchanan.
↑ Les restrictions imposées à l’émigration des ouvriers ont été rapportées en 1824. Mac Culloch.

%%%%%%%%%%%%%%%%%%%%%%%%%%%%%%%%%%%%%%%%%%%%%%%%%%%%%%%%%%%%%%%%%%%%%%%%%%%%%%%%
%                                  Chapitre 9                                  %
%%%%%%%%%%%%%%%%%%%%%%%%%%%%%%%%%%%%%%%%%%%%%%%%%%%%%%%%%%%%%%%%%%%%%%%%%%%%%%%%

\chapter{Des systèmes agricoles, ou de ces systèmes d’économie politique qui représentent le produit de la terre soit comme la seule, soit comme la principale source du revenu et de la richesse nationale}
\markboth{Des systèmes agricoles, ou de ces systèmes d’économie politique qui représentent le produit de la terre soit comme la seule, soit comme la principale source du revenu et de la richesse nationale}{}

Les systèmes fondés sur l’agriculture n’exigeront pas une aussi longue explication que celle qui m’a paru nécessaire pour le système fondé sur le commerce.
Ce système, qui représente le produit de la terre comme la seule source du revenu et de la richesse d’un pays, n’a jamais, autant que je sache, été adopté par aucune na­tion, et n’existe à présent qu’en France, dans les spéculations d’un petit nombre d’hom­mes d’un grand savoir et d’un talent distingué. Ce n’est sûrement pas la peine de discuter fort au long les erreurs d’une théorie qui n’a jamais fait et qui vraisem­blablement ne fera jamais de mal en aucun lieu du monde. je vais cependant tâcher de tracer le plus clairement possible les principaux traits de cet ingénieux système.
M. de Colbert, le célèbre ministre de Louis XIV, était un homme de probité, grand travailleur et possédant une parfaite connaissance des détails ; apportant à l’examen des comptes publics une grande sagacité jointe à beaucoup d’expérience ; en un mot, doué des talents les plus propres, en tout genre, à introduire de l’ordre et de la mé­tho­de dans les recettes et les dépenses du revenu de l’État. Malheureusement, ce ministre avait adopté tous les préjugés du système mercantile, système essentielle­ment forma­liste et réglementaire de sa nature, et qui ne pouvait guère manquer par là de convenir à un homme laborieux et rompu aux affaires, accoutumé depuis longtemps à régler les différents départements de l’administration publique, et à établir les formalités et les contrôles nécessaires pour les contenir chacun dans leurs attributions respectives. Il chercha à régler l’industrie et le commerce d’un grand peuple sur le même modèle que les départements d’un bureau ; et, au lieu de laisser chacun se diriger à sa manière dans la poursuite de ses intérêts privés, sur un vaste et noble plan d’égalité, de liberté et de justice, il s’attacha à répandre sur certaines branches d’industrie des privilèges extraordinaires, tandis qu’il chargeait les autres d’entraves non moins extraordinaires. non-seulement il était porté, comme les autres ministres de l’Europe, à encourager l’industrie des villes de préférence à celle des campagnes, mais encore, dans la vue de soutenir l’industrie des villes, il voulait même dégrader et tenir en souffrance celle des campagnes. Pour procurer aux habitants des villes le bon marché des vivres et encou­rager par là les manufactures et le commerce étranger, il prohiba totalement l’exportation des blés et, par ce moyen, ferma aux habitants des campagnes tous les marchés étrangers pour la partie, sans comparaison, la plus importante du produit de leur industrie. Cette prohibition, jointe aux entraves dont les anciennes lois provin­ciales de France avaient embarrassé le transport du blé d’une province à l’autre, ainsi qu’aux impôts arbitraires et avilissants qui se lèvent sur les cultivateurs dans presque toutes les provinces, découragea l’agriculture de ce pays et la tint dans un état de dégradation bien différent de l’état auquel la nature l’avait destinée à s’élever sur un sol aussi fertile et sous un climat aussi heureux. Cet état de découragement et de souffrance se fit sentir plus ou moins dans chacune des parties du royaume, et on procéda à différentes recherches pour en découvrir les causes. On s’aperçut bien qu’une de ces causes était la préférence que les institutions de M. de Colbert avaient donnée à l’industrie des villes sur celle des campagnes.
Si la branche est trop courbée dans un sens, dit le proverbe, il faut, pour la redres­ser, la courber tout autant dans le sens contraire. Il semble que ce soit sur cette maxime triviale que se sont dirigés les philosophes français, auteurs du système qui représente l’agriculture comme l’unique source du revenu et de la richesse d’un pays ; et si, dans le plan de M. de Colbert, l’industrie des villes avait certainement été éva­luée trop haut en comparaison de celle des campagnes, aussi, dans leur système, ils paraissent non moins certainement avoir compté celle-là pour trop peu.
Ils divisent en trois les différentes classes de peuple qu’on suppose contribuer, d’une manière quelconque, au produit annuel de la terre et du travail du pays. La première est la classe des propriétaires de terre ; la seconde est la classe des cultiva­teurs, fermiers et ouvriers de la campagne, qu’ils honorent en particulier du nom de classe productive ; la troisième est la classe des artisans, manufacturiers et mar­chands qu’ils affectent de dégrader en la désignant par la dénomination humiliante de classe stérile ou non productive.
La classe des propriétaires contribue à la formation du produit annuel par les dépenses qu’ils font, dans l’occasion, en amendements sur les terres, en constructions, en saignées et arrosements, clôtures et autres améliorations à faire ou à entretenir, et par le moyen desquelles les cultivateurs se trouvent en état, avec un même capital, de faire naître un plus grand produit et, par conséquent, de payer une plus forte rente. Cet accroissement de la terre peut être considéré comme l’intérêt ou le profit dû au propriétaire, en raison de la dépense ou du capital qu’il a employé de cette manière à améliorer sa terre. Ces sortes de dépenses sont nommées, dans ce système, dépenses foncières.
Les cultivateurs ou fermiers contribuent à la formation du produit annuel par les dépenses qu’ils appliquent à la culture, et qu’on distingue, dans ce système, en dépenses primitives et en dépenses annuelles. Les dépenses primitives consistent dans les instruments de labourage, le fonds de bestiaux, etc., ainsi que dans les semences et dans la subsistance de la famille du fermier, de ses valets et bestiaux de travail, pen­dant au moins une grande partie de la première année de son exploitation, ou jusqu’à ce qu’il puisse recevoir de la terre quelques rentrées. Les dépenses annuelles consis­tent dans les semences, l’entretien et réparation des instruments de labour, et dans la subsistance annuelle des valets et des bestiaux du fermier, aussi bien que de sa famille, autant qu’une partie de sa famille peut être regardée comme domestiques em­ployés à la culture. Cette portion du produit de la terre qui lui reste après le payement de la rente doit être suffisante, premièrement pour lui remplacer dans un espace de temps raisonnable, au moins dans le cours de son bail, la totalité de ses dépenses primitives, avec les profits ordinaires d’un capital, et secondement, pour lui remplacer annuellement la totalité de ses dépenses annuelles, avec les profits ordinaires d’un capital. Ces deux sortes de dépenses sont deux capitaux que le fermier emploie à la culture et, à moins qu’ils ne lui soient régulièrement remboursés avec un profit raisonnable, il ne peut pas soutenir son industrie au niveau des autres ; au contraire, il sera porté, par son intérêt personnel, à abandonner cet emploi le plus tôt possible, et à en chercher quelque autre. Cette portion du produit de la terre, qui est ainsi nécessaire pour mettre le fermier en état de continuer l’industrie qu’il a embrassée, doit être considérée comme un fonds consacré à la culture, sur lequel le propriétaire ne saurait étendre la main sans réduire nécessairement le produit de sa terre, et sans mettre le fermier, en peu d’années, hors d’état de payer non-seulement la rente qu’on lui aurait arrachée par violence, mais même la rente raisonnable que, sans cela, le propriétaire eût pu s’attendre à retirer de sa terre. La rente qui appartient proprement au propriétaire n’est autre chose que le produit net qui reste après qu’il a été satisfait complètement à toutes les dépenses dont il a fallu préalablement faire l’avance pour faire croître le produit brut ou produit total. C’est parce que le travail des cultivateurs, en outre du remboursement parfait de toutes ces dépenses nécessaires, rapporte encore un produit net comme on vient de le définir, que cette classe en particulier se trouve distinguée, dans ce système, par l’ho­no­rable dénomination de classe productive, Les dépenses primitives et annuelles, par la même raison, sont appelées, dans ce système, dépenses productives, parce qu’après avoir remplacé leur propre valeur, elles donnent encore lieu à la reproduction annu­elle de ce produit net.
Les dépenses foncières, comme on les appelle, ou celles que le propriétaire place en amélioration de sa terre, sont aussi, dans ce système, honorées de la dénomination de dépenses productives. jusqu’à ce que la totalité de ces dépenses, avec les profits ordinaires d’un capital, lui aient été complètement remboursés par le surcroît de rente qu’il retire de sa terre, ce surcroît de rente doit être regardé comme sacré et inviolable aux yeux de l’Église et du souverain ; il ne doit être assujetti ni à la dîme ni à l’impôt. S’il en est autrement, en décourageant l’amélioration de la terre, l’Église décourage l’accroissement futur de ses propres dîmes, et le roi, l’accroissement futur de la masse imposable. Par conséquent, comme dans un état de choses bien ordonné ces dépenses foncières, après avoir complètement reproduit leur propre valeur, occasionnent pareillement, en outre de cette reproduction, celle d’un produit net, au bout d’un certain temps on les considère aussi, dans ce système, comme dépenses productives.
Toutefois, les dépenses foncières du propriétaire, avec les dépenses primitives et annuelles du fermier, sont les trois seules espèces de dépenses qui soient, dans ce système, considérées comme productives.
Suivant cette manière d’envisager les choses, toutes autres dépenses et toutes au­tres classes de peuple, celles même qui, dans les idées ordinaires des hommes, sont regardées comme les plus productives, sont représentées ici comme totalement stériles ou non productives.
Les manufacturiers et artisans en particulier, dont l’industrie, d’après les idées communes, ajoute tant à la valeur des produits bruts de la terre, sont représentés dans ce système comme une classe de gens entièrement stériles et non productifs. Leur travail, dit-on, remplace seulement le capital qui les emploie, ainsi que les profits ordinaires de ce capital. Ce capital consiste dans les matières, outils et salaires que leur avance celui qui les met en œuvre, et c’est le fonds destin, à les tenir occupés et à les faire subsister. Les profits de ce capital sont le fonds destiné à la subsistance de celui qui les met en œuvre. Celui-ci, en même temps qu’il leur avance le fonds de matières, outils et salaires nécessaires pour les tenir occupés, s’avance aussi à lui-même ce qui est nécessaire à sa subsistance, et en général il proportionne cette subsistance au profit qu’il s’attend à faire sur le prix de leur ouvrage. À moins que le prix de l’ouvrage ne lui rembourse et la subsistance qu’il s’est avancée à lui-même, et les matériaux, outils et salaires qu’il a avancés à ses ouvriers, il est évident que cet ouvrage ne lui rendra pas toute la dépense qu’il y a mise. Par conséquent, les profits du capital employé en manufacture ne sont pas, comme la rente d’une terre, un produit net qui reste après le remboursement complet de toute la dépense indispen­sable avancée pour l’obtenir. Le capital du fermier lui rend un profit, aussi bien que celui du maître manufacturier, mais il rend encore de plus une rente à une autre personne, ce que ne fait pas le capital du manufacturier. Par conséquent, la dépense que l’on fait pour employer et faire subsister des artisans et ouvriers de manufacture, ne fait autre chose que de continuer, pour ainsi dire, l’existence de sa propre valeur, et elle ne produit aucune valeur nouvelle. C’est donc une dépense absolument stérile et non productive.
Au contraire, la dépense que l’on fait pour employer et faire subsister des fermiers et ouvriers de culture, outre qu’elle continue l’existence de sa propre valeur, produit encore une nouvelle valeur, qui est la rente du propriétaire. Cette dépense est donc productive.
Le capital employé dans le commerce est tout aussi stérile et non productif que le capital placé dans les manufactures. Il ne fait non plus que continuer l’existence de sa propre valeur, sans produire aucune valeur nouvelle. Ces profits ne sont que le rem­boursement de la subsistance que s’avance à soi-même celui qui emploie le capital, pendant le temps qu’il l’emploie, ou jusqu’à ce qu’il en ait reçu la rentrée. Ils ne sont que le remboursement d’une partie de la dépense qu’il faut nécessairement faire en employant ce capital.
Le travail des artisans et ouvriers de manufacture n’ajoute jamais la moindre chose à la valeur de la somme totale du produit brut de la terre. Il est bien vrai qu’il ajoute considérablement à la valeur de quelques parties de ce produit, vues séparément. Mais la valeur ajoutée à ces parties n’est précisément qu’un équivalent de la consom­mation d’autres parties de ce produit, à laquelle il donne lieu en même temps ; de manière que la valeur de la somme totale du produit ne se trouve, en aucun moment, augmentée de la moindre chose par ce travail. Par exemple, la personne qui fait la dentelle d’une très-belle paire de manchettes, fera quelquefois monter à 30 livres sterling la valeur de peut-être un denier de lin. Mais quoique, au premier coup d’œil, cette personne paraisse par là multiplier 7,200 fois environ la valeur d’une partie du produit brut, dans la réalité elle n’ajoute rien à la valeur de la somme totale du produit brut. La façon de cette dentelle lui coûte peut-être deux années de travail. Les 30 livres qu’elle en retire quand l’ouvrage est fini, ne sont autre chose que le rembour­sement de la subsistance qu’elle s’est avancée à elle-même durant les deux années qu’elle a été occupée à cet ouvrage. La valeur qu’elle ajoute au lin par le travail de chaque jour, de chaque mois, de chaque aimée, ne fait autre chose que remplacer la valeur de ce qu’elle consomme pendant ce jour, ce mois, cette année. Ainsi, il n’y a aucun instant dans lequel elle ait ajouté la plus petite chose à la valeur de la somme totale du produit brut de la terre, la portion de ce produit qu’elle va consommant continuellement étant toujours égale à la valeur qu’elle va produisant aussi continuel­lement. L’extrême pauvreté de la plupart des personnes employées à cette espèce de manufacture, si dispendieuse malgré sa frivolité, suffit bien pour nous convaincre que, pour l’ordinaire, le prix de leur travail n’excède pas la valeur de leur subsistance.
Il en est autrement du travail des fermiers et ouvriers de la campagne. La rente du propriétaire est une valeur que ce travail rend, d’ordinaire, continuellement produc­tive, vu qu’il remplace en outre, et le plus complètement possible, la totalité de la consommation des ouvriers et de celui qui les met en œuvre, ainsi que la totalité de la dépense avancée pour les employer et les faire subsister tous.
Les artisans, manufacturiers et marchands ne peuvent ajouter à la richesse et au revenu de la société que par leurs économies seulement, ou bien, suivant l’expression adoptée dans ce système, par des privations, c’est-à-dire en se privant de jouir d’une partie du fonds destiné à leur subsistance personnelle. Annuellement, ils ne reprodui­sent rien autre chose que ce fonds. À moins donc qu’annuellement ils n’en épargnent quelque partie, à moins qu’ils ne se privent annuellement de la jouissance de quelque portion de ce fonds, la richesse et le revenu de la société ne peuvent recevoir de leur industrie le plus petit degré d’augmentation. Les fermiers et ouvriers de la culture, au contraire, peuvent jouir complètement de tout le fonds destiné à leur subsistance personnelle, et cependant ajouter en même temps à la richesse et au revenu de la société. En outre de ce qui est destiné à leur subsistance personnelle, leur industrie rend annuellement encore un produit net dont la formation ajoute nécessairement à la richesse et au revenu de la société. Par conséquent, les nations telles que la France ou l’Angleterre, qui sont composées en grande partie de propriétaires et de cultivateurs, peuvent s’enrichir en travaillant et jouissant tout à la fois. Au contraire, les nations, telles que la Hollande, telles que Hambourg, qui sont principalement composées de marchands, de manufacturiers et d’artisans, ne peuvent devenir riches qu’à force d’économies et de privations. Comme des nations placées dans des circonstances aussi différentes se trouvent avoir un intérêt d’une nature très-différente, le caractère général du peuple doit se ressentir aussi de cette différence. Chez les nations de la première espèce, des manières libérales, franches et enjouées, le goût du plaisir et de la société, entrent naturellement dans ce caractère général. Chez les autres, on trouve de la mesquinerie, de la petitesse, des inclinations intéressées et égoïstes, et de l’éloi­gnement pour tous les amusements et toutes les jouissances sociales.
La classe non productive, celle des marchands, artisans et manufacturiers, est entre­tenue et employée entièrement aux dépens des deux autres classes, celle des propriétaires et celle des cultivateurs. Celles-ci lui fournissent à la fois les matériaux de son travail et le fonds de sa subsistance, le blé et le bétail qu’elle consomme pen­dant qu’elle est occupée à ce travail. Les propriétaires et les cultivateurs payent, en dernier résultat, les salaires de tous les ouvriers de la classe non productive et les profits de tous les entrepreneurs qui mettent ces ouvriers en œuvre. Ces ouvriers et ceux qui les mettent en œuvre sont, à proprement parler, les serviteurs des proprié­taires et des cultivateurs. Seulement, ce sont des serviteurs qui sont employés au-dehors de la maison, comme les serviteurs domestiques le sont au-dedans. Les uns et les autres n’en sont pas moins également entretenus aux dépens des mêmes maîtres. Le travail des uns et des autres est également non productif. Également il n’ajoute rien à la somme totale de la valeur du produit brut de la terre. Au lieu d’augmenter la valeur de cette somme totale, ce travail est une charge de ce produit, une dépense qu’il faut payer sur ce produit.
Toutefois, la classe non productive est non-seulement utile, mais extrêmement utile aux deux autres classes. C’est à la faveur de l’industrie des marchands, des artisans et des manufacturiers, que les propriétaires et les cultivateurs peuvent acheter des denrées étrangères, ainsi que les produits manufacturés de leur propre pays dont ils ont besoin, moyennant le produit d’une bien moindre quantité de leur travail, que celle qu’ils se trouveraient obligés d’y employer s’il leur fallait essayer, sans en avoir l’adresse ni l’habileté, soit d’exporter les unes, soit de fabriquer les autres pour leur usage personnel. La classe non productive débarrasse les cultivateurs d’une foule de travaux qui sans cela les distrairaient de la culture. La supériorité du produit qu’ils se trouvent en état d’obtenir, au moyen de ce que leurs soins ne sont pas détournés vers d’autres objets, suffit largement à payer toute la dépense que coûte la classe non pro­ductive, tant à eux qu’aux propriétaires. De cette manière l’industrie des mar­chands, artisans et manufacturiers, encore que tout à fait non productive par sa nature, contribue cependant indirectement à accroître le produit de la terre. Elle augmente les facultés productrices du travail productif, en le mettant à même de se consacrer tout entier à son véritable emploi, la culture de la terre ; et souvent l’homme dont le métier est le plus étranger à la charrue sert, par son travail, à faire aller la charrue plus facilement et plus vite.
L’intérêt des propriétaires et des cultivateurs ne peut jamais être de gêner ou de décourager en rien l’industrie des marchands, des artisans et des manufacturiers. Plus sera grande la liberté dont jouira la classe non productive, plus sera grande la concurrence dans tous les divers métiers qui composent cette classe, et plus alors les deux classes se trouveront fournies à bon marché, tant des denrées étrangères, que des produits manufacturés de leur propre pays.
L’intérêt de la classe non productive ne peut jamais être d’opprimer les deux autres. C’est le produit superflu de la terre, ou ce qui reste du produit, déduction faite pre­mièrement de la subsistance des cultivateurs, et secondement de celle des propriétaires, qui emploie et fait subsister la classe non productive. Plus ce superflu sera grand, et plus nécessairement sera abondant aussi le fonds qui emploie et entre­tient cette classe. L’établissement de la parfaite justice, de la parfaite liberté et de la parfaite égalité est le secret extrêmement simple d’assurer, de la manière la plus efficace, à toutes les trois classes le plus haut degré de prospérité.
Les marchands, artisans et manufacturiers de ces États purement commerçants, qui, tels que Hambourg et la Hollande, consistent principalement dans cette classe non productive, sont, de la même manière, employés et entretenus en entier aux frais de propriétaires et de cultivateurs de terres. La seule différence, c’est que ces pro­priétaires et cultivateurs sont, pour la plupart, placés à une distance beaucoup plus incommode des marchands, artisans et manufacturiers auxquels ils fournissent des matériaux à travailler et un fonds de subsistance ; qu’ils sont les habitants d’autres pays et les sujets d’autres gouvernements.
Néanmoins, ces États commerçants sont non-seulement utiles, mais extrêmement utiles aux habitants de ces autres pays. Ils remplissent, à un certain point, un vide très-important, et ils tiennent la place de marchands, d’artisans et de manufacturiers que les habitants de ces autres pays devaient trouver chez eux, mais qu’ils n’y trouvent pas, d’après quelque vice dans leur conduite politique.
L’intérêt des nations terriennes, si je puis m’exprimer ainsi, ne peut jamais être de décourager ou de ruiner l’industrie des nations marchandes, en imposant de gros droits sur leur commerce ou sur les marchandises qu’elles fournissent. Ces droits, en renchérissant les marchandises, ne servent qu’à rabaisser la valeur réelle du produit superflu des terres avec lequel, ou, ce qui revient au même, avec le prix duquel ces marchandises sont achetées. Ces droits ne servent qu’à décourager l’accroissement de cet excédent de produit et, par conséquent, l’amélioration et la culture des terres. L’expédient le plus sûr, au contraire, pour élever la valeur de cet excédent de produit, pour en encourager l’accroissement et, par conséquent, la culture et l’amélioration des terres, ce serait d’accorder au commerce des nations marchandes la plus entière liberté.
Cette parfaite liberté de commerce serait même pour les nations terriennes le plus sûr moyen de se procurer, au bout d’un certain temps, tous ces artisans, manufac­turiers et marchands dont elles manquent chez elles, et de remplir, de la manière la plus convenable et la plus avantageuse, le vide très-important qu’elles éprouvent à cet égard.
L’augmentation continuelle de l’excédent de produit de leurs terres viendrait à créer, au bout d’un certain temps, un capital plus grand que ce que l’amélioration et la culture des terres pourraient en employer avec un profit ordinaire, et l’excédent de ce capital servirait naturellement à employer des artisans et des manufacturiers dans l’intérieur. Or, ces artisans et manufacturiers, trouvant dans le pays même et les matériaux de leur ouvrage et le fonds de leur subsistance, pourraient tout d’un coup, même avec moins d’art et d’habileté, être à même de travailler à aussi bon marché que les artisans et manufacturiers de ces États commerçants, obligés de faire venir ces deux articles d’une plus grande distance. Même en supposant que, faute d’art et d’habileté, ils ne pussent pas, pour un certain temps, travailler à aussi bon marché, cependant, trouvant le débit sous leur main, ils seraient encore à même d’y vendre leur produit à aussi bon marché que celui des artisans et manufacturiers des États commerçants, qui ne pourrait être mis au marché qu’après un très-long trajet ; et comme leur art et leur habileté iraient en se perfectionnant, ils seraient bientôt en état de vendre à meilleur marché que les autres. Ainsi, les artisans et manufacturiers des États commerçants auraient bientôt, sur le marché de ces nations agricoles, des rivaux et des concurrents ; bientôt après, ils y seraient supplantés par ces mêmes rivaux qui offriraient à plus bas prix ; bientôt après enfin, ils se verraient obligés de s’en retirer tout à fait. En conséquence des progrès successifs de l’art et de l’habileté des ouvriers, le bon marché des produits manufacturés de ces nations agricoles étendrait, au bout d’un certain temps, au-delà du marché intérieur, la vente de ces produits, et les ferait rechercher sur les marchés étrangers, d’où ils finiraient peu à peu par exclure une grande partie des produits manufacturés des peuples purement commerçants.
Cette augmentation continuelle du produit tant brut que manufacturé de ces nations agricoles viendrait à créer, au bout d’un certain temps, un capital plus grand que ce que l’agriculture et les manufactures ensemble en pourraient tenir employé, avec un profit qui fût un taux ordinaire. Le surplus de ce capital se tournerait natu­rellement vers le commerce étranger, et serait employé à exporter aux nations étrangères les portions de ce produit, tant brut que manufacturé, qui se trouveraient excéder la demande du marché intérieur. Dans l’exportation de ce produit du pays, les marchands de ces nations agricoles auraient, sur ceux des peuples purement com­mer­çants, un avantage du même genre que celui qu’avaient leurs artisans et manufac­turiers sur ceux de ces mêmes peuples, l’avantage de trouver chez eux-mêmes cette cargaison, ces munitions et ces vivres que les autres seraient obligés d’aller chercher au loin. Par conséquent, avec moins d’art et d’habileté dans la navigation, ils seraient encore dans le cas de vendre sur les marchés étrangers leurs cargaisons à aussi bon marché que les marchands des peuples purement commerçants et, à égalité d’art et d’habileté, ils seraient en état de vendre à meilleur marché. Ces nations en viendraient donc bientôt à rivaliser avec les peuples commerçants dans cette branche de leur commerce étranger, et finiraient, au bout de quelque temps, par les en exclure tout à fait.
Ainsi, d’après ce noble et généreux système, la méthode la plus avantageuse, pour une nation à grand territoire, de faire naître chez elles des artisans, des manufacturiers et des marchands, c’est d’accorder la plus parfaite liberté commerciale aux artisans, aux manufacturiers et aux marchands de toutes les autres nations. Par là, elle élève la valeur du surplus du produit de ses terres, dont l’augmentation continuelle forme successivement un fonds qui fera nécessairement naître chez elle, au bout d’un certain temps, tous les artisans manufacturiers et marchands dont elle a besoin.
Quand, au contraire, une nation à grand territoire opprime, par des droits énormes ou par des prohibitions, le commerce des nations étrangères, et de toutes les espèces d’ouvrage de manufacture étrangère, elle nuit à ses propres intérêts de deux manières différentes. Premièrement, en faisant hausser le prix de toutes les denrées étrangères, elle fait baisser nécessairement la valeur réelle du surplus de produit de ses terres, avec lequel, ou, ce qui revient au même, avec le prix duquel elle achète ces denrées et marchandises étrangères. Secondement, en donnant à ses marchands, artisans et ma­nu­facturiers une sorte de monopole sur le marché intérieur, elle élève le taux des profits du commerce et des manufactures relativement à celui des profits de l’agriculture, et par là, ou elle enlève à l’agriculture une partie du capital qui y était employé auparavant, ou elle détourne d’y aller une partie du capital qui s’y serait porté sans cela. Par conséquent, une telle politique décourage l’agriculture de deux manières à la fois : d’abord en dégradant la valeur réelle de son produit et faisant baisser par là le taux de ses profits ; ensuite, en faisant hausser le taux des profits dans tous les autres emplois. C’est rendre, d’une part, l’agriculture moins lucrative et, de l’autre, le commerce et les manufactures plus lucratifs qu’ils n’auraient été sans cela ; en sorte que tout homme se trouve tenté, par son intérêt personnel, de retirer son capital et son industrie de la première, pour en porter autant qu’il peut dans les autres.
Quand même on supposerait qu’une nation à grand territoire pût parvenir, au moyen de ces mesures oppressives, à produire chez elle des artisans, des manufac­tu­riers et des marchands un peu plus tôt qu’elle ne l’aurait pu par la liberté du com­merce, chose qui ne laisse pas cependant d’être fort douteuse, toutefois elle les produirait, si on peut parler ainsi, d’une manière précoce et avant d’être parfaitement mûre pour cela. En se pressant de faire croître d’une manière trop hâtive une espèce d’industrie, elle affaiblirait une autre espèce d’industrie plus précieuse. En se pressant trop de donner naissance à une industrie qui ne fait que remplacer le capital qui la met en activité et un profit ordinaire, elle retarderait les progrès d’une autre industrie qui, après avoir remplacé ce capital et donné le profit ordinaire, rapporte en outre un produit net, une rente franche et libre au propriétaire. En donnant un encouragement prématuré à ce genre de travail qui est absolument stérile et non productif, elle arrê­terait le parfait développement des forces du travail qui est productif.
L’ingénieux et profond auteur de ce système, M. Quesnay, a représenté dans les formules arithmétiques, de quelle manière, suivant son système, la somme totale du produit annuel de la terre se distribue entre les trois classes ci-dessus, et comment le travail de la classe non productive ne fait que remplacer la valeur de sa consomma­tion, sans ajouter la moindre chose à la valeur de cette somme totale. La première de ces formules, qu’il a distinguée par excellence sous le nom de Tableau économique, représente la manière dont il suppose que cette distribution a lieu dans l’état de la plus parfaite liberté et par conséquent de la plus haute pros­périté ; dans un état de choses où le produit annuel est tel qu’il rend le plus grand produit net possible, et où chaque classe jouit de la part qui lui doit revenir dans la masse du produit annuel. Des formules subséquentes représentent la manière dont il suppose que cette distribution se fait sous différents régimes de règlements et d’entra­ves dans lesquels, ou la classe des propriétaires, ou la classe stérile et non productive est plus favorisée que la classe des cultivateurs, et dans lesquels l’une ou l’autre usurpe plus ou moins sur la part qui devrait justement revenir à cette classe produc­tive. Toute usurpation de ce genre, toute violation de cette distribution naturelle qu’établirait la plus parfaite liberté, doit infailliblement, selon ce système, diminuer plus ou moins, d’une année à l’autre, la valeur et la somme totale du produit annuel, et doit nécessairement occasionner un dépérissement graduel de la richesse et du revenu réel de la société, dépérissement dont les progrès seront plus rapides ou plus lents, selon les degrés de cette usurpation, selon que l’on aura plus ou moins violé cette distribution naturelle que la plus parfaite liberté ne manquerait pas d’établir. Ces formules subséquentes représentent les différents degrés de décadence, qui, suivant ce système, correspondent aux différents degrés dans lesquels aura été violée cette distribution naturelle des choses[1]. Quelques médecins spéculatifs se sont imaginé, à ce qu’il semble, que la santé du corps humain ne pouvait se maintenir que par un certain régime précis de diète et d’exercice dont on ne pouvait s’écarter le moins du monde, sans occasionner nécessairement un degré quelconque de maladie ou de dérangement proportionné au degré de cette erreur de régime. Cependant, l’expérience semble bien démontrer que le corps humain conserve, au moins dans toutes les apparences, le plus parfait état de santé sous une immense multitude de régimes divers, même avec des régimes que l’on croit généralement fort loin d’être parfaitement salutaires. Il paraî­trait donc que l’état de santé du corps humain contient en soi-même quelque principe inconnu de conservation, tendant à prévenir ou à corriger, à beaucoup d’égards, les mauvais effets d’un régime même très-vicieux. M. Quesnay, qui était lui-même méde­cin, et médecin très-spéculatif, paraît s’être formé la même idée du corps politique, et s’être figuré qu’il ne pourrait fleurir et prospérer que sous un certain régime précis, le régime exact de la parfaite liberté et de la parfaite justice. Il n’a pas considéré, à ce qu’il semble, que dans le corps politique l’effort naturel que fait sans cesse chaque individu pour améliorer son sort, est un principe de conservation capable de prévenir et de corriger, à beaucoup d’égards, les mauvais effets d’une économie partiale et même jusqu’à un certain point oppressive. Une telle économie, bien qu’elle retarde, sans contredit, plus ou moins le progrès naturel d’une nation vers la richesse et la prospérité, n’est pourtant pas toujours capable d’en arrêter totalement le cours, et encore moins de lui faire prendre une marche rétrograde. Si une nation ne pouvait prospérer sans la jouissance d’une parfaite liberté et d’une parfaite justice, il n’y a pas au monde une seule nation qui eût jamais pu prospérer. Heureusement que, dans le corps politique, la sagesse de la nature a placé une abondance de préservatifs propres à remédier à la plupart des mauvais effets de la folie et de l’injustice humaine, tout comme elle en a mis dans le corps physique pour remédier à ceux de l’intempérance et de l’oisiveté.
Néanmoins, l’erreur capitale de ce système paraît consister en ce qu’il représente la classe des artisans, manufacturiers et marchands, comme totalement stérile et non productive[2]. Les observations suivantes pourront faire voir combien est inexacte cette manière d’envisager les choses.
Premièrement, on convient que cette classe reproduit annuellement la valeur de sa propre consommation annuelle, et continue au moins l’existence du fonds ou capital qui la tient employée et la fait subsister. Mais, à ce compte, c’est donc très-impro­pre­ment qu’on lui applique la dénomination de stérile ou non productive. Nous n’appel­lerions pas stérile ou non reproductif un mariage qui ne reproduirait seulement qu’un fils et une fille pour remplacer le père et la mère, quoique ce mariage ne contribuât point à augmenter le nombre des individus de l’espèce humaine, et ne fit que conti­nuer la population telle qu’elle était auparavant. À la vérité, les fermiers et les ouvriers de la campagne, outre le capital qui les fait travailler et subsister, repro­dui­sent encore annuellement un produit net, une rente franche et quitte au propriétaire. Aussi, de même qu’un mariage qui donne trois enfants est certainement plus productif que celui qui n’en donne que deux, de même le travail des fermiers et ouvriers de la campagne est assurément plus productif que celui des marchands, des artisans et manufacturiers. Toutefois, la supériorité du produit de l’une de ces classes ne fait pas que l’autre soit stérile et non productive[3].
Secondement, sous ce même rapport, il paraît aussi tout à fait impropre de considérer les artisans, manufacturiers et marchands, sous le même point de vue que de simples domestiques. Le travail d’un domestique ne continue pas l’existence du fonds qui lui fournit son emploi et sa subsistance. Ce domestique est employé et entretenu finalement aux dépens de son maître, et le travail qu’il fait n’est pas de nature à pouvoir rembourser cette dépense. Son ouvrage consiste en services qui, en général, périssent et disparaissent à l’instant même où ils sont rendus, qui ne se fixent ni ne se réalisent en aucune marchandise qui puisse se vendre et remplacer la valeur de la subsistance et du salaire. Au contraire, le travail des artisans, marchands et manufac­turiers se fixe et se réalise naturellement en une chose vénale et échangeable. C’est sous ce rapport que, dans le chapitre où je traite du travail productif et du travail non productif, j’ai classé les artisans, les manufacturiers et les marchands parmi les ouvriers productifs, et les domestiques parmi les ouvriers stériles et non productifs[4].
Troisièmement, dans toutes les suppositions, il me semble impropre de dire que le travail des artisans, manufacturiers et marchands n’augmente pas le revenu réel de la société. Quand même nous supposerions, par exemple, comme on le fait dans ce système, que la valeur de ce que consomme cette classe pendant un jour, un mois, une année, est précisément égale à ce qu’elle produit pendant ce jour, ce mois, cette année, cependant il ne s’ensuivrait nullement de là que son travail n’ajoutât rien au revenu réel de la société, à la valeur réelle du produit annuel des terres et du travail du pays. Par exemple, un artisan qui, dans les six mois qui suivent la moisson, exécute pour la valeur de 10 livres d’ouvrage, quand même il aurait consommé pen­dant le même temps pour la valeur de 10 livres de blé et d’autres denrées nécessaires à la vie, ajoute néanmoins, en réalité, une valeur de 10 livres au produit annuel des terres et du travail de la société. Pendant qu’il a consommé une demi-année de revenu valant 10 livres en blé et autres denrées de première nécessité, il a en même temps produit une valeur égale en ouvrage, laquelle peut acheter pour lui ou pour quelque autre personne une pareille demi-année de revenu. Par conséquent, la valeur de ce qui a été tant consommé que produit pendant ces six mois, est égale non à 10, mais à 20 livres. Il est possible, à la vérité, que, de cette valeur, il n’en ait jamais existé, dans un seul instant, plus de 10 livres en valeur à la fois. Mais si les 10 livres vaillant, en blé et autres denrées de nécessité qui ont été consommées par cet artisan, eussent été consommées par un soldat ou par un domestique, la valeur de la portion existante du produit annuel, au bout de ces six mois, aurait été de 10 livres moindre de ce qu’elles s’est trouvée être, en conséquence du travail de l’ouvrier. Ainsi, quand même ou supposerait que la valeur produite par l’artisan n’est jamais, à quelque moment que ce soit, plus grande que la valeur par lui consommée[5], cependant la valeur totale des marchandises actuellement existantes sur le marché, à quelque moment qu’on la prenne, se trouve être, en conséquence de ce qu’il produit, plus grande qu’elle ne l’aurait été sans lui.
Quand les champions de ce système avancent que la consommation des artisans, manufacturiers et marchands est égale à la valeur de ce qu’ils produisent, vraisem­blablement ils n’entendent pas dire autre chose, sinon que le revenu de ces ouvriers ou le fonds destiné à leur subsistance est égal à cette valeur. Mais, s’ils s’étaient exprimés avec plus d’exactitude et qu’ils eussent seulement soutenu que le revenu de cette classe était égal à ce qu’elle produisait, alors il serait venu tout aussitôt à l’idée du lecteur que ce qui peut naturellement être épargné sur ce revenu doit nécessai­rement augmenter plus ou moins la richesse réelle de la société. Afin donc de pouvoir faire sortir de leur proposition quelque chose qui eût l’air d’un argument, il fallait qu’ils s’exprimassent comme ils l’ont fait, et encore cet argument, dans la supposition que les choses fussent, dans le fait, telles qu’ils les supposent, se trouve n’être nulle­ment concluant.
Quatrièmement, les fermiers et ouvriers de la campagne ne peuvent, non plus que les artisans, manufacturiers et marchands, augmenter le revenu réel de la société, le produit annuel de ses terres et de son travail, autrement que par leurs économies per­sonnelles. Le produit annuel des terres et du travail d’une société ne peut recevoir d’augmentation que de deux manières : ou bien, premièrement, par un perfection­nement survenu dans les facultés productives du travail utile actuellement en activité dans cette société, ou bien, secondement, par une augmentation survenue dans la quantité de ce travail.
Pour qu’il survienne quelque perfectionnement ou accroissement de puissance dans les facultés productives du travail utile, il faut ou que l’habileté de l’ouvrier se perfectionne, ou que l’on perfectionne les machines avec lesquelles il travaille. Or, comme le travail des artisans et manufacturiers est susceptible de plus de subdivi­sions que celui des fermiers ou ouvriers de la campagne, et que la tâche de chaque ouvrier y est réduite à une plus grande simplicité d’opérations que celle des autres, il est, par cette raison, pareillement susceptible d’acquérir l’un et l’autre de ces deux entes de perfectionnement dans un degré bien plus élevé[6]. À cet égard donc, la classe des cultivateurs ne peut avoir aucune espèce d’avantage sur celle des artisans et manufacturiers.
L’augmentation dans la quantité de travail utile actuellement employé dans une société dépend uniquement de l’augmentation du capital qui le tient en activité ; et à son tour, l’augmentation de ce capital doit être précisément égale au montant des épargnes que font sur leurs revenus ou les personnes qui dirigent et administrent ce capital, ou quelques autres personnes qui le leur prêtent. Si, comme ce système sem­ble le supposer, les marchands, artisans et manufacturiers sont naturellement plus disposés à l’économie et à l’habitude d’épargner que ne le sont les propriétaires et les cultivateurs, ils sont vraisemblablement d’autant plus dans le cas d’augmenter la quantité du travail utile employé dans la société dont ils font partie et, par consé­quent, d’augmenter le revenu réel de cette société, le produit annuel de ses terres et de son travail.
Cinquièmement, enfin, quand même on admettrait que le revenu des habitants d’un pays consiste uniquement, comme ce système paraît le supposer, dans la quantité de subsistance que peut leur procurer leur industrie, cependant, dans cette supposition même, le revenu d’un pays manufacturier et trafiquant doit être, toutes choses égales d’ailleurs, nécessairement toujours beaucoup plus grand que celui d’un pays sans trafic et sans manufactures. Au moyen du trafic et des manufactures, un pays peut annuellement importer chez lui une beaucoup plus grande quantité de subsistances que ses propres terres ne pourraient lui en fournir dans l’état actuel de leur culture. Quoique les habitants d’une ville ne possèdent souvent point de terres à eux, ils attirent cependant à eux, par leur industrie, une telle quantité du produit brut des terres des autres, qu’ils trouvent à s’y fournir, non-seulement des matières premières de leur travail, mais encore du fonds de leur subsistance. Ce qu’une ville est toujours à l’égard de la campagne de son voisinage, un État ou un pays indépendant peut souvent l’être à l’égard d’autres États ou pays indépendants. C’est ainsi que la Hollan­de tire des autres pays une grande partie de sa subsistance ; son bétail vivant du Holstein et du Jutland, et son blé de presque tous les différents pays de l’Europe.
Une petite quantité de produit manufacturé achète une grande quantité de produit brut. Par conséquent, un pays manufacturier et trafiquant achète naturellement, avec une petite partie de son produit manufacturé, une grande partie du produit brut des autres pays, tandis qu’au contraire un pays sans trafic et sans manufactures est, en général, obligé de dépenser une grande partie de son produit brut pour acheter une très-petite partie du produit manufacturé des autres pays. L’un exporte ce qui ne peut servir à la subsistance et aux commodités que d’un très-petit nombre de personnes, et il importe de quoi donner de la subsistance et de l’aisance à un grand nombre. L’autre exporte la subsistance et les commodités d’un grand nombre de personnes, et importe de quoi donner à un très-petit nombre seulement leur subsistance et leurs commo­dités. Les habitants de l’un doivent toujours nécessairement jouir d’une beaucoup plus grande quantité de subsistances que ce que leurs propres terres pourraient leur rap­porter dans l’état actuel de leur culture. Les habitants de l’autre doivent nécessai­rement jouir d’une quantité de subsistances fort au-dessous du produit de leurs terres.
Avec toutes ses imperfections, néanmoins, ce système est peut-être, de tout ce qu’on a encore publié sur l’économie politique, ce qui se rapproche le plus de la vérité, et sous ce rapport il mérite bien l’attention de tout homme qui désire faire un examen sérieux des principes d’une science aussi importante. Si, en représentant le travail employé à la terre comme le seul travail productif, les idées qu’il veut donner des choses sont peut-être trop étroites et trop bornées, cependant, en représentant la richesse des nations comme ne consistant pas dans ces richesses non consommables d’or et d’argent, mais dans les biens consommables reproduits annuellement par le travail de la société, et en montrant la plus parfaite liberté comme l’unique moyen de rendre cette reproduction annuelle la plus grande possible, sa doctrine paraît être, à tous égards, aussi juste qu’elle est grande et généreuse. Ses partisans sont très-nom­breux ; et comme les hommes se plaisent aux paradoxes et sont jaloux de paraître comprendre ce qui passe l’intelligence du vulgaire, le paradoxe qu’il soutient sur la nature non productive du travail des manufactures n’a peut-être pas peu contribué à accroître le nombre de ses admirateurs.
Ils formaient, il y a quelques années, une secte assez considérable, distinguée en France, dans la république des lettres, sous le nom d’Économistes. Leurs travaux ont certainement rendu quelques services à leur pays, non-seulement en appelant la discussion générale sur plusieurs matières qui n’avaient été, jusque-là, guère appro­fondies, mais encore en obtenant à un certain point, par leur influence, un traitement plus favorable pour l’agriculture de la part de l’administration publique. Aussi est-ce par une suite de leurs représentations que l’agriculture de France s’est vue délivrée de plusieurs des oppressions sous lesquelles elle gémissait auparavant. On a prolongé, de neuf années à vingt-sept, le terme pour lequel il est permis de passer un bail qui puisse avoir exécution contre tout acquéreur ou futur propriétaire d’une terre. Les anciens règlements provinciaux, qui gênaient le transport du blé d’une province du royaume à l’autre, ont été entièrement supprimés, et la liberté de l’exporter à tous les pays étrangers a été établie comme loi commune du royaume dans tous les cas ordinaires. Les écrivains de cette secte, dans leurs ouvrages, qui sont très-nombreux et qui traitent, non-seulement de ce qu’on nomme proprement l’économie politique, ou de la nature et des causes de la richesse des nations, mais encore de toute autre branche du système du gouvernement civil, suivent tous, dans le fond et sans aucune variation sensible, la doctrine de M. Quesnay. En conséquence, il y a peu de variété dans la plupart de leurs ouvrages. On trouvera l’exposition la plus claire et la mieux suivie de cette doctrine dans un petit livre écrit par M. Mercier de la Rivière, ancien intendant de la Martinique, intitulé : L’Ordre naturel et essentiel des sociétés politiques. L’admiration de la secte entière des économistes pour leur maître, qui était lui-même un homme d’une grande simplicité et d’une grande modestie, ne le cède en rien à celle que les philosophes de l’antiquité conservaient pour les fondateurs de leurs systèmes respectifs. « Depuis l’origine du monde », dit un auteur très-habile et très-aimable, le marquis de Mirabeau, « il y a eu trois grandes découvertes qui ont fourni aux sociétés politiques leur principale solidité, indépendamment de beaucoup d’autres découvertes qui ont contribué à les orner et à les enrichir. La première, c’est l’inven­tion de l’écriture, qui seule donne au genre humain la faculté de transmettre, sans altéra­tions, ses lois, ses conventions, ses annales et ses découvertes. La seconde est l’invention de la monnaie, le lien commun qui unit ensemble toutes les sociétés civilisées. La troisième, qui est le résultat des deux autres, mais qui les complète, puisqu’elle porte leur objet à sa perfection, est le Tableau économique, la grande découverte qui fait la gloire de notre siècle, et dont la postérité recueillera les fruits. 
Si l’économe politique des nations de l’Europe moderne a été plus favorable aux manufactures et au commerce étranger, qui constituent l’industrie des villes, qu’à l’agri­culture, qui constitue l’industrie des campagnes, celle d’autres nations a suivi un plan différent et a favorisé l’agriculture de préférence aux manufactures et au com­merce étranger.
La politique de la Chine favorise l’agriculture de préférence à toutes les autres industries. À la Chine, la condition d’un laboureur est, dit-on, autant au-dessus de celle d’un artisan, que dans la plupart des contrées de l’Europe la condition d’un arti­san est au-dessus de celle du laboureur. À la Chine, la grande ambition d’un hom­me est de se procurer la possession de quelque petit morceau de terre, soit en propriété, soit à bail ; et l’on dit que, dans ce pays, on obtient des baux à des condi­tions très-modérées, et que la jouissance du fermier y est assez assurée. Les Chinois font très-peu de cas du commerce étranger. Votre misérable commerce ! disaient ordinaire­ment, pour le désigner, les mandarins de Pékin dans leurs conversations avec M. de Lange, envoyé de Russie[7]. Les Chinois ne font que peu ou point de commerce étran­ger par eux-mêmes et dans leurs propres bâtiments, si ce n’est avec le japon, et ce n’est même que dans deux ou trois ports de leur royaume qu’ils admettent les vais­seaux des nations étrangères. Par conséquent, le commerce étranger se trouve de toute manière, à la Chine, resserré dans un cercle plus étroit que celui dans lequel il s’étendrait naturellement si les Chinois lui eussent laissé plus de liberté, soit dans leurs propres vaisseaux, soit dans ceux des nations étrangères[8].
Les ouvrages de manufacture contenant souvent une grande valeur sous un petit volume et pouvant, par cette raison, se transporter d’un pays à l’autre à moins de frais que la plupart des espèces de produit brut, sont, dans presque tous les pays, l’aliment principal du commerce étranger. En général aussi, dans des pays moins étendus et moins favorablement disposés pour le commerce intérieur que ne l’est la Chine, les manufactures ont besoin d’être soutenues par le commerce étranger. Sans un marché étranger fort étendu, elles ne pourraient guère prospérer, soit dans les pays dont le territoire est trop borné pour fournir un marché intérieur un peu considérable, soit dans ceux où la communication d’une province à l’autre est trop peu facile pour permettre aux marchandises d’un endroit de jouir de la totalité du marché intérieur que le pays pourrait fournir. Il ne faut pas oublier que la perfection de l’industrie ma­nu­facturière dépend entièrement de la division du travail ; et comme on l’a déjà fait voir, c’est l’étendue du marché qui règle nécessairement à quel degré peut être portée la division du travail dans un genre quelconque de manufacture[9]. Or, la grande étendue de l’empire de la Chine, la multitude immense de ses habitants, la variété de ses différentes provinces et, par conséquent, la grande variété de ses productions et la facilité des communications établies par la navigation entre la plus grande partie de ces provinces, rendent le marché intérieur de ce pays d’une si vaste étendue, qui est seul suffisant pour soutenir de très-grandes manufactures et admettre des subdivisions de travail très-considérables. Le seul marché intérieur de la Chine n’est peut-être pas fort inférieur en étendue au marché de tous les différents pays de l’Europe pris ensem­ble. Cependant, un commerce étranger plus étendu qui à ce vaste marché intérieur ajouterait encore le marché étranger de tout le reste du monde, surtout si une grande partie de ce commerce se faisait sur des vaisseaux nationaux, ne saurait guère man­quer d’augmenter de beaucoup les progrès des manufactures de la Chine et d’y perfec­tionner singulièrement, dans ce genre d’industrie, la puissance productive du travail. Avec une navigation plus étendue, la Chine en viendrait naturellement à apprendre l’emploi et la construction de toutes les différentes machines dont on fait usage dans les autres pays ; elle viendrait à s’instruire de tous les autres procédés utiles de l’art et de l’industrie qui sont mis en pratique dans toutes les diverses parties du monde. La conduite que suivent actuellement les Chinois ne leur offre guère d’occasions de se perfectionner par l’exemple de quelque autre nation, si ce n’est par celui de la nation japonaise. 
La politique de l’ancienne Égypte et celle du gouvernement des Gentous dans l’Indostan ont aussi, à ce qu’il semble, favorisé l’agriculture de préférence à toutes les autres industries.
Dans l’ancienne Égypte, ainsi que dans l’Indostan, la nation entière était divisée en différentes castes ou tribus, dont chacune était bornée, de père en fils, à un emploi ou classe d’emplois particuliers. Le fils d’un prêtre était nécessairement prêtre ; le fils d’un soldat, soldat ; le fils d’un laboureur, laboureur ; le fils d’un tisserand, tisserand ; le fils d’un tailleur, tailleur, etc. Dans l’un et l’autre de ces pays, la caste des prêtres tenait le premier rang, et celle des guerriers venait ensuite ; et chez ces deux peuples, la caste des fermiers et des laboureurs était supérieure à celle des marchands et des manufacturiers.
Le gouvernement de ces deux pays donnait une attention particulière aux intérêts de l’agriculture. Les ouvrages exécutés par les anciens souverains de l’Égypte, pour opérer une distribution convenable des eaux du Nil, ont été fameux dans l’antiquité, et les vestiges des ruines de quelques-unes de ces constructions font encore aujourd’hui l’admiration des voyageurs. Les travaux du même genre faits par les anciens sou­ve­rains de l’Indostan, pour distribuer avantageusement les eaux du Gange aussi bien que celles de beaucoup d’autres fleuves, paraissent n’avoir pas eu moins de grandeur, quoiqu’ils aient eu moins de célébrité. Aussi ces deux pays, quoique accidentellement sujets à des disettes, ont été remarques pour leur grande fertilité. Malgré l’immense population de l’un et de l’autre, ils étaient cependant, dans les années d’abondance ordinaire, en état d’exporter chez leurs voisins de grandes quantités de grains.
Par superstition, les anciens Égyptiens avaient de l’éloignement pour la mer, et comme la religion des Gentous ne permet pas à ceux qui la suivent d’allumer du feu sur l’eau ni, par conséquent, d’y préparer des aliments, elle leur défend, par le fait, tout voyage de longs cours par mer. Les Égyptiens et les Indiens ont dû se trouver nécessairement, pour l’exportation de leur surplus de produit, dans la dépendance de la navigation des autres nations ; et comme cette dépendance a dû resserrer leur mar­ché, elle a nécessairement par là découragé l’accroissement de ce surplus de produit. Elle a dû encore décourager l’accroissement du produit manufacturé, plus même que du produit brut. Les ouvrages de manufacture exigent un marché beaucoup plus éten­du que les parties les plus importantes du produit brut de la terre. Un seul cordonnier fera plus de trois cents paires de souliers dans une année, et sa famille ne lui en usera peut-être pas six paires. À moins donc qu’il n’ait pour pratiques au moins cinquante familles comme la sienne, il ne pourra pas débiter tout le produit de son travail. Les classes les plus nombreuses d’artisans, dans un grand pays, ne font guère plus d’un sur cinquante ou d’un sur cent, dans le nombre total des familles de ce pays ; mais le nombre des gens employés à l’agriculture, dans de grands pays tels que la France et l’Angleterre, a été supputé par quelques auteurs s’élever à la moitié, par d’autres au tiers de la population totale du pays, et je ne sache pas qu’aucun écrivain l’ait évalué au-dessous du cinquième[10]. Or, comme le produit de l’agriculture en France et en Angleterre est, pour la plus grande partie, consommé dans le pays, il faut, d’après ces calculs, pour chaque personne occupée à cet emploi, la pratique seulement d’une, de deux ou au plus de quatre familles comme la sienne, pour pouvoir débiter la totalité du produit de son travail. Par conséquent, au milieu du découragement qui résulte d’un marché très-borné, l’agriculture peut se soutenir beaucoup mieux que ne le peu­vent les manufactures. À la vérité, dans l’ancienne Égypte, ainsi que dans l’Indostan, le désavantage de manquer de marchés étrangers se trouvait compensé, à un certain point, par les avantages d’une quantité de moyens de navigation intérieure, qui ouvraient de la manière la plus utile et la plus commode, à chaque partie du produit des divers districts, le marché national dans sa plus parfaite étendue. Le vaste territoire de l’Indostan faisait de ce pays un immense marché intérieur, suffisant pour soutenir une multitude de manufactures diverses. Mais le territoire borné de l’ancien­ne Égypte, qui n’a jamais égalé celui de l’Angleterre en étendue, doit y avoir formé, dans tous les temps, un marché intérieur trop resserré pour supporter une grande variété de manufactures. Aussi le Bengale, la province de l’Indostan qui commu­nément exporte la plus grande quantité de riz, a toujours été plus remarquable pour l’exportation d’une multitude de divers ouvrages de manufacture, que pour celle de ses grains. Au contraire, l’Égypte ancienne, quoiqu’elle ait exporté quelques articles de manufacture, tels que ses belles toiles de lin et certains autres objets, a toujours été surtout renommée pour sa grande exportation de grains. Elle a été longtemps le grenier de l’empire romain.
Les souvenirs de la Chine, ceux de l’ancienne Égypte et ceux des différents royau­mes entre lesquels l’Indostan a été partagé à diverses époques, ont toujours tiré tout leur revenu, ou la plus grande partie, sans comparaison, de leur revenu, de quelque espèce d’impôt foncier ou de redevance foncière. Cet impôt foncier ou redevance foncière consistait, comme la dîme en Europe, en une portion déterminée, un cin­quième, dit-on, du produit de la terre, qui était livré en nature ou bien qu’on payait en argent d’après une évaluation fixe ; par conséquent, cet impôt variait d’une année à l’autre, suivant toutes les variations que le produit venait à essuyer. Dès lors il était naturel que ces souverains donnassent une attention particulière aux intérêts de l’agriculture, puisque, de sa prospérité ou de son dépérissement, dépendait si directe­ment l’accroissement ou la diminution annuelle de leur propre revenu.
La politique de Rome et celle des anciennes républiques de la Grèce, tout en ho­no­­rant l’agriculture plus que les manufactures et le commerce étranger, semblent cepen­dant s’être bien moins attachées à donner aucun encouragement formel et réflé­chi à la première de ces industries qu’à décourager les deux autres.
Dans plusieurs des anciens États de la Grèce, le commerce étranger était totale­ment prohibé, et dans plusieurs autres les occupations d’artisan et de manufacturier étaient réputées nuire à la force et à l’agilité du corps, parce que, l’empêchant de se livrer habituellement aux exercices militaires et gymnastiques, elles le rendaient plus ou moins incapable d’endurer les fatigues et d’affronter les périls de la guerre. De telles occupations étaient censées ne convenir qu’à des esclaves, et on défendait aux citoyens de s’y adonner. Dans les États même où cette défense n’eut pas lieu, tels qu’Athènes et Rome, le peuple était, par le fait, exclu de tous les métiers qui sont main­tenant exercés, pour l’ordinaire, par la dernière classe des habitants des villes. Ces métiers, à Rome et à Athènes, étaient remplis par les esclaves des riches, qui les exerçaient pour le compte de leurs maîtres, et la richesse, la puissance et la protection de ceux-ci mettaient le pauvre libre presque dans l’impossibilité de trouver le débit de son produit, quand ce produit venait en concurrence avec celui des esclaves du riche.
Mais les esclaves sont rarement inventifs, et les procédés les Plus avantageux à l’industrie, ceux qui facilitent et abrègent le travail, soit en fait de machines, soit en fait d’arrangement et de distribution de tâches, ont tous été inventés par des hommes libres. Si même un esclave s’avisait de proposer quelque moyen de ce genre, le maître serait très-disposé à regarder sa proposition comme suggérée par la paresse et par un désir d’épargner sa peine aux dépens du maître. Le pauvre esclave, au lieu de récom­pense, n’aurait vraisemblablement qu’une fort mauvaise réception à attendre, peut-être même quelque châtiment. Par conséquent, dans les manufactures qui vont par le moyen d’esclaves, il faut, en général, employer plus de travail pour exécuter la même quantité d’ouvrage, que dans celles qui vont par le moyen d’hommes libres. Par cette raison, l’ouvrage des manufactures de cette première espèce a dû, en général, être plus cher que celui des autres. M. de Montesquieu observe que les mines de la Hongrie, sans être plus riches que les mines de Turquie de leur voisinage, ont toujours été exploitées à moins de frais et, par conséquent, avec plus de profit. Les mines de la Turquie sont exploitées par des esclaves, et les bras de ces esclaves sont les seules machines que les Turcs se soient jamais avisés d’y employer. Les mines de la Hongrie sont exploitées par des hommes libres qui font usage d’une grande quantité de machines pour faciliter et abréger leur travail. D’après le peu que nous connaissons des prix des ouvrages de manufacture dans le temps des Grecs et des Romains, il parait que ceux du genre le plus fin étaient d’une cherté excessive. La soierie se vendait pour son poids d’or. Dans ces temps, à la vérité, ce n’était pas un ouvrage de fabrique européenne ; et comme elle était toute apportée des Indes orientales, la distance du transport peut, jusqu’à un certain point, rendre raison de l’énormité du prix. Cependant le prix qu’une dame payait quelquefois, dit-on, pour une pièce de très-belle toile, paraît avoir été tout aussi exorbitant ; et comme la toile venait toujours d’une fabrique européenne, ou, au plus loin, d’une manufacture d’Égypte, on ne peut rendre raison de l’énormité du prix que par la grande dépense de travail mise à cet ouvrage, et cette grande dépense de travail, à son tour, ne peut avoir eu d’autre cause que l’imperfection des machines dont on faisait usage. Le prix des belles étoffes de laine, quoiqu’il ne soit pas tout à fait aussi prodigieux, paraît cependant avoir été fort au-dessus des prix actuels.
Pline rapporte[11] que des draps teints d’une certaine façon coûtaient 100 deniers romains, ou 3 livres 6 sous 8 deniers la livre pesant. D’autres, teints d’une autre façon, coûtaient 1,000 deniers la livre, ou 33 livres 6 sous 9 deniers. Il faut se rappeler que la livre romaine ne contenait que douze de nos onces, avoir du poids. Il est vrai que ce haut prix, à ce qu’il semble, était dû principalement à la teinture. Mais si les draps, par eux-mêmes, n’eussent pas été beaucoup plus chers qu’aucun de ceux qu’on fabrique aujourd’hui, on n’aurait sûrement pas fait pour eux la dépense d’une teinture aussi précieuse ; la disproportion aurait été trop forte entre la valeur de l’accessoire et celle du principal. Mais ce qui passe toute croyance, c’est ce que rapporte le même auteur[12] du prix de certains triclinaires, espèces de coussins de laine dont on se servait dans les festins pour s’appuyer, quand on était couchés sur les lits qui entouraient la table ; suivant lui, quelques-uns de ces coussins auraient coûté plus de 30,000, d’autres plus de 300,000 livres[13], et il ne dit pas d’ailleurs que cet incroyable prix vint de la teinture. Le docteur Arbuthnot observe qu’il paraît y avoir eu, dans les anciens temps, beaucoup moins de variété dans l’habillement des gens du bon ton de l’un et de l’autre sexe, qu’il n’y en a dans les temps modernes ; et ce qui confirme cette observation, c’est le peu de diversité qui se trouve dans le costume des statues antiques. Il en infère que leur habillement était au total moins dispendieux que le nôtre, mais la conclusion ne paraît pas juste. Quand la dépense d’un habillement recherché est très-grande, il doit y avoir fort peu de variété dans les habits ; mais lorsqu’au moyen de la perfection que l’industrie et l’art des manufactures acquièrent dans leurs facultés productives, la dépense d’un habit de goût vient à être fort modique, alors naturellement les modes seront très-variées et les habits très-multipliés. Les riches ne pouvant plus se distin­guer par la dépense d’un habit, ils tâcheront naturellement de le faire par la multitude et la variété[14]. 
On a déjà observé que la branche la plus étendue et la plus importante du com­merce d’une nation était le commerce établi entre les habitants de la ville et ceux de la campagne. Les habitants de la ville tirent de la campagne le produit brut qui constitue à la fois la matière première de leur travail et le fonds de leur subsistance, et ils payent ce produit brut en renvoyant à la campagne une certaine portion de ce produit, manufacturée et préparée pour servir immédiatement à la consommation et à l’usage. Le commerce qui s’établit entre ces deux différentes classes du peuple consiste, en dernier résultat, dans l’échange d’une certaine quantité de produit brut contre une cer­taine quantité de produit manufacturé. Par conséquent, plus celui-ci est cher, plus l’autre sera à bon marché ; et tout ce qui tend, dans un pays, à élever le produit du prix manufacturé, tend à abaisser celui du produit brut de la terre et, par là, à décourager l’agriculture. Plus sera petite la quantité de produit manufacturé qu’une quantité donnée de produit brut (ou, ce qui revient au même, le prix d’une quantité donnée de produit brut), sera en état d’acheter, plus sera petite la valeur échangeable de cette quantité donnée de produit brut, et moins alors le propriétaire se sentira encouragé à augmenter la quantité de ce produit par des améliorations sur la terre, ou le fermier par une culture plus soignée. D’ailleurs, tout ce qui tend à diminuer dans un pays le nombre des artisans et des manufacturiers, tend à diminuer le marché intérieur le plus important de tous les marchés pour le produit brut de la terre, et tend par là à décou­rager encore l’agriculture.
Par conséquent, ces systèmes, qui, donnant à l’agriculture la préférence sur tous les autres emplois, cherchent à la favoriser en imposant des gênes aux manufactures et au commerce étranger, agissent contre le but même qu’ils se proposent et décou­ragent indirectement l’espèce même d’industrie qu’ils prétendent encourager. À cet égard, peut-être, ils sont encore plus inconséquents que le système mercantile lui-même. Celui-ci, en encourageant les manufactures et le commerce étranger de préfé­rence à l’agriculture, empêche une certaine portion du capital de la société d’aller au soutien d’une espèce d’industrie plus avantageuse, pour porter ce capital au soutien d’une autre qui ne l’est pas autant ; mais au moins encourage-t-il réellement, en dernier résultat, l’espèce d’industrie dont il a intention de favoriser les progrès, tandis qu’au contraire ces systèmes agricoles finissent réellement par jeter un véritable décou­ragement sur leur espèce favorite d’industrie. 
C’est ainsi que tout système qui cherche ou, par des encouragements extraor­di­naires, à attirer vers une espèce particulière d’industrie une plus forte portion du capi­ta1 de la société que celle qui s’y porterait naturellement, ou, par des entraves extraor­di­naires, à détourner forcément une partie de ce capital d’une espèce particulière d’in­dus­trie vers laquelle elle irait sans cela chercher un emploi, est un système réellement subversif de l’objet même qu’il se propose comme son principal et dernier terme. Bien loin de les accélérer, il retarde les progrès de la société vers l’opulence et l’agran­dis­sement réels ; bien loin de l’accroître, il diminue la valeur réelle du produit annuel des terres et du travail de la société.
Ainsi, en écartant entièrement tous ces systèmes ou de préférence ou d’entraves, le système simple et facile de la liberté naturelle vient se présenter de lui-même et se trouve tout établi. Tout homme, tant qu’il n’enfreint pas les lois de la justice, demeure en pleine liberté de suivre la route que lui montre son intérêt, et de porter où il lui plaît son industrie et son capital, concurremment avec ceux de toute autre classe d’hommes. Le souverain se trouve entièrement débarrassé d’une charge qu’il ne pour­rait essayer de remplir sans s’exposer infailliblement à se voir sans cesse trompé de mille manières, et pour l’accomplissement convenable de laquelle il n’y a aucune sagesse humaine ni connaissance qui puissent suffire, la charge d’être le surintendant de l’industrie des particuliers, de la diriger vers les emplois les mieux assortis à l’intérêt général de la société. Dans le système de la liberté naturelle, le souverain n’a que trois devoirs à rem­plir ; trois devoirs, à la vérité, d’une haute importance, mais clairs, simples et à la por­tée d’une intelligence ordinaire. Le premier, c’est le devoir de défendre la société de tout acte de violence ou d’invasion de la part des autres sociétés indépendantes. Le second, c’est le devoir de protéger, autant qu’il est possible, chaque membre de la société contre l’injustice ou l’oppression de tout autre membre, ou bien le devoir d’établir une administration exacte de la justice. Et le troisième, c’est le devoir d’ériger et d’entretenir certains ouvrages publics et certaines institutions que l’intérêt privé d’un particulier ou de quelques particuliers ne pourrait jamais les porter à ériger ou à entretenir, parce que jamais le profit n’en rembourserait la dépense à un particulier ou à quelques particuliers, quoiqu’à l’égard d’une grande société ce profit fasse beaucoup plus que rembourser les dépenses. 
Ces différents devoirs du souverain supposent nécessairement, pour les remplir convenablement, une certaine dépense ; et cette dépense aussi exige nécessairement un certain revenu pour la soutenir. Ainsi, dans le livre suivant, je tâcherai d’exposer, premièrement, quelles sont les dépenses nécessaires du souverain ou de la république ; quelles de ces dépenses doivent être défrayées par une contribution générale de la société entière, et quelles autres doivent l’être par la contribution d’une partie seule­ment de la société ou de quelques-uns de ses membres en particulier. Secondement, quelles sont les différentes méthodes de faire contribuer la société entière à l’acquit des dépenses qui sont à la charge de la société entière, et quels sont les principaux avantages et inconvénients de chacune de ces méthodes. Et troisièmement, quels sont les motifs et les causes qui ont amené presque tous les gouvernements modernes à aliéner et hypothéquer quelque partie de ce revenu ou à contracter des dettes, et quels ont été les effets de ces dettes sur la richesse réelle de la société, sur le produit annuel de ses terres et de son travail. Ainsi, le livre suivant se divisera naturellement en trois chapitres.
 
 
 
↑ Les tables économiques de M. Quesnay sont un essai malheureux de vouloir appliquer aux théories de l’économie la méthode employée dans les mathématiques. Ces deux sciences sont parfaitement distinctes : l’une est une science morale, l’autre traite des rapports des quantités fixes et déterminées. Les proportions suivant lesquelles, d’après M. Quesnay, les produits du sol se distribuent dans les différentes classes de la population, sont tout à fait conjecturales. Il n’a même pas essayé d’établir les bases de cette division tout imaginaire ; et quelle valeur peut-on attacher à des conclusions tirées de faits aussi arbitrairement posés ? Un raisonnement peu exact manquera toujours d’intérêt ; et quelque justes que les conclusions de M. Quesnay, tendant à la liberté du commerce, puissent être, leur valeur sera toujours affaiblie par la considération qu’elles ne sont pas basées sur des fondements solides. Buchanan.
↑
L’origine de cette erreur est dans l’idée que Quesnay et les économistes s’étaient faite de la nature et des causes du revenu. Ils avaient remarqué que les marchands et les fabricants ne faisaient que rentrer en quelque sorte dans leurs capitaux, y compris les salaires et les bénéfices ; tandis que l’industrie des cultivateurs leur offrait les mêmes salaires et bénéfices, outre le produit additionnel ou produit net, qui constitue les profits du propriétaire. Cette circonstance a fait croire aux économistes que l’agriculture était le seul emploi réellement productif, c’est-à-dire le seul qui fournit une quantité de produits supérieure à la consommation opérée par le travail. Et c’est sur cette hypothèse qu’ils ont construit leur théorie. Mais, s’ils avaient mieux observé les circonstances qui créent, et qui en même temps limitent et déterminent ces profits, ils n’en auraient pas tiré ces conclusions. Ils auraient vu alors que le sol ne donne pas de profit ou produit net, quand les meilleures terres seules sont mises en culture ; que ce produit n’est en définitive que la conséquence du décroissement de la fertilité du sol et de l’obligation dans laquelle nous sommes de recourir à des terres d’une qualité inférieure pour obtenir les provisions de nourriture nécessaires à l’accroissement de la population ; qu’il dépend de l’étendue des terres inférieures mises en culture, qu’il augmente à mesure qu’on les cultive, et qu’il diminue à mesure qu’on les laisse en jachère.
A. Smith n’a pas assez tenu compte de cette vérité, et c’est pour cela que sa réfutation du système des économistes est loin d’être satisfaisante. Mac Culloch.
↑
A. Smith diffère ici de très-peu de la théorie des économistes ; il prend seulement le mot improductif dans une autre acception ; il ne l’applique pas au travail de ceux qui ne produisent pas plus qu’ils ne consomment.
Si les économistes changeaient seulement ce terme dans leur théorie, A. Smith se trouverait d’accord avec eux, puisque, comme eux, il appelle le travail du fabricant et de l’artisan improductif, en tant qu’il n’ajoute pas aux richesses du pays. Buchanan.
Garnier a essayé de démontrer dans une longue note la vérité du système des économistes. Cette dissertation n’aurait plus aujourd’hui même l’intérêt d’une pièce, de controverse. Le système est jugé sans appel. A. B.
↑ Selon Mac Culloch et l’école à laquelle il appartient, la différence que Smith a essayé d’établir entre le travail des domestiques et celui des artisans, est aussi imaginaire que celle que les économistes ont voulu établir entre le travail des agriculteurs et celui des artisans et marchands. A. B.
↑
A. Smith, bien que d’accord avec les économistes sur le point capital, à savoir sur la prééminence de l’agriculture sur les autres industries, parait indécis et faible quand il combat les autres parties de leurs doctrines. Dans son raisonnement pour prouver que le travail de l’artisan est un travail productif, il admet, ainsi que les économistes, que ce travail ne puisse jamais augmenter le capital national ; mais il soutient en même temps qu’il est plus productif que celui des domestiques, qui consomment sans produire.
Les économistes peuvent facilement admettre ce dernier point, sans renoncer pour cela à leur dogme favori, qui ne reconnaît que le sol comme source unique de tout revenu et comme seule matière imposable. Il est singulier qu’Adam Smith, qui a si bien expliqué comment la division du travail amène l’augmentation du capital national, n’ait pas, dans sa doctrine, mieux attaqué ces idées des économistes ; il aurait, sans aucun doute, donné la meilleure réfutation de leur doctrine. Par l’amélioration dé l’industrie, par suite de la division du travail et de l’emploi des machines, les produits fabriqués sont devenus à très-bon marché, ce qui procure de l’avantage à la communauté. Mais c’est précisément la circonstance de ce bon marché qui diminue la Valeur des manufactures aux yeux des économistes, toujours embarrassés dans leurs idées d’un surcroit de production ; ils ne s’aperçoivent pas que, par la raison même que les manufactures ne donnent point ce surcroît, ou, en d’autres termes, par la raison même qu’elles produisent a bon marché, elles doivent tourner à l’avantage de la communauté. Si leur produit net était plus considérable, des particuliers pourraient bien s’enrichir, mais la communauté en tirerait des bénéfices moins grands. Buchanan.
↑ Voy. liv. I, chap. i.
↑ Voyez le Journal de M. de Lange, dans les Voyages de Bell, vol. II, pages 258, 276 et 293. Note de l’auteur.
↑ L’auteur exagère l’aversion des Chinois pour le commerce extérieur ; ils ne trafiquent pas seulement avec le Japon, mais avec toutes les îles indiennes, et leurs relations tendent de plus en plus à s’agrandir. D’ailleurs, nous ne savons s’il est permis d’alléguer ce qui se passe en ce pays, lequel est encore une terre inconnue. Il est probable que la prochaine ouverture des ports désignés dans le traité d’août 1842 avec les Anglais permettra désormais aux Européens de mieux étudier ce pays. A. B.
↑ Voy. liv. I, chap. iii.
↑ Conformément au cens de 1831, sur 3,414,175 familles, en Grande-Bretagne, 961,134 seulement étaient employées à l’agriculture. En Irlande, au contraire, sur 1,385,066 familles, 884,339 vivaient de la culture du sol. Mac Culloch.
↑ Liv. IX, chap. xxix.
↑ Liv. VIII, chap. xvviii.
↑ Le texte de Pline, selon les meilleures leçons, et dans l’édition dite Variorum, porte quadragies ; ce qui répond, d’après les calculs du docteur Arbuthnot, qui a adopté cette leçon, à 32,291 livres 13 schellings 4 deniers sterling, et ce qui est déjà bien assez incroyable. Mais Budée s’est avisée de lire quadringenties, ce qui d’après les mêmes calculs, donnerait 322,916 livres 13 schellings 4 deniers sterling, c’est-à-dire environ 7 à 8 millions de francs.
↑ Les calculs sur les prix des objets dans l’antiquité ont été empruntées par Adam Smith aux tables du docteur Arbuthnot, qui ne méritent pas grande confiance. Mac Cullcoh.

%%%%%%%%%%%%%%%%%%%%%%%%%%%%%%%%%%%%%%%%%%%%%%%%%%%%%%%%%%%%%%%%%%%%%%%%%%%%%%%%
%                                                                              %
%                                   Livre V                                    %
%                                                                              %
%%%%%%%%%%%%%%%%%%%%%%%%%%%%%%%%%%%%%%%%%%%%%%%%%%%%%%%%%%%%%%%%%%%%%%%%%%%%%%%%

\part{Du revenu du Souverain ou de la République}
\markboth{Du revenu du Souverain ou de la République}{}

%%%%%%%%%%%%%%%%%%%%%%%%%%%%%%%%%%%%%%%%%%%%%%%%%%%%%%%%%%%%%%%%%%%%%%%%%%%%%%%%
%                                  Chapitre 1                                  %
%%%%%%%%%%%%%%%%%%%%%%%%%%%%%%%%%%%%%%%%%%%%%%%%%%%%%%%%%%%%%%%%%%%%%%%%%%%%%%%%

\chapter{Des dépenses à la charge du Souverain et de la République}
\markboth{Des dépenses à la charge du Souverain et de la République}{}

SECTION PREMIÈRE
Des dépenses qu’exige la défense commune[1].

Le premier des devoirs du Souverain, celui de protéger la société contre la vio­lence et l’invasion d’autres sociétés indépendantes, ne peut se remplir qu’à l’aide d’une force militaire ; mais, dans les différents états de la société, dans ses différentes pério­des d’avancement, la dépense à faire tant pour préparer cette force militaire, en temps de paix, que pour l’employer en temps de guerre, se trouve être très-différente.
Chez les peuples chasseurs, ce qui est le premier degré et le plus informe de l’état social, tel que nous le trouvons parmi les naturels de l’Amérique septentrionale, tout homme est guerrier aussi bien que chasseur. Quand il va à la guerre ou pour défendre sa tribu, ou pour la venger des injures qu’elle a reçues de quelque autre tribu, il subsiste de son travail, comme quand il vit chez lui. Sa société, car dans cet état de choses il n’y a proprement ni souverain ni république, sa société n’a aucune dépense à faire soit pour le disposer à se rendre au champ de bataille, soit pour l’entretenir quand il y est.
Chez les peuples pasteurs, ce qui est un état de société plus avancé, tel que nous le voyons chez les Tartares et les Arabes, tout homme est de même guerrier. Ces nations, pour l’ordinaire, n’ont point d’habitations fixes, mais vivent sous des tentes et dans des espèces de chariots couverts qui se transportent aisément d’un lieu dans un autre. La tribu tout entière ou la nation change de situation selon les différentes saisons de l’année ou d’après d’autres circonstances. Quand ses troupeaux ont con­som­mé le pâturage d’une partie du pays, elle les mène à une autre, et de là à une troisième. Dans le temps de la sécheresse, elle descend sur le bord des rivières ; dans les temps humides, elle gagne les hauteurs. Quand une telle nation s’en va à la guerre, les guerriers ne laissent pas leurs troupeaux à la garde trop faible de leurs vieillards, de leurs femmes et de leurs enfants ; et d’un autre côté, les vieillards, les femmes et les enfants ne voudraient pas rester en arrière sans défense ai moyen de subsister. D’ailleurs, toute la nation, habituée à une vie errante, même en temps de paix, se met aisément en campagne en temps de guerre. Soit qu’elle marche comme armée, soit qu’elle chemine comme troupe de pasteurs, le genre de vie est à peu près le même, quoique l’objet qu’elle se propose soit très-différent. Ainsi ils vont tous ensemble à la guerre, et chacun fait du mieux qu’il peut. Chez les Tartares, on a vu souvent les femmes elles-mêmes se mêler à la bataille. S’ils sont victorieux, tout ce qui appartient à la tribu ennemie est le prix de la victoire ; mais s’ils sont vaincus, tout est perdu ; non-seulement les troupeaux, mais même les femmes et les enfants deviennent la proie du vainqueur. La plus grande partie même de ceux qui survivent à leur défaite sont obli­gés de se soumettre à lui pour pouvoir subsister. Le reste, pour l’ordinaire, se dissipe et se disperse dans le désert.
La vie ordinaire d’un Tartare ou d’un Arabe, ses exercices accoutumés, le prépa­rent à la guerre. Les passe-temps habituels de gens qui vivent en plein air sont de s’exercer à la course et à la lutte, de jouer du bâton, de lancer le javelot, de tirer de l’arc, et tous ces jeux sont des images de la guerre. Aujourd’hui, lorsqu’un Arabe ou un Tartare va en guerre, il subsiste de ses troupeaux qu’il mène avec lui, tout comme il fait en temps de paix. Son chef ou souverain, car ces nations ont toutes leur chef ou leur souverain, n’a aucune espèce de dépense à faire pour le disposer à se rendre au champ de bataille, et quand il y est rendu, l’espoir du pillage est la seule paie qu’il lui faut, et il n’en attend pas d’autre.
Une armée de chasseurs ne peut guère excéder deux ou trois cents hommes. La subsistance précaire qu’offre la chasse ne permettrait guère à un plus grand nombre de rester assemblés pendant un temps un peu long. Une armée de pasteurs, au contraire, peut quelquefois monter à deux ou trois mille hommes. Tant que rien n’arrête leurs progrès, ils peuvent aller d’un canton dont ils ont consommé l’herbe à un autre qui se trouve intact. Il semble qu’il n’y ait presque pas de bornes au nombre d’hommes qui peuvent ainsi marcher ensemble. Une nation de chasseurs ne peut jamais être redou­table pour les nations civilisées de son voisinage. Une nation de pasteurs peut l’être. Il n’y a rien de plus misérable qu’une guerre contre les Indiens dans l’Amérique septen­tri­onale ; il n’y a au contraire rien de plus terrible qu’une invasion de Tartares, telle qu’il en est souvent arrivé en Asie. L’expérience de tous les temps a vérifié l’opinion de Thucydide, que l’Europe et l’Asie ensemble ne pourraient résister aux Scythes réunis. Les habitants de ces plaines immenses, mais ouvertes de toutes parts, qui composent la Scythie ou la Tartarie, se sont souvent unis sous le commandement du chef de quelque horde ou tribu conquérante, et cette union a toujours été signalée par la ruine et la dévastation de l’Asie. Les naturels des déserts inhabitables de l’Arabie, cette autre grande nation de pasteurs, ne se sont jamais réunis qu’une fois, sous Maho­met et ses successeurs immédiats. Leur union, qui fut plutôt l’effet de l’enthousiasme religieux que celui de la conquête, a été signalée de la même manière. Si les peuples chasseurs de l’Amérique deviennent jamais peuples pasteurs, leur voisinage sera beaucoup plus dangereux pour les colonies européennes qu’il ne l’est à présent.
Dans un état de société encore plus avancé, chez les nations agricoles, qui n’ont que peu de commerce étranger, et qui ont, pour tout produit de manufacture, ces ouvrages grossiers et ces ustensiles de ménage que chaque famille fait elle-même pour son usage particulier, tout homme est aussi ou guerrier, ou tout prêt à le devenir. Ceux qui vivent de la culture des terres passent, en général, tout le jour en plein air et exposés à toutes les injures du temps. La dureté de leur genre de vie habituel les dispose aux fatigues de la guerre, avec lesquelles quelques-uns de leurs travaux ont une grande analogie. Le travail journalier d’un homme qui creuse la terre le prépare à travailler à une tranchée, et il saura fortifier un camp, comme il sait enclore le champ qu’il cultive. Les passe-temps ordinaires de ces cultivateurs sont les mêmes que ceux des pasteurs, et sont pareillement des images de la guerre ; mais comme les cultiva­teurs n’ont pas autant de loisir que les pasteurs, ils ne sont pas aussi souvent livrés à ces exercices. Ce sont bien des soldats, mais ce ne sont pas des soldats tout à fait aussi exercés. Tels qu’ils sont cependant, il est rare qu’ils coûtent aucune dépense au souverain ou à la république, quand il s’agit de les mettre en campagne.
L’agriculture, même dans son état le plus grossier et le plus informe, suppose un établissement, une sorte d’habitation fixe qu’on ne peut quitter sans essuyer une gran­de perte. Aussi, quand une nation de simples agriculteurs marche à la guerre, la totalité du peuple ne peut se mettre en campagne à la fois ; au moins faut-il que les vieillards, les femmes et les enfants restent au pays pour garder la maison. Mais tous les hommes en âge de porter les armes peuvent partir pour l’armée, et c’est ainsi qu’en ont souvent usé de petites peuplades de ce genre. Dans toute nation, les hommes en âge de porter les armes sont supposés former environ le quart ou le cinquième de tout le peuple. D’ailleurs, si la campagne commence après le temps des semailles et finit avant la moisson, le laboureur et ses principaux ouvriers peuvent quitter la ferme sans beaucoup de dommage. Celui-ci partira dans la confiance que les vieillards, les femmes et les enfants pourront bien suffire aux travaux à faire dans l’intervalle. Il ne se refusera donc pas à servir sans paie pendant une courte campagne, et très-souvent il n’en coûte pas plus au souverain ou à la république pour l’entretenir à l’armée que pour le préparer à s’y rendre. C’est de cette manière, à ce qu’il semble, que servirent les citoyens de tous les différents États de l’ancienne Grèce, jusqu’après la seconde guerre de Perse, et les Péloponésiens jusqu’après la guerre du Péloponèse. Thucydide observe qu’en général ces derniers quittaient la campagne pendant l’été, et retour­naient chez eux pour faire la moisson. Le peuple romain, sous ses rois et pendant les premiers âges de la république, servit de la même manière. Ce ne fut qu’à l’époque du siège de Véïes[2] que ceux qui restaient dans le pays commencèrent à contribuer à l’en­tre­tien de ceux qui étaient allés à la guerre. Dans les monarchies de l’Europe, qui furent fondées sur les ruines de l’empire romain, tant avant l’époque de ce qui s’ap­pelle proprement L’établissement du gouvernement féodal, que quelque temps après, les grands seigneurs, avec tous ceux qui étaient immédiatement sous leur dépendance, avaient coutume de servir la couronne à leurs propres frais. Au camp, tout comme chez eux, ils vivaient de leurs revenus personnels, et non d’aucune paie ou solde qu’ils reçussent du roi pour cet objet.
Dans un état de société plus avancé, deux différentes causes ont contribué à ren­dre absolument impossible, pour ceux qui prenaient les armes, de s’entretenir à leurs frais. Ces deux causes sont le progrès des manufactures et les perfectionnements qui s’introduisirent dans l’art de la guerre.
Quand même ce serait un laboureur qui serait employé dans une expédition, pour­vu qu’elle commence après les semailles et qu’elle finisse avant la moisson, l’interrup­tion de ses occupations ne lui causera pas toujours une diminution considérable de reve­nu. La plus grande partie de l’ouvrage qui reste à faire s’achève par la nature seule, sans qu’il ait besoin d’y mettre la main. Mais du moment qu’un artisan, un forgeron, un charpentier, un tisserand, par exemple, quitte son atelier, la source unique de son revenu est totalement arrêtée. La nature ne travaille pas pour lui ; il faut qu’il fasse tout par ses mains. Ainsi, quand il prend les armes pour la défense de l’État, n’ayant aucun revenu pour se soutenir, il faut bien qu’il soit entretenu aux frais de l’État. Or, dans un pays où une grande partie des habitants sont artisans et manufac­turiers, c’est nécessairement de ces classes qu’est tirée une grande partie des gens qui portent les armes et, par conséquent, il est indispensable que l’État les entretienne pendant tout le temps qu’ils sont employés à son service.
D’un autre côté, quand l’art de la guerre est devenu, par degrés, une science diffi­ci­le et compliquée ; quand le sort des armes n’a plus été déterminé, comme dans les premiers temps, par une seule bataille ou plutôt une mêlée sans règle et sans ordre ; mais quand une guerre vint à se prolonger pendant plusieurs campagnes, chacune des­quelles durait la plus grande partie de l’année, alors ce fut partout une nécessité absolue que l’État entretînt ceux qui s’armaient pour sa défense, au moins pendant le temps qu’ils étaient employés à ce service. Quelle que pût être, en temps de paix, l’occupation de ceux qui faisaient la guerre, un service si long et si dispendieux eût été pour eux une charge infiniment trop lourde. Aussi, après la seconde guerre de Perse, les armées d’Athènes semblent avoir été composées, en général, de troupes mer­cenaires, dont partie, à la vérité, étaient des citoyens, mais partie aussi des étran­gers, et tous également soldés et défrayés par l’État. Depuis le siège de Véïes, les armées romaines reçurent une paie pour leur service pendant le temps qu’elles res­taient sous les drapeaux. Dans les gouvernements soumis aux lois féodales, le service militaire, tant des grands seigneurs que de leurs vassaux immédiats, fut, après un certain espace de temps, changé partout en une contribution pécuniaire destinée à l’entretien de ceux qui servaient à leur place.
Le nombre de ceux qui peuvent aller à la guerre relativement à la population totale est nécessairement beaucoup moindre dans un État civilisé que dans une socié­té encore informe. Dans une société civilisée, les soldats étant entretenus en entier par le travail de ceux qui ne sont pas soldats, le nombre des premiers ne peut jamais aller au-delà de ce que les autres sont en état d’entretenir, en outre de ce qu’ils sont encore obligés de faire pour fournir tant à leur entretien qu’à celui des autres officiers civils, convenablement à ce qu’exige la condition de chacun d’eux. Dans les petits États agricoles de l’ancienne Grèce, un quart, dit-on, ou un cinquième de toute la nation se regardaient comme soldats, et prenaient les armes dans l’occasion. Chez les peuples civilisés de l’Europe moderne, on calcule généralement qu’on ne saurait employer comme soldats plus du centième des habitants, sans ruiner le pays par la dépense qu’entraîne leur service[3].
Chez les peuples anciens, la dépense de préparer le soldat à faire la guerre ne paraît être devenue un objet considérable que longtemps après l’époque où la dépense de son entretien, pendant son service, fut tombée entièrement à la charge de l’État. Dans toutes les différentes républiques de l’ancienne Grèce, l’apprentissage des exercices militaires était une partie indispensable de cette éducation à laquelle était obligé tout citoyen libre. Il y avait, à ce qu’il semble, dans chaque ville un lieu public où, sous la protection des magistrats, différents maîtres enseignaient aux jeunes gens ces exercices. Toute la dépense qu’un État de la Grèce ait jamais eu à faire pour préparer ses citoyens à la guerre paraît avoir consisté dans cette simple institution. Les exercices du Champ-de-Mars remplissaient, à Rome, le même objet que ceux du gymnase dans l’ancienne Grèce. Sous l’empire des lois féodales, le grand nombre d’ordonnances publiques portant que les habitants de chaque canton s’exerceront dans la pratique de tirer de l’arc, ainsi que dans plusieurs autres exercices militaires, eurent en vue le même avantage, mais ne paraissent pas avoir eu le même succès. Soit défaut d’intérêt de la part des officiers chargés de l’exécution de ces ordonnances, soit quel­que autre cause, il semble qu’elles ont été partout négligées ; et à mesure des progrès de ces gouvernements, on voit partout les exercices militaires tomber insensiblement en désuétude parmi le peuple.
Dans les anciennes républiques de la Grèce et de Rome, pendant toute la durée de leur existence, et sous les gouvernements féodaux, longtemps après leur premier établissement, le métier de soldat ne fut pas un métier distinct et séparé qui constituât la seule ou la principale occupation d’une classe particulière de citoyens. Tout sujet de l’État, quelque pût être le métier ou l’occupation ordinaire dont il tirait sa subsis­tance, se regardait aussi, en toutes circonstances, comme soldat et comme obligé à en faire le métier dans les occasions extraordinaires.
Cependant, l’art de la guerre étant, sans contredit, le plus noble de tous[4], devient naturellement, à mesure de l’avancement de la société, l’un des arts les plus compli­qués. Les progrès de la mécanique, aussi bien que celui d’autres arts avec lesquels il a une liaison nécessaire, déterminent le degré de perfection auquel il est susceptible d’être porté à une époque quelconque ; mais, pour qu’il atteigne jusqu’à ce point, il est indispensable qu’il devienne la seule ou la principale occupation d’une classe parti­culière de citoyens, et la division du travail n’est pas moins nécessaire au perfec­tionnement de cet art qu’à celui de tout autre. Dans les autres arts, la division du travail est l’effet naturel de l’intelligence de chaque individu, qui lui montre plus d’avantages à se borner à un métier particulier qu’à en exercer plusieurs ; mais c’est la prudence de l’État qui seule peut faire du métier de soldat un métier particulier, distinct et séparé de tous les autres. Un simple citoyen qui, en temps de paix et sans recevoir de l’État aucun encouragement, passerait en exercices militaires la plus grande partie de sa journée, pourrait sans doute se perfectionner beaucoup en ce genre et se procurer un divertissement très-noble ; mais à coup sûr ce ne serait pas un moyen de faire ses affaires. Si c’est pour lui une voie à l’avancement et à la fortune que de consacrer à cette occupation une grande partie de son temps, ce ne peut être que par l’effet de la sagesse de l’État ; et cette sagesse, les États ne l’ont pas toujours eue, même quand ils se sont vus dans une situation où la conservation de leur exis­tence exigeait qu’ils l’eussent[5].
Un pasteur de troupeaux a beaucoup de moments de loisir ; un cultivateur, dans l’état informe de la culture, en a quelques-uns ; un artisan ou ouvrier de manufacture n’en a pas du tout. Le premier peut, sans se faire tort, consacrer une grande partie de son temps à des exercices militaires ; le second peut y donner quelques heures ; mais le dernier ne peut pas employer ainsi un seul de ses moments sans éprouver quelque perte, et le soin de son intérêt personnel le conduit naturellement à abandonner tout à fait ces exercices. Les progrès de l’art du labourage, qui nécessairement viennent à la suite de ceux des autres arts et des manufactures) laissent bientôt au laboureur aussi peu de moments de loisir qu’à l’artisan. Les exercices militaires finissent par être tout aussi négligés par les habitants des campagnes que par ceux des villes, et la masse du peuple perd tout à fait le caractère guerrier. En même temps, cette richesse qui est toujours la suite du progrès des manufactures et de l’agriculture et qui, dans la réalité, n’est autre chose que le produit accumulé de ces arts perfectionnés, appelle l’invasion des peuples voisins. Une nation industrieuse et, par conséquent, riche, est celle de toutes les nations qui doit le plus s’attendre à se voir attaquer ; et si l’État ne prend pas quelques mesures nouvelles pour la défense publique, les habitudes naturelles du peuple le rendent absolument incapable de se défendre lui-même.
Dans cet état de choses, il n’y a, à ce qu’il me semble, que deux méthodes pour que l’État puisse pourvoir, d’une manière convenable, à la défense publique.
Il peut, en premier lieu, au moyen d’une police très-rigoureuse, malgré la pente de l’intérêt, du caractère et des inclinations du peuple, maintenir par force la pratique des exercices militaires, et obliger, ou tous les citoyens en âge de porter les armes, ou un nombre quelconque d’entre eux, à joindre à un certain point le métier de soldat à tout autre métier ou profession qu’ils se trouveront avoir embrassée.
Ou bien, en second lieu, en entretenant et occupant constamment à la pratique des exercices militaires un certain nombre de citoyens, il peut faire du métier de soldat un métier particulier, séparé et distinct de tous les autres.
Si l’État a recours au premier de ces deux expédients, on dit que sa force militaire consiste dans ses milices ; s’il a recours au second, qu’elle consiste dans des troupes réglées. La pratique des exercices militaires est la seule ou la principale occupation des troupes réglées, et l’entretien ou la paie que leur fournit l’État est le fonds princi­pal et ordinaire de leur subsistance. La pratique des exercices militaires n’est que l’occupation accidentelle des soldats de milices, et c’est d’une autre occupation qu’ils tirent le fonds principal et ordinaire de leur subsistance. Dans les milices, le caractère d’artisan, d’ouvrier ou de laboureur l’emporte sur celui de soldat ; dans les troupes réglées, le caractère de soldat l’emporte sur tout autre ; et c’est dans cette distinction que consiste, à ce qu’il semble, la différence essentielle de ces deux espèces de force militaire.
Il y a eu des milices de plusieurs sortes. Dans quelques pays, les citoyens destinés à la défense de l’État ont été seulement, à ce qu’il paraît, exercés, mais sans être, si je puis parler ainsi, enrégimentés, c’est-à-dire sans être divisés en corps de troupes distincts et séparés, ayant chacun ses propres officiers permanents, sous lesquels ils fissent leurs exercices. Dans les anciennes républiques de la Grèce et dans celle de Rome, à ce qu’il semble, tant que chaque citoyen restait dans ses foyers, il pratiquait ses exercices ou séparément et indépendamment de personne, ou avec ceux de ses égaux auxquels il lui plaisait de se réunir ; mais il n’était attaché à aucun corps parti­culier de troupes jusqu’au moment où on l’appelait pour se ranger sous les drapeaux. Dans d’autres pays, les milices ont été non-seulement exercées, mais encore enré­gimentées. En Angleterre, en Suisse et, je crois, dans tous les autres pays de l’Europe moderne, où l’on a établi quelque force militaire imparfaite de ce genre, tout homme de milice est, même en temps de paix, attaché à un corps particulier de troupes qui a ses propres officiers permanents sous lesquels il remplit ses exercices.
Avant l’invention de l’arme à feu, la supériorité était du côté de l’armée dans laquelle chaque soldat individuellement avait le plus d’habileté et de dextérité dans l’usage de ses armes. La force et l’agilité du corps étaient de la plus grande impor­tance, et décidaient ordinairement du sort des batailles ; mais cette habileté et cette dextérité dans l’usage des armes ne pouvaient s’acquérir que comme on les acquiert aujourd’hui dans l’escrime, en pratiquant, non en grands corps de troupes, mais indi­viduellement et séparément, dans une école particulière, sous un maître particulier, ou bien avec quelques égaux et quelques camarades. Depuis l’invention des armes à feu, la force et l’agilité du corps, et même une dextérité et une agilité extraordinaires dans l’usage des armes, sont d’une moindre utilité, quoiqu’il s’en faille de beaucoup cepen­dant qu’on doive les compter pour rien. Par la nature de l’arme, si le maladroit n’est nullement au niveau de l’homme habile, il s’en trouve toutefois moins éloigné qu’il ne l’était jadis. On suppose que toute l’habileté et toute la dextérité nécessaires pour l’usage de cette arme peuvent s’acquérir assez bien en s’exerçant par grands corps de troupes.
La régularité, l’ordre et la prompte obéissance au commandement sont, dans les armées modernes, des qualités d’une plus grande importance pour décider du sort des batailles, que l’habileté et la dextérité du soldat au maniement de ses armes[6]. Mais le fracas et la fumée des armées à feu, cette mort invisible à laquelle tout homme se sent exposé aussitôt qu’il arrive à la portée du canon, et longtemps avant qu’on puisse bien dire que la bataille est engagée, doivent rendre extrêmement difficile de maintenir à un certain point, même dès le commencement de nos batailles modernes, cette régu­larité, cet ordre et cette prompte obéissance. Dans les batailles anciennes, il n’y avait pas d’autre grand bruit que les cris des combattants ; il n’y avait pas de fumée, point de ces coups invisibles qui portent la mort ou les blessures. Tout homme, jusqu’au moment où quelque arme mortelle venait à l’approcher, voyait clairement qu’il n’avait rien auprès de lui qui menaçât sa vie. Dans cet état de choses, et avec des troupes qui avaient quelque confiance dans leur habileté et leur adresse à manier leurs armes, il devait être infiniment moins difficile de maintenir un certain degré d’ordre et de régularité, non-seulement dans le commencement, mais même dans tout le cours de ces batailles anciennes, et jusqu’à ce que l’une des deux armées fût en pleine déroute. Mais l’habitude de cette régularité, de cet ordre et de cette prompte obéissance au commandement est une chose qui ne petit s’acquérir que par des soldats exercés en grands corps de troupes.
Toutefois des milices, de quelque manière qu’elles soient exercées ou disciplinées, seront toujours très-inférieures à des troupes réglées et bien disciplinées.
Des soldats qui ne sont exercés qu’une fois par semaine, ou une fois par mois, ne peuvent jamais être aussi experts au maniement des armes que ceux qui sont exercés tous les jours ou tous les deux jours ; et quoique cette circonstance ne soit pas, dans nos temps modernes, d’une aussi grande importance qu’elle l’était dans les temps anciens, cependant la supériorité bien reconnue des troupes prussiennes, qui provient en très-grande partie, dit-on, d’une habileté supérieure dans leurs exercices, est bien une preuve qu’aujourd’hui même ce point est d’une grande utilité.
Des soldats qui ne sont tenus d’obéir à leur officier qu’une fois par mois ou par semaine, et qui, dans tout le reste du temps, ont la liberté de faire ce qui leur convient, sans avoir aucun compte à lui rendre, ne peuvent jamais être aussi contenus par sa présence, aussi bien disposés à une prompte obéissance, que ceux dont la conduite et la manière de vivre sont habituellement réglées par lui, et qui tous les jours de leur vie ne peuvent se lever ni se coucher, ou du moins se retirer dans leurs quartiers, que d’après ses ordres. Dans ce qui s’appelle la discipline ou l’habitude de la prompte obéissance, des milices doivent toujours être encore plus au-dessous des troupes réglées, qu’elles ne le seront dans ce qui s’appelle l’exercice ou l’usage et le manie­ment des armes. Or, dans la guerre moderne, l’habitude d’obéir au premier signal est d’une bien autre importance qu’une grande supériorité dans le maniement des armes.
Ces milices qui, comme celles des Arabes ou des Tartares, vont à la guerre sous les mêmes chefs auxquels elles sont accoutumées à obéir pendant la paix, sont sans comparaison les meilleures de toutes. Par leur respect envers leurs officiers, leur habitude d’obéir au premier mot, elles approchent le plus des troupes réglées. La milice des montagnards avait quelques avantages de la même espèce, quand elle ser­vait sous ses propres chefs. Cependant, comme les montagnards n’étaient pas des pasteurs errants, mais des pasteurs stationnaires, qu’ils avaient des demeures fixes et n’étaient pas, en temps de paix, accoutumés à suivre leurs chefs d’un endroit à un autre, aussi dans les temps de guerre ils étaient moins disposés à les suivre à des distances considérables ou à rester pendant longtemps de suite sous les armes. Quand ils avaient fait quelque butin, ils étaient fort empressés de retourner chez eux, et l’autorité du chef était rarement suffisante pour les retenir. Sous le rapport de l’obéissance, ils ont toujours été fort inférieurs à ce qu’on nous rapporte des Tartares et des Arabes. De plus, comme les montagnards, au moyen de leur vie sédentaire, passaient une moindre partie de leur temps en plein air, ils étaient aussi moins accou­tumés aux exercices militaires, et moins experts au maniement de leurs armes, que ne le sont, dit-on, les Arabes et les Tartares.
Il faut observer néanmoins que des milices, de quelque espèce qu’elles soient, qui ont servi sous les drapeaux pendant plusieurs campagnes successives, deviennent sous tous les rapports de vraies troupes réglées. Les soldats sont exercés chaque jour à l’usage des armes, et étant constamment sous le commandement de leurs officiers, ils sont habitués à cette prompte obéissance qui a lieu dans une armée toujours sur pied ; peu importe ce qu’ils étaient avant de prendre les armes. Après avoir fait quelques campagnes, ils acquièrent nécessairement le caractère de troupes de lignes. Si la guerre d’Amérique venait à traîner encore pendant une autre campagne, les milices américaines deviendraient, à tous égards, égales à ces troupes réglées qui, dans la dernière guerre, ne se montrèrent pas, pour le moins, inférieures en valeur aux vétérans les plus aguerris de la France et de l’Espagne.
Cette distinction une fois bien entendue, on trouvera que l’histoire de tous les siècles atteste la supériorité irrésistible qu’une armée de troupes réglées bien discipli­née a sur des milices.
Une des premières armées de troupes réglées dont nous ayons un rapport un peu clair dans des histoires revêtues de quelque authenticité, c’est celle de Philippe de Macédoine. Ses guerres fréquentes contre les Thraces, les Illyriens, les Thessaliens et quelques-unes des villes grecques du voisinage de la Macédoine, formèrent par degrés ses troupes (qui vraisemblablement n’étaient dans le commencement que des milices) à l’exactitude de discipline des troupes réglées. Quand il fut en paix, ce qui ne lui arriva que rarement et jamais pour longtemps de suite, il eut bien soin de ne pas licencier cette armée. Elle vainquit et subjugua, après avoir essuyé, à la vérité, une longue et vive résistance, ces milices si braves et si bien exercées des principales républiques de la Grèce, et ensuite avec très-peu d’efforts les milices efféminées et mal exercées du vaste empire des Perses[7]. La chute des républiques de la Grèce et l’empire des Perses fut l’effet de la supériorité irrésistible d’une armée de troupes réglées sur toute espèce de milices. C’est la première des grandes révolutions arrivées dans les affaires humaines, dont l’histoire nous ait conservé quelque compte clair et circonstancié.
La seconde est la chute de Carthage et l’élévation de Rome, qui en fut la conséquence. On peut très-bien expliquer par la même cause toutes les variations de fortune que subirent ces deux républiques.
Depuis la fin de la première guerre punique jusqu’au commencement de la secon­de, les armées de Carthage furent continuellement sous les armes, et employées sous trois grands généraux qui se succédèrent dans le commandement : Hamilcar, Asdrubal son gendre, et Annibal son fils. Le premier s’en servit pour punir la révolte des esclaves, ensuite pour subjuguer les nations de l’Afrique qui avaient secoué le joug, et enfin pour conquérir le vaste royaume d’Espagne. L’armée qu’Annibal con­duisit d’Espagne en Italie avait dû nécessairement, pendant ces différentes guerres, se former par degrés à la discipline exacte d’une armée de ligne. En même temps, les Romains, sans avoir été absolument toujours en paix, n’avaient cependant été enga­gés, dans cette période, dans aucune guerre d’une bien grande importance, et l’on convient généralement que leur discipline militaire était extrêmement relâchée. Les armées romaines qu’Annibal eut en face à la Trébie, à Trasimène et à Cannes, étaient des milices opposées à des troupes réglées ; il est vraisemblable que cette circons­tance contribua plus que toute autre à décider du sort de ces batailles[8].
L’armée de troupes réglées qu’Annibal laissa derrière lui en Espagne eut la même supériorité sur les milices que les Romains envoyèrent contre elle, et dans un espace de peu d’années, sous le commandement de son frère, le jeune Asdrubal, elle les chassa presque entièrement de cette contrée.
Annibal fut mal secouru par son pays. Les milices romaines, étant continuelle­ment sous les armes, devinrent, dans le cours de la guerre, des troupes réglées bien disciplinées et bien exercées, et la supériorité d’Annibal devint de jour en jour moins forte. Asdrubal jugea nécessaire de conduire au secours de son beau-frère, en Italie, toute ou presque toute l’armée de troupes réglées qu’il commandait en Espagne. On dit que, dans cette marche, il fut égaré par ses guides ; il se vit surpris et attaqué dans un pays qu’il ne connaissait pas, par une autre armée de troupes réglées, à tous égards égale ou supérieure à la sienne, et il fut entièrement défait.
Quand Asdrubal eut quitté l’Espagne, le grand Scipion ne trouva rien qu’on pût lui opposer que des milices inférieures aux siennes. Il défit et subjugua ces milices et, dans le cours de la guerre, celles qu’il commandait devinrent nécessairement des troupes réglées bien exercées et bien disciplinées. Ces troupes réglées furent ensuite menées en Afrique, où elles n’eurent en face que des milices. Pour défendre Carthage, il devint indispensable de rappeler les troupes réglées que commandait Annibal. On joignit à ces troupes les milices africaines, souvent battues et découragées par leurs fréquentes défaites, et celles-ci composaient, à la bataille de Zama, la plus grande partie de l’armée d’Annibal. L’événement de cette journée décida des destinées de ces deux républiques rivales.
Depuis la fin de la seconde guerre punique jusqu’à la chute de la république romai­ne, les armées de Rome furent, sous tous les rapports, des armées de troupes réglées. L’armée de Macédoine, ainsi composée de troupes réglées, ne laissa pas que de leur résister. Rome, au faîte même de sa grandeur, eut besoin de deux grandes guerres et de trois grandes batailles pour subjuguer ce petit royaume, dont la conquête eût vraisemblablement été encore bien plus difficile sans la lâcheté de son dernier roi. Les milices de toutes les nations civilisées de l’ancien monde, de la Grèce, de la Syrie et de l’Égypte, n’opposèrent aux troupes romaines qu’une faible résistance. Les mili­ces de quelques nations barbares se défendirent beaucoup mieux. Les milices scythes ou tartares, que Mithridate tira des contrées situées au nord du Pont-Euxin et de la mer Caspienne, furent les ennemis les plus formidables que les Romains aient eus en face depuis la seconde guerre punique. Les milices des Parthes et des Germains furent aussi toujours des forces respectables, et dans plusieurs circonstances elles remportèrent sur les armées romaines des avantages très-considérables. Toutefois, en général, quand les armées romaines étaient bien commandées, elles paraissent avoir été très-supérieures ; et si les Romains ne poursuivirent pas la conquête définitive de la Germanie et du royaume des Parthes, ce fut probablement parce qu’ils jugèrent que ce n’était pas la peine d’ajouter ces deux contrées barbares à un empire déjà trop étendu. Les anciens Parthes semblent avoir été un peuple d’extraction scythe ou tartare, et avoir toujours conservé en grande partie les mœurs de leurs ancêtres. Les anciens Germains étaient, comme les Scythes ou les Tartares, une nation de pasteurs errants qui marchaient à la guerre sous les mêmes chefs qu’ils étaient accoutumés à suivre dans la paix. Leurs milices étaient absolument de la même espèce que celles des Scythes ou Tartares, desquels aussi ils étaient vraisemblablement descendus.
Plusieurs causes différentes contribuèrent à relâcher la discipline des armées romaines. Une de ces causes fut peut-être son extrême sévérité. Dans les jours de leur grandeur, lorsque les Romains ne virent plus aucun ennemi capable de leur résister, ils mirent de côté leur armure pesante comme un fardeau inutile à porter, et ils négligèrent leurs pénibles exercices, comme des fatigues qu’il n’était pas nécessaire d’endurer. D’ailleurs, sous les empereurs, les troupes réglées des Romains, particu­liè­rement celles qui gardaient les frontières de la Germanie et de la Pannonie, devinrent redoutables pour leurs maîtres, contre lesquels elles mettaient souvent en opposition leurs propres généraux. Dans la vue de les rendre moins formidables, Dioclétien, suivant quelques auteurs, Constantin, suivant d’autres, commença le premier à les retirer de la frontière où elles avaient toujours été auparavant campées en grands corps, chacun en général de deux ou trois légions, et il les dispersa par petits corps dans les différentes villes des provinces, d’où on ne les fit jamais sortir que lorsqu’il devint nécessaire de repousser une invasion, Des soldats en petit corps de troupes, mis en quartiers dans des villes de commerce et de manufactures, et qui quittaient rare­ment leurs quartiers, devinrent eux-mêmes des artisans, des marchands et des ouvriers de manufacture. Le caractère civil finit par l’emporter sur le caractère mili­taire, et insensiblement les troupes réglées de l’empire romain dégénérèrent en milices corrompues, négligées et sans discipline, incapables de résister aux attaques de ces milices de Scythes et de Germains qui, bientôt après, envahirent l’empire d’Occident. Ce ne fut qu’en prenant à leur solde les milices de quelques-unes de ces nations pour les opposer à celles des autres, que les empereurs purent venir à bout de se défendre quelque temps. La chute de l’empire d’Occident est la troisième des grandes révo­lutions dans l’histoire du genre humain, dont les annales anciennes nous aient conser­vé quelque récit positif et circonstancié. Cette révolution fut opérée par la supériorité décidée que les milices d’une nation barbare ont sur celles d’une nation civilisée, que les milices d’un peuple pasteur ont sur celles d’un peuple de laboureurs, d’artisans et de manufacturiers. Les victoires remportées par des milices ne l’ont pas été, en général, sur des troupes réglées, mais sur d’autres milices qui leur étaient inférieures du côté de l’exercice et de la discipline. Telles furent les victoires remportées par les milices des Grecs sur celles de l’empire des Perses, et telles aussi furent celles que, dans des temps plus récents, les milices des Suisses remportèrent sur celles des Autrichiens et des Bourguignons.
La force militaire des nations scythes et germaines qui s’établirent sur les ruines de l’empire d’Occident continua pour quelque temps à être, dans leurs nouveaux établissements, de la même espèce qu’elle avait été dans leur pays originaire. Ce furent des milices de pasteurs et de laboureurs, qui marchaient, en temps de guerre, sous les ordres des mêmes chefs auxquels ils étaient accoutumés à obéir pendant la paix. Elles étaient, par conséquent, assez bien exercées et assez bien disciplinées. Cependant, à mesure qu’avançaient les arts et l’industrie, l’autorité des chefs vint insensiblement à déchoir, et la masse du peuple eut moins de temps à donner aux exercices militaires. Ainsi, l’exercice aussi bien que la discipline des milices féodales vinrent insensiblement à se perdre et, pour suppléer à leur défaut, l’usage des troupes réglées vint à s’introduire successivement. D’ailleurs, dès qu’une nation civilisée eut une fois adopté la ressource d’une armée de troupes réglées, il devint, pour ses voi­sins, indispensable de suivre son exemple. Ils sentirent bientôt que leur sûreté en dépendait, et que leurs milices étaient absolument incapables de résister aux attaques d’une armée de cette nature.
Les soldats qui composent des troupes réglées, sans avoir même jamais vu l’ennemi, ont souvent donné des preuves d’autant de courage que de vieilles troupes ; et du premier moment qu’ils sont entrés en campagne, on les a vus propres à faire face aux vétérans les mieux aguerris et les plus expérimentés. Lorsque, en 1756, l’armée de la Russie entra en Pologne, les soldats russes ne se montrèrent pas inférieurs en valeur aux soldats prussiens, qu’on regardait alors comme les vétérans les plus braves et les mieux exercés de l’Europe. Cependant il y avait alors près de vingt ans que l’empire de Russie jouissait d’une paix profonde, et il ne pouvait, à cette époque, avoir que très-peu de soldats qui eussent vu l’ennemi. Quand la guerre d’Espagne éclata, en 1739, l’Angleterre n’avait pas cessé d’être en paix depuis environ vingt-huit ans. Néanmoins la valeur de ses soldats, bien loin d’avoir été altérée par ce long repos, ne se montra jamais d’une manière plus distinguée que dans la tentative faite sur Cartha­gène, le premier exploit malheureux de cette guerre malheureuse. Dans une longue paix, les généraux peuvent peut-être oublier quelquefois leur habileté et leur adresse ; mais quand une armée de troupes réglées a toujours été bien tenue, on ne voit pas que les soldats aient jamais oublié leur valeur.
Quand une nation civilisée ne peut compter pour sa défense que sur des milices, elle est en tout temps exposée à être conquise par toute nation barbare qui se trouvera être dans son voisinage. Les conquêtes fréquentes que les Tartares ont faites de tous les pays civilisés de l’Asie sont une assez forte preuve de la supériorité des milices d’une nation barbare sur celles d’une nation civilisée. Une armée de troupes réglées bien tenue est supérieure à toute espèce de milices. Si une armée de ce genre ne peut jamais être mieux entretenue que par une nation civilisée et opulente, aussi est elle la seule qui puisse servir à une pareille nation de barrière contre les invasions d’un voisin pauvre et barbare. Ainsi, c’est par le moyen d’une armée de troupes réglées seulement que la civilisation peut se perpétuer dans un pays, ou même s’y conserver longtemps. 
Si ce n’est que par le moyen d’une armée de troupes réglées bien tenue qu’un pays civilisé peut pourvoir à sa défense, ce ne peut être non plus que par ce moyen qu’un pays barbare peut passer tout d’un coup à un état passable de civilisation. Une armée de troupes réglées fait régner avec une force irrésistible la loi du souverain jusque dans les provinces les plus reculées de l’empire, et elle maintient une sorte de gou­vernement régulier dans des pays qui, sans cela, ne seraient pas susceptibles d’être gouvernés. Quiconque examinera avec attention les grandes réformes faites par Pierre le Grand dans l’empire de Russie, verra qu’elles se rapportent presque toutes à l’établissement d’une armée de troupes bien réglées. C’est là l’instrument qui lui sert à exécuter et à maintenir toutes ses autres ordonnances. C’est à l’influence de cette armée qu’il faut attribuer en entier le bon ordre et la paix intérieure dont cet empire a toujours joui depuis cette époque.
Les hommes attachés aux principes républicains ont vu d’un œil inquiet une armée de troupes réglées, comme étant une institution dangereuse pour la liberté. Elle l’est, sans contredit, toutes les fois que l’intérêt du général et celui des principaux officiers ne se trouvent pas nécessairement liés au soutien de la constitution de l’État. Les troupes réglées que commandait César renversèrent la république romaine ; celles de Cromwell chassèrent le long parlement. Mais quand c’est le souverain lui-même qui est le général ; quand ce sont les grands et la noblesse du pays qui sont les princi­paux officiers de l’armée ; quand la force militaire est placée dans les mains de ceux qui ont le plus grand intérêt au soutien de l’autorité civile, parce qu’ils ont eux-mêmes la plus grande part de cette autorité, alors une armée de troupes réglées ne peut jamais être dangereuse pour la liberté. Bien au contraire, elle peut, dans certains cas, être favorable à la liberté. La sécurité qu’elle donne au souverain[9] le débarrasse de cette défiance inquiète et jalouse qui, dans quelques républiques modernes, semble épier jusqu’aux moindres de vos actions, et menace à tous les instants la tranquillité du citoyen. Lorsque la sûreté du magistrat, quoiqu’elle ait pour appui la partie la plus saine du peuple, est néanmoins mise en péril à chaque mécontentement populaire ; lorsqu’un léger tumulte est capable d’entraîner en peu d’instants une grande révolu­tion, il faut alors mettre en œuvre l’autorité tout entière du gouvernement pour étouf­fer et punir le moindre murmure, la moindre plainte qui s’élève contre lui. Au contraire, un souverain qui sent son autorité soutenue, non-seulement par l’aristocratie naturelle du pays, mais encore par une armée de troupes réglées en bon état, n’éprouve pas le plus léger trouble au milieu des remontrances les plus violentes, les plus insensées et les plus licencieuses. Il peut mépriser ou pardonner ces excès, sans aucun risque, et le sentiment de sa supériorité le dispose naturellement à agir ainsi. Ce degré de liberté, qui a quelquefois les formes de la licence, ne peut se tolérer que dans les pays où une armée de ligne bien disciplinée assure l’autorité souveraine. Ce n’est que dans ces pays qu’il n’est pas nécessaire pour la sûreté publique de confier au souverain quelque pouvoir arbitraire, même dans les occasions où cette liberté licencieuse se livre à des éclats indiscrets.
Ainsi, le premier des devoirs du souverain, celui de défendre la société des vio­lences et des injustices d’autres sociétés indépendantes, devient successivement de plus en plus dispendieux, à mesure que la société avance dans la carrière de la civili­sation. La force militaire de la société qui, dans l’origine, ne coûte aucune dé­pen­se au souverain, ni en temps de paix ni en temps de guerre, doit, à mesure des progrès de la civilisation, être entretenue à ses frais, d’abord en temps de guerre et, par la suite, dans le temps même de la paix.
Les grands changements que l’invention des armes à feu a introduits dans l’art de la guerre ont renchéri bien davantage encore la dépense d’exercer et de discipliner un nombre quelconque de soldats en temps de paix, et celle de les employer en temps de guerre. Leurs armes et leurs munitions sont devenues à la fois plus coûteuses. Un mousquet est une machine plus chère qu’un javelot ou qu’un arc et des flèches ; un canon et un mortier le sont plus qu’une baliste ou une catapulte. La poudre qui se dépense dans une revue moderne est absolument perdue, et cette dépense est un objet très-considérable. Dans une revue ancienne, les javelots qu’on lançait, les flèches qu’on décochait, pouvaient aisément se ramasser pour servir encore, et d’ailleurs elles étaient de bien peu de valeur. Non-seulement le mortier et le canon sont des machines beaucoup plus chères que la baliste ou la catapulte, mais ce sont encore des machines beaucoup plus pesantes, et elles exigent des dépenses bien plus fortes, non-seulement pour les préparer au service, mais encore pour les transporter. De plus, comme l’artillerie moderne a une extrême supériorité sur celle des Anciens, l’art de fortifier les villes pour les mettre en état de résister à l’attaque d’une artillerie si supérieure, même pour quelques semaines, est devenue une chose bien plus difficile et, par conséquent, beaucoup plus dispendieuse. Dans nos temps modernes, mille causes différentes contribuent à rendre plus coûteuse la dépense de la défense publique. Ce qui a extrêmement ajouté, à cet égard, aux effets nécessaires des progrès naturels de la civilisation, c’est une grande révolution survenue dans l’art de la guerre, dont un pur hasard, l’invention de la poudre, semble avoir été la cause.
Dans les guerres modernes, la grande dépense des armes à feu donne un avantage marqué à la nation qui est le plus en état de fournir à cette dépense et, par conséquent, à une nation civilisée et opulente sur une nation pauvre et barbare. Dans les temps anciens, les nations opulentes et civilisées trouvaient difficile de se défendre contre les nations pauvres et barbares. Dans les temps modernes, les nations pauvres et barbares trouvent difficile de se défendre contre les nations civilisées et opulentes. L’invention des armes à feu, cette invention qui paraît au premier coup d’œil une invention si funeste, est certainement favorable tant à la durée qu’à l’extension de la civilisation des peuples. 


SECTION SECONDE

Des dépenses qu’exige l’administration de la justice

Le second devoir du souverain, celui de protéger, autant qu’il est possible, chacun des membres de la société contre l’injustice ou l’oppression de tout autre membre de cette société, c’est-à-dire le devoir d’établir une administration de la justice, exige aussi des dépenses qui, dans les différentes périodes de la société, s’élèvent à des degrés fort différents.
Chez les nations de chasseurs, comme il n’y a presque aucune propriété, ou au moins aucune qui excède la valeur de deux ou trois journées de travail, il est rare qu’il y ait un magistrat établi ou une administration réglée de la justice. Des hommes qui n’ont point de propriété ne peuvent se faire de tort l’un à l’autre que dans leur personne ou leur honneur. Mais quand un homme tue, blesse, bat ou en diffame un autre, quoique celui à qui l’injure est faite souffre un dommage, celui qui fait l’injure n’en recueille aucun profit. Il en est autrement des torts qu’on fait à la propriété. Le profit de celui qui fait l’injure est souvent l’équivalent du dommage causé à celui à qui elle est faite : l’envie, le ressentiment ou la méchanceté sont les seules passions qui peuvent exciter un homme à faire injure à un autre, dans sa personne ou dans son honneur. Or, la plus grande partie des hommes ne se trouve pas très-fréquemment dominée par ces passions, et les hommes les plus vicieux ne les éprouvent même qu’accidentellement. D’ailleurs, quelque plaisir que certains carac­tères puissent trouver à satisfaire ces sortes de passions, comme une telle satisfaction n’est accompagnée d’aucun avantage réel ou permanent, la passion est ordinairement contenue, chez la plupart, par des considérations de prudence. Des hommes peuvent vivre en société, dans un degré de sécurité assez tolérable, sans avoir de magistrat civil qui les protège contre l’injustice de ces sortes de passions. Mais des passions qui opèrent d’une manière bien plus continue, des passions dont l’influence est bien plus générale, l’avarice et l’ambition chez l’homme riche, l’aversion pour le travail et l’amour du bien-être et de la jouissance actuelle chez l’homme pauvre, voilà les pas­sions qui portent à envahir la propriété. Partout où il y a de grandes propriétés, il y a une grande inégalité de fortunes. Pour un homme très-riche, il faut qu’il y ait au moins cinq cents pauvres ; et l’abondance où nagent quelques-uns suppose l’indigence d’un grand nombre. L’abondance dont jouit le riche provoque l’indignation du pauvre, et celui-ci, entraîné par le besoin et excité par l’envie, cède souvent au désir de s’empa­rer des biens de l’autre. Ce n’est que sous l’égide du magistrat civil que le possesseur d’une propriété précieuse, acquise par le travail de beaucoup d’années ou peut-être de plusieurs générations successives, peut dormir une seule nuit avec tranquillité ; à tout moment il est environné d’une foule d’ennemis inconnus qu’il ne lui est pas possible d’apaiser, quoiqu’il ne les ait jamais provoqués, et contre l’injustice desquels il ne saurait être protégé que par le bras puissant de l’autorité civile sans cesse levé pour les punir. Ainsi, l’acquisition d’une propriété d’un certain prix et d’une certaine étendue exige nécessairement l’établissement d’un gouvernement civil. Là où il n’y a pas de propriété, ou au moins de propriété qui excède la valeur de deux ou trois journées de travail, un gouvernement civil n’est pas aussi nécessaire.
Un gouvernement civil suppose une certaine subordination ; mais si le besoin du gouvernement civil s’accroît successivement avec l’acquisition de propriétés d’une certaine valeur, aussi les causes principales qui amènent naturellement la subordina­tion augmentent-elles de même successivement avec l’accroissement de ces pro­priétés.
Les causes ou les circonstances qui amènent naturellement la subordination, ou qui, antérieurement à toute institution civile, donnent naturellement à certains hom­mes une supériorité sur la plus grande partie de leurs semblables, peuvent se réduire à quatre.
La première de ces causes ou circonstances est la supériorité des qualités per­sonnelles, telles que la force, la beauté et l’agilité du corps ; la sagesse et la vertu, la prudence, la justice, le courage et la modération. En quelque période que ce soit de la société, les qualités du corps, à moins d’être soutenues par celles de l’âme, ne peuvent donner que peu d’autorité. Il faut être un homme très-fort pour contraindre, par la seule force du corps, deux hommes faibles à vous obéir. Il n’y a que les qualités de l’âme qui puissent donner une très-grande autorité. Néanmoins, ce sont des qualités invisibles, toujours contestables et généralement contestées. Il n’y a pas de société barbare ou civilisée qui ait trouvé convenable de fonder sur ces qualités invisibles les règles qui détermineraient les degrés de prééminence de rang et ceux de subor­dination, mais toutes ont jugé à propos d’établir ces règles sur quelque chose de plus simple et de plus sensible. 
La seconde de ces causes ou circonstances est la supériorité d’âge. Un vieillard, pourvu que son âge ne soit pas tellement avancé qu’on puisse le soupçonner de rado­ter, est partout plus respecté qu’un jeune homme, son égal en rang, en fortune et en mérite. Chez les peuples chasseurs, tels que les tribus des naturels de l’Amérique septentrionale, l’âge est le seul fondement du rang et de la présence ; chez eux le nom père est un terme de supériorité ; celui de frère est un signe d’égalité, et celui de fils un signe d’infériorité. Chez les nations les plus civilisées et les plus opulentes, l’âge règle le rang parmi ceux qui sont égaux, sous tous les autres rapports, et entre lesquels, par conséquent, il ne pourrait être réglé par aucune autre circonstance. Entre frères et sœurs, l’aîné a toujours le pas ; et dans la succession paternelle, tout ce qui n’est pas susceptible de se partager, mais qui doit aller en entier à quelqu’un, tel qu’un titre d’honneur, est le plus souvent dévolu à l’aîné. L’âge est une qualité simple et sensible qui ne fournit pas matière à contestation.
La troisième de ces causes ou circonstances, c’est la supériorité de fortune. Néanmoins, l’autorité qui résulte de la richesse, quoiqu’elle soit considérable dans toute période de la société, ne l’est peut-être jamais plus que dans l’état le plus infor­me où la société puisse admettre quelque notable inégalité dans les fortunes. Un chef de Tartares, qui trouve dans l’accroissement de ses troupeaux un revenu suffisant pour l’entretien d’un millier de personnes, ne peut guère employer ce revenu autrement qu’à entretenir mille personnes. L’état agreste de sa société ne lui offre aucun produit manufacturé, aucuns colifichets d’aucune espèce, pour lesquels il puisse échanger cette partie de son produit brut qui excède sa consommation. Les mille personnes qu’il entretient ainsi, dépendant entièrement de lui pour leur subsistance, doivent nécessairement servir à la guerre sous ses ordres, et se soumettre à ses jugements en temps de paix. Il est à la fois leur général et leur juge, et sa dignité de chef est l’état nécessaire de la supériorité de sa fortune. Dans une société civilisée et opulente, un homme peut jouir d’une fortune bien plus grande, sans pour cela être en état de se faire obéir par une douzaine de personnes. Quoique le produit de son bien soit suffi­sant pour entretenir plus de mille personnes, quoique peut-être dans le fait il les entretienne, cependant, comme toutes ces personnes payent pour tout ce qu’elles reçoi­vent de lui, comme il ne donne presque rien à qui que ce soit sans en recevoir l’équi­valent en échange, il n’y a presque personne qui se regarde absolument comme dans sa dépendance, et son autorité ne s’étend pas au-delà de quelques valets. Néanmoins, l’autorité que donne la fortune est très-grande, même dans une société civilisée et opulente. De toutes les périodes de la société, compatibles avec quelque notable inégalité de fortune, il n’en est aucune dans laquelle on ne se soit constamment plaint de ce que cette sorte d’autorité l’emportait sur celle de l’âge ou du mérite personnel. La première période de la société, celle des peuples chasseurs, n’admet pas cette sorte d’inégalité. La pauvreté générale établit une égalité générale, et la supériorité de l’âge ou des qualités personnelles est la faible, mais unique base de l’autorité et de la subordination. Il n’y a donc que peu ou point d’autorité ou de subordination dans cette période de la société. Le second âge de la société, celui des peuples pasteurs, com­porte une très-grande inégalité de fortune, et il n’y a pas de période où la supériorité de fortune donne une aussi grande autorité à ceux qui la possèdent. Aussi, n’y a-t-il pas de période où l’autorité et la subordination soient aussi complètement établies. L’autorité d’un chérif arabe est très-grande ; celle d’un kan tartare est totalement despotique.
La quatrième de ces causes ou circonstances est la supériorité de naissance. La supériorité de naissance suppose, dans la famille de celui qui y prétend, une ancienne supériorité de fortune. Toutes les familles sont également anciennes, et les ancêtres d’un prince, quoiqu’ils puissent être plus connus, ne peuvent néanmoins guère être plus nombreux que ceux d’un mendiant. L’ancienneté de famille signifie partout une ancienneté de richesse ou de cette espèce de grandeur qui est ordinairement la suite ou la compagne de la richesse. Une grandeur qui vient de naître est partout moins respectée qu’une grandeur ancienne. La haine qu’on porte aux usurpateurs, l’amour qu’on a pour la famille d’un ancien monarque, sont des sentiments fondés en grande partie sur le mépris que les hommes ont naturellement pour la première de ces sortes de grandeur, et leur vénération pour l’autre. De même qu’un officier militaire se soumet sans répugnance à l’autorité d’un supérieur par lequel il a toujours été com­mandé, mais ne pourrait supporter de voir son inférieur placé au-dessus de lui ; de même les hommes sont disposés à la soumission envers une famille à laquelle ils ont toujours été soumis, ainsi que leurs ancêtres ; mais ils frémissent d’indignation s’ils voient une autre famille, dans laquelle ils n’ont jamais reconnu de semblable supériorité, s’emparer du droit de les gouverner.
La distinction de naissance étant une suite de l’inégalité de fortune, ne peut avoir lieu chez des peuples chasseurs, parmi lesquels tous les hommes, étant égaux en fortune, doivent pareillement être à peu près égaux par la naissance. À la vérité, le fils d’un homme sage ou vaillant peut bien, même chez eux, être un peu plus considéré qu’un homme de mérite égal qui aura le malheur d’être fils d’un imbécile ou d’un lâche. Avec cela, la différence ne sera pas très-sensible, et je ne pense pas qu’il y ait jamais eu aucune grande famille dans le monde qui ait tiré toute son illustration de la sagesse et de la vertu de sa souche.
Chez des nations de pasteurs, non-seulement la distinction de naissance peut avoir lieu, mais même elle y existe toujours. Ces nations ne connaissent aucune espèce de luxe, et chez elles la grande richesse ne peut jamais être dissipée par des prodigalités imprudentes. Aussi, n’y a-t-il pas de nations qui abondent davantage en familles révé­rées et honorées comme comptant une longue suite d’ancêtres distingués et illustres, parce qu’il n’y a pas de nations chez lesquelles la richesse soit dans le cas de se perpétuer plus longtemps dans les mêmes familles.
La naissance et la fortune sont évidemment les deux circonstances qui contribuent le plus à placer un homme au-dessus d’un autre. Ce sont les deux grandes sources des distinctions personnelles, et ce sont, par conséquent, les causes principales qui établissent naturellement de l’autorité et de la subordination parmi les hommes. Chez des peuples pasteurs, chacune de ces deux causes opère dans la plénitude de sa force. Le grand pasteur ou grand propriétaire de troupeaux, considéré à cause de ses immenses richesses, respecté à cause du grand nombre de personnes qu’il fait subsister, vénéré à cause de la noblesse de sa naissance et de l’ancienneté immémo­riale de son illustre famille, a une autorité naturelle sur tous les pasteurs ou bergers inférieurs de sa horde ou de sa tribu. Il petit commander aux forces réunies d’un plus grand nombre d’hommes qu’aucun d’eux. Sa puissance militaire est plus grande que celle d’aucun d’eux. En temps de guerre, ils sont tous naturellement plus disposés à se ranger sous sa bannière que sous celle de toute autre ; ainsi, sa naissance et sa fortune lui donnent naturellement une sorte de pouvoir exécutif. D’un autre côté, en com­mandant une réunion de forces plus nombreuses qu’aucun d’eux, il est plus en état de contraindre celui d’entre eux qui aurait pu faire tort à quelque autre, à réparer ce dommage ; par conséquent, il est celui dans la personne duquel ceux qui sont trop faibles pour se défendre eux-mêmes voient naturellement un protecteur. C’est à lui qu’ils adressent leurs plaintes sur les injures qu’ils peuvent avoir reçues, et en pareil cas la personne même contre laquelle la plainte est portée se soumettra plus volontiers à ce qu’il interpose son autorité dans la querelle, qu’elle ne s’y soumettrait à l’égard de tout autre. Ainsi, sa fortune et sa naissance lui donnent naturellement une sorte de pouvoir judiciaire.
C’est dans l’âge des peuples pasteurs, la seconde période de l’état social, que l’inégalité de fortune commence d’abord à naître et à introduire parmi les hommes un degré d’autorité et de subordination qui ne pouvait y exister auparavant. Elle introduit par là jusques à un certain point ce gouvernement civil qui est indispensablement né­ces­saire pour que la société elle-même puisse se conserver ; et c’est tout naturellement, indépendamment même de la considération de cette nécessité, qu’elle l’introduit. Cette considération, sans contredit, vient ensuite contribuer pour beaucoup à mainte­nir et fortifier l’autorité et la subordination. Les riches, en particulier, sont nécessai­rement intéressés à appuyer un ordre de choses qui seul peut leur assurer la posses­sion de leurs avantages. Des hommes d’une richesse inférieure se lient à la défense de la propriété de ceux qui leur sont supérieurs en richesses, afin que ces derniers se lient à leur tour à la défense de leurs petites propriétés. Tous les pasteurs et bergers du second ordre sentent que la sûreté de leurs troupeaux dépend de la sûreté de ceux du grand pasteur ou berger ; que le maintient de la portion d’autorité dont ils jouissent dépend du maintien de la portion plus grande dont jouit celui-ci, et que c’est sur leur subordination envers lui que repose le pouvoir de tenir leurs inférieurs dans une pa­reille subordination envers eux-mêmes. Ils constituent une espèce de petite noblesse qui se sent intéressée à défendre leur propriété et à soutenir l’autorité de son petit souverain, afin qu’il soit en état lui-même de défendre leur propriété et de soutenir leur autorité. Le gouvernement civil, en tant qu’il a pour objet la sûreté des propriétés, est, dans la réalité, institué pour défendre les riches contre les pauvres, ou bien, ceux qui ont quelque propriété contre ceux qui n’en ont point.
Néanmoins, l’autorité judiciaire d’un pareil souverain, bien loin d’être pour lui un sujet de dépense, fut longtemps, au contraire, une source de revenu. Les personnes qui s’adressaient à lui pour avoir justice étaient toujours disposées à payer pour l’obtenir, et un présent ne manquait jamais d’accompagner la requête. De plus, quand l’autorité du souverain fut complètement établie, la personne jugée coupable, outre la satisfaction qu’elle était tenue de faire à la partie lésée, était encore obligée au paye­ment d’une amende envers le souverain. Elle avait causé une peine au souverain, elle avait troublé, elle avait rompu la paix de son seigneur roi, et pour cette offense on pensait qu’il était dû une réparation. Dans les gouvernements tartares de l’Asie, dans les gouvernements d’Europe, fondés par les nations scythes et germaines qui renver­sèrent l’empire romain, l’administration de la justice fut une source de revenu, tant pour le souverain que pour les chefs ou seigneurs subalternes qui exerçaient sous lui quelque juridiction soit sur quelque horde ou tribu particulière, soit sur quelque territoire du canton déterminé. Dans l’origine le souverain, ainsi que les chefs infé­rieurs, avaient coutume d’exercer en personne leur juridiction. Ensuite, ils trouvèrent partout plus commode d’en déléguer l’exercice à quelque substitut, bailli ou juge. Toutefois, ce substitut était toujours obligé de compter à son supérieur ou commettant des profits de justice. Il ne faut que lire les instructions[10] qui furent données aux juges de tournée du temps de Henri II, pour voir clairement que ces juges étaient des espèces de facteurs ambulants envoyés en tournée dans le pays pour lever quelques branches du revenu du roi. Dans ces temps-là, non-seulement l’administration de la justice fournissait des profits au souverain, mais même il paraît que l’un des principaux avantages qu’il se proposait en administrant la justice était de s’en faire un revenu.
Cette intention de se faire de l’administration de la justice une branche de revenu ne pouvait manquer de faire naître une foule d’énormes abus. La personne qui se présentait les mains bien garnies pour demander justice pouvait s’attendre à obtenir un peu plus que justice, pendant que celle qui la demandait avec de faibles présents devait s’attendre à avoir un peu moins. On pouvait aussi souvent différer de rendre justice, afin que les présents fussent répétés. D’un autre côté, l’amende encourue par la personne dont on se plaignait pouvait bien souvent suggérer de fortes raisons de la trouver dans son tort, même quand elle n’y aurait pas été réellement. Les anciennes histoires de chaque pays de l’Europe font foi que de tels abus n’étaient rien moins que rares. 
Quand le souverain ou chef exerçait en personne son autorité judiciaire, à quelque degré qu’il en abusât, il ne devait guère être possible d’obtenir réparation de l’abus, parce qu’il ne pouvait y avoir personne d’assez puissant pour l’appeler à rendre compte de sa conduite. À la vérité, lorsqu’il l’exerçait par le ministère d’un bailli, on pouvait quelquefois obtenir réparation des abus. Si c’était pour son profit personnel que le bailli eût commis une injustice, le souverain lui-même n’était pas toujours éloigné de le punir ou de l’obliger à réparer son tort. Mais si c’était pour le profit de son souverain, si c’était pour se rendre agréable à la personne qui l’avait mis en place et qui pouvait l’avancer, qu’il eût exercé quelque acte d’oppression, il devait être la plupart du temps tout aussi impossible d’en obtenir la réparation, que si c’eût été le souverain qui eût prévariqué lui-même. Aussi, dans tous les gouvernements barbares, et particulièrement dans tous les anciens gouvernements de l’Europe, qui furent établis sur les ruines de l’empire romain, l’administration de la justice paraît avoir été longtemps excessivement corrompue ; sous les meilleurs monarques, elle était encore bien loin d’être tout à fait équitable et impartiale, et sous les plus mauvais elle était indignement prostituée.
Chez les peuples pasteurs, le souverain ou chef n’étant autre chose que le pasteur le plus considérable de la horde ou de la tribu, il a, comme tous ses vassaux ou sujets, pour s’entretenir, le croît de ses propres troupeaux. Chez les peuples cultivateurs, qui ne font que sortir de la vie pastorale, et qui ne sont pas fort avancés au-delà (tels qu’étaient, à ce qu’il semble, les tribus des Grecs vers le temps de la guerre de Troie, ou nos ancêtres Scythes et Germains quand ils commencèrent à s’établir sur les ruines de l’empire d’Occident), le souverain ou chef n’est de même autre chose que le plus grand propriétaire du lieu, et il a de même pour s’entretenir, comme tout autre pro­prié­taire, le revenu qu’il tire de son propre bien, ou de ce qu’on appela depuis, dans l’Europe moderne, le domaine de la couronne. Dans les circonstances ordinaires, ses sujets ne contribuent en rien à son entretien, excepté quand ils se trouvent dans la nécessité de recourir à son autorité pour les protéger contre l’oppression de quelque autre sujet. Les présents qu’ils lui font dans de pareilles occasions constituent tout le revenu ordinaire, tous les émoluments que peut lui rapporter sa souveraineté sur eux, à cela près peut-être de quelques profits dans certaines occasions fort rares. Dans Homère, quand Agamemnon offre à Achille, pour obtenir son amitié, la souveraineté de sept villes grecques, le seul avantage qu’il annonce devant être le fruit de cet empire, c’est que le peuple l’honorera de présents. Tant que de pareils présents, tant que les émoluments de justice, ou ce qu’on pourrait appeler les honoraires de la cour, constituèrent ainsi tout le revenu ordinaire que le souverain retira de sa souveraineté, on ne dut guère s’attendre, on ne put même convenablement lui proposer qu’il renon­çât entièrement à ce produit. On pouvait seulement lui demander, et la proposition en fut souvent faite, de le régler et de le fixer. Mais, malgré ces règlements et ces fixa­tions, il était toujours extrêmement difficile, pour ne pas dire impossible, de trouver moyen d’empêcher qu’une personne qui était toute-puissante n’étendît les émoluments au-delà des fixations. Ainsi, tant que dura cet état de choses, il n’y avait presque aucune voie efficace de remédier à la corruption de la justice, résultat nécessaire de la nature incertaine et arbitraire de ces présents.
Mais lorsque, d’après différentes causes, et principalement d’après l’accroissement continuel des dépenses occasionnées par la nécessité de se défendre contre les invasions étrangères, le domaine particulier du souverain fut devenu tout à fait insuf­fisant pour défrayer la dépense de la souveraineté, et quand il fut nécessaire que le peuple, pour sa propre sûreté, contribuât à cette dépense par des impôts de diffé­ren­tes espèces, il paraît qu’il a été très-communément stipulé que, sous aucun prétexte, ni le souverain, ni ses baillis ou les juges ses substituts, ne pourraient accepter de présents pour l’administration de la justice. On trouva plus aisé, à ce qu’il semble, d’abolir totalement l’usage des présents, que de les régler et de les fixer d’une manière effica­ce. On attribua aux juges des salaires fixes, qui furent regardés, à leur égard, comme un dédommagement de ce qu’ils pouvaient perdre dans le partage des anciens émolu­ments de justice, comme aussi les impôts étaient, pour le souverain, un dédom­magement plus que suffisant de ce qu’il perdait sur cet article. Dès lors on dit que la justice serait administrée gratuitement.
Dans la réalité cependant, la justice ne fut jamais administrée gratuitement dans aucun pays. Au moins faut-il toujours que les parties salarient des procureurs et des avocats ; et si ceux-ci n’étaient pas payés, ils s’acquitteraient de leurs fonctions encore bien plus mal qu’ils ne le font aujourd’hui. Les salaires qu’on paie aux avocats et aux procureurs montent annuellement, dans chaque tribunal, à une somme beaucoup plus forte que les gages des juges. Quoique le payement de ces gages soit une dépense de la couronne, c’est une circonstance qui ne peut jamais diminuer de beaucoup les frais nécessaires d’un procès. Mais ce fut moins pour diminuer les frais de justice que pour prévenir la corruption des juges, qu’on les empêcha de recevoir aucun présent ou honoraire des parties.
Les fonctions de juges sont tellement honorables par elles-mêmes, qu’il y a tou­jours des hommes disposés à s’en charger, quoiqu’elles ne soient accompagnées que de très-faibles émoluments. Les fonctions subalternes de juge de paix, qui entraînent avec elles beaucoup de peine et qui, dans la plupart des circonstances, ne rendent aucuns émoluments, n’en sont pas moins l’objet de l’ambition de la plupart de nos propriétaires ruraux. Les gages forts ou faibles de tous les différents juges, ensemble toute la dépense qu’entraînent l’administration de la justice et son exécution, même dans les pays où cette dépense n’est pas conduite avec une très-grande économie, ne forment qu’une portion bien peu considérable de la dépense totale du gouvernement.
On pourrait, d’ailleurs, défrayer aisément toute la dépense de cette administration de la justice avec ce que payent les parties, comme honoraires de la cour et, sans exposer la justice à aucun véritable danger de corruption, on pourrait ainsi soulager entièrement le revenu public d’une charge permanente, quoique assez légère. Il est très-difficile de taxer d’une manière efficace les honoraires des cours de justice, quand une personne aussi puissante que le souverain y a sa part, et quand il en fait une branche importante de son revenu. Mais c’est une chose très-facile, quand le juge est la seule personne qui en recueille quelque profit. La loi vient aisément à bout de faire respecter le règlement par les juges, quoiqu’elle ne soit pas toujours en état de le faire respecter par le souverain. Quand les honoraires des cours sont réglés et fixés d’une manière bien précise, quand ils sont payés tous à la fois, à une certaine période du procès, entre les mains d’un caissier ou receveur, pour être par lui distribués, d’après des proportions déterminées, à chacun des juges, après la décision du procès et non avant, il semble qu’il n’y a pas là plus de danger de corruption que si ces honoraires étaient entièrement supprimés. Sans occasionner aucune augmentation considérable dans les frais de procédure, on pourrait rendre ces honoraires suffisants pour défrayer complètement la totalité des dépenses de l’administration judiciaire. S’ils n’étaient payés aux juges qu’après la fin du procès, ils seraient un mobile pour exciter le tribunal à mettre de la diligence dans l’examen et la décision des affaires. Dans les cours composées d’un nombre considérable de juges, en proportionnant la part de chaque juge au nombre de jours et d’heures qu’il aurait employés à l’examen du procès, soit dans la séance du tribunal, soit dans un comité appointé par la cour, ces honoraires pourraient donner quelque encouragement au zèle de chacun des juges. Le public n’est jamais mieux servi que quand la récompense vient après le service, et qu’elle est proportionnée à la diligence qu’on a mise à s’en acquitter. Dans les diffé­rents parlements de France, les honoraires de la cour, qui se nomment épices et vacations, constituent la plus grande partie, sans comparaison, des émoluments des juges. Toutes déductions faites, ce qui est payé net par le roi pour salaires ou gages à un juge ou conseiller au parlement de Toulouse, le second parlement du royaume en rang et en dignité, ne monte qu’à 150 liv. tournois, à peu près 5 liv. 11 sch. sterling par an. Il y a environ sept ans que, dans la même ville, cette somme était le taux ordinaire des gages annuels d’un laquais. La distribution de ces épices se fait aussi selon le travail de chaque juge. Un juge laborieux tire de son office un revenu assez honnête, quoique modique ; celui qui ne fait rien ne gagne guère que ses gages ou salaires. Ces parlements ne sont peut-être pas, à beaucoup d’égards, d’excellentes cours de justice ; mais jamais ils n’ont été accusés, pas même, à ce qu’il semble, jamais soupçonnés de corruption.
Il paraît que les honoraires de la cour formaient dans l’origine presque tout le revenu des différentes cours de justice en Angleterre. Chaque cour tâchait d’attirer à elle le plus d’affaires qu’elle pouvait, et par cette raison elle était disposée à prendre connaissance de beaucoup de procès qui, par leur nature, ne devaient pas être de sa compétence. La cour du banc du roi, instituée seulement pour le jugement des affaires criminelles, s’attribua la connaissance d’affaires purement civiles, le plaignant prétendant que le défendeur, en lui refusant justice, s’était rendu coupable de quelque crime ou de quelque, délit envers lui. La cour de l’échiquier, instituée pour connaître seulement de la perception des revenus du roi et recouvrement des deniers royaux, s’arrogea la connaissance de toutes autres dettes ou engagements, le demandeur alléguant qu’il ne pouvait payer le roi, faute d’être payé par son débiteur. En consé­quence de ces fictions, dans la plupart des affaires il dépendait totalement des parties de choisir le tribunal par lequel elles voulaient être jugées, et chaque cour, en jugeant avec plus de diligence et d’impartialité, s’efforça d’attirer à elle le plus de causes possible. Si les cours de justice en Angleterre sont aujourd’hui si parfaitement consti­tuées, nous en sommes peut-être originairement redevables, en grande partie, à cette émulation anciennement établie entre les juges respectifs qui les composaient, chaque juge tâchant, dans la cour dont il était membre, de trouver pour toute espèce d’injus­tice le remède le plus prompt et le plus efficace que la loi pût comporter. Dans le principe, les cours de loi n’accordaient pour infraction de contrat que des dommages-intérêts seulement. La cour de chancellerie, comme cour de conscience[11], fut la pre­mière qui prit sur elle de contraindre à l’exécution formelle des simples conventions. Quand l’infraction du contrat ne consistait que dans un non-payement de deniers, le dommage souffert par le créancier ne pouvait être réparé autrement qu’en ordonnant le payement ; ce qui était équivalent à une stricte exécution de la convention. Dans ce cas, le remède des cours de loi était suffisant. Il n’en était pas ainsi dans d’autres cas. Quand le tenancier poursuivait son seigneur pour l’avoir injustement évincé de son bail, les dommages-intérêts qu’on lui adjugeait n’équivalaient nullement pour lui à la jouissance de la terre. Aussi les causes de cette nature vinrent toutes, pendant quelque temps, à la cour de chancellerie ; ce qui ne fit pas peu de tort aux cours de loi. On prétend que ce fut pour ramener ces causes à leur tribunal, que les cours de loi ima­ginèrent cette action fictive qu’on nomme Writ d’expulsion, le remède le plus efficace contre une injuste expulsion ou une dépossession d’immeubles.
Un droit de timbre sur les actes de procédure dans chaque cour particulière, levé par la cour elle-même et appliqué à l’entretien des juges et autres officiers attachés au tribunal, pourrait de même fournir un revenu suffisant pour défrayer la dépense de l’administration de la justice, sans grever d’aucune charge le revenu général de la société. Dans ce cas, à la vérité, les juges pourraient être tentés de multiplier inutile­ment les procédures dans chaque cause, pour augmenter, autant que possible, le produit du droit de timbre. L’usage, dans l’Europe moderne, a été de régler, la plupart du temps, le payement des procureurs et greffiers des tribunaux d’après le nombre de pages de leurs écritures, le règlement exigeant toutefois que chaque page contînt tant de lignes, et chaque ligne tant de mots. Les procureurs et greffiers, pour augmenter leurs profits, ont imaginé de multiplier les mots sans aucune nécessité, à un tel point qu’il n’est pas, je crois, une cour de justice en Europe dont ils n’aient totalement corrompu le style. Une tentation pareille pourrait peut-être donner lieu à une corrup­tion du même genre dans les formes de la procédure[12].
Mais, soit qu’on imagine un moyen pour que l’administration judiciaire prenne sur elle-même de quoi fournir à ses dépenses, soit qu’on attribue aux juges, pour leur entretien, des salaires fixes tirés de quelque autre fonds, toujours ne paraît-il pas nécessaire que celui ou ceux auxquels est confié le pouvoir exécutif soient chargés de la direction de ce fonds ou du payement de ces salaires. Ce fonds pourrait être formé du revenu de quelques propriétés foncières, et chaque cour particulière être chargée d’administrer les propriétés destinées à fournir à son entretien. Ce fonds pourrait être fait aussi avec l’intérêt d’une somme d’argent, et la cour être chargée d’administrer le capital consacré à cet objet. Une portion (très-petite, à la vérité) des salaires des juges de la cour de session d’Écosse provient de l’intérêt d’une somme d’argent. Néanmoins, l’instabilité d’un tel fonds paraît le rendre peu propre à servir à l’entretien d’une institution dont la nature est d’être perpétuelle.
La séparation du pouvoir judiciaire d’avec le pouvoir exécutif est provenue, dans l’origine, à ce qu’il semble, de la multiplication des affaires de la société, en consé­quence des progrès de la civilisation. L’administration de la justice devint par elle-même une fonction assez pénible et assez compliquée pour exiger l’attention tout entière des personnes auxquelles elle était confiée. La personne dépositaire du pouvoir exécutif n’ayant pas le loisir de s’occuper par elle-même de la décision des causes privées, on commit un délégué pour les décider à sa place. Dans les progrès de la grandeur romaine, le soin des affaires politiques de l’État donna trop d’occu­pation au consul, pour qu’il pût vaquer à l’administration de la justice. On établit donc un prêteur pour juger à sa place. Dans le cours des progrès des monarchies européennes qui furent fondées sur les ruines de l’empire romain, les souverains et les grands seigneurs en vinrent partout à regarder l’administration de la justice comme une fonction à la fois trop fatigante et trop peu noble pour la remplir eux-mêmes en personne. Partout, en conséquence, ils s’en débarrassèrent en établissant un lieutenant, juge ou bailli.
Quand le pouvoir judiciaire est réuni au pouvoir exécutif, il n’est guère possible que la justice ne se trouve pas souvent sacrifiée à ce qu’on appelle vulgairement des considérations politiques. Sans qu’il y ait même aucun motif de corruption en vue, les personnes dépositaires des grands intérêts de l’État peuvent s’imaginer quelquefois que ces grands intérêts exigent le sacrifice des droits d’un particulier. Mais c’est sur une administration impartiale de la justice que reposent la liberté individuelle de chaque citoyen, le sentiment qu’il a de sa propre sûreté. Pour faire que chaque indivi­du se sente parfaitement assuré dans la possession de chacun des droits qui lui appartiennent, non-seulement il est nécessaire que le pouvoir judiciaire soit séparé du pouvoir exécutif, mais il faut même qu’il en soit rendu aussi indépendant qu’il est possible. Il faut que le juge ne soit pas sujet à être déplacé de ses fonctions, d’après la décision arbitraire du pouvoir exécutif ; il faut encore que le payement régulier de son salaire ne dépende pas de la bonne volonté ni même de la bonne économie de ce pouvoir.

SECTION TROISIÈME
Des dépenses qu’exigent les travaux et établissements publics

Le troisième et dernier des devoirs du souverain ou de la république est celui d’élever et d’entretenir ces ouvrages et ces établissements publics dont une grande société retire d’immenses avantages, mais qui sont néanmoins de nature à ne pouvoir être entrepris ou entretenus par un ou par quelques particuliers, attendu que, pour ceux-ci, le profit ne saurait jamais leur en rembourser la dépense. Ce devoir exige aussi, pour le remplir, des dépenses dont l’étendue varie selon les divers degrés d’avan­cement de la société. 
Après les travaux et établissements publics nécessaires pour la défense de la société et pour l’administration de la justice, deux objets dont nous avons parlé, les autres travaux et établissements de ce genre sont principalement ceux propres à faci­liter le commerce de la société, et ceux destinés à étendre l’instruction parmi le peuple.
Les institutions pour l’instruction sont de deux sortes celles pour l’éducation de la jeunesse, et celles pour l’instruction du peuple de tout âge.
Pour examiner quelle est la manière la plus convenable de pourvoir à la dépense de ces différentes sortes de travaux et établissements publics, je diviserai cette troi­sième section du premier chapitre en trois différents articles.

Article I.
Des travaux et établissements propres à faciliter le commerce de la société

§I. De ceux qui sont nécessaires pour faciliter le commerce en général

Il est évident, sans qu’il soit besoin de preuve, que l’établissement et l’entretien des ouvrages publics qui facilitent le commerce d’un pays, tels que les grandes routes, les ponts, les canaux navigables, les ports, etc., exigent nécessairement des degrés de dépense, qui varient selon les différentes périodes où se trouve la société. La dépense de la confection et de l’entretien des routes doit évidemment augmenter avec le pro­duit annuel des terres et du travail du pays, ou avec la quantité et le poids des mar­chandises et denrées au transport desquelles ces routes sont destinées. La force d’un pont doit nécessairement être proportionnée au nombre et au poids des voitures qu’il est dans le cas de supporter. La profondeur d’un canal navigable et le volume d’eau qu’il faut lui fournir doivent nécessairement être proportionnés au nombre et au port des bâtiments employés à transporter des marchandises sur ce canal ; enfin, il faut que l’étendue d’un port soit aussi proportionnée au nombre de vaisseaux qui sont dans le cas d’y chercher un abri.
Il ne paraît pas nécessaire que la dépense de ces ouvrages publics soit défrayée par ce qu’on appelle communément le revenu public, celui dont la perception et l’appli­cation sont, dans la plupart des pays, attribuées au pouvoir exécutif. La plus grande partie de ces ouvrages peut aisément être régie de manière à fournir un revenu particulier suffisant pour couvrir leur dépense, sans grever d’aucune charge le revenu commun de la société.
Une grande route, un pont, un canal navigable, par exemple, peuvent le plus sou­vent être construits et entretenus avec le produit d’un léger droit sur les voitures qui en font usage ; un port, par un modique droit de port sur le tonnage[13] du vaisseau qui y fait son chargement ou son déchargement. La fabrication de la monnaie, autre insti­tution destinée à faciliter le commerce, non-seulement couvre sa propre dépense dans plusieurs pays, mais même y rapporte un petit revenu ou droit de seigneuriage au souverain. La poste aux lettres, autre institution faite pour le même objet, fournit, dans presque tous les pays, au-delà du remboursement de toute sa dépense, un revenu très-considérable au souverain.
Quand les voitures qui passent sur une grande route ou sur un pont, ou les bateaux qui naviguent sur un canal, payent un droit proportionné à leur poids ou à leur port, ils payent alors pour l’entretien de ces ouvrages publics, précisément dans la proportion du déchet qu’ils y occasionnent. Il paraît presque impossible d’imaginer une manière plus équitable de pourvoir à l’entretien de ces sortes d’ouvrages. D’ailleurs, si ce droit ou taxe est avancé par le voiturier, il est toujours payé en définitive par le consom­mateur, qui s’en trouve chargé dans le prix de la marchandise. Néanmoins, comme les frais du transport sont extrêmement réduits au moyen de ces sortes d’ouvrages, la marchandise revient toujours au consommateur, malgré ce droit, à bien meilleur marché qu’elle ne lui serait revenue sans cela, son prix n’étant pas autant élevé par la taxe qu’il est abaissé par le bon marché du transport. Ainsi, la personne qui paie la taxe, en définitive, gagne plus par la manière dont cette taxe est employée, qu’elle ne perd par cette dépense. Ce qu’elle paie est précisément en proportion du gain qu’elle fait. Dans la réalité, le payement n’est autre chose qu’une partie de ce gain qu’elle est obligée de céder pour avoir le reste. Il paraît impossible d’imaginer une méthode plus équitable de lever un impôt.
Quand cette même taxe sur les voitures de luxe, sur les carrosses, chaises de pos­te, etc., se trouve être de quelque chose plus forte, à proportion de leur poids, qu’elle ne l’est sur les voitures d’un usage nécessaire, telles que les voitures de roulier, les chariots, etc., alors l’indolence et la vanité du riche se trouvent contribuer d’une ma­nière fort simple au soulagement du pauvre, en rendant à meilleur marché le transport des marchandises pesantes dans tous les différents endroits du pays.
Lorsque les grandes routes, les ponts, les canaux, etc., sont ainsi construits et entretenus par le commerce même qui se fait par leur moyen, alors ils ne peuvent être établis que dans les endroits où le commerce a besoin d’eux et, par conséquent, où il est à propos de les construire. La dépense de leur construction, leur grandeur, leur magnificence, répondent nécessairement à ce que ce commerce peut suffire à payer. Par conséquent, ils sont nécessairement établis comme il est à propos de les faire. Dans ce cas, il n’y aura pas moyen de faire ouvrir une magnifique grande route dans un pays désert, qui ne comporte que peu ou point de commerce, simplement parce qu’elle mènera à la maison de campagne de l’intendant de la province ou au château de quelque grand seigneur auquel l’intendant cherchera à faire sa cour. On ne s’avi­sera pas d’élever un large pont sur une rivière, à un endroit où personne ne passe, et seulement pour embellir la vue des fenêtres d’un palais voisin ; choses qui se voient quelquefois dans ces provinces où les travaux de ce genre sont payés sur un autre revenu que celui fourni par ces travaux mêmes.
Dans plusieurs endroits de l’Europe, la taxe ou droit de passage sur un canal est la propriété particulière de certaines personnes qui, pour leur intérêt, se trouvent obligées à l’entretien du canal. S’il n’est pas passablement entendu, la navigation cesse nécessairement tout à fait, et avec elle tout le profit que le droit pourrait rendre. Si ces droits étaient mis sous la régie de commissaires qui n’y eussent personnellement pas d’intérêt, ceux-ci pourraient apporter moins d’attention à l’entretien des ouvrages dont ces droits sont le produit. Le canal de Languedoc a coûté au roi de France et à la province au-delà de 13 millions de livres tournois, qui, à 28 livres le marc d’argent que valait la monnaie de France à la fin du dernier siècle, feraient plus de 900,000 livres sterling. Quand ce grand ouvrage fut achevé, on trouva que le meilleur moyen de s’assurer qu’il serait toujours tenu en bon état de réparation, c’était de faire présent du droit à Riquet l’ingénieur, qui avait fait le plan et conduit les travaux. Le revenu de ce droit constitue aujourd’hui une fortune très-considérable à différentes branches de la famille de cet artiste, qui ont, par conséquent, grand intérêt à tenir constamment cet ouvrage en bon état ; mais si ce droit eût été mis sous la régie de commissaires qui n’auraient pas eu le même intérêt, le produit eût peut-être été dissipé en dépenses inutiles et en vaine décoration, tandis qu’on aurait laissé tomber en ruine les parties les plus essentielles.
Les droits pour l’entretien d’une grande route ne pourraient pas, sans inconvénient, constituer une propriété particulière. Un grand chemin, quoique entièrement négligé, ne devient pas pour cela absolument impraticable, comme le serait un canal. Par conséquent, les propriétaires des droits perçus sur une route pourraient négliger tota­le­ment les réparations, et cependant continuer de lever, à très-peu de chose près, les mêmes droits. Il est donc à propos que les droits destinés à l’entretien d’un ouvrage de ce genre soient mis sous la direction de commissaires ou de préposés.
On s’est plusieurs fois plaint avec beaucoup de justice, en Grande-Bretagne, des abus commis par les préposés à la régie de ce produit ; on a dit qu’à un grand nombre de barrières le produit était plus du double de ce qui est nécessaire pour entretenir parfaitement la route, tandis que l’ouvrage y était souvent fait de la manière la plus défectueuse, et quelquefois même ne s’y faisait pas du tout. Il faut observer que le système de réparer les grandes routes au moyen de ces sortes de droits n’est pas d’une pratique fort ancienne ; il ne faut donc pas nous étonner qu’il n’ait pas encore été porté à ce degré de perfection dont il pourrait être susceptible. Si les emplois de cette régie sont souvent confiés à des gens mal choisis et peu dignes de confiance, et si l’on n’a pas encore établi des bureaux d’inspection et de comptabilité pour contrôler leur conduite et pour réduire le droit à ce qu’exige précisément le travail dont ils sont chargés, il faut attribuer ces défauts à la nouveauté de l’institution, qui doit aussi leur servir d’excuse, et il faut espérer que la sagesse du parlement y remédiera en grande partie avec le temps.
On suppose que l’argent perçu aux différentes barrières, dans la Grande-Bretagne, excède tellement ce qu’exige la réparation des routes, que les épargnes à faire sur ce revenu, en y apportant l’économie convenable, ont été regardées, même des ministres, comme une très-grande ressource, dont on pourrait tirer parti, dans un temps ou dans l’autre, pour les besoins de l’État. On a dit que le gouvernement, en se chargeant lui-même de la régie des barrières et en faisant travailler les soldats moyennant un très-léger surcroît de paie dont ils seraient fort satisfaits, pourrait tenir les routes en bon état, à beaucoup moins de frais, que ne peuvent le faire les préposés, ceux-ci n’ayant pas d’autres ouvriers à employer que des gens qui tirent de leurs salaires toute leur subsistance. On a prétendu qu’à ce moyen, sans mettre aucune nouvelle charge sur le peuple, on gagnerait un revenu de peut-être un demi-million[14], en sorte que les bar­rières se trouveraient contribuer à la dépense générale de l’État, de la même ma­nière que le fait maintenant la poste aux lettres.
Je ne doute pas qu’on puisse gagner par ce moyen un revenu considérable, quoi­que vraisemblablement pas à beaucoup près autant que l’ont supposé les auteurs de ce projet ; toutefois, ce plan en lui-même est susceptible de plusieurs objections très-importantes.
Premièrement, si les droits qui se perçoivent aux barrières pouvaient jamais être regardés comme une des ressources propres à fournir aux besoins de l’État, certai­nement ils viendraient à être augmentés à mesure que ces besoins seraient supposés l’exiger. Ainsi, d’après la politique adoptée en Angleterre, ils seraient vraisembla­blement augmentés avec promptitude ; la facilité avec laquelle on pourrait en retirer un grand revenu encouragerait probablement l’administration à user très-fréquemment de cette ressource. S’il est peut-être plus que douteux qu’avec toute l’économie imaginable on puisse venir à bout d’épargner un demi-million sur ces droits, tels qu’ils sont, au moins ne pourrait-on guère douter que, s’ils étaient doublés, on pourrait fort bien épargner un million sur ce produit, et peut-être deux si les droits étaient triplés[15]. De plus, ce grand revenu pourrait être levé sans qu’il fût besoin de commettre un seul employé de plus pour la perception. Mais les droits de barrières étant, dans ce but, continuellement augmentés, au lieu de faciliter le commerce intérieur du pays, com­me à présent, ils deviendraient bientôt pour lui une charge très-onéreuse. La dépense de transporter d’un endroit du royaume à l’autre des marchandises pesantes, serait bientôt tellement augmentée, par conséquent le marché pour toutes les marchandises de ce genre se resserrerait tellement, que leur production en serait en grande partie dé­cou­ragée et que les branches les plus importantes de l’industrie nationale se trouve­raient totalement anéanties.
En second lieu, une taxe sur les voitures, proportionnée à leur poids, quoiqu’elle soit un impôt très-légal quand son produit n’est appliqué à aucun autre objet qu’à la réparation des routes, devient un impôt très-illégal dès qu’on en applique le produit à une autre destination ou aux besoins généraux de l’État. Quand ce produit s’applique uniquement à la réparation de la route, chaque voiture est censée payer précisément pour le déchet que son passage occasionne. Mais quand il est employé à tout autre objet, chaque voiture est censée payer pour plus que ce déchet, et contribue à pour­voir à quelques autres besoins de l’État. Or, comme le droit de barrières fait hausser le prix des marchandises en raison de leur valeur, il est principalement payé par le con­som­mateur de denrées grossières et volumineuses, et non par ceux qui consomment des marchandises légères et précieuses. Ainsi, quel que fût le besoin de l’État auquel cette taxe serait destinée, c’est aux dépens du pauvre principalement, et non à ceux du riche qu’on pourvoirait à ce besoin ; c’est aux dépens de ceux qui sont le moins en état d’y contribuer, et non de ceux qui sont en état de le faire.
Troisièmement, si le gouvernement venait une fois à négliger la réparation des grandes routes, il serait bien plus difficile qu’il ne l’est à présent de contraindre les percepteurs du droit de barrières à en appliquer quelque chose à sa vraie destination. Ainsi, on pourrait lever sur le peuple un très-gros revenu sans qu’il y en eût la moin­dre partie appliquée au seul objet auquel doive jamais l’être un revenu levé de cette manière. Si la pauvreté et la basse condition des préposés à l’entretien des routes em­pê­chent aujourd’hui qu’on ne puisse aisément leur faire réparer les fautes de leur administration, dans le cas que l’on suppose ici, leur richesse et leur importance rendraient la chose dix fois plus difficile.
En France, les fonds destinés à l’entretien des grandes routes sont sous la direction immédiate du pouvoir exécutif. Ces fonds consistent en partie dans un certain nombre de journées de travail que les gens de la campagne, comme en beaucoup d’autres endroits de l’Europe, sont forcés d’employer à la réparation des chemins, et en partie dans une certaine portion du revenu général de l’État, que le roi juge à propos de retrancher de ses autres dépenses.
Par l’ancienne loi de la France, aussi bien que de la plupart des autres endroits de l’Europe, ces journées de travail ou corvées étaient sous la direction d’un magistrat local ou provincial qui ne dépendait pas immédiatement du conseil du roi. Mais dans l’usage actuel, les corvées ainsi que tout autre fonds que le roi juge à propos d’assigner pour la réparation des grands chemins dans une province ou généralité particulière, sont entièrement sous la direction de l’intendant, officier qui est nommé et révoqué par le conseil du roi, qui en reçoit les ordres, et qui correspond continuel­le­ment avec lui. Dans les progrès du despotisme, l’autorité du pouvoir exécutif absor­be successivement celle de tout autre pouvoir de l’État, et s’empare de l’administration de toutes les branches de revenu destinées à quelque objet public. Néanmoins, en France les grandes routes de poste, celles qui font la communication d’entre les grandes villes du royaume, sont en général bien tenues, et dans quelques provinces elles sont même de beaucoup au-dessus de la plupart de nos routes à barrières. Mais ce que nous appelons chemins de traverse, c’est-à-dire la très-majeure partie des che­mins du pays, sont totalement négligés, et dans beaucoup d’endroits sont absolument impraticables pour une forte voiture. En certains endroits il est même dangereux de voyager à cheval, et pour y passer avec quelque sûreté on ne peut guère se fier qu’à des mulets. Le ministre orgueilleux d’une cour fastueuse se plaira souvent à faire exécuter un ouvrage d’éclat et de magnificence, tel qu’une grande route qui est à tout moment sous les yeux de cette haute noblesse dont les éloges flattent sa vanité et contribuent de plus à soutenir son crédit à la cour. Mais ordonner beaucoup de ces petits travaux qui ne peuvent rien produire de très-apparent ni attirer les regards du voyageur ; de ces travaux, en un mot, qui n’ont rien de recommandable que leur extrême utilité, c’est une chose qui semble, à tous égards, trop mesquine et trop misé­rable pour fixer la pensée d’un magistrat de cette importance. Aussi, sous une pareille administration, les travaux de ce genre sont-ils presque toujours totalement négligés.
En Chine et dans plusieurs autres gouvernements de l’Asie, le pouvoir exécutif se charge de la réparation des grandes rouies et même de l’entretien des canaux navigables. Ces objets, dit-on, sont constamment recommandés au gouverneur de chaque province dans les instructions qu’on lui donne, et l’attention qu’il montre avoir donnée à cette partie de ses instructions détermine beaucoup le jugement que la cour porte de sa conduite. Aussi ajoute-t-on que cette branche d’administration est tenue dans tous ces pays avec le plus grand soin, et particulièrement à la Chine, ou, a ce que l’on prétend, les grandes routes, encore plus les canaux navigables, sont fort au-des­sus de tout ce qu’on connaît dans ce genre en Europe. Toutefois, ce qui nous a été rapporté sur ces sortes de travaux a été décrit, en général, par de pauvres voyageurs qui semblent s’être laissé étonner de tout, et souvent par des missionnaires stupides et menteurs. Peut-être que si ces travaux eussent été examinés par des yeux plus intel­ligents, ou que les rapports nous en eussent été faits par des témoins plus fidèles, ils ne nous paraîtraient plus aussi surprenants. Le compte que nous rend Bernier de quelques ouvrages de ce genre dans l’Indostan, se trouve extrêmement au-dessous de ce qui en avait été rapporté par d’autres voyageurs plus amateurs du merveilleux que lui. Il pourrait bien aussi en être dans ce pays-là comme en France, où les grandes routes, les grandes communications qui sont dans le cas de faire des sujets de con­versation à la cour ou dans la capitale, sont tenues avec soin, et tout le reste négligé. D’ailleurs, à la Chine, dans l’Indostan et dans plusieurs autres gouvernements de l’Asie, le revenu du souverain provient presque en entier d’une taxe ou revenu foncier qui monte ou qui baisse à mesure que monte ou baisse le produit annuel des terres. Par conséquent, dans ces pays-là, le grand intérêt du souverain, son revenu, est nécessairement et immédiatement lié à l’état de la culture des terres, à la quantité et valeur de leur produit. Or, pour rendre à la fois ce produit aussi fort et d’un aussi grand prix que possible, il est nécessaire de lui procurer un marché aussi étendu que possible et, par conséquent, d’établir entre toutes les différentes parties du pays la communication la plus libre, la plus facile et la moins coûteuse ; ce qui ne peut se faire que par le moyen des meilleures routes et des meilleurs canaux navigables. Mais, dans aucun endroit de l’Europe, le revenu du souverain ne procède principale­ment d’un impôt territorial et revenu foncier. Peut-être bien que, dans tous les grands royaumes de l’Europe, la plus grande partie de ce revenu dépend en dernier résultat du produit de la terre ; mais ce n’est pas d’une manière aussi évidente ni aussi immé­diate qu’il en dépend. Ainsi, en Europe, le souverain ne se sent pas aussi direc­tement intéressé à concourir à l’accroissement, tant en quantité qu’en valeur, du produit de la terre, ou bien à procurer à ce produit le marché le plus étendu, en entretenant de bonnes routes et de bons canaux.
Par conséquent, quand même il serait vrai, ce que je regarde comme fort douteux, que, dans quelques endroits de l’Asie, ce département de la police publique fût tenu par le pouvoir exécutif de manière à ne rien laisser à désirer, il n’y aurait pas néanmoins pour cela la moindre probabilité que, dans l’état actuel des choses, il pût être régi passablement bien par ce même pouvoir dans aucun endroit de l’Europe.
Cette espèce même de travaux publics qui sont de nature à ne pouvoir fournir aucun revenu par leur propre entretien, mais dont la commodité et l’avantage se bornent, presque en entier, à quelque lieu ou canton particulier, sera encore mieux entretenue par un revenu local ou provincial, sous la direction d’une administration locale ou provinciale, que par le revenu général de l’État, dont il faut nécessairement que la direction soit entre les mains du pouvoir exécutif. Si le pavé et l’illumination des rues de Londres étaient à la charge du Trésor public, y a-t-il quelque probabilité que ces rues fussent aussi bien pavées et aussi bien éclairées qu’elles le sont à présent, ou même à aussi peu de frais ? D’ailleurs, cette dépense, au lieu d’être défrayée par une taxe locale levée sur les habitants de chaque rue, paroisse ou quartier de Londres, serait, dans ce cas, défrayée par le revenu général de l’État, et supportée par tous les habitants du royaume, qui contribuent à former ce revenu, quoique la plus grande partie de ces habitants ne retire aucune espèce d’avantage de ce que les rues de Londres sont pavées et éclairées.
Quelque énorme que puissent paraître parfois les abus qui se glissent dans l’administration particulière d’un revenu local et provincial, dans la réalité, cependant, ce ne sont que des bagatelles en comparaison de ceux qui ont ordinairement lieu dans l’administration du revenu d’un grand empire et dans la manière de dépenser ce revenu. D’ailleurs, ils sont bien plus faciles à réformer. Sous la direction locale des juges de paix en Angleterre, les six journées de travail que les gens de la campagne sont obligés de donner à la réparation des grands chemins ne sont peut-être pas toujours employées de la manière la plus judicieuse, mais il ne se trouve presque jamais qu’elles soient exigées avec des formes dures ou oppressives. En France, sous l’administration des intendants, l’emploi n’en est pas toujours fait avec plus de discernement, mais la manière dont on les exige est souvent très- inhumaine et très-despotique. Les corvées, qui sont le nom qu’on donne à cette contribution, sont devenues, entre les mains de ces officiers, un des principaux instruments de leur tyrannie pour châtier la paroisse ou la communauté qui aura eu le malheur d’encourir leur disgrâce[16].

II. Des travaux et établissements publics qui sont nécessaires pour faciliter quelque branche particulière du commerce

L’objet des travaux et établissements publics dont on vient de parler, est de faciliter le commerce en général. Mais, pour faciliter quelques branches particulières, il faut des établissements qui exigent encore une dépense spéciale et extraordinaire.
Des branches particulières de commerce, qui se font avec des peuples barbares et non civilisés, exigent une protection extraordinaire. Un simple magasin ou comptoir ne suffirait pas pour la sûreté des marchandises de ceux qui trafiquent avec les côtes occidentales de l’Afrique. Il est indispensable que l’endroit où ces marchandises sont déposées soit en quelque sorte fortifié, pour les défendre contre les naturels du pays. Les désordres survenus dans le gouvernement de l’Indostan ont fait croire qu’une pareille précaution était nécessaire même chez ce peuple si doux et si soumis, et ce fut sous le prétexte de mettre les personnes et les propriétés à l’abri de la violence, que les compagnies des Indes, tant d’Angleterre que de France, ont obtenu la per­mission d’élever les premiers forts qu’elles ont occupés dans ce pays. Chez d’autres nations dont le gouvernement énergique ne souffrirait pas que des étrangers possé­dassent sur son territoire quelque lieu fortifié ; il peut être nécessaire d’entretenir un ambassadeur, un ministre ou un consul qui décide, d’après nos lois et nos usages, les différends survenus entre nos compatriotes, et qui, dans leurs contestations avec les naturels du pays, puisse, à la faveur de son caractère public, s’interposer avec plus d’autorité et leur prêter une protection plus puissante qu’ils ne pourraient l’attendre d’un simple particulier. Souvent les intérêts du commerce ont exigé qu’on entretînt des ministres dans des contrées étrangères, où des motifs de guerre ou d’alliance n’en auraient pas demandé. Le commerce de la compagnie de Turquie fut la première cause qui donna lieu à avoir un ambassadeur à Constantinople. Les premières ambas­sades de l’Angleterre en Russie n’eurent d’autre objet que des intérêts commerciaux. C’est probablement la communication constante que ces intérêts ont occasionnée entre les sujets des différents États de l’Europe, qui a introduit la coutume d’entre­tenir, dans tous les pays voisins, des ambassadeurs ou ministres qui y résident cons­tam­ment, même en temps de paix. Cette coutume, inconnue dans les anciens temps, ne parait pas remonter au-delà de la fin du quinzième siècle ou du commencement du seizième, c’est-à-dire de l’époque à laquelle le commerce commença à s’étendre à la plus grande partie des nations de l’Europe, et à laquelle elles commencèrent à s’occuper de ses intérêts.
Il paraîtrait assez raisonnable que la dépense extraordinaire à laquelle peut donner lieu la protection d’une branche particulière de commerce fût défrayée par un impôt modéré sur cette même branche ; par exemple, par un droit modique une fois payé par le commerçant la première fois qu’il entre dans ce genre de commerce, ou, ce qui est plus égal, par un droit particulier de tant pour cent sur les marchandises qu’il importe dans les pays avec lesquels se fait cette branche de commerce, ou sur celles qu’il en exporte. On dit que le premier établissement des droits de douane a eu pour cause la protection du commerce en général contre les pirates et les corsaires qui infestaient les mers. Mais, s’il a semblé raisonnable d’établir un impôt général sur le commerce pour subvenir à ce qu’exige la protection du commerce en général, il devrait paraître tout aussi raisonnable d’établir un impôt particulier sur une branche particulière de commerce, afin de défrayer la dépense extraordinaire qu’exige la protection de cette branche.
La protection du commerce en général a toujours été regardée comme essentiel­lement bée à la défense de la chose publique et, sous ce rapport, comme une partie nécessaire des devoirs du pouvoir exécutif. En conséquence, la perception et l’emploi des droits généraux de douanes ont toujours été laissés à ce pouvoir. Or, la protection d’une branche particulière de commerce est une partie de la protection générale du commerce et, par conséquent, une partie des fonctions de ce même pouvoir ; et si les nations agissaient toujours d’une manière logique, les droits particuliers perçus pour pourvoir à une protection particulière de ce genre auraient toujours été laissés pareil­lement à sa disposition. Mais, sur ce point comme sur beaucoup d’autres, les nations n’ont pas toujours agi conséquemment et, dans la plus grande partie des États com­merçants de l’Europe, des compagnies particulières de marchands ont eu l’adresse de persuader à la législature qu’elle devait confier à leurs soins cette partie des devoirs du souverain, ainsi que tous les pouvoirs qui y sont nécessairement attachés.
Quoique peut-être ces compagnies, en faisant à leurs propres dépens une expé­rience que l’État n’eût pas jugé prudent de faire lui-même, aient pu servir à introduire certaines branches nouvelles de commerce, à la longue, néanmoins, elles sont deve­nues partout ou nuisibles, ou inutiles au commerce, et elles ont fini par lui donner une fausse direction ou par le restreindre.
Si ces compagnies ne commercent pas à l’aide d’un fonds social, mais qu’elles soient tenues d’admettre toute personne ayant les qualités requises, en payant un droit d’entrée déterminé, et à la charge de se soumettre aux règlements de la compagnie (chaque membre commerçant sur ses propres fonds et à ses risques), on les appelle compagnies privilégiées. Quand elles commercent à l’aide d’un fonds social, chaque membre prenant sa part des profits ou des pertes communes, en proportions de sa mise, on les nomme compagnies par actions. Ces compagnies, soit privilégiées[17], soit par actions, ont quelquefois des privilèges exclusifs, et quelquefois elles n’en ont point.
Les compagnies privilégiées ressemblent, sous tous les rapports, aux corporations de métiers si communes dans les villes des divers pays de l’Europe, et ce sont des espèces de monopoles étendus à un grand nombre de personnes, telles que sont les corporations. De même qu’aucun habitant d’une ville ne peut exercer un métier incor­poré sans obtenir d’abord sa maîtrise dans la corporation ; de même, la plupart du temps, aucun sujet de l’État ne peut légalement exercer une branche de commerce étranger pour laquelle on a établi une compagnie privilégiée, sans devenir d’abord membre de cette compagnie. Le monopole est plus ou moins resserré, selon que les conditions pour l’admission sont plus ou moins difficiles à remplir, et selon que les directeurs de la compagnie ont plus ou moins d’autorité, ou qu’ils ont plus ou moins la faculté d’arranger les choses de manière à ce que la plus grande partie de ce com­merce soit réservée pour eux et leurs amis particuliers. Dans les plus anciennes compagnies privilégiées, les privilèges d’apprentissage ont été les mêmes que dans les autres corporations, et ils autorisaient celui qui avait servi son temps sous un membre de la compagnie à en devenir membre lui-même sans payer aucun droit d’entrée, ou en en payant un beaucoup moindre que celui que l’on exigeait des autres. L’esprit ordinaire de corporation domine dans toutes les compagnies privilégiées, partout où la loi ne lui prescrit pas de bornes. Quand on a laissé agir ces compagnies d’après leur pente naturelle, elles ont toujours cherché à assujettir le commerce à une foule de règlements onéreux, afin de restreindre la concurrence au plus petit nombre possible de personnes. Quand la loi les a empêchées d’agir de cette manière, elles sont devenues tout à fait inutiles et parfaitement nulles.
Les compagnies privilégiées pour le commerce étranger qui subsistent actuelle­ment dans la Grande-Bretagne sont : l’ancienne compagnie des commerçants à l’aventure, appelée communément aujourd’hui compagnie de Hambourg, la compa­gnie de Russie, la compagnie des Terres orientales, la compagnie de Turquie, et la compagnie d’Afrique.
Les conditions pour l’admission dans la compagnie de Hambourg sont aujour­d’hui, dit-on, extrêmement faciles, et les directeurs de cette compagnie ou n’ont pas le pouvoir d’assujettir ce commerce à quelques gênes ou règlements onéreux, ou au moins depuis longtemps ne l’exercent point. Il n’en a pas toujours été de même. Vers le milieu du dernier siècle, le droit d’entrée était de 50 liv., il a été une fois de 100 liv. ; on assure que la conduite de la compagnie était extrêmement oppressive. En 1643, 1645 et 1661, les drapiers et les corps des marchands de l’ouest de l’Angleterre se plaignirent au parlement de ceux qui composaient cette compagnie, comme de monopoleurs qui gênaient le commerce et opprimaient les manufactures du pays. Quoi­que ces plaintes n’aient donné lieu à aucun acte du parlement, elles ont néan­moins probablement intimidé assez la compagnie pour l’obliger à réformer sa conduite. Au moins, depuis ce temps, n’y eut-il plus de plaintes contre elle.
Par le statut des dixième et onzième années de Guillaume III, ch. vi, le droit d’entrée pour l’admission dans la compagnie de Russie fut réduit à 5 liv., et par celui de la vingt-cinquième de Charles II, chap. viii, le droit d’entrée pour l’admission dans la compagnie des Terres orientales[18], à 40 sch., tandis qu’au même temps on excepta de leur charte exclusive la Suède, le Danemark et la Norvège, tous les pays au nord de la mer Baltique. C’est vraisemblablement la conduite de ces compagnies qui a donné lieu à ces deux actes du parlement. Avant cette époque, sir Josias Child avait repré­senté ces deux compagnies et celle de Hambourg comme extrêmement oppressives, et il avait imputé à leur mauvaise administration le misérable état du commerce que nous faisions alors avec les pays compris dans leurs chartes respectives. Mais, si ces sortes de compagnies ne sont pas actuellement très-gênantes pour le commerce, au moins lui sont-elles certainement tout à fait inutiles. Être purement inutiles est peut-être, à la vérité, le plus grand éloge qu’on puisse jamais faire avec justice d’une com­pagnie privilégiée, et ces trois compagnies paraissent, dans leur état actuel, mériter cet éloge.
Le droit d’entrée pour l’admission dans la compagnie de Turquie était ancien­nement de 25 liv. pour toutes personnes au-dessous de vingt-six ans, et de 50 liv. pour toutes celles au-dessus de cet âge. Personne autre que les commerçants pro­prement dits n’y pouvait être admis ; restriction qui excluait tous les marchands en boutique et en détail. Par un des statuts de la compagnie, aucun ouvrage de fabrique anglaise ne pouvait être exporté en Turquie que dans des vaisseaux appartenant en commun à la compagnie ; et comme ces vaisseaux faisaient toujours voile du port de Londres, cette restriction limita le commerce à ce port dispendieux, et ne le permit qu’aux commerçants qui demeuraient à Londres et dans le voisinage. Par un autre de ces statuts, tout particulier résidant dans la distance de vingt milles de Londres, et non reçu bourgeois[19] de la ville, ne pouvait être admis comme membre ; autre restriction qui, jointe à la précédente, excluait nécessairement tout ce qui n’était pas reçu bourgeois de Londres. Comme le temps pour le chargement et le départ de ces vaisseaux de la compagnie dépendait totalement des directeurs, il leur était aisé de les remplir de leurs propres marchandises et de celles de leurs amis particuliers, à l’exclusion des autres, qui étaient censés avoir fait leurs demandes trop tard. Ainsi, dans cet état de choses, cette compagnie était, sous tous les rapports, un monopole très-sévère et très-oppressif. Ces abus donnèrent lieu à l’acte de la vingt-sixième année de Georges II, chap. xviii, qui réduisait le droit d’entrée pour l’admission à 20 liv. pour toutes personnes, sans distinction d’âge, et sans privilège quelconque, ni en faveur des commerçants proprement dits, ni en faveur des bourgeois de Londres, et qui accorda à toutes personnes ainsi admises la liberté d’exporter, de tous les ports de la Grande-Bretagne à l’un des ports de la Turquie, toutes marchandises anglaises dont l’exportation était permise, ainsi que d’importer de là toutes les marchandises turques dont l’importation n’était pas prohibée, en payant tant les droits généraux de douanes, que les droits particuliers établis pour subvenir aux dépenses nécessaires de la compagnie, et en se soumettant en même temps à l’autorité légitime des ambassa­deurs et consuls de la Grande-Bretagne résidant en Turquie, ainsi qu’aux statuts de la compagnie dûment arrêtés. Pour prévenir toute oppression dans la disposition de ces statuts, il fut ordonné par le même acte que, si sept membres de la compagnie se croyaient lésés par quelque statut porté depuis la date de cet acte, ils pourraient en appeler à la chambre de commerce et des colonies (à l’autorité de laquelle a mainte­nant succédé un comité de conseil privé), pourvu que l’appel fût porté dans les douze mois après que le statut aurait été arrêté ; et que, si sept membres se trouvaient lésés par quelque statut qui eût été arrêté avant la date de cet acte, ils pourraient interjeter un semblable appel, pourvu que ce fût dans les douze mois à partir de la date dudit acte. Cependant, l’expérience d’une année peut bien n’être pas toujours suffisante pour découvrir à tous les membres d’une grande compagnie les conséquences dangereuses d’un statut particulier ; et si plusieurs d’entre eux venaient à s’en apercevoir dans la suite, alors ni la chambre de commerce ni le comité du conseil ne pouvaient plus y rien réformer.
D’ailleurs, l’objet de la plus grande partie des statuts de toutes les compagnies privi­lé­giées, aussi bien que de toutes les autres corporations, est bien moins d’opprimer ceux qui sont déjà membres, que de décourager les autres de le devenir ; ce qui peut se faire non-seulement par de gros droits d’entrée, mais encore par beaucoup d’autres moyens. Le but constant de ces compagnies est toujours d’élever le taux de leurs profits aussi haut qu’elles le peuvent ; de tenir le marché aussi dégarni qu’il leur est possible, tant pour les marchandises dont elles font l’exportation, que pour celles qu’elles importent ; ce qui ne peut se faire qu’en gênant la concurrence ou en découra­geant de nouveaux concurrents de courir les hasards de ce commerce. D’ailleurs, un droit d’entrée, même de 20 livres seulement, s’il n’est peut-être pas assez fort pour décourager qui que ce soit d’entrer dans le commerce de Turquie, avec l’intention de continuer ce commerce, peut néanmoins l’être assez pour décourager un spéculateur de hasarder dans ce commerce une affaire particulière. Dans tout commerce quelcon­que, les marchands qui y ont fixé leur établissement, quand même ils ne seraient pas en corporation, se liguent naturellement pour faire monter leurs profits ; et il n’y a rien qui soit plus dans le cas de tenir en tout temps ces profits baissés à leur juste niveau, que la concurrence accidentelle de ces spéculateurs qui viennent par moments y tenter l’aventure. Quoique le commerce de Turquie paraisse avoir été à un certain point ouvert à tout le monde par cet acte du parlement, néanmoins beaucoup de gens le regardent encore comme bien loin d’être entièrement libre. La compagnie de Turquie contribue à entretenir un ambassadeur et deux ou trois consuls qui devraient, comme tous les autres ministres publics, être totalement entretenus aux frais de l’État, et tenir le commerce ouvert à tous les sujets de Sa Majesté. Les différentes taxes levées par la compagnie pour cet objet et pour d’autres arrangements de corporation pourraient fournir un revenu beaucoup plus que suffisant pour mettre l’État à même d’entretenir les ministres nécessaires.
Sir Josias Child a observé que, quoique les compagnies privilégiées eussent sou­vent entretenu des ministres publics, elles n’avaient néanmoins jamais entretenu de forts ou de garnisons dans les contrées où elles avaient commencé, tandis que les compagnies par actions l’ont souvent fait. En effet, les premiers paraissent être beaucoup moins propres que les autres pour faire faire ce genre de service. D’abord, les directeurs d’une compagnie privilégiée n’ont pas d’intérêt particulier à la prospérité du commerce de la compagnie en général, qui est l’objet pour lequel on entretient ces forts et ces garnisons. Le dépérissement de ce commerce général peut même souvent contribuer à l’avantage de leur commerce particulier, et il peut, en diminuant le nombre de leurs concurrents, les mettre à même d’acheter à meilleur marché et de vendre plus cher. Les directeurs d’une compagnie par actions, au contraire, n’ayant autre chose que leur part dans les profits qui se font avec le capital commun confié à leur administration, n’ont pas à eux de commerce particulier dont l’intérêt puisse être différent de celui du commerce général de la compagnie. Leur intérêt privé est lié à la prospérité de ce commerce général, et il est lié à l’entretien des forts et des garnisons destinés à les protéger. Par conséquent, ils sont plus dans le cas d’avoir cette attention soigneuse et continuelle qu’exige nécessairement cet entretien. En second lieu, les directeurs d’une compagnie par actions ont toujours le maniement d’un gros capital, celui qui compose le fonds de la société, duquel ils peuvent souvent employer une partie d’une manière convenable, à bâtir, à réparer et à entretenir ces forts et garnisons nécessaires. Mais les directeurs d’une compagnie privilégiée n’ayant le maniement d’aucun capital commun, n’ont pas d’autres fonds à employer à de telles dépenses que le revenu casuel provenant des droits d’entrée payés aux admissions, et des taxes de corporation établies sur le commerce de la compagnie. Ainsi, quand même ils auraient le même intérêt à veiller à l’entretien de forts et de garnisons semblables, ils ne pourraient guère avoir les mêmes moyens de rendre leur vigilance aussi efficace. L’entretien d’un ministre public n’exigeant presque aucune surveillance et n’occasion­nant qu’une dépense bornée et médiocre, c’est une chose beaucoup plus convenable à la constitution et aux facultés des compagnies privilégiées.
Cependant, longtemps après sir Josias Child, en 1750, on établit une compagnie privilégiée, la compagnie actuelle des marchands faisant le commerce d’Afrique, laquelle fut expressément chargée d’abord de l’entretien de tous les forts et garnisons de la Grande-Bretagne situés entre le cap Blanc et le cap de Bonne-Espérance, et ensuite de ceux seulement situés entre celui-ci et le cap Rouge. L’acte qui établit cette compagnie (de la vingt-troisième année de Georges II, chap. xxxi), paraît avoir en vue deux objets distincts : le premier, de restreindre d’une manière efficace cet esprit d’oppression et de monopole qui est naturel aux directeurs d’une compagnie privilé­giée ; le second, de les obliger, autant que possible, à donner à l’entretien des forts et garnisons une attention qu’il ne leur est pas naturel d’y donner.
Pour remplir le premier de ces deux objets, le droit d’entrée pour l’admission est fixé à 40 sch. Il est défendu à la compagnie de commercer en corps ou sur une asso­ciation de fonds ; d’emprunter de l’argent sous une obligation commune, ou d’établir aucune gêne sur le commerce, tout sujet de la Grande-Bretagne étant libre de faire ce commerce de toutes les places du royaume en payant le droit d’entrée. Le gouver­nement de la compagnie est composé d’un comité de neuf personnes qui s’assemblent à Londres, annuellement élues, par les bourgeois de Londres, Bristol et Liverpool, membres de la compagnie, et choisies en nombre égal dans chacune de ces villes. Il fut statué qu’un membre de la compagnie ne pourrait être continué dans sa place plus de trois ans consécutifs ; qu’un membre du comité pourrait être destitué par la chambre du commerce et des colonies (aujourd’hui par un comité du conseil), après avoir été entendu dans sa défense. Il est défendu aux membres composant le comité des neuf d’exporter des nègres de l’Afrique et d’importer aucune marchandises d’Afri­que en Grande-Bretagne. Mais comme ils sont chargés d’entretenir des forts et garnisons, ils peuvent, pour cet objet, exporter de la Grande-Bretagne en Afrique des marchandises et munitions de différentes sortes. Sur les fonds qu’ils touchent de la compagnie, il leur est alloué une somme qui ne peut excéder 800 livres pour les salaires de leurs secrétaires et agents à Londres, Bristol et Liverpool, le loyer de leur bureau à Londres et tous les autres frais de régie, agence et commission en Angleterre. Toutes ces dépenses défrayées, ils peuvent partager entre eux, comme ils le jugent à propos, ce qui reste de cette somme, à titre d’indemnité de leurs peines. D’après la constitution de cette compagnie, on aurait pu s’attendre que l’esprit du monopole y aurait été réprimé d’une manière efficace, et que le premier des deux objets de la loi aurait été suffisamment rempli. Toutefois, il paraîtrait qu’il ne l’a pas été. Quoique, par l’acte de la quatrième année de Georges III, chap. xx, le fort de Sénégal, avec toutes dépendances, eût été cédé à la compagnie des marchands faisant le commerce d’Afrique, cependant l’année suivante (par l’acte de la cinquième année de Georges III, chap. xliv), non-seulement le Sénégal et ses dépendances, mais toute la côte, depuis le port de Salé, au midi de la Barbarie, jusqu’au cap Rouge, fu­rent distraits de la juridiction de cette compagnie, remis entre les mains de la couron­ne, et le commerce de cette partie déclaré libre pour tous les sujets de Sa Majesté. La compagnie avait été soupçonnée de comprimer le commerce et de s’être attribué quelque monopole illégal. Il n’est cependant pas bien aisé de comprendre comment elle pouvait en venir à bout avec toutes les restrictions portées par l’acte de la vingt-troisième de Georges II. Toutefois, je remarque dans les débats imprimés de la Chambre des communes, qui ne sont pas toujours les registres les plus authentiques de la vérité, que ces accusations ont été portées contre la compagnie. Les membres du comité des neuf étant tous commerçants, et les gouverneurs et facteurs des différents forts et établissements de la compagnie étant sous leur dépendance, il n’est pas hors de vraisemblance que ceux-ci aient donné une attention plus particulière aux com­missions et expéditions de marchandises venant de la part des premiers ; ce qui aurait établi un véritable monopole.
Pour remplir le second objet de la loi, l’entretien des forts et garnisons, il leur a été accordé par le parlement une somme annuelle, montant communément à environ 13,000 liv. Pour justifier de l’emploi de cette somme, le comité est obligé de compter annuellement devant le baron cursitor de l’échiquier[20], et le compte est ensuite mis sous les yeux du parlement. Mais le parlement, qui donne si peu d’attention à l’emploi de millions, n’en donne vraisemblablement pas beaucoup à l’emploi d’une somme de 13,000 liv. par année, et le baron cursitor de l’Échiquier, par sa profession et le genre de son éducation, n’est pas probablement très-profondément versé dans la connais­sance des dépenses convenables pour des forts et garnisons. À la vérité, les capitaines des vaisseaux de Sa Majesté ou quelques autres officiers en commission, nommés par la chambre de l’amirauté, peuvent inspecter l’état des forts et garnisons, et faire le rapport de leurs observations à la chambre. Mais il ne paraît pas que cette chambre ait aucune juridiction directe sur le comité, ni qu’elle ait aucun pouvoir de punir ceux dont elle peut ainsi inspecter la conduite ; et d’ailleurs, les capitaines des vaisseaux de Sa Majesté ne sont pas censés toujours parfaitement instruits dans la science des fortifications. La destitution d’une place dont on ne peut pas jouir pour un plus long terme que trois années, et dont les émoluments légitimes, même pendant ce terme, sont si faibles, paraît être l’extrême punition à laquelle soit exposé un membre du comité, pour quelque faute que ce soit (excepté une malversation directe ou un dé­tour­nement de deniers, soit deniers publics, soit ceux de la compagnie) ; et la crainte d’une semblable punition ne peut jamais être un motif d’un assez grand poids pour l’engager à apporter une vigilance soigneuse et continuelle à laquelle il n’a pas d’autre intérêt qui l’oblige. Le comité a été accusé d’avoir expédié d’Angleterre des briques et de la pierre pour la réparation du château de la Côte-du-Cap, sur la côte de Guinée, chose pour laquelle le parlement avait accordé plusieurs fois une somme extraor­dinaire. De plus, ces briques et ces pierres, qui avaient été ainsi envoyées de si loin, se trouvèrent, dit-on, de si mauvaise qualité, qu’il fut nécessaire de rebâtir, depuis les fondations, les murs qui avaient été réparés avec ces matériaux. Les forts et garnisons qui sont au nord du cap Rouge, non-seulement sont entretenus aux frais de l’État, mais encore sont sous le gouvernement immédiat du pouvoir exécutif ; et pourquoi ceux situés au sud de ce cap, et qui sont aussi, en partie au moins, entretenus aux dépens de l’État, seraient-ils sous un autre gouvernement ? C’est ce dont il n’est pas aisé d’imaginer une bonne raison. Le but primitif ou le prétexte des garnisons de Minorque et de Gibraltar, ce fut la protection du commerce de la Méditerranée ; et cependant l’entretien et le gouvernement de ces garnisons ont toujours été commis, comme il est très-convenable, non pas a la compagnie de Turquie, mais au pouvoir exécutif. L’éclat et la dignité de ce pouvoir consistent, en grande partie, dans l’étendue de sa domination ; et il n’est guère vraisemblable qu’il manque d’attention dans tout ce qui est nécessaire pour défendre les domaines qui lui sont soumis. Aussi, les garnisons de Minorque et de Gibraltar n’ont-elles jamais été négligées. Si Minorque a été prise deux fois, et est probablement à présent perdue pour toujours, ce malheur même n’a jamais été imputé à aucune négligence du pouvoir exécutif. je ne voudrais pourtant pas qu’on pût croire que je prétends insinuer par là que l’une ou l’autre de ces deux garnisons si dispendieuses ait jamais été, même le moins du monde, nécessaire à l’objet pour lequel elles ont été originairement démembrées de la couronne d’Espagne. Ce démembrement n’a peut-être jamais eu d’autre véritable effet que d’aliéner de l’Angleterre le roi d’Espagne, son allié naturel, et de faire naître entre les deux bran­ches principales de la maison de Bourbon une alliance plus étroite et plus permanente que celle qu’eussent jamais pu produire les liens du sang.
Les compagnies par actions établies ou par charte royale, ou par acte du parle­ment, différent, à beaucoup d’égards, non-seulement des compagnies privilégiées, mais même des sociétés particulières de commerce[21]. 
Premièrement, dans une société particulière, aucun associé ne peut, sans le con­sen­tement de la société, transporter sa part d’associé à une autre personne, ou intro­duire un nouveau membre dans la société. Cependant, chaque membre peut, après un avertissement convenable, se retirer de l’association et demander le payement de sa portion dans les fonds communs de la société. Dans une société par actions, au con­traire, aucun membre ne peut demander à la compagnie le payement de sa part, mais chaque membre peut, sans le consentement de la compagnie, céder sa part d’associé à une autre personne, et par là introduire dans la compagnie un nouveau membre. La valeur d’une part ou action dans une société de ce genre est toujours le prix qu’on en trouvera sur la place, et ce prix peut être, sans nulle proportion, au-dessus ou au-dessous de la somme pour laquelle le propriétaire est crédité dans les fonds de la compagnie.
Secondement, dans une société particulière de commerce, chaque associé est obligé aux dettes de la société pour toute l’étendue de sa fortune. Dans une compa­gnie par actions, au contraire, chaque associé n’est obligé que jusqu’à concurrence de sa part d’associé.
Le commerce d’une compagnie par actions est toujours conduit par un corps de directeurs. À la vérité, ce corps est souvent sujet, sous beaucoup de rapports, au contrôle de l’assemblée générale des propriétaires. Mais la majeure partie de ces propriétaires ont rarement la prétention de rien entendre aux affaires de la compagnie, mais bien plutôt, quand l’esprit de faction ne vient pas à régner entre eux, tout ce qu’ils veulent c’est de ne se donner aucun souci là-dessus, et de toucher seulement l’année ou les six mois de dividende, tels que la direction juge à propos de les leur donner, et dont ils se tiennent toujours contents. L’avantage de se trouver absolument délivré de tout embarras et de tout risque au-delà d’une somme limitée, encourage beaucoup de gens (qui, sous aucun rapport, ne voudraient hasarder leur fortune dans une société particulière) à prendre part au jeu des compagnies par actions. Aussi, ces sortes de compagnies attirent à elles des fonds beaucoup plus considérables qu’aucune société particulière de commerce ne peut se flatter d’en réunir. Le capital de la compagnie de la mer du Sud se trouva monter une fois à plus de 33 millions 800 mille liv. Le capital, portant dividende, de la Banque d’Angleterre monte actuellement à 10 millions 780 mille liv. Néanmoins, les directeurs de ces sortes de compagnies étant les régisseurs de l’argent d’autrui plutôt que de leur propre argent, on ne peut guère s’attendre à ce qu’ils y apportent cette vigilance exacte et soucieuse que les associés d’une société apportent souvent dans le maniement de leurs fonds. Tels que les intendants d’un riche particulier, ils sont portés à croire que l’attention sur les petites choses ne conviendrait pas à l’honneur de leurs maîtres, et ils se dispensent très-aisément de l’avoir. Ainsi, la négligence et la profusion doivent toujours dominer plus ou moins dans l’administration des affaires de la compagnie. C’est pour cette raison que les compagnies par actions pour le commerce étranger ont rarement été en état de soutenir la concurrence contre les particuliers qui se sont aventurés dans le même commerce. Aussi, ont-elles très-rarement réussi sans l’aide d’un privilège exclusif, et souvent encore elles n’ont pas réussi même avec cette aide. Sans privilège exclusif, elles ont ordinairement mal dirigé le commerce dont elles se sont mêlées ; avec le privilège exclusif, elles l’ont mal dirigé et l’ont comprimé tout à la fois.
La compagnie royale d’Afrique, qui a précédé la compagnie actuelle d’Afrique, avait un privilège exclusif par charte ; mais, comme cette charte n’avait pas été confir­mée par acte du parlement, le commerce fut bientôt, après la Révolution, en consé­quence de la déclaration des droits, laissé ouvert à tous les sujets de Sa Majesté. La compagnie de la baie d’Hudson, quant à ses droits légaux, est dans la même situa­tion que la compagnie royale d’Afrique. Sa charte exclusive n’a pas été confirmée par acte du parlement. La compagnie de la mer du Sud, tant qu’elle demeura compagnie com­mer­çante, eut un privilège exclusif confirmé par acte du parlement, comme l’a pareil­le­ment la compagnie actuelle des marchands faisant le commerce aux Indes Orientales.
La compagnie royale d’Afrique s’aperçut bientôt qu’elle ne pouvait soutenir la con­­currence contre les particuliers qui se livraient à son genre de commerce, et dont pendant quelque temps, malgré la déclaration des droits, elle traita le commerce de commerce interlope et le persécuta même comme tel. Néanmoins, en 1698, ces com­merçants particuliers furent assujettis à un droit de 10 pour 100 sur presque toutes les différentes branches de leur commerce, pour servir à l’entretien des forts et garnisons de la compagnie. Mais, malgré cette forte taxe, la compagnie fut toujours hors d’état de soutenir la concurrence. Son capital et son crédit vinrent à dépérir successivement. En 1712, ses dettes étaient devenues si considérables, qu’on pensa qu’un acte du parlement était nécessaire, autant pour sa sûreté que pour celle de ses créanciers. Il fut statué qu’une délibération, consentie par les deux tiers de ses créanciers en nombre et en valeur, serait obligatoire contre le reste, tant à l’égard des délais qu’on pourrait accorder à la compagnie pour le payement de ses dettes, qu’à l’égard de tout autre accord qu’on pourrait trouver convenable de faire avec elle au sujet de ces dettes. En 1730, ses affaires étaient en un si grand désordre, qu’elle se trouva absolument hors d’état d’entretenir ses forts et garnisons, le seul objet ou prétexte de son institution. Depuis cette année jusqu’à sa dissolution finale, le parlement jugea indispensable de lui accorder pour cet objet une somme annuelle de 10,000 liv. et, en 1732, après avoir fait avec perte, pendant plusieurs années, le commerce de transporter des nègres aux Indes Occidentales, la compagnie prit à la fin le parti de l’abandonner tout à fait ; se contentant de vendre aux commerçants particuliers qui faisaient le commerce avec l’Amérique les nègres qu’elle achetait sur la côte, et d’employer ses agents à commercer avec l’intérieur de l’Afrique pour en avoir de la poudre d’or, des dents d’éléphant, des drogues pour la teinture, etc. Mais ses succès dans ce commerce borné ne furent pas plus grands que dans son premier commerce plus étendu. Ses affaires continuèrent toujours à aller par degrés de mal en pis, jusqu’à ce qu’enfin devenue, sous tous les rapports, une compagnie banqueroutière, elle fut dissoute par acte du parlement, et ses forts et garnisons remis entre les mains de la compagnie privilégiée qui existe aujourd’hui sous le nom de compagnie des marchands faisant le commerce d’Afrique. Avant que la compagnie royale d’Afrique fût érigée, il y avait eu succes­si­vement trois autres compagnies par actions établies l’une après l’autre pour le com­merce d’Afrique. Elles furent toutes également malheureuses. Cependant, elles eurent toutes des chartes exclusives, qui ne furent pas, à la vérité, confirmées par acte du parlement, mais qui n’en étaient pas moins dans ce temps-là réputées transmettre un véritable privilège exclusif. 
La compagnie de la baie d’Hudson, avant les malheurs qu’elle éprouva dans la dernière guerre, avait eu beaucoup plus de succès que la compagnie royale d’Afrique. Ses dépenses nécessaires sont beaucoup moindres. La totalité des personnes qu’elle entretient, dans les différents établissements qu’elle a décorés du nom de forts, n’excède pas, dit-on, cent vingt personnes, ce nombre est néanmoins tout ce qu’il faut pour préparer d’avance les fourrures et autres marchandises formant la cargaison de ses vaisseaux, qui, à cause des glaces, ne peuvent guère rester dans ces mers plus de six ou huit semaines. Des armateurs particuliers qui se livreraient à ce commerce ne pourraient pas, avant plusieurs années, se procurer l’avantage d’avoir ainsi leurs cargaisons préparées d’avance, et sans cela il ne paraît pas qu’il y ait possibilité de commercer à la baie d’Hudson ; d’ailleurs, le modique capital de la compagnie, qui, à ce qu’on dit, ne va pas au-delà de 110 mille livres, peut suffire pour la mettre à portée d’accaparer la totalité ou la presque totalité du commerce et du produit superflu du misérable pays (tout étendu qu’il soit) qui est compris dans sa charte ; aussi, aucun particulier n’a-t-il jamais essayé de commercer dans ce pays en concurrence avec elle. Par conséquent, si cette compagnie n’a pas, aux yeux de la loi, de droit à un com­merce exclusif, elle en a toujours joui par le fait. Par-dessus tout cela encore, on dit que le modique capital de cette compagnie est partagé entre un très-petit nombre de propriétaires. Or, une compagnie par actions composée d’un petit nombre d’action­naires, avec un capital modique, approche de très-près de la nature d’une société particulière de commerce, et peut être susceptible, à fort peu de chose près, du même degré de vigilance et d’attention. Il ne faut donc pas s’étonner si, en conséquence de ces différents avantages, la compagnie de la baie d’Hudson a pu, avant la dernière guerre, faire son commerce avec un degré de succès un peu considérable. Il ne paraît pourtant pas vraisemblable que ses profits aient jamais approché de ce que s’est figuré M. Dobbs. Un écrivain beaucoup plus judicieux et plus circonspect, M. Anderson, auteur du Traité historique et chronologique du commerce, observe avec beaucoup de justesse, qu’en examinant les comptes donnés par M. Dobbs lui-même, pendant plu­sieurs années de suite, des exportations et importations de la compagnie, et en mettant en ligne de compte les sommes convenables pour les risques et les frais extraordi­naires, il ne paraît pas que les profits de la compagnie soient dans le cas d’être enviés, ou qu’ils excèdent de beaucoup les profits ordinaires du commerce, en supposant même qu’ils les excèdent.
La compagnie de la mer du Sud n’a jamais eu ni forts ni garnisons à entretenir et, par conséquent, elle a toujours été exempte d’une grande dépense à laquelle sont sujettes les autres compagnies par actions pour le commerce étranger ; mais elle avait un immense capital divisé entre un nombre immense de propriétaires. On devait donc naturellement s’attendre à ce que l’imprévoyance, la négligence et la prodigalité ré­gne­raient dans toute l’administration de ses affaires. On ne connaît que trop l’extra­vagance et les manœuvres frauduleuses de ses projets d’agiotage, et ce serait une explication étrangère au sujet présent ; ses projets mercantiles n’ont pas été beaucoup mieux conduits. Le premier commerce qu’elle entreprit, ce fut celui de fournir de nègres les Indes Occidentales espagnoles ; elle avait le privilège exclusif de ce com­mer­ce, par suite de ce qu’on appela le contrat d’Asiento, à elle cédé par le traité d’Utrecht ; mais, comme il n’y avait pas lieu de s’attendre à ce qu’elle pût faire de grands profits à ce commerce, les compagnies française et portugaise, qui en avaient joui avant elle aux mêmes conditions, s’y étant ruinées l’une et l’autre, on lui permit, par forme de compensation, d’envoyer annuellement un vaisseau d’une charge déter­minée, pour commercer directement avec les Indes Occidentales espagnoles. De dix voyages qu’on permit de faire à ce vaisseau annuel[22], on dit qu’un seul, celui de la Royale Caroline, en 1731, lui a rapporté un bénéfice considérable, et qu’elle a été plus ou moins en perte dans presque tous les autres. Les facteurs et agents de la com­pa­gnie imputèrent ce mauvais succès aux extorsions et aux vexations du gouverne­ment d’Espagne ; mais c’était peut-être principalement à la prodigalité et aux dépré­dations de ces facteurs et agents eux-mêmes qu’il fallait l’attribuer ; on dit que quelques-uns d’eux ont fait de grandes fortunes, même dans l’espace d’une année. En 1734, la compagnie présenta au roi une pétition pour obtenir la permission de dispo­ser du commerce et du tonnage de son vaisseau annuel, à cause du peu de profits qu’elle y faisait, et d’accepter en équivalent ce qu’elle pourrait obtenir du roi d’Espagne. 
En 1724, cette compagnie avait entrepris la pêche de la baleine ; à la vérité, elle n’avait sur cet article aucun monopole ; mais tant qu’elle continua cette entreprise, il ne paraît pas qu’aucun autre sujet de la Grande-Bretagne se soit livré à ce genre de commerce. De huit voyages que ces vaisseaux firent au Groenland, un seul lui rapporta du bénéfice ; elle fut en perte dans les autres. Après son huitième et dernier voyage, quand elle eut vendu ses vaisseaux, agrès, munitions et ustensiles, elle trouva que la totalité de ses pertes dans cette branche, capital et intérêts compris, se montait au-delà de 237 mille livres.
En 1722, la compagnie présenta au parlement une pétition pour obtenir la per­mission de partager en deux portions égales son énorme capital de plus de 33 millions 800 mille livres, dont la totalité avait été prêtée au gouvernement, desquelles portions l’une, faisant moitié de ce capital ou plus de 16 millions 900 mille livres, serait mise sur le même pied que les autres annuités du gouvernement, et ne serait plus assujettie aux dettes ni aux pertes que les directeurs de la compagnie pourraient faire dans la poursuite de leurs projets de commerce, et l’autre moitié resterait, comme auparavant, fonds de com­merce, et assujettie à ces dettes et à ces pertes. La pétition était trop raisonnable pour n’être pas accueillie. En 1733, elle demanda au parlement, par une autre pétition, que les trois quarts de ses fonds de commerce fussent convertis en capital d’annuités, et qu’un quart seulement restât en fonds de commerce, c’est-à-dire exposé aux risques de la mauvaise administration de ses directeurs. Dans cet espace de temps, son capital d’annuités et son capital de commerce avaient été l’un et l’autre réduits de plus de 2 millions chacun, par plusieurs remboursements faits par le gouvernement ; de sorte que ce quart ne montait plus qu’à 3,662,784 liv. 8 sch. 6 den. En 1748, toutes les répétitions de la compagnie contre le roi d’Espagne, résultant du contrat de l’Asiento, furent abandonnées par le traité d’Aix-la-Chapelle, pour ce qui fut réputé en équi­valent[23] ; ceci mit fin à son commerce avec les Indes Occidentales espagnoles ; le reste de ses fonds de commerce fut converti en fonds d’annuités, et la compagnie cessa, sous tous les rapports, d’être une compagnie de commerce. 
J’aurais dû observer que, dans le commerce que fit la compagnie de la mer du Sud par le moyen de son vaisseau annuel, le seul commerce sur lequel on se soit jamais attendu qu’elle ait pu faire quelque profit considérable, elle ne fut pas sans concur­rents, soit dans le marché intérieur, soit dans le marché étranger. À Carthagène, à Porto-Bello, à la Vera-Cruz, elle avait contre elle la concurrence des marchands espa­gnols qui apportaient de Cadix à ces trois marchés des marchandises européennes de la même espèce que celles qui composaient la cargaison d’exportation de son vais­seau, et en Angleterre elle avait contre elle la concurrence des marchands anglais qui importaient de Cadix des marchandises des Indes Occidentales espagnoles, de même espèce que celles qui composaient sa cargaison d’importation. À la vérité, les mar­chandises tant des marchands anglais que des marchands espagnols étaient peut-être assujetties à des droits plus forts que celles de la compagnie ; mais probablement les pertes causées par la négligence, la profusion et les malversations des agents de la compagnie étaient une taxe beaucoup plus lourde que tous les droits possibles. Il paraît démontré par l’expérience la plus constante qu’une compagnie par actions ne saurait se soutenir avec succès dans aucune branche de commerce étranger, toutes les fois que les commerçants particuliers peuvent venir ouvertement et légalement en concurrence avec elle.
L’ancienne compagnie anglaise des Indes Orientales fut établie en 1600, par une charte de la reine Élisabeth. Dans les douze premiers voyages que ses vaisseaux firent aux Indes, il paraît qu’elle commerça comme compagnie privilégiée avec des capitaux séparés, quoique seulement dans les vaisseaux appartenant en commun à la com­pagnie. En 1612, elle s’organisa en société par actions. La charte était exclusive et, quoique non confirmée par acte du parlement, elle était dans ce temps-là réputée trans­mettre un privilège exclusif. Ainsi, pendant beaucoup d’années, elle ne fut pas très-contrariée par le commerce interlope. Son capital, qui n’alla jamais au-delà de 744000 liv., et dont l’action était de 50 liv., ne fut jamais assez exorbitant ni ses affaires assez étendues pour pouvoir fournir prétexte à beaucoup de négligence et de profusions, ou pour couvrir de grandes malversations. Malgré quelques pertes extra­ordinaires, causées en partie par la malveillance de la compagnie hollandaise des Indes Orientales, et en partie par d’autres accidents, elle fit le commerce avec beaucoup de succès pendant plusieurs années. Mais avec le temps, quand on vint à mieux entendre les principes de la liberté, on mit de plus en plus chaque jour en question jusqu’à quel point une charte royale, non confirmée par acte du parlement, pouvait donner un droit de privilège exclusif. Sur ce point, les décisions des cours de justice ne furent pas uniformes, mais elles varièrent avec l’autorité du gouvernement et l’esprit du temps. Le commerce interlope se multipliait au détriment de la com­pagnie, et vers la fin du règne de Charles Il, pendant tout celui de Jacques II et une partie de celui de Guillaume Ill, il réduisit la compagnie à une grande détresse. En 1689, le parlement reçut une soumission de faire au gouvernement une avance de 2 millions à 8 p. 100, sous condition que les souscripteurs seraient érigés en nouvelle compagnie des Indes Orientales, avec privilège exclusif. L’ancienne compagnie des Indes offrit 700,000 liv., presque le montant de son capital, à 4 p. 100, aux mêmes conditions. Mais telle était alors la situation du crédit public, qu’il convint mieux au gouvernement d’emprunter 2 millions à 8 p. 100, que 700,000 liv. à 4. On accepta la proposition des nouveaux souscripteurs, et une nouvelle compagnie des Indes Orientales fut établie en conséquence. L’ancienne compagnie eut pourtant le droit de continuer son commerce jusqu’en 1701. Elle avait en même temps eu l’habileté de souscrire, sous le nom de son trésorier, dans les fonds de la nouvelle, pour 315,000 liv. Par une négligence de rédaction dans l’acte du parlement qui investissait du com­merce aux Indes les souscripteurs de ce prêt de 2 millions, il n’était pas clairement exprimé qu’ils seraient obligés de s’unir tous en société par actions. Quelques com­merçants particuliers, dont les souscriptions montaient seulement à 7,200 livres, insistèrent sur le privilège de commercer séparément avec leurs propres fonds et à leurs risques. L’ancienne compagnie avait droit de commercer séparément sur ses anciens fonds jusqu’en 1701, et elle avait encore, tant avant qu’après ce terme,, tout comme les autres commerçants particuliers, le droit de prétendre commercer séparé­ment sur les 315,000 liv. de sa souscription dans les fonds de la nouvelle compagnie. La concurrence des deux compagnies entre elles et avec les commerçants particuliers les a, dit-on, presque ruinées toutes deux. Dans une autre occasion, en 1730, quand il fut proposé au parlement de mettre ce commerce sous la direction d’une compagnie privilégiée, et par là de le laisser en quelque sorte ouvert à tout le monde, la com­pagnie des Indes Orientales, en s’opposant à cette proposition, représenta, dans les termes les plus forts, quels avaient été jusqu’alors, suivant elle, les fâcheux effets de la concurrence ; cette concurrence, disait-elle, avait fait monter si haut le prix des marchandises dans l’Inde, qu’elles ne valaient pas la peine qu’on les y achetât, et en surchargeant le marché en Angleterre, elle y avait tellement fait baisser leur prix, qu’il n’y avait pas le moindre profit à faire. Que cette concurrence, en rendant l’approvi­sion­nement beaucoup plus abondant, ait extrêmement réduit le prix des marchandises de l’Inde sur le marché d’Angleterre, au grand avantage et à la grande commodité du public, c’est ce dont il n’est guère possible de douter ; mais qu’il ait beaucoup fait mon­ter leur prix sur le marché de l’Inde, c’est ce qui n’est guère vraisemblable, attendu que toutes les demandes extraordinaires que cette concurrence a pu occasionner ne doi­vent avoir été qu’une goutte d’eau dans l’immense océan du commerce des Indes. D’ailleurs, si l’augmentation de la demande fait quelquefois, dans les commen­ce­ments, monter le prix des marchandises, elle ne manque jamais de l’abaisser à la lon­gue. Cette augmentation encourage la production et augmente par là la concur­rence des producteurs, qui, pour se supplanter les uns les autres, ont recours à de nouvelles divisions de travail et à de nouveaux moyens de perfectionner l’industrie, auxquels ils n’auraient jamais pensé sans cela. Ces fâcheux effets dont se plaignait la compagnie, c’était le bon marché de la consommation et l’encouragement donné à la production, qui sont précisément les deux effets que se propose l’économie politique. En outre, on n’a pas laissé durer longtemps cette concurrence dont la compagnie faisait un portrait si lamentable. En 1702, les deux compagnies furent à un certain point réunies dans une société triple, dont la reine fut la troisième tête, et en 1708 elles furent parfaite­ment consolidées, par acte du parlement, en une compagnie subsistant actuellement sous le nom de compagnie des marchands unis pour le commerce aux Indes Orien­tales. On crut à propos d’insérer une clause dans cet acte, pour permettre à ceux qui faisaient le commerce séparément de le continuer jusqu’à la SaintMichel 1711 ; mais la même clause autorisa les directeurs à racheter, après un avertissement de trois an­nées, leur petit capital de 7,200 livres, et par là à convertir tout le capital de la com­pagnie en une mise commune de fonds. Par le même acte, le capital de la compagnie, en conséquence d’un nouveau prêt au gouvernement, fut porté, de 2 millions, à 3 millions 200,000 livres. En 1743, la compagnie avança un autre million au gouvernement. Ce million cependant, n’ayant pas été levé par un appel de fonds sur les actionnaires, mais par une vente d’annuités et en contractant, par la compagnie, des dettes par obligation, n’augmenta pas le capital sur lequel les actionnaires pouvaient prétendre un dividende. Il augmenta néanmoins le fonds de commerce de la compagnie, ce million étant assujetti aux pertes et aux dettes de la compagnie résultant de ses spéculations commerciales, tout comme y sont assujettis les autres 3 200 000 livres. Depuis 1708, ou au moins depuis 1711, cette compagnie étant débarrassée de tous concurrents, et en pleine et complète jouissance du monopole du commerce d’Angle­terre aux Indes Orientales, a commercé avec beaucoup de succès, et a donné sur les profits annuels un dividende modéré à ses actionnaires. Pendant la guerre de France, qui commença en 1741, elle se trouva, par l’ambition de M. Dupleix, gouver­neur français de Pondichéry, enveloppée dans les guerres du Carnate et dans les affaires politiques des princes indiens. Après plusieurs succès signalés et des pertes qui ne le furent pas moins, elle finit par perdre Madras, alors son principal établissement dans l’Inde. Il lui fut rendu par le traité d’Aix-la-Chapelle, et vers ce temps l’esprit de guerre et de conquête semble s’être emparé de ses agents dans l’Inde et ne les avoir plus quittés depuis. Pendant la guerre de France, qui commença, en 1755, les armes de la compagnie participèrent au bonheur général qui accompagna partout les dra­peaux de la Grande-Bretagne. Elle défendit Madras, prit Pondichéry, recouvra Calcutta et acquit un riche et vaste territoire, dont les revenus furent alors évalués à plus de 3 millions par an. Elle demeura en paisible possession de ce revenu pendant plusieurs années ; mais en 1767 le gouvernement revendiqua les acquisitions terri­toriales et le revenu en provenant, comme un droit appartenant à la couronne, et la compagnie consentit à payer au gouvernement, par forme de transaction sur cette prétention, 400,000 liv. par an. Elle avait avant ceci porté successivement son dividende environ de 6 à 10 p. 100, c’est-à-dire que, sur son capital de 3,200,000 liv., elle avait augmenté de 128,000 liv. la masse du dividende annuel, et que de 192,000 liv. elle l’avait portée à 320,000. Elle s’occupait vers cette époque de l’augmenter encore davantage, et de porter le taux du dividende à 12 et 112 pour 100, ce qui aurait rendu ses payements annuels à ses actionnaires égaux à ce qu’elle avait consenti à payer annuellement au gouvernement, c’est-à-dire 400 000 liv. Mais dans les deux années pendant lesquelles son accord avec le gouvernement devait avoir lieu, deux actes successifs du parlement lui défendirent d’élever davantage le taux du dividende. L’objet de ces actes était de la mettre à portée d’avancer un peu plus vite la liquidation de ses dettes, qu’on évaluait à cette époque au-delà de 6 à 7 millions sterling. En 1769, elle renouvela son accord avec le gouvernement pour cinq années de plus, et elle stipula que pendant le cours de ce terme il lui serait permis d’élever successi­ve­ment le taux du dividende jusqu’à 12 et 1/2 pour 100, en ne l’augmentant néanmoins jamais de plus de 1 pour 100 dans une année. Ainsi, cet accroissement de dividende, porté à son plus haut point, n’eût jamais grossi les payements annuels faits par la compagnie, tant à ses actionnaires qu’au gouvernement, que de 608 000 livres au-delà de ce qu’ils étaient avant ses acquisitions territoriales.
Nous avons déjà dit à quelle somme énorme on avait évalué le revenu de ces ac­quisitions territoriales, et par un compte rapporté en 1768 par la Cruttenden, vaisseau de la compagnie des Indes, le revenu net, toutes déductions faites et toutes charges militaires prélevées, fut porté à 2,048,747 liv. ; on annonça en même temps qu’elle possédait un autre revenu provenant en partie de terres, mais principalement de droits de douane qu’elle percevait à ses différents établissements, lequel montait à 439,000 liv. ; de plus, les profits de son commerce, d’après le témoignage rendu par son pré­sident devant la Chambre des communes, montaient à cette époque à 400,000 liv. au moins par an ; d’après celui de son agent comptable, à 500,000 liv. au moins ; d’après le compte le plus bas, ils étaient au moins égaux au plus fort dividende qui dût être payé à ses actionnaires. Un si grand revenu aurait certainement bien pu fournir à une augmentation de 608,000 liv. dans ses payements annuels et laisser, en outre, un très-gros fonds d’amortissement suffisant pour opérer en peu de temps la réduction de ses dettes ; néanmoins, en 1773, ses dettes, au lieu d’être réduites, se trouvèrent aug­men­tées de plusieurs articles, savoir : à la Trésorerie, une année arriérée du payement annuel de 400 000 liv. ; au bureau de douanes, des droits non acquittés ; à la Banque, une très-forte somme pour argent emprunté ; et quatrièmement enfin, des lettres de change tirées de l’Inde sur la compagnie, et imprudemment acceptées pour une valeur de plus de 1,200,000 liv. La détresse où la jetèrent toutes ces réclamations accumulées sur elle l’obligea non-seulement à réduire tout d’un coup son dividende à 6 pour 100, mais à se mettre à la merci du gouvernement et à solliciter d’abord la remise du paye­ment subséquent des 400,000 liv. annuelles, et ensuite un prêt de 1,400,000 liv. pour la sauver d’une banqueroute déclarée. Le grand accroissement de sa fortune n’avait, à ce qu’il semble, produit autre chose qu’un prétexte à ses agents de se livrer à de plus grandes profusions, et un moyen de couvrir de plus fortes malversations, les unes et les autres ayant augmenté même au-delà de la proportion de cette augmentation de fortune. La conduite de ses agents dans l’Inde, et la situation générale de ses affaires dans l’Inde et en Europe, furent le sujet d’une enquête parlementaire, en conséquence de laquelle on fit plusieurs changements très-importants dans la constitution de son gouvernement tant intérieur qu’extérieur. Ses principaux établissements dans l’Inde, Madras, Bombay et Calcutta, qui avaient été auparavant indépendants l’un de l’autre, furent soumis à un gouverneur général, assisté d’un conseil de quatre assesseurs, le parlement se réservant la première nomination de ce gouverneur et de ce conseil, dont la résidence fut fixée à Calcutta, établissement devenu aujourd’hui ce qu’était aupara­vant Madras, c’est-à-dire le plus important des établissements anglais dans l’Inde. Le tribunal du maire de Calcutta, institué dans l’origine pour le jugement des causes de commerce qui s’élevaient dans la ville et dans les environs, avait par degrés étendu sa juridiction à mesure de l’agrandissement de l’empire. On le réduisit alors, et on le borna à l’objet de son institution primitive ; on établit à sa place une nouvelle cour suprême de justice, composée d’un chef de justice et de trois juges à la nomination de la couronne. En Europe, on évalua à 1,000 liv. la quotité nécessaire pour autoriser un actionnaire à voter aux assemblées générales de la compagnie, au lieu de 500 liv., prix originaire d’une action ou intérêt dans les fonds de la compagnie. De plus, pour pouvoir voter même avec cette condition, il fut statué qu’il faudrait que le propriétaire de l’intérêt ou action de 1,000 liv. fût propriétaire au, moins depuis un an, s’il l’avait par achat et non par succession, au lieu de six mois, qui était le terme requis aupa­ravant. Le corps des vingt-quatre directeurs était élu auparavant tous les ans ; il fut alors statué que chaque directeur serait à l’avenir élu pour quatre années ; que cepen­dant six d’entre eux sortiraient de fonctions, par tour, chaque année, sans pouvoir être réélus à l’élection des six nouveaux directeurs de l’année suivante. On s’attendait qu’en conséquence de ces réformes les assemblées, tant des directeurs que des action­naires, seraient dans le cas de se conduire avec plus de dignité et plus de fermeté qu’elles n’en avaient ordinairement montre jusque-là. Mais il paraît impossible d’arri­ver par aucune réforme à rendre ces sortes d’assemblées, sous aucun rapport, propres à gouverner ou même à prendre quelque part dans le gouvernement d’un grand em­pire, parce que nécessairement la majeure partie des membres qui les composent auront toujours trop peu d’intérêt à la prospérité de cet empire, pour donner quelque attention sérieuse aux moyens qui pourraient atteindre ce but. Fort souvent un homme d’une grande fortune, quelquefois même un homme d’une fortune médiocre, veut acheter un intérêt de 1,000 liv. dans les fonds de la compagnie des Indes, uniquement pour l’influence qu’il espère acquérir par son droit de voter dans l’assemblée des propriétaires. Son action lui donne part, non pas à la vérité dans le droit de piller l’Inde, mais dans le droit de nommer ceux qui la pillent ; car, quoique cette nomina­tion se fasse par l’assemblée des directeurs, celle-ci est nécessairement plus ou moins sous l’influence des propriétaires d’actions, qui non-seulement élisent ces directeurs, mais quelquefois dirigent les nominations des agents dans l’Inde. Pourvu qu’il puisse jouir de cette influence pendant quelques années, et venir à bout de placer par là un certain nombre de ses amis, il ne s’occupe guère, le plus souvent, de ce que sera le dividende, ni même de ce que deviendra la valeur du capital sur lequel est fondé son droit de vote. Quant à la prospérité du grand empire dans le gouvernement duquel son droit de vote lui donne part, c’est ce dont il est très-rare qu’il se soucie le moins du monde. Il n’y a pas de souverains qui soient ou qui puissent jamais être, par la nature des choses, dans une aussi parfaite indifférence sur ce qui concerne le bonheur ou la misère de leurs sujets, la prospérité ou la ruine de leurs États, la gloire ou le déshon­neur de leur administration, que ne sont et que ne doivent y être nécessairement, par la force irrésistible des causes morales, la plus grande partie des propriétaires intéres­sés d’une pareille compagnie de commerce. Cette indifférence, en outre, était plus dans le cas d’augmenter que de diminuer, d’après quelques-uns des nouveaux arrange­ments qui avaient été faits en conséquence de l’enquête parlementaire. Par exemple, il fut déclaré par une résolution de la Chambre des communes, que quand les 1,400,000 liv. prêtées à la compagnie par le gouvernement seraient remboursées, et ses dettes contractées réduites à 1,500,000 liv., elle pourrait alors, et non avant, se partager un dividende de 8 p. 100 sur son capital, et que ce qui resterait en ses mains de revenus et profits nets serait divisé en quatre parts, trois desquelles seraient versées dans l’Échi­quier pour le service public, et la quatrième serait comme un fonds de réserve destiné ou à opérer une réduction ultérieure de ses dettes contractées, ou à acquitter d’autres charges ou besoins accidentels de la compagnie. Mais, si les membres de cette compagnie étaient de mauvais intendants et de mauvais souverains quand leurs revenus nets et leurs profits leur appartenaient en totalité et étaient à leur disposition, il n’y avait certainement pas lieu d’espérer qu’ils deviendraient meilleurs quand les trois quarts de leurs profits et revenus appartiendraient à d’autres, et que l’autre quart, sans cesser d’être appliqué à leur profit, ne le serait cependant que sous l’inspection et avec l’approbation d’autrui.
Peut-être la compagnie aimait-elle mieux que tout l’excédent restant après le dividende proposé de 8 pour 100 fût abandonné à ses propres agents et subalternes, pour qu’ils eussent le plaisir de le dissiper en profusions, ou le profit de le détourner par infidélité, plutôt que de voir ce surplus passer dans les mains d’une classe de gens avec lesquels un tel arrangement ne manquerait guère de la mettre en querelle. Il pouvait se faire que l’intérêt de ces agents et subalternes fût assez prédominant dans l’assemblée des propriétaires pour disposer quelquefois cette assemblée à soutenir les auteurs même de déprédations commises au mépris direct de sa propre autorité. Aux yeux de la majorité des propriétaires, ce pouvait être quelquefois une chose de moin­dre conséquence de soutenir l’autorité de leur propre assemblée, que de soutenir ceux qui auraient bravé cette même autorité.
Aussi, les mesures prises en 1773 ne mirent-elles pas fin aux désordres de l’administration de la compagnie dans l’Inde. Encore que, dans un accès passager de bonne conduite, elle eût amassé une fois, dans la trésorerie de Calcutta, plus de 3 millions sterling ; encore qu’elle eût ensuite étendu sa domination ou, si l’on veut, sa déprédation sur un vaste accroissement de territoire formé des contrées les plus riches et les plus fertiles de l’Inde, tout fut dissipé et détruit. La compagnie se trouva tout à fait hors d’état, faute d’y avoir été préparée, d’arrêter les incursions d’Hyder-Ali ou de lui résister, et par suite de ces désordres la compagnie se trouve aujourd’hui (1784) en une plus grande détresse que jamais, et réduite encore une fois à recourir à l’assis­tance du gouvernement pour échapper à une banqueroute imminente. Différents plans ont été proposés dans le parlement, de la part de tous les partis, pour arriver à une meilleure administration de ses affaires ; tous ces plans semblent être d’accord sur un point qui a toujours été, dans le fait, extrêmement évident, c’est que la compagnie est totalement incapable de gouverner ses possessions territoriales. La compagnie elle-même paraît convaincue de sa propre incapacité, au moins sur cet article, et d’après cela semble disposée à les céder au gouvernement.
Au droit de posséder des forts et garnisons dans les pays lointains et non civilisés est nécessairement lié le droit de faire la paix et la guerre dans ces pays. Les compagnies par actions qui ont eu le premier de ces droits ont constamment exercé l’autre, et il leur a été même fréquemment conféré d’une manière expresse. Une expé­rience récente n’a que trop fait connaître avec quelle légèreté capricieuse, avec quelle injustice, avec quelle cruauté, elles ont communément exercé ce terrible droit.
Quand une société de marchands entreprend, à ses propres dépens et à ses risques, d’établir quelque nouvelle branche de commerce avec des peuples lointains et non civilisés, il peut être assez raisonnable de l’incorporer comme compagnie par actions, et de lui accorder, en cas de réussite, le monopole de ce commerce pour un certain nombre d’années. C’est la manière la plus naturelle et la plus facile dont l’État puisse la récompenser d’avoir tenté les premiers hasards d’une entreprise chère et périlleuse, dont le public doit ensuite recueillir le profit. Un monopole temporaire de ce genre peut être justifié par les mêmes principes qui font qu’on accorde un semblable monopole à l’inventeur d’une machine nouvelle, et celui d’un livre nouveau à son auteur. Mais, à l’expiration du terme, le monopole doit certainement être supprimé ; les forts et garnisons, s’il a été jugé nécessaire d’en établir, doivent être remis entre les mains du gouvernement, à la charge par lui d’en rembourser la valeur à la compagnie, et le commerce doit demeurer ouvert à tous les sujets de l’État. Par un monopole per­pé­tuel, tous les autres citoyens se trouvent très-injustement grevés de deux différentes charges : la première résultant du haut prix des marchandises que, dans le cas d’un commerce libre, ils eussent achetées à beaucoup meilleur marché ; et la seconde résul­tant de l’exclusion totale d’une branche d’affaires à laquelle plusieurs d’entre eux auraient pu se livrer avec du profit et de l’agrément ; et c’est d’ailleurs pour en faire le plus indigne emploi qu’on les charge de ce double impôt ; c’est uniquement pour mettre la compagnie à même de soutenir la négligence, la prodigalité et les malver­sations de ses agents, dont la conduite désordonnée lui laisse rarement un dividende au-dessus du taux ordinaire des profits dans les commerces absolument libres, et très-souvent même le fait tomber beaucoup au-dessous. Cependant il paraîtrait, d’après l’ex­pé­rience, que sans le secours du monopole, une compagnie par actions ne saurait se soutenir longtemps dans une branche de commerce étranger. Acheter dans un marché, dans la vue de revendre avec profit dans un autre, quand il se trouve dans tous les deux beaucoup de concurrents ; épier non-seulement les variations acciden­telles de la demande, mais encore les variations bien plus grandes et bien plus fré­quen­tes de la concurrence ou de l’approvisionnement que les autres concurrents pourront amener au marché, en conséquence de l’état des demandes ; faire cadrer avec discernement et habileté, d’après toutes les circonstances, tant la qualité que la quan­tité de chaque assortiment de marchandises, c’est une sorte de petite guerre dont les opérations doivent changer à tout moment, et qui ne peut guère jamais être conduite avec succès, à moins d’une vigilance sans relâche et d’une attention toujours tendue, telles qu’il n’est pas possible d’en attendre pendant longtemps de la part d’une compagnie par actions. La compagnie des Indes Orientales a le droit, par acte du parlement, après la parfaite liquidation de son fonds, et à l’expiration de son privilège exclusif, de rester en société par actions et de continuer en corps à commercer aux Indes, concurremment avec le reste des sujets de la Grande-Bretagne. Mais dans une telle situation, selon toutes les probabilités, la supériorité qu’auraient sur elle les spéculateurs particuliers, du côté de l’attention et de la vigilance, la dégoûterait bien vite de continuer ce commerce.
Un auteur français, très-distingué par ses connaissances en matière d’économie politique, l’abbé Morellet, donne la liste de cinquante-cinq compagnies par actions pour le commerce étranger, qui se sont établies en divers endroits de l’Europe depuis 1600, et qui, selon lui, ont toutes failli par les vices de leur administration, quoi­qu’elles eussent des privilèges exclusifs. Il a été mal informé sur le compte de deux ou trois d’entre elles qui n’étaient pas des compagnies par actions, et qui n’ont pas failli ; mais en revanche il y a eu plusieurs compagnies par actions qui ont failli, et qu’il a omises.
Les seuls genres d’affaires qu’il paraît possible, pour une compagnie par actions, de suivre avec succès, sans privilège exclusif, ce sont celles dont toutes les opérations peuvent être réduites à ce qu’on appelle une routine, ou à une telle uniformité de mé­thode, qu’elle n’admette que peu ou point de variation. De ce genre sont : l˚ le com­merce de la banque ; 2˚ celui des assurances contre les incendies et contre les risques de mer et de capture en temps de guerre ; 3˚ l’entreprise de la construction et de l’entretien d’un canal navigable ; 4˚ une entreprise du même genre, celle d’amener de l’eau pour la provision d’une grande ville.
Quoique les principes du commerce de banque puissent paraître tant soit peu abs­traits et compliqués, cependant la pratique est susceptible d’en être réduite à des règles constantes. Se départir une seule fois de ces règles, en conséquence de quelque spéculation séduisante qui offre l’appât d’un gain extraordinaire, est une chose pres­que toujours extrêmement dangereuse, et très-souvent funeste à la compagnie de banque qui s’y expose. Mais la constitution d’une compagnie par actions rend, en général, ces compagnies plus fortement attachées aux règles qu’elles se sont une fois faites, qu’aucune société particulière. Aussi, les principales compagnies de banque de l’Europe sont-elles des compagnies d’actionnaires, dont la plupart conduisent très-heureusement leurs affaires sans aucun privilège exclusif. Le seul dont jouisse la banque d’Angleterre consiste en ce qu’aucune autre compagnie de banque en ce royau­me ne peut être composée de plus de six personnes. Les deux banques d’Édim­bourg sont des compagnies par actions, sans aucun privilège exclusif.
Quoique la valeur des risques, soit du feu, soit des pertes par mer ou par capture, ne puisse guère se calculer peut-être bien exactement, néanmoins elle est susceptible d’une évaluation en gros qui fait qu’on peut, à certain point, l’assujettir à une méthode et à des règles précises. Par conséquent, le commerce d’assurance peut être fait avec succès par une compagnie par actions, sans aucun privilège exclusif. La compagnie d’assurance de la ville de Londres, ni celle du change royal[24], n’ont aucun privilège de ce genre. 
Quand un canal navigable est une fois achevé, la direction de l’affaire devient tout à fait simple et facile, et elle peut se réduire à une méthode et à des règles constantes. On y peut même réduire la confection d’une de ces sortes d’ouvrages, puisqu’on peut contracter avec les entrepreneurs à tant par toise, à tant par écluse. On en peut dire autant d’un canal, d’un aqueduc ou d’un grand conduit destiné à amener l’eau pour la provision d’une grande ville. De telles entreprises peuvent donc être régies, et le sont aussi très-souvent, par des compagnies d’actionnaires, sans aucun privilège exclusif.
Cependant, il ne serait certainement pas raisonnable d’aller ériger, pour une entre­prise quelconque, une compagnie par actions, uniquement parce que cette compagnie serait capable de conduire l’entreprise avec succès, c’est-à-dire d’aller exempter un certain nombre de particuliers de quelques-unes des lois générales auxquelles tous leurs concitoyens sont assujettis, uniquement parce que ces particuliers, à l’aide de cette exemption, seraient en état de faire bien leurs affaires. Pour qu’un tel établis­sement soit parfaitement raisonnable, outre la condition expliquée ci-dessus, c’est-à-dire la possibilité de réduire l’entreprise à une méthode et à des règles constantes, il faut encore le concours de deux autres circonstances. La première, c’est qu’il soit évidemment démontré que l’entreprise est d’une utilité plus grande et plus générale que la plupart des entreprises ordinaires de commerce ; et la seconde, c’est qu’elle soit de nature à exiger un capital trop considérable pour être fourni facilement par une société particulière. Si un capital modéré suffisait pour l’entreprise, sa grande utilité seule ne serait pas une raison pour qu’on dût ériger une compagnie par actions, parce que, dans ce cas, il se présenterait bientôt des spéculateurs particuliers qui rempli­raient aisément la demande à laquelle cette entreprise aurait pour objet de répondre. Ces deux circonstances concourent dans les quatre genres de commerce dont il est question plus haut.
L’utilité considérable et générale du commerce de banque, quand il est conduit avec prudence, a été expliquée fort au long dans le deuxième livre de cet ouvrage. Mais une banque publique qui a pour objet de soutenir le crédit de l’État, et dans des besoins particuliers d’avancer au gouvernement la totalité du produit d’une taxe montant peut-être à plusieurs millions, une année ou deux avant qu’il puisse rentrer, une telle banque exige un plus grand capital qu’aucune société particulière n’en pourrait aisément réunir. 
Le commerce des assurances tend à donner une grande sécurité aux fortunes pri­vées, et en répartissant sur un très-grand nombre de têtes une perte qui pourrait ruiner un particulier, elle rend cette perte, pour la société tout entière, légère et facile à supporter. Mais, pour donner cette sécurité, il faut que les assureurs aient un très-gros capital. On dit qu’avant l’établissement des deux compagnies par actions pour le commerce d’assurance à Londres, il fut mis sous les yeux du procureur général une liste de cent cinquante assureurs particuliers qui avaient failli dans l’espace de quel­ques années.
C’est une chose assez évidente par elle-même que les canaux navigables et les ouvrages qui sont quelquefois nécessaires pour fournir d’eau une grande ville, sont extrêmement avantageux et d’une utilité générale, tandis qu’en même temps ils exi­gent souvent des dépenses plus fortes que n’en pourraient soutenir des fortunes particulières.
Excepté les quatre genres de commerce dont j’ai fait mention, je n’ai pu parvenir à m’en rappeler aucun autre dans lequel se trouvent concourir toutes les circonstances requises pour justifier l’établissement d’une compagnie par actions. La compagnie de Londres pour le cuivre anglais, la compagnie pour la fonte du plomb, la compagnie pour le poli des glaces, n’ont pas même le prétexte d’aucune utilité générale, ou seule­ment particulière, dans les objets dont elles s’occupent, et ces objets ne paraissent pas exiger des dépenses qui excèdent les facultés d’une réunion de plusieurs fortunes privées. Quant à la question de savoir si le genre de commerce que font ces compa­gnies est de nature à pouvoir se réduire à une méthode et à des règles assez précises pour qu’il soit susceptible du régime d’une compagnie par actions, ou si ces compa­gnies ont sujet de se vanter de profits extraordinaires, c’est ce dont je ne prétends pas être instruit. Il y a longtemps que la compagnie pour l’exploitation des mines est en banqueroute. Un intérêt dans les fonds de la compagnie des toiles d’Édimbourg se vend à présent fort au-dessous du pair, quoique moins au-dessous qu’il n’était il y a quelques années. Les compagnies par actions qui se sont établies dans la vue généreuse d’être utiles à l’État, en encourageant quelques manufactures particulières, outre le dommage qu’elles causent en faisant mal leurs propres affaires, et diminuant par là la masse générale des capitaux de la société, ne peuvent guère manquer encore, sous d’autres rapports, de faire plus de mal que de bien. Malgré les intentions les plus droites, la partialité inévitable de leurs directeurs pour quelques branches particulières de manufactures, dont les entrepreneurs viennent à bout de les séduire et de les dominer, jette nécessairement sur le reste un véritable découragement, et tend à rom­pre plus ou moins cette proportion naturelle qui s’établirait sans cela entre le profit et la sage industrie ; proportion qui est pour l’industrie générale du pays le plus grand et le plus efficace de tous les encouragements.

Article II.
De la dépense qu’exigent les institutions pour l’éducation de la jeunesse

Les institutions pour l’éducation de la jeunesse peuvent aussi, de la même ma­nière, fournir un revenu suffisant pour défrayer leur propre dépense. Le salaire ou honoraire que l’écolier paie au maître constitue naturellement un revenu de ce genre.
Lors même que la récompense du maître ne provient pas entièrement de cette source naturelle de revenu, il n’est pas encore nécessaire qu’elle soit puisée dans ce revenu général de la société, dont la perception et l’emploi sont délégués, dans la plupart des pays, au pouvoir exécutif. Aussi, dans la plus grande partie de l’Europe, la dotation des collèges ou écoles n’est point une charge de ce revenu général, ou n’en est qu’une très-faible. Partout cette dotation provient principalement de quelque reve­nu local ou provincial, de la rente de quelques biens-fonds, ou de l’intérêt de quelque somme d’argent donnée quelquefois par le souverain lui-même, et quelque­fois par un donateur particulier, et mise sous la régie d’administrateurs ou curateurs établis à cet effet.
Ces dotations publiques ont-elles contribué, en général, à accélérer le but de leur institution ? Ont-elles contribué à encourager la diligence des maîtres et à perfection­ner leurs talents ? Ont-elles dirigé le cours de l’éducation vers des objets qui soient, tant pour l’individu que pour la société, d’une plus grande utilité que ceux vers les­quels elle se serait dirigée d’elle-même ? Il ne serait pas, à ce qu’il semble, très-diffi­cile de répondre d’une manière au moins vraisemblable à chacune de ces ques­tions.
Dans chaque profession, les efforts de la plupart de ceux qui l’exercent sont toujours proportionnés à la nécessité qu’il y a pour eux d’en faire. Cette nécessité est plus grande pour ceux qui n’attendent leur fortune, ou même leur revenu et leur sub­sistance ordinaire, que des émoluments de leur profession. Pour acquérir cette fortu­ne, ou même pour gagner cette subsistance, il leur faut, dans le cours d’une année, exécuter une certaine quantité d’ouvrage d’une valeur connue, et si la concurrence est libre, la rivalité des concurrents, qui tâchent tous de s’exclure l’un l’autre de l’emploi commun, oblige chacun à s’efforcer d’exécuter son ouvrage avec un certain degré d’exactitude. Sans doute, la grandeur des objets auxquels on peut se flatter d’atteindre en réussissant dans certaines professions particulières, peut animer quelquefois le zèle de ce petit nombre d’hommes doués d’une ambition et d’une activité extraordinaires. Cependant, il est évident que pour donner naissance aux plus grands efforts, de grands projets ne sont pas nécessaires. La rivalité et l’émulation font de l’avantage d’exceller, même dans des professions obscures, un objet d’ambition, et souvent donnent lieu de déployer beaucoup d’énergie. Au contraire, les grands objets seuls, et sans le concours de la nécessité de l’application, ont rarement suffi pour produire quelque effort considérable de travail. En Angleterre, les succès dans la profession de légiste conduisent à de très-hauts objets d’ambition ; et cependant combien peu voit-on chez nous d’hommes nés dans l’aisance qui se soient jamais distingués dans cette profession !
Les dotations des collèges et des écoles ont nécessairement diminué plus ou moins chez les maîtres la nécessité de l’application ; leur subsistance, en tant qu’elle provient de leur traitement, dérive évidemment d’une source totalement indépendante de leur réputation et de leurs succès dans leurs professions particulières.
Dans quelques universités, le traitement fixe ne fait qu’une partie, et souvent qu’une faible partie des émoluments du maître, qui se composent principalement des honoraires ou rétributions qu’il reçoit de ses élèves. Dans ce cas, la nécessité de l’application n’est pas entièrement ôtée, quoique toujours elle soit plus ou moins diminuée. Ici, la réputation du maître dans sa profession est encore de quelque impor­tance pour lui ; il dépend encore pour quelque chose de l’attachement et de la recon­naissance de ceux qui ont suivi ses leçons, et du compte favorable qu’ils ont à rendre de lui ; et pour s’acquérir ces dispositions favorables, il n’a pas de voie plus sûre que de les mériter, c’est-à-dire de mettre tous ses soins et tous ses talents à remplir chaque partie de ses devoirs.
Dans d’autres universités, il est interdit au maître de recevoir aucun honoraire ou rétribution de ses élèves, et son traitement annuel constitue la totalité du revenu de sa place. Dans ce cas, son intérêt se trouve mis en opposition aussi directe que possible avec son devoir. L’intérêt de tout homme est de passer sa vie à son aise le plus qu’il peut, et si ses émoluments doivent être exactement les mêmes, soit qu’il remplisse ou non quelque devoir pénible, c’est certainement son intérêt (au moins dans le sens qu’on attache communément à ce mot), ou de négliger tout à fait ce devoir, ou bien, s’il est sous les yeux de quelque autorité qui ne lui permette pas d’agir ainsi, de s’en acquitter avec toute l’inattention et toute l’indolence que cette autorité voudra lui permettre. Si naturellement il a de l’activité et qu’il aime le travail, son intérêt est d’em­ployer cette activité à quelque chose dont il puisse retirer un avantage, plutôt qu’à l’acquittement d’un devoir qui ne peut lui en produire.
Si l’autorité à laquelle il est assujetti réside dans la corporation, le collège ou l’université dont il est membre lui-même, et dont la plupart des autres membres sont, comme lui, des personnes qui enseignent ou qui devraient enseigner, il est probable qu’ils feront tous cause commune pour se traiter réciproquement avec beaucoup d’in­dulgence, et que chacun consentira volontiers à ce que son voisin néglige ses devoirs, pourvu qu’on lui laisse aussi de son côté la faculté de négliger les siens. Il y a déjà plusieurs années qu’à l’université d’Oxford la plus grande partie des professeurs publics ont abandonné totalement jusqu’à l’apparence même d’enseigner.
Si l’autorité à laquelle il est soumis réside moins dans la corporation dont il est membre que dans quelque personne étrangère, telle, par exemple, que l’évêque du diocèse, le gouverneur de la province, ou peut-être quelque ministre d’État, dans ce cas, à la vérité, il n’est pas aussi probable qu’on lui laisse négliger tout à fait son de­voir. Cependant, tout ce que peuvent l’obliger à faire des supérieurs tels que ceux-ci, c’est d’être avec ses élèves un certain nombre d’heures, c’est-à-dire, de donner un certain nombre de leçons par semaine ou par année. Mais de quel genre seront ces leçons ? C’est ce qui dépendra toujours de l’activité et des soins du maître ; et cette activité, ces soins, seront vraisemblablement en proportion des motifs qu’il aura pour les donner. D’ailleurs, une juridiction étrangère telle que celle-là est sujette à être exercée à la fois avec ignorance et avec caprice. Par sa nature, elle est arbitraire et repose sur la discrétion des personnes qui en sont revêtues, lesquelles, n’assistant pas par elles-mêmes aux leçons du maître, peut-être même n’entendant rien aux sciences qu’il est chargé d’enseigner, ne sont guère en état de l’exercer avec discernement ; et puis, par suite de l’impertinence attachée aux grandes places, ces personnes sont fort souvent elles-mêmes très-indifférentes sur la manière dont elles exercent cette juridiction, et elles sont très-disposées à réprimander le maître ou à lui ôter sa place légèrement et sans motif raisonnable. Une pareille juridiction dégrade nécessairement celui qui y est soumis et, au lieu de tenir rang parmi les personnes les plus respec­tables de la société, il se trouve placé par là dans la classe avilie et méprisée. Une protection puissante est la seule sauvegarde qu’il puisse se donner contre les mauvais traitements auxquels il est exposé à tout moment ; et pour obtenir cette protection, le talent ou l’exactitude qu’il apportera dans l’exercice de sa profession est un moyen bien moins sûr qu’une soumission absolue à la volonté de ses supérieurs, et la disposition constante de sacrifier à cette volonté les droits, l’intérêt et l’honneur de la corporation dont il est membre. Il n’y a personne qui ait observé pendant quelque temps l’administration d’une université française, qui n’ait eu occasion de remarquer les effets inévitables d’une juridiction extérieure et arbitraire de ce genre.
Tout ce qui oblige un certain nombre d’étudiants à rester à un collège ou à une université, indépendamment du mérite ou de la réputation des maîtres, tend plus ou moins à rendre ce mérite ou cette réputation moins nécessaire.
Quand les privilèges des gradués dans les arts, dans le droit, dans la médecine et dans la théologie peuvent s’obtenir seulement par une résidence d’un certain nombre d’années dans les universités, ils entraînent nécessairement une quantité quelconque d’étudiants dans ces universités, indépendamment du mérite ou de la réputation des maîtres. Les privilèges des gradués sont une espèce de statuts d’apprentissage, qui ont contribué à perfectionner l’éducation, précisément comme les autres statuts d’appren­tissage ont contribué à perfectionner les arts et les manufactures.
Les fondations charitables pour des pensions d’écolier, bourses, etc., attachent néces­sai­rement un certain nombre d’écoliers à certains collèges, tout à fait indé­pendam­ment du mérite de ces collèges. Si ces fondations charitables avaient laissé aux écoliers la liberté de choisir leur collège, une pareille liberté aurait peut-être con­tri­bué à exciter, entre différents collèges un peu d’émulation. Un règlement qui, tout au contraire, défendait même aux membres indépendants de chaque collège particu­lier de le quitter et d’aller à un autre, sans avoir préalablement demandé et obtenu la permission de celui qu’on entend abandonner, tendait encore bien davantage à éteindre cette émulation.
Si, dans chaque collège, le maître ou précepteur destiné à instruire un écolier dans les différents arts et sciences n’était pas choisi librement par l’écolier, mais qu’il fût nommé par le chef du collège, et si, en cas de négligence, inaptitude ou mauvaise conduite de sa part, l’écolier n’était pas maître de le changer pour aller à un autre sans en avoir demandé et obtenu la permission, un pareil règlement tendrait beaucoup, non-seulement à éteindre toute émulation entre les différents maîtres d’un même collège, mais encore à diminuer pour tous les maîtres la nécessité des soins et de l’exactitude à l’égard de leurs élèves respectifs. De tels maîtres, quand même ils seraient bien payés par leurs écoliers, pourraient être tout aussi disposés à les négliger que ceux qui ne sont pas du tout payés par leurs écoliers, et qui n’ont d’autre récompense qu’un traitement fixe.
S’il arrive que le maître soit un homme de bon sens, ce doit être pour lui une chose assez pénible de sentir qu’en faisant ses leçons à ses écoliers, il leur lit ou leur débite du galimatias dépourvu de raison, ou quelque chose qui ne vaut guère mieux. Ce doit être aussi une chose assez désagréable d’observer que la plus grande partie de ses écoliers n’assistent pas à ses leçons, ou peut-être y assistent avec des signes marqués de négligence, de dédain ou de dérision. S’il est donc obligé de donner un certain nombre de leçons, ces motifs seuls, sans autre intérêt quelconque, pourront le disposer à prendre quelques soins pour que ses leçons soient passablement bonnes. Il y a néanmoins plusieurs expédients auxquels un maître peut avoir recours, et qui détruiront entièrement l’effet que ces motifs auraient pu avoir sur son activité. Au lieu d’expliquer lui-même à ses élèves la science dans laquelle il se propose de les instrui­re, le maître peut lire un livre qui traite de cette science ; et si ce livre est écrit dans une langue morte ou dans une langue étrangère, alors, en leur en faisant l’interpré­tation, ou, ce qui lui donnerait encore moins de peine, en leur faisant traduire à eux-mêmes, et leur entremêlant cette lecture, de temps en temps, de quelques remarques, il pourra se figurer avoir donné une leçon. Le plus léger degré de connaissances et d’application lui suffira pour remplir cette tâche sans s’exposer à la risée de ses auditeurs, ou sans être obligé de leur débiter des impertinences ou des absurdités. En même temps, la discipline établie dans le collège lui donne le moyen d’obliger ses élèves à assister le plus régulièrement possible à cette prétendue leçon, et de main­tenir entre eux, pendant tout le temps qu’elle dure, la contenance la plus décente et la plus respectueuse.
La discipline des collèges et des universités, en général, n’est pas instituée pour l’avantage des écoliers, mais bien pour l’intérêt, ou, pour mieux dire, pour la commo­dité des maîtres[25]. Son objet est de maintenir l’autorité du maître en toutes circons­tances, et de quelque manière qu’il se comporte, qu’il remplisse ses devoirs ou qu’il les néglige, d’obliger les écoliers, dans tous les cas, à se conduire à son égard comme s’il enseignait avec le plus grand talent et la plus parfaite exactitude. Elle semble supposer du côté du maître toute la sagesse et la vertu possible, et du côté des écoliers une extrême ineptie et une déraison complète. je ne crois pas cependant qu’il y ait d’exemples que, quand les maîtres s’acquittent réellement de leur devoir, la plupart des écoliers négligent le leur. Il n’est jamais besoin du secours de la contrainte pour obliger d’assister à des leçons qui méritent d’être écoutées, comme on le voit bien partout où se donnent de pareilles leçons, Sans doute, il faut bien, à un certain point, user de moyens de gêne et de rigueur pour obliger des enfants ou de très-petits garçons à prêter attention à ces parties de l’éducation qu’on croit nécessaire pour eux d’acquérir dans le cours de cette première période de la vie ; mais, passé l’âge de douze à treize ans, pourvu que le maître fasse bien son devoir, la contrainte ou la gêne ne peut plus guère être nécessaire pour diriger aucune partie de l’éducation. Telle est la disposition généreuse de la plupart des jeunes gens, que, bien loin d’être por­tés à négliger ou à tourner en ridicule les instructions d’un maître qui leur témoi­gne sérieusement l’intention de leur être utile, ils sont, au contraire, portés d’inclina­tion, en général, à lui pardonner un bon nombre d’inexactitudes dans l’accomplisse­ment de son devoir, et quelquefois même à cacher aux yeux publics beaucoup de négligences grossières[26].
Il est à remarquer que ces parties de l’instruction pour lesquelles il n’y a pas d’insti­tutions publiques sont, en général, les mieux enseignées. Quand un jeune homme va à une école d’armes ou de danse, il ne parvient pas toujours, il est vrai, à danser ou à faire des armes parfaitement ; mais il est bien rare qu’il n’y apprenne pas à danser ou à faire des armes. Les bons effets des écoles d’équitation ne sont pas communément aussi évidents. La dépense d’une école d’équitation est si forte, que dans la plupart des endroits c’est une institution publique. Les trois parties les plus essentielles de l’instruction littéraire, lire, écrire et compter, se donnent toujours plus communément dans des écoles particulières que dans des écoles publiques ; et aussi arrive-t-il très-rarement que personne manque d’acquérir ces connaissances au degré auquel il est nécessaire de les avoir.
Les écoles publiques en Angleterre sont beaucoup moins gâtées que les univer­sités. Dans les écoles on enseigne à la jeunesse, ou au moins elle peut apprendre le grec et le latin, qui est tout ce que les maîtres prétendent enseigner ou toute l’instruc­tion qu’on attend d’eux. Dans les universités, les sciences dans lesquelles ces corpo­rations sont destinées à instruire la jeunesse n’y sont point enseignées, et même la jeunesse n’y pourrait pas toujours trouver des moyens pour les y apprendre. La récompense du maître d’école dépend principalement, dans la plupart des cas, et presque entièrement dans quelques cas, des rétributions et honoraires qu’il reçoit de ses écoliers. Les écoles n’ont aucuns privilèges exclusifs. Il n’est pas nécessaire, pour obtenir les honneurs de gradué, que la personne rapporte un certificat constatant qu’elle a étudié un certain nombre d’années dans une école publique. Qu’à l’examen elle fasse voir qu’elle entend ce qu’on enseigne dans ces écoles, et on ne la ques­tionnera pas sur l’endroit où elle peut l’avoir appris.
Mais, me dira-t-on peut-être, les parties de l’instruction qui s’enseignent commu­nément dans les universités n’y sont pas, il est vrai, très-bien enseignées ; cependant, si elles ne l’étaient pas par le moyen de ces institutions, le plus souvent elles ne l’auraient pas été du tout, et alors le public aussi bien que les particuliers auraient eu vraiment à souffrir de cette lacune dans des parties aussi importantes de l’instruction.
Les universités actuelles de l’Europe étaient dans l’origine, pour la plupart, des corpo­rations ecclésiastiques instituées pour l’instruction des gens d’Église. Elles furent fondées par l’autorité du pape, et mises si absolument sous sa protection immédiate, que leurs membres, maîtres ou écoliers, avaient tous ce qu’on appelait alors le bénéfice de clergie, c’est-à-dire qu’ils étaient exempts de la juridiction civile des pays dans lesquels étaient situées leurs universités respectives, et qu’ils ne pouvaient être traduits que devant les tribunaux ecclésiastiques. Ce qu’on enseignait dans la majeure partie de ces universités était, conformément au but de leur institu­tion, ou la théologie, ou quelque chose qui était purement préparatoire aux études théologiques.
À l’époque à laquelle le christianisme commença à être la religion établie par la loi, un latin corrompu était devenu le langage vulgaire de toutes les parties occi­dentales de l’Europe. Aussi, le service divin et la traduction de la Bible qui se lisait dans les églises, étaient-ils l’un et l’autre dans ce mauvais latin, c’est-à-dire dans la langue ordinaire du pays. Après l’irruption des nations barbares qui renversèrent l’em­pire romain, le latin cessa peu à peu, par toute l’Europe, d’être la langue habituelle. Mais le peuple, par respect pour la religion, en conserva naturellement les formes et les cérémonies longtemps après que les circonstances qui les avaient d’abord intro­duites et qui les rendaient raisonnables, n’existèrent plus. Ainsi, quoique le latin ne fût plus entendu nulle part de la masse du peuple, tout le service divin continua toujours à être célébré dans cette langue. Dès lors il y eut deux langages différents établis en Europe, de la même manière que dans l’ancienne Égypte : la langue des prêtres et la langue du peuple ; la langue sacrée et la langue profane ; la langue des lettrés et celle des gens non lettrés. Or, il était nécessaire que les prêtres entendissent un peu cette langue sacrée et savante dans laquelle ils devaient officier, et par conséquent l’étude de la langue latine fut, dès l’origine, une partie essentielle de l’instruction dans les universités.
Il n’en était pas de même pour la langue grecque ni pour l’hébreu. Les infaillibles décrets de l’Église avaient prononcé que la traduction latine de la Bible, appelée communément la Vulgate, avait été, comme l’original, dictée par l’inspiration divine et que, par conséquent, elle avait la même autorité que les textes grec et hébreu. Ainsi, la connaissance de ces deux langues n’étant pas exigée comme une chose indispensable dans un ecclésiastique, leur étude, pendant un temps assez long, ne fit pas une partie nécessaire du cours ordinaire de l’éducation des universités. Il y a quelques universités en Espagne dans lesquelles, à ce qu’on m’a assuré, l’étude de la langue grecque n’a encore jamais fait partie du cours d’instruction. Les premiers réformateurs trouvèrent le texte grec du Nouveau Testament, et même le texte hébreu de l’Ancien, plus favorable à leurs opinions que la version de la Vulgate, qui avait été insensiblement accommodée, comme il est assez naturel de le présumer, au profit de la doctrine catholique. Ils s’attachèrent donc à exposer les nombreuses erreurs de cette version, ce qui mit le clergé romain dans la nécessité de la défendre ou de l’expliquer. Mais ceci ne pouvait guère se faire sans quelque connaissance des langues origi­nales ; ainsi leur étude s’introduisit peu à peu dans la majeure partie des universités, tant de celles qui embrassèrent la doctrine de la réformation, que de celles qui la rejetèrent. La langue grecque fut liée avec chaque partie de ce savoir classique qui, d’abord cultivé principalement par des catholiques et des Italiens, se trouva être en vogue absolument dans le même temps que la doctrine de la réformation vint à s’établir. Ainsi, dans la plupart des universités on enseigna cette langue préalablement à l’étude de la philosophie, et aussitôt que l’écolier eut fait quelques progrès dans le latin. L’hébreu n’ayant aucune liaison avec le savoir classique, et cette langue ne possédant, à l’exception des saintes Écritures, pas un seul livre un peu estimé, l’étude n’en commença communément qu’après celle de la philosophie, et quand l’étudiant fut entré dans la classe de la théologie.
Dans l’origine, on enseignait dans les universités les premiers éléments mêmes des langues grecque et latine, et dans quelques-unes on continue encore à les ensei­gner. Dans d’autres, on s’attend que l’étudiant aura probablement appris au moins les premiers éléments de l’une de ces langues ou de toutes les deux, dont l’étude continue toujours à faire partout une partie très-considérable de l’éducation des universités.
L’ancienne philosophie des Grecs était divisée en trois branches principales : la physique ou philosophie naturelle, l’éthique ou philosophie morale, et la logique. Cette division générale semble convenir parfaitement à la nature des choses.
Les grands phénomènes de la nature, les révolutions des corps célestes, les éclip­ses, les comètes, la foudre, les éclairs et les autres météores extraordinaires ; la géné­ration, la vie, la croissance et la dissolution des plantes et des animaux, sont autant d’objets qui, naturellement excitant l’étonnement, appellent nécessairement la curiosité de l’homme à rechercher leurs causes. La superstition essaya la première de satisfaire cette curiosité, en rapportant à l’action immédiate de quelque divinité tous ces objets surprenants. La philosophie vint ensuite, et chercha à en rendre compte d’après des causes plus familières aux hommes, ou auxquelles ils étaient plus habi­tués, que l’action d’une divinité. Comme ces grands phénomènes sont les premiers objets de la curiosité humaine, de même la science qui se propose de les expliquer a dû nécessairement être la première branche de philosophie qu’on ait cultivée. Aussi, les premiers philosophes dont l’histoire nous ait conservé quelque souvenir semblent-ils avoir été des philosophes adonnés à l’histoire naturelle[27]. 
Dans tout âge et dans tout pays du monde, les hommes ont dû observer avec attention les caractères, les intentions et les actions les uns des autres, et il a dû s’établir un grand nombre de règles ou de préceptes recommandables pour la conduite de la vie humaine, et consacrés par l’approbation générale. Dès que l’écriture se fut répandue, les hommes sages, ou ceux qui s’imaginaient l’être, cherchèrent naturelle­ment à augmenter le nombre de ces maximes généralement établies et respectées, et à exprimer leur propre sentiment sur ce qui était convenable ou ne l’était pas dans la conduite des hommes. Tantôt ils l’exprimèrent sous la forme plus adroite d’apologues, comme ce qu’on a appelé les Fables d’Ésope ; tantôt sous la forme plus simple d’apophtegmes ou de paroles sages, comme les Proverbes de Salomon, les Vers dorés de Théognis et de Procylide, et une partie des ouvrages d’Hésiode. Ils ont pu conti­nuer ainsi pendant longtemps à multiplier simplement le nombre de ces maximes de prudence ou de moralité, sans chercher même à les arranger dans un ordre méthodi­que ou très-distinct, encore bien moins à les lier entre elles par un ou plusieurs princi­pes généraux dont elles pussent toutes se déduire, comme des effets se déduisent de leurs causes naturelles. La beauté de l’arrangement systématique de différentes obser­vations, liées par un petit nombre de principes qui leur sont communs, se fit voir pour la première fois dans les essais informes imaginés dans ces anciens temps pour arriver à un système d’histoire naturelle. Par la suite, on essaya en morale quelque chose du même genre. On arrangea les préceptes du corps ordinaire de la vie dans un ordre méthodique, et on les lia ensemble par un petit nombre de principes généraux, de la même manière qu’on avait tâché d’arranger et de lier les phénomènes de la nature. La science qui se propose de rechercher et d’expliquer les principes généraux auxquels se lient les maximes particulières est ce qu’on appelle proprement la philosophie morale[28]. 
Différents auteurs donnèrent des systèmes différents, tant d’histoire naturelle que de philosophie morale. Mais les arguments qu’ils employaient à l’appui de ces diffé­rents systèmes, bien loin d’être toujours des démonstrations, n’étaient souvent au plus que de très-légères probabilités, et quelquefois de purs sophismes, qui n’avaient d’autre fondement que l’inexactitude et l’ambiguïté du langage. Dans tous les âges du monde, les systèmes spéculatifs ont été fondés sur des bases trop frivoles pour avoir jamais, dans aucune matière du plus mince intérêt pécuniaire, déterminé le jugement d’un homme d’un sens ordinaire. L’argumentation, ou ce qu’on peut appeler purement l’art des sophistes, n’a presque jamais eu aucune influence sur les opinions du genre humain, si ce n’est en matière de philosophie et de spéculation ; et très-souvent aussi, dans ces matières, c’est lui qui a eu la plus grande influence. Les champions de chaque système de philosophie naturelle et morale s’efforçaient de démontrer la fai­blesse des arguments avancés à l’appui des systèmes de leurs adversaires. En exami­nant ces arguments, ils furent nécessairement amenés à considérer la différence entre un argument probable et un argument démonstratif, entre un argument captieux et un argument concluant ; et des observations auxquelles donna lieu cette recherche approfondie dut naître naturellement la logique, ou la science des principes généraux qui constituent la manière de bien ou mal raisonner. Quoique postérieure en origine à la physique et à l’éthique, cependant, dans la plus grande partie des anciennes écoles de philosophie, mais non à la vérité dans toutes, on l’enseigna communément anté­rieu­rement à l’une et à l’autre de ces sciences. On a pensé, à ce qu’il semble, que l’écolier devait bien connaître d’abord la différence qui distingue un bon raisonnement d’avec un mauvais, avant d’être amené à raisonner sur des sujets aussi élevés.
À cette ancienne division de la philosophie en trois parties, on substitua, dans la plupart des universités de l’Europe, une autre classification en cinq parties.
Dans la philosophie ancienne, tout ce qui était enseigné sur la nature de l’âme ou sur celle de la Divinité faisait partie de la physique. Ces êtres, quelle que fût leur essence, faisaient partie du grand système de l’univers, et la partie qui produisait les effets les plus remarquables.
Tout ce que la raison humaine pouvait ou conclure ou conjecturer sur ces êtres formait, pour ainsi dire, deux chapitres, quoique deux chapitres, sans contredit, d’une très-haute importance, de la science qui se proposait d’expliquer l’origine et les révolutions du système général de l’univers. Mais dans les universités de l’Europe, où la philosophie ne fut enseignée que comme un accessoire de la théologie, il était naturel qu’on s’arrêtât plus longtemps sur ces deux chapitres que sur toute autre partie de la science. Ces deux chapitres furent successivement étendus de plus en plus et divisés en un grand nombre de chapitres secondaires, jusqu’à ce qu’enfin la doctrine des esprits, sur lesquels il y a si peu de chose à la portée de notre intelligence, vînt à occuper autant de place dans le système de la philosophie, que la doctrine des corps qui offrent un si vaste champ à nos connaissances. Us doctrines relatives à ces deux sujets furent considérées comme faisant deux sciences distinctes. Celle qui fut appelée pneumatique ou métaphysique fut mise en opposition à la physique, et fut cultivée, non- seulement comme la plus sublime des deux, mais encore comme la plus utile, vu la profession particulière à laquelle toute l’instruction était destinée. Le sujet réellement propre à l’observation et à l’expérience, le sujet qui, à l’aide d’une soi­gneuse attention, peut nous conduire à tant de découvertes utiles, se trouva presque totalement négligé. Le sujet qui fut extrêmement cultivé, ce fut celui dans lequel, après un petit nombre de vérités fort simples et presque évidentes, les plus profondes recherches ne peuvent plus découvrir que ténèbres et incertitudes, et ne peuvent, par conséquent, produire que des sophismes et des subtilités.
Quand ces deux sciences eurent été mises en opposition l’une avec l’autre, la comparaison qu’on fit entre elles deux donna naturellement naissance à une troisième, à celle qui fut appelée ontologie, ou la science qui traita des qualités et des attributs com­muns à la fois à l’un et à l’autre sujet des deux autres sciences. Mais si les sophismes et les subtilités composaient la plus grande partie de la métaphysique ou pneumatique des écoles, ils composaient la totalité du tissu si frivole et si inextricable à la fois de cette espèce de science qu’on nomma l’ontologie, à laquelle on a aussi quelquefois donné le nom de métaphysique.
L’objet, que se proposait dans ses recherches l’ancienne philosophie morale, était de connaître ce qui constitue la perfection et le bonheur de l’homme, considéré non-seulement comme individu, mais comme membre d’une famille, d’un État et de la grande société du genre humain. Cette philosophie traitait des devoirs comme de moyens pour arriver à la perfection et au bonheur de la vie humaine. Mais quand on en vint à n’enseigner la philosophie morale aussi bien que l’histoire naturelle que comme des connaissances accessoires à la théologie, alors des devoirs de la vie de l’homme furent traités principalement comme des moyens d’arriver au bonheur d’une vie future. Dans la philosophie ancienne, la perfection de la vertu était représentée comme devant nécessairement produire à celui qui la possédait le bonheur le plus parfait dans cette vie. Dans la philosophie moderne, on la représenta souvent comme étant en général, ou plutôt presque toujours, incompatible avec quelque degré de bonheur dans cette vie. Les seuls moyens de gagner le ciel furent la pénitence, les mortifications, les austérités et l’abaissement d’un moine, et non pas la conduite élevée, généreuse, énergique d’un homme. La doctrine des casuistes et une morale ascétique remplirent souvent la majeure partie de la philosophie morale des écoles. La plus importante sans comparaison de toutes les branches différentes de la philo­sophie devint de cette manière, sans comparaison, la plus corrompue de toutes.
Tel fut donc le cours ordinaire de l’éducation philosophique dans la plus grande partie des universités de l’Europe. On enseigna d’abord la logique ; l’ontologie vint au second rang ; la pneumatologie, comprenant la doctrine relative à la nature de l’âme et de la Divinité, fut mise au troisième ; vint ensuite, en quatrième ordre, un système abâ­tardi de philosophie morale, qui fut regardé comme lié immédiatement à la doc­trine de la pneumatologie, avec l’immortalité de l’âme, et avec les récompenses et les peines que l’on devait attendre de la justice de la Divinité dans une vie future ; un système bref et superficiel de physique terminait ordinairement le cours.
Les changements que les universités de l’Europe introduisirent ainsi dans l’ancien cours de philosophie furent tous imaginés pour l’éducation des ecclésiastiques, et pour faire de ce cours une introduction plus convenable à l’étude de la théologie. Mais tout ce qu’on y ajouta en subtilités et en sophismes, tout ce que ces changements y mêlèrent de morale ascétique et de doctrine de casuistes, ne contribua pas à le rendre plus propre à l’éducation des gens du monde, c’est-à-dire plus fait pour perfec­tionner les facultés de l’esprit ou les qualités du cœur.
Ce cours de philosophie est ce qu’on continue encore à enseigner dans la plupart des universités de l’Europe, avec plus ou moins de soin et d’exactitude, selon que la constitution de chacune de ces universités est de nature à rendre ce soin et cette exactitude plus ou moins nécessaires aux maîtres. Dans quelques-unes des plus riches et des mieux dotées, les professeurs se contentent d’enseigner quelques bribes et quel­ques morceaux décousus de ce cours corrompu, et encore pour l’ordinaire les enseignent-ils très-superficiellement et très-négligemment[29].
Les réformes et les progrès qui ont perfectionné, dans les temps modernes, plusieurs diverses branches de la philosophie, n’ont pas été, pour la plupart, l’ouvrage des universités, quoique sans doute elles en aient amené quelques-uns. En général même, les universités ont montré fort peu d’empressement à adopter ces réformes après qu’elles ont eu lieu ; et plusieurs de ces savantes sociétés ont préféré rester pendant longtemps comme des sanctuaires où les systèmes décriés et les préjugés surannés trouvaient encore refuge et protection après avoir été chassés de tout autre coin du monde. Les universités les plus riches et les mieux rentées ont été géné­ra­le­ment les plus tardives à adopter les réformes et les découvertes nouvelles, et ce sont elles qui ont fait voir le plus d’éloignement pour tout changement un peu considérable dans le plan d’éducation alors établi. Ces réformes s’introduisirent moins difficilement dans quelques universités plus pauvres, dans lesquelles les professeurs, comptant sur leur réputation pour la plus grande partie de leur subsistance, étaient obligés d’avoir plus d’égard aux opinions reçues dans le monde.
Mais quoique les écoles publiques et les universités de l’Europe fussent unique­ment destinées dans l’origine à l’éducation d’une profession particulière, celle des gens d’Église ; et quoique encore elles ne missent pas toujours beaucoup de soin à instruire leurs élèves dans les sciences mêmes qui passaient pour nécessaires à cette profession, cependant elles attirèrent à elles, peu à peu, l’éducation de presque toutes les autres personnes, et en particulier de presque tous les gens bien nés et ayant de la fortune. On ne sut pas trouver, à ce qu’il semble, une meilleure manière d’employer un peu fructueusement ce long intervalle qui sépare l’enfance d’avec cette période de la vie où les hommes commencent à s’appliquer sérieusement aux affaires du monde, aux affaires qui doivent les occuper pendant tout le reste de leurs jours. Cependant, la plus grande partie de ce qu’on enseigne dans les écoles et dans les universités ne semble pas ce qu’il y a de plus propre à les préparer à ses occupations.
En Angleterre, c’est une coutume qui gagne de plus en plus tous les jours que de faire voyager les jeunes gens dans les pays étrangers immédiatement au sortir de leurs classes, et sans les envoyer dans une université. Notre jeunesse, dit-on, revient au pays après avoir généralement beaucoup acquis dans ses voyages. Un jeune homme qui quitte le pays à dix-sept ou dix-huit ans et y rentre à vingt et un, revient avec trois ou quatre années de plus qu’il n’avait à l’époque de son départ, et à cet âge il est très-difficile de ne pas gagner d’une manière très-sensible en trois ou quatre ans. Dans le cours de ses voyages, il acquiert, en général, la connaissance d’une ou de deux langues étrangères, connaissance pourtant qui est rarement suffisante pour le mettre en état de les parler ou de les écrire correctement. À d’autres égards, il revient pour l’ordinaire plus suffisant, plus relâché dans ses mœurs, plus dissipé et moins capable d’aucune application sérieuse, ou pour l’étude ou pour les affaires, qu’il ne pourrait vraisemblablement l’être jamais devenu dans un si court espace de temps, s’il fût resté chez lui. En voyageant de si bonne heure, en perdant dans la dissipation la plus frivole les plus précieuses années de sa vie, éloigné de l’inspection et de la censure de ses parents et de sa famille, toutes les bonnes habitudes que les premières parties de son éducation auraient pu tendre à lui donner, au lieu d’être inculquées et fortifiées, s’affaiblissent et s’effacent presque nécessairement. Il n’y avait que le décri dans lequel les universités se sont laissées tomber d’elles-mêmes, qui fût capable de mettre en vogue une pratique aussi absurde que celle de voyager dans un âge si jeune. Un père, en envoyant son fils dans les pays étrangers, se délivre, au moins pour quelque temps, d’un objet aussi fatigant que celui d’un fils désœuvré, négligé, et qui se perd sous ses yeux.
Tels ont été les effets de quelques-unes des institutions modernes pour l’éducation de la jeunesse.
Des plans et des institutions différentes ont eu lieu, à ce qu’il semble, dans d’autres temps et chez d’autres nations.
Dans les républiques anciennes de la Grèce, tout citoyen libre était instruit, sous la direction du magistrat, dans les exercices gymnastiques et dans la musique. Les exer­cices gymnastiques avaient pour objet de lui fortifier le corps, de lui donner du courage, et de le préparer aux fatigues et aux dangers de la guerre ; et comme les mili­ces grecques, d’après tous les témoignages, ont été les meilleures qui eussent jamais existé dans le monde, il faut bien que cette partie de leur éducation publique ait par­faitement rempli l’objet de son institution. L’autre partie, la musique, avait pour objet, au moins suivant ce que nous en disent les philosophes et les historiens qui nous ont rendu compte de ces institutions, d’humaniser l’âme, d’adoucir le caractère et de disposer l’élève à remplir tous les devoirs civils et moraux de la vie publique et de la vie privée.
Dans l’ancienne Rome, les exercices du Champ-de-Mars avaient le même objet que ceux du gymnase dans l’ancienne Grèce, et ils semblent avoir aussi bien atteint leur but. Mais chez les Romains, il n’y avait rien qui répondît à l’éducation musicale des Grecs. Cependant la morale des Romains, dans la vie publique comme dans la vie privée, paraît avoir été non-seulement égale, mais de beaucoup supérieure à celle des Grecs sur tous les points. Quant à la supériorité de leur morale dans la vie privée, nous avons le témoignage exprès de Polybe et celui de Denis d’Halicarnasse, deux auteurs bien au fait des deux nations ; et d’un bout à l’autre, l’histoire des Grecs et des Romains atteste la supériorité de la morale publique des derniers. Le caractère doux et modéré des factions rivales est, à ce qu’il semble, la circonstance qui dépose le plus fortement en faveur de la morale publique chez un peuple libre. Or, les factions des Grecs furent presque toujours violentes et sanguinaires, tandis qu’à Rome, jusqu’au temps des Gracques, il n’avait pas encore été versé une seule goutte de sang dans une faction ; et dès le temps des Gracques, on peut regarder la république romaine comme réellement en dissolution. Ainsi, malgré les autorités si respectables de Platon, d’Aris­tote et de Polybe, et malgré les raisons fort ingénieuses dont M. de Montes­quieu cherche à appuyer ces autorités, il parait probable que l’éducation musicale des Grecs ne contribua guère à l’adoucissement de leurs mœurs, puisque celles des Romains, sans aucune éducation de ce genre, leur étaient au total supérieures. Le respect de ces anciens sages pour les institutions de leurs ancêtres les avait probablement disposés à trouver une profonde raison politique dans ce qui n’était peut-être autre chose qu’un antique usage, continué sans interruption depuis la période la plus reculée de ces sociétés jusqu’aux temps auxquels elles étaient parvenues à un degré considérable de raffinement. Chez toute nation barbare, la musique et la danse sont les principaux amusements, et ce sont les premiers talents à posséder pour faire les charmes de la société. Il en est actuellement ainsi chez les nègres de la côte d’Afrique ; c’était la même chose chez les anciens Celtes, chez les anciens Scandinaves et, comme nous pouvons le voir dans Homère, chez les anciens Grecs dans les temps antérieurs à la guerre de Troie. Quand les tribus grecques se formèrent en petites républiques, il était naturel que l’étude de ces arts agréables continuât pendant longtemps à faire partie de l’éducation publique et ordinaire du peuple.
Les maîtres qui instruisaient la jeunesse dans la musique ou dans les exercices militaires ne paraissent pas avoir été payés par l’État, ni même nommés par lui à cet emploi, soit à Rome, soit même à Athènes, la république de la Grèce sur les mœurs et les coutumes de laquelle nous avons le plus de lumières. L’État exigeait de chaque citoyen libre qu’il se rendît propre à défendre son pays à la guerre, et que dans cette vue il se formât aux exercices militaires. Mais il lui laissait la liberté du choix de ses maîtres pour s’y instruire, et ne lui faisait, à ce qu’il semble, aucune autre avance à cet égard, que celle du champ ou lieu public réservé pour ces exercices.
Dans les premiers âges des républiques grecque et romaine, il paraît que les autres par­ties de l’éducation consistaient à apprendre à lire, écrire et compter selon l’arith­métique du temps. Les plus riches citoyens acquéraient souvent ces connaissan­ces, à ce qu’il semble, par le secours de quelque précepteur domestique qui était, en général, ou un esclave ou un affranchi ; et les citoyens plus pauvres allaient pour le même objet aux écoles de certains maîtres qui faisaient métier d’enseigner ces choses moyen­nant une rétribution. Toutefois, ces parties de l’éducation étaient entièrement laissées aux soins des parents ou tuteurs de chaque individu. Il ne paraît pas que l’État se soit jamais attribué sur elles aucun droit de direction ni d’inspection. À la vérité, une loi de Solon dispensait les enfants de soutenir la vieillesse de leurs parents, lorsque ceux-ci avaient négligé de leur faire apprendre un métier ou un emploi lucratif.
Dans les progrès de la civilisation, quand la rhétorique et la philosophie vinrent à être en honneur, les gens d’une condition relevée avaient coutume d’envoyer leurs enfants aux écoles des philosophes et des rhéteurs, pour s’y instruire dans ces sciences que tout homme bien né se piquait de savoir. Mais ces écoles n’étaient pas entrete­nues par l’État ; pendant longtemps il ne fit simplement que les tolérer. Longtemps même la demande en fait de philosophie et de rhétorique fut si peu considérable, que les premiers maîtres qui s’annoncèrent pour professer l’une et l’autre de ces sciences, ne pouvant trouver d’occupation constante dans une seule ville, quelle qu’elle fût, furent obligés de voyager pour enseigner tantôt dans un endroit, tantôt dans l’autre. C’est ainsi que vécurent Zénon d’Élée, Protagoras, Gorgias, Hippias et plusieurs au­tres. À mesure que la demande vint à augmenter, les écoles de philosophie et celles de rhétorique devinrent stationnaires, d’abord à Athènes, et ensuite dans plusieurs autres villes. Toutefois, il ne paraît pas que l’État leur ait jamais donné d’autre encou­ra­gement que d’assigner à quelques-unes d’elles un lieu pour enseigner, ce qui fut fait aussi quelquefois par des donateurs particuliers. Ce fut l’État, à ce qu’il semble, qui assigna l’Académie à Platon, le Lycée à Aristote, et le portique à Zénon de Citta, le fondateur de la secte stoïcienne. Mais Épicure légua ses jardins à l’école qu’il avait fondée. jusque vers le temps de Marc-Antonin, on ne voit pas qu’aucun professeur ait été salarié par l’État, ou ait reçu d’autres émoluments que les honoraires ou rétribu­tions que lui payaient ses écoliers. La gratification que cet empereur philosophe accorda pour un maître de philosophie, comme nous l’apprend Lucien[30], ne dura vrai­semblablement pas au-delà de la vie de l’empereur. Nous ne voyons rien d’équiva­lent aux privilèges des gradués, ni qu’il fût nécessaire d’avoir suivi quelqu’une de ces écoles pour avoir la faculté de pratiquer un emploi ou une profession particulière. Si l’opinion qu’on se formait de leur utilité ne leur attirait pas d’écoliers, la loi ne forçait personne à y aller, ni ne récompensait personne pour y avoir été. Les maîtres n’avaient aucune espèce de juridiction sur leurs élèves, ni d’autre autorité que cette autorité naturelle que la supériorité de vertu et de talent donne toujours sur les jeunes gens à ceux qui sont chargés de quelques parties de leur éducation.
À Rome, l’étude des lois civiles faisait une partie de l’éducation, non de la plupart des citoyens, mais de quelques familles particulières. Cependant, les jeunes gens qui désiraient acquérir la connaissance des lois n’avaient pas d’école publique où ils pussent aller s’instruire, et la seule ressource qu’ils eussent pour les étudier, c’était de fréquenter la société de ceux de leurs parents et amis qui passaient pour savants en cette partie. Il n’est peut-être pas inutile de remarquer que, quoique les lois des Douze-Tables fussent pour la plupart copiées sur celles de quelques anciennes répu­bliques grecques, cependant il ne paraît pas que l’étude des lois ait jamais fait l’objet d’une science dans aucune république de la Grèce ; à Rome, elle fut de bonne heure une science, et elle donna aux citoyens qui avaient la réputation de l’entendre un lustre considérable. Dans les anciennes républiques de la Grèce, et particulièrement à Athènes, les cours ordinaires de justice consistaient en des portions nombreuses du peuple et, par conséquent, en des assemblées tumultueuses, qui le plus souvent décidaient au hasard ou selon que la clameur, la faction ou l’esprit de parti venait à entraîner la décision. La honte d’avoir rendu une sentence injuste, étant répartie entre cinq cents, mille ou quinze cents personnes (car quelques-unes de leurs cours étaient aussi nombreuses), devenait une charge assez peu sensible pour chaque individu. À Rome, au contraire, les principales cours de justice étaient composées d’un seul juge ou d’un petit nombre de juges, dont l’honneur ne pouvait manquer d’être extrêmement compromis par une décision injuste ou inconsidérée, attendu surtout qu’ils délibé­raient toujours en public. Dans les questions douteuses, le soin extrême que ces juges avaient de se garantir de tout reproche, faisait qu’ils cherchaient naturellement à se retrancher derrière l’exemple ou les jugements précédents des juges qui avaient siégé avant eux, ou dans la même cour, ou dans quelque autre. Cette attention à la pratique reçue et aux décisions précédentes fit que les lois romaines furent arrangées dans ce système régulier et méthodique dans lequel elles sont parvenues jusqu’à nous ; et une pareille attention, dans tout autre endroit où elle a eu lieu, a produit le même effet sur les lois du pays. Cette supériorité des mœurs des Romains sur celles des Grecs, si fort remarquée par Polybe et Denis d’Halicarnasse, fut due vraisemblablement à la constitution plus parfaite de leurs cours de justice, plutôt qu’à aucune des circons­tan­ces auxquelles ces auteurs l’attribuent. On dit que les Romains étaient particulière­ment distingués par un plus grand respect de la religion du serment. Mais des gens accoutumés à ne prêter de serment que devant une cour de justice éclairée et vigilante devaient naturellement avoir bien plus d’égard à la chose qu’ils avaient jurée, qu’un peuple habitué à remplir la même forme devant des assemblées populaires et tumultueuses.
On m’accordera sans peine que les talents civils et militaires des Grecs et des Romains étaient pour le moins égaux à ceux de quelque nation moderne que ce soit. Nous sommes plutôt portés, par préjugé, à en exagérer le mérite. Or, si l’on en excepte ce qui avait rapport aux exercices militaires, il ne paraît pas que l’État ait pris la moindre peine pour former ces grands talents ; car on ne me fera jamais croire qu’on en était redevable à l’éducation musicale des Grecs. Il n’y manqua cependant pas de maître, à ce que nous voyons, pour instruire les gens bien nés de ces différentes nations, dans tout art et toute science que leur état social pouvait leur rendre agréable ou nécessaire. La demande de ces sortes d’enseignement produisit ce qu’elle produit toujours, le talent de les donner ; et nous voyons que l’émulation, fruit nécessaire d’une concurrence illimitée, y porta ce talent à un très-haut degré de perfection. Par l’attention qu’excitaient les anciens philosophes, par l’empire qu’ils prenaient sur les opinions et les principes de leurs auditeurs, par la faculté qu’ils possédaient d’impri­mer un caractère et un ton particuliers à la conduite et à la conversation de ces auditeurs, ils paraissent avoir été extrêmement supérieurs à qui que ce soit de nos maîtres modernes.
De nos jours, l’activité des professeurs publics est plus ou moins émoussée par les circonstances qui les rendent plus ou moins indépendants de leur succès et de leur renommée dans leur profession. Les traitements fixes qu’ils reçoivent mettent aussi le maître particulier qui chercherait à entrer en concurrence avec eux, sur le même pied que serait un marchand qui voudrait commercer sans gratification, concurremment avec ceux qui en reçoivent une considérable dans leur commerce. S’il vend ses marchandises à peu près au même prix qu’eux, il ne peut pas avoir le même profit ; alors la pauvreté et la misère pour le moins, peut-être la ruine et la banqueroute, seront inévitablement son lot. S’il essaie de vendre ses marchandises beaucoup plus cher, il y a à parier qu’il aura si peu de chalands, que sa situation ne s’en trouvera pas beaucoup meilleure. D’ailleurs, les privilèges des gradués, dans beaucoup de pays, sont nécessaires ou au moins extrêmement avantageux à presque tous les hommes des professions savantes, c’est-à-dire à la plus grande partie de ceux qui ont besoin d’une éducation savante. Or, on ne peut obtenir ces privilèges qu’en suivant les leçons des professeurs publics. On aura beau suivre, avec la plus grande assiduité, les meilleures instructions possibles auprès d’un maître particulier, ce ne sera pas toujours un titre pour prétendre à ces privilèges. Ce sont toutes ces différentes causes qui font qu’un maître particulier, dans quelqu’une des sciences qu’on enseigne communément dans les universités, est, en général, regardé parmi nous comme de la dernière classe des gens de lettres. Un homme qui a quelque vrai talent ne saurait guère trouver de manière moins honorable et moins lucrative de l’employer. Il s’ensuit que les dota­tions des écoles et des collèges ont non-seulement nui à l’activité et à l’exactitude des professeurs publics, mais ont même rendu presque impossible de se procurer de bons maîtres particuliers[31].
S’il n’y avait pas d’institutions publiques pour l’éducation, alors il ne s’enseignerait aucune science, aucun système ou cours d’instruction dont il n’y eût pas quelque demande, c’est-à-dire aucun que les circonstances du temps ne rendissent ou néces­saire, ou avantageux, ou convenable d’apprendre. Un maître particulier ne trouverait jamais son compte à adopter, pour l’enseignement d’une science reconnue utile, quelque système vieilli et totalement décrié, ni à enseigner de ces sciences générale­ment regardées comme un pur amas de sophismes et de verbiage insignifiant, aussi inutile que pédantesque. De tels systèmes, de telles sciences ne peuvent avoir d’exis­tence ailleurs que dans ces sociétés érigées en corporation pour l’éducation ; sociétés dont la prospérité et le revenu sont, en grande partie, indépendants de leur réputation et totalement. de leur industrie. S’il n’y avait pas d’institutions publiques pour l’éduca­tion, on ne verrait pas un jeune homme de famille, après avoir passé par le cours d’études le plus complet que l’état actuel des choses soit censé comporter, et l’avoir suivi avec de l’application et des dispositions, apporter dans le monde la plus parfaite ignorance de tout ce qui est le sujet ordinaire de la conversation entre les personnes bien nées et les gens de bonne compagnie[32].
Il n’y a pas d’institutions publiques pour l’instruction des femmes et, en consé­quence, dans le cours ordinaire de leur éducation, il n’y a rien d’inutile, d’absurde ni de fantastique. On leur enseigne ce que leurs parents et tuteurs jugent nécessaire ou utile pour elles de savoir, et on ne leur enseigne pas autre chose[33]. Chaque partie de leur éducation tend évidemment à quelque but utile ; elle a pour objet ou de relever les grâces naturelles de leur personne, ou de former leur moral à la réserve, à la modestie, à la chasteté, à l’économie ; de les mettre dans le cas de devenir mères de famille, et de se comporter, quand elles le seront devenues, d’une manière convenable à cet état. Dans toutes les époques de sa vie, une femme sent qu’il n’y a aucune partie de son éducation dont elle ne tienne quelque avantage ou quelque agrément. Il arrive rarement que, dans aucun instant de sa carrière, un homme retire quelque utilité ou quelque plaisir de certaines parties de son éducation, qui en ont été les plus fatigantes et les plus ennuyeuses.
L’État ne devrait-il donc s’occuper en aucune manière, va-t-on me demander, de l’éducation du peuple ? Ou s’il doit s’en occuper, quelles sont les différentes parties de l’éducation auxquelles il devrait donner des soins dans les différentes classes du peuple ? Et de quelle manière doit-il donner ces soins ?
Dans certaines circonstances, l’état de la société est tel qu’il place nécessairement la plus grande partie des individus dans des situations propres à former naturellement en eux, sans aucuns soins de la part du gouvernement, presque toutes les vertus et les talents qu’exige ou que peut comporter peut-être cet état de société. Dans d’autres circonstances, l’état de la société est tel qu’il ne place pas la plupart des individus dans de pareilles situations, et il est indispensable que le gouvernement prenne quelques soins pour empêcher la dégénération et la corruption presque totale du corps de la nation.
Dans les progrès que fait la division du travail, l’occupation de la très-majeure partie de ceux qui vivent de travail, c’est-à-dire de la masse du peuple, se borne à un très-petit nombre d’opérations simples, très-souvent à une ou deux. Or, l’intelligence de la plupart des hommes se forme nécessairement par leurs occupations ordinaires. Un homme qui passe toute sa vie à remplir un petit nombre d’opérations simples, dont les effets sont aussi peut-être toujours les mêmes ou très-approchant les mêmes, n’a pas lieu de développer son intelligence ni d’exercer son imagination à chercher des expédients pour écarter des difficultés qui ne se rencontrent jamais ; il perd donc naturellement l’habitude de déployer ou d’exercer ces facultés et devient, en général, aussi stupide et aussi ignorant qu’il soit possible à une créature humaine de le devenir ; l’engourdissement de ses facultés morales le rend non-seulement incapable de goûter aucune conversation raisonnable ni d’y prendre part, mais même d’éprouver aucune affection noble, généreuse ou tendre et, par conséquent, de former aucun jugement un peu juste sur la plupart des devoirs même les plus ordinaires de la vie privée. Quant aux grands intérêts, aux grandes affaires de son pays, il est totalement hors d’état d’en juger, et à moins qu’on n’ait pris quelques peines très-particulières pour l’y préparer, il est également inhabile à défendre son pays à la guerre ; l’uniformité de sa vie séden­taire corrompt naturellement et abat son courage, et lui fait envisager avec une aversion mêlée d’effroi la vie variée, incertaine et hasardeuse d’un soldat ; elle affaiblit même l’activité de son corps, et le rend incapable de déployer sa force avec quelque vigueur et quelque constance, dans tout autre emploi que celui pour lequel il ’a été élevé. Ainsi, sa dextérité dans son métier particulier est une qualité qu’il semble avoir acquise aux dépens de ses qualités intellectuelles, de ses vertus sociales et de ses dispositions guerrières. Or, cet état est celui dans lequel l’ouvrier pauvre, c’est-à-dire la masse du peuple, doit tomber nécessairement dans toute société civilisée et avan­cée en industrie, à moins que le gouvernement ne prenne des précautions pour prévenir ce mal.
Il n’en est pas ainsi dans les sociétés qu’on appelle communément barbares : celles des peuples chasseurs, des pasteurs et même des agriculteurs, dans cet état informe de l’agriculture qui précède le progrès des manufactures et l’extension du commerce étranger. Dans ces sociétés, les occupations variées de chaque individu l’obligent à exercer sa capacité par des efforts continuels, et à inventer des expédients pour écarter des difficultés qui se présentent sans cesse. L’imagination y est tenue toujours en haleine, et l’âme n’a pas le loisir d’y tomber dans cet engourdissement et cette stupi­dité qui semblent paralyser l’intelligence de presque toutes les classes inférieures du peuple dans une société civilisée. Dans ces sociétés barbares, ou du moins qu’on nomme telles, tout homme est guerrier, comme on l’a déjà observé ; tout homme est aussi, à un certain point, homme d’État, et peut porter un jugement passable sur les affaires relatives à l’intérêt général de la société, et sur la conduite de ceux qui le gouvernent. Chez ces peuples, il n’y a presque pas un seul particulier qui ne puisse voir, au premier coup d’œil, jusqu’à quel point les chefs de la société sont bons juges en temps de paix et bons généraux en temps de guerre. À la vérité, dans une telle société, il n’y a guère de probabilité pour un homme d’y acquérir jamais cette perfec­tion et ce raffinement d’intelligence que certains hommes possèdent quel­quefois dans un état de civilisation plus avancé. Quoique, dans une société agreste, les occupations de chaque individu ne laissent pas que d’être fort variées, il n’y a pas une grande variété d’ occupations dans la société en général. Il n’y a guère d’homme qui ne fasse ou ne soit capable de faire presque tout ce qu’un autre homme fait ou peut faire. Tout homme a bien un certain degré de connaissance, d’habileté et d’imagination, mais il n’y a guère d’individu qui y possède ces qualités à un haut degré, quoique toutefois le degré auquel on les y possède communément soit, en général, tout ce qu’il faut pour conduire des affaires simples comme celles d’une telle société. Dans un État civilisé, au contraire, quoiqu’il y ait peu de variété dans les occupations de la majeure partie des individus, il y en a une presque infinie dans celles de la société en général. Cette multitude d’occupations diverses offre une variété innombrable d’objets à la médita­tion de ce petit nombre d’hommes qui, n’étant attachés à aucune occupation en parti­culier, ont le loisir et le goût d’observer les occupations des autres. En contemplant une aussi grande quantité d’objets variés, leur esprit s’exerce nécessairement à faire des combinaisons et des comparaisons sans fin, et leur intelligence en acquiert un degré extraordinaire de sagacité et d’étendue. Cependant, à moins qu’il n’arrive que ce petit nombre d’hommes se trouve placé dans des situations absolument particulières, leurs grands talents, tout honorables qu’ils sont pour eux-mêmes, contribuent fort peu au bonheur ou au bon gouvernement de la société dont ils sont membres. Malgré les talents relevés de ce petit nombre d’hommes distingués, tous les plus nobles traits du caractère de l’homme peuvent être en grande partie effaces et anéantis dans le corps de la nation.
L’éducation de la foule du peuple, dans une société civilisée et commerçante, exige peut-être davantage les soins de l’État que celle des gens mieux nés et qui sont dans l’aisance. Les gens bien nés et dans l’aisance ont, en général, dix-huit à dix-neuf ans avant d’entrer dans les affaires, dans la profession ou le genre de commerce qu’ils se proposent d’embrasser. Ils ont avant cette époque tout le temps d’acquérir, ou au moins de se mettre dans le cas d’acquérir par la suite toutes les connaissances qui peuvent leur faire obtenir l’estime publique ou les en rendre dignes ; leurs parents ou tuteurs sont assez jaloux, en général, de les voir ainsi élevés, et sont le plus souvent disposés à faire toute la dépense qu’il faut pour y parvenir. S’ils ne sont pas toujours très-bien élevés, c’est rarement faute de dépenses faites pour leur donner de l’éducation, c’est plutôt faute d’une application convenable de ces dépenses. Il est rare que ce soit faute de maîtres, mais c’est souvent à cause de l’incapacité et de la négligence des maîtres qu’on a, et de la difficulté ou plutôt de l’impossibilité qu’il y a de s’en procurer de meilleurs dans l’état actuel des choses. Et puis, les occupations auxquelles les gens bien nés et dans l’aisance passent la plus grande partie de leur vie ne sont pas, comme celles des gens du commun du peuple, des occupations simples et uniformes ; elles sont presque toutes extrêmement compliquées et de nature à exercer leur tête plus que leurs mains. Il ne se peut guère que l’intelligence de ceux qui se livrent à de pareils emplois vienne à s’engourdir faute d’exercice. D’un autre côté, les emplois des gens bien nés et ayant quelque aisance ne sont guère de nature à les enchaîner du matin au soir. En général, ils ne laissent pas d’avoir certaine quantité de moments de loisirs pendant lesquels ils peuvent se perfectionner dans toute branche de connaissances utiles ou agréables dont ils auront pu se donner les premiers éléments, ou dont ils auront pu prendre le goût dans la première époque de leur vie.
Il n’en est pas de même des gens du peuple ; ils n’ont guère de temps de reste à mettre à leur éducation. Leurs parents peuvent à peine suffire à leur entretien pendant l’enfance. Aussitôt qu’ils sont en état de travailler, il faut qu’ils s’adonnent à quelque métier pour gagner leur subsistance. Ce métier est aussi, en général, si simple et si uniforme, qu’il donne très-peu d’exercice à leur intelligence ; tandis qu’en même temps leur travail est à la fois si dur et si constant, qu’il ne leur laisse guère de loisir, encore moins de disposition à s’appliquer, ni même à penser à aucune autre chose.
Mais quoique dans aucune société civilisée les gens du peuple ne puissent jamais être aussi bien élevés que les gens nés dans l’aisance, cependant les parties les plus essentielles de l’éducation, lire, écrire et compter, sont des connaissances qu’on peut acquérir à un âge si jeune, que la plupart même de ceux qui sont destinés aux métiers les plus bas ont le temps de prendre ces connaissances avant de commencer à se met­tre à leurs travaux. Moyennant une très-petite dépense, l’État peut faciliter, peut encourager l’acquisition de ces parties essentielles de l’éducation parmi la masse du peuple, et même lui imposer, cri quelque sorte, l’obligation de les acquérir.
L’État peut faciliter l’acquisition de ces connaissances, en établissant dans chaque paroisse ou district une petite école où les enfants soient instruits pour un salaire si modique, que même un simple ouvrier puisse le donner ; le maître étant en partie, mais non en totalité, payé par l’État, parce que, s’il l’était en totalité ou même pour la plus grande partie, il pourrait bientôt prendre l’habitude de négliger son métier. En Écosse, l’établissement de pareilles écoles de paroisse a fait apprendre à lire à presque tout le commun du peuple, et même, à une très-grande partie, à écrire et à compter. En Angleterre, l’établissement des écoles de charité a produit un effet du même genre, mais non pas aussi généralement, parce que l’établissement n’est pas aussi universel­lement répandu. Si, dans ces petites écoles, les livres dans lesquels on enseigne à lire aux enfants étaient un peu plus instructifs qu’ils ne le sont pour l’ordinaire ; et si, au lieu de montrer aux enfants du peuple à balbutier quelques mots de latin, comme on fait quelquefois dans ces écoles, ce qui ne peut jamais leur être bon à rien, on leur enseignait les premiers éléments de la géométrie et de la mécanique, l’éducation littéraire de cette classe du peuple serait peut-être aussi complète qu’elle est suscep­tible de l’être. Il n’y a presque pas de métier ordinaire qui ne fournisse quelque occa­sion d’y faire l’application des principes de la géométrie et de la mécanique et qui, par conséquent, ne donnât lieu aux gens du peuple de s’exercer petit à petit et de se perfectionner dans ces principes qui sont l’introduction nécessaire aux sciences les plus sublimes, ainsi que les plus utiles.
L’État peut encourager l’acquisition de ces parties les plus essentielles de l’éduca­tion, en donnant de petits prix ou quelques petites marques de distinction aux enfants du peuple qui y excelleraient.
L’État peut imposer à presque toute la masse du peuple l’obligation d’acquérir ces parties de l’éducation les plus essentielles, en obligeant chaque homme à subir un examen ou une épreuve sur ces articles avant de pouvoir obtenir la maîtrise dans une corporation, ou la permission d’exercer aucun métier ou commerce dans un village ou dans une ville incorporée.
C’est ainsi que les républiques grecques et la république romaine, en facilitant les moyens de se former aux exercices militaires et gymnastiques, en encourageant la pratique de ces exercices, et en imposant à tout le corps de la nation la nécessité de les apprendre, entretinrent les dispositions martiales de leurs citoyens respectifs. Elles facilitèrent les moyens de se former à ces exercices, en ouvrant un lieu public pour les apprendre et les pratiquer, et en accordant à certains maîtres le privilège de les enseigner dans ce lieu. Il ne paraît pas que ces maîtres aient eu d’autre traitement ni aucune autre espèce de privilège. Leur récompense consistait entièrement dans ce qu’ils retiraient de leurs écoliers ; et un citoyen qui avait appris ces exercices dans les gymnases publics n’avait aucune espèce d’avantage légal sur un autre qui les aurait appris particulièrement, pourvu que celui-ci les eût également bien appris. Ces républiques encouragèrent la pratique de ces exercices, en accordant de petits prix et quelques marques de distinction à ceux qui y excellaient. Un prix remporté aux jeux Olympiques, Isthmiens ou Néméens, était un grand honneur, non-seulement pour celui qui le gagnait, mais encore pour sa famille et toute sa parenté. L’obligation où était chaque citoyen de servir un certain nombre d’années sous les drapeaux de la république, quand on l’y appelait, le mettait bien dans la nécessité d’apprendre ces exercices, sans lesquels il n’eût pas été propre à remplir son service.
Il ne faut que l’exemple de l’Europe moderne pour démontrer que, dans les progrès de la civilisation et de l’industrie, la pratique des exercices militaires, si le gouvernement ne se donne pas les soins propres à la maintenir, va insensiblement en déclinant, et avec elle le caractère martial du corps de la nation. Or, la sûreté d’une société dépend toujours plus ou moins du caractère guerrier de la masse du peuple. Dans les temps actuels, il est vrai, ce caractère seul, s’il n’était pas soutenu par une armée de ligne bien disciplinée, ne serait peut-être pas suffisant pour la défense et la sûreté nationales. Mais, dans une société où chaque citoyen aurait l’esprit guerrier, certainement il faudrait une armée de ligne moins forte. D’ailleurs, cet esprit guerrier diminuerait nécessairement de beaucoup les dangers réels ou imaginaires dont on croit communément qu’une armée de ligne menace la liberté ; de même qu’il facili­terait extrêmement les efforts de cette armée de ligne contre un ennemi étranger qui voudrait envahir le pays, de même aussi il opposerait à ces mêmes efforts une extrême résistance, si malheureusement ils étaient jamais dirigés contre la consti­tution de l’État.
Les anciennes institutions de la Grèce et de Rome ont, à ce qu’il semble, beaucoup mieux réussi à entretenir l’esprit martial dans le corps de la nation, que les établis­sements de nos milices modernes. Elles étaient beaucoup plus simples. Quand ces institutions étaient une fois établies, elles marchaient d’elles-mêmes, et il ne fallait que peu ou point d’attention de la part du gouvernement pour les maintenir en parfaite vigueur. Tandis que pour tenir la main même d’une manière tant soit peu passable à l’exécution des règlements compliqués de quelques-unes de nos milices modernes, il faut dans le gouvernement une vigilance active et continuelle, sans quoi ils ne man­quent jamais de tomber en désuétude, puis enfin dans un oubli total. D’ailleurs, les anciennes institutions avaient une influence beaucoup plus universelle. Par leur moyen, tout le corps de la nation était complètement formé à l’usage des armes, tandis que, par les règlements de nos milices modernes, il n’y a qu’une très-petite partie de la nation qui puisse être exercée, si l’on en excepte peut-être les milices de la Suisse. Or, un homme lâche, un homme incapable de se défendre ou de se venger d’un affront, manque d’une des parties les plus essentielles au caractère d’un homme. Il est aussi mutilé et aussi difforme dans son âme, qu’un autre l’est dans son corps lorsqu’il est privé de quelques-uns des membres les plus essentiels, ou qu’il en a perdu l’usage. Le premier est évidemment le plus affligé et le plus misérable des deux, parce que le bonheur et le malheur résidant entièrement dans la partie intellectuelle, ils doivent nécessairement dépendre davantage de l’état de santé ou de maladie de l’âme, de la régularité ou des vices de sa conformation, plutôt que de la constitution physique de l’individu. Quand même le caractère martial d’un peuple ne devrait être d’aucune utilité pour la défense de la société, cependant le soin de préserver le corps de la nation de cette espèce de mutilation morale, de cette honteuse difformité et de cette condition malheureuse qu’entraîne avec soi la poltronnerie, est une considération encore assez puissante pour mériter de la part du gouvernement la plus sérieuse atten­tion ; de même que ce serait un objet digne de la plus sérieuse attention d’empêcher qu’il ne se répandît parmi le peuple une lèpre ou quelque autre incommodité malpro­pre et répugnante, encore qu’elle ne fût ni mortelle ni dangereuse. Quand il ne pourrait résulter d’une telle attention aucun bien public qui fût positif, n’en serait-ce pas toujours un que d’avoir prévenu un aussi grand mal public ?
On en peut dire autant de la stupidité et de l’ignorance crasse qui semblent si souvent abâtardir l’intelligence des classes inférieures du peuple dans une société civilisée. Un homme qui n’a pas tout l’usage de ses facultés intellectuelles est encore plus avili, s’il est possible, qu’un poltron même ; il est mutilé et difforme, à ce qu’il semble, dans une partie encore plus essentielle du caractère de la nature humaine. Quand même l’État n’aurait aucun avantage positif à retirer de l’instruction des classes inférieures du peuple, il n’en serait pas moins digne de ses soins qu’elles ne fussent pas totalement dénuées d’instruction.
Toutefois, l’État ne retirera pas de médiocres avantages de l’instruction qu’elles auront reçue. Plus elles seront éclairées, et moins elles seront sujettes à se laisser égarer par la superstition et l’enthousiasme, qui sont chez les nations ignorantes les sources ordinaires des plus affreux désordres. D’ailleurs, un peuple instruit et intelli­gent est toujours plus décent dans sa conduite et mieux disposé à l’ordre, qu’un peuple ignorant et stupide. Chez celui-là, chaque individu a plus le sentiment de ce qu’il vaut et des égards qu’il a droit d’attendre de ses supérieurs légitimes, par conséquent il est plus disposé à les respecter. Le peuple est plus en état d’apprécier les plaintes intéres­sées des mécontents et des factieux ; il en est plus capable de voir clair au travers de leurs déclamations ; par cette raison, il est moins susceptible de se laisser entraîner dans quelque opposition indiscrète ou inutile contre les mesures du gouvernement. Dans des pays libres, où la tranquillité des gouvernants dépend extrêmement de l’opinion favorable que le peuple se forme de leur conduite, il est certainement de la dernière importance que le peuple ne soit pas disposé à en juger d’une manière capricieuse ou inconsidérée[34].

ARTICLE III.
Des dépenses qu’exigent les institutions pour l’instruction des personnes de tout âge.

Les institutions pour l’instruction des personnes de tout âge sont principalement celles qui ont pour objet l’instruction religieuse.
C’est un genre d’instruction dont l’objet est bien moins de rendre les hommes bons citoyens dans ce monde, que de les préparer pour un monde meilleur dans une vie future. Les maîtres qui enseignent la doctrine où est contenue cette instruction, de même que les autres maîtres, peuvent dépendre entièrement, pour leur subsistance, des contributions volontaires de leurs auditeurs, ou bien ils peuvent la tirer de quelque autre fonds auquel la loi de leur pays leur donne droit, tels qu’une propriété foncière, une dîme ou redevance territoriale, des gages ou appointements fixes. Leur activité, les efforts de leur zèle et de leurs moyens seront vraisemblablement beau­coup plus grands dans le premier cas que dans l’autre. Sous ce rapport, les professeurs de religions nouvelles ont toujours eu un avantage considérable en attaquant les systèmes religieux anciens et légalement établis, parce que dans ceux-ci le clergé, se reposant sur ses bénéfices, avait insensiblement négligé de maintenir, dans la masse du peuple, la dévotion et la ferveur de la foi, et que, s’abandonnant à l’indolence et à l’oisiveté, il était devenu absolument incapable de tout effort de vigueur, même pour défendre sa propre existence. Le clergé d’une religion tout établie et bien dotée finit par se composer d’hommes instruits et agréables, qui possèdent toutes les qualités des gens du monde, et qui peuvent prétendre à l’estime des personnes bien nées ; mais ces hommes sont dans le cas de perdre insensiblement les qualités tant bonnes que mauvaises qui leur donnaient de l’autorité et de l’influence sur les classes inférieures du peuple, et qui avaient peut-être été la cause primitive de succès et de l’établis­sement de leur religion. Un pareil clergé, quand il vient à être attaqué par une secte d’enthousiastes ardents et populaires, tout stupides et ignorants qu’ils soient, se sent aussi complètement dénué de défense, que les peuples indolents, efféminés et bien nourris des parties méridionales de l’Asie, quand ils furent envahis par les actifs, hardis et affamés Tartares du Nord. Un pareil clergé, dans une semblable occurrence, n’a pour l’ordinaire d’autre ressource que de s’adresser au magistrat civil, et de récla­mer sa force pour persécuter, détruire ou chasser ses adversaires comme des pertur­bateurs de la tranquillité publique. Ce fut ainsi que le clergé catholique romain mit en œuvre la puissance du magistrat civil contre les protestants, et l’Église d’Angleterre contre les dissidents ; c’est ainsi qu’en général toute secte religieuse, ayant une fois joui, pendant un siècle ou deux, de la sécurité d’un établissement légal, s’est trouvée incapable de faire aucune vigoureuse défense contre toute secte nouvelle qui a jugé à propos d’attaquer sa doctrine ou sa discipline. Dans ces occasions, l’avantage, en fait de savoir et de bons écrits, peut être quelquefois du côté de l’Église établie. Mais les finesses de la popularité, tous les talents propres à gagner des prosélytes, sont cons­tam­ment du côté des adversaires. En Angleterre, ces ressources sont depuis long­temps négligées par le clergé richement doté de l’Église établie, et elles sont princi­palement cultivées par les dissidents et par les méthodistes. Cepen­dant, les revenus indépendants qu’on a fondés en beaucoup d’endroits pour les professeurs de la doctrine des dissidents, au moyen de souscriptions volontaires, de fidéicommis et d’autres moyens d’éluder la loi, paraissent avoir extrêmement refroidi le zèle et l’activité de ces professeurs. Beaucoup d’entre eux sont devenus très-savants, gens d’esprit et pasteurs respectables ; mais ils ont, en général, cessé d’être des prêcheurs très-populaires. Les méthodistes, sans avoir la moitié du savoir des dissidents, ont beaucoup plus de crédit parmi le peuple.
Dans l’Église de Rome, le zèle et l’industrie du clergé inférieur sont bien plus soutenus par le puissant motif de l’intérêt personnel, que dans peut-être aucune église protestante légalement établie. Le clergé des paroisses, pour la plupart, tire une por­tion très-considérable de la subsistance des offrandes volontaires du peuple, source de revenu qu’il a mille moyens d’alimenter et de grossir à la faveur de la confession. Les ordres mendiants tirent toute leur subsistance de pareilles offrandes ; ils sont comme les hussards et l’infanterie légère de quelques armées : point de pillage, point de paie. Le clergé des paroisses ressemble à ces maîtres dont la récompense dépend en partie de leur traitement et en partie des honoraires ou rétributions qu’ils retirent de leurs élèves ; or, celles-ci dépendent toujours nécessairement, plus ou moins, de l’activité ou de la réputation du maître. Les ordres mendiants ressemblent aux maîtres dont la subsis­tance est tout entière fondée sur leur activité. Ils sont donc obligés de ne négli­ger aucun des moyens qui peuvent animer la dévotion du commun du peuple. Machia­vel observe, que, dans les treizième et quatorzième siècles, la dévotion et la foi languissantes de l’Église romaine reprirent une nouvelle vie par l’établissement des deux grands ordres mendiants de Saint-Dominique et de Saint-François. Dans les pays catholiques romains, l’esprit de dévotion est entretenu en totalité par les moines et par le clergé le plus pauvre des paroisses. Les grands dignitaires de l’Église, ornés de tous les agréments qui conviennent aux gens du monde et aux personnes de qua­lité, et quelquefois distingués par leurs connaissances, ont bien assez soin de mainte­nir la discipline nécessaire sur leurs inférieurs, mais ne se donnent guère la moindre peine pour l’instruction du peuple.
« La plupart des arts et des professions dans un État », dit l’historien philosophe le plus illustre de ce siècle[35], sont de telle nature, que, tout en servant l’intérêt général de la société, ils sont en même temps utiles et agréables à quelques particuliers ; et dans ce cas, la règle que doit se faire constamment le magistrat (excepté peut-être quand il s’agit d’introduire pour la première fois dans la société quelque art ou profes­sion nouvelle), c’est de laisser la profession à elle-même, et de s’en reposer pour son encouragement sur les particuliers qui en recueillent l’agrément ou l’utilité. Les arti­sans, en voyant leurs profits grossir à mesure qu’ils contentent leurs pratiques, redou­blent, autant qu’il est possible, de zèle et d’industrie ; et lorsque le cours naturel des choses n’est pas troublé par des mesures inconsidérées, on peut être assuré que la marchandise se trouvera, dans tous les temps, à très-peu de chose près, de niveau avec la demande.
Mais il y a aussi quelques métiers qui, quoique utiles et même nécessaires dans un État, ne rapportent cependant ni avantage ni agrément à aucun individu en parti­culier ; et le pouvoir souverain est obligé, à l’égard de ceux qui suivent ces sortes de professions, de s’écarter de sa règle générale de conduite. Il faut leur donner des encouragement publics, afin qu’ils trouvent les moyens de subsister ; et il faut encore s’occuper de prévenir la négligence à laquelle ils seront naturellement sujets à se laisser aller, et cela, soit en attachant des distinctions particulières à la profession, soit en établissant une subordination de rangs fort étendue et une stricte dépendance, soit enfin par quelque autre expédient. Les personnes employées dans les finances, dans la marine militaire et dans la magistrature, sont des exemples de cette classe de personnes.
On pourrait naturellement croire, au premier coup d’œil, que les ecclésiastiques appartiennent à la première classe, et que pour l’encouragement de cette profession, tout comme pour celle des jurisconsultes et des médecins, il faudrait s’en reposer, en toute sûreté, sur la libéralité de chaque particulier attaché à leur doctrine, et qui trouve de l’avantage ou de la consolation à user de leur ministère et de leur secours spirituel. Sans contredit, un surcroît d’encouragement de ce genre ne manquera pas d’aiguillonner leur activité et leur zèle ; sans contredit, leur habileté dans leur profes­sion, aussi bien que leur adresse à gouverner l’esprit du peuple, ne feront qu’aug­menter infailliblement, de jour en jour, par un redoublement continuel de leur part, de pratique, d’étude et d’attention.
Mais si nous examinons la chose plus attentivement, nous verrons que cette activité intéressée du clergé est ce que tout sage législateur doit s’attacher à prévenir, parce que, dans toute religion (excepté la véritable), elle est extrêmement dangereuse, et qu’elle a même une tendance naturelle à corrompre la vraie religion en y mêlant une forte dose de superstition, de sottises et de tromperies. Chacun de ces inspirés prédicants, pour se rendre plus cher et plus sacré aux yeux de ses fidèles, cherchera à exciter l’horreur la plus forte contre toutes les autres sectes, et mettra continuellement ses efforts à ranimer par quelque nouveauté la dévotion languissante de son auditoire. Dans la doctrine qu’on inculquera dans l’esprit du peuple, ni la vérité, ni la morale, ni la décence ne seront respectées. On prêchera de préférence toute maxime qui s’accor­dera mieux avec les affections désordonnées du cœur humain. Pour attirer la pratique à chaque conventicule particulier, on s’attachera à travailler, chaque jour avec plus d’adresse et d’activité, les passions et la crédulité de la populace. Au bout de tout, le magistrat civil finira par s’apercevoir qu’il a payé bien cher son économie prétendue d’épargner la dépense d’un établissement fixe pour les prêtres, et que dans la réalité la manière la plus avantageuse et la plus décente dont il puisse composer avec les gui­des spirituels, c’est d’acheter leur indolence en assignant des salaires fixes à leur pro­fession, et leur rendant superflue toute autre activité que celle qui se bornera sim­plement à empêcher leur troupeau d’aller s’égarer loin de leur bercail, à la recherche d’une nouvelle pâture ; et sous ce rapport les établissements ecclésiastiques, qui d’abord ont été fondés dans des vues religieuses, finissent cependant par servir avan­tageusement les intérêts politiques de la société. »
Mais, quels que puissent avoir été les bons ou mauvais effets de revenus indépendants qu’on a fondés pour le clergé, il est peut-être bien rare que ces effets soient entrés pour la moindre chose dans les motifs de ces fondations. Les temps où les controverses religieuses ont éclaté avec violence ont été, en général, des temps où les factions politiques ne se sont pas fait sentir avec moins de force. Dans ces occasions, chaque parti politique a trouvé ou imaginé qu’il était dans son intérêt de se liguer avec l’une ou l’autre des sectes religieuses opposées. Mais ceci ne pouvait se faire qu’en adoptant, ou au moins en favorisant la doctrine de cette secte particulière. Celle qui avait eu le bonheur de se lier au parti triomphant partageait nécessairement dans les fruits de la victoire de son allié, dont la faveur et la protection la mettaient bientôt en état de dominer sur tous ses adversaires, et de les réduire au silence jusqu’à un certain point. Ces adversaires, en général, s’étaient ligués avec les ennemis de la faction victorieuse et, par conséquent, étaient eux-mêmes les ennemis de cette faction. Le clergé de cette secte particulière, devenu ainsi complètement maître du champ de bataille, et ayant porté au plus haut degré de force son influence et son autorité sur la masse du peuple, se vit assez puissant pour en imposer même aux chefs et aux principaux de sa faction amie, et pour obliger les magistrats civils à respecter ses opinions et ses volontés. Sa première demande fut, pour l’ordinaire, que ces ma­gistrats abattraient et feraient taire toute autre secte ; et la seconde, qu’ils lui assu­re­raient un revenu indépendant. Comme ce clergé, le plus souvent, ne laissait pas que d’avoir beaucoup contribué à la victoire, il paraissait assez juste qu’il eût aussi quel­que part dans la dépouille ; et puis, il commençait à se lasser d’avoir à gagner le peuple, et de dépendre de ses caprices pour subsister. Ainsi, en faisant cette demande, il ne consulta que son bien être et sa commodité, sans beaucoup s’embarrasser de l’effet qui en pourrait résulter dans l’avenir, quant à l’influence et à l’autorité de son ordre. Le magistrat civil, qui ne pouvait satisfaire à la demande du clergé qu’en lui cédant quelque chose qu’il aurait beaucoup mieux aimé prendre ou garder pour lui-même, mit rarement un grand empressement à la lui accorder. Toutefois, la nécessité l’obligea à se soumettre à la fin, quoique ce ne fût souvent qu’après beaucoup de délais, de défaites ou d’excuses supposées.
Mais si la politique n’eût jamais appelé la religion à son aide, si la faction triomphante n’eût jamais été forcée d’adopter la doctrine d’une secte plutôt que celle d’une autre, alors quand elle aurait remporté la victoire, elle aurait vraisemblablement traité toutes les sectes diverses avec indifférence et impartialité, et elle aurait laissé tout individu libre de choisir son prêtre et sa religion comme il jugerait à propos. Sans doute il y aurait eu, dans ce cas, une grande multitude de sectes religieuses. Vraisem­blablement, presque chaque congrégation différente aurait fait par elle-même une petite secte, ou se serait plu à établir de son chef quelques points particuliers de doctrine. Chaque maître en ce genre de profession se serait vu dans la nécessité de faire tous ses efforts et de mettre en œuvre toutes ses ressources, tant pour se conserver ses disciples que pour en augmenter le nombre. Mais, comme tout autre maître de la même profession se serait vu dans la même nécessité de son côté, le succès d’aucun de ces maîtres ou d’aucune de leurs sectes n’aurait pu être très-grand. Le zèle actif et intéressé des maîtres en fait de religion ne peut être dangereux et inquiétant que dans le cas où il n’y aurait qu’une seule secte tolérée dans la société, ou que la totalité d’une immense société serait divisée en deux ou trois grandes sectes, les maîtres dans chaque secte agissant alors de concert et sous l’influence d’une subordination et d’une discipline régulières. Mais ce zèle ne peut être de la moindre conséquence quand toute la société est partagée en deux ou trois centaines, ou peut-être en autant de milliers de petites sectes, dont aucune ne peut être assez considé­rable pour troubler la tranquillité publique. Les maîtres dans chaque secte, se voyant entourés de toutes parts de plus d’adversaires que d’amis, se trouveront bientôt obligés de prendre des manières franches et un esprit de modération, vertus si rares parmi les maîtres ou profès de ces grandes sectes dont la doctrine, étant soutenue par le magistrat civil, est un objet de vénération pour la presque totalité des habitants de grands et puissants empires, et qui ne voient autour d’eux, par conséquent, que des sectateurs, des- disciples et d’humbles admirateurs. Les maîtres dans chaque petite secte, se trouvant presque isolés, seraient obligés de respecter ceux de presque toute autre secte, et ce qu’ils se verraient forcés de se céder mutuellement les uns aux autres, tant pour leur avantage que pour leur agrément réciproques, finirait vraisem­blablement par réduire avec le temps la doctrine de la plupart d’entre eux à cette religion pure et raisonnable, purgée de tout mélange d’absurdités, d’impostures ou de fanatisme, telle que les hommes sages dans tous les âges du monde ont désiré la voir établie, mais telle que la loi positive ne l’a peut-être encore jamais établie et pro­bablement ne l’établira jamais dans aucun pays, parce qu’en matière de religion la loi positive a toujours été, et vraisemblablement sera toujours, plus ou moins soumise à l’influence des superstitions ou de l’enthousiasme populaire.
Ce plan de gouvernement ecclésiastique, ou, pour mieux dire, de suppression de tout gouvernement ecclésiastique, était celui que se proposait d’établir en Angleterre, vers la fin des guerres civiles, la secte dite des Indépendants, une secte, sans aucun doute, d’enthousiastes effrénés. Si ce projet eût été réalisé, encore qu’il fût provenu d’une origine extrêmement peu philosophique, il aurait vraisemblablement, depuis ce temps jusqu’à nos jours, amené, à l’égard de toute espèce de principe religieux, cet esprit de modération et de calme que donne la philosophie. Ce régime a été établi dans la Pennsylvanie, où, quoique les quakers se trouvent former la secte la plus nombreuse, cependant la loi, dans la réalité, n’en favorise aucune plus que l’autre ; aussi dit-on qu’il y a fait naître partout cette modération et ce calme philosophiques.
Mais quand même, en traitant avec une parfaite égalité toutes les sectes religieu­ses, on ne parviendrait pas à amener parmi toutes celles d’un même pays, ni même dans la plupart d’entre elles, ce caractère de modération et cet esprit de tolérance, cependant, pourvu que ces sectes fussent suffisamment nombreuses, et chacune d’elles conséquemment trop faible pour pouvoir troubler la tranquillité publique, le zèle excessif de chaque secte pour sa doctrine particulière ne pourrait guère produire d’effets très-nuisibles ; au contraire, il pourrait même produire quelque bien, et si le gouvernement était parfaitement décidé à les abandonner toutes à elles-mêmes, en les obligeant pourtant à rester tranquilles les unes à l’égard des autres, il n’y a pas de doute qu’elles n’en vinssent bientôt d’elles-mêmes à se subdiviser assez promptement pour devenir en peu de temps aussi nombreuses qu’on pourrait le désirer.
Dans toute société civilisée, dans toute société où la distinction des rangs a été une fois généralement établie, il y a toujours eu deux différents plans ou systèmes de morale ayant cours en même temps : l’un, fondé sur des principes rigoureux, et qui peut s’appeler le système rigide ; l’autre, établi sur des principes libéraux, et que je nomme système relâché. Le premier est, en général, admiré et révéré par le commun du peuple ; l’autre est communément plus en honneur parmi ce qu’on appelle les gens comme il faut, et c’est celui qu’ils adoptent. Le degré de blâme que nous portons sur les vices de légèreté, ces vices qui naissent volontiers d’une grande aisance et des excès de gaieté et de bonne humeur, est ce qui semble constituer la véritable distinc­tion entre ces deux plans ou systèmes opposés. Dans le système libéral ou de morale relâchée, le luxe, la gaieté folle et même la joie déréglée, l’amour du plaisir poussé jusqu’à un certain degré d’intempérance, les fautes contre la chasteté, au moins chez un des deux sexes, etc., pourvu que ces choses ne soient pas accompagnées d’indé­cences grossières et n’entraînent ni fausseté ni injustice, sont en général traitées avec une assez grande indulgence, et sont très-aisément excusées, même entièrement par­données. Dans le système rigide, au contraire, ces excès sont regardés comme une chose détestable dont il faut s’éloigner avec horreur. Les vices qu’engendre la légèreté sont toujours ruineux pour les gens du peuple, et il ne faut souvent qu’une semaine de dissipation et de débauche pour perdre à jamais un pauvre ouvrier, et pour le pousser par désespoir jusqu’aux derniers crimes. Aussi, ce qu’il y a de mieux et de plus rangé parmi les gens du peuple a-t-il toujours fui et détesté ces sortes d’excès, qu’il sait par expérience être si funestes aux gens de sa sorte. Au contraire, même plusieurs années passées dans les excès et le désordre peuvent ne pas entraîner la ruine de ce qu’on appelle un homme comme il faut, et les personnes de cette classe sont très-disposées à regarder comme un des avantages de leur fortune la faculté de pouvoir se permettre quelques excès, et comme un des privilèges de leur état la liberté d’en user ainsi sans encourir la censure et les reproches. Aussi, parmi les personnes de leur condition, regardent-elles de pareils excès avec assez peu de désapprobation, et ne les blâment-elles que très-légèrement ou point du tout.
Presque toutes les sectes religieuses ont pris naissance parmi les masses popu­laires, et c’est de cette classe qu’elles ont, en général, tiré leurs premiers et leurs plus nombreux prosélytes. Aussi le système de morale rigide a-t-il été adopté presque constamment par ces sectes, ou au moins à très-peu d’exceptions près, car il y en a bien quelques-unes à faire. Ce système était le plus propre à mettre la secte en honneur parmi cet ordre de peuple, auquel elle s’adressait toujours quand elle com­mençait à proposer son plan de réformes sur les choses précédemment établies. Plusieurs d’entre ces sectaires, peut-être la plus grande partie, ont même tâché de se donner du crédit en raffinant sur ce système d’austérité, et en le portant jusqu’à la folie et à l’extravagance, et très-souvent ce rigorisme outré a servi plus que toute autre chose à leur attirer les respects et la vénération du peuple.
Un homme ayant de la naissance et de la fortune est, par son état, un membre distingué d’une grande société, qui a les yeux ouverts sur toute sa conduite, et qui l’oblige par là à y veiller lui-même à chaque instant. Son autorité et sa considération dépendent en très-grande partie du respect que la société lui porte. Il n’oserait pas faire une chose qui pût le décrier ou l’avilir, et il est obligé à une observation très-exacte de cette espèce de morale aisée ou rigide que la société, par un accord général, prescrit aux personnes de son rang et de sa fortune. Un homme de basse condition, au contraire, est bien loin d’être un membre distingué d’une grande société. Tant qu’il demeurera à la campagne, dans un village, on peut avoir les yeux sur sa conduite, et il peut être obligé de s’observer. C’est dans cette situation, et dans celle-là seulement, qu’on peut dire qu’il a une réputation à ménager. Mais sitôt qu’il vient dans une grande ville, il est plongé dans l’obscurité la plus profonde ; personne ne le remarque ni ne s’occupe de sa conduite ; il y a dès lors beaucoup à parier qu’il n’y veillera pas du tout lui-même, et qu’il s’abandonnera à toutes sortes de vices et de débauche honteu­se. Il ne sort jamais plus sûrement de cette obscurité, sa conduite n’excite jamais autant l’attention d’une société respectable, que lorsqu’il devient membre de quelque petite secte religieuse ; dès ce moment, il acquiert un degré de considération qu’il n’avait jamais eu auparavant. Tous les frères de sa secte sont intéressés, pour l’hon­neur de la secte, à veiller sur sa conduite ; et s’il cause quelque scandale, s’il vient à trop s’écarter de cette austérité de mœurs qu’ils exigent presque toujours les uns des autres, ils s’empressent de l’en punir par ce qui est toujours une punition très-sévère, même quand il n’en résulte aucun effet civil, l’expulsion ou l’excommunication de la secte. Aussi, dans les petites sectes religieuses, les mœurs des gens du peuple sont pres­que toujours d’une régularité remarquable et, en général, beaucoup plus que dans l’église établie. Souvent, à la vérité, les mœurs de ces petites sectes ont été plutôt dures que sévères, et même jusqu’à en être farouches et insociables.
Il y a néanmoins deux moyens très-faciles et très-efficaces qui, réunis, pourraient servir à l’État pour corriger sans violence ce qu’il y aurait de trop austère ou de vraiment insociable dans les mœurs de toutes les petites sectes entre lesquelles le pays serait divisé.
Le premier de ces deux moyens, c’est l’étude des sciences et de la philosophie, que l’État pourrait rendre presque universelle parmi tous les gens d’un rang et d’une for­tune moyenne, ou plus que moyenne, non pas en donnant des gages à des professeurs pour en faire des paresseux et des négligents, mais en instituant même dans les sciences les plus élevées et les plus difficiles quelque espèce d’épreuve ou d’examen que serait tenue de subir toute personne qui voudrait avoir la permission d’exercer une profession libérale, ou qui se présenterait comme candidat pour une place hono­rable ou lucrative. Si l’État mettait cette classe de personnes dans la nécessité de s’instruire, il n’aurait besoin de se donner aucune peine pour les pourvoir de maîtres convenables. Elles sauraient bien trouver tout de suite elles-mêmes de meilleurs maîtres que tous ceux que l’État eût pu leur procurer. La science est le premier des antidotes contre le poison de l’enthousiasme et de la superstition ; et dès que les clas­ses supérieures du peuple seraient une fois garanties de ce fléau, les classes inférieures n’y seraient jamais exposées.
Le second de ces moyens, c’est la multiplicité et la gaieté des divertissements publics. Si l’État encourageait, c’est-à-dire s’il laissait jouir d’une parfaite liberté tous ceux qui, pour leur propre intérêt, voudraient essayer d’amuser et de divertir le peuple, sans scandale et sans indécence, par des peintures, de la poésie, de la musique et de la danse, par toutes sortes de spectacles et de représentations dramatiques, il viendrait aisément à bout de dissiper dans la majeure partie du peuple cette humeur sombre et cette disposition à la mélancolie, qui sont presque toujours l’aliment de la superstition et de l’enthousiasme. Tous les fanatiques agitateurs de ces maladies populaires ont toujours vu les divertissements publics avec effroi et avec courroux. La gaieté et la bonne humeur qu’inspirent ces divertissements étaient trop incompa­tibles avec cette disposition d’âme qui est la plus analogue à leur but, et sur laquelle ils peuvent le mieux opérer. D’ailleurs, les représentations dramatiques, souvent en exposant leurs artifices au ridicule et quelquefois même à l’exécration publique, furent, pour cette raison, de tous les divertissements publics, l’objet le plus particulier de leur fureur et de leurs invectives.
Dans un pays où la loi ne favoriserait pas les maîtres ou profès d’une religion plus que ceux d’une autre, il ne serait pas nécessaire qu’aucun d’eux se trouvât sous une dépendance particulière ou immédiate du souverain ou du pouvoir exécutif, ni que celui-ci eût à se mêler de les nommer ou de les destituer de leurs emplois. Dans un pareil état de choses, à n’aurait pas besoin de s’embarrasser d’eux le moins du monde, si ce n’est pour maintenir la paix entre eux comme parmi le reste de ses sujets, c’est-à-dire de les empêcher de se persécuter, de se tromper ou de s’opprimer l’un l’autre. Mais il en est tout autrement dans les pays où il y a une religion établie ou dominante. Dans ce cas, le souverain ne peut jamais se regarder en sûreté, à moins qu’il n’ait les moyens de se donner une influence considérable sur la plupart de ceux qui enseignent cette religion. 
Le clergé de toute église établie constitue une immense corporation ; les membres de cette corporation peuvent agir de concert et suivre leurs intérêts sur un même plan et avec un même esprit, autant que s’ils étaient sous la direction d’un seul homme, et très-souvent aussi y sont-ils. Leur intérêt, comme membres d’un corps, n’est jamais le même que celui du souverain, et lui est même quelquefois directement opposé. Leur grand intérêt est de maintenir leur autorité dans le peuple, et cette autorité dépend de l’importance et de l’infaillibilité prétendue de la totalité de la doctrine qu’ils lui inculquent ; elle dépend de la nécessité prétendue d’adopter chaque partie de cette doctrine avec la foi la plus implicite, pour éviter une éternité de peines. Que le souve­rain s’avise imprudemment de paraître s’écarter ou de douter lui-même du plus petit article de leur doctrine, ou qu’il essaie par humanité de protéger ceux auxquels il arrive de faire l’un ou l’autre, alors l’honneur jaloux et chatouilleux d’un clergé qui ne sera en aucune manière sous sa dépendance se trouve à l’instant provoqué à le proscrire comme un profane, et à s’armer de toutes les terreurs de la religion pour forcer le peuple à transporter son obéissance à quelque prince plus soumis et plus orthodoxe. Qu’il essaie de résister à quelques-unes de leurs prétentions ou de leurs usurpations, le danger ne sera pas moins grand. Les princes qui ont osé tenter ce genre d’opposition contre l’Église, outre le crime de rébellion, ont généralement encore été chargés pas surcroît du crime d’hérésie, en dépit de toutes les protestations les plus solennelles de leur foi et de leur humble soumission à tout article de croyance qu’elle jugerait à propos de leur prescrire. Mais l’autorité que donne la religion l’emporte sur toute autre autorité. Les craintes qu’elle inspire absorbent toutes les autres craintes. Quand des professeurs de religion légalement établis propagent parmi le peuple quelque doctrine subversive de l’autorité du souverain, celle-ci ne peut être maintenue que par la force seulement ou par le secours d’une puissante armée. Une armée même, dans ce cas, ne peut donner au souverain une sécurité durable, parce que, si les soldats ne sont pas étrangers (et il est fort rare qu’ils le soient), s’ils sont tirés de la masse du peuple, comme cela doit être presque toujours, il y a à présumer qu’ils seront bientôt corrompus eux-mêmes par cette doctrine populaire. Les révo­lutions continuelles que fit naître à Constantinople l’esprit turbulent du clergé grec, tant que subsista l’empire d’Orient ; les convulsions fréquentes qui éclatèrent dans toutes les parties de l’Europe par suite du caractère factieux et remuant du clergé romain pendant le cours de plusieurs siècles, démontrent assez combien sera toujours incertaine et précaire la situation d’un souverain qui n’a pas les moyens convenables d’exercer son influence sur le clergé de la religion établie et dominante de son pays.
Il est assez évident par soi-même que des articles de foi, ainsi que toutes les matières spirituelles, ne sont pas du département d’un souverain temporel, qui, à quelque point qu’il puisse posséder les qualités propres à protéger le peuple, est rarement censé posséder celles propres à l’instruire et à l’éclairer. Ainsi, pour tout ce qui concerne ces matières, son autorité ne peut guère contrebalancer l’autorité réunie du clergé de l’église établie. Cependant, sa sûreté personnelle et la tranquillité de l’État peuvent très-souvent dépendre de la doctrine que le clergé jugera à propos de répandre sur de pareilles matières. Comme le prince ne peut donc guère s’opposer directement à la décision des membres de ce corps avec assez de poids et d’autorité, il est nécessaire qu’il soit à portée d’influer sur cette décision ; et il ne saurait y influer qu’autant qu’il pourra s’attacher, par des craintes ou des espérances, la majorité des individus de cet ordre. La crainte d’une destitution ou autre punition pareille, et l’espérance d’une promotion à un meilleur bénéfice, sont propres à remplir cet objet.
Dans toutes les églises chrétiennes, les bénéfices ecclésiastiques sont des espèces de franches tenures dont le titulaire a la jouissance, non pas à simple volonté, mais pendant toute sa vie et tant qu’il se comporte bien. Si les bénéficiers tenaient ces biens à un titre plus précaire, et s’ils étaient sujets à en être expulsés au plus léger déplaisir qu’ils auraient causé au souverain ou à ses ministres, il leur serait peut-être impossible de conserver aucune autorité sur le peuple ; et celui-ci, ne les regardant plus alors que comme des mercenaires dépendant de la cour, ne croirait plus à la bonne foi de leurs exhortations. Mais si le souverain s’avisait d’employer la violence ou quelque voie irrégulière pour priver de leurs bénéfices un certain nombre de gens d’Église, par la raison peut-être qu’ils auraient propagé avec un zèle plus qu’ordinaire quelque doc­trine séditieuse ou favorable à une faction, il ne ferait, par une telle persécution, que les rendre, eux et leurs doctrines, dix fois plus populaires et, par conséquent, dix fois plus dangereux et plus embarrassants qu’ils ne l’étaient auparavant. La crainte est presque toujours un mauvais ressort de gouvernement, et elle ne devrait surtout être jamais employée contre aucune classe d’hommes qui ait la moindre prétention à l’indépendance. En cherchant à les effrayer, on ne fait qu’aigrir leur mauvaise humeur et les fortifier dans une résistance, qu’avec des manières plus douces on aurait pu les amener peut-être aisément ou à modérer, ou à abandonner tout à fait. Il est bien rare que le gouvernement de France ait jamais réussi, par les moyens violents qu’il a ordi­nairement mis en œuvre pour obliger les parlements ou cours souveraines de justice à enregistrer quelque édit qui n’était pas populaire.
Cependant, le moyen qu’il employait communément, qui était l’emprisonnement de tous les membres réfractaires, était bien, à ce qu’on pourrait croire, assez énergi­que. Les princes de la maison de Stuart eurent quelquefois recours à de pareilles violences pour venir à bout de quelques-uns des membres du parlement d’Angleterre et, en général, ils ne les trouvèrent pas moins intraitables. On manie aujourd’hui le parlement d’Angleterre d’une autre manière ; et pour prouver qu’on aurait pu encore plus aisément manier, par les mêmes moyens, tous les parlements de France, il ne faut que la petite expérience que fit le duc de Choiseul sur le parlement de Paris, il y a environ douze ans. On n’a pas suivi cette expérience ; car, encore que les voies de persuasion et de ménagement soient toujours les ressorts les plus sûrs et les plus faciles pour gouverner, tout comme la force et la violence sont les plus mauvais et les plus dangereux, cependant tel est l’insolent orgueil naturel à l’homme, qu’il dédaigne presque toujours de faire usage du bon ressort, à moins qu’il ne puisse ou qu’il n’ose se servir du mauvais. Le gouvernement de France a pu et a osé employer la force et, par conséquent, il a dédaigné de se servir des voies de ménagement et de persuasion. Mais, à ce qu’il semble, je crois, par l’expérience de tous les siècles, il n’y a pas de classe d’hommes avec lesquels il soit si dangereux, ou plutôt si complètement funeste d’employer la contrainte et la violence, que le clergé d’une église établie, environné de la considération publique. Les droits, les privilèges, la liberté personnelle de tout individu ecclésiastique qui est bien avec son ordre, sont plus respectés, dans les gouvernements même les plus despotiques, que ceux de toute autre personne à peu près égale en rang et en fortune. Cela est ainsi dans tous les différents degrés du des­po­tisme, depuis le gouvernement doux et modéré de Paris, jusqu’au gouvernement violent et terrible de Constantinople. Mais si cette classe d’hommes ne peut être me­née par force, on peut se la concilier tout aussi aisément qu’une autre ; la sûreté du souverain, non moins que la tranquillité publique, semblent dépendre, en très-grande partie, des moyens qu’a le souverain de s’attacher ces hommes-là, et ces moyens semblent consister en entier dans les bénéfices qu’il a à répandre parmi eux.
Dans l’ancienne constitution de l’Église catholique romaine, l’évêque de chaque diocèse était élu par les suffrages réunis du clergé et du peuple de la ville épiscopale. Le peuple ne conserva pas longtemps son droit d’élection et pendant tout le temps qu’il le conserva, il agit presque toujours sous l’influence du clergé, qui dans ces matières spirituelles, semblait être son guide naturel. En outre, le clergé se lassa bientôt de prendre la peine de se concilier le peuple, et trouva plus commode d’élire lui-même ses évêques. L’abbé fut élu de même par les religieux du monastère, au moins dans la plus grande partie des abbayes. Tous les bénéfices ecclésiastiques inférieurs compris dans le diocèse étaient à la collation de l’évêque, qui les conférait à ceux des ecclésiastiques qu’il jugeait à propos d’en investir. De cette manière tous les bénéfices ecclésiastiques furent à la disposition du clergé. Quoique le souverain pût avoir quelque influence indirecte sur les élections, et qu’il fût quelquefois d’usage de lui demander son consentement pour élire, ainsi que son approbation de l’élection, cependant il n’avait aucun moyen direct et suffisant de se concilier le clergé. Chaque homme d’église était naturellement bien moins porté, par son ambition, à faire sa cour à son souverain qu’à son propre ordre, duquel seul il pouvait espérer quelque avancement.
Dans la plus grande partie de l’Europe, le pape attira insensiblement à lui, d’abord la collation de presque tous les évêchés et abbayes, ou de ce qu’on appelait bénéfices consistoriaux, et ensuite, sous divers prétextes et par diverses manœuvres, il s’attribua celle de la plus grande partie des bénéfices inférieurs compris dans chaque diocèse, n’en laissant à l’évêque guère plus que ce qui était purement nécessaire pour lui don­ner une autorité décente sur son clergé particulier. Par cet arrangement, la condition du souverain fut encore pire qu’elle n’avait été auparavant. Le clergé de tous les différents pays de l’Europe vint ainsi à se former en une espèce d’armée spirituelle, dispersée à la vérité dans différents quartiers, mais dont tous les mouvements et toutes les opérations purent alors être conduits par une seule tête et dirigés sur un plan uniforme. Le clergé de chaque pays particulier pouvait être regardé comme un déta­chement de cette année, duquel les opérations étaient au besoin soutenues et secon­dées par tous les autres détachements cantonnés dans les pays environnants. Non-seulement chacun de ces détachements fut indépendant du souverain du pays dans lequel il était cantonné et qui le faisait subsister, mais il était sous la dépendance d’un souverain étranger qui pouvait un jour tourner les armes de ce détachement contre le souverain de ce même pays, et soutenir celui-là avec les armes de tous ses autres détachements.
Ces armes étaient les plus formidables qu’on puisse imaginer. Dans l’ancien état de l’Europe, avant l’établissement des arts et des manufactures, les richesses du clergé lui donnaient sur la masse du peuple la même espèce d’influence que celle qu’avaient les grands barons sur leurs vassaux, tenanciers et gens de leur suite. Dans les grands domaines dont la piété trompée, tant des princes que des particuliers, avait gratifié l’Église, il y avait des juridictions établies de la même nature que celles des grands barons, et par la même cause. Dans ces grands domaines, le clergé ou ses baillis pouvaient aisément maintenir la paix sans le soutien ou l’assistance du roi ni d’aucune autre personne, et ni le roi ni aucune autre personne n’eussent pu y maintenir la paix sans le soutien et l’assistance du clergé. Ainsi, les juridictions du clergé dans ses baronies ou manoirs particuliers étaient tout aussi indépendantes et tout aussi exclu­sives de l’autorité des cours du roi, que les juridictions des grands seigneurs tempo­rels. Les tenanciers du clergé étaient, comme ceux des grands barons, presque tous amovibles à volonté, entièrement dépendants de leurs seigneurs immédiats et, par conséquent, dans le cas d’être appelés à tout moment pour porter les armes dans toutes les querelles dans lesquelles le clergé jugeait à propos de les engager. En outre des revenus de ces domaines, le clergé possédait encore dans les dîmes une très-forte portion des revenus de tous les autres domaines, dans chaque royaume de l’Europe. Les revenus provenant de ces deux sources différentes se payaient, pour la plus grande partie, en nature : en grains, vin, bestiaux, volailles, etc. ; la quantité excédait considérablement ce que le clergé en pouvait consommer lui-même, et il n’y avait ni arts ni manufactures contre le produit desquels il pût échanger ce superflu. Le clergé ne pouvait tirer parti de cette énorme surabondance autrement qu’en l’employant com­me les grands barons employaient le même superflu de leurs revenus, à entretenir l’hospitalité la plus libérale, à faire des charités sans bornes. Aussi dit-on que l’hospi­talité et la charité de l’ancien clergé étaient immenses. Non- seulement il faisait subsister presque tous les pauvres dans chaque royaume, mais encore il y avait une quantité de chevaliers et de gentilshommes qui n’avaient pas d’autres moyens de vivre que d’aller voyageant de monastère en monastère sous prétexte de dévotion, mais dans la réalité pour profiter de l’hospitalité du clergé. Les gens de la suite de certains prélats étaient souvent aussi nombreux que ceux des plus grands seigneurs laïques ; et les gens à la suite du clergé, pris ensemble, étaient peut-être plus nombreux que ceux de tous les seigneurs laïques. Il régnait toujours beaucoup plus d’union entre les seigneurs ecclésiastiques qu’entre les autres ; les premiers étaient soumis à une discipline réglée et subordonnée à l’autorité du pape, les autres n’étaient soumis à aucu­­ne discipline ou subordination réglée ; au contraire, ils étaient presque tous égale­ment jaloux les uns des autres et du roi. Ainsi, quand même les tenanciers et gens de la suite du clergé eussent été tous ensemble moins nombreux que ceux des grands seigneurs laïques (et probablement les tenanciers de ceux-ci l’étaient beaucoup moins), cependant l’union qui régnait dans cet ordre l’aurait toujours rendu plus redoutable que l’autre. Et puis, l’hospitalité et la charité exercées par le clergé don­naient non-seulement une grande force temporelle à son commandement, mais aug­men­taient encore extrêmement le poids de ses armes spirituelles. Ces vertus lui assuraient les respects et la vénération la plus profonde dans toutes les classes inférieures du peuple, dont un grand nombre d’individus étaient constamment nourris par lui, et presque tous, au moins de temps en temps. Tout ce qui appartenait, tout ce qui avait quelque rapport avec un ordre aussi populaire, ses possessions, ses privilè­ges, sa doctrine, tout paraissait nécessairement sacré aux yeux du vulgaire, et toute violation réelle ou supposée de quelqu’une de ces choses était le comble de la profanation et du sacrilège. Si dans ces temps donc le souverain trouvait souvent de la difficulté à résister à une confédération de quelques grands seigneurs, il ne faut pas s’étonner qu’il en dût trouver encore bien davantage à résister à la force réunie du clergé de ses propres États, soutenue par celle du clergé de tous les États voisins. Dans de telles circonstances, ce qui doit étonner, ce n’est pas qu’il ait été quelquefois obligé de plier, mais c’est qu’il ait jamais pu se croire en état de se soutenir.
Les privilèges du clergé de ces anciens temps, qui nous semblent les plus absur­des, à nous qui vivons dans le temps actuel, par exemple son exemption totale de la juridiction séculière, ou ce qu’on appelle en Angleterre le bénéfice de clergie, étaient une suite naturelle ou plutôt nécessaire de cet état de choses. Combien n’eût-il pas été dangereux pour le souverain de vouloir punir un homme d’église pour un crime quelconque, si l’ordre dont celui-ci était membre avait été disposé à le protéger, et à représenter ou les preuves comme trop faibles pour convaincre un aussi saint person­nage, ou le châtiment comme trop sévère pour être infligé à celui dont la religion avait rendu la personne sacrée ! Dans de pareilles circonstances, le souverain n’avait rien de mieux à faire que de le laisser juger par les tribunaux ecclésiastiques, qui, pour l’honneur même de leur ordre, étaient intéressés à prévenir, autant que possible, parmi leurs membres, les crimes d’éclat, ou même ces actions scandaleuses faites pour aliéner l’esprit du peuple.
Dans l’état des choses qui eut lieu presque par toute l’Europe pendant le cours des dixième, onzième, douzième et treizième siècles, et quelque temps encore tant avant qu’après cette période, la constitution de l’Église de Rome peut être regardée comme la combinaison la plus formidable qui ait été formée contre l’autorité et la sûreté du gouvernement civil, aussi bien que contre la liberté, la raison et le bonheur du genre humain, qui ne peuvent jamais régner et prospérer que sous la protection du gouver­nement civil. Dans cette constitution, les impostures et les illusions les plus grossières de la superstition se trouvèrent si fortement liées aux intérêts privés d’une immense multitude de gens, qu’elles étaient hors de toute atteinte des traits de la raison humaine ; car, encore bien que la raison eût peut-être pu venir à bout de dévoiler, même aux yeux du commun du peuple, quelques-unes de ces erreurs superstitieuses, elle n’aurait néanmoins jamais pu détacher entièrement les liens de l’intérêt privé. Si cette constitution n’eût eu d’autres attaques à essuyer que les faibles efforts de la raison, elle aurait sans doute duré à jamais. Mais cet édifice immense et si habilement construit, que toute la sagesse et toute la vertu humaine n’eussent jamais pu ébranler, encore moins renverser, s’est vu par le cours naturel des choses, d’abord affaibli, ensuite en partie démoli, et peut-être ne lui faut-il plus aujourd’hui que quelques siè­cles encore pour qu’il s’écroule tout à fait.
Les progrès successifs des arts, des manufactures et du commerce, les mêmes causes qui détruisirent la puissance des seigneurs, ont détruit de la même manière, dans la majeure partie de l’Europe, toute la puissance temporelle du clergé. Le produit des arts, des manufactures et du commerce offrit au clergé, tout comme aux sei­gneurs, quelque chose à échanger contre le superflu du produit brut de ses terres, et lui fit voir ainsi tous les moyens de dépenser la totalité de ses revenus en jouissances personnelles, sans être obligé d’en faire une aussi grande part aux autres. Peu à peu, sa charité devint moins étendue, son hospitalité moins généreuse et moins prodigue. Sa suite devint, par conséquent, moins nombreuse, et par degrés elle finit par se réduire tout à fait à rien. Comme les seigneurs, le clergé désira aussi retirer de plus fortes rentes de ces domaines, afin de les dépenser de la même manière, en jouis­sances personnelles, en sottises et en faste puéril. Or, cette augmentation de rente ne put s’obtenir qu’en accordant aux tenanciers de plus longs baux, ce qui rendit ceux-ci en grande partie indépendants. Ce fut ainsi que se relâchèrent et tombèrent enfin peu à peu ces liens d’intérêt qui attachaient au clergé les classes inférieures du peuple. Ils se relâchèrent et tombèrent même plus tôt encore que ceux qui attachaient les mêmes classes du peuple aux seigneurs, parce que les bénéfices de l’Église étant, pour la plus grande partie, de bien moindres domaines que les terres des seigneurs, le possesseur de chaque bénéfice fut bien plus tôt mis à même de dépenser tout son revenu au profit de sa personne. La puissance des seigneurs était encore en pleine vigueur dans la plus grande partie de l’Europe, pendant la majeure partie des quatorzième et quinzième siècles ; mais le pouvoir temporel du clergé, cet empire absolu qu’il avait eu autrefois sur la masse du peuple, était dès lors extrêmement déchu. La puissance de l’Église, à cette époque, était à peu près réduite, presque par toute l’Europe, à celle que pouvait lui donner son autorité spirituelle, et encore cette autorité spirituelle fut-elle fort affai­blie quand elle eut cessé d’être soutenue par la charité et par l’hospitalité du clergé. Les classes inférieures du peuple cessèrent de voir dans cet ordre, comme elles avaient fait auparavant, leur asile dans la disgrâce, leur soutien dans l’indigence. Au contraire, elles ne virent qu’avec éloignement et indignation la vanité, le luxe et les folles dépenses du riche clergé, qui prodiguait ouvertement à ses plaisirs ce qui avait toujours été considéré jusque-là comme le patrimoine des pauvres.
Dans ce nouvel état de choses, les souverains de différents États de l’Europe tâchèrent de recouvrer l’influence qu’ils avaient eue autrefois dans la disposition des grands bénéfices de l’Église, en s’occupant de faire rendre aux doyen et chapitre de chaque diocèse l’ancien droit d’élire leur évêque, et aux moines de chaque abbaye celui d’élire leur abbé. Le rétablissement de cet ordre ancien fut l’objet de plusieurs statuts portés en Angleterre pendant le cours du quatorzième siècle, particulièrement de celui qui fut appelé le statut des proviseurs[36], et de la pragmatique sanction établie en France dans le quinzième siècle. Il devint nécessaire, pour la validité des élections, que le souverain y eût préalablement donné son consentement, et en même temps qu’il agréât ensuite la personne élue ; et quoique l’élection fût toujours censée libre, il eut néanmoins tous les moyens indirects que lui fournissait nécessairement sa position, pour prendre de l’influence sur le clergé de ses États. D’autres règlements tendant au même but furent établis dans d’autres endroits de l’Europe ; mais nulle part avant la réformation, à ce qu’il semble, le pouvoir du pape sur la collation des grands bénéfices de l’Église ne fut aussi efficacement et aussi universellement restreint qu’en France et en Angleterre. Vint ensuite, dans le seizième siècle, le concordat, qui donna aux rois de France le droit absolu de présentation à tous les grands bénéfices et bénéfices consistoriaux de l’Église gallicane.
Depuis l’établissement de la pragmatique sanction et du concordat, le clergé de France a, en général, montré moins de respect pour les décrets de la cour papale, que le clergé de tout autre pays catholique. Dans toutes les querelles que son souverain a eues avec le pape, ce clergé a presque toujours pris le parti du premier. L’indépen­dance où est le clergé de France de la cour de Rome paraît être principalement fondée sur la pragmatique sanction et le concordat. Dans les temps plus reculés de la monarchie, on trouve le clergé de France tout aussi dévoué au pape que le clergé de tout autre pays. Quand Robert, le second roi de la troisième race, fut frappé par la cour de Rome de la plus injuste des excommunications, ses propres domestiques, dit-on, jetaient aux chiens les mets qui sortaient de sa table, et se gardaient bien de toucher à rien de ce qui avait été souillé par le contact d’une personne frappée d’un tel anathème. On peut bien présumer, sans crainte de se tromper, que c’était le clergé du royaume qui leur prescrivait cette conduite.
Ainsi, le droit de collation aux grands bénéfices de l’Église, ce droit pour le soutien duquel la cour de Rome avait souvent ébranlé et quelquefois renversé les trônes de quelques-uns des plus grands souverains du monde chrétien, se trouva restreint, modifié ou même tout à fait anéanti dans plusieurs endroits de l’Europe, même avant l’époque de la réformation. Comme le clergé eut alors moins d’influence sur le peuple, l’État eut plus d’influence sur le clergé. Ainsi, le clergé eut à la fois et moins de pouvoir pour troubler l’État, et moins de penchant à le faire. Tel était l’état de décadence où était tombée l’autorité de l’Église de Rome, quand les disputes qui donnèrent naissance à la réformation éclatèrent en Allemagne et se répandirent bientôt par toute l’Europe.
La doctrine nouvelle obtint partout une grande faveur populaire ; elle était propa­gée avec tout l’enthousiasme du zèle qui anime communément l’esprit de parti quand il attaque une autorité reconnue. Les maîtres de cette doctrine, quoique peut-être à d’autres égards aussi peu instruits que la plupart des théologiens qui défendaient les dogmes reçus, semblent, en général, avoir été mieux au fait de l’histoire ecclésias­tique, ainsi que de l’origine et des progrès de ce système d’opinions sur lequel était fondée l’autorité de l’Église, et ils avaient par là de l’avantage dans toutes les disputes. L’austérité de leurs mœurs leur donnait du crédit sur le vulgaire, qui mettait en opposition la stricte régularité de leur conduite avec la vie déréglée de la plupart des membres de son clergé. Ils possédaient aussi, à un bien plus haut degré que leurs adversaires, tous les arts de la popularité et celui de se faire des prosélytes ; arts que les puissants et magnifiques enfants de l’Église avaient depuis longtemps négligés comme à peu près inutiles. Quelques-uns embrassèrent la nouvelle doctrine par rai­son ; beaucoup par amour pour la nouveauté ; un bien plus grand nombre encore par haine et par mépris pour le clergé dominant. Mais ce qui attira vers elle une foule sans comparaison plus nombreuse, ce fut cette éloquence ardente, passionnée et fanatique, quoique souvent rustique et grossière, avec laquelle elle fut presque partout prêchée[37]. 
Le succès de cette nouvelle doctrine fut si grand et si général, qu’elle fournit aux princes qui se trouvaient alors être mal avec la cour de Rome le moyen de détruire aisément dans leurs États l’église dominante ; et celle-ci, qui avait perdu le respect et la vénération des classes inférieures du peuple, ne pouvait guère opposer de résis­tance. La cour de Rome avait désobligé quelques-uns des petits princes du nord de l’Allemagne, qu’elle avait probablement regardés comme trop peu importants pour valoir la peine d’être ménagés. En conséquence, ceux-ci établirent généralement la religion réformée dans leurs États. La tyrannie de Christiern II et de Troll, archevêque d’Upsal, mit Gustave Vasa à même de les chasser l’un et l’autre de Suède. Le pape prit le parti du tyran et de l’archevêque, et Gustave Vasa ne trouva aucune difficulté à établir la réforme en Suède. Christiern II fut ensuite déposé du trône de Danemark, où sa conduite l’avait rendu aussi odieux qu’en Suède. Le pape cependant se montra encore disposé à le favoriser, et Frédéric de Holstein, qui était monté sur le trône à sa place, se vengea du pape en suivant l’exemple de Gustave. Les magistrats de Berne et de Zurich, qui n’avaient pas de querelle particulière avec le pape, établirent avec grande facilité la réformation dans leurs cantons respectifs, où, par une imposture un peu plus grossière encore que leurs tromperies ordinaires, quelques gens du clergé venaient tout nouvellement de rendre leur ordre entier odieux et méprisable[38].
Dans une situation aussi critique, la cour papale avait bien assez à faire de cultiver l’amitié des puissants monarques de France et d’Espagne, dont le dernier était à cette époque empereur d’Allemagne. Avec leur assistance elle put venir à bout, quoique non sans de grandes difficultés et beaucoup de sang répandu, ou d’empêcher totale­ment la réformation dans leurs États, ou d’en arrêter un moment les progrès. Elle était assez disposée aussi à traiter le roi d’Angleterre avec une grande complaisance ; mais les circonstances voulurent qu’elle n’eût pu agir ainsi sans offenser un monarque encore plus puissant, Charles V, roi d’Espagne et empereur d’Allemagne. Aussi, si Henri VIII lui-même ne reconnut pas les principaux articles de la doctrine de la réformation, au moins la faveur générale que cette doctrine avait acquise le mit-elle à même de supprimer tous les monastères dans ses États, et d’y abolir l’autorité de l’Église romaine. Quoiqu’il n’ait pas été plus loin, c’en était assez pour faire plaisir aux champions de la réformation qui, s’étant rendus maîtres du gouvernement sous son fils et successeur, achevèrent sans la moindre difficulté l’ouvrage commencé par le père.
Dans quelques pays, comme l’Écosse, où le gouvernement était anti-populaire et très-peu solidement établi, la réformation fut assez forte, non-seulement pour ren­verser l’Église, mais encore pour renverser l’État, qui voulut essayer de soutenir l’Église.
Entre les sectateurs de la réformation répandus dans tous les différents pays de l’Europe, il n’y avait pas de tribunal général qui pût, comme celui de la cour de Rome ou comme un concile œcuménique, régler entre eux tous les sujets de controverse, et prescrire à tous, avec une irréfragable autorité, les limites précises de l’orthodoxie. Quand donc ceux de la religion réformée dans un pays venaient à différer l’opinion avec leurs frères d’un autre pays, comme il n’y avait pas de juge commun auquel ils pussent appeler, la dispute ne pouvait jamais être décidée, et il s’éleva beaucoup de ces sortes de disputes parmi eux ; celles relatives au gouvernement de l’église et au droit de conférer les bénéfices ecclésiastiques étaient peut-être celles qui intéressaient le plus la paix et le bien-être de la société civile ; elles donnèrent, en conséquence, lieu aux deux parties ou sectes principales qui divisent les disciples de la réformation, les sectes calviniste et luthérienne, les seules parmi eux dont la doctrine et la discipline aient encore jamais été légalement établies en Europe.
Les partisans de Luther, ainsi que ce qu’on appelle l’Église anglicane, conservè­rent plus ou moins le gouvernement épiscopal, maintinrent une subordination dans le clergé, donnèrent au souverain la disposition de tous les évêchés et autres bénéfices consistoriaux dans ses États, et le rendirent par là le véritable chef de l’église ; et sans ôter à l’évêque le droit de collation aux bénéfices inférieurs dans son diocèse, non-seulement ils admirent quant à ces bénéfices mêmes, mais encore ils favorisèrent le droit de présentation, tant chez le souverain que chez les autres patrons laïques. Ce système de gouvernement ecclésiastique fut dès le commencement favorable à la paix et au bon ordre, ainsi qu’à la soumission envers l’autorité civile. Ainsi n’a-t-il jamais été l’occasion d’aucun trouble ou commotion civile dans aucun des pays où il a été une fois établi. L’Église d’Angleterre, en particulier, s’est toujours glorifiée avec rai­son de la loyauté irréprochable de ses principes[39]. Sous un pareil régime, ceux qui composent le clergé cherchent naturellement à gagner l’estime du souverain, de la cour, de la noblesse et des personnes distinguées du pays, par l’influence desquelles ils espèrent principalement obtenir de l’avancement. Ils font la cour à ces patrons, quelquefois sans doute par de basses flatteries et de viles complaisances, mais bien souvent aussi par la culture de ces arts qui attirent le plus l’attention des gens riches et distingués et sont, par conséquent, la voie la plus sûre d’acquérir leur estime, par des connaissances dans toutes les diverses branches utiles et agréables des sciences, par la noblesse et la décence de leurs manières, par la sociabilité de leur humeur et le bon ton de leur conversation ; enfin, par le mépris dont ils font profession pour ces austé­rités absurdes et hypocrites que les fanatiques prêchent et se piquent de pratiquer afin d’attirer sur eux la vénération du petit peuple, et de lui rendre odieux la plupart de ceux des classes supérieures qui se dispensent ouvertement de pareilles momeries. Cependant un tel clergé, en se rendant aussi agréable aux personnes du premier ordre de la société, est très-disposé à négliger totalement les moyens de conserver de l’influence et du crédit sur les dernières classes ; il sera écouté, estimé et respecté de ses supérieurs, mais devant ses inférieurs il sera souvent hors d’état de défendre avec succès et d’une manière convaincante pour un tel auditoire ses principes sages et modérés, contre le plus ignorant des enthousiastes qui jugera à propos de les attaquer.
Les partisans de Zwingle, ou pour mieux dire ceux de Calvin, donnèrent, au contraire, au peuple de chaque paroisse, dans tous les cas de vacance, le droit d’élire son propre pasteur, et établirent en même temps la plus parfaite égalité dans le clergé. Tant que la première partie de cette institution resta en vigueur, il paraît qu’elle n’a produit autre chose que de la confusion et des désordres, et qu’elle tendit à corrom­pre également les mœurs du clergé et celles du peuple. L’autre partie paraît n’avoir jamais eu que des effets parfaitement conformes au but de l’institution.
Tant que le peuple de chaque paroisse conserva ce droit d’élection, il ne fit pres­que toujours que suivre l’influence du clergé et, en général, celle des plus fanatiques et des plus turbulents de cet ordre. Les ecclésiastiques, pour conserver leur influence dans ces élections populaires, devinrent pour la plupart et affectèrent de se montrer fanatiques, encouragèrent le fanatisme dans le peuple et donnèrent presque toujours la préférence aux plus fanatiques d’entre les candidats. La moindre affaire, la nomina­tion d’un simple prêtre de paroisse, suffit pour occasionner le plus souvent des contestations violentes, non-seulement dans la paroisse, mais encore dans toutes les paroisses voisines, qui manquaient rarement de prendre parti dans la querelle. S’il arrivait que la paroisse fût située dans une grande ville, un tel événement divisait les habitants en deux partis ; et quand il se trouvait que cette ville formait elle-même une petite république, ou bien qu’elle était le chef-lieu ou la capitale d’une petite répu­blique, ce qui est le cas de la plupart des villes considérables de la Suisse et de la Hollande, chaque misérable dispute de ce genre, en excitant l’animosité de toutes les autres factions, menaçait encore de laisser après elle à la fois et un nouveau schisme dans l’Église, et une nouvelle faction dans l’État. En conséquence, dans ces petites républiques, le magistrat sentit de bonne heure la nécessité, pour maintenir la tran­quillité publique, de se saisir lui-même du droit de présenter à tous les bénéfices vacants. En Écosse, le pays le plus étendu dans lequel ait jamais été établie cette forme presbytérienne dans le gouvernement de l’Église, les droits de patronage furent, dans le fait, abolis par l’acte qui établit les presbytéries[40], au commencement du règne de Guillaume III. Cet acte, du moins, investit certaines classes du peuple de chaque paroisse du pouvoir d’acheter, pour une très-petite somme, le droit d’élire leur propre pasteur. On laissa subsister environ vingt-deux ans le régime établi par cet acte : mais ce régime fut aboli par le statut de la dixième année de la reine Anne, chapitre XII, à cause des troubles et des désordres qu’avait causés presque partout ce mode populaire d’élection. Cependant, dans un pays aussi étendu que l’Écosse, un tumulte dans une paroisse éloignée n’était pas autant dans le cas de troubler la tranquillité du gouver­nement qu’il l’eût été dans un plus petit État. L’acte de la dixième année de la reine Anne rétablit le droit de patronage. Mais quoiqu’en Écosse la loi donne le bénéfice, sans exiger aucune autre condition, à la personne présentée par le patron, cependant l’Église exige quelquefois (car à cet égard elle n’a pas été très-uniforme dans ses décisions) un certain concours ou agrément de la part du peuple, avant de conférer à la personne présentée ce qu’on appelle la charge des âmes ou la juridiction ecclésias­ti­que sur la paroisse. Au moins quelquefois, sous le prétexte affecté de conserver la paix dans la paroisse, elle diffère de mettre le pasteur en possession jusqu’à ce qu’on ait pu avoir ce concours de la part du peuple. Les menées particulières du clergé du voisinage, quelquefois pour obtenir cet agrément populaire, mais plus souvent encore pour l’empêcher, et les moyens de popularité qu’il se ménage pour se mettre à même d’intriguer avec plus de succès dans de pareilles occasions, sont peut-être la cause principale de cet ancien levain de fanatisme qui se fait sentir encore en Écosse dans le clergé et parmi le peuple.
L’égalité que la forme presbytérienne du gouvernement ecclésiastique établit dans le clergé consiste d’abord dans l’égalité d’autorité ou de juridiction ecclésiastique, et secondement dans l’égalité de bénéfices. Dans toutes les églises presbytériennes, l’égalité d’autorité est parfaite ; il n’en est pas de même de celle des bénéfices. En outre, la différence entre un bénéfice et un autre est rarement assez considérable pour que le possesseur même du petit bénéfice puisse être tenté de faire bassement la cour aux patrons afin d’en obtenir un meilleur. C’est ordinairement par des moyens plus honnêtes et plus relevés que, dans toutes les églises presbytériennes où les droits de patronage sont généralement établis, que le clergé cherche à se concilier la faveur de ses supérieurs ; c’est par ses connaissances et son savoir, par une conduite irrépro­chable, par la fidélité et l’exactitude avec laquelle il remplit ses devoirs. Les patrons mêmes se plaignent souvent de l’indépendance de caractère chez les ecclésiastiques, à laquelle ils donnent volontiers le nom d’ingratitude et d’oubli des bienfaits passes, mais qui, a en juger le moins favorablement, est au plus une indifférence tout natu­rel­lement produite par la certitude de n’avoir plus aucun bienfait de ce genre à attendre à l’avenir. En nul endroit de l’Europe peut-être, on ne saurait trouver une classe d’hommes plus instruits, plus décents, plus indépendants et plus respectables que la plupart des ecclésiastiques presbytériens de Hollande, de Genève, de la Suisse et de l’Écosse.
Quand les bénéfices de l’église sont à peu près tous égaux, aucun d’eux ne peut être fort considérable, et cette médiocrité dans les bénéfices, quoiqu’il ne faille pas la porter trop loin, a toutefois des effets très-favorables. Il n’y a que les mœurs les plus exemplaires qui puissent donner de la dignité à un homme d’une très-modique fortune. Les vices qu’entraînent la frivolité et la vanité le rendraient nécessairement ridicule et, d’ailleurs, seraient presque aussi ruineux pour lui que pour les gens du peuple. Ainsi, dans sa conduite privée, il est obligé de suivre ce système de morale que le peuple respecte le plus. Il gagne l’estime et l’affection des gens de cette classe par le genre de vie même que son intérêt seul et sa position le porteraient à adopter. Il est regardé par eux avec ce sentiment de bienveillance que nous portons naturelle­ment à quelqu’un qui se rapproche un peu de notre propre condition, mais qui nous semble fait pour une plus relevée. Naturellement aussi, leur bienveillance excite la sienne ; il met plus de soin à les instruire, plus d’attention à les seconder, plus de zèle à les soulager ; il ne méprise même pas les préjugés de gens qui sont disposés à lui être si favorables, et il ne prend jamais avec eux ces airs dédaigneux et arrogants que nous trouvons si souvent dans l’orgueilleux dignitaire d’une église opulente et richement dotée. Aussi le clergé presbytérien a-t-il plus d’influence sur l’esprit du peuple que n’en a peut-être le clergé de toute autre église établie ; et ce n’est, en consé­quence, que dans les seuls pays presbytériens que nous verrons jamais le peuple com­plètement converti à la croyance de l’église établie, sans qu’aucun moyen de persécution ait été employé.
Dans les pays où les bénéfices de l’église sont pour la plus grande partie très-modiques, une chaire dans une université est, en général, une meilleure place qu’un bénéfice ecclésiastique. Dans ce cas, les universités peuvent prendre avec choix tous leurs membres dans la totalité des gens d’église du pays, qui constituent partout la classe, sans comparaison, la plus nombreuse de gens de lettres. Dans ceux, au con­traire, où les bénéfices de l’Église sont en grande partie d’un revenu très-considérable, natu­rellement l’Église enlève aux universités la plupart de leurs gens de lettres distin­gués, qui trouvent toujours quelque patron jaloux de leur procurer un bon bénéfice. Dans le premier de ces deux cas, il y a à parier que le clergé n’offrira qu’un petit nombre de gens de mérite, et ceux-ci encore parmi les membres les plus jeunes de cet ordre, qui vraisemblablement en seront aussi tirés avant d’avoir pu acquérir assez de connaissances et d’expérience pour lui être d’une grande utilité. M. de Voltaire observe que le père Porée, jésuite (médiocrement distingué dans la république des lettres), était le seul professeur qu’on eût vu en France, dont les ouvrages valussent la peine d’être lus. Dans un pays qui a été aussi fécond en gens de lettres du premier talent, il peut paraître assez extraordinaire qu’il y ait eu à peine un d’entre eux profes­seur dans une université. Le célèbre Gassendi était, dans les premières années de sa vie, professeur à l’université d’Aix. Aux premières étincelles de génie qu’il fit paraître, on lui représenta qu’en se mettant dans l’église il pourrait trouver facilement les moyens de vivre avec plus d’aisance et de repos, et qu’il serait ainsi dans une position plus favorable pour continuer ses études ; et il suivit aussitôt ce conseil. La remarque de M. de Voltaire peut s’appliquer, à ce que je crois, non-seulement à la France, mais à tous les pays catholiques romains. Il est très-rare que nous trouvions, chez aucun, un homme de lettres distingué qui soit professeur d’une université, excepté peut-être dans les chaires de droit et de médecine, professions dans lesquelles l’église n’est pas aussi à même de puiser. Après l’Église de Rome, celle d’Angleterre est, sans compa­raison, la plus opulente et la mieux rentée de toutes les églises chrétiennes. Aussi, en Angleterre, l’église est occupée continuellement à épuiser les universités de leurs membres les plus studieux et les plus habiles, et il serait aussi rare que dans les pays catholiques romains d’y trouver un ancien professeur de collège, connu et cité en Europe comme un homme de lettres du premier ordre. À Genève, au contraire, dans les cantons suisses protestants, dans, les pays protestants de l’Allemagne, en Hollan­de, en Écosse, en Suède et en Danemark, les gens de lettres les plus distingués que ces pays aient produits avaient été, non pas tous, à la vérité, mais sans comparaison la plus grande partie, professeurs dans les universités. Dans ces pays, ce sont les univer­sités, au contraire, qui épuisent continuellement l’ église glise de tous les gens de lettres supérieurs qui peuvent s’y trouver.
C’est peut-être une chose qui mérite d’être observée, que si nous en exceptons les poètes, un petit nombre d’orateurs et quelques historiens, la très-majeure partie des autres gens de lettres d’un ordre supérieur, tant à Rome que dans la Grèce, paraissent avoir été des professeurs publics ou particuliers, et généralement des professeurs de philosophie ou de rhétorique. On trouvera cette observation constamment vraie depuis le temps de Lysias et d’Isocrate, de Platon et d’Aristote, jusqu’à ceux de Plutar­que et d’Épictète, de Suétone et de Quintilien[41]. Il semble, en effet, que la méthode la plus efficace pour rendre un homme parfaitement maître d’une science particulière, c’est de lui imposer la nécessité d’enseigner cette science régulièrement chaque année. Étant obligé de parcourir tous les ans la même carrière, pour peu qu’il soit bon à quelque chose, il se met nécessairement en peu d’années complètement au fait de chaque partie de sa matière ; et s’il lui arrivait, dans une année, de se former sur quelque point particulier une opinion trop hâtive, quand il vient l’année suivante à repasser sur le même objet dans le cours de ses leçons, il y a à parier qu’il réformera ses idées. Si l’emploi d’enseigner une science est certainement l’emploi naturel de celui qui est purement homme de lettres, c’est aussi peut-être le genre d’éducation le plus propre à en faire un homme vraiment profond en savoir et en connaissances. La médiocrité des bénéfices ecclésiastiques tend naturellement à attirer la plupart des gens de lettres du pays où cette circonstance se rencontre, vers le genre d’emploi dans lequel ils peuvent être le plus utiles au public, et en même temps à leur donner la meilleure éducation peut-être qu’ils soient capables de recevoir ; elle tend à rendre leur savoir aussi solide et aussi profond que possible, et de plus à lui donner la direction la plus utile qu’il puisse prendre. Il est à observer que le revenu de l’Église établie (à l’exception seulement des parties de ce revenu qui peuvent provenir de terres ou de domaines particuliers) est une branche du revenu général de la société, qui se trouve ainsi détourné pour un objet fort étranger à la dépense de l’État.
La dîme, par exemple, est un véritable impôt territorial qui ôte aux propriétaires des terres la faculté de pouvoir contribuer aussi largement qu’ils pourraient le faire sans cela à la défense publique. Or, la rente de la terre est, suivant quelques person­nes, la source unique, et suivant d’autres, la source principale qui fournit en dernier résultat de quoi pourvoir aux besoins de l’État dans toutes les grandes monarchies. Plus il va de cette source à l’Église, moins sans contredit on en peut réserver pour l’État. On peut poser comme maxime certaine que, toutes choses supposées égales d’ailleurs, plus l’Église est riche, plus nécessairement alors ou le souverain ou le peuple sera pauvre et, dans les deux cas, l’État nécessairement moins capable de se défendre. Dans plusieurs pays protestants, et particulièrement dans tous les cantons suisses protestants, avec les revenus qui appartenaient anciennement à l’Église catholique romaine, les dîmes et les biens-fonds ecclésiastiques, on a pu former un fonds suffisant, non-seulement pour fournir des salaires convenables au clergé, mais pour défrayer encore, avec peu ou point d’addition, toutes les autres dépenses de l’État. Les magistrats du puissant canton de Berne, en particulier, ont accumulé sur les épargnes de ce fonds une très-forte somme qu’on croit monter à plusieurs millions, dont partie est déposée dans un trésor public, et partie placée à intérêt, dans ce qu’on appelle les fonds publics, chez différentes nations de l’Europe qui sont grevées d’une dette, principalement celles de France et d’Angleterre. je ne prétends pas savoir à quoi peut monter le total de ce que coûte à l’État l’église de Berne ou de tout autre canton protestant. Il paraît, d’après un compte très-exact, que la totalité du revenu de l’église d’Écosse, y compris la glèbe ou les biens-fonds ecclésiastiques, ainsi que la rente de leurs manses ou maisons d’habitation, portée à une évaluation raisonnable, se montait, en 1755, à une somme de 68,514 livres 1 sch. 5 deniers 1/12 seulement. Ce revenu très-modique fournit une subsistance décente à neuf cent quarante-quatre ministres. Toute la dépense de l’église, y compris ce qu’il fallut allouer accidentellement pour constructions et réparations des églises et des maisons de ministres, ne peut être censée aller fort au-delà de 80 ou 85,000 livres par an.
L’Église la plus opulente du monde chrétien ne maintient pas mieux l’uniformité de croyance, la ferveur de la dévotion, l’esprit d’ordre, la bonne conduite et la sévérité de mœurs dans la masse du peuple, que cette église d’Écosse si pauvrement dotée. Elle produit aussi pleinement qu’aucune autre que ce puisse être tous les bons effets civils et religieux qu’on peut attendre d’une église établie. La plupart des églises protestantes de Suisse, qui, en général, ne sont pas mieux dotées que l’église d’Écosse, produisent tous ces effets, et à un degré encore plus marqué. Dans la majeure partie des cantons protestants, on ne trouverait pas une seule personne qui ne fit profession d’être de l’église établie. Il est vrai que si quelqu’un professe une autre religion, la loi l’oblige à quitter le canton ; mais une loi aussi rigoureuse ou plutôt réellement aussi oppressive n’aurait jamais pu s’exécuter dans ces pays de liberté, si les soins du clergé n’eussent pas d’avance converti au culte établi toute la masse du peuple, à l’exception peut-être seulement d’un petit nombre d’individus. Aussi, dans quelques endroits de la Suisse, où, par l’union accidentelle d’un pays protestant et d’un pays catholique romain, la conversion n’a pas été complète, les deux religions sont non-seulement tolérées, mais elles sont toutes deux légalement établies. Pour qu’un service quelconque soit rempli d’une manière convenable, il faut, à ce qu’il semble, que son salaire ou sa récompense soit proportionné le plus exactement possible à la nature du service. Si un service est beaucoup trop peu payé, il y a fort à craindre qu’il ne souffre de l’incapacité et de la bassesse de la plupart de ceux qui y seront employés ; s’il est beaucoup trop payé, il y a à craindre peut-être qu’il ne souffre encore plus de leur insouciance et de leur paresse. Un homme qui jouit d’un gros revenu, de quelque profession qu’il puisse être, s’imagine devoir vivre comme les autres personnes qui ont un pareil revenu, et pouvoir donner une grande partie de son temps aux plaisirs, à la vanité et à la dissipation. Mais chez un ecclésiastique, un pareil train de vie non-seulement consume un temps qui devrait être consacré aux devoirs de sa place, mais encore détruit presque entièrement aux yeux des gens du peuple ce caractère de sainteté, qui peut seul le mettre en état de remplir ses devoirs avec le poids et l’autorité convenables.

SECTION QUATRIÈME.
Des dépenses nécessaires pour soutenir la dignité du souverain.

Outre les dépenses nécessaires pour mettre le souverain en état de remplir ses différents devoirs, il y a encore une certaine dépense qu’exige le soutien de sa dignité. Cette dépense varie, tant avec les différentes périodes d’avancement de la société, qu’avec les différentes formes du gouvernement.
Dans une société opulente et industrieuse, où toutes les différentes classes du peuple sont entraînées de jour en jour à faire plus de dépense dans leur logement, dans leur ameublement, dans leur table, dans leurs habits et dans leur train, on ne peut guère s’attendre à ce que le souverain résistera seul au torrent de la mode. Il en vient donc aussi naturellement ou plutôt nécessairement à faire plus de dépense dans chacun de ces différents articles, et sa dignité semble lui prescrire d’en user ainsi.
Comme sous le rapport de la dignité un monarque est plus élevé au-dessus de ses sujets que le premier magistrat d’une république quelconque ne peut jamais être censé l’être au-dessus de ses concitoyens, il faut aussi une plus grande dépense pour soutenir cette dignité plus élevée. Naturellement, nous nous attendons à trouver plus de splendeur dans la cour d’un roi que dans la maison d’un doge ou d’un bourgmestre.

CONCLUSION.

Les dépenses qu’exige la défense publique, et celle pour soutenir la dignité du premier magistrat, sont faites, les unes et les autres, pour l’avantage commun de toute la société. Il est donc juste que ces dépenses soient défrayées par une contribution générale de toute la société, à laquelle chaque différent membre contribue, le plus équitablement possible, dans la proportion de ses facultés.
La dépense qu’exige l’administration de la justice peut aussi sans doute être regardée comme faite pour l’avantage commun de toute la société. Il n’y aurait donc rien de déraisonnable quand cette dépense serait aussi défrayée par une contribution générale. Cependant les personnes qui donnent lieu à cette dépense sont celles qui, par des actions ou des prétentions injustes, rendent nécessaire le recours à la protection des tribunaux ; comme aussi les personnes qui profitent le plus immédiatement de cette dépense sont celles que le pouvoir judiciaire a rétablies ou maintenues dans leurs droits ou violés, ou attaqués. Ainsi, les dépenses d’adminis­tration de la justice pourraient très-convenablement être payées par une contribution particulière, soit de l’un ou de l’autre, soit de ces deux différentes classes de personnes à mesure que l’occasion l’exigerait, c’est-à-dire par des honoraires ou vacations payés aux cours de justice. Il ne peut y avoir nécessité de recourir à une contribution générale de toute la société, que pour la conviction de ces criminels qui n’ont person­nellement ni bien ni fonds quelconque sur lequel on puisse prendre ces vacations.
Ces dépenses locales ou provinciales dont l’avantage est borné à la même localité, telles, par exemple, que celles pour la police d’une ville ou d’un district, doivent être défrayées par un revenu local ou provincial, et ne doivent pas être une charge du revenu général de la société. Il n’est pas juste que toute la société contribue pour une dépense dont une partie seulement de la société recueille le fruit.
La dépense d’entretenir des routes sûres et commodes et de faciliter les communi­cations est sans doute profitable à toute la société et, par conséquent, on peut sans injusti­ce la faire payer par une contribution générale. Cependant, cette dépense pro­fite plus immédiatement et plus directement à ceux qui voyagent ou qui transpor­tent des marchandises d’un endroit dans un autre, et à ceux qui consomment ces mar­chandises. Les droits de barrières, sur les grands chemins en Angleterre, et ceux appelés péages dans d’autres pays, mettent cette dépense en totalité sur ces deux différentes sortes de personnes, et par là dégrèvent le revenu général de la société d’un fardeau considérable.
La dépense des institutions pour l’éducation publique et pour l’instruction reli­gieuse est pareillement sans doute une dépense qui profite à toute la société, et qui par conséquent peut bien, sans injustice, être défrayée par une contribution générale. Cependant, il serait peut-être aussi convenable, et même quelque peu plus avantageux qu’elle fût payée en entier par ceux qui profitent immédiatement de cette éducation et de cette instruction, ou par la contribution volontaire de ceux qui croient avoir besoin de l’une ou de l’autre.
Quand les établissements ou les travaux publics qui profitent à toute la société ne peuvent être entretenus en totalité, ou ne sont pas, dans le fait, entretenus en totalité par la contribution de ceux des membres particuliers de la société qui profitent le plus immédiatement de ces travaux, il faut que le déficit, dans la plupart des circonstances, soit comblé par la contribution générale de toute la société.
Le revenu général de la société, outre la charge de pourvoir aux dépenses de la défense publique et à celle que demande la dignité du premier magistrat, est donc encore chargé de remplir le déficit de plusieurs branches particulières de revenu.
Je vais tâcher d’exposer dans le chapitre suivant quelles sont les sources de ce revenu général ou du revenu de l’État.
 
 
 
↑ La grande question qui doit être ici prise en considération est évidemment de savoir, non point quels sont les frais de défense, mais quel système de défense sera le meilleur, quoi qu’il en puisse coûter. Mais Adam Smith, s’étant borné à des recherches qui ont trait aux richesses des nations, s’est, par le plan même de son ouvrage, dispensé de la discussion des questions qui intéressent vivement les hommes d’État. Il les traite, non point comme des questions politiques, ce qu’elles sont en réalité, mais comme des questions d’économie. Il montre d’abord comment elles se rattachent au bien-être général, avant d’en entreprendre la discussion. Il a ainsi, sans aucune nécessité, entravé le cours de ses recherches par une règle qu’il est obligé d’enfreindre avant d’aborder son sujet ; et on peut remarquer que, bien qu’il ne prétende traiter que le côté économique de la question, en tant qu’il fait partie du plan de son ouvrage, il envisage pourtant la question politique, recherchant avant tout, non point le système le moins coûteux, mais le meilleur. Ses vues sur cette matière ne sont pourtant pas complètes ; il se borne à constater un seul point, savoir, la supériorité d’une armée régulière sur toute autre espèce de force, et il attribue à l’oubli de cette maxime toutes les révolutions qui ont bouleversé les États. Buchanan.
↑ Ceux qui fout la guerre et qui, par conséquent, ne peuvent rien faire pour leur entretien, doivent être entretenus par l’industrie des autres ; et, dans un État purement militaire, le commerce et l’agriculture doivent être cultivés assez pour que ceux qui restent dans leurs foyers puissent subvenir aux besoins de ceux qui se battent pour eux. C’est alors, quand une petite portion de la population se charge des fournitures de guerre, qu’une nation peut se servir de sa population comme d’un instrument de guerre. C’est d’après ce module que se font formées les républiques belliqueuses de la Grèce et de Rome, et jamais, depuis, dans des États d’une égale étendue, un aussi grand déplacement de forces militaires ne s’est vu. Dans les temps modernes, à l’époque de l’invasion de leur pays en 1792, les Français, dans la défense de leur patrie, ont montré un zèle et un enthousiasme dignes du patriotisme des anciennes républiques, et ils finirent par devenir formidables aux États environnants. Mais, même dans cette guerre, les Français n’avaient jamais sous les armes une aussi grande portion de leur population que les ancieunes républiques de Rome et de la Grèce. Buchanan.
↑ Depuis les guerres de la révolution, l’Europe est devenue plus belliqueuse ; et on a calculé que maintenant 1 sur 70 de la population de chaque pays est destiné au service militaire. De si grands efforts ne ruinent pas précisément, mais ils appauvrissent les pays qui les font. Buchanan.
↑ Sous quel rapport l’art de la guerre peut-il être appelé le plus noble des arts ? La guerre, sans doute, développe toutes ces grandes qualités de l’àme qui étonnent et éblouissent les hommes ; mais elle n’ouvre pas la même carrière aux facultés de l’esprit. La théorie de la guerre est bientôt apprise, et sa pratique n’offre point de difficultés, l’esprit restant calme et pouvant exécuter facilement ce qu’il a saisi sans effort. Sous ce rapport donc, l’art de la guerre ne sera pas le plus noble des arts ; et quand on considère qu’il fait couler le sang par torrents et qu’il porte partout la misère et la destruction, on n’hésitera pas, en exceptant seulement le cas de défense, à le regarder comme atroce et barbare. La guerre, le mépris du danger, et la hardiesse seront toujours populaires ; mais qu’y a-t-il, sous ces dehors de générosité et de grands sentiments, de réellement admirable dans le caractère d’un soldat, qui ne fait que marcher aveuglément à la suite d’un chef victorieux, victime lui-même de son ambition, et sans égards pour les malheurs qui accompagnent ses triomphes ? En réfléchissant que la gloire du soldat naît des souffrances de l’humanité, il nous parait douteux qu’on puisse applaudir à un art qui ne s’exerce que par la destruction de la félicité humaine. Gibbon remarque avec justesse que, tant que les hommes exalteront plutôt ceux qui les écrasent que leurs véritables bienfaiteurs, la guerre sera toujours considérée comme le chemin de la gloire. Buchanan.
↑ Les opinions émises dans ce paragraphe sont, comme il est facile de le croire, sans aucun fondement. Nous avons déjà essayé de démontrer qu’il n’y a pas de motifs pour admettre que les agriculteurs sont plus intelligents que les travailleurs employés dans les manufactures et dans le commerce, et que l’intelligence de ces derniers souffre de ce que, par suite de la division du travail dans les fabriques, ils sont obligés de faire toujours la même chose. C’est précisément le contraire qui a lieu. La population des manufactures est généralement plus instruite que celle des campagnes, et son intelligence s’est développée en raison de l’accroissement du nombre et de la plus grande division du travail. L’idée que le travail dans les manufactures détruit chez les hommes les vertus sociales et militaires est plus fausse encore. Les villes elles pays, dans les temps anciens et modernes, qui ont été les plus avancés dans les arts de l’industrie et du commerce, se sont en même temps le plus distingués par leur patriotisme et leur courage. Il n’est pas nécessaire de sortir de l’Angleterre pour trouver des preuves irrécusables des erreurs contenues dans ce paragraphe. Nos manufactures ont atteint un développement inouï pendant les derniers cinquante ans ; et la division du travail est poussée plus loin en Angleterre que dans les autres pays ; mais, bien que le gouvernement n’ait rien fait pour son instruction , peut-on dire que la population des fabriques soit devenue stupide et ignorante ? Que les hommes travaillant dans la manufacture soient moins capables que ceux des campagnes de connaître les intérêts du pays, et qu’ils seraient incapables de le défendre en cas de guerre ? Toutes ces assertions sont sans aucun fondement. C’est un de ces cas très-rares où le jugement d’Adam Smith s’est laissé influencer par d’anciens préjugés : il aurait dû savoir que le régiment de chevau-légers du général Elliot, qui s’est tant distingué pendant la guerre de Sept Ans, a été en grande partie composé des tailleurs de la capitale. Quant à l’observation que les manufactures affaiblissaient les forces physiques et militaires, il suffit de rappeler que, pendant la dernière guerre, ce furent les villes manufacturières et commerçantes qui fournissaient les contingents de troupes les plus considérables. Des faits de cette importance prouvent, au delà de toute contestation, que, quels que puissent être les changements introduits dans les mœurs de notre nation, nos troupes sont aussi capables que jamais de supporter des fatigues et de montrer du courage et de la résolution. Mac Culloch.
↑ Cette distinction entre la guerre ancienne et la guerre moderne nous parait tout à fait imaginaire. Rien, assurément, ne pouvait être plus terrible que le choc dans les batailles anciennes, et pour que les soldats tinssent ferme dans cette rencontre terrible, il fallait que les habitudes d’ordre et de discipline fussent aussi fortes que dans les temps modernes. Buchanan.
↑ Les troupes grecques égalaient en bravoure les Macédoniens. Mais ce fut plutôt à son génie qu’à la supériorité de ses soldats que Philippe dut la conquête de la Grèce. Buchanan.

↑ Dans toutes ces batailles, ce fut plutôt la science d’Annibal que la valeur de ses troupes qui décida du succès. À Trasimène, et à Cannes particulièrement, les dispositions de la bataille avaient été très-savantes. La première de ces batailles eut lieu dans les Apennins. Annibal avait, par différentes ruses, attiré ses ennemis dans un défilé étroit entre les montagnes et le- lac de Trasimène. Le gros de son année était sur les hauteurs ; après que les Humains se furent engagés dans le défilé, il tomba sur eux, et ils étaient vaincus de toutes parts avant qu’ils eussent eu le temps de se ranger en bataille. À Cannes, il dut la victoire à l’art avec lequel il fit manœuvrer son centre ; il le fit avancer en forme convexe au commencement de la bataille ; puis, le faisant reculer à mesure que les Romains attaquaient, il en renversa complètement la forme : les rangs se refermèrent sur le centre ennemi cl l’enfoncèrent de tous les côtés. Tous ces mouvements furent exécutes avec une exactitude parfaite, et l’armée romaine, après une résistance héroïque, fut complètement anéantie, sans doute en partie par la valeur des ennemis, mais surtout par le génie supérieur de leur grand capitaine. Buchanan.

↑ Chez une nation éclairée et avec un gouvernement libre, une armée permanente n’est pas seulement inutile, mais encore dangereuse, puisque évidemment elle met le pouvoir entre les mains du souverain. Les lois et les institutions les plus sages ne sont d’aucune utilité si les garanties de l’exécution leur manquent ; et comment une pareille garantie pourrait-elle exister, quand le souverain dispose d’un instrument de violence aussi formidable ? Le soldat par profession n’est pas très-disposé à soutenir la cause de la liberté et de l’ordre ; il est violent par état, et il est toujours prêt à seconder les vues de ses chefs. Adam Smith a dit avec justesse, qu’avec une armée régulière et permanente, le souverain peut dédaigner toutes les démonstrations injustes, séditieuses et turbulentes ; mais ne peut-il pas également dédaigner toute espèce de représentations ? Et les hommes au pouvoir, appuyés surtout sur une bonne armée, ne sont-ils pas presque toujours enclins à regarder comme séditieuses et turbulentes les démonstrations qui ont pour but d’examiner leur conduite ? Nous ne trouvons pas qu’ils supportent mieux « les excès d’une liberté turbulente », parée qu’ils ont une armée permanente. L’Amérique n’a point d’armée permanente, et pourtant les hommes d’État dans ce pays sont plus librement interrogés que partout ailleurs. La loi, dans ce pays, ne regarde pas la vérité envers les hommes au pouvoir comme un libelle et une offense, et ce degré de liberté n’existe dans aucun autre pays. Buchanan.
↑ On les trouve dans l’Histoire d’Angleterre, par Tyrrel. (Note de l’auteur)
↑ On distingue en Angleterre les cours de loi et les cours de conscience ou d’équité ; ces dernières différent par la forme de procéder, et elles sont moins astreintes, pour le fond de leurs jugements, à suivre strictement la lettre de la loi ; mais elles peuvent se décider sur la bonne foi des parties, ou, comme on le disait pour nos justices consulaires, ex æquo et bono.
↑ Toutes ces turpitudes sont encore autorisées aujourd’hui par les divers codes de procédure de l’Europe et exploitées sans miséricorde par les avoués, les huissiers et les gens de loi. Elles font la honte et le fléau de notre temps. Quand donc commencera la croisade qui doit y mettre un terme ? A. B.
↑ Nombre de tonneaux qui forment la contenance du vaisseau.
↑ Depuis la publication des deux premières éditions de cet ouvrage, j’ai eu de fortes raisons de croire que la totalité des droits de barrière perçus en Grande-Bretagne ne produit pas un revenu net d’un demi-million, somme qui, sous la régie du gouvernement, ne suffirait pas pour tenir en bon état cinq des principales routes du royaume. (Note de l’auteur).

↑ J’ai maintenant de bonnes raisons de croire que toutes ces sommes conjecturales sont beaucoup trop fortes. (Note de l’auteur).
↑ Il est inutile de faire remarquer que ces observations d’Adam Smith s’appliquent à un ordre de choses qui n’est plus. Depuis que les Français sont rentrés en possession de leurs droits, et principalement depuis 1830, la législation des routes a été refaite sur des bases plus équitables, et la France d’aujourd’hui ne ressemble plus guère à la France de 1775. A. B.
↑ L’expression joint stock compagnies, qui est dans l’original, ne peut être traduite d’une manière tout à fait exacte en français : société commerciale ne dirait pas assez ; société en commandite dirait trop, car les Anglais n’ont pas, à proprement parler, de sociétés en commandite. J’ai adopté le terme de compagnie par actions, de préférence à tout autre, parce qu’il se rapproche le plus de l’idée anglaise. Le comte Garnier s’était servi des mots compagnies en société de fonds, qui ne signifient rien. Qu’est-ce que des compagnies en société ? A. B.
↑ Cette compagnie se nomme aussi compagnie de la mer du Nord, pour la distinguer de la compagnie du Levant, ou autrement de Turquie.
↑ Freeman, c’est-à-dire ayant le droit de maîtrise dans une corporation de métier ou de commerce, droit qui s’acquiert par l’apprentissage, ou par l’argent, ou par concession. À ce titre est attaché le droit de concourir aux offices municipaux et aux élections des membres du parlement, représentants de la cité, ville ou bourg.
↑ Les officiers composant la cour de l’échiquier ont tous le titre de baron. L’un d’eux a la dénomination de cursitor. Ses fonctions principales consistent à signer en chef les actes émanés de cette cour.
↑ Compagnies par actions (exclusive or joint stock companies). Une compagnie par actions est une société qui possède un capital social déterminé et divisé en un nombre plus ou moins grand d’actions transférables ; elle est administrée au profit des actionnaires par un corps de directeurs élus, et obligés de rendre compte de leur gestion. Quand une fois toutes les actions ou portions du capital social out été souscrites, nul ne peut devenir membre de la compagnie sans avoir préalablement acheté une ou plusieurs actions appartenant aux membres déjà existants. Les membres n’agissent jamais individuellement. Les décisions sont prises en commun ; leur exécution est confiée aux directeurs ou aux agents employés par eux. D’après le droit commun de l’Angleterre, tous les membres d’une compagnie par actions sont liés entre eux ; ils restent solidairement responsables, sur leurs fortunes, des dettes de la compagnie. Ils peuvent faire des arrangements entre eux, ayant pour but de limiter leurs obligations mutuelles ; mais, à moins d’être autorisés par une autorité compétente à changer leurs statuts, ils restent indéfiniment responsables vis-à-vis du public. Le Parlement limite quelquefois la solidarité des actionnaires des compagnies de ce genre établies par un statut jusqu’à concurrence du montant des actions souscrites par eux. On supposait, jusqu’à une époque récente, qu’une charte d’incorporation accordée par la couronne devait avoir le même effet ; mais, par l’acte 6 (rendu sous Georges IV, ch. ixxvi), la couronne est expressément investie du droit d’accorder des chartes d’incorporation portant que les membres des compagnies seraient individuellement responsables, dans des limites et avec des restrictions qui seraient jugées convenables. Depuis, on a très-souvent accordé des chartes à l’effet de rendre des compagnies capables de poursuivre ou d’être poursuivies en justice, au nom de plusieurs de ses fonctionnaires, sans que la responsabilité des actionnaires vis-à-vis du public en soit aucunement limitée. Cette limitation ne peut être implicitement reconnue ni par une charte ni par un acte du Parlement ; elle existe seulement quand elle est expressément mentionnée.
Utilité des compagnies par actions. Quand le capital requis pour une entreprise excède les forces d’un seul homme, une association devient indispensable pour son exécution. Toutes les fois que les chances du succès d’une entreprise seront douteuses et qu’un laps de temps plus ou moins long sera nécessaire pour en voir la fin, un seul individu, quoique prêt à contribuer pour sa part avec d’autres, ne voudra point, quand même il en aura les moyens, assumer toute la responsabilité de l’affaire. De là la nécessité et l’avantage des compagnies ou des associations. Nous leur devons les canaux qui traversent ce pays dans toutes les directions ; la construction des docks et des grands magasins, l’institution des principales banques et des assurances, une foule d’établissements d’utilité publique, enfin, que l’association des hommes et des capitaux a seule rendus possibles.
Compagnies privilégiées (open or regulated companies). Les affaires de ces compagnies ou associations sont conduites par des directeurs employés par les membres. La compagnie n’a pas de fonds commun. Chaque individu paye une somme en entrant, ou, ce qui a lieu plus ordinairement, une contribution annuelle. Un droit pouvant être affecté aux affaires de la compagnie est quelquefois imposé sur l’importation et l’exportation des marchandises dans les pays avec lesquels la compagnie fait le commerce. Les sommes ainsi acquises sont employées par les directeurs à envoyer des ambassadeurs, des consuls et autres fonctionnaires publics, capables de faciliter les entreprises commerciales, ou à construire des factoreries, à équiper des croiseurs, etc. Les membres d’une pareille compagnie font le commerce avec leurs propres capitaux, et à leurs risques personnels. Une compagnie privilégiée, en définitive, n’est qu’un moyen de faire payer à ceux qui sont engagés dans une certaine branche du commerce les frais généraux ou politiques rendus indispensables, sauf à laisser aux individus toute latitude dans les entreprises particulières. La formation d’une pareille compagnie sera le meilleur mode d’assurer à une certaine branche du commerce la protection que le gouvernement refuserait ou ne serait pas en droit d’accorder. Ce mode d’association, tout en établissant une protection sûre, laisse aux particuliers toute liberté d’action.
Quant à ce qui concerne la protection, on pourrait peut-être admettre, avec Adam Smith, qu’une compagnie par actions est mieux appropriée à cet effet qu’une compagnie privilégiée. Les directeurs de cette classe d’associations, dit Adam Smith, n’ont aucun intérêt dans le commerce général de la compagnie, au profit de laquelle des vaisseaux de guerre, des factoreries et des forts doivent être construits, lis sont capables de négliger ces intérêts et de ne penser qu’à leurs propres affaires. Dans les compagnies par actions, au contraire, les intérêts des directeurs s’identifient avec ceux de la compagnie. Ils n’ont pas de capitaux particuliers engagés dans le commerce ; leurs profits dépendent uniquement de l’emploi avantageux et prudent du fonds commun, et il est à présumer qu’ils rechercheront par tous les moyens possibles de faire prospérer les entreprises communes. D’un autre côté, il peut arriver que les directeurs d’une compagnie par actions ne sachent pas s’arrêter au point juste ; ils ont presque toujours essayé d’étendre les relations commerciales par la force et de devenir plutôt des rois que des marchands. Cette dernière circonstance était même assez facile à prévoir, attendu que la considération et le patronage résultant de cette politique devaient être pour eux d’une plus grande importance qu’une augmentation modeste des dividendes de leur capital. Quand ils ont été à même de pouvoir l’entreprendre, ils n’ont jamais reculé devant l’emploi de la force pour mener à bout leurs projets ; et, au lieu de se contenter de magasins et de factoreries, ils ont construit des fortifications, engagé des troupes et fait la guerre. Les compagnies privilégiées ont procédé autrement. Leurs affaires sous leur propre contrôle ont été conduites d’une manière modeste et économique ; leurs établissements n’ont été que des factoreries, et elles se sont rarement laissé entraîner par les idées de conquête et de domination.
Si donc nous les considérons simplement comme des machines du commerce, nous ne devons pas douter de la supériorité des compagnies privilégiées sur les compagnies par actions. Les dernières ont en outre un grave défaut, c’est d’exclure complètement l’industrie et la rivalité des individus. Quand une compagnie de ce genre est on possession d’un privilège particulier, elle fera certainement tout pour son propre intérêt, quelque préjudiciable qu’il puisse être au public. Si elle a le monopole du commerce d’un pays particulier ou d’une marchandise particulière, elle ne manquera pas, en s’emparant du marché intérieur et extérieur, de vendre les marchandises qu’elle importe ou exporte à des prix d’une hausse artificielle. Son but est, non point d’employer des capitaux considérables, mais de réaliser de grands bénéfices sur des capitaux relativement petits. La conduite de la compagnie hollandaise des Indes Orientales, qui brûlait les épices pour que la trop grande quantité n’en fit pas baisser les prix, peut servir d’exemple de la manière dont agissent de pareilles associations. Les hommes voudront toujours vendre au plus haut prix possible ; délivrés de la concurrence et protégés par le privilège du monopole, ils n’hésiteront pas à élever les prix aussi haut que le leur permettra la concurrence des acheteurs, et ils réaliseront ainsi de très-gros bénéfices. Cependant, malgré tous ces avantages, les compagnies, à cause de la négligence, de la profusion et du gaspillage inséparables de la direction des grandes associations, se sont presque toujours endettées. La compagnie des Indes Orientales a perdu beaucoup dans le commerce, et sans les revenus de l’Inde, elle aurait déjà cessé d’exister. Acheter sur un marché, vendre avec profit sur un autre, suivre exactement toutes les variations qui surviennent dans les prix, dans la provision et les demandes des marchandises ; connaître les besoins des différents marchés, et conduire ensuite les opérations de la manière la plus convenable et la plus économique, ce sont là des choses qui exigent une grande vigilance et une attention soutenue, et qu’on ne pourra jamais obtenir des directeurs et employés d’une grande compagnie par actions ; de là il est souvent arrivé que des particuliers aient réussi dans certaines branches du commerce qui avaient ruiné les compagnies.
Constitution des compagnies. Quand une demande est soumise au Parlement afin d’obtenir un acte d’incorporation accordant à plusieurs individus le droit de se constituer en compagnie par actions pour l’exécution d’une entreprise utile, il faut bien se garder de leur accorder des privilèges qui pourront devenir préjudiciables au public. Quand une compagnie est formée pour la construction d’un dock, d’une route ou d’un canal, il sera nécessaire, pour que des particuliers s’engagent dans l’entreprise, de leur accorder des privilèges pour un certain nombre d’années. Mais, si d’autres personnes étaient à jamais empêchées de construire de nouveaux docks, d’ouvrir de nouvelles lignes de communication, il en résulterait pour le public un dommage durable. Il sera très-utile, par exemple, de former une compagnie ayant pour but de conduire de l’eau dans une ville ; mais, s’il n’y avait pas d’autres sources dans le voisinage que celles sur lesquelles la compagnie a acquis des droits, elle pourrait, si l’acte d’incorporation ne le lui interdit, élever le prix de l’eau d’une manière exorbitante et réaliser de grands bénéfices au préjudice du public. Ainsi, toutes les fois qu’il s’agira de la construction d’un canal, d’un chemin de fer, il sera d’une bonne politique de régler les taux du prix pour les différents services, et de limiter également les dividendes en fixant un maximum au delà duquel ils ne pourront plus être augmentés, en stipulant à cet effet que, dans le cas où le taux du prix établi par la compagnie s’élèverait au-dessus du maximum des dividendes et des frais de l’exploitation, elle serait tenue de le réduire jusqu’au rétablissement du niveau ; ou, dans le cas où elle refuserait d’accepter cette condition, on pourrait exiger que le surplus des dividendes fût affecté à l’amortissement du capital de l’association, de manière qu’à la fin les dépenses servant au payement des dividendes se trouveraient abolies. Si ce principe avait été appliqué aux premiers canaux qu’on a construits en Angleterre, le transport des marchandises sur les lignes de communication les plus importantes ne coûterait presque rien maintenant, et on aurait obtenu ce résultat sans que le nombre de ces entreprises en fùl diminué. Il y a très-peu de personnes qui, au moment où elles s’engagent dans de pareilles entreprises, s’attendent à plus de dix ou de douze pour cent de bénéfices ; elles seraient même toutes prêtes à s’engager si elles pouvaient seulement en espérer autant. N’est-il pas alors du devoir du gouvernement de faire en sorte que, dans le cas d’un succès inattendu, le public puisse en tirer quelque avantage? Ici la concurrence ne peut pas rétablir le niveau. Ceux qui viennent les premiers s’emparent de la meilleure, sinon de l’unique ligne propre à l’établissement d’un canal ou d’un chemin de fer ; ils obtiendront ainsi un véritable monopole sans qu’on puisse les en déposséder. Il y a donc avantage à stipuler le taux des prix et le maximum des dividendes ; sans décourager les entreprises, on aura garanti les intérêts du public. Quand, à avantage égal pour le public, une entreprise pourra être formée par des particuliers tout aussi bien que par des compagnies, ou quand les risques et les difficultés ne sont pas trop grands, on ferait bien de n’accorder aucun privilège et de les traiter, sous tous les rapports, comme de simples particuliers. Mac Culloch.

↑ Ce vaisseau se nommait vaisseau de permission ; il devait être du port de cinq cents tonneaux ; un quart du profit appartenait à Sa Majesté catholique, et en outre 5 pour 100 sur les trois autres quarts de ce profit.
↑ Non pas par le traité d’Aix-la-Chapelle, mais par un traité signé à Buen-Retiro, le 3 octobre 1750, par lequel le roi d’Espagne s’oblige envers le roi d’Angleterre à payer 100,000 liv. sterl. à la compagnie de l’Asiento, pour tous droits, demandes et prétentions.
↑ Ces deux compagnies d’assurance sont établies à Londres, en vertu de patentes du même jour, 8 janvier 1720.
↑ Adam Smith se montre ici d’une partialité étrange. Il semblerait au contraire que la discipline des collèges est surtout favorable aux études, et par conséquent aux écoliers plutôt qu’aux maîtres, à qui le désordre ne saurait profiter sans doute, mais qui en souffriraient beaucoup moins, assurément, que leurs élèves. Nous n’admettons pas non plus ce que l’auteur ajoute un peu plus bas de l’inutilité de la discipline pour les jeunes gens, après l’âge de douze à treize ans. C’est principalement à cet âge, et même après vingt ans, que la sévérité d’une règle nous parait indispensable. On n’a qu’à observer ce qui se passe dans nos grandes écoles libres, telles que les facultés de droit et de médecine, ainsi que dans les universités anglaises et allemandes, pour se bien convaincre des inconvénients de la tolérance extrême qui y règne, et du dommage qu’en éprouvent ces milliers de candidats avortés, dont les échecs sont beaucoup moins dus à la faiblesse de leur intelligence qu’à celle de leurs supérieurs. Je suis persuadé, pour mon compte, et je crois avoir le droit de le dire après une expérience de plus de vingt ans dans la direction et l’enseignement de la jeunesse, que l’Europe voit s’évanouir chaque année d’immenses ressources intellectuelles par suite de l’indiscipline qui est tolérée dans les grands établissements d’instruction publique. Le désordre est incomparablement plus frappant dans les facultés et universités que dans les collèges. En France surtout, cette plaie appelle à un très-haut degré la sollicitude des hommes sérieux. Quelque opinion qu’on ait du régime impérial, lu discipline sévère qu’il avait établie dans l’université était un pas vers le progrès, j’ai presque dit la source de tous les progrès ; la liberté qui y règne aujourd’hui est un obstacle fâcheux, sans parler des autres. Pour moi, je préfère ici les idées de Napoléon à celles d’Adam Smith. A. B.
↑ Quelle illusion, pour un ancien professeur ! A. B.
↑ Nous avons cru devoir traduire par histoire naturelle les mois natural philosophy, dont les Anglais se servent pour caractériser cette science. M. le sénateur Garnier avait adopté l’expression philosophie naturelle qui n’a aucun sens dans notre langue, ou qui du moins n’a jamais été prise dans l’acception d’histoire naturelle. A. B.
↑ La morale est plutôt l’affaire du sentiment que du raisonnement, et il n’est pas facile de voir quelle influence la philosophie pourrait exercer sur elle. Car le philosophe, que peut-il expliquer que nous ne sachions déjà ?
Si nous étendons le champ de la morale jusqu’aux opérations de l’esprit, nous avons, sans aucun doute, un plus vaste espace ouvert à des recherches subtiles et ingénieuses. Il est agréable de scruter les facultés de noire âme et de tracer les lignes qui lient les différentes sensations entre elles. Mais une pareille étude ne mérite pas le nom de science, puisqu’elle n’apporte aucun résultat nouveau. La science de l’âme (psychologie), comme on l’appelle maintenant, a été trop vantée par les philosophes modernes, et ou avait les espérances les plus exagérées de son influence sur la société et la vie. Mais il est évident que les maux qui affectent le monde viennent des imperfections de la nature humaine, trop profondes pour être modifiées ou éloignées par ces spéculations bizarres. C’est l’égoïsme naturel à l’homme qui, en le poussant à chercher son bien-être par tous les moyens possibles, empêche la perfection de la société. L’homme se trompe, non par ignorance, mais eu dépit de ses connaissances ; et, pour le corriger de ses fautes, il faut moins éclairer son esprit qu’améliorer son cœur. Mais comment produire un pareil effet ? On croyait que la philosophie fournirait quelque moyen d’action nouveau pour empêcher les hommes de s’abandonner à leurs mauvais penchants. Mais pouvons-nous supposer que, par une simple analyse de ses facultés, l’homme deviendra un être nouveau ! Si cela ne peut pas avoir lieu, toutes les améliorations auxquelles s’attendaient les admirateurs de cette science ne seront qu’imaginaires : l’égoïsme continuera d’être le moteur principal de toutes les actions. La fraude, la violence, la cruauté continueront de régner, et la société sous les dehors de l’ordre, vue de près, se. présentera sous de bien noires couleurs. — On peut ajouter encore. : tandis que dans les sciences naturelles les progrès de nos connaissances sont manifestes, et qu’il nous est possible de marquer les points que de nouvelles investigations ont éclaircis, la philosophie ou la métaphysique ne s’est enrichie d’aucune nouvelle découverte. Ses partisans parlent beaucoup de ce que dans l’avenir elle est appelée à produire, mais ils gardent le silence sur ce qu’elle a produit dans le passé ; et, si nous devons juger de l’avenir par le passé, notre foi dans les améliorations futures n’est rien moins que solide. Buchanan.
↑ Aujourd’hui que la philosophie est enseignée très-compendieusement et très-sérieusement, sommes-nous plus avancés ? A. B.
↑ « L’empereur a fondé pour chaque secte une chaire de philosophie. Les honoraires en sont assez considérables, et les stoïciens, les disciples de Platon, ceux d’Épicure et d’Aristote, y ont une égale part. Lorsqu’un de ces professeurs vient à mourir, un autre lui succède, nommé par le suffrage et d’après l’examen des philosophes les plus habiles. Or, le prix du combat n’est pas, comme dans Homère, une peau de bœuf, mais 10,000 drachmes payées au vainqueur chaque année, à condition de donner des leçons à la jeunesse. » (Lucien, Traduction de Belin de Malin, tome. III, page 529.)
↑ Il nous semble qu’Adam Smith a poussé bien loin ici l’amour de la concurrence. A. B.
↑ Ce triste résultat n’est pas la conséquence de l’existence des institutions publiques pour l’éducation, mais de l’esprit de système qui domine chez ceux qui les dirigent. Les professeurs de l’université de France sont généralement des hommes de mérite ; mais ils sont condamnés à enseigner des choses inutiles au plus grand nombre de leurs auditeurs. A. B.
↑ Malheureusement on ne leur enseigne rien.
↑ Il suffît de lire ces belles pages de Smith pour apprécier le reproche d’indifférence sociale adressé à l’auteur par quelques prétendus économistes de nos jours. A. B.
↑ Hume, Histoire d’Angleterre.
↑ Le mot anglais provisor désigne ceux qui sollicitaient des bulles du pape pour se faire investir du bénéfice ou dignité ecclésiastique, ou qui se prévalaient de pareilles bulles. Os bulles s’appelaient provision ou expectative, parce qu’elles nommaient un successeur par avance et en attendant la vacance du bénéfice.
↑ La réforme, sans contredit le coup le plus terrible porté à l’Église romaine, n’était-elle pas due à la raison humaine ? Les prédications de Luther contre les indulgences ne s’adressaient-elles pas à la raison humaine ? Et la controverse, que voulait-elle ? sinon porter la conviction dans les esprits ? Les circonstances dont parle Adam Smith ont ajouté aux efforts de la raison, mais ce fut elle qui avait donné la première impulsion. Ce fut la raison humaine qui brisa le joug de la superstition, et qui depuis nous a préservés de toute rechute. Buchanan.
↑ Voyez l’Essai sur les mœurs et l’esprit des nations, par Voltaire, chap. cxxix.
↑ S’il est permis de juger d’une Église par ses œuvres, l’Église d’Angleterre est bien certainement la plus détestable institution humaine qui ait abusé du sentiment religieux, après l’inquisition d’Espagne. On pourrait la caractériser par trois mots : hypocrisie, bigoterie, cupidité. C’est la honte de l’Angleterre. A. B.
↑ Les presbyléries sont des chambres ecclésiastiques composées des ministres d’un district et d’un ancien par paroisse ; elles se forment en assemblée tous les mois : leur attribution est d’examiner les candidats qui aspirent au ministère et de déposer les ministres qui ont encouru la destitution. La réunion des presbytérien compose les synodes provinciaux et le synode général.
↑ L’auteur avait ajoute en cet endroit, dans sa première édition, et a retranché dans les éditions postérieures, la phrase suivante : « Plusieurs de ceux sur lesquels nous n’avons pas la certitude qu’ils aient été professeurs publics, ont été, à ce qu’il semble, instituteurs particuliers. Nous savons que Polybe était l’instituteur particulier de Scipiou Émilien, et il y a des raisons assez plausibles de croire que Denis d’Halicarnasse avait rempli les mêmes fonctions auprès des enfants de Marcus et de Quintus Ciréron. »

%%%%%%%%%%%%%%%%%%%%%%%%%%%%%%%%%%%%%%%%%%%%%%%%%%%%%%%%%%%%%%%%%%%%%%%%%%%%%%%%
%                                  Chapitre 2                                  %
%%%%%%%%%%%%%%%%%%%%%%%%%%%%%%%%%%%%%%%%%%%%%%%%%%%%%%%%%%%%%%%%%%%%%%%%%%%%%%%%

\chapter{Des sources du Revenu général de la société ou du revenu de l’État}
\markboth{Des sources du Revenu général de la société ou du revenu de l’État}{}

Le revenu qui doit pourvoir non-seulement aux dépenses de la défense publique et à celles que demande la dignité du premier magistrat, mais encore à toutes les autres dépenses nécessaires du gouvernement, pour lesquelles la constitution de l’État n’a pas assigné de revenu particulier, peut être tiré, soit, en premier lieu, de quelques fonds qui appartiennent en particulier au souverain ou à la république, et qui soient indépendants du revenu du peuple, soit, en second lieu, du revenu du peuple.

SECTION PREMIÈRE.
Des fonds ou sources du revenu qui peuvent appartenir particulièrement au souverain ou à la république.

Les fonds ou sources de revenu qui peuvent particulièrement appartenir au souve­rain ou à la république consistent nécessairement ou en capitaux, ou en fonds de terre[1].
Le souverain, comme tout autre capitaliste, peut retirer un revenu de son capital, soit en l’employant lui-même, soit en le prêtant à d’autres. Dans le premier cas, son revenu consiste en profits ; dans le second, en intérêts. 
Le revenu d’un chef arabe ou tartare consiste en profits ; il provient principalement du lait et du croît de ses bestiaux et de ses troupeaux, dont il surveille lui-même la direction, étant le premier pasteur ou berger de sa horde ou de sa tribu. Ce n’est cependant que dans ce premier état agreste et informe du gouvernement civil que le profit a jamais pu faire la principale partie du revenu public d’un État monarchique.
De petites républiques ont quelquefois tiré un revenu considérable de profits provenant d’affaires de commerce. On dit que la république de Hambourg s’en fait un avec les profits d’un magasin de vin et d’une boutique de pharmacie[2]. Ce ne peut pas être un très-grand État que celui dont le souverain a le loisir de mener un commerce de marchand de vin ou d’apothicaire.
Le profit d’une banque publique a été une source de revenu pour des États plus considérables ; c’est ce qui s’est vu non-seulement à Hambourg, mais encore à Venise et à Amsterdam. Quelques personnes ont même pensé qu’un revenu de cette sorte ne serait pas indigne de l’attention d’un empire aussi puissant que la Grande-Bretagne. En comptant le dividende ordinaire de la banque d’Angleterre à 5 1/2 pour 100, et son capital à 10,780,000 livres, le profit annuel, toutes dépenses de régie prélevées, peut monter, dit-on, à 592,900 livres. Le gouvernement pourrait, à ce qu’on prétend, emprunter ce capital à l’intérêt de 3 pour 100, et en prenant lui-même la régie de la banque, il pourrait faire par an un profit clair de 269,500 liv. L’administration rangée, vigilante et économe d’une aristocratie, telle que celles de Venise et d’Amsterdam, est extrêmement propre, à ce qu’il semble d’après l’expérience, à régir une entreprise de commerce de ce genre. Mais c’est une chose qui ne laisse pas d’être pour le moins beaucoup plus douteuse que de savoir si la conduite d’une pareille affaire peut être confiée avec sûreté à un gouvernement tel que celui d’Angleterre, qui, quels que puissent être d’ailleurs ses avantages, n’a jamais été cité pour sa bonne économie ; qui, en temps de paix, s’est en général conduit avec la prodigalité, l’abandon et l’insou­ciance naturelle peut-être aux monarchies, et qui a constamment agi, en temps de guerre, avec tous les excès et l’instabilité ordinaire aux démocraties.
Les postes sont, à proprement parler, une entreprise de commerce ; le gouver­nement fait l’avance des frais d’établissement des différents bureaux, ceux de l’achat ou du louage des chevaux et voitures nécessaires, et il s’en rembourse, avec un gros profit, par les droits perçus sur ce qui est voituré. C’est peut-être la seule affaire de commerce qui ait été conduite avec succès, je crois, par toute espèce de gouverne­ment. Le capital qu’il s’agit d’avancer n’est pas très-considérable. Il n’y a pas de secret ni de savoir-faire dans une pareille besogne. Les rentrées sont non-seulement assu­rées, mais elles se font immédiatement. 
Les princes cependant se sont souvent engagés dans beaucoup d’autres projets de commerce, et n’ont pas dédaigné de chercher, comme des particuliers, à améliorer leur fortune en courant les hasards de différentes spéculations commerciales de la classe ordinaire ; ils n’ont jamais réussi, et il est à peu près impossible qu’il en soit autrement avec la prodigalité qui règne communément dans la gestion de leurs affai­res. Les agents d’un prince regardent la fortune de leur maître comme inépuisable ; ils ne s’embarrassent pas du prix auquel ils achètent ; ils ne s’inquiètent guère à quel prix ils vendent ; ils ne comptent pas davantage ce qu’il leur en coûte pour transporter les marchandises d’un endroit dans un autre. Ces agents vivent souvent dans la profusion, comme les princes, et quelquefois aussi, malgré toutes ces profusions, et par la manière dont ils savent régler leurs comptes, ils acquièrent des fortunes de princes. C’est ainsi, à ce que nous dit Machiavel, que les agents de Laurent de Médicis, qui n’était pas un prince dépourvu de talents, menaient son commerce. La république de Florence fut obligée plusieurs fois de payer les dettes dans lesquelles l’avaient jetée leurs extravagances ; aussi trouva-t-il à propos d’abandonner le métier de marchand, métier auquel sa famille était originairement redevable de sa fortune, et d’employer par la suite ce qui lui restait de cette fortune, ainsi que les revenus publics dont il avait la disposition, à des dépenses et à des entreprises plus dignes du poste qu’il occupait.
Il semble qu’il n’y ait pas deux caractères plus incompatibles que celui de mar­chand et celui de souverain. Si l’esprit mercantile des directeurs de la compagnie des Indes anglaise en fait de très-mauvais souverains, l’esprit de souveraineté paraît aussi les avoir rendus de très-mauvais marchands. Tant qu’ils ne furent que marchands, ils conduisirent leur commerce avec succès, et se virent en état de payer sur leurs profits un dividende honnête à leurs actionnaires. Depuis qu’ils sont devenus souverains, ils se sont vus obligés, avec un revenu qui était originairement, à ce qu’on dit, de plus de 3 millions sterling, d’implorer humblement des secours extraordinaires du gouverne­ment, pour éviter une banqueroute imminente. Dans la première organisation de la compagnie, ses facteurs dans l’Inde se regardaient comme des commis de marchands ; dans l’organisation actuelle, ses facteurs se regardent comme des ministres de souverains.
Un État peut quelquefois composer une partie de revenu public avec l’intérêt d’une somme d’argent, comme avec les profits d’un capital. S’il a amassé un trésor, il peut prêter une partie de ce trésor, soit à des États étrangers, soit à ses propres sujets.
Le canton de Berne tire un revenu considérable du prêt d’une partie de son trésor aux États étrangers, c’est-à-dire du placement qu’il en a fait dans les fonds publics de différentes nations de l’Europe qui ont des dettes, principalement dans ceux de France et d’Angleterre. La sûreté d’un tel revenu dépendra de plusieurs conditions : 1° de la sûreté des fonds dans lesquels il est placé, et de la bonne foi du gouvernement qui a le maniement de ces fonds ; 2° de la certitude ou du moins de la probabilité qu’on restera en paix avec la nation débitrice. Dans le cas d’une guerre, il pourrait bien se faire que le premier de tous les actes d’hostilité, de la part de la nation débitrice, fût une confis­cation des fonds du créancier. Cette mesure politique de prêter de l’argent aux États étrangers est, autant que je puis savoir, particulière au canton de Berne.
La ville de Hambourg[3] a établi une espèce de bureau de prêt public, qui prête de l’argent aux sujets de l’État sur des gages, à l’intérêt de 6 pour 100. Ce bureau de prêt ou lombard, comme on l’appelle, rapporte à l’État, à ce qu’on prétend, un revenu de 150,000 écus[4], qui, à 4 sch. 6 den. pièce, font 33,730 liv. sterling.
Le gouvernement de Pennsylvanie, sans amasser de trésor, trouva une manière de prêter à ses sujets, non pas de l’argent, à la vérité, mais ce qui équivaut à de l’argent. Il avança à des particuliers, à intérêt et sur des sûretés en biens-fonds de la valeur du double, des papiers de crédit ou billets d’État, remboursables dans les quinze années de leur date, transmissibles néanmoins de main en main, comme des billets de banque, et qui étaient déclarés, par un acte de l’assemblée, offres légales de payement pour toutes dettes entre habitants de la province. Par là, il se fit un petit revenu qui ne laissa pas que d’avancer considérablement le payement des dépenses annuelles de ce gouvernement réglé et économe, dont toutes les charges ordinaires allaient à environ 4,500 livres. Le succès d’une ressource de ce genre a dû dépendre de trois différentes circonstances : 1° du besoin d’un instrument de commerce outre l’or et l’argent circulant, ou de la demande d’un capital en choses consommables, tel qu’on n’ait pu se le procurer sans envoyer au-dehors, pour l’acheter, la plus grande partie de l’or et de l’argent du pays ; 2° du bon crédit du gouvernement, qui s’est servi de cette ressource ; 3° de la modération avec laquelle on en fait usage, la valeur totale de ces billets de crédit n’ayant jamais excédé celle de la monnaie d’or et d’argent qui eût été nécessaire pour faire marcher la circulation, s’il n’y eût pas eu de billets. La même ressource a été adoptée en différentes occasions, par plusieurs autres colonies américaines ; mais, faute de cette modération, elle a produit, dans la plupart de ces colonies, plus de désordres que d’avantages.
Toutefois, la nature mobile et périssable du crédit et des capitaux ne permet pas qu’on puisse s’en reposer sur eux pour former la principale base de ce revenu assuré, solide et permanent, qui seul peut donner au gouvernement de la sécurité et de la dignité. Aussi ne paraît-il pas que, parmi les grandes nations avancées au-delà de l’état pastoral, le gouvernement ait jamais fondé sur de pareilles ressources une gran­de partie du revenu public.
La terre est un fonds d’une nature plus stable et plus permanente et, en consé­quence, une rente de terres a formé souvent la principale source du revenu public, chez de grandes nations qui avaient déjà dépassé de fort loin l’âge des peuples pasteurs. Les républiques anciennes de la Grèce et de l’Italie ont pendant longtemps tiré, du produit ou de la rente des terres publiques, la majeure partie du revenu qui fournissait aux dépenses nécessaires de l’État. Les rentes de terres de la couronne ont constitué, pendant longtemps, la plus grande partie du revenu des anciens souverains de l’Europe.
La guerre et les préparatifs de guerre sont les deux circonstances qui occasion­nent, dans les temps modernes, la plus grande partie de la dépense nécessaire à tous les grands États. Mais, dans les anciennes républiques de la Grèce et de l’Italie, tout citoyen était soldat, et c’était à ses propres dépens qu’il servait et qu’il se préparait à servir. Ainsi, aucune de ces deux circonstances ne pouvait occasionner de dépense considérable pour l’État. La rente d’un domaine très-modique pouvait largement suffire à couvrir toutes les autres dépenses du gouvernement.
Dans les anciennes monarchies de l’Europe, les mœurs et les usages des temps préparaient suffisamment à la guerre la masse des sujets ; et quand ils entraient en campagne, d’après la nature des services féodaux auxquels ils étaient obligés, ils devaient ou s’entretenir à leurs frais, ou être entretenus aux frais de leurs seigneurs immédiats, sans occasionner au souverain aucune nouvelle charge. Les autres dépen­ses du gouvernement étaient pour la plupart très-modiques. On a vu que l’admi­nis­tration de la justice, au lieu d’être une cause de dépense, était une source de revenu. Trois journées de travail des gens de la campagne avant la moisson, et trois journées après, étaient regardées comme un fonds suffisant pour la construction et l’entretien de tous les ponts, grandes routes et autres travaux publics, que le commerce du pays était censé exiger. Dans ces temps-là, la principale dépense du souverain consistait, à ce qu’il semble, dans l’entretien de sa maison et des personnes de sa suite ; aussi les officiers de sa maison étaient-ils alors les grands officiers de l’État ; le grand-trésorier recevait ses rentes ; le grand-maître et le grand-chambellan présidaient à sa dépense domestique ; le soin de ses étables et écuries était confié au grand-connétable et au grand-maréchal. Ses maisons étaient toutes bâties en forme de châteaux forts, et étaient, à ce qu’il semble, les principales forteresses qu’il possédât ; les gardiens ou concierges de ces maisons ou châteaux pouvaient être regardés comme des espèces de gouverneurs militaires, et il paraît que c’étaient les seuls officiers militaires qu’il fallût entretenir en temps de paix. Dans un tel état de choses, la rente d’un vaste domaine pouvait très-bien, dans les circonstances ordinaires, défrayer toutes les dépenses nécessaires du gouvernement.
Dans l’état actuel de la plupart des monarchies civilisées de l’Europe, la rente de la totalité des terres du pays, régies comme elles le seraient vraisemblablement si elles appartenaient toutes à un seul propriétaire, monterait peut-être à peine au revenu ordinaire qu’on lève sur le peuple, même dans les temps de paix. Par exemple, le revenu ordinaire de la Grande-Bretagne, y compris non-seulement ce qui est néces­saire pour pourvoir à la dépense courante de l’année, mais encore ce qu’il faut pour payer l’intérêt de la dette publique et pour amortir une partie du capital de cette dette, se monte à plus de 10 millions par année. Or, la taxe foncière, à 4 sch. par livre, ne va pas à 2 millions par an. Cette taxe foncière, comme on l’appelle, est cependant censée faire le cinquième, non-seulement de la rente de toutes les terres, mais encore de celle de toutes les maisons, et de l’intérêt de tous les capitaux, à l’exception seulement de ceux prêtés à l’État et de ceux employés, comme capital de fermier, à la culture des terres. Une partie très-considérable du produit de cette taxe procède de loyers de maisons et d’intérêts de capitaux. La taxe foncière de la cité de Londres, par exemple, à 4 s. pour livre, monte à 123,399 liv. 6 s. 7 ci. ; celle de la cité de Westminster, à 63,092 liv. 1 s. 5 d. ; celle des palais de Whitehall et de Saint-James, à 30,754 liv. 6 s. 3 d. Il y a de même une certaine portion de la taxe foncière, assise sur toutes les autres cités et villes incorporées du royaume, et qui provient presque tout entière ou de loyers de maisons, ou de ce qui est censé être l’intérêt de capitaux prêtés ou placés dans le commerce. Ainsi, d’après l’évaluation sur laquelle la Grande-Bretagne est imposée à la taxe foncière, la somme totale des revenus provenant des rentes de toutes les terres, de celles de toutes les maisons et de l’intérêt de tous les capitaux, en en exceptant seulement ce qui est ou prêté à l’État, ou employé à la culture de la terre, n’excède pas 10 millions sterling par année, le revenu ordinaire que le gouvernement lève sur le peuple, encore dans les temps de paix. Il est bien vrai que l’évaluation sur laquelle la Grande-Bretagne est imposée à la taxe foncière est, en prenant la totalité du royaume en masse, de beaucoup au-dessous de la véritable valeur, quoique, dans plusieurs comtés et districts particuliers, elle soit à très-peu de chose près, à ce qu’on dit, portée à son véritable taux. La seule rente des terres, sans y comprendre les loyers des maisons ni les intérêts de capitaux, a été estimée par plusieurs personnes à 20 millions ; estimations faite en grande partie au hasard, et qu’on peut supposer, à ce que j’imagine, aussi bien au-dessus qu’au-dessous de la vérité. Mais si les terres de la Grande-Bretagne, dans l’état actuel de leur culture, ne rapportent pas une rente de plus de 20 millions par an, elles pourraient bien ne pas rapporter la moitié, très-probablement même pas le quart de cette rente, si elles appartenaient toutes à un seul propriétaire, et qu’elles fussent mises sous la régie insouciante, dispendieuse et oppressive de ses agents et préposés. Les terres du domaine de la couronne de la Grande-Bretagne ne rapportent pas actuellement le quart de la rente qu’on pourrait probablement leur faire rendre si elles étaient propriétés particulières. Si les terres de la couronne étaient plus étendues, il est probable qu’elles seraient encore plus mal régies.
Le revenu que le corps entier du peuple retire de la terre est en raison, non de la rente de la terre, mais de son produit[5]. La totalité du produit annuel des terres de chaque pays, si l’on en excepte ce qui est réservé pour semences, est ou annuellement consommée par la masse du peuple, ou échangée contre quelque autre chose qui est consommée par elle. Tout ce qui tient le produit de la terre au-dessous du point où il serait monté sans cela, diminue le revenu de la masse du peuple, encore plus qu’il ne diminue celui des propriétaires de terres. La rente de la terre, cette portion du produit qui appartient aux propriétaires, n’est pas censée excéder de beaucoup, en quelque endroit que ce soit de la Grande-Bretagne, le tiers du produit total. Si la terre qui, dans un tel état de culture, rapporte une rente de 10 millions sterling par an, pouvait, avec une autre culture, rapporter une rente de vingt (la rente étant, dans l’un et l’autre cas, supposée former le tiers du produit), le revenu des propriétaires serait seulement de 10 millions par an moindre de ce qu’il eût été dans ce meilleur état de culture ; mais le revenu de la masse du peuple serait de 30 millions moindre de ce qu’il pourrait être, sauf à déduire seulement la valeur des semences. La population du pays serait moindre de tout le nombre d’hommes que 30 millions par an (déduisant toujours les semences) pourraient faire subsister, selon la manière de vivre et de consommer usitée parmi les diverses classes de gens entre lesquelles le reste se distribuait.
Quoiqu’il n’y ait actuellement en Europe aucun État civilisé, de quelque nature qu’il soit, qui tire la plus grande partie de son revenu public de rentes de terres appartenant à l’État, cependant, dans toutes les grandes monarchies de l’Europe, il reste encore beaucoup de vastes étendues de terrain qui sont la propriété de la cou­ronne. Ce sont en général des forêts, et des forêts quelquefois où vous pourriez voyager plusieurs milles sans y trouver à peine un seul arbre ; autant de pays vraiment désert et absolument perdu, aux dépens du produit national ainsi que de la population. Dans chacune des grandes monarchies de l’Europe, la vente des terres de la couronne produirait une très-grosse somme d’argent, qui, appliquée au payement de la dette publique, pourrait dégager de toute hypothèque une portion de revenu infiniment plus grande que ces terres n’en ont jamais rapporté à la couronne. Dans les pays où les terres en grande valeur et dans le meilleur état de culture, qui produisent, au moment de la vente, à peu près le plus fort revenu qu’elles puissent rendre, sont communément vendues au denier 30, on pourrait bien s’attendre à ce que les terres de la couronne, point améliorées, mal cultivées et affermées à si bas prix, se vendissent aisément au denier 40, 50 ou même 60. La couronne se trouverait immédiatement en jouissance du revenu que l’argent de cette vente servirait à dégager de toute hypothèque. Au bout de quelques années, elle aurait encore acquis un autre revenu. Quand ces terres seraient devenues des propriétés particulières, elles seraient, au bout de peu d’années, des terres en valeur et bien cultivées. L’accroissement de produit qui en résulterait augmenterait la population du pays, en ajoutant au revenu du peuple et à ses moyens de consommation. Or, le revenu que retire la couronne des droits de douane et de ceux d’accise grossirait nécessairement avec le revenu et la consommation du peuple.
Quoique le revenu que la couronne tire de ses domaines fonciers, dans une monar­chie civilisée, ne paraisse rien coûter aux particuliers, c’est peut-être pourtant, dans le fait, celui de tous les revenus dont elle jouit qui, à égalité de produit, coûte le plus cher à la société. Ce serait, dans tous les cas, l’intérêt de la nation de remplacer ce revenu de la couronne par quelque autre revenu égal, et de partager ces terres entre des particuliers ; ce qui ne pourrait peut-être se faire mieux qu’en les mettant publi­que­ment à l’enchère.
Les seules terres qui devraient, à ce qu’il semble, appartenir à la couronne, dans une grande monarchie civilisée, ce sont les terres destinées à la magnificence et à l’agrément, telles que les parcs, jardins, promenades publiques, etc., toutes posses­sions qui sont regardées partout comme objets de dépense, et non comme source de revenu.
Ainsi, des capitaux ou des domaines publics (les deux seules sources de revenu qui puissent appartenir, comme propriété particulière, au souverain ou à la républi­que), étant les uns et les autres des moyens aussi impropres qu’insuffisants pour couvrir les dépenses ordinaires d’un grand État civilisé, il en résulte que ces dépenses doivent nécessairement être, pour la majeure partie, défrayées par des impôts d’une espèce ou d’une autre, au moyen desquels le peuple, avec une partie de ses propres revenus particuliers, contribue à composer au souverain ou à l’État ce qu’on nomme un revenu public. 



SECTION SECONDE.
Des impôts.

On a vu, dans le premier livre de ces Recherches, que le revenu particulier des individus provient, en dernier résultat, de trois sources différentes : la rente, les profits et les salaires. Tout impôt doit, en définitive, se payer par l’une ou l’autre de ces trois différentes sortes de revenus, ou par toutes indistinctement. Je tâcherai d’exposer, du mieux qu’il me sera possible, les effets, 1° de ces impôts qu’on a inten­tion de faire porter sur les rentes[6] ; 2° de ceux qu’on a intention de faire porter sur les profits ; 3° de ceux qu’on veut faire porter sur les salaires ; 4° de ceux qu’on veut faire porter indistinctement sur toutes ces trois différentes sources de revenu particulier.
L’examen séparé de ces quatre différentes espèces d’impôts divisera cette seconde section du présent chapitre en quatre articles, dont trois exigeront plusieurs autres subdivisions.
On verra, par l’examen qui va suivre, que plusieurs de ces impôts ne sont pas supportés, en définitive, par le fonds ou la source du revenu sur laquelle on avait eu l’intention de les faire porter.
Avant d’entrer dans l’examen de ces impôts en particulier, il est nécessaire de faire précéder la discussion par les quatre maximes suivantes sur les impôts en général.
Première maxime. Les sujets d’un État doivent contribuer au soutien du gouver­nement, chacun le plus possible en proportion de ses facultés, c’est-à-dire en propor­tion du revenu dont il jouit sous la protection de l’État.
La dépense du gouvernement est, à l’égard des individus d’une grande nation, comme les frais de régie sont à l’égard des copropriétaires d’un grand domaine, qui sont obligés de contribuer tous à ces frais à proportion de l’intérêt qu’ils ont respec­tivement dans ce domaine.
Observer cette maxime ou s’en écarter, constitue ce qu’on nomme égalité ou iné­galité dans la répartition de l’impôt. Qu’il soit, une fois pour toutes, observé que tout impôt qui tombe en définitive sur une des trois sortes de revenus seulement, est nécessairement inégal, en tant qu’il n’affecte pas les deux autres. Dans l’examen sui­vant des différentes sortes d’impôts, je ne reviendrai guère davantage sur cette espèce d’inégalité ; mais je bornerai le plus souvent mes observations à cette autre espèce d’inégalité qui provient de ce qu’un impôt particulier tombe d’une manière inégale même sur le genre particulier de revenu sur lequel il porte.
Deuxième maxime. La taxe ou portion d’impôt que chaque individu est tenu de payer doit être certaine, et non arbitraire. L’époque du payement, le mode du payement, la quantité à payer, tout cela doit être clair et précis, tant pour le contribuable qu’aux yeux de toute autre personne. Quand il en est autrement, toute personne sujette à l’impôt est plus ou moins mise à la discrétion du percepteur, qui peut alors ou aggraver la taxe par animosité contre le contribuable, ou bien, à la faveur de la crainte qu’a celui-ci d’être ainsi surchargé, extorquer quelque présent ou quelque gratification. L’incertitude dans la taxation autorise l’insolence et favorise la corruption d’une classe de gens qui est naturellement odieuse au peuple, même quand elle n’est ni insolente ni corrompue. La certitude de ce que chaque individu a à payer est, en matière d’imposition, une chose d’une telle importance, qu’un degré d’inégalité très-considérable, à ce qu’on peut voir, je crois, par l’expérience de toutes les nations, n’est pas, à beaucoup près, un aussi grand mal qu’un très-petit degré d’incertitude.
Troisième maxime. Tout impôt doit être perçu à l’époque et selon le mode que l’on peut présumer les moins gênants pour le contribuable. Un impôt sur la rente des terres ou le loyer des maisons, payable au même terme auquel se payent pour l’ordinaire ces rentes ou loyers, est perçu à l’époque à laquelle il est à présumer que le contribuable peut plus commodément l’acquitter, ou quand il est le plus vraisemblable qu’il a de quoi le payer. Tout impôt sur les choses consom­ma­bles qui sont des articles de luxe, est payé en définitive par le consommateur, suivant un mode de payement très-commode pour lui. Il paie l’impôt petit à petit, à mesure qu’il a besoin d’acheter ces objets de consommation. Et puis, comme il est le maître d’acheter ou de ne pas acheter ainsi qu’il le juge à propos, ce sera nécessairement sa faute s’il éprouve jamais quelque gêne considérable d’un pareil impôt.
Quatrième maxime.[7] Tout impôt doit être conçu de manière à ce qu’il fasse sortir des mains du peuple le moins d’argent possible au delà de ce qui entre dans le Trésor de l’État, et en même temps a ce qu’il tienne le moins longtemps possible cet argent hors des mains du peuple avant d’entrer dans ce Trésor. Un impôt peut, ou faire sortir des mains du peuple plus d’argent que ne l’exigent les besoins du Trésor public, ou tenir cet argent hors de ses mains plus longtemps que ces mêmes besoins ne l’exigent, de quatre manières, savoir : 1° la perception de l’impôt peut nécessiter l’emploi d’un grand nombre d’officiers dont les salaires absorbent la plus grande partie du produit de l’impôt, et dont les concussions personnelles établissent un autre impôt additionnel sur le peuple ; 2° l’impôt peut entraver l’industrie du peuple et le détourner de s’adonner à de certaines branches de commerce ou de travail, qui fourniraient de l’occupation et des moyens de subsistance à beaucoup de monde. Ainsi, tandis que d’un côté il oblige le peuple à payer, de l’autre il diminue ou peut-être anéantit quelques-unes des sources qui pourraient le mettre plus aisément dans le cas de le faire ; 3° par les confiscations, amendes et autres peines qu’encourent ces malheureux qui succombent dans les tentatives qu’ils ont faites pour éluder l’impôt, il peut souvent les ruiner et par là anéantir le bénéfice qu’eût recueilli la société de l’emploi de leurs capitaux. Un impôt inconsidérément établi offre un puissant appât à la fraude. Or, il faut accroître les peines de la fraude à proportion qu’augmente la tentation de frauder. La loi, violant alors les premiers principes de la justice, commence par faire naître la tentation, et punit ensuite ceux qui y succombent ; et ordinairement elle enchérit aussi sur le châtiment, à proportion qu’augmente la circonstance même qui devrait le rendre plus doux, c’est-à-dire la tentation de commettre le crime[8]. L’impôt, en assujettissant le peuple aux visites réitérées et aux recherches odieuses des percepteurs, peut l’exposer à beaucoup de peines inutiles, de vexations et d’oppressions ; et quoique, rigoureusement parlant, les vexations ne soient pas une dépense, elles équivalent certainement à la dépense aux prix de laquelle un homme consentirait volontiers à s’en racheter. C’est de l’une ou de l’autre de ces quatre manières différentes que les impôts sont souvent onéreux au peuple, dans une proportion infiniment plus forte qu’ils ne sont profitables au souverain.
La justice et l’utilité évidente des quatre maximes précédentes ont fait que toutes les nations y ont eu plus ou moins égard. Toutes les nations ont fait de leur mieux pour chercher à rendre leurs impôts aussi également répartis, aussi certains, aussi commodes pour le contribuable, quant à l’époque et au mode de payement, et aussi peu lourds pour le peuple, à proportion du revenu qu’ils rendaient au prince, qu’elles ont pu l’imaginer. L’examen qui suit, dans lequel nous passerons très-succinctement en revue quelques-uns des principaux impôts qui ont eu lieu en différents temps et en diffé­rents pays, fera voir que les efforts de toutes les nations à cet égard ne leur ont pas également bien réussi.

ARTICLE I.
Impôts sur les rentes de terres et loyers de maisons.
§I. Impôts sur les rentes de terres.

Un impôt sur le revenu territorial peut être établi d’après un cens fixe, chaque district étant évalué à un revenu quelconque, dont l’évaluation ne doit plus changer par la suite ; ou bien il peut être établi de manière à suivre toutes les variations qui peuvent survenir dans le revenu réel de la terre, c’est-à-dire, de manière à monter ou baisser avec l’amélioration ou le dépérissement de sa culture[9]. 
Un impôt territorial qui est établi, comme celui de la Grande-Bretagne, d’après un cens fixe et invariable, a bien pu être égal à l’époque de son premier établissement ; mais il devient nécessairement inégal dans la suite des temps, en conséquence des degrés inégaux d’amélioration ou de négligence dans la culture des différentes parties du pays. En Angleterre, l’évaluation d’après laquelle a été faite l’assiette de l’impôt territorial ou taxe foncière sur les différents comtés et paroisses, par l’acte de la quatrième année de Guillaume et Marie, a été fort inégal, même à l’époque de son premier établissement. À cet égard donc, cette taxe choque la première des quatre règles exposées ci-dessus ; elle est parfaitement conforme aux trois autres. Elle est on ne peut pas plus certaine. L’époque du payement de la taxe étant la même que celle du payement des rentes, est aussi commode qu’elle peut l’être pour le contribuable. Quoique le propriétaire soit, dans tous les cas, le vrai contribuable, la taxe est pour l’ordinaire avancée par le tenancier, auquel le propriétaire est obligé d’en tenir compte dans le payement de la rente ou fermage. Cette taxe est levée par un beaucoup plus petit nombre d’officiers que toute autre taxe rendant à peu près le même revenu. Comme cette taxe ne monte pas quand la rente vient à monter, le souverain n’a point de part dans le profit des améliorations faites par le propriétaire. Ces améliorations contribuent quelquefois, à la vérité, à soulager la cote des autres propriétaires du même district ; mais le surcroît de taxe que cette circonstance occasionnera quelque­fois sur une propriété particulière est toujours si peu de chose, qu’il ne peut jamais avoir pour effet de décourager les améliorations, ni de tenir le produit de la terre au-dessous du degré auquel il tend à s’élever. La taxe n’ayant aucune tendance à diminuer la quantité de ce produit, elle ne peut en avoir à en faire hausser le prix ; elle n’entrave nullement l’industrie du peuple ; elle n’assujettit le propriétaire à aucun autre inconvénient qu’à l’inconvénient inévitable de payer l’impôt.
Cependant, l’avantage qu’a retiré le propriétaire de cette constance invariable dans l’élévation sur laquelle toutes les terres de la Grande-Bretagne sont imposées à la taxe foncière, doit être principalement attribué à des circonstances tout à fait étrangères à la nature de la taxe.
Cet avantage est dû en partie à la grande prospérité du pays, dans presque toutes ses parties, les rentes de presque tous les biens-fonds de la Grande-Bretagne ayant été continuellement en augmentant, et presque aucune d’elles n’ayant baissé depuis l’époque où l’évaluation a été faite pour la première fois. Ainsi, les propriétaires ont presque tous gagné la différence d’entre la taxe qu’ils auraient eue à payer d’après la rente actuelle de leurs terres, et celle qu’ils payent à présent d’après l’ancienne évalua­tion. Si l’état du pays eût été différent, et que les rentes eussent été insensiblement en baissant en conséquence d’un dépérissement dans la culture, les propriétaires auraient presque tous perdu cette même différence. Dans l’état de choses qui s’est trouvé avoir lieu depuis la révolution, la constance de l’évaluation a été avantageuse au proprié­taire, et contraire à l’intérêt du Trésor public. Dans un état de choses différent, elle aurait été avantageuse au Trésor et contraire à l’intérêt du propriétaire.
Comme la taxe est payable en argent, de même l’évaluation de la terre est exprimée en argent. Depuis l’établissement de cette évaluation, la valeur de l’argent s’est maintenue d’une manière assez uniforme, et il n’y a eu aucune altération dans l’état de la monnaie, soit quant au poids, soit quant au titre. Si l’argent eût haussé considérablement de valeur, comme il parait avoir fait dans le cours des deux siècles qui ont précédé la découverte des mines de l’Amérique, la constance de l’évaluation se serait trouvée être fort dure pour le propriétaire. Si l’argent eût baissé considé­rablement de valeur, comme il a fait certainement pendant environ un siècle au moins après la découverte de ces mines, la même constance d’évaluation aurait extrêmement réduit cette branche du revenu du souverain. S’il avait été fait quelque changement considérable dans l’état des monnaies, soit en réduisant la même quantité d’argent à une dénomination plus basse, soit en l’élevant à une dénomination plus haute ; qu’une once d’argent, par exemple, au lieu d’être taillée en 5 schellings et 2 pence, eût été taillée en pièces dénommées seulement 2 schellings 7 pence, ou en pièces qu’on eût au contraire élevées, dans leur dénomination, jusqu’à 10 schellings 4 pence, le revenu du propriétaire y aurait perdu dans le premier cas, et celui du souverain dans le second.
Ainsi, dans des circonstances qui auraient différé en quelque chose de celles qui se trouvent avoir eu lieu, cette constance d’évaluation aurait pu entraîner de très-grands inconvénients, ou pour les contribuables, ou pour le revenu public. Cependant, dans la suite des temps, il faut bien qu’à une époque ou à l’autre ces circonstances arrivent. Or, quoique jusqu’à présent nous ayons vu que les empires n’étaient pas moins périssables que tous les autres ouvrages des hommes, cependant tout empire se flatte d’une durée éternelle. Ainsi, toute institution que l’on a établie pour être aussi permanente que l’empire lui-même, devrait être de nature à se prêter à toutes les circonstances, et non pas à certaines circonstances seulement ; ou bien elle devrait être appropriée à ces circonstances qui sont nécessaires et, par conséquent, sont toujours les mêmes, mais non pas à celles qui sont passagères, et qui sont l’effet du hasard ou des besoins du moment.
Cette classe de gens de lettres français, qui s’appellent économistes, vantent comme le plus équitable de tous les impôts un impôt sur le revenu des terres, qui suit toutes les variations du revenu, c’est-à-dire qui s’élève et qui baisse d’après l’amélio­ration ou le dépérissement de la culture. Tous les impôts, à ce qu’ils prétendent, retombent en dernière analyse sur le revenu de la terre, et doivent, par conséquent, être établis avec égalité sur le fonds qui doit définitivement les payer. Que tous les impôts doivent porter aussi également qu’il est possible sur le fonds qui doit définiti­vement les payer, c’est une vérité constante. Mais, sans entrer dans une discussion qui serait ici déplacée, de tous les arguments métaphysiques par lesquels ils soutiennent leur ingénieuse théorie, le coup d’œil suivant suffira pour faire voir quels sont les impôts qui tombent en définitive sur le revenu de la terre, et quels sont ceux qui tombent sur quelque autre source du revenu[10] 
Dans le territoire de Venise, toutes les terres labourables qui sont données à bail à des fermiers sont taxées au dixième de la rente[11]. Les baux sont enregistrés dans un registre public que tiennent les officiers du revenu dans chaque province ou district. Quand le propriétaire fait valoir lui-même ses terres, elles sont évaluées sur une juste estimation, et on lui accorde une déduction du cinquième de l’impôt, de manière que pour ces sortes de terres il paie seulement 8 au lieu de 10 pour 100 du revenu qu’on lui suppose.
Un impôt territorial de cette espèce est certainement plus égal que la taxe foncière d’Angleterre. Il pourrait peut-être n’être pas tout à fait aussi certain, et l’assiette de l’impôt pourrait souvent occasionner beaucoup plus d’embarras au propriétaire. La perception pourrait bien aussi en être beaucoup plus dispendieuse.
Cependant, il ne serait pas impossible d’imaginer un genre de régie capable de prévenir en grande partie cette incertitude, et qui amenât en même temps quelque modé­ration dans la dépense.
On pourrait, par exemple, obliger le propriétaire et le tenancier conjointement à faire enregistrer le bail dans un registre public. On pourrait porter des peines conve­nables contre tout déguisement ou fausse déclaration sur quelque clause du bail, et si une partie de l’amende était applicable à l’une ou à l’autre des deux parties qui aurait dénoncé et convaincu l’autre, pour cause de déguisement ou fausse déclaration de ce genre, une telle disposition produirait immanquablement l’effet de les empêcher de se concerter entre elles pour frauder le revenu public ; un tel enregistrement ferait connaître d’une manière suffisante toutes les clauses du bail.
Quelques propriétaires, au lieu d’augmenter le fermage, prennent un pot-de-vin ou deniers d’entrée au renouvellement du bail. Cette méthode est le plus souvent la ressource d’un prodigue qui vend, pour une somme d’argent comptant, un revenu futur d’une beaucoup plus grande valeur ; elle est donc, le plus souvent, nuisible au propriétaire ; elle est souvent nuisible au fermier, et est toujours nuisible à la société. Elle ôte souvent au fermier une si grande partie de son capital, et diminue tellement par là ses moyens de cultiver la terre, qu’il trouve plus de difficulté à payer une petite rente, qu’il n’en aurait eu sans cela à en payer une grosse. Tout ce qui diminue en lui les moyens de cultiver, tient nécessairement la partie la plus importante du revenu de la société au-dessous de ce qu’elle aurait été sans cela. En rendant l’impôt sur ces sortes de deniers d’entrée bien plus fort que sur les fermages ordinaires, on viendrait à bout de décourager cette pratique nuisible ; ce qui ferait l’avantage de toutes les différentes parties intéressées, du propriétaire, du fermier, du souverain et de toute la société.
Il y a certains baux où l’on prescrit au fermier un mode de culture, dans lesquels on le charge d’observer une succession particulière de récoltes pendant toute la durée du bail. Cette condition, qui est presque toujours l’effet de l’opinion qu’a le proprié­taire de la supériorité de ses propres connaissances (opinion très-mal fondée la plupart du temps), doit être regardée comme un surcroît de fermage, comme une rente en services, au lieu d’une rente en argent. Pour décourager cette pratique, qui, en général, est une sottise, on pourrait évaluer cette sorte de rente à quelque chose plus haut que les rentes ordinaires en argent et, par conséquent, l’imposer un peu davantage.
Quelques propriétaires, au lieu d’une rente en argent, exigent une rente en nature ; en grains, bestiaux, volailles, vin, huile, etc. D’autres aussi exigent une rente en services. De pareilles rentes sont toujours plus nuisibles au fermier qu’elles ne sont avantageuses pour le propriétaire. Elles ont l’inconvénient d’ôter au premier plus d’argent qu’elles n’en donnent à l’autre, ou au moins de tenir l’argent hors des mains du fermier, sans profit pour le propriétaire. Partout où elles ont lieu, les tenanciers sont pauvres et misérables, et précisément selon que cette pratique est plus ou moins générale. En évaluant de même ces sortes de rentes plus haut que les rentes ordinaires en argent et, par conséquent, en les taxant à quelque chose plus haut, on parviendrait peut-être à faire tomber un usage nuisible à la société.
Quand le propriétaire aime mieux faire valoir par ses mains une partie de ses terres, on pourrait évaluer d’après une estimation arbitrale faite par des fermiers et des propriétaires du canton, et on pourrait lui accorder une réduction raisonnable de l’impôt, comme c’est l’usage dans le territoire de Venise, pourvu que le revenu des terres qu’il ferait valoir n’excédât pas une certaine somme. Il est important que le propriétaire soit encouragé à faire valoir par lui-même une partie de sa terre. Son capital est généralement plus grand que celui du tenancier, et avec moins d’habileté il peut souvent donner naissance à un plus gros produit. Le propriétaire peut, sans se gêner, faire des essais et il est, en général, disposé à en faire. Une expérience qu’il aura faite sans succès ne lui cause qu’une perte modique. Celles qui lui réussissent contribuent à l’amélioration et à la meilleure culture de tout le pays. Il pourrait être bon cependant que la réduction de l’impôt ne l’encourageât à cultiver qu’une certaine étendue seulement de ses domaines. Si les propriétaires allaient, pour la plus grande partie, essayer de faire valoir par eux-mêmes la totalité de leurs terres, alors, au lieu de tenanciers sages et laborieux qui sont obligés, par leur propre intérêt, de cultiver aussi bien que leur capital et leur habileté peuvent le comporter, le pays se remplirait de régisseurs et d’intendants paresseux et corrompus, dont la régie pleine d’abus dégraderait bientôt le culture de la terre, et affaiblirait son produit annuel, non-seule­ment au détriment du revenu de leurs maîtres, mais encore aux dépens de la branche la plus importante du revenu général de la société.
Un pareil système d’administration dans un impôt de ce genre pourrait peut-être le dégager de toute incertitude capable d’occasionner quelque oppression ou quelque gêne au contribuable ; il pourrait servir en même temps à introduire, dans la méthode ordinaire de gouverner les terres, un plan ou une sorte de police très-capable d’accélérer dans le pays, d’une manière sensible, les progrès de l’amélioration et de la bonne culture[12].
Les frais de perception d’un impôt territorial variable à chaque variation de revenu seraient sans doute de quelque chose plus forts que ceux d’un impôt toujours établi sur une évaluation fixe. Il faudrait nécessairement quelque surcroît de dépense, tant pour les bureaux d’enregistrement qu’il serait à propos d’établir dans les différents districts du pays, que pour les évaluations successives qu’il faudrait faire de temps à autre, quant aux terres que les propriétaires préféreraient faire valoir par eux-mêmes. Néanmoins, tout ce surcroît de dépense pourrait être fort modique et fort au-dessous de celle qu’entraîne la perception de quantité d’autres impôts qui ne rendent qu’un reve­nu très-peu considérable en comparaison de celui qu’on pourrait espérer d’un impôt de ce genre.
L’objection la plus importante qui se présente, à ce qu’il semble, contre un impôt territorial aussi variable, c’est le découragement qu’il pourrait donner à l’amélioration des terres. Certainement, le propriétaire serait moins disposé à faire des améliorations quand le souverain, qui ne contribuerait en rien à la dépense, viendrait prendre part au profit de l’amélioration. On pourrait peut-être prévenir jusqu’à cette objection, en laissant au propriétaire, avant de commencer son amélioration, la faculté de faire constater, contradictoirement avec les officiers du revenu public, la valeur actuelle de sa terre, d’après l’évaluation arbitrale d’un certain nombre de propriétaires et de fer­miers du canton, également choisis par les deux parties, et en le taxant, confor­mément à cette évaluation, pour tel nombre d’années qu’on pourrait juger capable de l’indemniser complètement. Un des principaux avantages qu’on se propose dans l’établissement d’un impôt de ce genre, c’est d’attirer l’attention du souverain vers l’amélioration des terres, par la considération de l’augmentation qui en résulterait pour son propre revenu[13]. Par conséquent, l’abonnement accordé au propriétaire à titre d’indemnité ne devrait guère être beaucoup plus long qu’il ne serait nécessaire pour remplir cet objet, de peur que, l’intérêt du souverain se trouvant à un trop grand éloignement, son attention ne vînt à en être découragée. Il vaudrait pourtant mieux que le terme de cet abonnement fût de quelque chose trop long, plutôt que d’être le moins du monde trop court. Le motif d’aiguillonner l’attention du souverain ne saurait compenser, dans aucun cas, le moindre découragement donné à celle du propriétaire. L’attention du souverain ne peut jamais aller tout au plus qu’à une considération très-vague et très-générale de ce qui est le plus propre à contribuer à la meilleure culture de la majeure partie des terres de ses États. L’attention du propriétaire est une considération très-particulière et très-détaillée de tout ce qui est dans le cas de lui faire tirer le parti le plus avantageux de chaque pouce de terre dans son domaine. L’atten­tion principale du gouvernement, ce doit être d’encourager, par tous les moyens qui sont en son pouvoir, l’attention tant du propriétaire que du fermier, en les laissant l’un et l’autre chercher leur intérêt à leur manière et selon leur propre jugement ; en donnant à l’un et à l’autre la plus parfaite sécurité de jouir dans toute sa plénitude du fruit de leur industrie, et en procurant à l’un et à l’autre le marché le plus étendu pour chaque partie de leur produit, au moyen de l’établissement des communications les plus sûres et les plus commodes, tant par eau que par terre, dans toute l’étendue de ses États, aussi bien que de la liberté d’exportation la plus illimitée aux États de tous les autres princes.
Si, au moyen d’un pareil système d’administration, un impôt de ce genre pouvait être ménagé de manière non-seulement à n’apporter aucun découragement à l’amélio­ration des terres, mais au contraire à lui donner quelque degré d’encouragement, il ne paraît pas qu’il soit dans le cas d’entraîner avec lui aucune espèce d’inconvénient pour le propriétaire, excepté toujours l’inconvénient qui est inévitable, celui d’être obligé de payer l’impôt.
Au milieu de toutes les variations qu’éprouverait la société dans les progrès ou dans le dépérissement de son agriculture, au milieu de toutes les variations qui surviendraient dans la valeur de l’argent, ainsi que de celles qui auraient lieu dans l’état des monnaies, un impôt de ce genre s’ajusterait aussitôt lui-même, et sans qu’il fût besoin d’aucune attention de la part du gouvernement, à la situation actuelle des choses ; et au milieu de tous ces différents changements, il se trouverait toujours cons­tamment d’accord avec les principes de justice et d’égalité. Il serait donc beaucoup plus propre à être établi comme règlement perpétuel et inaltérable, ou comme ce qu’on appelle loi fondamentale de l’État, que tout autre impôt dont la perception serait à toujours réglée d’après une évaluation fixe[14]. 
Quelques États, au lieu de se servir de l’expédient simple et naturel de l’enre­gis­trement des baux, ont eu recours à la méthode pénible et dispendieuse d’un arpentage et d’une évaluation actuelle de toutes les terres du pays ; ils ont craint probablement que le preneur et le bailleur ne vinssent à se concerter ensemble pour cacher les clau­ses réelles du bail, dans la vue de frauder le fisc. Le grand cadastre d’Angleterre est, à ce qu’il semble, le résultat d’un arpentage général de ce genre, fait avec une très-grande exactitude.
Dans les anciens États du roi de Prusse, l’impôt territorial est assis d’après un arpentage et une évaluation actuelle, qu’on revoit et qu’on charge de temps en temps . Selon cette évaluation, les propriétaires laïques payent de 20 à 25 pour 100 de leur revenu ; les propriétaires ecclésiastiques de 40 à 45 pour 100. L’arpentage et l’éva­luation générale de la Silésie ont été faits par ordre du roi actuel et, à ce qu’on dit, avec beaucoup d’exactitude. Suivant cette évaluation, les terres appartenant à l’évêque de Breslau sont taxées à 25 pour 100 de leur revenu ; les autres revenus ecclésias­tiques des deux religions à 50 pour 100 ; les commanderies de l’ordre Teutonique et de l’ordre de Malte à 40 pour 100 ; les terres tenues en fief noble à 38 1/3 pour 100 ; celles tenues en roture à 35 1/3 pour 100.
L’arpentage et l’évaluation générale de la Bohême ont été, dit-on, l’ouvrage de plus de cent années. Cette opération ne fut terminée qu’après la paix de 1748, par les ordres de l’impératrice-reine actuelle . L’arpentage général du duché de Milan, qui fut commencé sous Charles VI, ne fut termine qu’après 1760. On le regarde comme une des opérations de ce genre les plus exactes et les mieux soignées qui aient jamais été faites. L’arpentage général de la Savoie et du Piémont a été exécuté par les ordres du feu roi de Sardaigne[15].
Dans les États du roi de Prusse, les revenus ecclésiastiques sont imposés beau­coup plus haut que ceux des propriétaires laïques. Le revenu de l’Église est, pour la plus grande partie, une charge sur les revenus des terres. Il arrive rarement qu’aucune partie en soit appliquée à l’amélioration de la terre, ou qu’elle y soit employée de manière à contribuer d’une façon quelconque à l’accroissement du revenu de la masse du peuple. Sa Majesté prussienne a vraisemblablement pensé, d’après cela, qu’il était raisonnable que ce revenu contribuât de quelque chose de plus que les autres au soulagement des besoins de l’État. Dans quelques pays, les terres de l’Église sont exemp­tes de tout impôt ; dans d’autres, elles sont imposées plus faiblement que les autres terres ; dans le duché de Milan, les terres que l’Église possédait avant 1575 sont taxées à l’impôt sur le pied de 1/3 seulement de leur valeur.
En Silésie, les terres de la noblesse sont taxées à 3 pour 100, plus que celles tenues en roture. Sa Majesté prussienne a vraisemblablement pensé que les honneurs et privilèges de différentes sortes attachés aux premières étaient pour le propriétaire une compensation suffisante d’une légère augmentation dans l’impôt, tandis qu’en même temps l’infériorité humiliante des dernières se trouverait en quelque sorte adou­cie par un avantage dans le taux de la taxation. Dans d’autres pays, au lieu d’adou­cir cette inégalité, le système d’imposition l’aggrave encore. Dans les États du roi de Sardaigne et dans ces provinces de France qui sont sujettes à ce qu’on appelle la taille réelle ou foncière, l’impôt porte entièrement sur les terres tenues en roture. Les terres de la noblesse en sont exemptes.
Un impôt territorial assis d’après un arpentage et une évaluation générale, quelque égal qu’il puisse être dans sa première assiette, doit nécessairement, dans le cours d’un espace de temps peu considérable, devenir inégal. Pour prévenir cette inégalité, il faudrait, de la part du gouvernement, une pénible et continuelle attention à toutes les variations qui peuvent survenir dans la valeur et dans le produit de chacune des différentes fermes du pays. Les gouvernements de Prusse, de Bohême de Sardaigne et du duché de Milan exercent actuellement une surveillance de ce genre ; surveillance si peu conforme à la nature d’un gouvernement, qu’il y a à présumer qu’elle ne sera pas d’une longue durée et que, si on la continue, elle occasionnera probablement à la longue beaucoup plus d’embarras et de vexations aux contribuables, qu’elle ne pourra jamais leur procurer de soulagement.
En 1666, l’assiette de la taxe réelle ou foncière de la généralité de Montauban fut faite d’après un arpentage et une évaluation qu’on dit avoir été très-exacts[16]. Vers 1727, cette assiette était devenue tout à fait inégale. Pour remédier à cet inconvénient, le gouvernement ne trouva pas de meilleur expédient que de réimposer sur toute la généralité un contingent additionnel de 120,000 livres. Ce nouveau contingent est réparti sur tous les différents districts sujets à la taille, d’après les bases de l’ancienne assiette ; mais on le lève seulement sur ceux qui, dans l’état actuel des choses, se trouvent imposés trop bas par la première assiette, et on l’applique au dégrèvement de ceux qui, par cette même assiette, se trouvent surimposés. Par exemple, deux dis­tricts, dont l’un, dans l’état actuel des choses, devrait être imposé à 900 livres, et l’autre à 1,100 livres, se trouvent, par l’ancienne assiette, imposés l’un et l’autre à 1,000 livres. Chacun de ces deux districts est réimposé, par le contingent additionnel, à 100 livres. Mais cette taxe additionnelle ne se lève que sur le district trop peu imposé, et elle s’applique en entier au soulagement du district trop imposé, qui par conséquent ne paie plus que 900 livres. Le gouvernement ne gagne ni ne perd à cette imposition additionnelle, qui est entièrement appliquée à remédier aux inégalités résultant de la première assiette. L’application est absolument réglée par l’intendant de la généralité et laissée à sa discrétion ; par conséquent, elle doit être en grande partie arbitraire.


Des impôts qui sont proportionnés au produit de la terre, et non au revenu du propriétaire.

Des impôts sur le produit de la terre sont, dans la réalité, des impôts sur la rente ou fermage ; et quoique l’avance en soit primitivement faite par le fermier, ils sont toujours supportés en définitive par le propriétaire. Quand il y a une certaine portion du produit à réserver pour l’impôt, le fermier calcule, le plus juste qu’il peut le faire, à combien pourra se monter, une année dans l’autre, la valeur de cette portion, et il fait une réduction proportionnée dans le fermage qu’il consent à payer au propriétaire. Il n’y a pas un fermier qui ne calcule par avance à combien pourra se monter, une année dans l’autre, la dîme ecclésiastique, qui est un impôt foncier de ce genre.
La dîme et tout autre impôt de ce genre sont, sous l’apparence d’une inégalité par­faite, des impôts extrêmement inégaux ; une portion fixe du produit étant, suivant la différence des circonstances, l’équivalent de portions très-différentes du revenu ou fermage. Dans certaines terres excellentes, le produit est si abondant qu’une moitié de ce produit suffit largement pour remplacer au fermier son capital employé à la culture, et encore les profits ordinaires qu’un capital ainsi placé rend dans le canton. L’autre moitié, ou, ce qui revient au même, le prix de l’autre moitié, est ce qu’il serait en état de payer au propriétaire comme rente ou fermage, s’il n’y avait pas de dîme. Mais si on vient à lui emporter pour la dîme un dixième de ce produit, il faut alors qu’il exige une réduction d’un cinquième sur le fermage, autrement il ne pourrait pas retirer son capital avec les profits ordinaires. Dans ce cas, la rente du propriétaire, au lieu de se monter à une moitié ou cinq dixièmes du produit total, ne montera qu’à quatre dixièmes de ce produit. Dans de mauvaises terres, au contraire, le produit est quelquefois si maigre et la dépense de culture si forte, qu’il faut quatre cinquièmes du produit total pour remplacer au fermier son capital avec le profit ordinaire. Dans ce cas, quand même il n’y aurait pas de dîme, le revenu du propriétaire ne monterait toujours pas à plus d’un cinquième ou de deux dixièmes du produit total. Mais si le fermier est tenu de payer pour la dîme un dixième du produit, il faut bien qu’il exige du propriétaire une réduction égale dans le fermage qu’il a à lui payer, lequel, par ce moyen, ne sera plus qu’un dixième seulement du produit de la terre. Sur le revenu des excellentes terres, la dîme peut quelquefois n’être qu’un impôt du cinquième seule­ment, ou de 4 schellings pour livre, tandis que sur celui des mauvaises terres elle peut être quelquefois un impôt de moitié ou de 10 schellings pour livre.
Si la dîme est le plus souvent un impôt très-inégal sur les revenus, elle est aussi tou­jours un très-grand sujet de découragement, tant pour les améliorations du proprié­taire que pour la culture du fermier. L’un ne se hasardera pas à faire les améliorations les plus importantes, qui, en général, sont les plus dispendieuses ; ni l’autre à faire naître les récoltes du plus grand rapport, qui en général aussi sont celles qui exigent les plus grands frais, lorsque l’Église, qui ne contribue en rien à la dépense, est là pour emporter une si grosse portion du profit. La dîme a été longtemps cause que la culture de la garance a été confinée aux Provinces-Unies, pays qui, étant presbytérien, et pour cette raison affranchi de cet impôt destructeur, a joui en quelque sorte, contre le reste de l’Europe, du monopole de cette drogue si utile pour la teinture. Les dernières tentatives qu’on a faites en Angleterre pour y introduire la culture de cette plante n’ont eu lieu qu’en conséquence du statut qui porte que 5 sch. par acre tiendront lieu de toute espèce de dîme quelconque sur la garance.
Dans plusieurs contrées de l’Asie, l’État, comme l’église dans la majeure partie de l’Europe, est entretenu principalement par un impôt territorial, proportionné au pro­duit de la terre et non pas au revenu du propriétaire. En Chine, le revenu principal du souverain consiste dans un dixième du produit de toutes les terres de l’empire. Cependant ce dixième est établi sur une évaluation tellement modérée, que dans la plupart des provinces on dit qu’il n’excède pas un trentième du produit ordinaire. -L’impôt ou redevance foncière qu’il était d’usage de payer au gouvernement maho­métan du Bengale, avant que ce pays tombât dans les mains de la compagnie anglaise des Indes Orientales, montait, à ce qu’on prétend, à un cinquième environ du produit. L’impôt territorial de l’ancienne Égypte montait pareillement, dit-on, à un cinquième.
On assure qu’en Asie cette espèce d’impôt territorial fait que le souverain prend intérêt à la culture et à l’amélioration des terres. Aussi les souverains de la Chine, ceux du Bengale, pendant que ce pays était sous le gouvernement des Mahométans, et ceux de l’ancienne Égypte, ont-ils toujours été, dit-on, extrêmement soigneux de faire faire et d’entretenir de bonnes routes et des canaux navigables, dans la vue d’aug­menter autant que possible tant la quantité que la valeur de chaque partie du produit de la terre, en procurant à chacune de ses parties le marché le plus étendu que leurs États lui pussent fournir. Mais la dîme de l’église est divisée en portions si petites, qu’aucun des décimateurs ne peut avoir un intérêt de ce genre. Le ministre d’une paroisse ne trouverait guère son compte à faire une route ou un canal dans un endroit du pays un peu éloigné, afin d’ouvrir un marché plus étendu au produit des terres de sa paroisse. Quand de pareils impôts sont destinés à l’entretien de l’église, ils entraînent avec eux autre chose que des inconvénients. 
Les impôts sur le produit des terres peuvent être perçus ou en nature, ou bien en argent, d’après une certaine évaluation.
Un ministre de paroisse, ou un propriétaire peu riche qui vit dans sa terre, peut trouver quelquefois certain avantage à recevoir en nature, l’un sa dîme, l’autre ses fermages. La quantité à recueillir est si petite, et le terrain sur lequel ils ont à recueil­lir est si borné, qu’ils peuvent bien l’un et l’autre surveiller par leurs yeux, dans tous leurs détails, la perception et la destination de ce qui leur revient. Mais un grand et riche propriétaire, vivant dans la capitale, courrait risque d’avoir beaucoup à souffrir de la négligence et encore plus de la malversation de ses agents et préposés, si on lui payait de cette manière les fermages de domaines situés dans des provinces éloignées. La perte que le souverain aurait à essuyer par les abus et les déprédations des percep­teurs de l’impôt serait encore nécessairement bien plus grande. Les domestiques du particulier le plus insouciant sont encore peut-être beaucoup plus sous les yeux de leur maître que les agents du prince le plus soigneux ne sont sous les siens. Et un revenu public payable en nature aurait tellement à souffrir de la mauvaise adminis­tration des collecteurs et régisseurs, qu’il n’arriverait jamais jusque dans le Trésor du prince qu’une très-faible partie de ce qui aurait été levé sur le peuple. On dit pourtant qu’en Chine une portion du revenu public se perçoit de cette manière. Les mandarins et les autres employés à la levée de l’impôt ne manqueront pas sans doute de trouver leur intérêt à laisser continuer une méthode de perception qui a tant d’avantages sur toute espèce de payement en argent, pour faciliter et courir les abus.
Un impôt sur le produit de la terre, qui se perçoit en argent, peut être perçu sur une évaluation qui varie avec toutes les variations du prix du marché, ou bien d’après une évaluation toujours fixe, un boisseau de blé froment, par exemple, étant toujours évalué au même prix en argent, quel que puisse être l’état du marché. Le produit de l’impôt, s’il est perçu de la première manière, ne sera sujet à d’autres variations que celles du produit réel de la terre, et à celles qui résultent de l’état de progrès ou de dépérissement de la culture. Mais si l’impôt est perçu de l’autre manière, alors son produit variera non-seulement avec les variations qui surviendraient dans le produit de la terre, mais encore avec celles qui pourraient survenir, tant dans la valeur des métaux précieux, que dans la quantité de ces métaux contenue, en différents temps, dans les monnaies d’une même dénomination. Le produit du premier de ces impôts sera toujours proportionné à la valeur du produit réel de la terre ; le produit du second pourra, en différents temps, être dans des proportions très-différentes avec cette valeur.
Quand, au lieu d’une certaine quotité du produit de la terre ou du prix d’une certaine quotité, on paie une somme fixe en argent pour tenir complètement lieu de tout impôt ou dîme, alors l’impôt devient, dans ce cas, précisément de même nature que la taxe foncière d’Angleterre. Il ne monte ni ne baisse avec le revenu de la terre ; il n’encourage ni ne décourage l’amélioration et la culture. Dans la plupart de ces parois­ses, qui payent ce qu’on appelle le modus ou abonnement pour tenir lieu de toute autre dîme, cette dîme est un impôt de ce genre. Dans le Bengale, sous le gou­ver­ne­ment mahométan, au lieu d’un prélèvement en nature du cinquième du produit, la perception avait lieu, dans la plupart des districts ou zemindarats du pays, d’après un abonnement semblable, qui était, à ce qu’on dit, très-modéré. Quelques-uns des facteurs de la Compagnie des Indes, sous prétexte de rétablir à sa vraie valeur le reve­nu public, ont changé, dans quelques provinces, cet abonnement en un payement en nature. Sous leur régime, un pareil changement doit, selon toute apparence, non-seulement décourager la culture, mais encore ouvrir de nouvelles sources aux abus déjà si multipliés dans la perception du revenu public ; aussi ce revenu est-il extrême­ment tombé au-dessous de ce qu’il était, à ce qu’on assure, quand la compagnie a commencé à en prendre la régie. Il se peut que les facteurs de la compagnie aient trouvé leur compte à un tel changement de perception, mais c’est vraisemblablement au détriment de l’intérêt de leurs maîtres et de celui du pays[17]. 

\$ II. Impôts sur les loyers de maisons.
On peut supposer le loyer d’une maison divisé en deux parties, dont l’une constitue proprement le loyer du bâtiment ou rente de la superficie ; l’autre s’appelle communément le loyer du sol ou rente du fonds de terre.
Le loyer du bâtiment est l’intérêt ou profit du capital dépense a construire la maison. Pour mettre le commerce d’un entrepreneur de bâtiments[18] au niveau de tous les autres commerces, il est nécessaire que ce loyer soit suffisant, premièrement, pour lui rapporter le même intérêt que celui qu’il aurait retiré de son capital en le prêtant sur de bonnes sûretés et, deuxièmement, pour tenir constamment la maison en bon état de réparation, ou, ce qui revient au même, pour remplacer, dans un certain espace d’années, le capital qui a été employé à la bâtir. Le loyer que rend un bâtiment, ou le profit ordinaire de l’argent placé en bâtiments, est donc réglé partout par le taux ordinaire de l’intérêt de l’argent. Si le taux de l’intérêt au cours de la place est à 4 p. 100, le revenu d’une maison qui, la rente du sol payée, rapporte 6 ou 6 1/2 pour 100 sur la totalité des dépenses de construction, peut bien être censé rendre à l’entre­preneur de la construction un profit suffisant. Quand le taux de l’intérêt est de 5 pour 100, il faut peut-être que ce revenu aille à 7 et 7 1/2 pour 100. S’il arrivait que le commerce d’un entrepreneur de maisons rapportât un profit beaucoup plus grand que celui-ci, à proportion de l’intérêt courant de l’argent, ce commerce enlèverait bientôt tant de capital aux autres branches de commerce, qu’il ramènerait ce profit à son juste niveau. S’il venait, au contraire, à rendre beaucoup moins, les autres commerces lui enlèveraient bientôt tant de capital, que le profit remonterait encore au niveau des autres.
Tout ce qui excède, dans le loyer total d’une maison, ce qui est suffisant pour rapporter ce profit raisonnable, va naturellement au loyer du sol ; et quand le pro­priétaire du sol et le propriétaire de la superficie sont deux personnes différentes, c’est au premier le plus souvent que se paie la totalité de cet excédent. Cette augmentation de loyer est le prix que donne le locataire de la maison pour quelque avantage de situation, réel ou réputé tel. Dans les maisons des champs situées à une certaine distance des grandes villes, et où il y a abondance de terrain libre pour construire, le loyer du sol n’est presque rien, ou n’est pas plus que ce que rendrait le fonds sur lequel est la maison, s’il était mis en culture. Dans les maisons de campagne voisines de quelque grande ville, ce loyer du sol est quelquefois beaucoup plus haut, et on paie souvent assez cher la beauté ou la commodité particulière de la situation. Les loyers du sol sont, en général, le plus haut possible dans la capitale, et surtout dans ces quartiers recherchés où il se trouve y avoir la plus grande demande de maisons, quelles que puissent être les causes de cette demande, soit raison de commerce et d’affaires, soit raison d’agrément et de société, ou simplement affaire de mode et de vanité.
Un impôt sur les loyers de maison, payable par le locataire, et proportionné au revenu total que rendrait chaque maison, ne pourrait pas influer, du moins pour longtemps, sur les revenus que rendent les superficies ou bâtiments. Si l’entrepreneur de constructions n’y trouvait pas le profit raisonnable qu’il s’attend à faire, il serait forcé de quitter le métier ; ce qui, faisant monter la demande de bâtiments, ramènerait en peu de temps le profit de ce commerce à son juste niveau avec le profit des autres. Un pareil impôt ne porterait pas non plus totalement sur le loyer du sol, mais il se partagerait de manière à tomber, partie sur l’habitant de la maison, partie sur le propriétaire du sol.
Par exemple, supposons qu’une personne juge que ses facultés lui permettent de dépenser pour son loyer 60 liv. par an, et supposons aussi qu’on vienne à établir sur les loyers un impôt de 4 sch. pour livre ou d’un cinquième, payable-par le locataire ; dans ce cas, un loyer de 60 liv. lui coûtera 72 liv. par an ; ce qui est 12 liv. de plus que ce qu’elle se croit en état de donner. Il faudra donc qu’elle se contente d’une maison moindre ou d’un logement de 50 liv. de loyer, lesquelles, jointes aux 10 liv. de surcroît qu’elle est obligée de payer pour l’impôt, lui compléteront la somme de 60 liv. par an, qui est la dépense qu’elle se juge en état de faire ; et à l’effet de payer l’im­pôt, elle renoncera en partie au surcroît de commodités que lui eût procuré une maison de 10 livres de plus de loyer. Je dis qu’elle y renoncera en partie, car il n’arrivera guère qu’elle soit obligée de renoncer en entier à ce surcroît de commodités, parce que, par une suite des effets de l’impôt, elle viendrait à bout de se procurer, pour ses 50 liv. par an, un meilleur logement qu’elle n’aurait pu l’avoir s’il n’y eût pas eu d’impôt. En effet, de même que cette sorte d’impôt, en écartant ce concurrent particulier, doit diminuer la concurrence pour les logements de 60 liv. de loyer, de même elle doit aussi diminuer la concurrence pour ceux de 50 liv., et de la même manière pour le logement de toute autre somme, excepté ceux du loyer le plus bas, pour lesquels elle augmentera au contraire la concurrence pendant quelque temps. Or, les loyers de chaque classe de logements pour laquelle la concurrence aura diminué, baisseront nécessairement plus ou moins. Cependant, comme aucune partie de cette réduction de loyer ne peut affecter, du moins pour un temps considérable, le loyer de la superficie, la totalité de la réduction doit nécessairement tomber, à la longue, sur le loyer du sol. Par conséquent, le payement final de cet impôt tombera en partie sur le locataire de la maison, qui, pour en payer sa part, aura été obligé de sacrifier une partie de sa commodité, et en partie sur le propriétaire du sol, qui, pour en payer sa part, sera obligé de sacrifier une partie de son revenu. Dans quelle proportion ce payement final se partagera-t-il entre eux ? C’est ce qui n’est peut-être pas très-facile à décider. Ce partage se ferait probablement d’une manière très-différente dans des circonstances différentes, et un impôt de ce genre, d’après des circonstances diffé­rentes, affecterait d’une manière très-inégale le locataire de la maison et le pro­priétaire du soi.
Ce serait entièrement de l’inégalité accidentelle avec laquelle ce partage viendrait à se faire, que procéderait l’inégalité avec laquelle un impôt de ce genre tomberait sur les différents propriétaires de terrains bâtis. Mais l’inégalité avec laquelle cet impôt tomberait sur les différents locataires de maisons, procéderait non-seulement de cette cause, mais encore d’une autre. Dans des degrés différents de fortune, la proportion entre la dépense qu’un particulier affecte à son loyer et sa dépense totale, n’est pas la même ; elle est probablement la plus forte possible dans le plus haut degré de fortune, elle va en diminuant successivement dans les degrés inférieurs, de manière qu’en général, dans le degré le plus bas de fortune, elle est la plus faible possible. Les premiers besoins de la vie font la grande dépense du pauvre. Il a de la difficulté à se procurer de la nourriture, et c’est à en avoir qu’il dépense la plus grande partie de son petit revenu. Le luxe et la vanité forment la principale dépense du riche, et un loge­ment vaste et magnifique embellit et étale, de la manière la plus avantageuse, toutes les autres choses du luxe et de vanité qu’il possède. Aussi un impôt sur les loyers tomberait, en général, avec plus de poids sur les riches, et il n’y aurait peut-être rien de déraisonnable dans cette sorte d’inégalité. Il n’est pas très-déraisonnable[19] que les riches contribuent aux dépenses de l’État, non-seulement à proportion de leur revenu, mais encore de quelque chose au-delà de cette proportion.
Quoiqu’à quelques égards le loyer des maisons ressemble au fermage des terres, il y a cependant un point dans lequel il en diffère essentiellement. Le fermage des terres se paie pour l’usage d’une chose productive[20] ; la terre qui le paie le produit. Le loyer des maisons se paie pour l’usage d’une chose non productive ; ni la maison, ni le terrain sur lequel elle est assise, ne produisent rien. Ainsi, la personne qui paie ce loyer doit le tirer de quelque autre source de revenu distincte et indépendante de la chose pour laquelle elle le paie. Il faut qu’un impôt mis sur les loyers, en tant qu’il tombe sur les locataires, soit tiré de la même source que le loyer lui-même ; il faut que ceux-ci le payent sur leur revenu, que ce revenu provienne de salaire de travail, ou de profit de capitaux ou de rente de terre. En tant que cet impôt tombe sur des locataires, il est du nombre de ces impôts qui ne portent pas sur une seule source de revenu, mais sur toutes les trois sources indistinctement, et il est, à tous égards, de même nature qu’un impôt sur toute autre espèce de denrée consommable. En général, il n’y a peut-être pas un seul article de dépense ou de consommation qui soit plus dans le cas de faire juger de la manière large ou étroite dont un homme règle sa dépense totale, que la quotité de son loyer. Un impôt proportionnel sur cet article de la dépense des particuliers rendrait peut-être plus de revenu qu’on n’a pu encore en retirer jusqu’à présent de la même source dans aucun pays de l’Europe. Si l’impôt, à la vérité, était très-fort, la plus grande partie des gens tâcheraient d’y échapper, autant qu’ils pour­raient, en se contentant de logements médiocres, et en tournant vers quelque autre objet la partie la plus considérable de leur dépense.
On pourrait aisément s’assurer avec assez d’exactitude de la quotité du loyer, au moyen d’une mesure de police du même genre que celle qui serait nécessaire pour s’assurer de la quotité ordinaire du revenu foncier. Les maisons non habitées ne devraient pas payer l’impôt. Un impôt sur ces maisons tomberait entièrement sur le propriétaire, qui serait ainsi impose pour une chose qui ne lui rapporterait ni commodité ni revenu. Les maisons habitées par le propriétaire devraient être impo­sées, non pas d’après la dépense qu’aurait pu coûter leur construction, mais d’après le loyer qu’elles seraient dans le cas de rapporter si elles étaient louées, arbitré à une juste évaluation. Si elles étaient imposées d’après la dépense qu’aurait coûtée leur construction, un pareil impôt de 3 ou 4 sch. pour livre, joint aux autres impôts, ruinerait presque toutes les grandes et riches familles de ce pays et, je crois, de tout autre pays civilisé. Quiconque examinera avec attention les différentes maisons de la ville et de la campagne de quelques-unes des plus grandes et des plus opulentes familles du royaume, trouvera qu’au taux de 6 1/2 seulement ou 7 pour 100 sur les frais originaires de construction, leur loyer est à peu près égal à la totalité du revenu net de leurs biens. C’est la dépense accumulée de plusieurs générations successives, appliquée à des objets d’une grande beauté et d’une grande magnificence à la vérité, mais d’une valeur échangeable extrêmement modique, à proportion de ce qu’ils ont coûté[21].
Le loyer du sol est encore un sujet plus propre à être imposé que le loyer des maisons. Un impôt sur le loyer du sol ne ferait pas monter le loyer des maisons. Il tomberait en entier sur le propriétaire du sol, qui agit toujours en monopoleur et qui exige le loyer le plus fort qu’il soit possible de donner pour l’usage de son terrain. Les offres sont plus ou moins fortes, selon que les concurrents sont plus riches ou plus pauvres, ou qu’ils sont en état de satisfaire, à plus ou moins de frais, la fantaisie qu’ils ont pour tel coin de terrain en particulier. Dans tout pays, le plus grand nombre de concurrents riches est dans la capitale, et c’est là aussi qu’on trouvera toujours les loyers du sol les plus forts. Comme un impôt sur les loyers du sol ne pourrait pas augmenter le moins du monde la richesse de ces concurrents, ils ne seraient pas vrai­semblablement pour cela disposés à payer plus cher l’usage du terrain. Il importerait fort peu que l’impôt dût être avancé par le locataire ou par le propriétaire du sol. Plus le locataire serait obligé de payer pour l’impôt, moins il serait d’humeur à payer pour le terrain, de manière que le payement définitif de l’impôt retomberait en entier sur le propriétaire du terrain. Le loyer du sol des maisons non habitées ne devrait pas payer d’impôt.
Les loyers du sol et des fermages ordinaires des terres sont une espèce de revenu dont le propriétaire jouit, le plus souvent, sans avoir ni soins ni attention à donner. Quand une partie de ce revenu lui serait ôtée pour fournir aux besoins de l’État, on ne découragerait par là aucune espèce d’industrie. Le produit annuel des terres et du travail de la société, la richesse et le revenu réel de la masse du peuple, pourraient toujours être les mêmes après l’impôt comme auparavant. Ainsi, les loyers du sol et les fermages ordinaires des terres sont peut-être l’espèce de revenu qui est le mieux dans le cas de supporter l’assujettissement à un impôt qui lui soit particulier.
Les loyers du sol semblent même, à cet égard, un sujet plus propre à supporter une imposition particulière que les fermages ordinaires des terres. Le fermage ordi­naire des terres est le plus souvent dû, en partie au moins, à l’attention et à la bonne administration du propriétaire. Un impôt très-lourd pourrait décourager beaucoup trop cette attention et cette bonne administration. Les loyers du sol, en tant qu’ils excèdent le revenu ordinaire des terres, sont dus en entier au bon gouvernement du souverain, qui, par la protection qu’il assure à l’industrie du peuple en général ou des habitants de quelque lieu particulier, les met d’autant plus en état de payer, pour le terrain sur lequel ils bâtissent, un prix supérieur à sa valeur réelle, ou bien d’offrir au propriétaire du sol plus que la compensation de la perte résultant de cet emploi de sa terre. Il est parfaitement raisonnable qu’un fonds qui doit s’on existence à la bonne conduite du gouvernement de l’État soit imposé d’une manière particulière, ou contribue de quel­que chose de plus que la plupart des autres fonds au soutien du gouvernement.
Quoiqu’en plusieurs différents pays de l’Europe il y ait eu des impôts établis sur les loyers de maisons, je ne sache cependant pas que dans aucun on ait jamais consi­déré les loyers du sol comme un sujet distinct et séparé d’imposition. Les auteurs des impôts ont apparemment trouvé de la difficulté à constater quelle portion de loyer devrait être considérée comme loyer du sol, et quelle portion devrait l’être comme loyer de la superficie ; il ne serait pourtant pas très-difficile, à ce qu’il semble, de distinguer ces deux parties du loyer l’une de l’autre.
Dans la Grande-Bretagne, le loyer des maisons est censé imposé dans la même pro­portion que les revenus fonciers, par l’impôt nommé taxe foncière annuelle. L’élé­vation d’après laquelle est faite l’assiette de chaque paroisse et district est toujours la même. Cette évaluation a été extrêmement inégale dans son origine, et continue toujours à l’être. Dans la plus grande partie du royaume, cet impôt porte toujours plus faiblement sur les loyers de maisons que sur les revenus purement fonciers. Dans un petit nombre de districts seulement, qui furent, dans l’origine, taxés très-haut, et dans lesquels les loyers de maisons ont baissé considérablement, on dit que la taxe foncière de 3 ou 4 sch. pour livre se trouve monter à la juste proportion du véritable loyer des maisons. Les maisons qui ne sont pas louées, quoique assujetties à l’impôt par la loi, en sont exemptées, dans la plupart des districts, par une faveur de ceux qui font la répartition ; et cette exemption occasionne quelquefois de légères variations dans la cote particulière de chaque maison, quoique le montant du contingent pour le district soit toujours le même. Les augmentations qui surviennent dans la quotité des loyers par de nouvelles constructions, par des réparations, etc., vont à la décharge du district ; ce qui occasionne encore d’autres variations dans la cote particulière des maisons individuelles.
Dans la province de Hollande[22], chaque maison est imposée à 2 1/2 pour 100 de sa valeur, sans avoir égard au loyer actuel qu’elle rend, sans même qu’on sache si elle est louée ou non. Il semble un peu dur d’obliger un propriétaire à payer un impôt pour une maison qui n’est pas louée, et dont il ne tire aucun revenu, surtout un impôt aussi lourd. En Hollande, où le cours de l’intérêt n’excède pas 3 pour 100, 2 1/2 pour 100 sur la valeur totale d’une maison doit faire le plus souvent plus d’un tiers du loyer de la superficie, peut-être même du loyer total du sol et de la superficie. À la vérité, l’estimation sur laquelle les maisons sont imposées, quoique très-inégale, est, à ce qu’on dit, toujours au-dessous de la valeur réelle. Quand une maison est rebâtie, amélio­rée ou agrandie, on fait une nouvelle estimation, et la maison est imposée en conséquence.
Ceux qui ont inventé les différents impôts établis en Angleterre sur les maisons en différents temps, semblent s’être imaginé qu’il y avait de grandes difficultés à consta­ter d’une manière passablement exacte la valeur réelle de chaque maison. Ils ont, en conséquence, réglé leur impôt d’après quelque circonstance plus en évidence, et telle qu’ils l’ont probablement jugée devoir être le plus souvent en proportion avec la valeur du loyer.
Le premier impôt de ce genre fut la taxe du fouage ou de 2 schellings par chaque feu. Pour constater combien il y avait de feux dans chaque maison, il était nécessaire que le collecteur de l’impôt en vît toutes les chambres. Cette visite désagréable jeta de l’odieux sur l’impôt. En conséquence, il fut aboli peu après la révolution, comme une tache de servitude.
L’impôt du même genre qui suivit celui-ci fut une taxe de 2 schellings sur chaque maison qui était habitée. Une maison ayant dix fenêtres payait 4 schellings de plus ; une maison à vingt fenêtres et au-delà, payait 8 schellings. Cette taxe fut ensuite changée, en ce que les maisons de vingt fenêtres et de moins de trente furent impo­sées à 10 schellings, et celles de trente fenêtres et au-delà, à 20 schellings. Le nombre des fenêtres peut se compter le plus souvent du dehors et, dans tous les cas, sans entrer dans toutes les chambres de la maison. Ainsi, la visite du collecteur fut moins choquante pour cet impôt que pour la taxe du fouage.
Cet impôt fut ensuite révoqué, et on établit à la place la taxe des fenêtres, qui a subi aussi plusieurs changements et augmentations. La taxe des fenêtres, telle qu’elle subsiste aujourd’hui (janvier 1775), outre le droit de 3 schellings sur chaque maison en Angleterre, et de 1 schelling sur chaque maison en Écosse, établit sur chaque fenêtre un droit qui, en Angleterre, augmente par degrés, depuis 2 pence, qui est le taux le plus bas pour les maisons qui n’ont pas plus de sept fenêtres, jusqu’à 2 schel­lings, qui est le taux le plus haut pour les maisons qui ont vingt-cinq fenêtres et au-delà[23].
La principale objection contre tous les impôts de cette espèce, c’est leur inégalité, et la pire de toutes les inégalités, puisqu’ils portent souvent avec plus de poids sur le pauvre que sur le riche. Une maison de 10 livres de loyer, dans une ville de province, peut quelquefois avoir plus de fenêtres qu’une maison à Londres, de 500 livres de loyer ; et quoiqu’il y ait à parier que le locataire de la première soit beaucoup moins riche que celui de l’autre, cependant, en tant que sa contribution aux charges de l’État est réglée par la taxe des fenêtres, celui-là contribuera plus que le dernier. Ces impôts sont donc directement contraires à la première des quatre maximes que nous avons établies ci-dessus. Ils ne paraissent pas beaucoup choquer aucune des trois autres.
La tendance naturelle de la taxe des fenêtres et de tous les autres impôts sur les maisons est de faire baisser les loyers. Plus un homme paie pour l’impôt, moins il est évidemment en état de payer pour le loyer. Cependant, depuis l’établissement de la taxe des fenêtres, les loyers de maisons ont au total augmenté, plus ou moins, dans presque chaque ville ou village de la Grande-Bretagne que j’ai pu observer. Tel a été presque partout l’effet de la demande toujours croissante de logements, qu’elle a fait monter les loyers plus que la taxe des fenêtres n’a pu les faire baisser ; preuve à ajouter à tant d’autres de la grande prospérité du pays et de l’accroissement du revenu de ses habitants. Si ce n’eût été la taxe, les loyers auraient monté vraisemblablement encore plus haut.

ARTICLE II.
Impôts sur le profit ou sur le revenu provenant de capitaux[24].

Le revenu ou profit qui provient d’un capital se divise naturellement en deux portions : 1° celle qui paie l’intérêt et qui appartient au propriétaire du capital ; 2° celle qui excède ce qui est nécessaire pour le payement de l’intérêt.
Cette dernière portion du profit ne peut évidemment être directement imposée ; elle est la compensation, et le plus souvent elle n’est rien de plus qu’une compensation très-modérée des risques et de la peine d’employer le capital. Il faut que celui qui emploie le capital ait cette compensation, autrement il ne peut, sans nuire à ses intérêts, en continuer l’emploi. S’il était donc imposé directement à proportion du profit total qu’il retire, il serait obligé, ou d’élever le taux de son profit, ou de rejeter sur l’intérêt de l’argent[25], c’est-à-dire de payer moins d’intérêt. S’il élevait le taux de son profit à proportion de l’impôt, alors, quoique l’impôt pût être avancé par lui, cepen­dant le payement définitif tomberait en entier sur l’une ou sur l’autre de deux classes de gens différentes, selon les différentes manières dont il emploierait le capital dont il a la direction. S’il l’employait, comme capital de fermier, à la culture de la terre, il ne pourrait faire hausser le taux de son profit qu’en retenant par ses mains une plus forte portion du produit de la terre, ou, ce qui revient au même, le prix d’une plus forte portion de ce produit ; et comme cela ne pourrait se faire qu’en réduisant le fermage, le payement définitif de l’impôt tomberait sur le propriétaire. S’il employait le capital comme capital de commerce ou de manufacture, il ne pourrait hausser le taux de son profit qu’en augmentant le prix de ses marchandises, auquel cas le payement final de l’impôt tomberait totalement sur les consommateurs de ses marchandises. En supposant qu’il n’élevât point le taux de son profit, il serait obligé de rejeter tout l’impôt sur cette portion du profit qui était destinée à payer l’intérêt de l’argent ; il rendrait moins d’intérêt pour tout ce qu’il aurait emprunté de capital et, dans ce cas, tout le poids de l’impôt porterait sur l’intérêt de l’argent. Tout l’impôt dont il ne pourrait pas se décharger d’une de ces manières, il serait obligé de s’en décharger de l’autre[26].
L’intérêt de l’argent paraît, au premier coup d’œil, un objet aussi susceptible d’être imposé directement que le revenu foncier. De même que le revenu foncier ou fermage de terre, c’est un produit net qui reste, toute compensation pleinement faite des risques et de la peine d’employer le capital. De même qu’un impôt sur le fermage des terres ne peut faire monter le fermage plus haut, parce que le produit net qui reste après le remplacement du capital du fermier, avec un profit raisonnable, ne peut pas être plus grand après qu’avant l’impôt, de même, et par la même raison, un impôt sur l’intérêt de l’argent ne pourrait hausser le taux de l’intérêt, si l’on suppose que la quantité de capitaux ou d’argent à prêter dans le pays reste la même après l’impôt qu’elle était auparavant, tout comme la quantité de terre. On a fait voir dans le premier livre[27] que le taux ordinaire du profit se détermine sur le rapport existant entre la quantité de capitaux à employer et la quantité d’emplois ou d’affaires qui exigent des capitaux. Or, la quantité d’emplois pour les capitaux, ou d’affaires qui en deman­dent, ne peut être augmentée ni diminuée par un impôt sur l’intérêt de l’argent. Si donc la quantité de capitaux à employer n’était ni augmentée ni diminuée par l’impôt, le taux ordinaire du profit resterait nécessairement le même. Or, la portion de ce pro­fit nécessaire pour indemniser de ses risques et de sa peine celui qui emploie le capital, resterait pareillement la même ; car ces risques et cette peine n’éprouveraient aucune sorte de changement. Par conséquent le résidu, cette portion qui appartient au propriétaire du capital et qui paie l’intérêt de l’argent, resterait nécessairement aussi le même. Ainsi, au premier coup d’œil, l’intérêt de l’argent paraît être un objet aussi propre à être imposé directement, que le revenu de la terre.
Il y a cependant deux différentes circonstances qui rendent l’intérêt de l’argent un sujet d’imposition directe, beaucoup moins convenable que le revenu de la terre.
Premièrement, la quantité et la valeur de la terre qu’un homme possède ne peuvent jamais être un secret, et peuvent toujours se constater avec une grande précision. Mais la somme totale de ce qu’il possède en capital est presque toujours un secret, et on ne peut guère s’en assurer avec une certaine exactitude ; elle est d’ailleurs sujette à varier presque à tout moment. Il ne se passe guère une année, souvent pas un mois, quelquefois pas un seul jour, sans qu’elle augmente ou diminue plus ou moins. Une inquisition sur la situation des affaires privées de chaque individu, et une inquisition qui, pour faire cadrer l’impôt avec cette situation, épierait toutes les fluctuations de sa fortune, serait une source si féconde de vexations continuelles et interminables, que personne au monde ne pourrait la supporter[28].
En second lieu, la terre est une chose qui ne peut s’emporter, tandis que le capital peut s’emporter très-facilement. Le propriétaire de terre est nécessairement citoyen du pays où est situé son bien. Le propriétaire de capital est proprement citoyen du monde, et il n’est attaché nécessairement à aucun pays en particulier. Il serait bientôt disposé à abandonner celui où il se verrait exposé à des recherches vexatoires qui auraient pour objet de le soumettre à un impôt onéreux, et il ferait passer son capital dans quelque autre lieu où il pourrait mener ses affaires et jouir de sa fortune à son aise. En emportant son capital, il ferait cesser toute l’industrie que ce capital entre­tenait dans le pays qu’il aurait quitté. C’est le capital qui met la terre en culture ; c’est le capital qui met le travail en activité. Un impôt qui tendrait à chasser les capitaux d’un pays tendrait d’autant à dessécher toutes les sources du revenu, tant du souverain que de la société. Ce ne seraient pas seulement les profits de capitaux, ce seraient encore la rente de la terre et les salaires du travail qui se trouveraient nécessairement plus ou moins diminués par cette émigration de capitaux.
Aussi, les nations qui ont essayé d’imposer le revenu provenant de capitaux ont été obligées, au lieu d’une inquisition rigoureuse de cette espèce, de se contenter d’une appréciation très-vague, et par conséquent, plus ou moins arbitraire. L’extrême inégalité et l’extrême incertitude d’un impôt aussi vaguement assis ne peuvent être compensées que par son extrême modération, en conséquence de laquelle chaque particulier se trouve taxé si fort au-dessous de son véritable revenu, qu’il ne s’inquiète guère que son voisin soit taxé de quelque chose encore plus bas.
Par l’impôt appelé, en Angleterre, la taxe foncière, ou a eu l’intention d’imposer les capitaux dans la même proportion que les terres. Quand la taxe sur les terres était à 4 schellings pour livre, ou au cinquième du revenu présumé, on a entendu que le capital serait imposé au cinquième de l’intérêt présumé ; lorsque la présente taxe foncière annuelle a été établie pour la première fois, le taux légal de l’intérêt était à 6 pour 100. En conséquence, chaque 100 livres de capital furent censées imposées à 24 schellings, la cinquième partie de 5 livres. Depuis que le taux légal de l’intérêt a été réduit à 6 pour 100, chaque 100 livres de capital sont censées être imposées seule­ment à 20 schellings. La somme d’impôt à lever par ce qu’on nomme la taxe foncière a été divisée entre les campagnes et les principales villes. La plus grosse partie de cette somme a été mise sur les campagnes ; et de celle qui a été mise sur les villes, la plus forte portion a été assise sur les maisons. Ce qui est resté à asseoir sur le capital ou commerce des villes (car on n’eut pas intention d’imposer le capital employé à la culture des terres) s’est trouvé fort au-dessous de la valeur réelle de ce capital ou de ce commerce. Ainsi, toutes les inégalités qui purent se rencontrer dans l’assiette primiti­ve ne donnèrent lieu à aucune plainte sensible. Chaque paroisse et district continue encore à être taxé pour ses terres, ses maisons et ses capitaux, selon l’assiette primitive ; et la prospérité presque universelle du pays, qui a extrêmement fait monter la valeur de toutes ces choses dans la plupart des endroits, a rendu ces inégalités d’une importance bien moindre aujourd’hui ; et puis, le contingent assigné à chaque district restant toujours le même, l’incertitude de cet impôt, en tant qu’il porte sur le capital du particulier, a été extrêmement diminuée, outre qu’elle est devenue d’une bien moindre conséquence. Si la majeure partie des terres d’Angleterre ne sont pas imposées à la taxe foncière pour la moitié de leur valeur actuelle, la majeure partie du capital de l’Angleterre est peut-être à peine imposée au cinquantième de sa valeur actuelle. Dans de certaines villes, la totalité de la taxe foncière est assise sur les maisons, comme à Westminster, où les capitaux et le commerce en sont affranchis. Il n’en est pas de même à Londres.
Dans tous les pays, on a évité avec grand soin toute recherche rigoureuse sur la situation des affaires privées des particuliers.
À Hambourg[29], chaque habitant est obligé de payer à l’État un quart pour 100 de tout ce qu’il possède ; et comme la richesse du peuple de Hambourg consiste principalement en capital, on peut considérer cet impôt comme un impôt sur les capitaux. Chacun se taxe soi-même, et met annuellement, en présence du magistrat, une certaine somme d’argent dans la caisse publique, en déclarant sur serment que cette somme est le quart pour 100 de tout ce qu’il possède, mais sans déclarer quel en est le montant, ou sans qu’on puisse lui faire aucune question sur cet article. Cet impôt passe pour être acquitté en général avec une grande fidélité. Dans une petite république, où le peuple a une confiance entière dans ses magistrats, où il est convaincu que l’impôt est nécessaire aux besoins de l’État et croit qu’il sera fidèlement appliqué à sa destination, on peut quelquefois s’en reposer, pour le payement de l’impôt, sur la bonne volonté et sur la bonne foi des contribuables. Cette pratique n’est pas particulière aux Hambourgeois.
Le canton d’Unterwald en Suisse est fréquemment ravagé par des orages et des inondations, et se trouve exposé par là à des dépenses extraordinaires. Dans ces occa­sions, le peuple s’assemble, et chacun déclare, dit-on, avec la plus grande sincérité, ce qu’il a de bien, afin d’être taxé en conséquence. À Zurich, la loi ordonne que, dans le cas de nécessité, chacun sera taxé à proportion de son revenu, dont il est obligé de déclarer le montant sous serment. Ils n’ont pas, à ce qu’on assure, la moindre défiance qu’aucun de leurs concitoyens ait l’intention de les tromper. À Bâle, le principal revenu de l’État provient d’un petit droit de douane sur les marchandises exportées. Tous les citoyens font serment de payer, à chaque trimestre, la totalité des impôts établis par la loi. On s’en rapporte à chaque marchand et même à chaque cabaretier, pour tenir lui-même un état des marchandises qu’il vend au-dedans et au-dehors du territoire. À la fin du trimestre, il envoie son état au trésorier, avec le montant de l’impôt calculé au bas. On n’a pas de soupçons que le revenu public souffre de cette confiance[30]. 
Dans ces cantons suisses, on ne regarde pas, à ce qu’il semble, comme une loi très-fâcheuse d’obliger chaque citoyen à déclarer publiquement, sous serment, le montant de sa fortune. À Hambourg, une telle loi passerait pour la plus dure possible. Des marchands engagés dans des entreprises de commerce hasardeuses tremblent tous à l’idée d’être obligés d’exposer à chaque instant le véritable état de leurs affaires. Ils sentent bien qu’il ne leur arriverait que trop souvent de voir par là leur crédit ruiné et leurs projets avortés. Des gens sages et économes, qui sont étrangers à tous les projets de ce genre, ne s’imaginent pas avoir besoin d’une pareille réticence.
En Hollande, bientôt après l’élévation du dernier prince d’Orange au stathoudérat, on établit sur la totalité de la fortune de chaque citoyen un impôt de 2 p. 100, ou un cinquantième denier, comme on l’appela. Chaque citoyen se taxa lui-même, et paya sa contribution de la même manière qu’à Hambourg ; en général on présume que l’impôt fut payé avec une grande fidélité. Le peuple avait à cette époque le plus grand attachement pour son nouveau gouvernement, qu’il venait d’établir par une insur­rection générale. L’impôt ne devait se payer qu’une seule fois, pour soulager l’État dans une occurrence particulière. Il est vrai qu’il était trop lourd pour être permanent. Dans un pays où le taux courant de l’intérêt n’excède guère 3 p. 100, un impôt de 2 p. 100 se monte à 13 schellings 4 deniers par livre du plus haut revenu net qu’on puisse tirer communément d’un capital. C’est un impôt que très-peu de gens seraient en état de payer sans prendre plus ou moins sur leurs capitaux. Dans une nécessité particu­lière, par un grand élan de patriotisme, le peuple peut faire un effort extraordinaire, et sacrifier même une partie de ses capitaux individuels pour soulager l’État. Mais il est impossible qu’il continue ce sacrifice pendant quelque temps ; et s’il le faisait, l’impôt le ruinerait bientôt si complètement, qu’il le réduirait tout à fait à l’impuissance de soutenir l’État.
L’impôt établi sur les capitaux en Angleterre par le bill de la taxe foncière, en le supposant même dans toute sa proportion avec le capital, n’a pas pour objet de retrancher sur le capital ni d’en prendre la moindre partie. On a seulement eu l’inten­tion de mettre sur l’intérêt de l’argent un impôt proportionné à celui sur le revenu des terres, de manière que, quand ce dernier impôt est à 4 schellings pour livre, l’autre soit aussi à 4 schellings pour livre. L’impôt de Hambourg et les impôts encore plus modérés d’Unterwald et de Zurich sont de même des impôts qu’on a voulu mettre, non sur les capitaux, mais sur l’intérêt ou le revenu net des capitaux. L’impôt de Hollande est un impôt qu’on a entendu mettre sur les capitaux mêmes[31].
Impôts qui portent particulièrement sur les profits de certains emplois

Dans quelques pays, on a établi des impôts extraordinaires sur les profits de capitaux, quelquefois sur ceux employés dans des branches particulières de com­merce, et quelquefois même sur ceux placés dans l’agriculture. 
En Angleterre, la taxe sur les colporteurs et marchands ambulants, celle sur les chaises à porteurs et carrosses de place, et celle que les cabaretiers payent pour une permission de vendre en détail de l’ale et des liqueurs spiritueuses, sont des impôts de la première espèce. Pendant la dernière guerre, on proposa une autre taxe de la même espèce sur les boutiques. La guerre ayant été, disait-on, entreprise pour la défense du commerce du pays, les marchands, qui doivent en recueillir le fruit, doivent con­tribuer à la soutenir. 
Cependant, un impôt mis sur les profits des capitaux employés dans une branche particulière de commerce ne peut jamais tomber en définitive sur le marchand, parce qu’il faut que celui-ci trouve, dans tous les cas, le profit raisonnable de son com­merce, et il ne peut guère avoir rien de plus que ce profit raisonnable, quand la concur­rence est libre. Mais un tel impôt retombe toujours sur le consommateur, qui est obligé de payer (et en général avec encore une surcharge), dans le prix de la marchandise, l’impôt qui a été avancé par le marchand.
Quand un impôt de ce genre est proportionné au commerce que fait le marchand, il est payé, en définitive, par le consommateur, et ne pèse en aucune manière sur le marchand. Quand il n’est pas ainsi proportionné, mais qu’il est le même sur tous les marchands, alors, quoiqu’il soit payé aussi en définitive par le consommateur, néan­moins il favorise les gros marchands et pèse sur les petits. La taxe de 5 schellings par semaine sur chaque carrosse de place[32], et de 10 schellings par an sur chaque chaise à porteurs (sous le rapport de l’avance que sont obligés de faire ceux qui tiennent ces sortes de voitures) est un impôt assez exactement proportionné à l’étendue de leur commerce respectif. Cet impôt ne favorise pas le gros marchand, et ne pèse pas d’une manière oppressive sur le petit. La taxe de 20 sch. par an pour une permission de vendre de l’ale, de 40 sch. pour celle de vendre des liqueurs spiritueuses, et de 40 sch. de plus pour la permission de vendre du vin, étant la même pour tous les détaillants, doit nécessairement donner quelque avantage au gros marchand, et peser sur le petit d’une manière un peu oppressive. Le premier doit trouver plus de facilité à se rembourser de l’impôt dans le prix de sa marchandise, que n’en trouve le dernier. Toutefois, la modicité de cet impôt rend cette inégalité d’une moindre importance, et il y a peut-être bien des personnes qui trouvent assez à propos qu’on ôte un peu aux petits cabarets l’envie de se multiplier. La taxe sur les boutiques devait, selon le projet, être la même sur toutes les boutiques ; elle n’aurait pu guère exister autrement. Il aurait été impossible de proportionner, avec un degré passable d’exactitude, la taxe d’une boutique à l’étendue du commerce qui s’y faisait, à moins de pousser les recherches à un point qui aurait été absolument insupportable dans un pays fibre. Si la taxe avait été considérable, elle aurait écrasé les petits marchands, et mis par force tout le commerce dans les mains des gros. La concurrence des premiers étant écartée, les derniers auraient joui d’un monopole dans leur commerce, et, comme les autres monopoleurs, ils se seraient bientôt ligués entre eux pour élever leurs profits beau­coup au-delà de ce qui eût été nécessaire pour le payement de la taxe. Le payement définitif de cette taxe, au lieu de tomber sur le maître de la boutique, serait retombé sur le consommateur, avec une surcharge considérable au profit du maître de la boutique. Ces raisons firent rejeter le projet de la taxe sur les boutiques, à la place de laquelle on établit le subside de 1759[33].
Ce qu’on appelle en France la taille personnelle est peut-être l’impôt le plus important qui soit levé dans aucun lieu de l’Europe sur les profits des capitaux placés dans l’agriculture.
Dans l’état de désordre où était l’Europe sous l’empire du gouvernement féodal, le souverain était obligé de se contenter d’imposer ceux qui étaient trop faibles pour se refuser au payement de l’impôt. Les grands seigneurs, quoique disposés à lui prêter secours dans des occasions particulières, n’entendaient pas s’assujettir à un impôt permanent, et il n’était pas assez fort pour les y contraindre. Les cultivateurs des terres, par toute l’Europe, étaient pour la plupart originairement des serfs. Dans la plus grande partie de l’Europe, ils furent affranchis peu à peu. Quelques-uns d’eux acquirent la propriété de certaines terres qu’ils tinrent en roture ou à autre titre servile, quelquefois relevant du roi, quelquefois relevant de quelque autre grand seigneur, comme en relevaient en Angleterre nos anciens tenants-par-copie[34]. D’autres, sans acquérir la propriété, obtinrent des baux à longs termes des terres qu’ils cultivaient pour leur seigneur, et par là ils se trouvèrent moins dépendants de lui. Il semble que l’orgueil des grands seigneurs ait vu d’un œil chagrin et envieux le degré d’indé­pendance et de prospérité auquel cette classe d’hommes était venue à bout de s’élever, et ils consentirent volontiers à ce qu’elle fût imposée par le souverain. Dans quelques pays, cet impôt fut borné aux terres qui étaient tenues en roture à titre de propriété, et dans ce cas la taille était appelée réelle. L’impôt territorial établi par le feu roi de Sardaigne, et la taille établie dans les provinces du Languedoc, de la Provence, du Dauphiné et de la Bretagne, dans la généralité de Montauban et dans les élections d’Agen et de Condom, ainsi que dans quelques autres districts de la France, sont des impôts sur des terres tenues en roture à titre de propriété. Dans d’autres pays, l’impôt fut établi sur les profits présumés de tous ceux qui tenaient à ferme ou à bail des terres appartenant à autrui, quelle que fût la nature de la terre, noble ou roturière, dans la personne du propriétaire, et dans ce cas la taille était appelée personnelle. Dans la plupart des provinces de France qu’on nomme pays d’élection, la taille est de ce genre. La taille réelle, n’étant imposée que sur une partie seulement des terres du pays, est nécessairement un impôt inégal, mais non toujours arbitraire, bien qu’il le soit dans quelques occasions. La taille personnelle étant un impôt qu’on entend proportionner aux profits d’une certaine classe de gens, profits sur lesquels on ne peut que conjecturer, l’impôt est à la fois inégal et arbitraire.
Le montant de la taille personnelle imposée en France en ce moment (1775), dans les vingt généralités qu’on nomme pays d’élection, est de 40 107 239 liv. 16 s. tournois[35]. La proportion selon laquelle cette somme est assise sur les différentes provinces varie d’année en année, d’après les rapports qui se font au conseil du roi de l’état bon ou mauvais de la récolte, aussi bien que d’autres circonstances qui peuvent augmenter ou diminuer dans ces provinces leurs moyens respectifs de payer l’impôt. Chaque généralité se divise en un certain nombre d’élections, et la proportion dans laquelle la somme imposée sur toute la généralité se partage entre les différentes élections varie pareillement d’une année à l’autre, d’après les rapports faits au conseil sur les moyens de payer de chacune de ces élections. Il paraît impossible qu’avec les meilleures intentions le conseil puisse jamais proportionner, avec une exactitude un peu tolérable, l’une ou l’autre de ces deux assiettes aux facultés réelles de la province ou district sur lequel elles sont établies. Le conseil le plus équitable sera toujours dans le cas de se méprendre par ignorance ou par défaut d’informations exactes. La propor­tion que chaque paroisse doit supporter dans le total du contingent de l’élection, et celle que chaque individu doit supporter dans le contingent particulier de sa paroisse, sont de même l’une et l’autre sujettes à varier d’une année à l’autre, d’après ce que les circonstances sont supposées exiger. Dans le premier cas, ce sont les officiers de l’élection qui jugent de ces circonstances ; dans le dernier cas, ce sont ceux de la paroisse, et ils sont les uns et les autres plus ou moins soumis à l’influence de l’auto­rité de l’intendant. Ces répartiteurs de l’impôt sont sujets, dit-on, à faire de fréquentes erreurs, non-seulement par ignorance et par défaut de bonnes informations, mais encore par complaisance, par esprit d’animosité et par d’autres motifs particuliers. Il est évident qu’aucun de ceux qui sont sujets à un pareil impôt ne peut jamais, avant que sa cote d’imposition soit faite, être certain de ce qu’il aura à payer. Il ne peut même en être certain après que sa cote a été réglée. Si l’on a taxé quelqu’un qui aurait dû être exempt, ou si quelqu’un a été taxé au-delà de la proportion dans laquelle il doit l’être, quoiqu’ils soient, dans ce cas, obligés l’un et l’autre de commencer toujours par payer, cependant s’ils se plaignent et parviennent à faire valoir leur réclamation, alors toute la paroisse est réimposée dans l’assiette de l’année prochaine pour les rembourser. Si quelqu’un des contribuables devient insolvable ou tombe en faillite, le collecteur est obligé de faire l’avance de la cote de ce contribuable, et l’année suivante on réimpose toute la paroisse pour rembourser le collecteur. Si le collecteur lui-même vient à faire banqueroute, la paroisse qui l’a choisi est obligée de répondre pour lui au receveur général de l’élection. Mais, comme il pourrait être trop embarrassant pour le receveur général d’avoir à contrôler toute la paroisse, il s’adresse, à son choix, à cinq ou six des plus riches contribuables, et les oblige à tenir compte des deniers perdus par l’insolvabilité du collecteur ; ensuite, pour rembourser ces cinq ou six habitants, on réimpose la paroisse. Ces réimpositions sont toujours en sus de la taille particulière de l’année dans laquelle on les établit.
Quand il se trouve un impôt établi sur les profits des capitaux dans une branche particulière de commerce, les marchands ont tous bien soin de ne pas mettre au marché plus de marchandises que ce qui peut s’y vendre à un prix suffisant pour les rembourser de l’avance de l’impôt. Quelques-uns d’eux retirent du commerce une partie de leurs capitaux, et le marché est moins garni qu’il ne l’était auparavant. Le prix de la marchandise vient à monter, et le payement définitif de l’ impôt retombe sur le consommateur. Mais, quand il y a un impôt établi sur les profits des capitaux placés dans l’agriculture, ce n’est pas l’intérêt des fermiers de retirer de cet emploi une partie de leurs capitaux. Chaque fermier tient une certaine quantité de terre pour laquelle il paie une rente ou fermage. Une certaine quantité de capital est nécessaire pour cultiver convenablement cette quantité de terre ; et si le fermier s’avisait de retirer une partie de ce capital nécessaire, il y a à parier qu’il ne serait pas par là plus en état de payer soit l’impôt soit le fermage. Pour suffire à payer l’impôt, ce ne peut jamais être son intérêt de diminuer de son produit, ni, par conséquent, de moins approvisionner le marché qu’auparavant. L’impôt ne le mettra donc jamais a même d’élever le prix de son produit de manière à se rembourser de l’impôt en en rejetant le payement définitif sur le consommateur. Il faut pourtant que le fermier, comme tout autre commerçant, ait son profit raisonnable, autrement il renoncerait à son métier. Après l’établissement d’un impôt de ce genre, il ne peut plus se procurer de profit raisonnable qu’en payant un moindre fermage au propriétaire. Plus il est obligé de payer en impôt, moins il est en état de payer en fermage. Un impôt de cette sorte, établi pendant le cours d’un bail, peut sans doute écraser, même ruiner le fermier ; mais, au renouvellement du bail, il faut toujours que l’impôt retombe sur le propriétaire.
Dans les pays où la taille personnelle existe, le fermier est ordinairement imposé à proportion du capital qu’il paraît employer à la culture ; c’est ce qui fait qu’il n’ose souvent avoir un bon attelage de chevaux ou de bœufs, mais qu’il tâche de cultiver avec les instruments de labour les plus chétifs et les plus mauvais possible ; il se défie tellement de la justice de ceux qui doivent l’imposer à la taille, qu’il fait semblant d’être pauvre, et qu’il cherche à paraître presque hors d’état de rien payer, dans la crainte d’être obligé de payer trop. Par cette misérable politique, il n’entend peut-être pas ses intérêts le mieux possible, et probablement il perd plus par la diminution du produit qu’il n’épargne par celle de l’impôt. Quoique, par une suite de cette méchante culture, le marché soit sans doute un peu plus mal pourvu, cependant la légère hausse de prix que cela pourrait occasionner, qui n’est pas même dans le cas de pouvoir indemniser le fermier de la diminution de produit, est encore bien moins dans le cas de lui donner le moyen de payer plus de fermage à son propriétaire. Le public, le fermier, le propriétaire, tous souffrent plus ou moins de cette culture dégradée. J’ai déjà eu occasion d’observer, dans le troisième livre de ces Recherches, que la taille personnelle tend, de mille manières différentes, à décourager la culture et, par consé­quent, à tarir la principale source de richesses de tout grand pays.
Ce qu’on appelle capitation dans les provinces de la partie méridionale de l’Amé­rique septentrionale et dans les îles des Indes occidentales, et qui est un impôt annuel de tant par tête de nègre, est proprement un impôt sur les profits d’une certaine espèce de capital employé en agriculture. Comme les planteurs sont à la fois, pour la plupart, fermiers et propriétaires, le payement définitif de l’impôt tombe sur eux en leur qualité de propriétaires, sans aucune répétition.
Les impôts de tant par tête sur les serfs employés à la culture ont été, à ce qu’il semble, autrefois communs dans toute l’Europe. Il subsiste actuellement dans l’empire de Russie un impôt de ce genre[36]. C’est probablement pour cette raison que les capita­tions de toute espèce ont souvent été représentées comme des signes de servitude. Cependant tout impôt est, pour la personne qui paie, un signe de liberté et non pas de servitude. Il marque que cette personne est soumise, à la vérité, à un gouvernement, mais aussi qu’elle a quelque propriété, et ne peut être elle-même par conséquent la propriété d’un maître. Une capitation sur des esclaves est totalement différente d’une capitation sur les hommes libres ; la dernière se paie par les personnes mêmes sur lesquelles elle est imposée ; l’autre se paie par une classe de personnes différente de celle qui est imposée. La dernière est entièrement arbitraire ou entièrement inégale, et le plus souvent elle est à la fois l’une et l’autre ; la première, quoique inégale à quelques égards, des esclaves différents étant de valeur différente, n’est nullement arbitraire. Tout maître qui sait le nombre de ses esclaves sait d’une manière précise ce qu’il a à payer. Ces deux genres différents d’impôt, étant appelés du même nom, ont été regardés comme de même nature.
Les taxes qui sont imposées en Hollande sur les domestiques mâles et femelles, sont des impôts sur les dépenses et non pas sur les capitaux, et à cet égard elles ressemblent aux impôts établis sur les choses de consommation. La taxe d’une guinée par tête par chaque domestique mâle, qui vient d’être établie dernièrement dans la Grande-Bretagne, est un impôt du même genre. Les personnes de la classe moyenne sont celles sur lesquelles elle pèse le plus. Un homme qui a 200 livres de rente peut avoir un domestique mâle ; mais un homme de 10 000 liv. de rente n’en aura pas cinquante. Elle ne touche point à la classe des pauvres.
Les impôts qui portent sur les profits de capitaux dans certains emplois en particulier ne peuvent jamais influer sur l’intérêt de l’argent. Personne ne voudra prêter à ceux qui exercent l’emploi sujet à l’impôt, à un intérêt moindre qu’à ceux qui exercent les emplois qui n’y sont pas sujets. Les impôts qui portent généralement sur les revenus provenant de capitaux dans tous les emplois, si le gouvernement cherche à les lever avec un certain degré d’exactitude, retomberont la plupart du temps sur l’intérêt de l’argent. Le vingtième, ou vingtième denier en France, est un impôt de même nature que ce qu’on appelle en Angleterre la taxe foncière, et il est de même assis sur les revenus provenant de terres, de maisons et de capitaux. Quoique, en ce qui concerne les capitaux, cet impôt ne soit pas assis avec une très-grande rigueur, cependant il l’est avec beaucoup plus d’exactitude que la partie de la taxe foncière d’Angleterre qui porte sur le même objet ; il tombe en entier, dans plusieurs circons­tances, sur l’intérêt de l’argent. On aliène souvent de l’argent en France par ce qu’on appelle contrat de constitution de rente, c’est-à-dire pour des annuités perpétuelles rachetables en tout temps par le débiteur, en remboursant par lui la somme origi­nairement avancée, mais dont le rachat n’est pas exigible par le créancier, si ce n’est dans de certains cas. Quoique le vingtième soit levé très-exactement sur toutes ces annuités, a ne paraît pas néanmoins qu’à en ait fait hausser le taux.

SUPLLÉMENT AUX ARTICLES I ET II.
Impôts sur la valeur capitale des terres, maisons et fonds mobiliers.

Tant qu’une propriété reste entre les mains du même possesseur, tous les impôts permanents dont elle peut être grevée, quels qu’ils soient, n’ont jamais pour objet de rien retrancher ni de rien prendre de sa valeur capitale ; ils ne sont qu’un prélèvement d’une partie du revenu qui en provient. Mais, quand la propriété vient à changer de mains, quand elle est transmise du mort au vif ou entre-vifs, on a souvent établi sur elle des impôts de nature à emporter nécessairement une partie de sa valeur capitale.
La transmission des propriétés de tout genre du mort au vif, et le transport entre-vifs des propriétés immobilières, comme terres et maisons, sont des actes qui, de leur nature, sont publics et notoires, ou qui ne peuvent rester longtemps secrets. Ces actes peuvent donc être imposés directement. Les transports de capitaux ou de propriétés mobilières faits entre-vifs pour des prêts d’argent, sont souvent des conventions cachées, et peuvent toujours être faits en secret. Il n’est donc pas aisé de les imposer directement. On les a imposés indirectement de deux manières différentes : la pre­mière, en exigeant que l’acte qui contient l’obligation de payer fût écrit sur du papier ou du parchemin qui eût acquitté un droit du timbre déterminé, sous peine de nullité de l’acte ; la deuxième, en exigeant, sous la même peine de nullité, que cet acte fût enregistré dans un registre public ou secret, et en imposant des droits sur cet enregis­trement. Les droits de timbre et ceux d’enregistrement ont souvent été établis de même sur les actes de transmission de propriétés immobilières entre personnes vivan­tes, transmissions cependant qu’il eût été facile d’imposer directement.
Le vingtième denier des successions ou vicesima hœreditatum, imposé par Auguste sur les Romains, était un impôt sur la transmission de propriété du mort au vif. Dion Cassius[37], l’auteur qui parle de cet impôt avec le moins d’obscurité, dit qu’il fut établi sur toutes les successions, legs et donations à cause de mort, excepté ceux faits aux plus proches parents ou aux pauvres.
L’impôt établi en Hollande sur les successions[38] est de même nature. Les succes­sions collatérales sont taxées, depuis 5 jusqu’à 30 pour 100 de toute la valeur de la succession, à raison de la proximité du degré de parenté. Les legs ou donations testamentaires à des collatéraux sont assujettis aux mêmes droits. Celles d’un mari à sa femme ou d’une femme à son mari sont taxées au 50e denier. La succession lugubre, luctuosa hœreditas, par laquelle les ascendants succèdent aux descendants, est taxée au 20e denier seulement. Les successions directes ou celles des descendants qui succèdent aux ascendants ne payent point de droits. La mort est, pour des enfants qui vivent dans la même maison que lui, un événement qui n’amène guère aucune augmentation de fortune, mais qui entraîne souvent une diminution considérable de revenu par la perte de son industrie, ou d’une charge dont il était revêtu, ou de quelque rente viagère dont il avait la jouissance. Un impôt qui aggraverait encore leur perte en leur enlevant une partie de sa succession serait cruel et oppressif. Cependant, il peut quelquefois en être autrement à l’égard des enfants qui sont ce qu’on appelle, dans le langage des lois romaines, émancipés, et dans celui des lois d’Écosse, établis hors de la famille, c’est-à-dire qui ont reçu leur portion, qui ont une famille à eux, et sont entretenus par des moyens distincts et indépendants de ceux de leur père. Tout ce qui reviendrait à ces enfants de la succession de leur père serait une véritable addition à leur fortune, et pourrait peut-être en conséquence, sans autre inconvénient que ceux qui sont inséparables de tous les droits de cette espèce, être assujetti à un impôt.

Les droits cruels établis par les lois féodales étaient des impôts sur la transmission des terres, tant du mort au vif qu’entre-vifs. Dans les anciens temps, ces droits consti­tuaient, par toute l’Europe, une des principales branches du revenu de la couronne.
L’héritier de tout vassal immédiat de la couronne payait un certain droit, en général une année de revenu, en recevant l’investiture du domaine. Si l’héritier était mineur, tous les revenus du domaine, tant que durait la minorité, étaient dévolus au seigneur, sans aucune autre charge que l’entretien du mineur et le payement du douaire de la veuve, quand il se trouvait qu’elle avait un assigné sur la terre. Quand le mineur arrivait à sa majorité, il était encore dû au seigneur un autre droit appelé relief, qui, en général, montait de même à une année de revenu. Une longue minorité, qui aujourd’hui donne les moyens d’éteindre toutes les charges d’un grand domaine et de rétablir une famille dans son ancien état de splendeur, ne pouvait pas alors avoir de pareils effets. La suite ordinaire d’une longue minorité était la ruine d’une grande terre, et non sa libération. 
Par la loi féodale, le vassal ne pouvait pas aliéner sans le consentement de son supérieur, qui, en général, exigeait un pot-de-vin ou une composition pour le donner. Ce pot-de-vin, qui était d’abord arbitraire, vint à être réglé, dans la plupart des pays, à une portion déterminée du prix de la terre. Dans quelques pays où la plus grande partie des autres coutumes féodales sont tombées en désuétude, cet impôt sur l’aliéna­tion des terres continue toujours de faire une branche considérable du revenu du souverain. Dans le canton de Berne, il se monte jusqu’au sixième du prix de tous les fiefs nobles, et au dixième de tous les biens en roture[39]. Dans le canton de Lucerne, l’impôt sur la vente des terres n’est pas universel, et il n’a lieu que dans certains districts. Mais, si une personne vend sa terre pour quitter le territoire, elle paie 10 pour 100 du prix de la vente[40]. Il existe dans beaucoup d’autres pays des droits du même genre, soit sur la vente de toutes les terres, soit sur la vente des terres seule­ment qui sont tenues à un certain titre, et ces droits forment une branche plus ou moins considérable du revenu du souverain.
Des conventions de ce genre peuvent être imposées indirectement par le moyen de droits de timbre ou de droits d’enregistrement, et ces droits peuvent être ou ne pas être proportionnés à la valeur de l’objet qui est transporté.
Dans la Grande-Bretagne, les droits de timbre sont plus ou moins forts, plutôt d’après la nature particulière de l’acte, que d’après la valeur de la chose transportée (car un papier timbré de 18 pence ou d’une demi-couronne suffira pour une obliga­tion, à quelque somme d’argent qu’elle se monte). Le plus fort droit n’excède pas 6 liv. sur chaque feuille de papier ou peau de parchemin, et ces gros droits portent principalement sur des dons et concessions de la couronne, et sur certains actes de procédure, sans aucun égard à la valeur de l’objet[41]. Il n’y a pas de droits en Grande-Bretagne sur l’enregistrement ; ce sont des officiers qui tiennent le registre, et ces vocations ne vont guère au-delà du juste salaire de leur travail. La couronne n’en retire aucun revenu.
En Hollande[42], il y a des droits de timbre et des droits d’enregistrement, qui sont, dans certains cas, proportionnés à la valeur de la propriété transportée, et ne le sont pas dans d’autres. Tous testaments doivent être écrits sur du papier timbré, qui coûte depuis 3 pence ou 3 stivers[43] la feuille, jusqu’à 300 florins, valant environ 27 livres 10 schellings de notre monnaie. Si le timbre du papier est d’un prix inférieur à celui dont le testateur aurait dû se servir, sa succession est dévolue au fisc. Ce droit de timbre se paie indépendamment de tous les autres impôts sur les successions. Excepté les lettres de change et quelques autres billets de commerce, tous autres actes, promesses et contrats sont assujettis au timbre. Ce droit cependant ne monte pas à proportion de la valeur de l’objet. Toutes ventes de terres ou de maisons, et toutes hypothèques sur les unes et les autres doivent être enregistrées, et payent à l’État, pour l’enregistrement, un droit de 2 1/2 p. 100 du montant du prix de l’hypothèque. Ce droit est étendu à la vente de tous vaisseaux et bâtiments du port de plus de deux tonneaux, pontés ou non pontés. On les considère apparemment comme des maisons sur l’eau. La vente des meubles, quand elle est ordonnée par une cour de justice, est assujettie à un droit de 2 1/2 p. 100.
En France, il y a des droits de timbre et des droits d’enregistrement. Les premiers sont regardés comme une branche des aides ou accise, et ils sont levés, dans les provinces où ces droits ont lieu, par les employés aux aides. Les derniers sont regar­dés comme une branche du domaine de la couronne, et ils sont levés par une classe d’employés.
Ces modes d’imposition, par droits de timbre et par droits d’enregistrement, sont d’une invention très-moderne. Cependant, dans le cours seulement d’un peu plus d’un siècle, les droits de timbre sont devenus presque universels en Europe, et les droits d’enregistrement sont devenus extrêmement communs. Il n’y a pas d’art qu’un gouver­nement apprenne plus tôt d’un autre, que celui de puiser l’argent dans les poches du peuple.
Les impôts sur les transmissions de propriété du mort au vif tombent, définiti­vement aussi bien qu’immédiatement, sur la personne à laquelle la propriété est transmise. Les impôts sur les ventes de terres tombent en totalité sur le vendeur ; le vendeur est presque toujours dans la nécessité de vendre, et dès lors obligé de prendre le prix qu’il peut avoir ; l’acheteur n’est presque jamais dans la nécessité d’acheter, et ne donne, par conséquent, que le prix qu’il lui plaît de donner ; il examine ce que la terre lui coûtera tant en achat qu’en impôts ; plus il sera obligé de payer comme impôt, moins il sera disposé à donner comme prix. De tels impôts tombent donc presque toujours sur une personne qui est déjà dans un état de nécessité, et ils doivent être sou­vent, par conséquent, durs et oppressifs. Les impôts sur la vente des maisons nouvel­lement bâties, quand la superficie est vendue sans le sol, tombent ordinai­rement sur l’acheteur, parce qu’il faut que l’entrepreneur de la construction ait en général son profit ; autrement il faudrait qu’il abandonnât le métier. Ainsi, si celui-ci avance l’impôt, il faut qu’il en soit remboursé par l’acheteur. Les impôts sur la vente des maisons anciennement bâties, par la même raison que ceux sur la vente des terres, tombent en général sur le vendeur, qui, le plus souvent, par arrangement d’affaires ou par nécessité, est obligé de vendre. Le nombre de maisons nouvellement bâties qui sont annuellement mises en vente, se règle plus ou moins sur la demande. À moins que la demande ne soit telle que l’entrepreneur de bâtiments trouve son profit, toutes les dépenses payées, il ne bâtira plus de maisons. Le nombre de maisons ancienne­ment bâties qui, en quelque temps que ce soit, se trouvent être à vendre, est déterminé par des circonstances accidentelles, dont la plus grande partie n’a pas de rapport avec la demande. Deux ou trois grandes banqueroutes dans une ville de commerce feront mettre au marché une quantité de maisons qu’il faudra vendre au prix qu’on pourra en avoir. Les impôts sur la vente des terrains à bâtir tombent en totalité sur le vendeur, par la même raison que ceux sur la vente des terres. Les droits de timbre et les droits d’enregistrement des promesses et contrats pour argent prêté, tombent en entier sur l’emprunteur et, dans le fait, ils sont toujours payés par lui. Les droits de la même espèce sur les actes de procédure tombent en entier sur les plaideurs ; ils réduisent, pour les deux parties, la valeur de l’objet en litige. Plus il vous en coûte pour acquérir une propriété, moins elle a nécessairement pour vous de valeur nette quand elle est acquise.
Tous les impôts établis sur les mutations de toute espèce de propriété, en tant qu’ils diminuent la valeur capitale de cette propriété, tendent à diminuer le fonds destiné à l’entretien du travail productif ; tous sont plus ou moins des impôts dissipa­teurs, entamant les capitaux de gens qui n’entretiennent que des ouvriers productifs, pour grossir le revenu du souverain, qui n’entretient guère que la classe non productive.
De tels impôts, même lorsqu’ils sont proportionnés à la valeur de la propriété transmise, sont toujours inégaux, la fréquence des mutations n’étant pas toujours la même dans des propriétés et des valeurs égales. Quand ils ne sont pas proportionnés à cette valeur (ce qui est le plus ordinaire pour la plupart des droits de timbre et d’enre­gistrement), ils sont encore bien plus inégaux ; ils ne sont à aucun égard arbitraires, et ils sont ou peuvent être, pour tous les cas, parfaitement clairs et certains. Quoiqu’ils tombent quelquefois sur une personne qui n’a pas beaucoup de moyens de payer, cependant l’époque du payement est le plus souvent assez commode pour elle ; le plus souvent, elle doit avoir de l’argent au moment où l’impôt est exigible. Ces impôts se lèvent à très-peu de frais, et en général ils n’assujettissent les contribuables à aucune autre incommodité au-delà de celle qui est toujours inévitable, celle de payer l’impôt.
En France, on ne se plaint pas beaucoup des droits de timbre ; on se plaint beau­coup de ceux d’enregistrement, qu’on y nomme contrôle. lis donnent lieu, à ce qu’on prétend, à quantité d’exactions de la part des employés de la ferme générale qui perçoivent cet impôt, arbitraire et incertain à beaucoup d’égards. Dans la plupart des écrits qui ont paru contre le système actuel des finances de France, les abus du contrôle forment un des principaux griefs. Cependant l’incertitude de la taxe n’est pas, à ce qu’il semble, un inconvénient qui soit essentiellement de la nature de ces sortes d’impôts. Si les plaintes du peuple sont bien fondées, il faut que les abus proviennent bien moins de l’impôt en lui-même, que du manque de clarté et de précision dans la teneur des édits ou des lois qui l’ont établi.
L’enregistrement des hypothèques et, en général, de tous droits sur les propriétés immobilières, donnant une grande sûreté aux créanciers et aux acquéreurs, est une formalité extrêmement avantageuse au public. Celui de la plupart des actes de tout autre genre est souvent incommode et même dangereux pour les particuliers, sans aucun avantage pour le public. Tous registres reconnus pour devoir rester secrets ne devraient jamais exister ; le crédit des particuliers ne devrait pas reposer sur une garantie aussi fragile que la probité et la discrétion des employés subalternes du revenu public. Or, partout où on a fait de la formalité de l’enregistrement une source de revenu pour le souverain, les employés à l’enregistrement ont été communément multipliés sans fin, tant pour les actes qui devaient être enregistrés, que pour ceux qui ne devaient pas l’être. En France, il y a plusieurs sortes différentes de registres secrets. Cet abus, s’il n’est pas, comme il faut en convenir, un effet nécessaire de ces sortes d’impôts, en est au moins un effet très-naturel.
Des droits de timbre, tels que ceux qui existent en Angleterre sur les cartes et les dés, sur les papiers-nouvelles et feuilles périodiques, etc., sont proprement des impôts sur la consommation ; le payement final tombe sur les personnes qui font consomma­tion ou usage de ces sortes de marchandises. Des droits de timbre, tels que ceux sur les permissions pour vendre en détail de l’ale, du vin et des liqueurs spiritueuses, quoiqu’on ait peut-être entendu les faire tomber sur les profits des détaillants, sont pareillement payés, en définitive, par ceux qui consomment ces liqueurs. Quoique ces sortes d’impôts portent le même nom que les- droits de timbre sur les mutations de propriété dont il est fait mention ci-dessus, et quoiqu’ils soient levés par les mêmes officiers et de la même manière, ils sont cependant d’une nature tout à fait différente, et portent sur des fonds absolument différents.

ARTICLE III.
Impôts sur les salaires du travail.

Deux circonstances différentes, comme j’ai tâché de le faire voir dans le premier livre, règlent partout nécessairement le salaire des ouvriers, savoir : la demande de travail, et le prix moyen ou ordinaire des denrées. La demande de travail, selon qu’elle se trouve aller en augmentant, ou rester stationnaire, ou aller en décroissant, règle différemment la nature de la subsistance du travailleur, et détermine le degré auquel cette subsistance sera ou abondante, ou médiocre, ou chétive. Le prix moyen et ordinaire des denrées détermine la quantité d’argent qu’il faut payer à l’ouvrier pour le mettre, une année dans l’autre, à même d’acheter cette subsistance abondante, mé­dio­cre ou chétive. Ainsi, tant que la demande de travail et le prix des denrées restent les mêmes, un impôt direct sur les salaires du travail ne peut avoir d’autre effet que de les faire monter de quelque chose plus haut que l’impôt. Supposons, par exemple, que dans un endroit particulier la demande de travail et le prix des denrées soient tels qu’ils portent le taux ordinaire des salaires du travail à 10 sch. par semaine, et que l’on vienne à mettre un impôt d’un cinquième ou de 4 sch. pour livre sur les salaires du travail. Si la demande de travail et le prix des denrées restaient les mêmes, il faudrait toujours nécessairement que l’ouvrier, dans cet endroit, gagnât une sub­sistance telle qu’elle ne pourrait pas s’acheter à moins de 10 sch. par semaine, ou bien que, l’impôt payé, il lui restât par semaine 10 sch. francs pour salaire. Mais pour lui laisser cette quotité de salaire après le payement de l’impôt que nous supposons, il faut que les salaires montent aussitôt dans cet endroit, non pas à 12 sch. seulement par semaine, mais à 12 sch. 6 den., c’est-à-dire que, pour le mettre à même de payer un impôt d’un cinquième, il faut nécessairement que ses salaires haussent aussitôt, non pas d’un cinquième seulement, mais d’un quart. Quelle que soit la proportion dans laquelle est établi l’impôt, dans tous les cas il est indispensable que les salaires haussent, non-seulement dans cette proportion, mais encore dans une proportion plus forte. Si l’im­pôt, par exemple, était d’un dixième, les salaires monteraient bientôt nécessaire­ment, non pas d’un dixième seulement, mais d’un huitième.
Ainsi, quand même un impôt direct sur les salaires du travail serait payé par les mains mêmes de l’ouvrier, on ne pourrait pas dire proprement qu’il fait l’avance de l’impôt, du moins si la demande de travail et le prix moyen des denrées restaient les mêmes après l’impôt qu’auparavant. Dans tous les cas d’une telle supposition, la personne qui met immédiatement l’ouvrier en œuvre serait obligée d’avancer, non-seulement l’impôt, mais quelque chose de plus que l’impôt. Le payement définitif retomberait sur des personnes différentes, selon la différence des circonstances. La hausse que l’impôt occasionnerait dans les salaires du travail des ouvriers de manu­facture serait avancée par le maître manufacturier, qui serait à la fois dans la nécessité et dans le droit de la reporter, avec un profit, sur le prix de ses marchan­dises. Ainsi, le payement définitif de ce surhaussement de salaire, y compris le profit additionnel du maître manufacturier, retomberait sur le consommateur. L’élévation qu’un tel impôt occasionnerait dans les salaires du travail de la campagne serait avancée par le fermier, qui serait obligé alors d’employer un plus gros capital pour entretenir le même nombre d’ouvriers qu’auparavant. Pour se rembourser de ce capital plus élevé, ainsi que des profits ordinaires des capitaux, il serait nécessaire qu’il retînt par ses mains une plus forte portion du produit de la terre, ou, ce qui revient au même, le prix d’une plus forte portion et, par conséquent, qu’il rendît moins de fermage au proprié­taire. Ainsi, dans ce cas, le payement définitif de cette élévation de salaire, y compris le profit additionnel du fermier qui l’aurait avancé, retomberait sur le propriétaire.
Dans tous les cas, un impôt direct sur les salaires du travail doit nécessairement occasionner à la longue une plus forte diminution dans la rente de la terre, et en même temps une plus grande élévation dans le prix des objets manufacturés, que n’en aurait pu occasionner d’une part ni de l’autre une autre imposition d’une somme égale au produit de cet impôt, qui aurait été convenablement assise, partie sur le revenu de la terre et partie sur les objets de consommation.
Si les impôts directs sur les salaires du travail n’ont pas toujours occasionné dans ces salaires une hausse proportionnée, c’est parce qu’ils ont, en général, occasionné une baisse considérable dans la demande de travail. Le déclin de l’industrie, la dimi­nution des moyens d’occupation pour le pauvre, et le décroissement du produit annuel des terres et du travail du pays, sont en général les effets qu’ont amenés de pareils impôts. Cependant, par une suite de ces impôts, le prix du travail doit toujours être plus haut qu’il ne l’eût été sans eux, dans l’état actuel de la demande ; et cette éléva­tion de prix, y joignant le profit de ceux qui en font l’avance, doit toujours être payée en définitive par les propriétaires et les consommateurs[44].
Un impôt sur les salaires des travaux de campagne ne fait pas hausser le prix du produit brut de la terre en proportion de l’impôt, par la même raison qu’un impôt sur les profits du fermier ne fait pas hausser ce prix dans cette proportion.
Tout absurdes cependant, tout destructifs que sont de tels impôts, ils ont lieu dans plusieurs pays. En France, cette partie de la taille qu’on impose sur l’industrie des ouvriers et journaliers dans les villages, est proprement un impôt de cette espèce. On compte leurs salaires selon le taux commun du district où ils demeurent ; et, afin qu’ils soient le moins possible exposés à une surcharge, on évalue leur gain annuel sur le pied de deux cents jours ouvrables seulement dans l’année[45]. La cote de chaque indi­vidu est changée d’une année à l’autre, suivant les différentes circonstances qui peu­vent survenir, desquelles est juge le collecteur ou le commissaire que nomme l’inten­dant pour l’assister. En Bohême, en conséquence du changement qui a com­mencé en 1747, dans le système des finances de ce royaume, il y a un impôt extrêmement lourd sur l’industrie des gens d’arts et métiers. Ils sont divisés en quatre classes. La première paie 100 florins par an, ce qui, à 22 den. 1/6 le florin, monte à 9 liv. 7 sch. 6 den. ; la seconde classe est taxée à 70 ; la troisième à 50, et la quatrième, qui comprend les artisans des villages et la plus basse classe de ceux des villes, à 25 florins[46].
Quant aux artisans et aux personnes qui exercent des professions libérales, le prix de leur travail garde nécessairement, comme j’ai cherché à le faire voir dans le livre 1er, une certaine proportion avec les gains des métiers inférieurs. Ainsi, un impôt sur la récompense d’un tel travail ne pourrait avoir d’autre effet que de la faire monter de quelque chose plus haut qu’en proportion de l’impôt. S’il ne la faisait pas monter ainsi, alors les arts de génie et les professions libérales, ne se trouvant plus à leur niveau relativement à tous les autres métiers et professions, seraient tellement abandonnés, qu’ils remonteraient bientôt à ce niveau.
Les émoluments des charges, offices et places de faveur ne sont pas, comme ceux des métiers et professions, réglés par l’effet de la libre concurrence du marché et, par conséquent, ils n’observent pas toujours une juste proportion avec ce qu’exige la nature de l’emploi. Dans la plupart des pays, ils sont peut-être plus hauts que ce qu’elle exige, attendu que les personnes qui ont l’administration du gouvernement sont, en général, disposées à se récompenser elles-mêmes, ainsi que tous ceux qui sont sous leur dépendance immédiate, plutôt au-delà que dans la juste mesure. Ainsi, les émoluments des places et offices peuvent fort bien, le plus souvent, supporter une imposition particulière. D’ailleurs, les personnes qui remplissent les emplois et char­ges publiques, principalement celles qui ont les places les plus lucratives, sont, dans tous les pays, les objets de l’envie générale ; et un impôt sur leurs émoluments, quand même il serait un peu plus fort que sur toute autre espèce de revenu, est toujours un impôt très-bien vu du peuple. En Angleterre, par exemple, lorsque toute autre espèce de revenu était censée imposée par la taxe foncière à 4 schellings pour livre, ce fut une mesure très-populaire que d’établir un impôt bien réellement de 5 schellings 6 deniers par livre sur les traitements des places et offices excédant 100 livres par année, excepté les pensions des branches cadettes de la famille royale, la paie des officiers de terre et de mer, et quelques autres emplois moins sujets à être exposés à l’envie. Il n’y a pas, en Angleterre, d’autre impôt direct sur les salaires du travail.

ARTICLE IV.
Impôts qu’on a intention de faire porter indistinctement sur toutes les différentes espèces de revenus.

Les impôts qu’on a intention de faire porter indistinctement sur toute espèce de revenu, ce sont les impôts de capitation et les impôts sur les objets de consommation. Il faut que ’ces impôts soient indistinctement payés par les revenus quelconques que peuvent posséder les contribuables, par la rente de leurs terres, par les profits de leurs capitaux, ou par les salaires de leur labeur.

§I. Impôts de capitation.

Les impôts de capitation deviennent entièrement arbitraires, si on essaie de les proportionner à la fortune ou au revenu de chaque contribuable. L’état de la fortune d’un particulier varie d’un jour à l’autre ; et à moins d’une inquisition plus insuppor­table que quelque impôt que ce puisse être, et renouvelée au moins une fois chaque année, il n’est pas possible de faire autre chose que de l’apprécier par conjecture. Ainsi, l’assiette d’un tel impôt doit donc le plus souvent dépendre des dispositions bonnes ou mauvaises de ceux qui la font et, par conséquent, il doit être totalement arbitraire et incertain.
Si l’impôt de capitation est assis, non dans la proportion de la fortune présumée, mais dans celle du rang du contribuable, alors il devient entièrement inégal, les de­grés de fortune étant souvent inégaux à égalité de rang.
Ainsi un pareil impôt, quand on veut essayer de le rendre égal, devient totalement incertain et arbitraire ; et quand on veut essayer de le rendre certain et hors de l’arbi­traire, il devient tout à fait inégal. Que l’impôt soit léger ou qu’il soit lourd, l’incer­ti­tude de ce qu’on a à payer est toujours une chose dure. Si l’impôt est léger, on peut bien supporter un certain degré d’inégalité ; mais l’inégalité sera absolument insup­portable si l’impôt est lourd.
Dans les différents impôts par tête qui eurent lieu en Angleterre sous le règne de Guillaume III, les contribuables furent taxés, pour la plupart, selon leur rang, comme ducs, marquis, comtes, vicomtes, barons, écuyers, simples gentilshommes, les aînés et cadets des pairs, etc. Tous les marchands en boutique et gens de métier ayant plus de 300 livres de bien, c’est-à-dire les plus distingués de cette classe, furent soumis à une même taxe, quelque grande que pût être la différence entre leurs facultés. On regarda plus à leur état qu’à leur fortune. Plusieurs de ceux qui, dans le premier impôt de ce genre, avaient été taxés selon leur fortune présumée, furent ensuite taxés selon leur état. Les avocats, procureurs et mandataires judiciaires qu’on avait taxés, dans la première assiette de cet impôt, à 3 schellings par livre de leur revenu présumé, furent ensuite taxés comme simples gentilshommes ou gens vivant noblement. Dans l’assiette d’un impôt qui n’était pas fort lourd, on a trouvé qu’un certain degré d’inégalité était plus aisé à supporter que le moindre degré d’incertitude.
Dans la capitation qui a été levée en France sans aucune interruption depuis le commencement du siècle présent, les classes les plus élevées sont taxées, selon leur rang, sur un tarif invariable, et les dernières classes selon leur- fortune présumée, et par une assiette qui varie d’une année à l’autre. Les officiers de la maison du roi, les juges et autres officiers des cours supérieures de justice, les officiers militaires, etc., sont taxés de la première manière. Les classes inférieures du peuple dans les pro­vinces sont taxées de la seconde manière. En France, les grands se soumettent sans peine à un certain degré d’inégalité dans un impôt qui, à leur égard, n’est pas fort lourd ; mais ils ne pourraient pas supporter d’être imposés arbitrairement par un inten­dant. Dans ce pays, les classes inférieures du peuple sont bien obligées de souffrir patiemment les formes que leurs supérieurs jugent à propos de leur prescrire.
En Angleterre, les différents impôts par tête n’ont jamais rendu la somme qu’on en avait attendue, ou qu’on a supposé qu’ils auraient produite s’ils eussent été levés exactement. En France, la capitation rend toujours le produit qu’on s’attend à en retirer. Quand le gouvernement doux de l’Angleterre a fait sur les diverses classes du peuple l’assiette d’un impôt par tête, il s’est contenté de ce que cette assiette s’est trouvée avoir produit, et il n’a exigé aucune compensation pour la perte que l’État avait à essuyer par le fait de ceux qui ne pouvaient pas payer, ou de ceux qui ne voulaient pas payer ; car il y en avait beaucoup de ce nombre, et qui, par l’indulgence qu’on mettait dans l’exécution de la loi, n’étaient pas contraints au payement. Le gouvernement de France, qui est plus sévère, impose à chaque généralité une certaine somme qu’il faut que l’intendant trouve comme il pourra. Si une province se plaint d’être surtaxée, elle peut obtenir, dans l’assiette de l’année suivante, une réduction proportionnée à la surcharge de l’année précédente ; mais il faut toujours payer en attendant. Pour que l’intendant fût sûr de trouver dans sa généralité la somme à laquelle elle est taxée, il a été autorisé à l’imposer à une plus forte somme ; de manière à ce que les non-valeurs résultant du défaut de payement ou de manque de facultés de quelques-uns des contribuables, pussent être compensées par la surcharge des autres ; et jusqu’en 1765, la fixation de cette charge supplémentaire a été entièrement laissée à sa discrétion. À la vérité, cette année-là, le conseil se ressaisit de ce pouvoir. L’auteur des Mémoires sur les impositions de la France, qui a écrit d’après d’excellentes infor­mations, observe que dans la capitation des provinces, la portion qui tombe sur la noblesse et sur les privilégiés exempts de taille est la moins considérable ; la plus forte portion tombe sur les personnes sujettes à la taille, qui sont imposées à la capitation à tant par livre de ce qu’elles payent pour cet autre impôt.
Les impôts de capitation, pour ce qui s’en lève sur les classes inférieures du peu­ple, sont des impôts directs sur les salaires du travail, et ils entraînent à leur suite tous les inconvénients résultant de cette nature d’impôt.
Les impôts de capitation se perçoivent à peu de frais ; et quand ils sont exigés à la rigueur, ils rapportent à l’État un revenu très-assuré. C’est pour cette raison que les impôts de capitation sont très-ordinaires dans les pays où l’on fait peu de cas du bien-être, de la tranquillité et de la sécurité des classes inférieures du peuple. Néanmoins, un grand empire n’a jamais retiré de ces sortes d’impôts qu’une petite partie de son revenu public, et les plus grosses sommes qu’ils aient jamais rendues auraient pu être levées de quelque autre manière moins incommode pour le peuple.

§II. Impôts sur les objets de consommation.

Il paraît que c’est l’impossibilité d’imposer le peuple par une capitation propor­tionnée au revenu de chaque contribuable, qui a fait imaginer les impôts sur les objets de consommation. L’État, ne sachant comment faire pour imposer le revenu de ses sujets directement et dans de justes proportions, tâche de l’imposer indirectement en mettant un impôt sur les dépenses, parce qu’on suppose que ces dépenses pour chaque particulier seront le plus souvent, à très-peu de chose près, proportionnées à son revenu. On impose les dépenses en imposant les objets de consommation qui font la matière de ces dépenses.
Les objets de consommation sont de nécessité, ou de luxe.
Par objets de nécessité, j’entends non-seulement les denrées qui sont indispensa­ble­ment nécessaires au soutien de la vie, mais encore toutes les choses dont les hon­nêtes gens, même de la dernière classe du peuple, ne sauraient décemment manquer, selon les usages du pays. Par exemple, une chemise, strictement parlant, n’est pas une chose nécessaire aux besoins de la vie. Les Grecs et les Romains vivaient, le pense, très-commodément, quoiqu’ils n’eussent pas de linge. Mais aujourd’hui, dans presque toute l’Europe, un ouvrier à la journée, tant soit peu honnête, aurait honte de se mon­trer sans porter une chemise ; et un tel dénuement annoncerait en lui cet état de misère ignominieuse dans lequel on ne peut guère tomber que par la plus mauvaise conduite. D’après les usages reçus, les souliers sont devenus de même, en Angleterre, un des besoins nécessaires de la vie. La personne la plus pauvre de l’un et de l’autre sexe, pour peu qu’elle respecte les bienséances, rougirait de se montrer en public sans souliers. En Écosse aussi, d’après les usages, cette chaussure est un des premiers besoins de la vie pour la dernière classe, mais parmi les hommes seulement ; il n’en est pas de même, dans cette classe, pour les femmes, qui peuvent très-bien aller nu-pieds sans qu’on en ait plus mauvaise opinion d’elles. En France, les souliers ne sont d’absolue nécessité ni pour les hommes ni pour les femmes ; les gens de la dernière classe du peuple, tant d’hommes que femmes, y paraissent publiquement, sans s’avilir, tantôt en sabots, tantôt pieds nus[47]. Ainsi, par les choses nécessaires à la vie, j’entends non -eulement ce que la nature, mais encore ce que les règles convenues de décence et d’honnêteté ont rendu nécessaire aux dernières classes du peuple. Toutes les autres choses, je les appelle luxe, sans néanmoins vouloir, par cette dénomination, jeter le moindre degré de blâme sur l’usage modéré qu’on peut en faire. La bière et l’ale, par exemple, dans la Grande-Bretagne, et le vin, même dans les pays vignobles, je les appelle des choses de luxe. Un homme, de quelque classe qu’il soit, peut s’abstenir totalement de ces liqueurs, sans s’exposer pour cela au moindre reproche. La nature n’en a fait des choses nécessaires au soutien de la vie, et l’usage n’a établi nulle part qu’il fût contre la décence de s’en passer.
Comme partout le salaire du travail se règle en partie par la demande de travail, et en partie par le prix moyen des choses nécessaires à la subsistance, tout ce qui fait monter ce prix moyen doit nécessairement faire monter les salaires, de manière que l’ouvrier soit toujours à même d’acheter cette quantité de choses nécessaires que l’état de la demande de travail exige qu’il ait, quantité réglée par l’état croissant, station­naire ou décroissant de cette demande . Un impôt sur les choses nécessaires ne peut manquer de faire monter leur prix quelque peu plus haut que le montant de l’impôt, parce que le marchand qui fait l’avance de l’impôt doit, en général, s’en faire rem­bourser avec un profit. Ainsi, fi faut nécessairement qu’un pareil impôt amène dans le salaire du travail un surhaussement proportionné à celui qui arrive dans le prix de ces choses.
C’est ainsi qu’un impôt sur les choses nécessaires à la vie opère exactement de la même manière qu’un impôt direct sur les salaires du travail. Quand même l’ouvrier paierait cet impôt par ses mains, on ne pourrait pas dire proprement, au moins pour un temps considérable, qu’il en fait même l’avance. Il faut toujours, à la longue, que l’avance de cet impôt lui soit faite par celui qui le met immédiatement en ouvrage, au moyen d’une augmentation dans le taux de son salaire. Celui-ci, s’il est maître manufacturier, reportera cette élévation de salaire, et encore son profit avec, sur le prix de ses marchandises ; de manière que le payement définitif de l’impôt, accru de cette surcharge, retombera sur le consommateur. Si le maître de l’ouvrier est un fermier, ce payement définitif, y compris une pareille surcharge, retombera sur le fermage du propriétaire.
Il n’en est pas de même des impôts sur ce que j’appelle choses de luxe, même sur celles dont le pauvre fait le plus d’usage. Une hausse dans le prix des denrées imposées n’entraînera pas nécessairement une hausse dans le salaire du travail. Un impôt sur le tabac, par exemple, quoique ce soit une chose de luxe à l’usage du pauvre aussi bien que du riche, ne fera pas hausser les salaires. Quoiqu’il soit imposé, en Angleterre, à trois fois son prix originaire, et en France à quinze fois ce prix, cependant il ne paraît pas que ces droits énormes aient produit aucun effet sur les salaires du travail. On en peut dire autant des impôts sur le thé et sur le sucre, qui sont devenus, en Angleterre et en Hollande, des choses de luxe à l’usage des dernières classes du peuple ; de ceux sur le chocolat, qui a acquis la même importance, à ce qu’on dit, en Espagne. Les différents impôts qu’on a établis en Grande-Bretagne, dans le cours de ce siècle, sur les liqueurs spiritueuses, ne passent pas pour avoir produit quelque effet sur les salaires du travail. La hausse occasionnée dans le prix du porter par un impôt additionnel de 3 sch. par baril de bière forte, n’a pas fait monter, à Londres, les salaires du travail de manœuvre[48].
Le haut prix des denrées de cette espèce ne fait pas nécessairement que les classes inférieures du peuple aient moins qu’auparavant le moyen d’élever leurs familles. À l’égard d’un homme pauvre qui est rangé et laborieux, des impôts sur ces sortes de denrées agissent comme des lois somptuaires, et le disposent ou à modérer, ou à cesser tout à fait l’usage des choses superflues qu’il ne peut plus suffire à se procurer sans se gêner. Loin que ces impôts lui retranchent rien des moyens d’élever sa famille, souvent peut-être, par une suite de cette frugalité forcée, ils contribuent à y ajouter. Ce sont les pauvres laborieux et économes qui, en général, élèvent les plus nombreuses familles, et qui fournissent principalement à la demande qu’on fait de travail utile. Il est vrai que tous les pauvres ne sont pas rangés et laborieux, et que ceux qui sont sans ordre et sans conduite pourraient bien continuer à se permettre l’usage de ces sortes de denrées après l’élévation du prix tout comme auparavant, sans songer à la gêne que ces habitudes pourraient mettre dans leurs ménages. Néanmoins, ces gens dérangés n’élèvent guère de familles nombreuses ; leurs enfants, en général, périssent par défaut de soins, par vice de régime et faute d’une nourriture ou saine, ou assez abondante. Si la force de leur constitution l’emporte sur les risques auxquels les expose la mauvaise conduite de leurs parents, encore arrive-t-il que les mauvais exemples placés à tous moments sous leurs yeux corrompent ordinairement leurs mœurs, de manière que, au lieu d’être utiles à la société par leur industrie, ils devien­nent des fléaux publics par leurs vices et leurs dérèglements. Ainsi, quand même l’élévation du prix dans les choses de luxe à l’usage des pauvres viendrait à augmenter de quelque chose la gêne et la misère de ces ménages dérangés, et à leur ôter en partie les moyens d’élever des enfants, il est probable qu’il n’en résulterait pas une grande diminution dans la population utile du pays.
Toute élévation dans le prix moyen des choses nécessaires à la vie, à moins qu’elle ne soit compensée par une augmentation proportionnée dans le taux des salai­res du travail, doit nécessairement diminuer plus ou moins, parmi les gens pauvres, le moyen d’élever de nombreuses familles et, par conséquent, de fournir à la demande qui s’y fait de travail utile, quel que puisse être l’état de cette demande, croissant, stationnaire ou décroissant, ou quel que soit le mouvement qu’il imprime à la population, progressif, ou stationnaire, ou rétrograde.
Les impôts sur les choses de luxe n’ont aucune tendance à faire monter le prix d’aucune autre marchandise que de celles qui sont =posées. Les impôts sur les choses de nécessité, en faisant monter les salaires du travail, tendent nécessairement à faire monter le prix de tous les objets manufacturés et, par conséquent, à en diminuer la vente et la consommation. Les impôts sur les choses de luxe sont payés, en définitive, par les consommateurs de la chose imposée, sans aucune répétition de leur part. Ils tombent indistinctement sur toutes espèces de revenus, salaires de travail, profits de capitaux et rentes de terre. Les impôts sur les choses de nécessité, pour ce qui porte sur la classe pauvre et ouvrière, sont payés en définitive, partie par les propriétaires dans le déchet que souffrent leurs revenus fonciers, et partie par les riches consom­ma­teurs, propriétaires et autres, dans le surhaussement de prix des choses manufac­turées, et toujours ils sont payés avec une surcharge considérable. L’élévation du prix de ces choses manufacturées, qui sont de véritables choses de nécessité, et qui sont destinées à la consommation du pauvre, des grosses étoffes de laine par exemple, doit nécessairement être compensée chez le pauvre par l’élévation de son salaire. Si les classes supérieures et moyennes entendaient bien leur intérêt, elles devraient toujours s’opposer à tous impôts sur les choses nécessaires à la vie, tout comme aux impôts directs sur les salaires du travail. Le payement définitif des uns, aussi bien que des autres, retombe en entier sur elles, et toujours avec une surcharge considérable. Il retombe avec plus de poids surtout sur le propriétaire, qui paie toujours doublement ou à deux différents titres : comme propriétaire, par la réduction de son revenu, et comme riche consommateur, par l’augmentation de sa dépense. L’observation faite par sir Matthieu Decker, qu’il y a des impôts qui sont quelquefois répétés et accu­mulés cinq ou six fois dans le prix de certaines marchandises, est parfaitement juste à l’égard des impôts sur les choses nécessaires à la vie. Par exemple, dans le prix du cuir, il faut que vous payiez non-seulement l’impôt sur le cuir des souliers que vous portez, mais encore une partie de cet impôt sur les souliers que portent le cordonnier et le tanneur[49]. Il faut que vous payiez de plus pour l’impôt sur le sel, sur le savon et sur les chandelles que consomment ces ouvriers pendant le temps qu’ils emploient à travailler pour vous, et puis encore pour l’impôt sur le cuir qu’usent le faiseur de sel, le faiseur de savon et le faiseur de chandelles, pendant qu’ils travaillent pour ces mêmes ouvriers.
Dans la Grande-Bretagne, les principaux impôts sur les choses de nécessité sont ceux sur les quatre denrées que je viens de nommer : le sel, le cuir, le savon et la chandelle.
Le sel est un objet d’imposition très-ancien et très-universel. Il était imposé chez les Romains, et il l’est actuellement, je crois, dans tous les endroits de l’Europe. La quantité annuellement consommée par un individu est si petite et peut s’acheter si aisément à mesure du besoin, qu’on a pensé, à ce qu’il semble, qu’un impôt, même assez lourd sur cette denrée, ne serait guère sensible pour personne. Il est imposé, en Angleterre, à 3 schellings 4 deniers le boisseau, environ trois fois le prix originaire de cette denrée[50]. En quelques autres pays, l’impôt est encore plus fort. Le cuir est vrai­ment une chose de nécessité. L’usage du linge a aussi rendu le savon indispensable. Dans des pays où les nuits d’hiver sont longues, la chandelle devient un véritable élément de travail. Le cuir et le savon sont imposés, dans la Grande-Bretagne, à 3 demi-pence la livre ; les chandelles à 1 penny[51] ; impôts qui peuvent monter, sur le prix originaire du cuir, à environ 8 ou 10 pour 100, sur celui du savon à environ 20 ou 25 pour 100, et sur celui de la chandelle, à environ 14 ou 15 pour 100, et qui ne laissent pas que d’être encore très-lourds, quoique bien moins que celui sur le sel. Comme ces quatre denrées sont vraiment des choses de première nécessité, des impôts aussi lourds sur de tels articles doivent infailliblement augmenter de quelque chose la dépense du pauvre rangé et laborieux, et doivent par conséquent faire hausser plus ou moins les salaires de son travail[52].
Dans un pays où les hivers sont aussi froids qu’ils le sont dans la Grande-Bretagne, le feu, pendant cette saison, est, dans le sens le plus étroit du mot, une chose de première nécessité, non-seulement pour la préparation des aliments, mais encore pour que maintes espèces différentes d’ouvriers qui travaillent dans des en­droits clos puissent endurer la rigueur du temps ; et le charbon de terre est, de tous les chauffages, le plus économique. Le prix du chauffage a une si grande influence sur celui du travail, que par toute la Grande-Bretagne les fabriques se sont retirées principalement dans les pays de charbon de terre, les autres endroits du pays n’étant pas en état de travailler à aussi bon marché, à cause du haut prix de cet article de première nécessité. D’ailleurs, dans quelques manufactures, le charbon est un instrument nécessaire de métier, comme dans celles de verrerie, de fer, et de tous les autres métaux. S’il y avait quelque cas où une prime pût être une chose raisonnable, ce serait peut-être celle qu’on accorderait pour transporter le charbon de terre des endroits du pays dans lesquels il est abondant, à ceux qui en manquent. Mais la législature, au lieu d’une prime, a établi un impôt de 3 sch. 3 deniers par tonneau, sur le charbon transporté par mer le long des côtes ; ce qui, sur la plupart des espèces de charbon, est plus de 60 p. 100 du prix originaire de cette denrée à la mine[53]. Le charbon transporté par terre ou bien par eau, dans l’intérieur du pays, ne paie pas de droit. Où cette marchandise est naturellement à bon marché, on la consomme franche de droit ; où elle est naturellement chère, elle est chargée, pour le consommateur, d’un droit fort lourd.
Si de tels impôts font monter le prix de la subsistance et, par conséquent, les salaires du travail, ils rapportent en outre au gouvernement un revenu considérable qu’il ne pourrait pas aisément trouver de toute autre manière. Il peut donc y avoir de bonnes raisons pour les continuer. La prime à l’exportation des grains, en tant qu’elle tend, dans l’état actuel du labourage, à faire monter le prix de cet article de première nécessité, produit tous les mêmes mauvais effets, et au lieu de fournir aucun revenu au gouvernement, elle lui cause souvent une dépense énorme. Les gros droits sur l’importation des blés étrangers, qui, dans les années d’une abondance moyenne, équivalent à une prohibition ; et la prohibition absolue d’importer soit du bétail vivant, soit des viandes salées, prohibition qui a lieu dans l’état ordinaire de la loi, et qui à présent, à cause de la disette, se trouve suspendue pour un temps limité à l’égard de l’Irlande et de nos colonies, toutes ces institutions ont tous les mauvais effets des impôts établis sur les choses de première nécessité, et ne produisent aucun revenu au gouvernement. Il n’est pas besoin d’autre chose, à ce qu’il semble, pour faire révoquer de semblables règlements, que de bien convaincre le public de la futilité du système par suite duquel ils ont été établis.
Les impôts sur les choses de première nécessité sont beaucoup plus forts, dans un grand nombre d’autres pays, qu’ils ne le sont dans la Grande-Bretagne. Dans plusieurs pays, il y a des droits à payer sur la farine et la fleur de farine quand on moud le blé au moulin, et sur le pain quand on le cuit au four. En Hollande, le prix en argent du pain qui se consomme dans les villes est, à ce qu’on croit, doublé par des impôts de ce genre. À la place d’une partie de ces impôts, les gens qui vivent à la campagne payent tant par tête chaque année, selon l’espèce de pain qu’ils sont censés consommer. Ceux qui mangent du pain de froment payent 3 florins 15 stivers, environ 6 sch. 9 deniers 1/2. On dit que ces impôts et quelques autres du même genre, en faisant monter le prix du travail, ont ruiné la plupart des manufactures de Hollande[54]. Des impôts semblables, quoique pas tout à fait aussi lourds, existent dans le Milanais, dans les États de Gênes, dans le duché de Modène, dans les duchés de Parme, Plaisance et Guastalla, et dans l’État de l’Église. Un auteur français[55] de quelque réputation a proposé de réformer les finances de son pays, en substituant à la plus grande partie des autres impôts cette espèce d’impôt, la plus ruineuse de toutes. Il n’y a rien de si absurde, dit Cicéron, qui n’ait été avancé par quelque philosophe.
Les impôts sur la viande de boucherie sont encore plus communs que ceux sur le pain. À la vérité, on peut mettre en doute si la viande de boucherie est nulle part une chose de première nécessité. Il est bien connu par l’expérience, que sans recourir à aucune viande on peut trouver la nourriture la plus abondante, la plus saine, la plus substantielle et la plus agréable dans les grains et autres végétaux, avec l’aide du lait, du fromage et du beurre, ou bien de l’huile quand on ne peut avoir de beurre. Il n’y a pas d’endroits où les règles de la décence exigent qu’un homme mange de la viande, comme elles exigent dans plusieurs qu’il ait une chemise ou des souliers.
Les objets de consommation, soit de nécessité, soit de luxe, peuvent être imposés de deux différentes manières. On peut faire payer au consommateur une somme annuelle pour pouvoir consommer ou faire usage de marchandises d’une certaine espèce, ou bien on peut imposer les marchandises pendant qu’elles sont dans les mains du marchand et avant qu’elles aient passé dans celles du consommateur. Les objets de consommation qui durent un temps considérable avant d’être totalement consommés sont ceux qui sont les plus propres à être imposés de la première manière ; ceux dont la consommation se fait immédiatement, ou au moins plus promptement, sont les plus propres à être imposés de l’autre manière. La taxe sur les carrosses et celle sur la vaisselle sont des exemples du premier de ces deux modes d’imposition. La plupart des autres droits d’accise et de douane sont des exemples du dernier.
Un carrosse bien ménagé peut servir dix ou douze ans. On pourrait bien l’imposer une fois pour toutes, avant qu’il sortît des mains du carrossier. Mais il est certainement plus commode pour l’acheteur de payer 4 livres par an pour le privilège de rouler carrosse[56], que de payer tout à la fois 40 ou 48 livres par surcroît de prix au carrossier, ou une somme équivalente à celle que l’impôt est dans le cas de lui coûter pendant le temps qu’il se servira du même carrosse. De même, un service de vaisselle peut durer plus d’un siècle. Il est certainement plus commode pour le consommateur de payer 5 sch. par an pour chaque cent onces de vaisselle, c’est-à-dire près de 1 pour 100 de la valeur, que de racheter cette longue annuité sur le pied du denier 25 ou 30, ce qui renchérirait le prix d’au moins 25 ou 30 pour 100[57]. Les différents impôts qui portent sur les maisons sont certainement bien plus aisés à payer par des payements modiques faits tous les ans, que par une taxe fort lourde et équivalente, imposée sur la première bâtisse ou vente de la maison.
C’était un projet fort connu, proposé par sir Matthieu Decker, d’imposer de cette manière toute espèce de marchandises, même celles dont la consommation se fait immédiatement et très-promptement, le marchand ne faisant aucune avance pour l’im­pôt, mais le consommateur payant une certaine somme annuelle pour la permis­sion de consommer certaines marchandises. Le but de son projet était de donner de l’extension à toutes les branches différentes de commerce étranger, et particulière­ment au commerce de transport, par la suppression de tous les droits sur l’importation et sur l’exportation, ce qui mettrait le marchand en état d’employer la totalité de ses capitaux et de son crédit en acquisition de marchandises et frais de bâtiments, sans en distraire aucune partie pour l’avance de l’impôt. Cependant il y a, à ce qu’il semble, quatre objections fort importantes à faire contre le plan d’imposer de cette manière des marchandises dont la consommation se fait immédiatement ou dans un temps fort court.
Premièrement, l’impôt serait inégal ou ne serait pas si bien proportionné à la dépense et à la consommation des différents contribuables, qu’il l’est dans la manière ordinaire d’imposer. Les taxes sur l’ale, le vin et les liqueurs spiritueuses, dont l’avan­ce se fait par les marchands, sont en définitive payées par les différents consomma­teurs, dans la proportion exacte de leur consommation respective. Mais si la taxe se payait en achetant une permission pour boire de ces liqueurs, le consommateur frugal serait, à proportion de sa consommation, imposé bien plus durement que le consom­mateur buveur. Un ménage qui recevrait beaucoup de monde à sa table serait imposé bien plus doucement qu’un autre qui n’aurait que très-peu de convives.
Secondement, ce mode d’imposition de payer par année, par semestre ou par quar­tier, une permission pour consommer certaines marchandises, diminuerait extrême­ment une des principales commodités des impôts sur les choses d’une prompte consommation, c’est-à-dire la facilité de payer petit à petit. Dans le prix de 3 pence 1/2 que se paie à présent le pot de porter, les différentes taxes sur la drêche, le hou­blon et la bière, y compris le profit extraordinaire dont le brasseur charge la mar­chandise pour avoir avancé ces taxes, peuvent se monter peut-être à environ 3 demi-pence. Si un ouvrier peut sans se gêner, dépenser ces 3 demi-pence, il achète un pot de porter. S’il ne le peut pas, il se contente d’une pinte, et comme ce qu’on épargne est autant de gagné, sa tempérance lui fait ainsi gagner 1 farting[58]. Il paie l’impôt petit à petit selon qu’il est en état de le payer, et quand il a le moyen de le payer. Chaque acte de payement est parfaitement volontaire, et il est le maître de s’en dispenser si cela lui convient mieux. Troisièmement, ces sortes d’impôts auraient moins l’effet de lois somptuaires. Quand la permission sera une fois achetée, que le consommateur boive beaucoup ou boive très-peu, l’impôt sera le même pour lui. Quatrièmement, s’il fallait qu’un ouvrier payât en une seule fois par année, par demi-année ou par quartier, un impôt égal à ce qu’il paie à présent sans embarras ou presque sans embarras, sur chacun des différents pots ou pintes de porter qu’il boit dans un pareil espace de temps, la somme pourrait souvent le gêner extrêmement. Ainsi, il paraît évident qu’un pareil mode d’imposition ne pourrait jamais, à moins de beaucoup de gêne et d’oppression pour les contribuables, produire un revenu approxi­ma­tivement égal à ce qu’on retire par le mode actuel d’imposition sans opprimer per­sonne. Néanmoins, dans plusieurs pays, des denrées dont la consommation se fait immédiatement ou dans un temps fort court, sont imposées de cette manière. En Hollande, on paie tant par tête pour la permission de boire du thé. J’ai déjà parlé d’un impôt sur le pain, qui est perçu de la même manière, quant au pain qui se mange dans les fermes et dans les villages.
Les droits d’accise sont principalement imposés sur les marchandises du produit du pays, et destinées à sa consommation. Ils ne sont imposés que sur un petit nombre d’espèces de marchandises dont l’usage est le plus général. Il ne peut jamais y avoir matière à incertitude, ou sur les marchandises qui sont sujettes à ces droits, ou sur le droit particulier auquel telle espèce de marchandises est assujettie. Ces droits portent presque en totalité sur ce que j’appelle choses de luxe, excepté toujours les quatre espèces de droits dont j’ai fait mention, qui sont ceux sur le sel, le savon, le cuir et les chandelles, auxquels on pourrait peut-être ajouter ceux sur le verre commun[59].
Les droits de douane ou traites[60] sont beaucoup plus anciens que ceux d’accise. Il paraît qu’ils ont été nommés coutumes (customs), pour désigner des payements coutu­miers qui étaient en usage depuis un temps immémorial. Ils ont été regardés dans l’origine, à ce qu’il me semble, comme des impôts sur les profits des marchands. Dans les temps barbares de l’anarchie féodale, les marchands, ainsi que tous les autres habitants des bourgs, n’étaient guère autrement regardés que comme des serfs affran­chis, dont on méprisait la personne et dont on enviait les profits. La haute noblesse, qui avait consenti que les profits de ses propres tenanciers fussent taillés par le roi, ne fit nulle difficulté de lui laisser prendre aussi la taille sur une classe d’hommes qu’elle avait bien moins d’intérêt à protéger. Dans ces temps d’ignorance, on n’était pas en état de comprendre que les profits des marchands ne sont pas de nature à être imposés directement, ou que le payement définitif de tout impôt assis de cette manière doit toujours retomber avec une surcharge considérable sur les consommateurs.
Les gains des marchands étrangers furent vus avec bien plus de défaveur encore que ceux des marchands anglais. Il était donc naturel que ceux des premiers fussent imposés plus durement que ceux des autres. Cette distinction entre les droits perçus sur les marchands étrangers et ceux perçus sur les marchands anglais, qui commença d’abord par esprit d’ignorance, a été continuée ensuite par esprit de monopole, ou dans la vue de donner un avantage à nos marchands, tant sur notre marché que sur le marché étranger.
Les anciens droits de coutumes ou de douane, avec cette seule distinction, furent imposés également sur toute espèce de marchandises, sur les choses de nécessité aussi bien que sur celles de luxe, sur les objets exportés tout comme sur les objets importés. Pourquoi, à ce qu’on semble s’être imaginé, celui qui trafique d’une espèce de denrée serait-il mieux traité que celui qui trafique d’une autre ? Ou pourquoi le marchand qui exporte serait-il plus favorisé que le marchand qui importe ?
Les anciens droits de douane étaient divisés en trois branches. Le premier, et peut-être le plus ancien de tous ces droits, était celui sur la laine et sur le cuir. Il paraît avoir été principalement ou même tout à fait un droit sur l’exportation. Lorsque les manufactures d’étoffes de laine commencèrent à être établies en Angleterre, de peur que le roi ne se trouvât, par l’exportation des draps, perdre une partie de ses droits de douane sur la laine, on établit sur ceux-ci un droit pareil. Les autres deux branches étaient : 1° un droit sur le vin, qui, étant établi à raison de tant par tonneau, fut nom­mé tonnage ; 2° un droit sur toutes les autres marchandises, qui, étant établi à tant par livre de leur valeur supposée, fut appelé pondage. Dans la quarante-septième année d’Edouard III, il fut établi un droit de 6 den. par livre sur toutes marchandises exportées et importées, excepté les laines, les peaux garnies de leur laine, le cuir et les vins, qui furent assujettis à des droits particuliers. Dans la quatorzième de Richard Il, ce droit fut porté à 1 sch. par livre ; mais, trois années après, il fut remis à 6 deniers. Dans la deuxième année de Henri IV, il fut porté à 8 deniers, et dans la quatrième du même règne à 1 sch. Il resta à 1 sch. par livre depuis cette époque jusqu’à la neuvième de Guillaume III. Les droits de tonnage et de pondage furent, en général, accordés au roi par un seul et même acte du parlement, et on les appela le subside de tonnage et pondage. Le subside de pondage étant resté pendant si longtemps sur le pied de 1 sch. par livre ou de 5 p. 100, un subside, dans le langage des douanes, devint la dénomination d’un droit général de ce genre, de 5 p. 100. Ce subside, qu’on nomme aujourd’hui l’ancien subside, continue toujours à se percevoir d’après le livre du tarif dressé dans la douzième année de Charles II. On dit que la méthode de constater par un livre de tarif la valeur des marchandises sujettes à ce droit remonte au-delà du règne de Jacques la. Le nouveau subside établi par les neuvième et dixième années de Guillaume III fut un droit additionnel de 5 p. 100 sur la plus grande partie des marchandises. Le tiers de subside et les deux tiers de subside formèrent entre eux un autre droit de 5 p. 100, dont ils étaient les parties intégrantes. Le subside de 1747 fut un quatrième droit de 5 p. 100 sur la plus grande partie des marchandises, et celui de 1759 un cinquième droit qui ne porta que sur quelques espèces particulières de marchandises. Outre ces cinq subsides, il a été établi acciden­tel­le­ment une grande multitude d’autres droits divers sur des espèces particulières de marchandises, tantôt dans la vue de subvenir au besoin de l’État, et tantôt dans la vue de diriger et de régler le commerce du pays suivant les principes du système mer­cantile.
Ce système a pris faveur successivement de plus en plus. L’ancien subside était imposé indistinctement sur l’exportation aussi bien que sur l’importation. Les quatre subsides subséquents, ainsi que les autres droits qui ont été depuis imposés acciden­tel­lement sur des espèces particulières de marchandises, ont tous été, à très-peu d’exceptions près, mis en totalité sur l’importation. La plus grande partie des anciens droits sur l’exportation des marchandises du cru du pays ou de ses fabriques ont été modifiés, ou tout à fait supprimés. On a même accordé des primes à l’exportation de quelques-unes de ces marchandises. Quant aux droits établis à l’importation de mar­chan­dises étrangères, on a accordé, lors de l’exportation de ces mêmes marchandises, le retour ou restitution, quelquefois de la totalité, et le plus souvent d’une partie du droit. On ne restitue à l’exportation qu’une moitié des droits établis sur l’importation par l’ancien subside ; mais la totalité de ceux établis par les derniers subsides et par les autres impôts est restituée de la même manière, sur la plus grande partie des marchan­dises. Ces grâces toujours croissantes en faveur de l’exportation, et ces décourage­ments contre l’importation, n’ont souffert que peu d’exceptions, qui regardent princi­pa­­lement les matières premières de quelques manufactures. Quant à celles-ci, nos marchands et manufacturiers voudraient qu’elles pussent leur revenir au meilleur mar­ché possible, et qu’elles fussent payées le plus cher possible par leurs rivaux et con­cur­rents dans les autres pays. C’est par cette raison qu’on laisse quelquefois importer, franches de tous droits, des matières premières de l’étranger ; par exemple, des laines d’Espagne, du lin et du fil écru pour toiles. L’exportation des matières premières produites chez nous, et de celles qui sont le produit particulier de nos colonies, a quelquefois été prohibée et quelquefois assujettie à des droits plus forts. L’exportation des laines anglaises a été prohibée. Celle du castor, soit en peau, soit en poil, et celle de la gomme du Sénégal, ont été assujetties à de plus forts droits, la Grande-Bretagne ayant gagné à peu près le monopole de ces marchandises par la conquête du Canada et du Sénégal[61].
Que ce système mercantile n’ait pas été très-favorable au revenu de la masse du peuple, au produit annuel des terres et du travail du pays, c’est ce que j’ai tâché de montrer dans le IVe livre de cet ouvrage. Il ne paraît pas qu’il ait été plus favorable au revenu du souverain, au moins quant à cette partie du revenu qui dépend des droits de douane.
En conséquence de ce système, l’importation de plusieurs sortes de marchandises a été totalement prohibée. Cette prohibition a, dans quelques circonstances, entière­ment empêché, et dans d’autres extrêmement diminué l’importation de ces marchandi­ses, en réduisant les marchands importateurs à la nécessité de les faire entrer en fraude. Elle a entièrement empêché l’importation des étoffes de laine fabriquées chez l’étranger, et elle a extrêmement diminué celle des soieries et des velours étrangers. Dans ces différentes circonstances, elle a de même anéanti totalement le revenu que les douanes auraient eu à percevoir sur ces importations.
Les gros droits qu’on a établis sur l’importation de plusieurs différentes espèces de marchandises étrangères, dans la vue d’en décourager la consommation dans la Grande-Bretagne, n’ont servi, la plupart du temps, qu’à encourager leur entrée en frau­de, et dans tous les cas ils ont réduit le revenu des douanes au-dessous de ce qu’au­­raient rapporté des droits plus modérés. Le mot du docteur Swift, que, dans l’arithmétique des douanes, « deux et deux, au lieu de faire quatre, ne font souvent qu’un », est d’une vérité parfaite à l’égard de ces gros droits, qu’on n’aurait jamais pen­sé à établir si le système du commerce ne nous eût appris à employer la plupart du temps l’impôt comme instrument, non de revenu, mais de monopole.
Les primes qui sont quelquefois accordées à l’exportation du produit et des ouvrages de fabrique du pays, ainsi que les retours ou restitutions de droits que l’État paie lors de la réexportation de la plupart des marchandises étrangères, ont donné nais­­san­ce à un grand nombre de fraudes et à une espèce de contrebande plus destruc­tive du revenu public qu’aucune autre. Tout le monde sait que, pour obtenir la prime ou la restitution des droits, les marchandises sont quelquefois chargées sur un vais­seau et mises en mer, mais bientôt après débarquées clandestinement dans quel­que endroit du pays. La défalcation qu’occasionnent dans le revenu des douanes les gratifications et drawbacks, dont il y a une grande partie obtenue frauduleusement, est un objet énorme. Dans l’année qui a fini au 5 janvier 1755, le produit total des douanes montait à 5 068 000 livres. Les primes qui furent payées sur ce revenu, quoiqu’il n’y eût pas cette année de prime sur le blé, montèrent à 167 800 livres. Les retours ou restitutions de droits qui furent payés sur les acquits et certificats montè­rent à 2 156 800 livres. Les primes et drawbacks ensemble formèrent un total de 2 324 600 livres. En conséquence de ces déductions, le revenu des douanes ne monta plus qu’à 2 743 400 livres ; de laquelle dernière somme déduisant 287900 livres pour frais de régie consistant en appointements et autres dépenses accessoires, le revenu net des douanes, pour cette année, se trouva être de 2 45 5 500 livres. Ainsi, les frais de régie vont à environ 5 ou 6 p. 100 du revenu brut des douanes, et à quelque chose de plus que 10 p. 100 sur ce qui reste de ce revenu, déduction faite de ce qui se paie en primes et restitutions de droits[62].
Au moyen des droits énormes dont sont chargées presque toutes les marchandises à l’importation, nos marchands importateurs font entrer en fraude le plus possible, et font leur déclaration aux registres des douanes pour le moins possible. Nos mar­chands exportateurs, au contraire, font déclaration aux registres de plus que ce qu’ils exportent réellement ; quelquefois par vanité et afin de se faire passer pour gens qui font de grosses affaires dans ce genre de marchandises qui ne payent pas de droits, et quelquefois aussi afin de gagner une prime ou un drawback. En conséquence de toutes ces fraudes différentes, nos exportations paraissent, sur le registre des douanes, l’emporter de beaucoup sur nos importations ; ce qui fait un merveilleux sujet de triom­phe pour les politiques subtils qui regardent ce qu’ils appellent la balance du commerce comme l’infaillible mesure de la prospérité nationale.
Toutes les marchandises importées, à moins qu’elles ne jouissent d’une exemption particulière (et ces exemptions ne sont pas très-nombreuses), sont sujettes à quelques droits de douane. Si l’on importe une marchandise qui ne se trouve pas mentionnée dans le livre du tarif, elle est taxée à 4 sch. 9 deniers 9/20 par chaque 20 sch. de sa valeur, sur la déclaration assermentée du marchand qui l’importe, c’est à-dire à cinq subsides ou cinq droits de pondage. Le livre du tarif est extrêmement étendu, et contient l’énumération d’une très-grande multitude d’articles, dont un grand nombre très-peu en usage et, par conséquent, très-peu connus. C’est pour cela qu’il est souvent difficile de décider sous quel article il faut classer une espèce particulière de mar­chandises et, par conséquent, quel droit elle doit payer. Il y a telles méprises à cet égard qui ruinent quelquefois l’officier de la douane, et il y en a très-fréquemment qui causent beaucoup d’embarras, de frais et de vexations au marchand importateur. Ainsi, sous le rapport de la clarté, de la précision et de la classification, les droits de douane sont fort inférieurs à ceux d’accise.
Pour que la plus grande partie des membres d’une société contribuent au revenu public à proportion de leur dépense respective, il n’est pas nécessaire, à ce qu’il sem­ble, que chaque article particulier de cette dépense se trouve imposé. Le revenu que produisent les droits d’accise passe pour tomber sur les contribuables d’une manière aussi égale que le revenu qui se lève aux douanes, et cependant les droits d’accise ne sont imposés que sur un petit nombre d’articles seulement, d’un usage et d’une con­som­mation plus générale. Beaucoup de gens ont pensé qu’avec un régime bien entendu, les droits de douane pourraient de même être restreints à un petit nombre d’articles seulement, sans aucune perte pour le revenu public, et avec de grands avantages pour le commerce étranger[63]. 
Les articles tirés de l’étranger, qui sont d’un usage et d’une consommation plus gé­né­rale dans la Grande-Bretagne, consistent pour le présent, à ce qu’il semble, princi­pa­lement en vins et eaux-de-vie, en quelques-unes des productions de l’Amérique et des Indes Occidentales, comme sucre, rhum, tabac, noix de cacao, etc., et en quelques-unes de celles des Indes Orientales, comme thé, café, porcelaine, épice de toute espèce, différentes sortes d’étoffes, etc. Ces divers articles fournissent peut-être maintenant la plus grande partie du revenu qu’on retire des droits de douane. Les impôts qui subsistent à présent sur les articles de manufacture étrangère, si vous en exceptez les droits sur le peu qu’en contient l’énumération ci-dessus, sont des impôts établis, pour la plupart, non pas en vue d’augmenter le revenu public, mais en vue d’assurer un monopole ou de donner à nos marchands un avantage dans notre marché intérieur. Si l’on supprimait toutes les prohibitions, et qu’on assujettît tous les objets de fabrique étrangère à des droits modérés, et tels que l’expérience les démontrerait propres à rendre sur chaque article le plus gros revenu à l’État, alors nos propres ouvriers se trouveraient jouir encore, sur notre marché, d’un avantage assez considé­rable, et l’État retirerait un très-gros revenu d’une foule d’articles d’importation dont à présent quelques-uns ne lui en rapportent aucun, tandis que d’autres lui en rapportent un presque nul.
Les droits élevés, soit en diminuant la consommation des marchandises imposées, soit en encourageant la contrebande, rendent souvent au gouvernement un plus faible revenu que celui qu’il aurait retiré de droits plus modiques.
Quand la diminution de revenu est l’effet d’une diminution de consommation, il ne peut y avoir qu’un remède, c’est de réduire les droits.
Quand la diminution du revenu est l’effet de l’encouragement donné à la contre­bande, on peut y remédier de deux manières, ou en diminuant la tentation de frauder, ou en augmentant les difficultés de la contrebande. On ne peut diminuer la tentation qu’en réduisant les droits, et on ne peut augmenter les difficultés qu’en établissant le système d’administration qui est le plus propre à empêcher la contrebande.
L’expérience démontre, je crois, que les lois de l’accise arrêtent et gênent d’une manière bien plus efficace les manœuvres de la contrebande que ne le font les lois de douanes. On pourrait beaucoup ajouter aux difficultés de la contrebande, en intro­duisant dans les douanes un système d’administration aussi semblable à celui de l’ac­ci­se que pourrait le comporter la nature différente de ces deux sortes de droits. Beaucoup de gens ont pensé qu’on pourrait très-aisément venir à bout d’opérer ce changement[64].
Par exemple, le marchand qui importerait les marchandises sujettes à quelques droits de douane pourrait avoir la faculté de les faire transporter dans son magasin particulier, ou, à son choix, de les placer dans un magasin qu’il se procurerait à ses frais ou que lui procurerait le gouvernement, mais qui dans tous les cas serait sous la clef de l’officier de la douane, et ne pourrait jamais être ouvert qu’en sa présence[65]. Si le marchand préférait faire transporter ses marchandises à son magasin particulier, alors il serait tenu de payer immédiatement les droits, et ne pourrait plus par la suite en espérer aucune restitution ; ce magasin serait, dans tous les moments, sujet à la visite et à l’examen de l’officier de la douane, à l’effet par lui de s’assurer jusqu’à quel point la quantité des marchandises contenues se trouve répondre à celle pour laquelle on a payé les droits. Si le marchand préférait les placer dans le magasin public, alors il n’aurait aucun droit à payer jusqu’au moment où il les en ferait sortir pour la consom­ma­tion intérieure. S’il les faisait sortir pour l’exportation, elles seraient franches de droits, à condition par le marchand d’une sûreté suffisante que les marchandises se­ront réellement exportées. Les marchands qui font commerce de ces sortes de mar­chan­dises, soit en gros, soit en détail, seraient à tous les instants sujets à la visite et à l’inspection de l’officier de la douane, et seraient tenus de justifier par des certi­ficats en bonne forme du payement des droits sur toute la quantité contenue dans leurs bou­tiques ou magasins. Les droits qu’on appelle droits d’accise sur le rhum importé sont actuellement perçus de cette manière, et il serait peut-être possible d’étendre à tous les droits sur les marchandises importées le même système d’administration, pourvu toujours que ces droits fussent, comme les droits d’accise, bornés à un petit nombre d’espèces de marchandises d’un usage et d’une consommation générale. S’ils éten­daient à toutes les espèces de marchandises, comme ils font à présent, il ne serait pas aisé de trouver des magasins publics d’une assez grande étendue, et il y a certai­nes marchandises d’une nature très-délicate et dont la conservation exige beaucoup de soins et d’attention, que le marchand n’oserait pas placer ailleurs que dans son propre magasin.
Si, au moyen d’un pareil système d’administration, on pouvait empêcher que la contrebande se fit en une quantité un peu considérable, même en supposant des droits assez forts ; si chaque droit était, au besoin, ou augmenté, ou modéré, suivant qu’il serait présumé devoir, d’une manière ou de l’autre, rendre à l’État le plus de revenu, l’imposition étant toujours employée comme moyen de revenu, et jamais comme moyen de monopole, alors il ne paraît pas hors de vraisemblance que des droits sur l’importation seulement d’un petit nombre d’espèces de marchandises d’un usage et d’une consommation générale pourraient rendre à l’État un revenu au moins égal au revenu net actuel des douanes, et qu’ainsi les droits de douanes pourraient être portés au même degré de simplicité, de certitude et de précision que ceux d’accise. Avec un tel système, on épargnerait en entier ce que perd aujourd’hui le revenu public par des drawbacks sur des réexportations de marchandises étrangères qu’on fait ensuite ren­trer dans le pays et qui y sont consommées. À cet article d’économie, qui serait lui seul très-considérable, si l’on ajoutait encore la suppression de toutes les primes à l’exportation des marchandises du produit national (dans tous les cas où ces primes ne seraient pas dans la réalité des restitutions de quelques droits d’accise qui auraient été avancés auparavant), il n’est guère possible de douter qu’après des changements et réformes de ce genre, le revenu net des douanes ne montât largement à ce qu’il n’a jamais pu rendre jusqu’à présent. S’il est évident que le revenu public n’aurait aucune perte à souffrir de ce changement de système, il ne l’est pas moins que le commerce et les manufactures du pays y gagneraient un avantage extrêmement considérable. Le commerce sur les marchandises non Imposées, qui formeraient sans comparaison le plus grand nombre, serait parfaitement libre et pourrait s’étendre, tant en importation qu’en exportation, à toutes les parties du monde, avec tous les avantages possibles. Au nombre de ces marchandises seraient compris tous les articles servant aux pre­miers besoins de la vie, et tous ceux qui sont matières premières de manufacture. Com­­me la libre importation des objets servant aux premiers besoins de la vie con­tribue à réduire leur prix moyen sur le marché national, elle tendrait d’autant à faire baisser le prix en argent du travail, mais sans rien retrancher de sa récompense réelle ; car la valeur de l’argent est en raison de la quantité d’objets de première néces­sité qu’on peut acheter, au lieu que la valeur des objets de première nécessité est absolument indépendante de la quantité d’argent qu’on pourrait avoir à leur place. La diminution du prix en argent du travail amènerait nécessairement une diminution proportionnée dans celui de tous les objets de manufacture nationale, qui gagneraient par là un avantage sur tous les marchés étrangers. Le prix de certains articles de manufacture diminuerait dans une proportion encore plus forte par la libre impor­tation des matières premières à leur état brut. Si l’on pouvait importer, franches de droits, les soies non ouvrées de la Chine et de l’Indostan, les fabricants d’étoffes de soie en Angleterre pourraient très-facilement supplanter ceux de France et d’Italie, par l’infériorité du prix de la fabrication. Il n’y aurait pas besoin de prohiber l’importation des soieries et des velours étrangers. Le bon marché de la marchandise assurerait à nos ouvriers, non-seulement le marché national en entier, mais encore de très-fortes commandes chez l’étranger. Le commerce même des marchandises imposées mar­cherait avec bien plus d’avantage qu’à présent. Si ces marchandises étaient tirées des lieux publics d’entrepôt pour être exportées à l’étranger, étant dans ce cas exemptes de tout droit, ce genre de commerce serait parfaitement libre. Dans un tel système, le com­merce de transport de toute espèce de marchandise quelconque jouirait de tous les avantages possibles. Si les marchandises étaient retirées de l’entrepôt pour être consommées dans l’intérieur, alors le marchand importateur, qui ne serait pas obligé d’avancer l’impôt avant qu’il se fût présenté une occasion de vendre ses marchandises ou à quelque autre marchand, ou à quelque consommateur, pourrait toujours suffire à les vendre à meilleur marché qu’il n’eût pu le faire s’il eût été obligé de faire l’avance de l’impôt au moment de l’importation. Ainsi, avec les mêmes impôts, le commerce étranger de consommation, même en marchandises sujettes à l’impôt, pourrait par ce moyen marcher avec beaucoup plus d’avantage qu’il ne peut le faire à présent.
Le but du fameux projet d’accise de sir Robert Walpole était d’établir, à l’égard du vin et du tabac, un plan assez semblable à celui que je viens d’exposer ici. Mais, quoi­que le bill qui en fut alors porté au parlement ne comprît que ces deux marchan­dises, cependant on croit généralement que ce n’était qu’un acheminement à un plan beau­coup plus étendu. L’esprit de faction, combiné avec l’intérêt des marchands contre­bandiers, suscita contre ce bill une clameur tellement violente, quoique fort injuste, que le ministre crut à propos de laisser tomber le bill, et la crainte de rencontrer une semblable opposition a empêché jusqu’à présent tous ses successeurs de reprendre le projet.
Les droits sur les objets de luxe tirés de l’étranger et importés pour la consom­ma­tion intérieure, quoique acquittés quelquefois par la classe pauvre, portent néanmoins principalement sur les personnes de la classe moyenne ou supérieure ; tels sont, par exemple, les droits sur les vins étrangers, sur le café, le chocolat, le thé, le sucre, etc.
Les droits sur les choses de luxe les moins chères, produites dans le pays et destinées à la consommation intérieure, portent d’une manière fort égale sur les per­sonnes de toutes les classes, à proportion de leur dépense respective. Le pauvre paie les droits sur la drêche, le houblon, la bière et l’ale, à raison de sa consommation per­sonnelle ; le riche les paie, tant sur sa consommation personnelle que sur celle de ses domestiques.
Il faut observer que la somme totale de la consommation que font les classes inférieures du peuple, ou celles qui sont au-dessous de la classe moyenne, est dans tout pays beaucoup plus grande, non-seulement en quantité, mais en valeur, que la consommation de la classe moyenne et de celles qui sont au-dessus de cette classe. La somme totale de la dépense des classes inférieures est beaucoup plus forte que celle des classes supérieures. En premier lieu, la presque totalité du capital de chaque pays se distribue annuellement parmi les classes inférieures du peuple, comme salaires de travail productif. En second lieu, une grande partie des revenus provenant des rentes de terre et des profits de capitaux se distribue annuellement dans les mêmes classes, comme salaires et entretien de domestiques et autres salariés non productifs. Troisiè­mement, il y a quelques parties de profits de capitaux qui appartiennent à ces mêmes classes, comme revenu provenant de l’emploi de leurs petits capitaux. La somme de tous les profits qui se font annuellement par de petits merciers, artisans et détaillants de toutes les espèces, est partout un objet très-considérable et forme une portion très-importante du produit annuel. Quatrièmement enfin, il y a quelque partie même des rentes de terre qui appartient à ces mêmes classes, dont une part considérable à ceux qui sont tant soit peu au-dessous de la classe moyenne, et une petite part même à ceux qui sont absolument au dernier rang, de simples manouvriers possédant quelquefois en propriété une acre ou deux de terre. Ainsi, quoique la dépense de ces classes infé­rieures, en ne voyant que l’individu, soit fort peu de chose, cependant la niasse totale de cette dépense, en prenant ces classes collectivement, forme toujours la très-majeure partie de la dépense totale de la société ; ce qui reste du produit annuel des terres et du travail du pays pour la consommation des classes supérieures étant toujours de beaucoup moindre, non-seulement quant à la quantité, mais quant à la valeur. Ainsi, entre les impôts établis sur les dépenses, ceux qui portent principale­ment sur la dépense des classes supérieures, sur la portion la plus petite du produit annuel, promettent un revenu public beaucoup moindre que ceux qui portent indis­tinctement sur les dépenses communes à toutes les classes du peuple, ou même que ceux qui portent principalement sur la dépense des classes inférieures ; ceux-là doi­vent moins rendre que ceux qui portent indistinctement sur la totalité du produit annuel, ou même que ceux qui portent principalement sur la portion la plus forte de ce produit. Aussi, de tous les différents impôts mis sur la dépense, le plus productif, sans comparaison, est le droit d’accise sur les matières premières et la fabrication des liqueurs fermentées et spiritueuses qui se font dans le pays ; et cette branche de l’acci­se porte considérablement, on peut même dire principalement, sur la dépense des clas­ses les plus modestes de la population. Dans l’année qui a fini le 5 juillet 1775, le produit total ou brut de cette branche de l’accise s’est monté à 3,341,337 livres 9 schellings 9 deniers[66].
Il faut toujours se rappeler cependant qu’il n’y a que la dépense de luxe des classes inférieures du peuple, et non celle de nécessité, qui doive être imposée. Tout impôt sur leur dépense nécessaire porterait tout entier en définitive sur les classes supé­rieures, sur la portion la plus petite du produit annuel, et non sur la plus forte. Un im­pôt de ce genre a nécessairement, dans tous les cas, pour effet d’élever les salaires ou de diminuer la demande du travail. Il ne pourrait pas faire hausser les salaires du travail sans rejeter sur les classes supérieures la charge finale de l’impôt. Il ne pourrait pas diminuer la demande de travail sans affaiblir le produit annuel des terres et du travail du pays, la source qui nécessairement fournit, en dernière analyse, à tous les impôts. Quel que puisse être l’état auquel un impôt de ce genre réduise la demande de travail, cet impôt a toujours nécessairement l’effet d’élever les salaires plus haut qu’ils n’auraient été sans lui dans cet état ; et il faut nécessairement, dans tous les cas, que le payement de cette élévation de salaire retombe en dernier résultat sur les classes supérieures du peuple.
Les liqueurs fermentées et les liqueurs spiritueuses que l’on fait chez soi, pour son usage particulier et non pour les vendre, ne sont assujetties à aucun droit d’accise dans la Grande-Bretagne. Cette exemption, dont l’objet est d’épargner aux ménages particuliers le désagrément des visites et des perquisitions du collecteur d’impôt, fait que la charge de ces droits porte souvent d’une manière bien plus légère sur les riches que sur les pauvres. Il n’est pas fort ordinaire, à la vérité, de distiller des liqueurs spi­ri­tueuses pour son usage particulier, quoique cela se fasse pourtant quelquefois. Mais dans la province, une grande partie des personnes de la classe moyenne, et presque tous les ménages riches et considérables, brassent leur bière chez eux. Par consé­quent, leur bière forte leur coûte 8 sch. par baril[67] de moins qu’elle ne coûte au brasseur ordinaire, auquel il faut son profit, sur l’impôt comme sur tous les autres frais dont il fait l’avance. Ainsi, ces ménages-là doivent boire leur bière à 9 ou 10 sch. au moins de meilleur marché par baril que ne revient une boisson de même qualité aux classes inférieures, qui pourtant trouvent plus commode d’acheter leur bière, petit à petit, à la brasserie ou au cabaret ; de même, la drêche qui se fait dans un ménage pour l’usage de la maison n’est pas assujettie aux visites et aux perquisitions du percepteur de l’impôt ; mais dans ce cas, il faut que la maison paie un abonnement de 7 sch. 6 den. par tête, pour l’impôt. Ces 7 sch. 6 den. forment le montant du droit d’accise sur dix boisseaux de drêche, et c’est sans doute tout ce que peuvent consommer les membres d’un ménage frugal, pris indistinctement, hommes, femmes et enfants. Mais dans de grandes et riches maisons de province, où l’on reçoit beaucoup de monde, les bois­sons faites de drêche qui se consomment entre les membres de la famille, ne forment qu’une très-petite partie de ce qui s’en boit dans la maison. Cependant, soit à cause de l’abonnement qu’il faut payer, soit pour d’autres raisons, il n’est pas, à beaucoup près, aussi ordinaire de faire chez soi de la drêche pour son usage, que d’y brasser de la bière. Il est difficile d’imaginer aucune bonne raison pour que ceux qui brassent ou qui distillent pour leur usage particulier ne soient pas assujettis à payer un abonne­ment de la même espèce.
On a dit souvent qu’au lieu de tous ces gros droits imposés sur la drêche, sur la bière et sur l’ale, on pourrait procurer un plus gros revenu à l’État par un droit bien plus léger imposé sur la drêche, attendu que les occasions de frauder sont bien plus aisées et plus fréquentes dans une brasserie que dans une fabrique de drêches, et attendu que ceux qui brassent pour leur usage particulier sont exempts de payer soit des droits, soit un abonnement pour les droits ; ce qui n’a pas lieu à l’égard de ceux qui font de la drêche pour leur usage particulier.
Dans la brasserie de porter à Londres, un quarter de drêche est ordinairement brassé en plus de deux barils et demi de porter, quelquefois en trois. Les différents impôts sur la drêche montent à 6 sch. par quarter[68] ; ceux sur la bière forte et l’ale à 8 sch. par baril[69]. Ainsi, dans une brasserie de porter, les différents impôts sur la drêche, la bière et l’ale vont de 26 à 30 sch. sur le produit d’un quarter de drêche. Dans les brasseries pour le débit ordinaire des provinces, un quarter de drêche n’est guère brassé en moins de deux barils de bière forte et un baril de petite bière ; souvent il l’est en deux barils et demi de bière forte. Les différents impôts sur la petite bière montent à 1 sch. 4 pence par baril[70]. Ainsi, dans les brasseries de province, les diffé­rents impôts sur la drêche, la bière et l’ale ne vont guère à moins de 23 sch. 4 den., et souvent ils vont à 26 sch. sur le produit d’un quarter de drêche. Par conséquent, en faisant une évaluation moyenne pour tout le royaume, le montant total des droits sur la drêche, la bière et l’ale ne peut être estimé à moins de 24 ou 25 sch. sur le produit d’un quarter de drêche. Or, en supprimant tous les différents droits sur la bière et sur l’ale, et en triplant la taxe sur la drêche, ou en la portant de 6 sch. à 18 sch. par quarter de drêche, on pourrait, à ce qu’on prétend, trouver, avec cette seule taxe, un plus gros revenu que celui qu’on retire à présent de toutes ces taxes plus fortes.
 	L	 s.	d.
En 1772, l’ancienne taxe[71] sur la drêche a produit	722,023	     11	11
      la taxe additionnelle[72]	356,776	7	9 3/4
En 1773, l’ancienne taxe a produit	561,627	3	7 1/2
      la taxe additionnelle	278,650	15	3 3/4
En 1774, l’ancienne taxe a produit	624,614	17	5 3/4
      la taxe additionnelle	310,745	2	8 1/2
En 1775, l’ancienne taxe a produit	657,357	0	8 1/4
      la taxe additionnelle	323,785
12
6 1/4
Quatre années	3,835,580
12
0 1/4
Taux moyen de ces quatre années	958,895
3
0 3/16
En 1772, l’accise des provinces	1,243,128	5	3
      les brasseries pour Londres	408,260	7	2 3/4
En 1773, l’accise des provinces	1,245,808	3	3
      les brasseries pour Londres	405,406	17	10 1/2
En 1774, l’accise des provinces	1,246,373	14	5 1/2
      les brasseries pour Londres	320,601	18	0 1/4
En 1775, l’accise des provinces	1,214,583	6	1
      les brasseries pour Londres	463,670
7
0 1/4
Quatre années	6,547,832
19
2 1/4
Taux moyen de ces quatre années	1,636,958	4	9 1/2
A quoi ajoutant le taux moyen ci-dessus de la taxe sur la drêche, ou	958,895	3	0 3/17
Le total de ces différents impôts monte à	2,595,853
7
9 11/16
Or, en triplant la taxe sur la drêche ou en la portant de 6 sch. à 18 sch. par quarter de drêche, ce seul impôt aurait produit	2,876,685	9	0 9/16
Somme qui excède la précédente de	280,832	1	 2 14/16

À la vérité, dans l’ancienne taxe sur la drêche est compris un droit de 4 schellings sur le muid de cidre[73], et un autre de 10 sch. sur celui du mum[74]. En 1774, la taxe sur le cidre ne produisit que 3 083 livres 6 schellings 8 deniers ; vraisemblablement elle fut au-dessous du produit auquel elle monte habituellement, tous les différents droits sur le cidre ayant rendu moins qu’à l’ordinaire cette année-là. Le droit sur le muni, quoi­que beaucoup plus fort, est encore d’un moindre produit, à cause du peu de con­som­mation qui se fait de cette boisson. Mais, pour balancer le montant ordinaire de ces deux taxes, quel qu’il puisse être, il y a aussi de compris dans ce qu’on appelle l’accise des provinces, 1° l’ancienne accise de 6 schellings 8 deniers sur le muid de cidre ; 2° une pareille taxe de 6 schellings 8 deniers sur le muid de verjus ; 3° une autre de 8 schellings 9 deniers sur le muid de vinaigre, et enfin une quatrième taxe de 11 pence sur le gallon d’hydromel[75]. Le produit de ces quatre différents impôts doit probable­ment faire plus que balancer le produit des droits imposés sur le cidre et sur le muni par ce qu’on appelle la taxe annuelle sur la drêche.
La drêche se consomme non-seulement pour la brasserie de la bière et de l’ale, mais encore pour la fabrication de ce qu’on appelle petits vins[76] et esprits[77]. Si l’impôt sur la drêche venait à être porté à 18 sch. par quarter, il paraîtrait nécessaire de faire quelque réduction sur les différents droits d’accise qui sont imposés sur ces différen­tes espèces particulières de petits vins et d’esprits dont la drêche compose un des éléments. Dans ce qu’on nomme esprit de drêche, elle ne fait pour l’ordinaire qu’un tiers des ingrédients, les deux autres tiers étant ou d’orge non fermentée, ou moitié orge et moitié froment. Dans les distilleries où se fait l’esprit de drêche, la facilité et la tentation de frauder les droits sont bien plus grandes l’une et l’autre que dans une bras­serie ou bien dans une fabrique de drêche : la facilité, à cause du plus petit volu­me de la marchandise et de sa plus grande valeur, et la tentation, à cause des droits qui sont plus forts et qui montent à 3 schellings 10 deniers 2/3 par gallon d’esprit[78]. En augmentant les droits sur la drêche et en réduisant ceux sur la fabrication des liqueurs distillées, on diminuerait à la fois et la facilité, et la tentation de frauder ; ce qui pourrait encore donner lieu d’autant à une augmentation de revenu public.
Il y a déjà quelque temps que l’intention de la législature est de décourager la consommation des liqueurs spiritueuses, parce qu’on suppose qu’elles tendent à ruiner la santé du peuple et à corrompre ses mœurs. D’après cette politique, il ne faudrait pas que la réduction des impôts sur les distilleries fût assez forte pour causer une dimi­nution dans le prix de ces liqueurs. Les liqueurs spiritueuses pourraient rester toujours aussi chères qu’elles l’aient jamais été, tandis qu’en même temps les boissons saines et fortifiantes, telles que la bière et l’ale, auraient considérablement baissé de prix. Ainsi, le peuple serait en partie soulagé de l’un des fardeaux dont il se plaint aujour­d’hui le plus, tandis qu’en même temps le revenu public recevrait une augmentation considérable.
Les objections du docteur Davenant contre cette réforme du système actuel des droits d’accise ne paraissent pas fondées. Ces objections consistent à dire que l’impôt, au lieu de se répartir, comme à présent, avec assez d’égalité sur le profit du fabricant de drêche, sur celui du brasseur et sur celui du débitant, porterait en entier, pour ce qui doit atteindre le profit, sur celui du fabricant de drêche ; que le fabricant de drêche ne pourrait pas si aisément retirer le montant de l’impôt en élevant le prix de sa drêche, que le font le brasseur et le débitant en augmentant le prix de la boisson, et qu’un impôt aussi lourd sur la drêche pourrait faire diminuer le revenu et le profit des terres cultivées en orge.
Un impôt ne peut jamais réduire pour un temps considérable le taux du profit dans un commerce ou métier particulier, celui-ci devant toujours garder son niveau avec les autres commerces et métiers du canton. Les droits actuels sur la drêche, la bière et l’ale n’ont pas d’effet sur les profits de ceux qui trafiquent sur ces sortes de denrées, lesquels se remboursent tous de l’impôt, avec un profit additionnel, par l’élévation du prix de leur marchandise. À la vérité, un impôt pourrait rendre la marchandise sur laquelle il est établi tellement chère, qu’il en diminuerait la consommation. Mais la consommation de la drêche se fait en boissons et liqueurs composées avec cette denrée, et un impôt de 18 schellings par quarter de drêche ne pourrait guère rendre ces boissons plus chères que les différentes taxes d’à présent, montant à 24 ou 25 sch., ne peuvent le faire. Ces boissons, au contraire, tomberaient probablement à meilleur marché, et il a lieu de supposer que la consommation en augmenterait plutôt que de diminuer.
Il n’est pas très-aisé de comprendre pourquoi le fabricant de drêche trouverait plus de difficulté à se rembourser de 18 sch. par une élévation dans le prix de sa drêche, que n’en trouve à présent le brasseur à se rembourser de 24 ou 25, quelquefois de 30 sch., par l’accroissement du prix de sa boisson. Le fabricant de drêche, à la vérité, au lieu d’un droit de 6 sch., serait obligé d’en avancer un de 18 sch., sur chaque quarter de drêche ; mais le brasseur est obligé à présent d’avancer un droit de 24 ou 25, quelquefois de 30 sch. sur chaque quarter de drêche qu’il brasse en boisson. Il n’y aurait pas pour le fabricant de drêche plus d’incommodité à faire l’avance d’un impôt plus faible, qu’il n’y en a aujourd’hui pour le brasseur à faire l’avance d’un plus fort. Le fabricant n’est pas absolument tenu de garder dans ses greniers une provision de drêche qui attende plus longtemps le débit, que ne l’attend la provision de bière et d’ale dans les celliers du brasseur. Ainsi, le premier peut souvent avoir la rentrée de ses fonds aussi promptement que l’autre. Mais quelque inconvénient qu’il pût y avoir pour le fabricant de drêche à être obligé de faire l’avance d’un impôt plus lourd, il serait aisé d’y remédier en lui accordant quelques mois de plus de crédit que ce qu’on en accorde aujourd’hui communément au brasseur. 
Il n’y a autre chose qu’une diminution dans la demande de l’orge, qui puisse dimi­nuer la rente et le profit des terres ensemencées en cette nature de grain. Or, un changement de système qui réduirait de 24 ou 25 sch. à 18 sch. seulement les droits imposés sur un quarter de drêche brassé en bière ou en ale, serait dans le cas d’augmenter la demande plutôt que de la diminuer. D’ailleurs, il faut toujours que la rente et le profit des terres en orge soient à peu près égaux à ceux des autres terres également bien cultivées. S’ils étaient au-dessous, fi y aurait bientôt une partie des terres en orge qui serait consacrée à une autre culture ; et s’ils étaient plus forts, il y aurait bientôt plus de terre employée à produire de l’orge. Quand le prix ordinaire de quelque produit particulier de la terre est monté à ce qu’on peut appeler prix de monopole, un impôt sur cette production fait baisser nécessairement la rente et le profit de la terre où elle croît. Si l’on mettait un impôt sur le produit de ces vignobles précieux dont les vins sont trop loin de remplir la demande effective pour que leur prix ne monte pas toujours au-delà de la proportion naturelle du prix des productions des autres terres également fertiles et également bien cultivées, cet impôt aurait nécessairement l’effet de faire baisser la rente et le profit de ces vignobles. Le prix de ces vins étant déjà le plus haut qu’on puisse en retirer relativement à la quantité qui en est communément envoyée au marché, il ne pourrait pas s’élever davantage, à moins qu’on ne diminuât cette quantité. Or, on ne saurait diminuer cette quantité sans qu’il en résultât une perte encore plus forte, parce que la terre où ils croissent ne pourrait pas être remise en un autre genre de culture dont le produit fût de valeur égale ; ainsi, tout le poids de l’impôt porterait sur la rente et le profit du vignoble ; à vrai dire, il porterait sur la rente. Chaque fois qu’on a proposé d’établir un nouvel impôt sur le sucre, nos planteurs se sont toujours plaints que le poids de ces sortes d’impôts portait en entier sur le producteur et nullement sur le consommateur, celui-là n’ayant jamais trouvé moyen d’élever le prix de son sucre, après l’impôt, plus haut qu’il n’était aupa­ra­vant. Le prix aurait donc été, avant l’impôt, à ce qu’il semble, un prix de monopole, et l’argument qu’on mettait en avant pour prouver que le sucre n’était pas un article propre à être imposé, était peut-être une bonne démonstration du contraire, les gains des monopoleurs, de quelque part qu’ils puissent venir, étant certainement l’objet le plus propre à supporter une imposition. Mais le prix ordinaire de l’orge n’a jamais été un prix de monopole ; la rente et le profit des terres en orge n’ont jamais été au-delà de leur proportion naturelle avec ceux des autres terres également fertiles et également bien cultivées. Les différents impôts qui ont été établis sur la drêche, la bière et l’ale, n’ont jamais fait baisser le prix de l’orge, n’ont jamais réduit la rente et le profit des terres en orge. Le prix de la drêche a monté certainement, pour le brasseur, à propor­tion des impôts mis sur cette denrée ; et ces impôts, ensemble les différents droits sur la bière et l’ale, ont constamment fait monter le prix de ces denrées pour le consom­mateur, ou bien, ce qui revient au même, ils en ont fait baisser la qualité. Le payement définitif de ces impôts est retombé constamment sur le consommateur et non sur le producteur.
Les seules personnes qui seraient dans le cas de souffrir du changement de systè­me qu’on propose ici, ce sont celles qui brassent pour leur usage particulier. Mais l’exemption dont les classes supérieures du peuple jouissent aujourd’hui d’impôts très-lourds qui sont payés par l’ouvrier et l’artisan, est certainement la faveur la plus injuste et la plus contraire à l’égalité ; il faudrait la supprimer, même quand le change­ment proposé ne devrait jamais avoir lieu. C’est pourtant vraisemblablement l’intérêt de cette classe supérieure qui a empêché jusqu’à présent une réforme propre à amener à la fois de l’augmentation dans le revenu de l’État et du soulagement pour le peuple.
Outre ces sortes de droits, tels que ceux d’accise et de douane mentionnés ci-dessus, il y en a plusieurs autres qui influent sur le prix des marchandises d’une ma­nière plus inégale et plus indirecte. De ce genre sont les droits qu’on nomme en France péages, qui étaient nommés droits de passage au temps des anciens Saxons, et qui semblent avoir été, dans l’origine, établis pour le même objet que nos droits de barrières, ou ceux perçus sur les canaux et les rivières navigables, en vue de pourvoir à l’entretien de la route ou de la navigation. La manière la plus convenable d’imposer ces droits, quand ils sont appliqués à leur véritable objet, est de taxer la marchandise d’après son volume ou son poids. Comme c’étaient, dans l’origine, des droits locaux et provinciaux destinés à des dépenses locales et provinciales, la régie en fut confiée le plus souvent à la ville, paroisse ou seigneurie particulière dans laquelle ils étaient per­çus, ces communautés étant censées, d’une manière ou de l’autre, responsables du juste emploi des derniers. Le souverain, qui n’est tenu à aucune responsabilité, s’est emparé, dans plusieurs pays, de la régie de ces droits ; et quoiqu’il ait, la plupart du temps, extrêmement augmenté le droit, il a fort souvent négligé totalement d’en faire la juste application. Si jamais les droits qui se perçoivent aux barrières des grandes routes, en Angleterre, venaient à faire une des ressources du gouvernement, il ne faut que l’exemple de tant d’autres nations pour nous faire voir quelles en seraient vraisemblablement les conséquences. Ces sortes de droits sont, sans contredit, payés en définitive par le consommateur, mais le consommateur n’est pas imposé à propor­tion de la dépense qu’il fait au moment où il paie ; il n’est pas imposé d’après la valeur, mais d’après le poids ou le volume de la chose qu’il consomme. Lorsque de tels droits sont réglés, non sur le poids ou le volume des marchandises, mais sur leur valeur présumée, alors ils deviennent proprement une sorte de droit d’accise ou de droit de douane intérieure, qui entrave successivement la plus importante de toutes les branches de commerce, c’est-à-dire le commerce intérieur du pays.
Dans quelques petits États, il y a des droits semblables à ces droits de passage, imposés sur les marchandises qui traversent le territoire, par terre ou par eau, pour passer d’un pays étranger dans un autre. Ces droits se nomment, dans certains pays, droits de transit. Quelques-uns des petits États d’Italie, qui sont situés sur les bords du Pô et des rivières qui se jettent dans ce fleuve, tirent un revenu de droits de cette espèce. Ces droits sont supportés en entier par les étrangers, et ce sont peut-être les seuls droits qu’un État puisse imposer sur les sujets d’un autre, sans mettre aucune espèce d’entrave à l’industrie ou au commerce des siens. Le droit de transit le plus important qui existe dans le monde est celui que lève le roi de Danemark sur tous les vaisseaux marchands qui traversent le Sund[79].
Quoique ces sortes d’impôts sur les objets de luxe, tels que sont la plus grande partie des droits de douane et d’accise, portent indistinctement sur toutes les diffé­rentes espèces de revenu, et soient payés définitivement et sans répétition par le consommateur quelconque des marchandises sur lesquelles ils sont établis, cependant ces impôts ne portent pas d’une manière égale ou proportionnée sur le revenu de chaque individu. Comme c’est le caractère et le penchant naturel de chaque homme qui détermine le degré de consommation qu’il fait, chaque homme se trouve contri­buer plutôt selon la nature de ses inclinations que selon son revenu. Le prodigue contribue au-delà de la juste proportion ; l’homme parcimonieux contribue en deçà de cette proportion ; pendant sa minorité, un homme doué d’une grande fortune contribue ordinairement de fort peu de chose, par sa consommation, au soutien de l’État, dont la protection est pour lui la source d’un gros revenu. Ceux qui résident en pays étranger ne contribuent en rien, par leur consommation, au soutien du gouvernement du pays dont ils tirent leur revenu. Si, dans ce dernier pays, il n’y avait pas d’impôt territorial ni aucun droit considérable sur les mutations des propriétés mobilières ou immobi­lières, comme cela est en Irlande, des personnes absentes pourraient ainsi jouir d’un gros revenu à la faveur de la protection d’un gouvernement aux besoins duquel elles ne contribueraient pas pour un sou. Cette inégalité sera vraisemblablement plus forte qu’ailleurs dans un pays dont le gouvernement est à quelques égards subordonné et dépendant du gouvernement d’un autre pays. Les personnes qui possèdent les propriétés les plus étendues dans le pays dépendant aimeront mieux, en pareil cas, choisir leur résidence dans le pays qui gouverne. L’Irlande est précisément dans cette situation, et ainsi il ne faut pas nous étonner si la proposition de mettre un impôt sur les absents est, dans ce pays, si favorablement accueillie par l’opinion publique. Il serait peut-être assez difficile de constater quelle sorte ou quel degré d’absence devrait mettre un homme dans le cas d’être imposé comme absent, ou bien à quelle époque précise l’impôt serait réputé commencer ou cesser. Néanmoins, si vous en exceptez cette situation tout à fait particulière, toute espèce d’inégalité dans la contribution individuelle des particuliers qui peut naître de ces sortes d’impôts se trouve plus que compensée par la circonstance même qui est la source de ces inéga­lités : la circonstance que la contribution de chaque individu est absolument volontaire, cet individu étant parfaitement le maître de consommer ou de ne pas consommer la marchandise sujette à l’impôt. Aussi, quand ces sortes d’impôts sont assis d’une manière convenable, et qu’ils ne portent que sur des marchandises propres à être imposées, ils sont partout payés avec moins de murmure que tout autre. Quand ils sont avancés par le marchand ou le manufacturier, alors le consommateur qui les paie en définitive vient bientôt à les confondre avec le prix même de la marchandise, et à ne s’apercevoir presque pas qu’il paie l’impôt.
Ces sortes d’impôts sont ou peuvent être parfaitement exempts de toute incerti­tude, c’est-à-dire qu’ils peuvent être assis de manière à ne laisser aucun doute, ni sur ce qu’il y a à payer, ni sur le moment où il faut payer, ni sur la quotité ni sur l’époque du payement. Quelles que soient les incertitudes qui puissent se présenter quelquefois, ou dans les droits de douane de la Grande-Bretagne, ou dans les autres droits de même espèce établis dans d’autres pays, elles ne proviennent nullement de la nature de ces impôts, mais d’un défaut d’exactitude ou de précision dans les termes de la loi qui les a établis.
Les impôts sur les objets de luxe se payent en général, et peuvent toujours se payer petit à petit, ou bien au fur et à mesure que le contribuable a besoin d’acheter les objets sur lesquels ces impôts sont établis. Ils sont, ou au moins ils peuvent être les plus commodes de tous les impôts pour l’époque et pour le mode de payement. Ainsi, en résumé, ces sortes d’impôts sont peut-être aussi conformes que tout autre aux trois premières des quatre règles générales qui concernent les impositions. Ils choquent, sous tous les rapports, la quatrième de ces règles.
À proportion de la somme que ces sortes d’impôts font entrer dans le Trésor public de l’État, ils prennent plus d’argent au peuple ou lui en tiennent plus longtemps hors des mains, que ne fait presque toute autre espèce d’impôt. Ils produisent ce mau­vais effet, à ce qu’il semble, de toutes les quatre manières différentes dont il soit possible de le produire.
Premièrement, la perception de ces sortes d’impôts, même quand ils sont établis de la manière la plus judicieuse, exige un grand nombre de bureaux de douane et d’officiers d’accise, dont les salaires et les rétributions casuelles établissent sur le peu­ple un véritable impôt qui ne rapporte rien au Trésor de l’État. Cependant il faut avouer que cette dépense est, en Angleterre, plus modérée que dans la plupart des autres pays. Dans l’année qui a fini au 5 juillet 1775, le produit total ou brut des différents droits qui sont sous la régie des commissaires de l’accise, en Angleterre, s’est monté à une somme de 5,507,300 liv. 18 sch. 8 d. 1/4, dont la perception ne coûta guère plus de 5 et demi pour 100. Il faut cependant déduire de ce produit total ce qui a été payé en primes et en restitutions de droits sur l’exportation des marchandises sujettes à l’accise ; ce qui abaissera le produit net au-dessous de 5 millions[80]. La per­cep­tion du droit sur le sel, qui est aussi un droit d’accise, mais qui est sous une régie différente, est beaucoup plus dispendieuse. Le revenu net des douanes ne monte pas à 2 millions et demi, dont la perception coûte plus de 10 pour 100 en salaires d’em­ployés et autres accessoires[81]. Mais les rétributions casuelles des employés des douanes sont partout beaucoup plus fortes que leurs salaires ; dans certains ports, elles sont plus du double ou du triple de ces salaires. Ainsi, si les salaires des employés et autres dépenses à accessoires montent à plus de 10 pour 100 sur le revenu net des douanes, la totalité des frais de perception, tant en salaires qu’en casuel, peut bien aller à plus de 20 ou 30 pour 100. Les employés de l’accise ne reçoivent que peu ou point de rétributions casuelles, et l’administration de cette branche du revenu public étant, comme établissement, d’une date plus récente, est en général moins entachée de corruption que celle des douanes, dans laquelle le temps a introduit et comme auto­risé une foule d’abus. On croit qu’en reportant sur la drêche tout le revenu qui se perçoit à présent par les différents droits sur la drêche et les liqueurs et boissons de drêche, il y aurait à faire, sur les frais annuels de l’accise, une économie de 50,000 livres. On ferait encore vraisemblablement une beaucoup plus grande économie dans les frais annuels des douanes, en bornant les droits de douane à un petit nombre d’espèces de marchandises, et en faisant la perception de ces droits suivant les lois de l’accise.
Secondement, ces sortes d’impôts occasionnent nécessairement quelques entraves ou quelques découragements dans certaines branches d’industrie. Comme ils font toujours hausser le prix de la marchandise imposée, ils en découragent d’autant la consommation et, par conséquent, la production. Si c’est une marchandise du sol ou des fabriques du pays, il en arrivera que moins de travail sera employé à la faire croître ou à la produire. Si c’est une marchandise étrangère dont l’impôt augmente ainsi le prix, les marchandises de même sorte qui se font dans le pays pourront, à la vérité, gagner par là quelque avantage sur le marché intérieur, et il pourra se faire qu’à l’aide de ce moyen une plus grande quantité de l’industrie nationale se tourne vers la fabrication de cette marchandise. Mais si cette élévation de prix sur une marchandise étrangère peut encourager l’industrie nationale dans une branche particulière, il dé­cou­rage nécessairement cette industrie dans presque toute autre branche. Plus le manu­facturier de Birmingham achètera cher son vin étranger, plus alors cette partie de ses quincailleries avec lesquelles, ou ce qui revient au même, avec le prix des­quelles il l’achète, sera nécessairement vendue à bon marché. Par conséquent, cette partie de ses quincailleries se trouvera être pour lui d’une moindre valeur, et il sera d’autant moins encouragé à la fabriquer. Plus les consommateurs d’un pays payent cher le produit surabondant d’un autre, plus ils vendent nécessairement à bas prix cette partie de leur propre produit surabondant avec lequel, ou ce qui est la même chose, avec le prix duquel ils l’achètent. Cette partie de leur produit surabondant de­vient alors pour eux d’une moindre valeur, et ils sont moins encouragés à en aug­menter la quantité. Par conséquent, tout impôt sur les choses de consommation tend à réduire, au-dessous de ce qu’elle serait sans cela, la quantité de travail productif em­ployée soit à préparer la marchandise imposée, c’est une marchandise du produit du pays, soit à préparer celles avec lesquelles elle est achetée, si c’est une mar­chandise étrangère. De plus, ces impôts dérangent toujours plus ou moins la direction naturelle de l’industrie nationale, et la forcent à prendre une direction toujours différente et, en général, moins avantageuse que celle qu’elle aurait suivie d’elle-même.
Troisièmement, l’espoir d’échapper par les fraudes et les contrebandes à ces sortes d’impôts donne fréquemment lieu à des confiscations, à des amendes et à d’autres peines qui ruinent totalement le délinquant, homme sans contredit extrêmement blâ­ma­ble d’enfreindre les lois de son pays, mais qui néanmoins se trouve être fort souvent une personne incapable de violer celles de la justice naturelle, et née pour faire, à tous égards, un excellent citoyen, si les lois de son pays ne se fussent avisées de rendre criminelles des actions qui n’ont jamais reçu de la nature un tel caractère. Dans ces gouvernements corrompus, qui donnent lieu tout au moins de soupçonner de grandes profusions et d’énormes abus dans l’application du revenu public, les lois établies pour protéger ce revenu sont très-peu respectées. Il n’y a pas beaucoup de gens qui se fassent conscience de frauder les droits quand ils trouvent une occasion sûre et facile de le faire sans se parjurer. Témoigner quelque scrupule d’acheter des marchandises de contrebande (ce qui est pourtant évidemment encourager la violation des droits de l’impôt et le parjure que cette violation entraîne toujours avec elle) serait regardé, dans presque tous les Pays, comme un de ces traits de pédantisme et d’hypo­crisie qui, bien loin de faire un bon effet sur l’esprit de personne, ne servent qu’à don­ner une opinion plus désavantageuse de la probité de celui qui affecte un tel rigorisme de morale. Cette indulgence du public encourage le contrebandier à conti­nuer un mé­tier dans lequel on l’accoutume à ne voir, en quelque sorte, qu’une inno­cente indus­trie, et quand il a se trouve près d’être atteint par les rigoureuses lois de l’impôt, il est le plus souvent disposé à défendre par la force ce qu’il a pris l’habitude de considérer comme sa propriété. Après avoir débuté par être souvent plutôt imprudent que crimi­nel, il finit presque toujours par devenir un des plus audacieux et des plus déterminés violateurs de lois de la société. Par la ruine du contrebandier, son capital, qui avait servi auparavant à entretenir du travail productif, se trouve absorbé ou dans le revenu de l’État, ou dans celui d’un officier du fisc, et sert ainsi à entretenir du travail non productif au détriment de la masse des capitaux de la société, ainsi que de l’industrie utile qu’un tel capital aurait pu mettre en activité.
Quatrièmement, ces sortes d’impôts, en assujettissant les citoyens, ou au moins ceux qui, par métier, tiennent les marchandises imposées à des visites fréquentes et à des recherches toujours odieuses de la part des percepteurs de l’impôt, exposent de temps en temps ces citoyens à quelques vexations, et toujours pour le moins à beaucoup d’embarras et d’importunités. Et si ces importunités, comme on l’a déjà dit, ne sont pas, rigoureusement parlant, une dépense, elles sont du moins équivalentes à la somme que chacun donnerait volontiers pour s’en exempter. Les lois de l’accise, quoiqu’elles aillent plus sûrement au but pour lequel elles ont été faites, sont, sous ce rapport, plus vexatoires que celles des douanes. Quand un marchand a importé des marchandises sujettes à quelques droits de douanes, qu’il a payé ces droits et serré ses marchandises dans son magasin, il n’est plus assujetti, la plupart du temps, à essuyer de nouveaux embarras ni aucune importunité de la part de l’officier de la douane. Il n’en est pas ainsi des marchandises sujettes aux droits d’accise. Les officiers de l’accise, avec leurs visites et leurs recherches continuelles, ne laissent pas au mar­chand ou fabricant un moment de repos. Les droits d’accise sont, pour cette raison, plus mal vus du peuple que les droits de douane, et il en est de même des officiers qui en font la perception. Quoiqu’en général ces officiers s’acquittent peut-être en tout point de leur devoir aussi bien que ceux des douanes, cependant, comme ce devoir les oblige à être fort souvent importuns à certains de leurs voisins, ils en contractent pour l’ordinaire, à ce qu’on croit, une sorte de rudesse de caractère moins commune parmi les autres. Toutefois, il se peut très-bien que cette observation ne soit qu’une pure suggestion de la part des marchands et fabricants prévaricateurs, dont la vigilance de ces officiers prévient ou découvre, souvent les fraudes.
Néanmoins, les inconvénients qui sont peut-être, jusqu’à un certain point, insépa­rables des impôts sur les consommations, sont pour le peuple d’Angleterre aussi légers et aussi peu sensibles qu’ils puissent l’être en tout pays où les besoins du gouvernement seront à peu près aussi exigeants. Notre régime n’est pas parfait et il serait susceptible d’amendement ; mais il est aussi bon ou meilleur même que celui de la plupart de nos voisins.
Par suite de l’opinion que les droits sur les marchandises de consommation étaient des impôts mis sur le profit des marchands, ces droits, dans certains pays, ont été répétés sur chaque vente successive de la marchandise. Si le profit du marchand impor­tateur ou du marchand fabricant subissait un impôt, l’égalité semblait exiger que le profit de l’acheteur intermédiaire qui intervenait entre l’un ou l’autre de ceux-là et le consommateur, fût pareillement imposé. C’est sur ce principe que paraît avoir été établi le fameux alcavala d’Espagne. Ce fut d’abord un impôt de 10 pour 100, ensuite de 14 pour 100 ; il est à présent seulement de 6 pour 100 sur la vente de toute espèce de propriété mobilière ou immobilière, et il est répété chaque fois que la propriété est vendue[82]. La perception de cet impôt exige une multitude d’agents qui puissent suffire à empêcher le transport d’une marchandise, non- seulement d’une province à l’autre, mais même d’une boutique à une autre. Il assujettit non-seulement ceux qui trafiquent sur certaines espèces de marchandises, mais encore ceux qui trafiquent sur quelque espèce que ce soit, tous les fermiers, tous les fabricants, tous les commerçants et les marchands en boutique, aux visites et recherches continuelles des percepteurs de l’impôt. Dans un pays où un pareil impôt est établi, on ne peut presque nulle part rien produire ni faire pour être vendu au loin. Il faut, dans toute l’étendue du pays, que le produit local se proportionne dans chaque endroit particulier à la consommation du lieu seulement. Aussi est-ce à l’alcavala que don Ustaritz impute la ruine des manu­fac­tures en Espagne. Il aurait pu aussi lui imputer de même le dépérissement de l’agriculture, car ce droit frappe non-seulement les articles de manufacture, mais encore le produit brut de la terre.
Dans le royaume de Naples, il y a un impôt de même nature de 3 pour 100 sur la valeur de toutes les conventions et, par conséquent, sur toutes les ventes. Il est moins lourd que celui d’Espagne ; et puis la plupart des villes et paroisses ont la faculté de payer un abonnement pour tenir lieu de cet impôt. Elles perçoivent cet abonnement dans la forme qui leur convient le mieux et en général de manière à ne donner aucune interruption au commerce intérieur du lieu. Ainsi l’impôt de Naples n’est pas, à beaucoup près, aussi ruineux que celui d’Espagne.
Le système uniforme d’imposition qui, a quelques légères exceptions près, existe dans toutes les différentes parties des royaumes unis de la Grande-Bretagne, laisse une liberté presque entière au commerce intérieur du pays, à celui qui se fait dans l’intérieur des terres, comme à celui qui se fait par les côtes. Le commerce intérieur par terre jouit d’une liberté presque parfaite, et la majeure partie des marchandises peuvent être transportées d’un bout du royaume à l’autre sans qu’il soit besoin d’aucun congé ou laissez-passer, et sans qu’on ait à essuyer aucune question, aucune visite ou recherche de la part des agents de l’impôt. Il y a un petit nombre d’exceptions, mais elles ne sont pas de nature à causer la moindre interruption à aucune des branches importantes du commerce qui se fait par terre dans l’intérieur du pays. À la vérité, pour les marchandises qui sont transportées par mer le long des côtes, on exige des certificats ou des acquits. Cependant, si l’on en excepte le charbon de terre, presque tout le reste est franc de droits. Cette liberté du commerce intérieur, qui est l’effet de l’uniformité du système d’imposition, est peut-être une des causes principales de la prospérité de la Grande-Bretagne, tout pays vaste étant nécessairement. le marché le plus étendu et le plus avantageux pour la majeure partie des productions de l’industrie nationale. Si l’on pouvait, par une suite de cette uniformité de système, étendre la même liberté de commerce à l’Irlande et aux colonies, il est probable qu’on verrait encore augmenter à la fois la grandeur générale de l’empire et la prospérité de cha­cune de ses parties.
En France, la diversité des lois fiscales établies dans les différentes provinces exige une foule d’employés pour border, non-seulement les frontières du royaume, mais celles de presque chaque province particulière, soit afin d’empêcher l’importa­tion de certaines marchandises, soit afin de les soumettre au payement de certains droits ; ce qui ne met pas peu d’obstacles au commerce intérieur du pays. Quelques provinces ont eu la faculté de s’abonner pour la gabelle ou impôt sur le sel ; d’autres en sont totalement exemptes. Quelques provinces sont affranchies de la vente exclusive du tabac dont jouissent les fermiers-généraux dans la plus grande partie du royaume. Les aides, qui répondent à l’accise d’Angleterre, sont très-différentes dans les diffé­rentes provinces. Quelques provinces en sont exemptes et payent un abonnement ou équivalent. Dans les provinces d’aides, où ces droits sont en ferme, il y a une quantité de droits locaux qui ne s’étendent pas au-delà d’une ville ou d’un district particulier. Les traites, qui répondent à nos droits de douane, divisent le royaume en trois grandes parties : la première comprend les provinces sujettes au tarif de 1664, qui sont appe­lées provinces des cinq grosses fermes, et dans lesquelles sont comprises la Picardie, la Normandie et la plupart des provinces de l’intérieur du royaume ; la deuxième com­prend les provinces sujettes au tarif de 1667, qui sont appelées provinces réputées étrangères, et renferme la plus grande partie des provinces frontières ; la troisième comprend les provinces traitées à l’égal de l’étranger, ou qui, parce qu’on leur laisse liberté de commerce avec les pays étrangers, sont assujetties, dans leur commerce avec les autres provinces de France, à tous les droits que payent les pays étrangers. Ces provinces sont l’Alsace, les trois évêchés de Metz, Toul et Verdun, et les trois villes de Dunkerque, Bayonne et Marseille. Dans les provinces dites des cinq grosses fermes (ainsi appelées à cause d’une ancienne division des droits de traites en cinq grandes branches dont chacune formait originairement une ferme particulière, et qui sont actuellement réunies en une seule), ainsi que dans les provinces dites réputées étrangères, il y a une quantité de droits locaux qui ne s’étendent pas au-delà d’une ville et d’un district particulier. Il y en a quelques-uns de ce genre, même dans les provinces dites traitées à l’égal de l’étranger, particulièrement dans la ville de Marseille. Il n’est pas besoin de faire remarquer jusqu’à quel point et les entraves du commerce intérieur du pays, et le nombre des employés, doivent être multipliés pour garder les frontières de ces différentes provinces et districts sujets à des systèmes d’imposition aussi variés.
Outre les entraves générales qui naissent de cette complication dans le système des lois fiscales, le commerce des vins, la production peut-être la plus importante de la France, après le blé, est assujetti, dans la plupart des provinces, à des entraves parti­culières provenant de la faveur accordée aux vignes de certaines provinces et can­tons particuliers de préférence à celles des autres pays. On trouvera, je crois, que les provinces dans lesquelles ce genre de commerce est le moins chargé d’entraves sont celles qui ont le plus de célébrité pour leurs vins. Le marché plus étendu dont jouissent ces provinces encourage la bonne administration pour la culture des vignes et pour la préparation ultérieure des vins.
Un système aussi bigarré et aussi compliqué dans les lois de l’impôt n’est pas une chose particulière à la France[83]. Le petit duché de Milan est divisé en six provinces, dans chacune desquelles il y a un système différent d’imposition à l’égard de plusieurs espèces différentes d’objets de consommation. Le territoire encore plus petit du duché de Parme est divisé en trois ou quatre sections, dont chacune a de même son système particulier d’imposition. Avec une administration aussi absurde, il faut tout l’avantage du climat et toute la fertilité du sol pour empêcher ces pays de retomber bientôt au dernier état de misère et de barbarie.
Les impôts sur les consommations peuvent se percevoir par une régie dont les administrateurs sont nommés par le gouvernement et sont immédiatement respon­sa­bles envers lui ; dans ce cas, le revenu est variable d’une année à l’autre, selon les varia­tions qui surviennent dans le produit de l’impôt ; ou bien ces impôts peuvent être affermés moyennant un prix fixe annuel, le fermier ayant la liberté de nommer ses employés, lesquels, bien qu’obligés de percevoir l’impôt selon les formes prescrites par la loi, sont néanmoins sous son inspection immédiate et sont immédiatement responsables envers lui.
La manière la plus avantageuse et la plus économique de percevoir un impôt ne peut jamais être de l’affermer. Outre ce qui est nécessaire pour payer le prix du bail, les salaires des employés et tous les frais d’administration, il faut toujours que le fermier retire encore du produit de l’impôt un certain profit tout au moins propor­tionné aux avances qu’il fait, aux risques qu’il court, à la peine qu’il se donne, ainsi qu’aux connaissances et à l’habileté qu’exige la conduite d’une affaire aussi fortement compliquée et d’un si grand intérêt. Le gouvernement, en établissant immédiatement et sous sa propre inspection une régie de la même espèce que celle qu’établit le fer­mier, pourrait au moins économiser ce profit qui est presque toujours exorbitant. Pour prendre à ferme une branche considérable du revenu public, il faut un grand capital et un grand crédit, circonstances qui seules seraient suffisantes pour restreindre la con­currence des entrepreneurs à un extrêmement petit nombre de personnes. Dans le petit nombre de personnes qui ont ce capital et ce crédit, un bien plus petit nombre encore a les connaissances et l’expérience nécessaires, autre circonstance qui resserre davantage la concurrence. Ce nombre si petit de personnes entre lesquelles est ainsi limitée la concurrence, trouve qu’il est bien plus de leur intérêt de se concerter ensem­ble, d’être coassociés au lieu d’être rivaux, et quand la ferme est mise aux enchères, de ne faire d’offres que fort au-dessous de la valeur réelle du bail. Dans les pays où les revenus de l’État sont affermés, les fermiers sont, en général, les hommes les plus opulents ; leurs richesses seules suffiraient pour exciter l’indignation publique, et la sotte vanité qui accompagne presque toujours ces fortunes subites, la ridicule osten­tation avec laquelle ils étalent pour l’ordinaire leur opulence, allument encore davantage cette indignation.
Le fermier du revenu public ne trouve jamais trop de rigueur dans les lois desti­nées à punir toute tentative faite pour échapper au payement de l’impôt. Il n’a pas d’entrailles pour des contribuables qui ne sont pas ses sujets, et qui pourraient tous faire banqueroute le lendemain de l’expiration du bail, sans que son intérêt en souffrît le moins du monde. Dans les moments où l’État a les plus grands besoins, et où néces­sairement le souverain est le plus jaloux que ses revenus soient payés avec exacti­tude, alors le fermier ne manque pas de crier qu’à moins de quelques lois plus rigoureuses que celles en vigueur, il lui sera impossible de payer même le prix ordinaire du bail. Dans ces instants de détresse publique, il n’y a guère moyen de disputer sur ce qu’il demande. En conséquence, les lois de l’impôt deviennent de plus en plus cruelles. C’est dans les pays où la plus grande partie du revenu public est en ferme qu’on trouvera toujours les lois fiscales les plus dures et les plus sanguinaires. Au contraire, les plus douces sont dans les pays où le revenu de l’État est perçu sous l’inspection immédiate du souverain. Un mauvais prince même sentira pour son peuple plus de compassion qu’on n’en peut jamais attendre des fermiers du revenu. Il sait bien que la grandeur permanente de sa famille est fondée sur la prospérité du peuple, et jamais il ne voudra sciemment, pour son intérêt personnel du moment, anéantir les sources de cette prospérité. Il n’en est pas de même des fermiers de son revenu ; c’est sur la ruine du peuple, et non pas sur sa prospérité, qu’ils trouveront le plus souvent à fonder leur fortune.
Quelquefois non-seulement l’impôt est affermé pour un prix fixe annuel, mais encore le fermier a de plus le monopole de la marchandise imposée. En France, les impôts sur le sel et sur le tabac sont levés de cette manière. En pareil cas, le fermier lève sur le peuple deux énormes profits au lieu d’un, le profit de fermier et le profit, encore bien plus exorbitant, de monopoleur. Le tabac étant un objet de luxe, on laisse chacun maître d’en acheter ou de n’en pas acheter, comme il lui plaît. Mais le sel étant un objet de nécessité, on oblige chacun d’en acheter du fermier une certaine quantité, parce que s’il n’achetait pas du fermier cette quantité, il serait présumé l’acheter de quelque contrebandier. Les droits sur l’une et l’autre de ces denrées sont excessifs. En conséquence, la tentation de frauder est irrésistible pour une foule de gens, tandis qu’en même temps la rigueur de la loi et la vigilance des employés font de cette tenta­tion la cause d’une ruine presque inévitable. La contrebande sur le sel et sur le tabac envoie chaque année aux galères plusieurs centaines de personnes, outre un nombre considérable qu’elle conduit au gibet. Ces impôts, levés de cette manière, rapportent au gouvernement un très-gros revenu. En 1767, la ferme du tabac fut affermée pour 22,541,278 livres tournois par an, celle du sel pour 36,492,404 livres tournois. Le bail pour l’un comme pour l’autre objet devait commencer en 1768 et durer six années. Ceux qui comptent pour rien le sang du peuple en comparaison du revenu du prince peuvent approuver peut-être cette méthode de lever l’impôt. Dans plusieurs autres pays, il a été établi sur le sel et sur le tabac des impôts et des monopoles semblables, particulièrement dans les domaines du roi de Prusse, dans ceux de l’Autriche et dans la plupart des États d’Italie.
En France, la plus grande partie du revenu actuel de la couronne provient de huit sources différentes : la taille, la capitation, les deux vingtièmes, les gabelles, les aides, les traites, le domaine et la ferme du tabac. Les cinq derniers objets sont affer­més dans la plupart des provinces. Les trois premiers sont perçus dans tout le royaume par une administration qui est sous la direction et l’inspection immédiate du gouvernement, et il est généralement reconnu qu’en proportion de l’argent que ces trois impôts lèvent sur le peuple, ils en font entrer dans le trésor du prince plus que ne font les cinq autres, dont l’administration est beaucoup plus chère et plus ruineuse.
Dans leur état actuel, les finances de France sont susceptibles de trois réformes très-simples et très-évidentes. Premièrement, en supprimant la taille et la capitation, et en augmentant le nombre des vingtièmes, de manière à ce qu’ils produisent un revenu additionnel égal au montant de ces autres impôts, on conserverait à la couronne tout son revenu ; on pourrait diminuer de beaucoup les frais de perception ; on épargnerait aux classes inférieures du peuple toutes les vexations que lui causent la taille et la capitation, et les classes supérieures pourraient n’être pas plus foulées qu’elles ne le sont aujourd’hui, pour la plus grande partie. Le vingtième, comme je l’ai déjà observé, est un impôt, à très-peu de chose près, de même nature que ce qu’on appelle la taxe foncière en Angleterre. Le fardeau de la taille, comme tout le monde en convient, retombe, en définitive, sur le propriétaire de la terre ; et comme la plus grande partie de la capitation est assise sur ceux qui sont sujets à la taille, à tant par livre de ce dernier impôt, le payement définitif de la majeure partie de celui-là doit pareillement retomber sur le propriétaire. Ainsi, quand on augmenterait le nombre des vingtièmes de manière à leur faire produire un revenu additionnel égal au montant de ces deux autres impôts, les classes supérieures pourraient n’être pas plus foulées qu’elles ne le sont aujourd’hui. Beaucoup de particuliers seraient, sans contredit, plus chargés qu’ils ne le sont, à cause des extrêmes inégalités avec lesquelles la taille est assise, pour l’ordi­naire, sur les terres et les fermiers des différents propriétaires. L’intérêt person­nel des sujets qui sont ainsi traités avec faveur, et l’opposition qu’ils ne manqueront pas de susciter, sont les premiers et les plus puissants obstacles que rencontrerait une telle réforme ou toute autre du même genre. Secondement, en soumettant à un régime uniforme, dans toutes les différentes parties du royaume, la gabelle, les aides, les impôts sur le tabac, tous les différents droits de traites et d’accise, ces impôts pour­raient être levés à beaucoup moins de frais, et le commerce intérieur du royaume pourrait devenir aussi libre que celui de l’Angleterre. Troisièmement enfin, en mettant tous les impôts sous une régie soumise à la direction et à l’inspection immédiate du gouvernement, les profits énormes des fermiers généraux pourraient bien être ajoutés aux revenus de l’État. Il y a à parier que l’opposition résultant de l’intérêt individuel de quelques particuliers ne réussira pas moins a empêcher ces deux projets de réforme que le premier.
Le système d’imposition établi en France paraît inférieur, à tous égards, à celui de la Grande-Bretagne. Dans la Grande-Bretagne, on lève annuellement 10 millions ster­ling[84] sur une population de moins de 8 millions de têtes, sans qu’on puisse dire qu’il y ait quelque classe particulière qui soit sous l’oppression. D’après les recherches de l’abbé d’Expilly et les observations de l’auteur de l’Essai sur la législation et le commerce des grains, il paraît vraisemblable que la France, y compris les provinces de Lorraine et de Bar, renferme environ 23 ou 24 millions d’habitants, trois fois peut-être autant qu’en contient la Grande-Bretagne. Le sol et le climat de France sont meilleurs que ceux de la Grande-Bretagne. Les progrès de la culture et de l’industrie y datent d’une époque beaucoup plus reculée, et la France est, par cette raison, mieux approvisionnée de toutes ces choses qui exigent un long temps pour être produites et accumulées, telles que les grandes villes et des maisons commodes et bien bâties, tant à la ville que dans les campagnes. En songeant à tous ces avantages, on aurait lieu de s’attendre qu’un revenu de 30 millions sterling, pour le soutien de l’État, pourrait être levé en France avec aussi peu de difficultés qu’un revenu de 10 millions l’est dans la Grande-Bretagne. Cependant, la totalité du revenu entrant dans le Trésor public de France, à l’époque de 1765 et 1766, d’après les meilleurs renseignements que j’ai pu me procurer (quoique j’avoue qu’ils sont encore très-imparfaits), varie pour l’ordinaire entre 308 et 325 millions de livres tournois, c’est-à-dire qu’elle n’allait pas à 15 millions sterling, pas à la moitié de ce qu’on aurait dû espérer si, relativement à la population, le peuple eût contribué dans la même proportion que le peuple de la Grande-Bretagne. C’est pourtant une chose généralement reconnue, qu’en France le peuple souffre infiniment plus d’oppression par les impôts que celui de la Grande-Bretagne. Néanmoins, après la Grande-Bretagne, la France est certainement, de tous les grands empires de l’Europe, celui qui jouit du gouvernement le plus doux et le plus modéré.
On dit qu’en Hollande les impôts très-lourds sur les choses de première nécessité ont ruiné les principales manufactures, et menacent peu à peu d’un semblable dépérissement jusqu’aux pêcheries et au commerce de la construction des vaisseaux. Les impôts sur les choses de nécessité sont peu importants dans la Grande-Bretagne, et jusqu’à présent ils n’ont amené la destruction d’aucun genre de manufacture ; les impôts qui pèsent de la manière la plus fâcheuse sur les manufactures en Angleterre, ce sont quelques droits sur l’importation de matières premières non ouvrées, particu­lière­ment sur les soies écrues. En Hollande, toutefois, le revenu des États-Généraux et des villes se monte, à ce qu’on dit, à plus de 5,250,000 liv. sterling ; et comme on ne peut guère supposer que la population des Provinces-Unies aille à plus d’un tiers de celle de la Grande-Bretagne, il faut que, proportion gardée, les habitants de ces pro­vinces soient beaucoup plus fortement imposés.
Après que tous les objets propres à supporter une imposition ont été épuisés, si les besoins de l’État viennent encore à exiger de nouveaux impôts, il faut bien les établir sur des objets qu’il ne serait pas convenable d’imposer. Ainsi, ces impôts sur les cho­ses de première nécessité ne sont pas un motif d’attaquer la sagesse de cette républi­que, qui, pour conquérir et maintenir son indépendance, s’est vue, malgré son extrême économie, entraînée dans des guerres dispendieuses et réduite à contracter des dettes immenses. Les seuls pays de la Hollande et de la Zélande exigent, en outre, une dépense considérable pour conserver leur existence, c’est-à-dire pour se préserver d’être engloutis par la mer ; ce qui doit avoir contribué à augmenter extrêmement la mas­se des impôts dans ces deux provinces. La forme républicaine du gouvernement semble être la principale base de la grandeur actuelle de la Hollande. Les proprié­taires de grands capitaux, les grandes familles commerçantes ont, en général, dans l’administration de ce gouvernement, ou une part directe, ou une influence indirecte. C’est en considération de l’autorité et de l’importance que cette situation leur procure, qu’ils se décident à vivre dans un pays où leur capital leur rendra moins de profits s’ils ne l’emploient eux-mêmes, et moins d’intérêts s’ils le prêtent à d’autres, et dans lequel le revenu extrêmement modique qu’ils peuvent retirer de ce capital achètera encore bien moins de choses utiles et commodes, qu’il n’en aurait acheté dans tout autre coin de l’Europe. En dépit de tous les désavantages du pays, la résidence de toutes ces personnes opulentes y tient toujours nécessairement en activité un certain degré d’industrie. Toute calamité publique qui détruirait la forme républicaine du gouver­nement, qui ferait tomber toute l’administration entre les mains de nobles et de militaires, qui anéantirait entièrement l’importance de ces riches commerçants, leur rendrait bientôt leur existence désagréable dans un pays où ils ne pourraient plus guè­re espérer une grande considération. Ils transporteraient aussitôt leur séjour ainsi que leurs capitaux dans quelque autre pays, et alors l’industrie et le commerce de la Hollande ne tarderaient pas à suivre les capitaux auxquels ils doivent leur activité[85].
 
 
 
↑ Voyez Mémoire concernant les droits et impositions en Europe, tome Ier, page 73. Cet ouvrage est une compilation faite par ordre de la cour pour servir à une commission créée, il y a quelques années, à l’effet de rechercher les moyens convenables de réformer les finances de France. L’état des impôts de la France, qui remplit trois volumes in-4°, peut être regardé comme parfaitement authentique ; celui des impositions des autres nations de l’Europe a été compilé d’après les informations qu’ont pu se procurer les ministres français auprès des différentes cours ; il est beaucoup plus court, et probablement il n’est pas tout à fait aussi exact que celui des impôts de la France. (Note de l’auteur.)
↑ Il existe dans nos finances quelques parties de revenu qui ont le caractère de revenu territorial. Tel est le produit des mines, qui est d’environ 180,000 fr. ; celui des salines de l’Est, dont la ferme annuelle est de 2,400 fr. ; quelques propriétés de Pondichéry, dont le gouvernement retire, année commune, 1,000 fr. ; et enfin les forêts de l’État, dont le produit brut annuel est de 18 millions et demi, qui peuvent être regardés comme autant de revenus fonciers. Des motifs d’intérêt public exigent que ces diverses propriétés restent entre les mains du gouvernement. Il faut, poulies grandes constructions et principalement pour celles de la marine, des bois conservés sur pied pendant plus d’un siècle, temps indispensable pour leur faire acquérir le volume, l’élévation et la solidité nécessaires; mais il est difficile que des particuliers puissent se soumettre à une aussi longue privation de revenu. À cela près de quelques cas d’exception qui sont rares, la propriété foncière n’est jamais plus profitable au pays que lorsqu’elle est placée sous la direction de l’intérêt privé.
Il y a en France une quantité assez considérable de terres qui sont réputées propriétés communales et qu’on a tenté plusieurs fois, sans succès, de convertir en propriétés individuelles, faute d’avoir su vaincre la résistance opposée par quelques intérêts locaux.
Ces terres, qui composent plusieurs millions d’arpents, sont, pour la plus grande partie, des vaines pâtures ou de mauvaises broussailles sans produit régulier. Le prétendu droit de propriété des communes n’est fondé que sur une possession immémoriale dont on doit rapporter l’origine à la fin du dixième siècle, époque à laquelle les seigneurs titulaires de bénéfices civils ou militaires usurpèrent l’hérédité de ces bénéfices et voulurent les assurer à l’ainé de leurs enfants mâles, conformément aux règles du droit féodal qui prit naissance à cette époque. Dans l’ancien droit des Francs, le souverain était censé propriétaire de la totalité des terres du royaume, et il concédait aux dignitaires de sa cour et aux principaux chefs de ses armées la jouissance de divers domaines, comme gage attaché à leurs offices ou fonctions. Mais lorsque, sur la lin de la seconde race, les seigneurs investis de ces bénéfices se liguèrent entre eux pour démembrer la couronne et se créer des fiefs héréditaires, pour trouver moins d’obstacle au succès de cette grande entreprise et mettre dans leurs intérêts la population des campagnes, ils délaissèrent aux habitants des communes et des villages toutes les portions de terres sur lesquelles ceux-ci avaient coutume de faire paître leurs bestiaux ou de ramasser du bois pour leur chauffage. C’est ainsi que les habitants d’une même commune commencèrent une possession collective qui s’est toujours continuée depuis, mais qui n’a jamais pu acquérir les véritables caractères d’une propriété. En effet, pour devenir propriétaire, même par prescription et sans titre direct, il faut être jouissant de ses droits ou faire partie d’une association légalement constituée. Ce ne peut être qu’en vertu de lettres de corporation régulièrement délivrées qu’une collection de personnes prend fictivement le caractère d’individualité et devient apte à exercer les droits et actions qui n’appartiennent qu’aux individus. Or, quoique longtemps après ce commencement de possession une grande partie des communes de France aient reçu de nos rois des chartes d’affranchissement et des concessions de libertés et de privilèges, avec l’autorisation de se choisir des magistrats ou syndics à l’effet de stipuler et défendre les droits de la communauté, cependant nous ne connaissons aucune commune dans laquelle les bourgeois soient constitués en corps de société, de manière à pouvoir posséder indivisément entre eux un corps quelconque de propriété foncière. Tout particulier peut s’établir dans la commune où il lui plaît rie faire sa résidence, et par son fait seul il devient membre de la commune, sans qu’il ait besoin du consentement ou de l’admission des autres habitants du même village, et sans que ceux-ci aient le droit de contester sa résidence et sa participation aux distributions de fouage et autres jouissances communes, sorte de droit purement précaire et de tolérance, qu’il perd de même par son éloignement de la commune, sans le pouvoir céder ou transmettre à personne. On voit qu’une jouissance de ce genre, quelque longue qu’elle puisse être, n’est pas de nature à pouvoir jamais prendre la consistance d’un droit de propriété. Dans les principes de la législation qui régit aujourd’hui la France, la commune n’est autre chose qu’une simple division administrative ou section de population, de même genre que les arrondissements et les cantons ; et les citoyens qui se trouvent compris dans cette division n’ont entre eux aucun lien d’association, soit conventionnelle, soit légale, qui puisse faire reposer sur leurs titres, ni séparément ni collectivement, un droit de propriété commune et indivise. Ainsi, ce qu’on nomme abusivement propriété communale ne réside réellement sur aucune tête ayant capacité légale de posséder. Eu rendant ces propriétés à la circulation générale, on n’attenterait au droit de personne ; on attacherait un véritable propriétaire à des propriétés qui n’en ont encore aucun, et on restituerait à la culture une quantité considérable de terre sans produit qui contribuerait à augmenter le revenu public et les revenus particuliers, qui ajouterait aux moyens de travail et de subsistance, et étendrait d’autant la masse de la population. Garnier.
↑ Voyez Mémoires sur les droits et impositions, tome Ier, page 73.
↑ Ou rixdales, valant environ 5 fr. 20 c.
↑ Adam Smith observe ici avec raison que le revenu du corps entier du peuple d’un pays est en raison, non de la rente de la terre, mais de son produit. La totalité du produit annuel des terres, ajoute-t-il, si on en excepte ce qui est réservé pour semences, est ou annuellement consommé par la masse du peuple, ou échangé contre quelque autre produit qui est consommé par elle. Ainsi, ou doit distinguer le revenu imposable, c’est-à-dire celui qui constitue le revenu disponible des propriétaires fonciers, d’avec le revenu ou produit annuel dont le corps de la nation tire sa subsistance.
En France, d’après les nombreux renseignements qui ont été recueillis pendant une suite d’années, par les commissaires du gouvernement, sur les bases combinées du prix courant des baux, de celui des ventes dos biens-fonds et des comparaisons des cantons cadastrés, et suivant le résultat que le ministère a publié en 1821, il paraît que le revenu imposable à la contribution foncière dans les quatre-vingt-six départements de la France peut être évalué à 1 milliard 580 millions. Mais, dans cette somme se trouve compris le revenu ou valeur locative des maisons et bâtiments portés sur les mêmes rôles de contribution que les terres cultivées et productives. Ces loyers ne sont toutefois que des revenus fictifs qui donnent bien un revenu au propriétaire auquel ils sont payés, mais qui ne donnent aucun revenu à la nation ; et comme ces sortes de biens ne produisent rien par eux-mêmes, il faut que ceux qui en acquittent le loyer tirent ce qu’ils payent pour cet article de leur dépense, de quelque source de revenu qui leur soit propre. On ne peut donc pas comprendre ce genre de produit dans le revenu national, dans lequel le peuple puise ses moyens de subsistance. On estime généralement que, dans la totalité des évaluations du revenu imposable à la contribution foncière, 1rs maisons et bâtiments entrent pour un cinquième. En partant de cette supposition, si de la somme ci-dessus de 1 milliard 580 millions on déduit un cinquième, les quatre cinquièmes restants, qui sont de 1 milliard 200 millions, peuvent être regardés comme la valeur du revenu des terres cultivées et productives, ou de la rente qu’en retirent les propriétaires fonciers, déduction faite de tous frais de culture. Cette évaluation se trouve assez conforme à celle qui fut faite en 1791 par Lavoisier. Les recherches et les travaux auxquels le savant académicien se livra à cette époque, avec un zèle et un dévouement sans bornes, pour arriver à une appréciation exacte du revenu net des terres productives, l’amenèrent à un résultat de 1 milliard 200 millions. Si maintenant on fait attention à la quantité de terrains non encore cultivés qui ont été mis en culture depuis ces trente années, tels que les enclos, cours et cloîtres des couvents et maisons religieuses, les cimetières, promenades, parcs, avenues et emplacements de châteaux et maisons d’agrément, on sera fermement convaincu qu’en n’ajoutant qu’un vingtième à l’évaluation de 1791, on reste encore au-dessous de la véritable valeur du produit de la France.
Eu partant, toutefois, de cette évaluation, la somme de 1 milliard 264 millions ne représenterait encore que la part du produit annuel qui est dévolue aux propriétaires fonciers, ce qui ne fournit qu’une portion aliquote du revenu total. En prenant en masse tout le territoire, et pour se prémunir contre toute exagération, on peut estimer la rente ou fermage du propriétaire au quart de la récolte. Dans des cantons fertiles et bien cultivés, le fermage va jusqu’au tiers de la récolte et même au delà ; il est rare qu’il descende fort au-dessous du quart. On ne courra donc pas le risque de porter trop haut le revenu territorial de la France, ou la masse de ses produits annuels, à quatre fois le produit net, ce qui forme un total de 5 milliards 56 millions *. .
Cette somme de 5 milliards 56 millions peut être regardée comme l’équivalent des fruits de toute sorte qui, année moyenne, à mesure des récoltes successives, entrent dans les granges, greniers et celliers des cultivateurs, en nature de blé, grains des diverses espèces, fourrages, fruits, légumes, crû de bestiaux, laines, lin, chanvre, soie, huile, bois, charbon et autres denrées consommables, de quelque nature que ce soit, ce qui forme le fonds sur lequel doit subsister tout le corps du peuple, ainsi que le fonds de toutes les matières premières sur lesquelles tous les genres d’industrie ont à s’exercer. Si cette somme de 5 milliards 56 millions est divisée par les trente millions d’individus de tout âge et de tout sexe dont on suppose que notre population est composée, on aura pour chaque tète une valeur de 168 fr. 50 c. ; et, en calculant pour chaque famille cinq individus, savoir, le père, la mère, les deux enfants destinés à les remplacer l’un et l’autre dans la génération suivante, et un troisième pour couvrir les chances ordinaires de la mortalité jusqu’à ce remplacement, on aura, pour chaque famille, un revenu de 842 fr. 30 c, provenant soit de rente de terre, soit de salaire, soit de profits de capital, soit enfin de quelque autre source de revenu, comme rente ou pension sur l’État ou sur les particuliers, traitement, gages, exercice d’industrie quelconque, etc.
Maintenant, il faut observer que cette masse de valeurs diverses, en entrant dans la circulation et par l’effet du mouvement général qui lui est imprimé, subit des changements continuels au moyen des échanges, et principalement par l’échange non interrompu des subsistances contre le travail. La portion de cette masse de produits, qui est en nature de blé ou autre substance alimentaire, va journellement se consommer pour nourrir les artisans et ouvriers des manufactures, qui au fur et à mesure remplacent leur consommation par une quantité correspondante d’ouvrage fait, et reportent ainsi sur la matière première qui a passé par leurs mains la valeur des substances qui leur ont été livrées. Ainsi, à mesure que décroit la masse du blé disponible qui se rend dans les marchés, il y a plus de laine, de lin, de soie, filés ou tissés en lainages, en toile, en rubans; plus de cuir préparé, plus de bois et de fer travaillé. La somme des valeurs est bien à peu près la même, au total ; mais dans le jeu de cette vaste machine, dont les innombrables ressorts sont dans une activité continuelle, toutes les valeurs soumises à son action changent sans cesse de forme, de nature, comme de place. La plupart de ces denrées travaillées se rapprochent de plus en plus des consommateurs, et pour arriver dans leurs mains elles passent successivement dans les ateliers du fabricant, puis dans les magasins du marchand en gros, puis enfin dans la boutique du détaillant, en acquérant toujours en valeur le déficit des subsistances consommées par les agents qui ont concouru à opérer ces transports, jusqu’à ce qu’enfin, quand elles ont achevé leur révolution à travers tous les canaux de l’industrie et du commerce, elles entrent dans le fonds de consommation de chaque individu ou de chaque ménage. Là, elles se réunissent et se confondent avec le fonds déjà existant en provisions de bouche, en vêtements, meubles, ustensiles provenant des revenus des années précédentes et qui ne sont pas entièrement consommés.
Une partie du revenu national, travaillée ou non travaillée, est envoyée au dehors pour y être échangée contre les productions étrangères qui entrent dans la consommation française.
Enfin, ce revenu fournit, non-seulement à toutes les dépenses privées, mais encore aux dépenses publiques, au moyen des taxes et impôts que prélève le gouvernement, tant sur la part disponible qui est dévolue aux propriétaires, aux fermiers, que sur toutes les autres parties de ce produit, même sur celles qui sont destinées à la nourriture et à l’entretien des ouvriers de la culture. Mais il ne faut pas perdre de vue que si l’impôt est une dépense pour celui qui le supporte, et que s’il opère à l’égard de celui-ci un retranchement dans ce qui était destiné à ses consommations personnelles, il ne forme point une diminution dans la masse du revenu national, et ne fait que transporter à une autre personne le droit de consommer ce qui a été perçu sur le redevable. L’impôt, loin de rien retrancher de la somme des objets consommables, est une source de revenu pour une partie considérable de la nation, tels que les créanciers de la dette publique, les pensionnaires de l’État, les fonctionnaires de tout ordre, les agents, préposés et salariés du gouvernement.
Les divers articles du revenu national qui ne sont pas consommés dans le cours de l’année par la personne qui avait droit de les consommer, forment un surcroit disponible pour la consommation de l’année suivante, et ils contribuent à composer, pour celui qui eu a fait l’épargne, un capital dont il peut, dans la suite, retirer un profit pour grossir d’autant son revenu privé et même ajouter au revenu national, en mettant en activité quelque nouvelle branche de travail. Quelque faible qu’on puisse supposer le montant de toutes ces épargnes partielles dans le cours d’une seule année, on sent néanmoins que, dans la durée d’un siècle, elles doivent donner lieu à une accumulation extrêmement considérable. Garnier.
↑ Le mot rente est toujours pris ici pour le revenu net de la terre.
↑ Si l’on applique au système actuel de nos impôts les quatre maximes établies en cet endroit par Adam Smith, on reconnaîtra que ce système est peut-être le moins défectueux qu’il soit possible d’adapter à un État aussi vaste, aussi riche, aussi peuplé, et dans lequel une grande variété de productions de la terre, du commerce et de l’industrie, ont fait naître tant de sortes diverses de richesses dont les éléments sont absolument inappréciables.
La première de ces maximes, qui veut que chaque citoyen soit imposé dans la proportion de ses facultés, ne peut guère s’entendre que des facultés apparentes et susceptibles d’évaluation, c’est-à-dire des facultés résultant d’une propriété foncière. Après plus de vingt-cinq années d’efforts et de tentatives dispendieuses, le gouvernement est parvenu à atteindre, autant que la sagesse d’une administration prudente et réservée doit la chercher, l’égalité de répartition de la contribution foncière entre les principales divisions du territoire. Cette contribution, qui, à compter du 1er juillet 1821, ne monte pas en principal au dixième du produit net, et qui, en y joignant les 40 centimes additionnels a ce principal, n’excède guère le huitième, paraîtra sans doute bien modérée, si on la compare aux charges dont est grevée la propriété foncière dans toutes les autres monarchies de l’Europe.
Quant aux facultés personnelles qui ne dérivent pas d’une possession territoriale, et qui sont à peu près impossibles à évaluer, puisqu’elles diffèrent dans des proportions considérables entre des fabricants ou des commerçants qui exercent le même genre de négoce ou d’industrie, d’après les quantités respectives de leur capital, de leur crédit et de plusieurs autres ressources qui échapperaient à toutes les investigations ; que même elles diffèrent entre des salariés du même métier lorsque l’un d’eux est chargé d’une famille nombreuse et que l’autre, veuf ou garçon, n’a que sa personne à entretenir, en sorte que le dernier trouve dans la même espèce et quantité de travail trois ou quatre fois plus de moyens de jouissance personnelle que le premier ; la seule voie qu’ait l’administration pour apprécier un genre de facultés qui se dérobe à toute espèce de recherches et qui se refuse à toute mesure générale, c’est de les juger par le signe le moins équivoque, par ses effets les plus ordinaires et les plus naturels, la consommation de l’individu, parce que dans le cours commun des choses, et à peu d’exceptions près,chaque individu est disposé à consacrer à des jouissances et commodités personnelles tout ce qui lui reste de disponible, quand il a satisfait aux besoins impérieux de la première nécessité. Ainsi, par des taxes sur le loyer, sur le mobilier, sur les boissons, les viandes, les assaisonnements, les denrées coloniales, les tabacs, les articles de vêtement et de chauffage, le gouvernement vient a bout de reprendre une portion du revenu disponible de chaque particulier, et de retrancher au profit de l’État le superflu des gains et bénéfices individuels. De même par des droits de greffe, de timbre ou d’hypothèque, par des taxes sur les transactions, promesses et contrats, il parvient à atteindre au passage certains capitaux mobiliers qui tendent toujours à se cacher, et qui ne se montrent au jour que par occasion et quand ils peuvent le faire avec profit.
La proportion entre la masse totale des impôts directs et la somme des taxes indirectes, telle qu’elle se trouve réglée dans notre système actuel d’impositions, parait être celle qui s’accorde le mieux avec les quantités respectives des fortunes immobilières et des revenus mobiliers et industriels. La contribution foncière n’entre guère que pour un quart dans la masse totale des impôts annuels. Ainsi, les taxes indirectes et droits de consommation, qui Ont pour but de compenser les inégalités inconnues et accidentelles des revenus privés de toute espèce, en frappant indistinctement sur tous les particuliers, quelle que soit la source dont ils dérivent leurs moyens de subsistance, sont, en somme totale, trois fois plus forts que la contribution foncière, assise sur une seule source particulière de revenu.
La connaissance certaine et précise de ce que chaque contribuable a à payer, ce qui fait l’objet de la deuxième maxime, est une condition qui se trouve parfaitement remplie par la publicité donnée à tous les règlements et à toutes les ordonnances relatives aux impositions. Tous les impôts, sans exceptions, sont établis par des lois généralement connues, et la quotité des droits est réglée par des tarifs que chacun peut consulter. Les taxes sur les articles de consommation journalière sont à la vérité le plus souvent avancées par les marchands qui débitent ces denrées, et qui font entrer le montant de leur avance dans le prix de la marchandise ; mais la libre concurrence dans tous les genres de commerce ne permet pas que le débitant élève le prix de la denrée au delà de ce dont elle est réellement renchérie par l’effet de la taxe, autrement l’abus serait facilement reconnu, et il s’exposerait à perdre ses pratiques.
La troisième maxime est celle dont on s’est le plus écarté dans notre système d’imposition ; mais il est aisé de voir que cette déviation est loin d’être au préjudice du gouvernement et des contribuables. La contribution foncière, ainsi que toutes celles qui sont levées sur des rôles nominatifs sur lesquels chaque contribuable est porté pour sa cote individuelle, sont exigibles par douzièmes de mois en mois, et non pas, comme le conseille Adam Smith, à l’époque des termes où se payent les loyers et les fermages. L’expérience a démontré les avantages de la méthode adoptée en France, et qui, à ce que je puis croire, est particulière à notre pays. Les fermages, dans la plus grande partie des départements, se payent par semestre ; les loyers de maison se payent à Paris par trimestre, et dans plusieurs autres grandes villes du royaume, l’usage est de les payer de six eu six mois. Si le payement de l’impôt foncier était réglé sur ces échéances, le contribuable aurait à payer à la fois une moitié de son imposition annuelle, au lieu que la charge, divisée par douzième, lui semble presque insensible ; et comme cette dette est prévue, il se prépare d’avance à l’acquitter. C’est ce morcellement de. l’impôt par petites fractions qui met le percepteur à portée de se prêter aux arrangements du redevable et de lui ménager quelques facilités, ce qui établit des relations de confiance entre le percepteur et les propriétaires de son canton. Aussi jamais, a aucune époque, la contribution foncière n’a été acquittée plus régulièrement et avec moins d’exercice des voies de contrainte ; et, lorsque l’année est terminée, il n’y a presque aucune partie du recouvrement en arrière, ce qui est certainement très-remarquable dans un pays où il n’y a pas moins d’un million de propriétaires fonciers, qui, entre eux tous, forment un nombre de cotes différentes qui n’est pas moindre de dix millions et demi.
Enfin, conformément à ce que prescrit la quatrième maxime, l’argent levé pour l’impôt se trouve à la disposition du Trésor au moment même de la perception. Les impôts directs et les taxes indirectes sont également versés dans les caisses des receveurs-généraux des finances, qui sont autant de caisses dépendantes du Trésor royal ; et comme ces receveurs font passer tous les dix jours au ministre l’état de leur situation, le gouvernement dispose aussitôt de tous les fonds libres pour effectuer les payements locaux, que le receveur exécute moyennant un droit de commission, ce qui épargne le transport des espèces ainsi que tous frais et retards inutiles.
Les formes de la perception des taxes sont aussi douces qu’il soit possible, et, à moins de violences exercées par les fraudeurs, les peines encourues se bornent le plus souvent à des confiscations et à des amendes. Les visites et perquisitions à domicile ne s’exercent que contre des débitants de boissons au détail, et cet assujettissement est un des inconvénients attaches au genre de commerce qu’ils ont entrepris et auxquels ils ont dû s’attendre. La vigilance et la sévérité des douanes ne s’exercent que sur un rayon peu étendu du voisinage des frontières, et ne peuvent causer aucune incommodité aux citoyens qui se soumettent aux lois et rougiraient de se livrer au métier honteux de contrebandier ou d’en favoriser les coupables manœuvres.
Il n’y a aucun système d’imposition, dans quelque pays que ce puisse être, qui ne soit susceptible de beaucoup d’objections et qui, sous beaucoup de rapports, ne donne prise à la critique ; mais on ne craint pas d’assurer qu’il n’en est aucun qui donne moins lieu aux plaintes et aux murmures du peuple, aucun qui s’approche le plus de la justice et de l’égalité, aucun enfin qui soit moins onéreux aux citoyens, à proportion des produits abondants et réguliers qu’il donne au gouvernement, que le système d’imposition auquel la France est soumise depuis le commencement de ce siècle. Garnier.

↑ Voyez Esquisse de l’histoire de l’homme, page 474 et suiv. (Cet ouvrage est de lord Kaimes.) (Note de l’auteur.)
↑ L’impôt foncier qui consisterait en une somme d’argent fixe et invariable assise sur chaque fonds de terre dans la proportion de son produit moyen et ordinaire, d’après une évaluation une fois faite, présente ce grand avantage, que le. propriétaire du fonds ainsi imposé n’est pas détourné de faire des amendements et améliorations sur sa terre par la crainte que le fisc ne lui enlève une partie des surcroîts de produit dont ces améliorations auront été la cause. Aussi cette considération a-t-elle séduit beaucoup de gens ; et, de nos jours, il semble que l’opinion la plus généralement répandue est celle qui préfère ce mode d’impôt à tout autre. L’Assemblée Nationale, qui fut réunie en France en 1789, était fortement préoccupée de cette idée, lorsque, changeant les formes établies jusqu’alors pour l’assiette de. l’impôt foncier, elle jugea à propos de fixer à 240 millions le total de la contribution foncière qu’elle croyait devoir être levée sur la France, dont le produit net avait été évalué à cinq fois cetle somme.
Ce contingent général une fois fixé, elle décréta qu’il serait réparti entre les divers départements et districts qui divisaient le territoire du royaume. Mais la nature n’a pas réparti la richesse foncière d’une manière uniforme sur la surface de la France ; et pour juger dans quel rapport la richesse territoriale d’un département en particulier était à la richesse totale du royaume, il aurait fallu des informations locales et des connaissances positives dont les premiers éléments étaient encore à chercher. Aussi la répartition présenta-t-ellc les inégalités les plus choquantes : tel département se trouva grevé au sixième, tel autre au seizième ou dix-septième de son produit net ; et, après trente ans, ce désordre subsistait encore, quelques moyens qu’on ait mis en œuvre pour y remédier. On a cru devoir recourir à un arpentage et à une évaluation générale des propriétés, mais cette entreprise gigantesque, qui avait déjà dévoré 40 millions, a donné des résultats si défectueux qu’on a été forcé de l’abandonner. Cet exemple, joint à tant d’autres, concourt à démontrer combien il serait difficile de concilier la fixité de l’impôt avec une égalité tant soit peu tolérable.
Mais, indépendamment même de cette considération, Adam Smith n’adopte point cette fixité de l’impôt, et présente d’autres objections qui la font rejeter. Il reconnaît l’avantage d’encourager le propriétaire à faire ses améliorations, en le débarrassant de tout partage avec le fisc, mais il pense aussi qu’il n’est pas bon de pousser cette mesure au point de rendre le souverain totalement désintéressé dans l’amélioration future des terres, et presque étranger aux progrès de l’agriculture dans ses États. Il regarde, au contraire, comme très-politique d’attirer l’attention du prince, par la vue de son propre intérêt, vers toutes les mesures propres à favoriser l’accroissement du revenu territorial. Il observe de plus que, dans une longue suite d’années, l’argent peut éprouver des variations dans sa valeur, ou la monnaie subir des changements ; ce qui mettrait l’impôt fixe au-dessous ou au-dessus de la limite dans laquelle on aurait eu l’intention de le circonscrire. Au reste, il propose (pag. 511) un moyen simple et facile d’obtenir tout l’avantage qu’on recherche dans la fixité de l’impôt ; ce serait d’affranchir de toute augmentation d’impôt, pour un certain nombre d’années, le propriétaire qui aurait déclaré l’intention où il serait de faire sur sa terre des améliorations.
La fixité de l’impôt présente encore une autre sorte d’inconvénient bien plus grave, que Smith n’a pas dû prévoir, et dont, après lui, les finances d’Angleterre ont fourni un exemple. L’impôt, ainsi converti en une redevance fixe, perd son véritable caractère, qui est celui d’un tribut annuel d’une portion du produit, d’un sacrifice momentané et volontaire, d’un secours accordé à l’État par les propriétaires ; il prend la forme d’une rente foncière ou cens perpétuel que le gouvernement est bientôt porté à considérer comme une propriété domaniale. Il n’y a qu’une telle illusion qui a pu déterminer M. Pitt, en 1798, à proposer au Parlement une des mesures les plus iniques et les plus attentatoires au droit sacré de la propriété, en faisant passer une loi qui oblige chaque propriétaire foncier à racheter la taxe foncière dont sa terre était grevée, à raison de vingt années de cette taxe, et six, dans un terme de cinq années ; et, à faute de faire, dans le temps prescrit, ses offres de rachat, autorise les commissaires nommés à cet effet, à mettre en vente ce prétendu capital. Toute personne a été admise à acquérir, et en vertu d’une telle acquisition, ce tiers acquéreur a pu devenir créancier privilégié du montant de la taxe annuelle, comme d’une rente réelle, foncière et perpétuelle, qui aurait été créée sur le fonds.
Le résultat de cette opération n’a été, en définitive, autre chose qu’une taxe ou subvention extraordinaire, levée sur les propriétaires fonciers, pour être employée au remboursement d’une partie de la dette publique. Le propriétaire qui a fourni ses deniers pour ce rachat n’a point libéré son domaine de la charge à jamais inhérente à la propriété foncière, la charge de contribuer directement ou indirectement aux besoins présents et futurs du gouvernement, de la protection duquel elle tient toute sa valeur. Cette condition est inséparablement attachée à la qualité de propriétaire, et on ne pourrait les disjoindre sans ébranler tous les fondements de l’édifice politique. Aussi, dès l’année qui suivit celle dans laquelle fut porté l’acte de rachat de la rente foncière, il fut établi une taxe sur les revenus, qui n’était qu’une véritable taxe foncière, pour la partie que les propriétaires de terre eurent à supporter dans ce nouvel impôt. Garnier.
↑ L’idée que le surplus que produit le sol, en dehors des salaires et profits, doive constituer le fond de toute imposition, ressort nécessairement de la théorie des Économistes, qui soutiennent que le sol est l’unique source de la richesse. Si A. Smith eût été suffisamment pénétré de l’erreur de cette théorie, il aurait vu qu’il n’était nullement nécessaire d’entamer une discussion métaphysique pour arriver à une réfutation complète. L’hypothèse des Économistes se base sur la considération que l’exploitation du sol crée un surplus net. Mais ce surplus naît, ainsi que nous l’avons démontré, du prix élevé des produits de la terre. L’avantage qui en résulte est donc tout entier pour le propriétaire, au préjudice du consommateur. La communauté ne gagne donc aucune augmentation de capital, puisque ce surplus n’est, en définitive, qu’un revenu transporté d’une classe à une autre ; il ne peut donc pas en conséquence fournir une nouvelle matière imposable. Le revenu a déjà existé entre les mains de ceux qui achètent ; il y serait resté si les produits fonciers étaient à un plus bas prix, et on aurait pu l’y imposer tout aussi bien qu’entre les mains des propriétaires fonciers, dans lesquelles le prix élevé de ces produits l’a fait passer. Il n’y a donc pas de raison de dire que le revenu de la communauté vient seulement du sol. Le sol fournit en effet des moyens de subsistance, et des matières premières ; mais le travail qui façonne ces matières crée également un revenu. Les revenus de la société viennent donc en partie du sol, et en partie du travail. Le sol, avec un bon système de culture, et le travail, quand il est judicieusement divisé, soutenu par l’emploi des machines, produisent donc en commun un accroissement de revenu ; la communauté devient plus riche, et c’est sur cette augmentation de richesses, qu’elle vienne de la culture du sol ou du travail, que se prélèvent les impôts. Quand les salaires excèdent ce qui est nécessaire pour la subsistance du cultivateur, pourquoi ne payerait-il pas l’impôt sur le surplus ? C’est ainsi qu’un système de taxation doit procéder, l’impôt enlève une part de leur revenu à ceux qui le payent, quelle que soit d’ailleurs la source de ce revenu. Le zèle des Économistes pour arriver à une application de leurs doctrines parait avoir été grand. Cette doctrine fut sur le point d’être mise en pratique dans l’administration de M. Turgot, qui fut contrôleur-général des finances en France, et qui dans ses écrits s’était toujours montré partisan du système de l’impôt territorial. Cette mesure fut empêchée par le renvoi de M. Turgot. L’administration de M. Turgot s’était signalée par une série d’actes salutaires au bien public ; mais dans cette occasion, il faut blâmer la légèreté avec laquelle cet homme d’État, sur la foi d’une doctrine non encore éprouvée, projetait une mesure qui aurait chargé une seule classe de la communauté de tout le fardeau des impôts. Buchanan.
↑ Mémoires concernant les droits et impositions, pages 240 et 241.
↑ L’idée d’améliorer l’agriculture en imposant des produits nous paraît tout à fait bizarre, et elle est, par parenthèse, en opposition directe avec le raisonnement de Smith, qui, dans un autre endroit, fait ressortir ce qu’il y a d’impolitique à vouloir contrôler les particuliers dans le maniement de leurs affaires. Le meilleur encouragement qu’on puisse donner à une industrie consiste à lui laisser la libre jouissance de ses produits. Imposer les produits du sol n’est pas, par conséquent, un moyen habile pour améliorer la culture, puisque, en diminuant la part du propriétaire foncier, il ne l’encourage point à l’exploitation de ses terres. Les propriétaires fonciers se soucient certainement moins de produire pour les autres que pour eux-mêmes, et ce seul effet de la taxe suffirait pour balancer la bonne influence qu’elle pourrait peut-être exercer sur l’agriculture, en produisant un meilleur mode de culture. L’impôt est rarement un bon instrument de régularisation, et quand la loi n’impose pas une prohibition directe, toute autre espèce de restriction indirecte peut être facilement éludée. Le législateur n’a pas à discuter les différents modes de culture; et quand il veut s’en mêler, il peut être sûr que, outre qu’il causera du préjudice aux autres, il donnera en même temps la mesure de sa propre ignorance. Buchanan.
↑ Il n’y a pas de raison pour que l’agriculture, plus qu’une autre industrie, demande les soins particuliers du souverain. Son amélioration appartient à une nombreuse classe d’individus, qui tous ont dans leurs affaires la même portion d’intelligence que les autres hommes dans les leurs ; et si, malgré leurs soins, l’agriculture ne prospère point, on ne gagnera probablement pas beaucoup à la confier à la surveillance insouciante et ignorante du gouvernement. Buchanan.
↑ La proposition d’Adam Smith est bonne en théorie, mais un impôt foncier variable est toujours une source de vexations, et très-souvent d’oppression ; une pareille taxe nuira à l’amélioration de la culture, le propriétaire n’ayant pas le même intérêt à améliorer s’il est obligé d’admettre l’État au partage des bénéfices. Il n’est pas du tout nécessaire que l’État ait sa part dans les revenus du sol. Si le revenu public suffit à tous les besoins, pourquoi en chercher davantage, et pourquoi l’État, en conséquence, chercherait-il à se mêler des améliorations du commerce ou de l’agriculture du pays*.  ? Buchanan.

↑ Mémoires concernant les droits, etc., t. Ier, p. 280, etc., et p. 287 jusqu’à 316.
↑ Mémoires concernant les droits, etc., t. II, p. 139 et suiv.
↑
Le point de vue pris par A. Smith dans le développement de l’opération des taxes sur les revenus des terres nous parait complètement faux. Il ne fait point de distinction entre les taxes sur les revenus proprement dits, c’est-à-dire sur les sommes que rapporte l’exploitation du sol, et les taxes sur les revenus, dans le sens populaire du mot, c’est-à-dire sur le total de la somme payée, non-seulement pour le sol, mais aussi pour les bâtiments, s’il y en a, les rigoles et haies, et les améliorations de tout genre. A. Smith a pensé que les taxes sur les revenus du sol, dans le sens populaire et étendu du mot, tombaient entièrement sur le propriétaire. Il est évident que ceci estime erreur. La somme payée aux propriétaires pour l’exploitation du sol serait entièrement absorbée par la taxe, qu’il serait impossible aux propriétaires de faire partager leurs charges aux autres ; mais, en tant que la rente compte dans le revenu du capital affecté aux améliorations et bâtiments, aucune taxe ne saurait l’enlever aux propriétaires.
Dans la pratique, il est impossible dans un pays ancien, organisé et bien cultivé, de diviser le revenu brut dans les différentes parties qui le composent, ou de distinguer entre la somme payée pour l’exploitation du sol et celle payée pour le capital qui y aura été dépensé. Mais supposons un instant qu’une pareille séparation puisse se faire, alors la première partie, ou celle payée pour l’exploitation du sol, formant un surplus sur les frais de production, il est clair qu’elle pourra être entièrement enlevée par la taxe, sans que cela affecte d’autres intérêts que ceux des propriétaires. La taxe la plus lourde ne contribuerait pas à faire hausser le prix des matières premières, car rien ne peut affecter ce prix, sans affecter en même temps les frais de la production. Or, la rente des terres est tout à fait indépendante des frais de production, elle ne peut même pas exister avant que le fermier soit rentré dans toutes les dépenses faites pour porter ses produits au marché, et avant qu’il ait retiré les bénéfices convenables de son capital engagé dans la construction des bâtiments, haies, dans les semences, le travail, instruments, etc. Les prix des produits ne seront donc pas affectés, bien que lu taxe absorbe tout le revenu de la terre, c’est-à-dire la somme entière payée pour l’exploitation du sol.
Il serait impossible au gouvernement, quand même il serait disposé à le faire, d’enlever par une taxe directe le total de la rente du propriétaire, c’est-à-dire le total de la somme payée non-seulement pour l’exploitation du sol, mais aussi pour les bâtiments et constructions. Car un impôt qui frapperait n’importe de quelle façon le revenu du capital employé en améliorations, aurait pour effet de faire hausser le prix des matières premières, et retomberait nécessairement sur le consommateur. La rente pour l’exploitation du sol appartient aux propriétaires, non pas comme cultivateurs, mais comme propriétaires ; il n’en est pas de même de la portion du revenu payée pour améliorations et constructions. Ces améliorations, ils les ont faites en leur qualité de cultivateurs, et il est alors évident qu’une taxe qui frapperait le revenu de ce capital affecterait infailliblement les prix des matières premières. Supposons, par exemple, que le total de la rente d’une ferme s’élève à 500 l. sterl. par an. dont la moitié, ou 250 l. sterl., serait payée comme intérêt du capital employé en améliorations. Si dans un cas pareil une taxe de 10 pour 100 était imposée sur ce revenu, la moitié seulement, ou 25 liv. sterl., serait entièrement à la charge du propriétaire. D’abord, il est vrai, le total de la taxe pèserait sur lui ; mais la moitié de cette somme serait évidemment à déduire de l’intérêt du capital affecté aux améliorations, et non pas de la rente du sol proprement dite. Les propriétaires des terres seraient alors naturellement dans une position moins favorable que les autres producteurs ; ils se verraient obligés de ne pas placer des capitaux dans l’exploitation, avant qu’une hausse dans les prix des céréales et des autres matières premières, produite soit par la diminution de la quantité, soit par l’accroissement de la demande, les mette dans la même position que les autres producteurs, c’est-à-dire jusqu’à ce qu’ils aient obtenu le taux des profits communs et ordinaires provenant du capital affecté aux améliorations. Il est donc évident que si par une taxe exclusive le Trésor peut absorber tout le revenu provenant de l’exploitation du sol, l’autre portion, provenant du rapport des capitaux engagés dans les constructions et améliorations, ne saurait rester à la longue affectée par une taxe de celle espèce, et qu’en définitive le niveau entre le rapport des produits de la terre et celui des autres productions se rétablirait bientôt.
Du point de vue pratique, des taxes sur le revenu des terres seront toujours une mesure des plus injustes et des plus impolitiques. Il est, comme nous avons déjà démontré, tout à fait impossible de diviser cette rente dans ses éléments, et de constater d’une manière précise la part appartenant au revenu net du sol, et celle provenant des capitaux engagés dans les améliorations. Il n’y a pas deux agriculteurs qui, dans l’examen d’un cas particulier, arrivent, si ce n’est par hasard, au même résultat, et les juges les plus compétents affirment qu’en général une pareille distinction devient impossible. Quand donc, en conséquence, une taxe est imposée sur lu rente de la terre, elle sera nécessairement proportionnée au montant du total, sans distinction des sources d’où elle provient. Une pareille taxe a toujours été et sera inévitablement toujours un obstacle invincible à toute amélioration ; car, la taxe frappant les capitaux dépensés en améliorations, empêchera l’emploi de nouveaux capitaux. L’injustice de cet impôt n’est pas moins claire. Supposons que deux propriétaires fonciers se partagent d’une manière égale un revenu de 1,000 liv. sterl. par an ; la propriété de l’un consistera en terres d’une excellente qualité, qui n’ont besoin, pour être mises en culture, que d’un capital relativement petit ; celle de l’autre consistera en terres d’une qualité inférieure, et aura exigé des capitaux considérables pour son exploitation. La taxe enlèvera dans le revenu de la première une portion due à la faveur de la nature ; mais dans le revenu de la seconde, elle n’atteindra que le travail et l’industrie de l’homme. De là l’injustice manifeste des taxes sur le revenu foncier. Nous doutons qu’il soit possible de rien imaginer qui fût plus contraire aux vrais principes. et plus défavorable aux progrès de la culture. La contribution foncière, en France, était «ne taxe de ce genre ; et elle fait le sujet des plaintes de tous les écrivains qui s’occupent do l’agriculture de la France.
Mac Culloch.
↑ Il s’agit d’un entrepreneur qui bâtit à ses frais et risques, pour vendre ou louer ensuite la construction.
↑ Voilà une indication qui n’est guère suivie en matière de taxations, quoiqu’elle soit parfaitement juste. A. B.
↑ Cette opinion s’accorde avec la théorie d’Adam Smith sur les dépenses productives ; mais elle n’est pas juste. Une maison qu’habite un particulier n’est pas toujours directement productive, mais il est clair qu’elle peut l’être indirectement; car comment la production pourrait-elle avoir lieu, si ceux qui y sont engagés n’avaient pas un abri et les autres avantages que donnent les maisons ? La nourriture, qui fait vivre l’ouvrier, n’est pas plus nécessaire qu’une maison. Si la première est productive, la seconde le sera également.
Mac Culloch.
↑ Depuis la première publication de cet ouvrage, il a été établi un impôt à peu près conforme aux principes exposés ci-dessus. (Note de l’auteur.)
↑ Mémoires concernant les droits, etc., p. 223.
↑ Depuis 1798, la taxe des fenêtres est, pour une maison de six fenêtres, 6 schell. au total ; dix fenêtres payent 1 liv. 12 sch. ; vingt payent 7 liv. 12 sch. ; cinquante payent 20 liv. 10 sch. ; cent payent 37 liv. Les nombres intermédiaires ont leur tarif à proportion. La taxe la plus haute, qui a lieu pour cent quatre-vingts fenêtres et plus, est de 61 liv. (environ 1,460 fr.)
↑ Sur toutes les questions d’impôts, le lecteur consultera avec fruit les chap. viii et xvii des Principes de l’économie politique et de l’impôt, par David Ricardo, et le livre VI des Nouveaux principes d’économie politique, de M. de Sismondi. A. B.
↑ Communément, on ne parle de l’intérêt de l’argent que lorsque le capital a été emprunté par celui qui l’emploie. Cependant, que ce capital soit ou ne soit pas emprunté, il peut toujours être supposé tel dans tous les cas ; car le commerçant qui opère avec un capital à lui en propre retire dans son profit ce qui représente l’intérêt, tel qu’il l’eût payé si le capital eût été d’emprunt. C’est dans ce sens que doit être pris ici le mot d’intérêt de l’argent.
↑ Adam Smith raisonne ici comme si le taux des profits était fixe et inévitable, et qu’il ne pût diminuer par suite d’une nouvelle taxe ; il pense qu’elle serait payée soit par une hausse proportionnée des prix, soit par une réduction dans le taux des intérêts. Mais il n’explique pas pourquoi le taux des profits ne peut pas être réduit par une taxe générale tout aussi bien que létaux des intérêts. Si la taxe ne frappait que certaines branches de commerce, les capitaux s’en retireraient, et les profits s’élèveraient ainsi en proportion de la taxe. Mais quand l’imposition est générale, les capitaux ne peuvent pas se retirer, et il en résulte une baisse dans le taux des profits et dans le taux des intérêts. Si le taux des profits était de 10 pour 100 et que la taxe imposée s’élevât à 2 pour 100, l’intérêt du capital baisserait à coup sûr, puisque le rapport du capital deviendrait moindre. Mais il n’est pas probable que le fardeau pèserait entièrement sur les intérêts, il atteindrait en même temps les intérêts et les profits. Buchanan.
↑ Chap. ix.
↑ Par l’établissement dans ce pays d’une taxe sur la propriété ou plutôt sur les revenus, le taux des intérêts est, comme toute autre espèce de revenu, soumis à la contribution d’un dixième ; et comme c’est toujours l’emprunteur qui le paye et qui le fait entrer dans son compte en déduction de l’intérêt qui est dû au porteur, il ne manque pas d’être exactement perçu. La taxe sur les revenus du commerce est, sans aucun doute, souvent éludée ; et pour prévenir cette fraude, les employés chargés de prélever cet impôt sont autorisés à se livrer-à des recherches que, selon Adam Smith, aucun peuple ne voudrait longtemps supporter. Buchanan.
↑ Mémoires concernant les droits, etc, tome I, page 74.
↑ Ibid, tome I, pages 163, 166 et 171.
↑ Les opinions d’Adam Smith relativement à l’influence des taxes sur les profits des différentes entreprises nous paraissent plus erronées encore que celles développés sur les taxes du revenu foncier. Il suppose qu’une pareille taxe, qu’elle frappe les profits de toutes les affaires, ou qu’elle n’atteigne qu’une ou plusieurs d’entre elles, ne sera jamais payée d’une manière permanente par ceux qu’elle aura frappés d’abord ; que les producteurs et les commerçants élèveront les prix de leurs marchandises à proportion de la taxe, de façon qu’en général le payement se fera non par eux, mais par les consommateurs. Une courte discussion démontrera l’erreur de cette opinion. Pour mettre en plus grande évidence ce que nous venons de dire à ce sujet, nous diviserons nos observations en deux parties. Nous supposerons, dans la première, que la taxe frappe les profits de tous les capitaux ; et dans la seconde, qu’elle n’est pas générale, et qu’elle affecte seulement les profits d’une ou de plusieurs branches de l’industrie.

Si la taxe était universelle, elle frapperait évidemment les profits seuls sans modifier les prix des marchandises ou la distribution du capital ; nous avons démontré, en traitant des effets des variations dans le taux des salaires et profits sur les valeurs des marchandises, que tout ce qui affectait différentes classes de producteurs dans une mesure égale, ne pouvait changer ni leur position relative ni la valeur de leurs marchandises. Le même cas se présente quand il s’agit de la taxe en question. Un fabricant de lin ou de coton, frappé d’une taxe de 5 ou de 10 pour 100 sur ses profits, ne sera en aucune façon dans une position moins favorable que les autres, si tous se trouvent imposés dans la même proportion. Il est par conséquent évident que, dans ces circonstances, il ne pourra point éviter cette taxe en changeant d’affaires ; les capitaux, par conséquent, ne changeront pas d’emploi. Comme la taxe n’augmente pas la quantité de travail requise pour la fabrication des marchandises, les frais de production ne deviendront pas plus considérables ; la provision et la demande n’éprouveront pas de variation, et comme elle affectera plutôt les profits que les capitaux, les moyens de production ne seront pas diminués par suite de son établissement. Les moyens d’acheter de ceux qui vivent de profits seront sans doute diminués par suite de l’imposition de cette taxe ; mais, comme les moyens d’acheter dont disposeraient le gouvernement et ses agents qui perçoivent la taxe, seront augmentés en proportion de la diminution qu’auront éprouvée les contribuables, l’ensemble des demandes de la société restera le même ; et comme la taxe ne pourra ni diminuer la quantité du capital, ni affecter sa distribution, ni même diminuer la faculté d’acheter de ses produits, il est évident qu’elle n’occasionnera aucune variation dans le prix des marchandises. L’effet immédiat d’une taxe sur les profits, égale et universelle, sera de les faire baisser dans une même proportion, liais comme la possibilité d’accumuler des capitaux est toujours proportionnée au taux des profits, il s’ensuivra que la tendance et l’effet de pareilles taxes, quand elles sont assez élevées pour ne pas être balancées par l’industrie et l’économie, seront d’empêcher l’accumulation des capitaux et l’accroissement de la population.

Mais, si la taxe n’était pas universelle et qu’elle affectât seulement les profits des capitaux engagés dans une ou plusieurs affaires, les effets en seraient tout à fait différents. En ce cas, elle ferait hausser les prix, et ne retomberait par conséquent sur les capitalistes qu’autant qu’ils consommeraient leurs produits. Supposons, par exemple, qu’une taxe de 10 pour 100 fût mise exclusivement sur les profits des chapeliers., il serait alors facile de démontrer que le prix des chapeaux augmentera en proportion ; car sans cette augmentation, les chapeliers gagneraient moins que les fabricants engagés dans d’autres affaires, et ils seraient forcés de retirer leurs capitaux de cet emploi ; les capitaux continueraient ainsi d’être retirés jusqu’à ce que, par la diminution de la provision des chapeaux, leur prix se fût élevé de manière à fournir le taux des profits ordinaires. Par la même raison, une taxe sur les profits du tailleur, du fermier, du [cordonnier, produirait une augmentation dans le prix des marchandises qu’ils portent sur le marché. Dans ce cas, les producteurs ont toujours la possibilité d’élever les prix et de faire porter la taxe sur les consommateurs, parce qu’ils peuvent retirer des capitaux des emplois dont les profits sont frappés d’une taxe pour les placer là où les profits ne le sont pas. Mais, quand toutes les affaires sont imposées d’une manière égale, cette ressource échappe aux capitalistes, et ils n’ont aucun moyen d’élever les prix ou d’éviter la taxe. Mac Culloch.
↑ La taxe sur les carrosses de place ou fiacres est de 10 sch. par semaine depuis 1784.
↑ En 1786, M. Pitt essaya d’établir une taxe sur les boutiques, graduée sur le loyer ; elle excita de grands murmures, et on fut obligé de la révoquer en 1789.
↑ Copy-holders, ainsi nommés parce qu’ils possédaient sans titre direct, et que leur droit n’était fondé que sur la prescription et la coutume locale ; qu’ainsi, pour justifier de ce droit, ils se faisaient délivrer copie ou extrait des registres de leur baron, ce qui attestait, 1° l’ancienneté de leur possession ; 2° la nature et quotité des services auxquels ils étaient tenus par la coutume de la baronie.
↑ Mémoires concernant les droits, etc., tome I, page 17.
↑ C’est celui qu’on nomme obroc. On peut consulter utilement, sur les effets de cet impôt, le chapitre vu des Nouveaux principes d’économie politique de M. de Sismondi, et le Cours de M. Storch. A. B.
↑ Lib. LV. Voyez aussi Burman, de Vectigalibus Pop. Rom., cap. x ; et Bouchaud, de l’Impôt du vingtième sur les successions. (Note de l’auteur.)
↑ Mémoires concernant les droits, etc., tome I, page 225.
↑ Mémoires concernant les droits, etc. tome I, page 154.
↑ Idem, page 157.
↑ Ces droits ont été fort augmentés et multipliés depuis quelques années. Il a été établi de plus, en 1796, un droit sur les legs en ligne collatérale, qui monte jusqu’à 6 pour 100 du montant des legs, quand les légataires sont étrangers ou parents éloignés.
↑ Mémoires concernant les droits, etc., tome I, pages 223, 224 et 225.
↑ Le stiver, qui est la vingtième partie du florin, vaut environ 11 centimes.
↑ Les raisons qui font supposer à Adam Smith qu’une taxe sur le travail aurait pour suite une augmentation analogue aux prix des salaires, sont peu satisfaisantes ; ses vues sur cette matière se rattachent à sa théorie sur l’état invariable du taux des salaires, que j’ai essayé de réfuter ailleurs. Son argumentation parait se réduire à ceci : l’état d’une société, selon qu’il est en progrès, qu’il reste stationnaire, ou qu’il décline, détermine les moyens d’existence du travailleur ; ils sont ainsi ou abondants, ou modiques, ou bornés ; et comme les salaires se règlent d’après ce principe, il paraît admettre qu’aucune cause ne saurait altérer ce rapport. Dans une société en progrès, le travail est sans doute bien rétribué, parce que les demandes sont très-grandes ; et dans une société stationnaire ou en déclin, il est mal payé, parce que les demandes baissent. Mais, après avoir reçu la rétribution due à son travail, le travailleur peut-il avoir quelque recours contre celui qui l’a employé, parce qu’il sera forcé de dépenser une partie de ce salaire en impôts ? Il n’y a aucune loi qui autorise une pareille supposition. Après avoir reçu son salaire, le travailleur porte à ses propres risques et périls le fardeau de toutes les exactions auxquelles on l’expose, car il n’a à sa disposition aucun moyen coercitif pour exiger un remboursement de celui qui lui a payé la rétribution convenable de son travail. S’il était réduit au strict nécessaire, il ne pourrait pas supporter une pareille réduction de ses salaires, il ne pourrait plus soutenir sa famille ; mais, comme les salaires du travail lui permettent de se procurer une plus grande aisance, quelquefois même des objets de luxe, il a toujours de quoi payer l’impôt. Ce que l’impôt lui enlève serait dépensé en jouissances auxquelles il est ainsi obligé de renoncer. Les taxes sur le travail, ou sur des articles à l’usage du travailleur, ont pour effet de diminuer l’aisance du travailleur ; elles augmentent ses privations et tendent à dégrader la condition des classes ouvrières. Adam Smith suppose que l’effet inévitable d’une taxe sur les objets de première nécessité sera d’en rendre l’acquisition impossible au travailleur, et de produire ainsi, par contre-coup, une hausse dans les salaires. Mais ceci ne peut être admis que dans le cas où le travailleur serait réduit au strict nécessaire ; car, s’il peut vivre dans l’aisance, il retranchera du superflu pour payer l’impôt sur le travail ou sur les objets de première nécessité. Qu’il y ait donc une taxe directe sur le travail, ou une taxe sur des objets qu’Adam Smith appelle articles de luxe, les effets sur la condition du travailleur resteront toujours les mêmes ; car, du moment qu’il est obligé de régler ses dépenses de manière à pouvoir payer la taxe, il n’est d’aucune importance de savoir de quelle espèce de jouissance il est forcé de se priver.

Adam Smith modifie, il est vrai, son opinion relativement à l’influence d’une taxe directe sur les salaires du travail, eu ajoutant que c’est seulement quand les demandes du travail restent les mêmes, que les salaires du travailleur s’élèvent en proportion de la taxe. Mais pourquoi admettre que les demandes resteront les mêmes, quand les salaires auront augmenté ? C’est toujours la demande qui règle le prix du travail, et si les demandes n’augmentent pas, les prix ne s’élèvent pas non plus. Il est également contraire aux principes en économie politique, de supposer que la demande restera la même, malgré une hausse dans les prix ; les demandes du travail, comme celles des marchandises, baissent a mesure que les prix s’élèvent. Si le travailleur exigeait une augmentation de salaire proportionnée à la taxe, les demandes du travail diminueraient immédiatement, et il serait bientôt forcé de se contenter des anciens salaires. Pour payer la taxe, le travailleur sera donc obligé de réduire ses dépenses, en se passant des objets qui ne sont pas absolument nécessaires. Buchanan.
↑ Mémoires concernant les droits, etc., tome II, page 108.
↑ Idem, tome III, page 67.
↑ Les choses sont bien changées en France depuis qu’Adam Smith écrivait ces lignes.
↑ Plusieurs de ces articles, notamment le thé, le sucre, le tabac et les liqueurs spiritueuses, ont subi, depuis quelques années, de fortes augmentations de droits. En 1790, le tabac a été retiré de la régie des douanes et transporté à celle de l’accise : il paye 1 sch. 7 d. par livre ; c’est plus que six fois le prix d’achat, s’il ne coûte que 3 d. la livre, comme on l’a dit aux Communes en 1784.
↑ Il est certain que la taxe sur une marchandise élèvera le prix de tous les articles dans la confection desquels elle entre. Une taxe sur le charbon et les chandelles, par exemple, fera hausser le prix des marchandises dans In production desquelles on en consomme. Mais on ne voit pas, pourquoi, quand une taxe est mise sur le cuir, le consommateur payerait la part du cordonnier. Ceux qui font le commerce d’une marchandise imposée avancent d’abord le montant de cette taxe, qui leur rentre quand la marchandise est vendue. Mais la portion qu’ils consomment eux-mêmes reste naturellement à leur charge, car la circonstance qu’ils font le commerce de cette marchandise ne peut pas les placer dans une position exceptionnelle.
↑ En 1798 le droit a été porté à 5 sch. par boisseau. M. Pitt évaluait à un demi-boisseau la consommation annuelle d’une famille pauvre.
↑ Le droit sur la chandelle avait été porté à 1 den. 1/2 : il n été, en 1792, diminué d’un demi-denier.
↑ Quand la concurrence des ouvriers entre eux ne fait pas baisser ces salaires. A. B.
↑ Ce droit sur le cabotage du charbon est actuellement de 8 sch. 10 d. par chaldron pour le port de Londres, et 5 sch. 6 d. pour les autres ports.
↑ Voyez Mémoire concernant les droits, etc., pages 210 et 211.
↑ Le Réformateur, par Cliquot de Blervache, inspecteur-général du commerce, Amsterdam, 1756
↑ Les droits imposés en 1785 et 1789 sur les carrosses vont à 8 liv. sterl. par voiture à quatre roues, avec une augmentation progressive sur les deuxième et troisième voilures, outre le droit de 1 liv. sterl. sur le premier cheval et le droit progressif sur les autres : ces derniers droits ont été augmentés en 1796, 1797 et 1801. Le triplement des taxes assises, qui ont eu lieu en 1798, a porté sur ces droits qui en font partie.
↑ Ce droit sur les ouvrages d’orfèvrerie, qui fait partie de ceux du timbre, est maintenant de 8 sch. par once d’or, et de 6 d. par once d’argent, une fois payés.
↑ Ce calcul parait exiger quelque éclaircissement. Le pot de porter, tout impôt déduit, eût coûté 2 pence. L’ouvrier, à cause de l’impôt, ne pouvant acheter le pot, se contente de la pinte ou moitié du pot, laquelle, tout impôt compris, lui coûte 1 penny 1/2 ; donc il a réellement économisé ; de penny ou un farting, et cette épargne est l’effet de l’impôt.
↑ Ces droits se montent à environ 8 sch. par quintal sur le verre pour vitre, et à moitié sur le verre pour bouteilles.
↑ En anglais, customs.
↑ Le commerce de la gomme a reçu depuis une immense extension, et, malgré ses vicissitudes, notre colonie du Sénégal a acquis beaucoup d’importance.
A. B.
↑ En 1798 le revenu brut des douanes à monté à 7,789,658 liv. sterl. ; les frais de régie, à 414,166 liv. ; les déductions pour gratifications, à 507,221 liv. ; celles pour retours de droits, à 1,229,622 liv. ; autres dépenses prélevées sur ce produit, 77,493 liv. ; le produit net s’est trouvé être de 5,561,136 liv. ; les frais de régie ont fait environ 7 1/2 pour 100 du produit brut, et environ 5 1/2 pour 100 du produit net.
↑ C’est ce qui ne saurait manquer d’arriver parmi nous, le jour où la raison publique aura parfaitement compris la portée du dommage causé à la richesse des nations par le système des douanes. A. B.
↑ M. Pitt a exécuté une partie de ce plan en réunissant à la régie de l’accise plusieurs branches de revenu qui dépendaient des douanes ou d’autres régies particulières, notamment l’impôt du tabac, du sel. etc.
↑ Comme on voit, Adam Smith a exposé ici le premier les avantages de la création des entrepôts, que ses compatriotes ont élevés à un si haut rang d’utilité sous le nom de docks, et dont l’organisation laisse encore tant à désirer parmi nous. A. B.
↑ Cette même branche de l’accise a donné en 1798 un produit brut de 5,595,4151. sterl., sans y comprendre l’accise d’Écosse. Les vins et les liqueurs spiritueuses de l’étranger ont en outre donné lieu à plus de 1,850,000 liv. sterl, de droits d’accise.
↑ Ce droit a reçu en 1790 une augmentation qui est de 8 d. par baril si la bière est pour la consommation de Londres, et de 10 d. si elle est destinée aux provinces. dans de grandes et riches maisons de province.
↑ Il est à 6 sch. 6 d. par quarter de huit boisseaux.
↑ Voyez la note de la page précédente.
↑ A été augmenté de 2 den. par boisseau en 1790.
↑ Ce droit, qui date de 1697, était annuellement voté par le Parlement : il était de 6 den. par boisseau.
↑ Ce droit additionnel, établi en 1760, était de 3 den. par boisseau ; les deux droits réunis étaient de 9 den. par boisseau ou 6 sch. par quarter de huit boisseaux. — Voyez à la fin du volume la Table de conversion de toutes les monnaies, poids et mesures en usage en Angleterre.
↑ Ce muid est de 63 gallons.
↑ Espèce de bière dans la composition de laquelle entrent beaucoup d’ingrédients et plantes aromatiques : elle se fabrique beaucoup en Allemagne, et principalement à Brunswick. On l’appelle aussi bière de Brunswick.
↑ Il y a deux espèces de boissons faites avec le miel et l’eau, auxquelles on ajoute quelques épices et un peu de levure de bière : l’une se nomme mead, l’autre meteglin ; elles diffèrent très-peu.
↑ On donne le nom générique de vins, en Angleterre, aux liqueurs fermentées qu’on retire des différents fruits ou végétaux les plus susceptibles de la fermentation vineuse ou spiritueuse. Les petits vins ou vins factices se nomment aussi vins doux (sweets), ou vins du pays (home-made). On trouve dans l’Art de la cuisine et office, par Farley, les recettes de plus de quarante sortes différentes de ces vins.
↑ Ce sont des espèces d’eaux-de-vie qu’on extrait de la bière, du cidre, du poiré, de l’hydromel, du riz, du sucre, etc. À un certain degré de force, on les nomme esprits à l’épreuve : ce sont ceux dont il est ici question, Plus rectifiés et au-dessus de l’épreuve, ils payent des droits plus forts.
↑ Quoique les droits directement imposés sur les esprits ne montent qu’à 12 sch. 6 den. par gallon, ceux-ci ajoutés aux droits sur les petits vins dont ces esprits sont extraits, montent à 3 sch. 10 den. 2/3. Les petits vins et les esprits sont taxés aujourd’hui, pour prévenir les fraudes, d’après la jauge même des matières en fermentation. (Note de l’auteur.)
↑ On évalue à huit ou neuf mille le nombre de vaisseaux de toutes nations qui passent annuellement le Sund. En 1796 il monta à douze mille. Le droit qu’ils payent va environ à 1 3/4 pour 100 de la valeur sur toutes les marchandises. On peut l’évaluer à 75 rixdalles par chaque vaisseau l’un dans l’autre ; ce qui formerait un revenu de 600,000 rixdalles, outre ce que payent les vaisseaux pour l’entretien des feux, bouées, signaux, etc. Le droit se paye en rixdalles espèces, qui valent environ 5 fr. 30 cent.
↑ Le produit net de cette année, toutes dépenses et charges déduites, a monté à 4,975,652 liv. 19 sch. 6 den. (Note de l’auteur.) — En 1798 ce produit net, non compris l’accise de l’Écosse, s’est élevé à 9,873,618 liv. sterl.
↑ Les frais de perception sont maintenant dans une bien moindre proportion avec le produit, parce que celui-ci a été fort augmenté, et la régie améliorée. À la fin de 1799, les frais de douanes étaient à 5 5/8 pour 100 du produit ; ceux de l’accise et du timbre à 3 1/4, pour 100. A. B.
↑ Mémoires concernant les droits, etc., tome I, page 455.
↑ Il est inutile de rappeler que tout ce système a été aboli dès les premiers jours de la révolution française. A. B.
↑ On en a levé six fois davantage pendant la guerre, et on lèvera nécessairement dans l’état de paix plus de 34 millions sterl. par année.
↑ L’expérience n’a point justifié cette prédiction d’Adam Smith.
*On voit par celle note que le commentateur Buchanan appartient a l’école absolue, qui ne veut de l’intervention du gouvernement en aucune manière dans Ici affaires de l’industrie humaine. Il est inutile de réfuter une telle hérésie. L’influence des gouvernements est comme celle des saisons, bonne ou mauvaise, selon la prédominance variable des bons et des mauvais jours ; mais elle est incontestable. C'est l'affaire de la politique de veiller a ce que cette influence soit la meilleure possible dam l'intérêt général. A. B.

%%%%%%%%%%%%%%%%%%%%%%%%%%%%%%%%%%%%%%%%%%%%%%%%%%%%%%%%%%%%%%%%%%%%%%%%%%%%%%%%
%                                  Chapitre 3                                  %
%%%%%%%%%%%%%%%%%%%%%%%%%%%%%%%%%%%%%%%%%%%%%%%%%%%%%%%%%%%%%%%%%%%%%%%%%%%%%%%%

\chapter{Des dettes publiques}
\markboth{Des dettes publiques}{}

Quand la société est encore dans cet état informe qui précède les progrès des manufactures et l’extension du commerce, quand ces objets dispendieux de luxe que le commerce et les manufactures peuvent seuls y introduire sont entièrement incon­nus, alors, comme j’ai cherché à le faire voir dans le troisième livre de ces Recher­ches, celui qui possède un grand revenu n’a pas d’autre manière de le dépenser et d’en jouir que de l’employer à faire subsister autant de monde à peu près que ce revenu peut en nourrir. On peut dire en tout temps d’un grand revenu, qu’il consiste dans le pouvoir de commander une grande quantité de choses nécessaires aux besoins de la vie. Dans cet état encore informe, le payement de ce revenu se résout communé­ment en une immense provision de choses de première nécessité, en denrées propres à fournir une nourriture simple et de grossiers vêtements, en blé et bétail, en laine et peaux crues. Quand ni le commerce ni les manufactures ne fournissent d’objets d’échan­ge contre lesquels le propriétaire de toutes ces denrées puisse échanger tout ce qu’il en possède au-delà de sa consommation propre, il ne peut faire autre chose de cette quantité surabondante que d’en nourrir et d’en habiller à peu près autant de monde qu’elle peut en nourrir et en habiller. Dans cet état de choses, la principale dépense que puissent faire les riches et les grands consiste en une hospitalité sans luxe et des libéralités sans ostentation. Mais, comme j’ai cherché pareillement à le montrer dans le même livre[1], ces sortes de dépenses sont de nature à ne pas ruiner aisément ceux qui les font. Parmi les plaisirs personnels, au contraire, il n’y en a peut-être pas de si frivole qui n’ait quelquefois ruiné ceux qui s’y sont livrés, et même des hommes qui n’étaient pas dépourvus de jugement. La passion des combats de coqs n’en a-t-elle pas ruiné beaucoup ? Mais je ne crois pas qu’il y ait beaucoup d’exemples de gens réduits à la misère par une hospitalité ou des libéralités du genre de celles dont je parle, quoique l’hospitalité de luxe et les libéralités d’ostentation en aient ruiné un grand nombre. Le long temps pendant lequel, sous le régime féodal, les terres demeuraient dans la même famille, est une preuve suffisante de la disposition géné­rale de nos ancêtres à ne pas dépenser au-delà de leurs revenus. Quoique l’hospitalité rustique, continuellement exercée par les grands propriétaires, ne nous semble peut-être guère compatible avec cet esprit d’ordre que nous regardons volontiers comme inséparable d’une vraie économie, cependant nous serons bien obligés de convenir qu’ils ont été au moins assez économes pour n’avoir pas communément dépensé tout leur revenu. Il y avait une partie de leurs laines et de leurs peaux qu’ils trouvaient à vendre pour de l’argent. Peut-être dépensaient-ils une portion de cet argent à acheter le peu d’objets de luxe et de vanité que les circonstances du temps pouvaient leur fournir ; mais il paraît aussi qu’une autre portion était communément mise en réserve. Il est vrai qu’ils ne pouvaient guère faire autre chose de l’argent qu’ils épargnaient que de thésauriser. Il eût été déshonorant pour un gentilhomme de faire le commerce, et il l’eût été encore bien davantage de prêter de l’argent à intérêt ; ce qui était alors regardé comme de l’usure, et prohibé par la loi. D’ailleurs, dans ces temps où régnaient la violence et les désordres, il était a propos d’avoir sous la main un trésor en argent, pour pouvoir, dans le cas où on serait chassé de sa demeure, emporter avec soi, dans un lieu de sûreté, quelque chose d’une valeur connue. Les mêmes violences qui obli­geaient à thésauriser obligeaient pareillement à cacher son trésor. Une preuve assez claire de l’usage où on était alors d’amasser des trésors et de les cacher, c’est la grande quantité de trésors trouvés, c’est-à-dire de trésors qu’on découvrait sans en connaître le propriétaire. Ces trésors étaient regardés alors comme une branche importante du revenu du souverain. Aujourd’hui, tous les trésors trouvés du royaume feraient peut-être à peine une branche importante dans le revenu d’un particulier un peu riche.
La même disposition à épargner et à thésauriser avait gagné le souverain aussi bien que les sujets, comme on l’a observé dans le IVe livre. Chez des nations qui ne connaissent guère le commerce ni les manufactures, le souverain est dans une situa­tion qui le dispose naturellement à cet esprit d’économie nécessaire pour amasser. Dans un tel état de choses, le train de la dépense, même chez un souverain, ne peut prendre sa direction d’après ce vain orgueil qui aime à s’environner d’une cour brillante et fastueuse. L’ignorance des temps fournit très-peu de ces colifichets qui constituent la recherche de la parure. Les armées de troupes réglées ne sont pas alors nécessaires ; de sorte que la dépense même du souverain ne peut guère consister en autre chose qu’en libéralités envers ses tenanciers, et en hospitalité envers les gens de sa suite. Mais les libéralités et l’hospitalité conduisent bien rarement à des profusions excessives, tandis que la vanité y mène presque toujours. Aussi, comme on l’a déjà observé, tous les anciens souverains de l’Europe avaient-ils des trésors ; et actuelle­ment, dit-on, il n’y a pas de chef de Tartares qui n’en ait un.
Dans un pays commerçant où abondent tous les objets de luxe les plus dispen­dieux, naturellement le souverain, de même que tous les grands propriétaires de ses États, dépense à ces fantaisies une grande partie de son revenu. Son pays et les pays voisins lui fournissent en abondance toutes ces bagatelles précieuses qui composent la pompe éblouissante, mais vaine, des cours. Pour un étalage du même genre, quoique d’un ordre inférieur, ses nobles renvoient leur suite, affranchissent leurs te­nan­ciers de toute dépendance, et finissent par devenir insensiblement aussi nuls que la plupart des riches bourgeois de ses États. Les mêmes passions frivoles qui dirigent la conduite de ces nobles influent sur celle du chef. Comment pourrait-on s’imaginer qu’il sera le seul riche de ses États qui soit insensible à ce genre de plaisir ? En sup­posant qu’il n’aille pas jusqu’à dépenser dans ces vains amusements, comme il n’est que trop présumable qu’il le fera, assez de son revenu pour que les forces destinées à la défense de l’État en souffrent sensiblement, au moins ne peut-on guère s’attendre qu’il n’y dépense pas toute cette partie de revenu que n’absorbe pas l’entretien de ces forces. Sa dépense ordinaire prend le niveau de son revenu ordinaire, et on est fort heureux si bien souvent elle ne monte pas au-delà. Il ne faut plus espérer qu’il amasse de trésor, et quand les besoins extraordinaires exigeront des dépenses imprévues, il faudra nécessairement qu’il recoure à ses sujets pour en obtenir une aide extraor­dinaire. Le feu roi de Prusse et celui régnant sont les seuls grands princes de l’Europe, depuis la mort de Henri IV, roi de France, en 1610, qui passent pour avoir amassé un trésor un peu considérable. Cet esprit d’épargne qui conduit à amasser est devenu presque aussi étranger aux républiques qu’aux gouvernements monarchiques. Les républiques d’Italie, les Provinces-Unies des Pays-Bas, sont toutes endettées. Le canton de Berne est la seule république de l’Europe qui ait amassé un trésor de quel­que importance. Les autres républiques de la Suisse n’en ont point. Le goût d’un faste quelconque, celui au moins de la magnificence des bâtiments et autres embellisse­ments publics, domine souvent tout autant dans le sénat si modeste en apparence d’une petite république, que dans la cour dissipée du plus grand monarque.
Le défaut d’économie, en temps de paix, impose la nécessité de contracter des dettes en temps de guerre. Quand survient la guerre, il n’y a dans le Trésor que l’ar­gent nécessaire pour faire aller la dépense ordinaire de l’établissement de paix. Cependant alors il faut établir les dépenses sur un pied trois ou quatre fois plus fort pour pourvoir à la défense de l’État et, par conséquent, un revenu trois ou quatre fois plus fort que le revenu du temps de paix devient indispensablement nécessaire. Supposons même que le souverain ait sous sa main des moyens d’augmenter sur-le-champ son revenu à proportion de l’augmentation de sa dépense, moyen qu’il n’a presque jamais, encore le produit des impôts dont il faut tirer cette augmentation de revenu ne commencera-t-il à rentrer dans le Trésor que dix ou douze mois peut-être après que ces impôts auront été établis. Mais au moment même où commence la guerre, ou plutôt au moment même où elle menace de commencer, il faut que l’armée soit augmentée ; il faut que la flotte soit équipée ; il faut que les villes de garnison soient mises en état de défense ; il faut que cette armée, cette flotte, ces garnisons soient approvisionnées de vivres, d’armes et de munitions. C’est une énorme dépense actuelle qui doit parer à ce moment de danger actuel, et il n’y a pas moyen d’attendre les rentrées lentes et successives des nouveaux impôts. Dans ce besoin urgent, le gouvernement ne saurait avoir d’autre ressource que celle des emprunts.
Ce même état d’activité commerçante où se trouve la société, cet état qui, par l’action de diverses causes morales, met ainsi le gouvernement dans la nécessité d’emprunter, fait naître aussi chez les sujets et les moyens, et la volonté de prêter. Si cet état amène avec soi, pour l’ordinaire, la nécessité d’emprunter, il amène en même temps avec soi la facilité de le faire[2]. 
Un pays qui abonde en marchands et en manufacturiers abonde nécessairement en une classe de gens à qui non-seulement leurs propres capitaux, mais encore les capitaux de tous ceux qui leur prêtent de l’argent ou leur confient des marchandises, passent aussi fréquemment ou plus fréquemment par les mains, que ne le fait à un particulier son propre revenu lorsque, sans se mêler d’aucune affaire de commerce, il se borne à vivre de ses rentes. Le revenu de ce particulier ne peut lui passer par les mains régulièrement qu’une fois dans tout le cours de l’année. Mais la masse totale des capitaux et du crédit d’un commerçant dont le négoce est de nature à lui donner des rentrées très-promptes, peut quelquefois lui passer par les mains deux, trois ou quatre fois par an. Par conséquent, un pays qui abonde en marchands et manufac­turiers abonde nécessairement en une classe de gens qui ont en tout temps la faculté d’avancer, s’il leur convient de le faire, de très-grosses sommes d’argent au gouvernement : de là provient, dans les sujets d’un État Commerçant, le moyen qu’ils ont de prêter.
Le commerce et les manufactures ne peuvent guère fleurir longtemps dans un État qui ne jouit pas d’une administration bien réglée de la justice, dans lequel on ne sent pas la possession de ses propriétés parfaitement garantie, dans lequel la foi des conventions n’est pas appuyée par la loi, et dans lequel on ne voit pas l’autorité publique prêter sa force d’une manière constante et réglée pour contraindre au payement de leurs dettes tous ceux qui sont en état de les acquitter. En un mot, le commerce et les manufactures seront rarement florissants dans un État où la justice du gouvernement n’inspirera pas un certain degré de confiance. Cette même con­fiance qui dispose de grands commerçants et de grands manufacturiers à se reposer sur la protection du gouvernement pour la conservation de leur propriété, dans les circonstances ordinai­res, les dispose à confier à ce gouvernement, dans les occasions extraordinaires, l’usa­ge même de cette propriété. En prêtant des fonds au gouverne­ment, ils ne se retran­chent rien, même pour le moment, des moyens de faire marcher leur commerce et leurs manufactures. Au contrait même, ils ajoutent souvent à ces moyens. Les besoins de l’État rendent le gouvernement très-disposé, dans la plupart des occasions, à emprunter à des conditions extrêmement avantageuses pour le prêteur. L’engagement que l’État prend envers le créancier primitif, ainsi que les sûretés accessoires de cet engagement, sont de nature à pouvoir se transmettre à tout autre créancier et, vu la confiance générale qu’on a dans la justice de l’État, on les vend, pour l’ordinaire, sur la place, à un prix plus haut que celui qui a été payé dans l’origine. Le marchand ou capitaliste se fait de l’argent en prêtant au gouvernement, et au lieu de diminuer les capitaux de son commerce, c’est pour lui une occasion de les augmenter. Ainsi, en général, il regarde comme une grâce du gouvernement d’être admis pour une portion dans la première souscription ouverte pour un nouvel emprunt ; de là la bonne volonté ou le désir que les sujets d’un État commerçant ont de lui prêter.
Le gouvernement d’un tel État est très-porté à se reposer sur les moyens ou la bonne volonté qu’ont ses sujets de lui prêter leur argent dans les occasions extraordi­naires. Il prévoit la facilité qu’il trouvera à emprunter, et pour cela il se dispense du devoir d’épargner.
Dans une société encore peu civilisée, il n’y a pas de ces grands capitaux qu’em­ploient le commerce et les manufactures. Les particuliers qui thésaurisent tout ce qu’ils peuvent ménager, et qui cachent leur trésor, n’agissent ainsi que par la défiance où ils sont de la justice du gouvernement, par la crainte qu’ils ont que, si l’on venait à leur savoir un trésor et à en connaître la place, ils n’en fussent bientôt dépouillés. Dans un tel état de choses, il y a bien peu de gens en état de prêter de l’argent au gouvernement dans ses besoins extraordinaires, et il n’y a personne qui en ait la bonne volonté. Le souverain sent qu’il lui faut pourvoir d’avance à de tels besoins par des épargnes, parce qu’il prévoit l’impossibilité absolue d’emprunter. Cette dernière considération ajoute encore à la disposition naturelle où il est de faire des épargnes.
Le progrès des dettes énormes qui écrasent à présent toutes les grandes nations de l’Europe, et qui probablement les ruineront toutes à la longue, a eu un cours assez uniforme. Les nations, comme les particuliers, ont commencé, en général, par em­prunter sur ce qu’on peut appeler le crédit personnel, sans assigner ou hypothéquer de fonds particuliers pour le payement de la dette ; et quand cette ressource leur a man­qué, elles en sont venues à emprunter sur des assignations ou sur l’hypothèque de fonds particuliers.
Ce qu’on appelle la dette non fondée de la Grande-Bretagne est contractée dans la première de ces deux manières. Elle consiste, partie en une dette qui ne porte pas, ou du moins est censée ne pas porter d’intérêt, et qui ressemble aux dettes que fait un particulier sur un compte courant, et partie en une dette portant intérêt, qui ressemble à celles qu’un particulier contracte sur des billets ou promesses. Les dettes qui ont pour cause, soit des services extraordinaires, soit des services pour lesquels il n’y a pas de fonds de fait, ou bien qui ne sont pas payés à l’époque où ils sont rendus ; une partie de l’extraordinaire de l’armée, de la marine et de l’artillerie ; l’arriéré des subsides qui se payent aux princes étrangers, celui des salaires des gens de mer, etc., constituent ordinairement une dette de la première sorte. Les billets de la marine et de l’échiquier, qui ont été émis tantôt en payement des dettes ci-dessus, et tantôt pour d’autres objets, constituent une dette de la seconde sorte ; les billets de l’Échiquier portant intérêt du jour de leur émission, et les billets de la marine six mois après la leur. La Banque d’Angleterre, soit en escomptant volontairement ces billets pour leur valeur au cours de la place, soit en convenant avec le gouvernement, par des arrangements particuliers, de soutenir la circulation des billets de l’Échiquier, c’est-à-dire de les recevoir au pair, et de bonifier l’intérêt qui se trouve être alors échu, en maintient la valeur et en facilite la circulation ; ce qui met souvent le gouvernement à même de contracter une très-forte dette de cette espèce. En France, où il n’y a pas de banque, les billets de l’État[3] se sont quelquefois vendus à 60 et 70 pour 100 de perte. Pendant la grande refonte de la monnaie, sous le roi Guillaume, quand la Banque d’Angleterre jugea nécessaire de suspendre ses opérations accoutumées, les billets de l’Échiquier et les coupons[4] ont subi, à ce qu’on dit, de 25 à 60 pour 100 de perte ; ce qui provenait, en partie sans doute, du peu de solidité qu’on supposait dans le gouverne­ment établi par la révolution, mais en partie aussi de ce que ces effets n’étaient pas soutenus par la Banque.
Lorsque cette ressource a été épuisée, et qu’il est devenu nécessaire, pour faire de l’argent, de donner une assignation ou hypothèque sur quelque branche particulière du revenu public pour le payement de la dette, le gouvernement a fait ceci, en diverses occasions, de deux manières différentes. Quelquefois il a donné cette assignation ou hypothèque pour un court espace de temps seulement, pour une année ou quelques années, par exemple ; et quelquefois il l’a donnée à perpétuité. Dans le premier cas, le fonds assigné était censé suffisant pour payer, dans ce temps limité, l’intérêt et le principal de l’argent emprunté. Dans l’autre cas, il était censé suffisant pour payer l’inté­rêt seulement ou une annuité perpétuelle équivalant à l’intérêt, le gouvernement ayant la faculté de racheter en tout temps cette annuité en remboursant le principal emprunté. Quand on empruntait de la première manière, cela s’appelait emprunter par anticipation ; et de l’autre, emprunter en faisant fonds à perpétuité, ou tout simplement en faisant fonds[5].
Dans la Grande-Bretagne, la taxe foncière et celle sur la drêche[6] sont régulière­ment anticipées tous les ans, en vertu d’une clause d’emprunt qui est insérée constamment dans les actes qui les imposent. Les sommes pour lesquelles ces taxes sont accordées sont, en général, avancées par la Banque d’Angleterre à un intérêt qui, depuis la révolution, a varié de 8 pour 100 à 3 pour 100, et elle reçoit son rembour­sement à mesure que le produit rentre successivement. S’il y a un déficit, ce qui arrive toujours, il y est pourvu dans ce qui est accordé pour les besoins de l’année suivante. La seule branche considérable du revenu public qui ne soit pas encore aliénée par une hypothèque à perpétuité, est ainsi régulièrement dépensée avant qu’elle soit rentrée. Comme un dissipateur sans prévoyance, à qui ses besoins toujours urgents ne permet­tent pas d’attendre le payement régulier de son revenu, l’État est dans la pratique constante d’emprunter de ses propres facteurs et agents, et de leur payer des intérêts pour l’usage de son propre argent.
Sous le règne du roi Guillaume, et pendant une grande partie de celui de la reine Anne, avant que nous nous fussions aussi familiarisés que nous le sommes aujour­d’hui avec la pratique de fonder à perpétuité, la plus grande partie des nouveaux impôts n’étaient établis que pour un terme court, pour quatre, cinq, six ou sept ans seulement ; et une grande partie des fonds accordés par le parlement, chaque année, consistait en emprunts sur des anticipations du produit de ces impôts. Le produit étant fort souvent insuffisant pour rembourser, dans le terme limité, le principal et l’intérêt de l’argent emprunté, il se forma des déficits, et, pour les couvrir, il devint nécessaire de proroger le terme.
En 1697, par le statut de la huitième année de Guillaume III, chapitre xx, les déficits de plusieurs impôts furent rejetés sur ce qu’on appela alors le premier fonds ou hypothèque générale, consistant en une prolongation, jusqu’au 1er août 1706, de plusieurs différents impôts qui auraient dû expirer dans un terme plus court, et dont le produit fut réuni en un fonds général. Les déficits dont on chargea cette prorogation d’impôts montaient à 5,150,459 liv. 14 sch. 9 d. 1/4.
En 1701, ces droits, avec quelques autres, furent encore continués, pour la même cause, jusqu’au 1er août 1710, et furent appelés le deuxième fonds ou hypothèque générale. Les déficits dont ce deuxième fonds fut chargé montaient à 2,055,999 liv. 7 sch. 11 d. 1/2.
En 1707, ces droits furent continués de nouveau jusqu’au 1er août 1712, comme fonds pour de nouveaux emprunts, et ils furent appelés le troisième fonds ou hypo­thèque générale. La somme empruntée sur ce fonds fut de 983,254 liv. 11 sch. 9 d. 1/4.
En 1708, ces droits (à l’exception de l’ancien subside de tonnage et pondage, dont une moitié seulement composa partie de ce fonds et d’un droit sur l’importation des toiles d’Écosse, qui a été supprimé par les clauses de l’union) furent tous continués, comme fonds pour de nouveaux emprunts, jusqu’au 1er août 1714, et ils furent appe­lés le quatrième fonds ou hypothèque générale. La somme empruntée sur ce fonds fut de 925,176 liv. 9 sch. 2 d. 1/4.
En 1709, ces droits (à l’exception de l’ancien subside de tonnage qui fut alors tout à fait retiré de ce fonds) furent tous encore continués, pour la même cause, jusqu’au 1er août 1716, et ils furent appelés le cinquième fonds ou hypothèque générale. La somme empruntée sur ce fonds fut de 922,029 liv. 6 sch.
En 1710, ces droits furent encore continués jusqu’au 1er août 1720, et furent appelés le sixième fonds ou hypothèque générale. La somme empruntée sur ce fonds fut de 1,296,552 liv. 9 sch. 11 d. 3/4.
En 1711, les mêmes droits (qui étaient ainsi à cette époque chargés de quatre différentes anticipations), ensemble plusieurs autres droits, furent continués pour toujours, et il en fut fait un fonds pour payer l’intérêt du capital de la compagnie de la mer du Sud, qui avait avancé cette année au gouvernement, pour payer les dettes et bonifier des déficits de taxes, une somme de 9,177,967 liv. 15 sch. 4 d., le plus gros emprunt qui eût été fait jusqu’alors.
Avant cette période, qui est la principale, autant que j’aie pu l’observer, les seuls impôts qui eussent été établis à perpétuité pour payer l’intérêt d’une dette, étaient ceux destinés à payer l’intérêt de l’argent avancé au gouvernement par la Banque et la compagnie des Indes, et de celui qu’on espérait qui serait avancé (mais qui ne l’a jamais été) par une banque territoriale projetée. Les fonds avancés par la Banque à cette époque montaient à 3,375,027 liv. 17 sch. 10 d. 1/2, pour lesquels il lui était payé une annuité ou intérêt de 206,501 liv. 13 sch. 5 d. Les fonds avancés par la compagnie des Indes montaient à 3,200,000 liv., pour lesquels il lui était payé une annuité ou intérêt de 160,000 liv., les fonds de la Banque étant à 6 pour 100 d’intérêt, et ceux de la compagnie des Indes à 5 pour 100.
En 1715, par le statut de la première année de George Ier, chap. xii, les différents impôts qui avaient été hypothéqués pour payer l’annuité de la Banque, ensemble plusieurs autres impôts qui furent rendus pareillement perpétuels par cet acte, furent tous réunis dans un fonds commun appelé le fonds agrégé, lequel fut chargé, non-seulement du payement de l’annuité de la Banque, mais encore de diverses autres annuités et payements de différentes sortes. Ce fonds fut ensuite augmenté par le statut de la troisième année de George Ier, chap. viii, et par celui de la cinquième de George Ier, chap. iii, et les différents droits qui y furent alors ajoutés furent pareillement rendus perpétuels.
En 1717, par le statut de la troisième année de George 1er, chap. vit, plusieurs autres impôts furent rendus perpétuels et réunis dans un autre fonds commun appelé le fonds général, destiné au payement de quelques annuités, montant en totalité à 724,849 liv. 6 sch. 10 d. 112.
En conséquence de ces différents actes, la plus grande partie des impôts, qui n’avaient été auparavant anticipés que pour un terme d’années assez court, furent rendus perpétuels pour faire un fonds destiné au payement, non pas du capital, mais de l’intérêt seulement de l’argent qui avait été emprunté sur les impôts par différentes anticipations successives.
Si l’on n’eût jamais fait d’emprunt que sur anticipation, il n’aurait fallu que quelques années pour la libération du revenu public, sans autre attention de la part du gouvernement que celle de ne pas surcharger le fonds anticipé en le chargeant de plus de dettes qu’il n’en pouvait payer dans le terme limité, et de ne pas anticiper une seconde fois avant l’expiration de la première anticipation. Mais il paraît qu’une telle attention a été impossible pour la plupart des gouvernements de l’Europe. Ils ont souvent surchargé le fonds anticipé, même dès la première anticipation, et quand cela ne s’est pas trouvé fait ainsi, ils n’ont généralement pas manqué de le surcharger en anticipant une seconde et une troisième fois avant l’expiration de la première antici­pation. Le fonds devenant de cette manière absolument insuffisant pour payer le principal et l’intérêt de l’argent emprunté, il fut nécessaire de le charger de l’intérêt seulement, ou d’une annuité perpétuelle égale à l’intérêt, et ces anticipations ainsi faites sans prévoyance rendirent indispensable la pratique plus ruineuse de faire des fonds à perpétuité. Mais quoique, par cette pratique, la libération du revenu public se trouve nécessairement renvoyée d’une période fixe à une autre tellement indéfinie qu’il y a fort à croire qu’elle n’arrivera jamais ; cependant, comme dans tous les cas on peut se procurer, par cette nouvelle pratique, une plus forte somme d’argent que par l’ancienne forme des anticipations, celle-là, dès que les hommes ont été familiarisés avec elle, a été universellement préférée à l’autre dans les grands besoins de l’État. Se tirer des besoins du moment est toujours l’objet qui occupe d’une manière principale ceux qui sont le plus immédiatement chargés de l’administration des affaires publi­ques. Quant à la libération future du revenu public, c’est un soin qu’ils laissent à la postérité.
Pendant le règne de la reine Anne, le taux de l’intérêt au cours de la place était tombé de 6 à 5 p. 100, et dans la douzième année de son règne on déclara 5 p. 100 l’intérêt le plus haut qu’il fût permis de prendre pour argent prêté entre particuliers. Bientôt après que la plus grande partie des impôts temporaires de la Grande-Bretagne eurent été rendus perpétuels et distribués dans les différents fonds, le fonds agrégé, le fonds de la mer du Sud et le fonds général, les créanciers de l’État, comme ceux des particuliers, furent amenés à accepter 5 p. 100 pour l’intérêt de leur argent ; ce qui procura une épargne de 1 p. 100 sur le capital de la plus grande partie des dettes qui avaient été ainsi fondées à perpétuité, ou d’un sixième de la plus grande partie des annuités qui se payaient sur les trois grands fonds ci-dessus. Cette épargne laissa, dans le produit des différents impôts qui avaient été réunis dans ces fonds, un excédant considérable au delà de ce qui était nécessaire pour payer les annuités dont ils se trouvaient alors chargés, et elle fut la base de ce qui a été appelé depuis le fonds d’amortissement. En 1717, cet excédent faisait un objet de 323,434 liv. 7 sch. 7 den. 1/2 ; en 1727, l’intérêt de la plus grande partie de la dette publique fut encore réduit et mis à 4 p. 100, et en 1753 et 1757, à 3 1/2 et 3 p. 100, toutes réductions qui gros­sirent encore le fonds d’amortissement.
Un fonds d’amortissement, quoique institué pour payer les dettes anciennes, facilite extrêmement les moyens d’en contracter de nouvelles. C’est un fonds subsi­diaire qu’on a toujours sous la main prêt à être hypothéqué pour venir au secours de quelque autre fonds douteux, et sur lequel on se propose d’emprunter de l’argent dans une nécessité publique. On verra tout à l’heure si le fonds d’amortissement de la Grande-Bretagne a été plus souvent appliqué à l’une de ces deux destinations qu’à l’autre[7]. 
Outre ces deux méthodes d’emprunter sur des anticipations et sur des fonds faits à perpétuité, il y a deux autres méthodes qui tiennent entre celles-là une sorte de milieu. C’est celle d’emprunter sur des annuités à terme, et celle d’emprunter sur des annuités viagères. 
Pendant les règnes de Guillaume et de la reine Anne, on emprunta fréquemment de très-grosses sommes sur des annuités a terme, dont le terme fut tantôt plus long, tantôt plus court. En 1693, il fut passé un acte pour emprunter 1 million sur une annuité de 14 p. 100, ou de 140,000 liv. par année pour seize ans. En 1691, fut passé un acte pour emprunter 1 million sur annuités viagères, à des conditions qui paraî­traient très-avantageuses aujourd’hui ; mais la souscription ne fut pas remplie. Dans le cours de l’année suivante, on bonifia le déficit en empruntant sur annuités viagères à 14 p. 100 ou à un peu plus du denier 7. En 1695, les personnes qui avaient acheté ces annuités furent autorisées à pouvoir les échanger contre d’autres annuités de quatre-vingt-seize années, en payant dans le trésor de l’Échiquier 63 p. 100, c’est-à-dire que la différence entre 14 p. 100 viagers et 14 p. 100 pendant quatre-vingt-seize ans, fut vendue pour 63 liv. ou bien au denier 4 1/2. Telle était pourtant l’opinion sur le peu de solidité du gouvernement, que de telles conditions même attirèrent fort peu d’acheteurs. Sous le règne de la reine Anne on emprunta souvent en différentes cir­cons­tances sur des annuités viagères et sur des annuités à terme de trente-deux, de quatre-vingt-neuf, de quatre-vingt-dix-huit et de quatre-vingt-dix-neuf ans. En 1719, les propriétaires d’annuités de trente-deux années furent invités à accepter, en rem­placement de ces annuités, des fonds de la compagnie de la mer du Sud, sur le pied du denier 11 1/2, c’est-à-dire équivalant à onze années et demie de leurs annuités, plus une quantité additionnelle de ces mêmes fonds, équivalant au montant des arrérages qui se trouvaient alors leur être dus sur ces annuités. En 1720, la majeure partie des autres annuités, tant à long qu’à court terme, furent converties en souscriptions dans les mêmes fonds. Les annuités à long terme montaient, à cette époque, à 666,821 liv. 8 sch. 3 den. 1/2 par an. Au 5 janvier 1775, ce qui en restait encore ou ce qui n’était pas converti en souscriptions ne montait plus qu’à 136,453 liv. 12 sch. 8 den.
Pendant le cours des deux guerres qui ont commencé en 1739 et en 1755, on emprunta peu sur annuités à terme ou sur annuités viagères. Cependant une annuité, pour avoir un terme de quatre-vingt-dix-huit ou quatre-vingt-dix-neuf années, vaut à peu près autant d’argent qu’une annuité perpétuelle, et devrait être, à ce qu’il semble d’abord, un moyen pour emprunter à peu près autant. Mais ceux qui achètent des effets publics dans la vue d’assurer des établissements à leur famille ou de faire un placement pour la postérité la plus reculée, ne se soucieraient guère de placer leur argent dans un effet dont la valeur va toujours en diminuant ; et les personnes de cette espèce font une portion très-considérable des propriétaires et acquéreurs de fonds publics. Ainsi, quoiqu’une annuité pour un long terme d’années ait, à très-peu de chose près, la même valeur intrinsèquement qu’une annuité perpétuelle, cependant elle ne trouvera pas, à beaucoup près, le même nombre d’acheteurs. Ceux qui sous­crivent pour un nouvel emprunt du gouvernement, songeant en général à revendre le plus tôt possible leurs souscriptions, préfèrent de beaucoup une annuité perpétuelle rachetable à la volonté du Parlement, à une annuité non rachetable pour un long terme d’années, et seulement de la même somme. La valeur de la première peut être regardée comme étant la même ou à très-peu de chose près la même en tout temps ; et par conséquent, comme effet commerçable et transmissible, elle est plus commode que l’autre.
Pendant le cours des deux dernières guerres ci-dessus, les annuités, soit à terme, soit viagères, n’ont guère été accordées que comme des primes en faveur des souscripteurs à un nouvel emprunt, en sus de l’annuité rachetable ou de l’intérêt sur le crédit duquel le nouvel emprunt était censé fait. On les créa, non pas comme faisant proprement partie du fonds sur lequel on empruntait, mais comme un surcroît d’en­cou­ragement pour le prêteur.
Les annuités viagères ont été, suivant les circonstances, créées de deux différentes manières, ou sur des vies séparées, ou sur des lots de plusieurs vies jointes ; ce qui fut nommé en français tontine, du nom de leur inventeur. Quand les annuités sont créées sur des vies séparées, la mort de chaque individu rentier dégrève le revenu public de la charge qu’y apportait sa rente. Quand on crée des annuités par tontines, la libé­ration du revenu publie ne commence qu’à la mort de la totalité des rentiers compris dans le même lot ou classe ; ce qui peut quelquefois composer un nombre de vingt ou trente personnes, dont les survivants succèdent aux annuités de tous ceux qui meurent avant eux, le dernier survivant succédant aux annuités de la classe entière. On peut, avec la même portion de revenu public, faire plus d’argent en empruntant par tontines, qu’en empruntant par des annuités sur des vies séparées. Une annuité avec un droit de survivance a réellement plus de valeur qu’une annuité pareille sur une tête séparée ; et vu la confiance que tout homme a naturellement dans sa bonne fortune, principe sur lequel est fondé le succès de toutes les loteries, une pareille annuité se vend toujours pour quelque chose de plus qu’elle ne vaut. Dans les pays où il est d’usage que le gouvernement emprunte sur des annuités, les tontines sont, par cette raison, préférées généralement aux annuités sur des têtes séparées. L’expédient qui fait trouver le plus d’argent est presque toujours préféré à celui qui pourrait faire espérer une plus prompte libération du revenu public.
En France, il y a une beaucoup plus grande portion de la dette publique qui consiste en annuités viagères, qu’en Angleterre. D’après un Mémoire présenté au roi par le parlement de Bordeaux, en 1764, la totalité de la dette publique de France est évaluée à 2 milliards 400 millions de livres tournois, dont il y a 300 millions, c’est-à-dire un huitième de toute la dette, qui forme le capital converti en rentes viagères. Ces rentes elles-mêmes sont calculées à 30 millions par an, le quart de 120 millions tournois, a quoi est porté l’intérêt de la totalité de la dette. Je sais fort bien que ces évaluations ne sont pas très-exactes ; mais ayant été représentées par une compagnie aussi respectable comme approchant de la vérité, j’imagine qu’on peut bien les considérer comme telles. Cette différence dans le mode d’emprunter entre la France et l’Angleterre ne provient pas de ce que l’un de ces deux gouvernements s’inquiète plus que l’autre de la libération du revenu public ; elle provient en entier de la différence dans les vues et les intérêts qui dirigent les prêteurs.
En Angleterre, le siège du gouvernement étant dans la plus grande ville com­mer­çante du monde, les commerçants sont en général les gens qui avancent de l’argent au gouvernement. Ils n’entendent pas, en faisant cette avance, diminuer les capitaux de leur commerce ; ils comptent bien, au contraire, les augmenter, et s’ils ne s’attendaient Pas à vendre avec profit leur part de souscription dans un nouvel emprunt, ils ne souscriraient jamais. Mais si, en avançant leur argent, il leur fallait acheter, au lieu d’annuités perpétuelles, des annuités viagères seulement, soit sur leurs têtes, soit sur d’autres, ils ne seraient pas toujours aussi assurés de pouvoir les vendre avec profit. Des annuités sur leurs têtes se vendraient toujours avec perte, parce qu’un homme n’ira jamais donner, d’une annuité sur la tête d’un tiers à peu près du même âge et de même santé que lui, le prix qu’il donnerait d’une annuité sur sa propre tête. À la vérité, une annuité sur la tête d’un tiers est sans contredit de la même valeur pour l’ache­teur que pour le vendeur ; mais sa valeur réelle n’en commence pas moins à diminuer du moment où elle est créée, et continue à diminuer toujours de plus en plus tant qu’elle subsiste. Une telle annuité ne peut donc jamais constituer un effet com­mer­çable aussi commode qu’une annuité perpétuelle, dont la valeur réelle peut être censée toujours la même ou très-approximativement la même.
En France, le siège du gouvernement n’étant pas dans une grande ville commer­çante, les commerçants n’y composent pas une portion aussi considérable de ceux qui avancent de l’argent au gouvernement. Les gens intéressés dans les finances, les fermiers généraux, les receveurs des impôts qui ne sont pas en ferme, les banquiers de la cour, etc., forment la majeure partie de ceux qui avancent leur argent dans tous les besoins publics. Ces gens-là sont ordinairement des gens d’une naissance com­mune, mais puissamment riches et souvent fort vains. Ils sont trop haut pour épouser leurs égales, et les femmes de qualité rougiraient de s’allier à eux.
Ils prennent donc souvent le parti de rester célibataires ; et n’ayant point de famille de leur chef, ne prenant pas grand intérêt à leurs parents qu’ils ne se soucient même pas toujours de reconnaître, ils n’ont d’autre désir que de passer leur vie dans l’éclat et l’opulence, et ne s’inquiètent pas que leur fortune finisse avec eux. D’ailleurs, la quantité de gens riches qui ont de l’éloignement pour le mariage, ou qui se trouvent dans une situation à ce que cet état leur soit ou peu convenable, ou moins commode, est bien plus grande en France qu’en Angleterre. Pour de pareilles gens qui ne s’embarrassent que peu ou point du tout de la postérité, il n’y a rien de plus commode que de pouvoir échanger leur capital contre un revenu qui doit durer tout juste aussi longtemps et pas plus longtemps qu’ils ne le souhaitent. 
La dépense ordinaire de la plus grande partie des gouvernements modernes, en temps de paix, étant égale ou à peu près égale à leur revenu ordinaire, quand la guerre survient, ils n’ont ni la volonté ni les moyens d’augmenter leur revenu à proportion de l’augmentation de leur dépense. Ils n’en ont pas la volonté dans la crainte de heurter le peuple, qu’un accroissement si fort et si subit d’impôt dégoûterait bien vite de la guerre ; ils n’en ont pas les moyens, parce qu’ils ne sauraient guère trouver de nouvel impôt suffisant pour produire le revenu dont ils ont besoin. La facilité d’emprunter les délivre de l’embarras que leur auraient causé sans cela cette crainte et cette impuis­sance. Au moyen de la ressource des emprunts, une augmentation d’impôts fort modérée les met à même de lever assez d’argent d’année en année pour soutenir la guerre ; et au moyen de la pratique de faire des fonds perpétuels ils se trouvent en état, avec la plus petite augmentation possible dans les impôts, de lever annuellement les plus grosses sommes d’argent. Dans de vastes empires, les gens qui vivent dans la capitale et dans les provinces éloignées du théâtre des opérations militaires ne ressen­tent guère, pour la plupart, aucun inconvénient de la guerre, mais ils jouissent tout à leur aise de l’amusement de lire dans les gazettes les exploits de leurs flottes et de leurs armées. Pour eux, cet amusement compense la petite différence des impôts qu’ils payent à cause de la guerre, d’avec ceux qu’ils étaient accoutumés à payer en temps de paix. Ils voient ordinairement avec déplaisir le retour de la paix, qui vient mettre fin à leurs amusements, et à mille espérances chimériques de conquête et de gloire nationale qu’ils fondaient sur la continuation de la guerre.
À la vérité, il est rare que le retour de la paix les soulage de la plupart des impôts mis pendant la guerre. Ces impôts sont affectés au payement des intérêts de la dette que la guerre a forcé de contracter. Si, par-delà le payement des intérêts de cette dette et l’acquit des dépenses ordinaires du gouvernement, l’ancien revenu, joint aux nouveaux impôts, produisait quelque excédent de revenu, peut-être pourrait-on le convertir en un fonds d’amortissement destiné au remboursement de la dette. Mais, en premier heu, ce fonds d’amortissement, quand même on supposerait qu’il ne fût jamais détourné de sa destination, est en général absolument disproportionné avec ce qu’il faudrait pour rembourser toute la dette occasionnée par la guerre, dans un espace de temps tel que celui pendant lequel on peut raisonnablement s’ attendre à conserver la paix ; et, en second lieu, ce fonds est presque toujours appliqué à quelque autre objet.
Les nouveaux impôts ont été mis dans la seule vue de payer l’intérêt de l’argent emprunté sur eux. S’ils produisent plus, c’est pour l’ordinaire ce à quoi on n’a pas son­gé ; c’est un produit sur lequel on n’a pas compté, et qui, par conséquent, ne peut pas être fort considérable. En général, les fonds d’amortissement ne sont guère résul­tés d’un excédent d’impôts levés au-delà de ce qui était nécessaire pour payer l’intérêt ou l’annuité originairement assignée sur ces impôts ; ils sont bien plutôt provenus de quelque réduction subséquemment faite dans cet intérêt. Celui de la Hollande, en 1655, et celui de l’État ecclésiastique en 1685, ont été l’un et l’autre créés de cette manière ; de là vient l’insuffisance ordinaire de ces sortes de fonds.
Pendant la paix la plus profonde, il survient divers événements qui exigent une dépense extraordinaire ; et le gouvernement trouve toujours plus commode de satis­faire à cette dépense, en détournant le fonds d’amortissement de sa destination, qu’en mettant un nouvel impôt. Tout nouvel impôt est senti sur-le-champ plus ou moins par le peuple. Il occasionne toujours quelque murmure et ne passe pas sans rencontrer de l’opposition. Plus les impôts ont été multipliés, plus on presse fortement chaque article d’imposition, et plus alors le peuple crie contre tout impôt nouveau, plus il de­vient difficile de trouver un nouvel objet imposable ou de porter plus haut les impôts déjà établis. Mais une suspension momentanée du rachat de la dette n’est pas sentie immédiatement par le peuple, et ne cause ni plaintes ni murmures. Emprunter sur le fonds d’amortissement est une ressource facile et qui se présente d’elle-même pour sortir de la difficulté du moment. Plus la dette publique se sera accumulée, plus il sera devenu indispensable de s’occuper sérieusement de la réduire, plus il est dan­ge­reux, ruineux même de détourner la moindre partie du fonds d’amortissement, moins alors il est à présumer que la dette publique puisse être réduite à un degré un peu consi­dérable ; plus il faut s’attendre, plus il est infaillible que le fonds d’amor­tissement sera détourné pour couvrir toute la dépense extraordinaire qui peut survenir en temps de paix. Quand une nation est déjà surchargée d’impôts, il n’y a que les besoins impé­rieux d’une nouvelle guerre, il n’y a que l’animosité de la vengeance nationale ou l’inquiétude pour la sûreté de la patrie qui puisse amener le peuple à se soumettre, avec un peu de patience, au joug d’un nouvel impôt : de là vient que les fonds d’amor­tissement sont si ordinairement détournés de leur destination[8].
Dans la Grande-Bretagne, du moment que nous avons eu recours à l’expédient ruineux de faire des fonds perpétuels, la réduction de la dette publique, en temps de paix, n’a jamais gardé aucune espèce de proportion avec son accumulation en temps de guerre.
Ce fut dans la guerre qui commença en 1688, et qui fut terminée par le traité de Ryswick en 1697, que furent jetés les fondements de cette dette énorme qui pèse aujourd’hui sur la Grande-Bretagne. Au 31 décembre 1697, la dette publique de l’Angleterre, tant ce qui était fondé que ce qui était non fondé, se montait à 21,515,742 livres 13 sch. 8 d. 1/2. Une grande partie de cette dette avait été contractée sur des anti­ci­pations à court terme, et une partie sur des annuités viagères ; de manière qu’avant le 31 décembre 170 1, en moins de quatre années, il avait été amorti, tant par des remboursements que par les extinctions, une somme de 5,121,041 liv. 22 sch. 0 den. 3/4 den., la plus grande réduction qui ait jamais été faite depuis dans la dette publi­que en un espace de temps aussi court. Le restant de la dette se trouva donc n’être plus que de 16,394,701 liv. 1 sch. 7 den. 1/4.
Dans la guerre qui commença en 1702, et qui fut terminée par le traité d’Utrecht, la dette publique grossit encore davantage. Au 31 décembre 1714, elle se montait à 53,681,076 liv. 5 sch. 6 den 1/12. Les souscriptions qui furent faites des annuités à long et à court terme, dans les fonds de la compagnie de la mer du Sud, ajoutèrent au capital de la dette publique, de manière qu’au 31 décembre 1722 il s’élevait à 55,282,978 liv. 1 sch. 3 den. 5/6. La réduction de la dette commença en 1723, et elle alla si lentement qu’au 31 décembre 1739, pendant dix-sept années d’une profonde paix, la totalité des remboursements faits n’excéda pas 8,323,354 liv. 17 sch. 11 den. 3/12, le capital de la dette publique se montant, à cette époque, à 46,954,623 liv. 3 sch. 4 den. 7/12.
La guerre d’Espagne, qui commença en 1739, et la guerre de France, qui la suivit de près, portèrent la dette plus haut qu’elle n’avait encore été ; et, au 31 décembre 1748, après la guerre terminée par le traité d’Aix-la-Chapelle, le capital dû était de 78,293,313 liv. 1 sch. 11 den. 3/4. La paix la plus profonde, prolongée pendant dix-sept années de suite, n’avait ôté de cette dette que 8,328,354 liv. 17 sch. 11 den. 3/12. Une guerre de moins de neuf ans de durée y ajouta 31,338,689 liv. 18 sch. 6 den. 1/6[9].
Pendant l’administration de M. Pelham, on réduisit l’intérêt de la dette publique, ou du moins on prit des mesures pour qu’il se trouvât réduit de 4 à 3 pour 100[10]. On augmenta le fonds d’amortissement et on acheta une partie de la dette publique. En 1755, avant que la dernière guerre eût éclaté, la dette fondée de la Grande-Bretagne se montait à 72,289,673 liv. Au 5 janvier 1763, à la conclusion de la paix, la dette fondée se trouva être de 122,603,336 liv. 8 sch. 2 den. 1/4. La dette fondée fut réglée à 13,927,589 liv. 2 sch. 2 den. Mais la dépense dont la guerre avait été la source ne prit pas fin par la conclusion de la paix, de manière que, bien qu’au 5 janvier 1764 (partie à cause d’un nouvel emprunt, partie parce qu’on avait fondé une portion de la dette non encore fondée) le capital de la dette fondée se trouvât porté jusqu’à 129,586,739 liv. 10 sch. 1 den. 3/4, il restait encore, suivant l’auteur des Considérations sur le commerce et les finances de la Grande-Bretagne, qui a écrit d’après de très-bons renseignements, une dette non fondée, qui fut portée au compte de cette année et de la suivante et qui n’allait pas à moins de 9,975,017 liv. 12 sch. 2 den. 15/44. Ainsi en 1764, la dette publique de la Grande-Bretagne, tant fondée que non fondée, se mon­tait, d’après cet auteur, à 139,561,807 liv. 2 sch. 4 den. 1/11. De plus, les annuités viagères qui avaient été créées comme primes pour les souscripteurs dans le nouvel emprunt de 1757, estimées sur le pied du denier 14, furent portées pour 472,500 liv. ; et les annuités à long terme d’années, créées pareillement comme primes en 1761 et 1762, estimées sur le pied du den. 27 1/2, furent comptées pour 6,826,875 liv.[11]. Pendant une paix de sept années environ, l’administration sage et vraiment patriotique de M. Pelham ne put venir à bout de rembourser 6 millions sur l’ancienne dette ; et pendant une guerre de même durée à peu près, de nouvelles dettes furent contractées pour plus de 75 millions.
Au 5 janvier 1775, la dette fondée de la Grande-Bretagne s’élevait à 124,996,086 liv. sch. 6 den. 1/4 ; la dette non fondée, sans y comprendre une forte dette de la liste civile, allait à 4,150,236 liv. 3 sch. 11 den. 7/8. L’une et l’autre réunies formaient un total de 129,146,322 liv. 2 sch. 6 den. 1/8. D’après ce compte, la totalité des rembour­sements faits sur la dette pendant onze années d’une paix profonde, ne montait qu’à 10,415,484 liv. 16 sch. 9 den. 7/8 ; encore cette légère réduction de la dette n’est-elle pas tout le fruit d’épargnes sur le revenu ordinaire de l’État. Plusieurs sommes pro­ve­nant d’objets étrangers, et totalement indépendantes de ce revenu ordinaire avaient contribué à cette réduction. Parmi ces objets, on ne peut compter le sou pour livre additionnel à la taxe foncière pour trois années, les deux millions reçus de la com­pagnie des Indes Orientales pour indemnité de ses acquisitions territoriales, et les 110,000 liv. reçues de la Banque pour le renouvellement de sa charte. Il faut ajouter à ceci diverses autres sommes qui, étant des produits de la dernière guerre, devraient peut-être venir en déduction des dépenses qu’elle a coûtées. Les principales de ces sommes sont :
Le produit des prises françaises	690,449	l.	18 sch.	9 d.
La composition faite pour les prisonniers français	670,000		-	-
Ce qui a été reçu de la vente des îles cédées[12]	95,600		-	-
Total	1,455,949		18	9
En ajoutant à la somme ci-dessus la balance des comptes du comte de Chatam et de M. Calcraft, et d’autres restes du même genre sur les fonds de l’armée, ensemble ce qui a été reçu de la banque de la compagnie des Indes, et le sou pour livre additionnel de la taxe foncière, le total ira bien largement au-delà de cinq millions. Ainsi, ce qui a été racheté de la dette depuis la paix, sur les économies du revenu ordinaire de l’État, n’a pas été, une année dans l’autre, à un demi-million par an. Sans contredit, le fonds d’amortissement en a été considérablement augmenté depuis la paix, au moyen des remboursements faits sur la dette, de la réduction des 4 pour 100 rachetables remis à 3 pour 100, et des annuités viagères qui se sont éteintes ; et si la paix pouvait durer, on pourrait peut-être économiser aujourd’hui un million par an sur le revenu, pour servir à la liquidation de la dette. Aussi a-t-on remboursé un autre million dans le cours de l’année dernière ; mais en même temps il y a une énorme dette de la liste civile qui reste sans être payée, et nous voici maintenant enveloppés dans une nouvelle guerre qui peut bien, dans ses progrès, devenir tout aussi dispendieuse qu’aucune de nos guer­­res précédentes[13]. Vraisemblablement la nouvelle dette qui va se trouver contrac­tée avant la fin de la campagne prochaine égalera, à peu de chose près, tout ce qui a été remboursé de l’ancienne avec les économies faites sur le revenu ordinaire de l’État. Ce serait donc une pure chimère que de s’attendre à voir jamais la dette publique com­plètement acquittée par le moyen d’épargnes, quelles qu’elles fussent, sur le revenu ordinaire tel qu’il subsiste à présent.
Il y a un auteur[14] qui a représenté les fonds publics des différentes nations endettées de l’Europe, et spécialement ceux de l’Angleterre, comme l’accumulation d’un grand capital ajouté aux autres capitaux du pays, au moyen duquel son commerce a acquis une nouvelle extension, ses manufactures se sont multipliées, et ses terres ont été cultivées et améliorées beaucoup au-delà de ce qu’elles l’eussent été au moyen de ses autres capitaux seulement.
Cet auteur ne fait pas attention que le capital avancé au gouvernement par les premiers créanciers de l’État était, au moment où ils ont fait cette avance, une portion du produit annuel, qui a été détournée de faire fonction de capital pour être employée à faire fonction de revenu, qui a été enlevée à l’entretien des ouvriers productifs pour servir à l’entretien de salariés non productifs, et pour être dépensée et dissipée dans le cours, en général, d’une seule année, sans laisser même l’espoir d’aucune reproduction future. À la vérité, en retour du capital par eux avancé, ils ont obtenu une annuité dans les fonds publics, qui le plus souvent valait au moins autant. Sans contredit, cette annuité leur a remplacé leur capital, et les a mis en état de faire aller leur commerce et leurs affaires avec tout autant et peut-être plus d’étendue qu’auparavant, c’est-à-dire qu’ils se sont trouvés à même d’emprunter à des tiers un nouveau capital sur le crédit de cette annuité, ou bien, en la vendant, de retirer de quelque tierce personne un autre capital à elle appartenant, égal ou supérieur à celui qu’ils avaient avancé au gouvernement. Mais ce nouveau capital qu’ils ont ainsi acheté ou emprunté de tierces personnes, il fallait bien qu’il existât dans le pays auparavant, et qu’il y fût déjà employé, comme le sont tous les capitaux, à entretenir du travail productif. Quand ce capital est venu à passer dans les mains de ceux qui avaient avancé leur argent au gouvernement, s’il était pour eux, à certains égards, un nouveau capital, il n’en était pas un nouveau pour le pays ; ce n’était autre chose qu’un capital retiré de certains emplois particuliers pour être tourné vers d’autres. Bien qu’il remplaçât pour eux ce qu’ils avaient avancé au gouvernement, il ne le remplaçait pas pour le pays. S’ils n’eussent point fourni leur capital au gouvernement, il y aurait eu alors dans le pays deux capitaux au lieu d’un, deux portions du produit annuel au lieu d’une, employées à entretenir du travail productif.
Lorsque pour couvrir la dépense du gouvernement on lève un revenu, dans le cours de l’année, avec le produit de quelque impôt libre et non déjà hypothéqué, il n’y a alors qu’une certaine portion du revenu des particuliers qui soit ôtée à l’entretien d’une espèce de travail non productif, pour aller à l’entretien d’une autre espèce de travail du même genre. Il y aurait eu sans doute quelque portion de ce que ces particuliers payent pour ces impôts, qui aurait été accumulée par eux en capital, et qui aurait, par conséquent, servi à entretenir du travail productif ; mais la plus grande partie aurait été dépensée et, par conséquent, employée à entretenir du travail non productif. Sans doute, quand la dépense publique est défrayée de cette manière, elle empêche plus ou moins qu’il ne se fasse des accumulations de nouveaux capitaux, mais au moins elle n’entraîne pas nécessairement la destruction de quelque capital actuellement existant.
Lorsque la dépense publique est défrayée par des créations de fonds, alors elle est défrayée par la destruction annuelle de quelque capital qui avait existé auparavant dans le pays, par le détournement de quelque portion du produit annuel qui était aupa­ravant destinée à entretenir du travail productif, et qui va à l’entretien du travail non productif. Néanmoins, comme dans ce cas les impôts sont plus légers qu’ils ne l’ eussent été si on eût levé, dans le cours de l’année, un revenu suffisant pour défrayer la même dépense, dès lors le revenu privé des citoyens est nécessairement moins chargé et, par conséquent, on ôte beaucoup moins aux moyens qu’ils peuvent avoir d’épargner et d’accumuler en capital une partie de ce revenu. Si la méthode de créer des fonds détruit plus l’ancien capital que ne le fait la méthode de pourvoir aux dépenses publiques par un revenu levé à mesure dans le cours de l’année, d’un autre côté cette première méthode empêche moins que l’autre la formation ou l’acquisition de quelque nouveau capital. Avec le système de créer des fonds perpétuels, l’écono­mie et l’industrie des particuliers peuvent réparer plus aisément les brèches que font de temps en temps au capital général de la société les dissipations et les profusions du gouvernement.
Ce n’est néanmoins que pendant la durée de la guerre que le système de créer des fonds perpétuels a cet avantage sur l’autre système. Si l’on pourvoyait toujours aux dépenses de la guerre avec un revenu qui se levât dans le cours de l’année, les impôts dont on tirerait ce revenu extraordinaire ne dureraient pas alors plus longtemps que la guerre elle-même. Si les moyens d’accumuler étaient moindres chez les particuliers tant que durerait la guerre, ils seraient aussi plus grands pendant la paix qu’ils ne l’auraient été avec le système des fonds perpétuels. La guerre n’aurait entraîné la destruction nécessaire d’aucun des anciens capitaux, et la paix aurait amené l’accu­mulation d’un nombre plus grand de nouveaux. Les guerres seraient, en général, plus promptement terminées, et on les entreprendrait avec moins de légèreté. Le peuple, sentant tout le poids du fardeau de la guerre pendant le temps qu’elle durerait, en deviendrait bientôt las, et le gouvernement ne se trouverait plus obligé, par condes­cendance pour ses fantaisies, de la continuer plus longtemps qu’il ne serait nécessaire. La perspective des charges lourdes et inévitables qu’amènerait la guerre empêcherait aussi le peuple de la vouloir trop légèrement, et à moins d’un intérêt réel et solide qui en valût la peine. Ainsi, ces périodes pendant lesquelles s’affaibliraient les moyens que les particuliers ont d’amasser des capitaux seraient à la fois plus rares et d’une plus courte durée. Celles, au contraire, où ces moyens auraient toute leur force, se­raient beaucoup plus durables qu’elles ne peuvent l’être avec le système des fonds perpétuels.
D’ailleurs, quand la création des fonds perpétuels a fait un certain progrès, alors la quantité d’impôts permanents dont elle grève les particuliers affaiblit quelquefois tout autant, même pendant la paix, les moyens d’amasser des capitaux que l’autre système le ferait en temps de guerre. Le revenu public de la Grande-Bretagne, en temps de paix, se monte à présent à plus de 10 millions par an. S’il était libre et sans hypothè­que, il serait suffisant, avec une bonne administration, pour soutenir la guerre la plus vigoureuse sans contracter un sou de dettes nouvelles. Le revenu privé des habitants de la Grande-Bretagne est à présent aussi chargé en temps de paix[15], leurs moyens pour accumuler sont autant affaiblis qu’ils eussent pu l’être pendant le temps de la guerre la plus dispendieuse, si le funeste système des fonds perpétuels n’eût jamais été adopté.
Dans les payements qui se font des intérêts de la dette publique, a-t-on dit, c’est la main droite qui paie à la main gauche. L’argent ne sort pas du pays. C’est seulement une partie du revenu d’une classe d’habitants qui est transportée à une autre classe, et la nation n’en est pas d’un denier plus pauvre.
Cette apologie est tout à fait fondée sur les idées sophistiques de ce système mercantile que j’ai combattu dans le livre IVe de ces Recherches, et après la longue réfutation que j’ai faite de ce système, il est peut-être inutile d’en dire davantage sur cette matière. C’est supposer d’ailleurs que la totalité de la dette publique appartient aux habitants de ce pays ; ce qui ne se trouve nullement vrai, les Hollandais, aussi bien que les autres nations étrangères, ayant une part très-considérable dans nos fonds publics. Mais quand même la totalité de la dette appartiendrait à des nationaux, ce ne serait pas une raison de conclure qu’elle n’est pas un mai extrêmement pernicieux[16]. 
La terre et les capitaux sont les deux sources primitives de tous revenus, tant publics que particuliers. Les capitaux payent les salaires du travail productif de quel­que manière qu’il soit employé, dans l’agriculture, dans les manufactures ou dans le commerce. L’administration de ces deux sources primitives de revenu appartient à deux différentes classes de personnes, les propriétaires de terre et les possesseurs de capitaux, ou ceux qui les font valoir. 
Le propriétaire de terre, pour conserver son revenu, est intéressé à tenir son bien en aussi bon état qu’il lui est possible, en bâtissant et réparant les logements de ses fermiers, en faisant et en entretenant les saignées et les clôtures nécessaires, et toutes ces autres améliorations dispendieuses qu’il appartient proprement au propriétaire de faire et d’entretenir. Mais une excessive contribution foncière peut retrancher une si forte part du revenu du propriétaire, et les divers droits sur les choses propres aux besoins et aisances de la vie peuvent tellement diminuer la valeur réelle de ce revenu déjà réduit, que le propriétaire se trouve tout à fait hors d’état de faire ou d’entretenir ces améliorations dispendieuses. Cependant, quand le propriétaire cesse de remplir sa partie, il est absolument impossible que le fermier continue à remplir la sienne. À mesure qu’augmente l’état de gêne du propriétaire, il faut de toute nécessité que la culture du pays aille en dépérissant.
Quand, par l’effet de la multiplicité des impôts sur les choses propres aux besoins et aisances de la vie, les capitalistes et ceux qui font valoir des capitaux viennent à s’apercevoir que, quelque revenu qu’ils puissent retirer de leurs fonds, ce revenu n’achètera jamais, dans le pays où ils sont, la même quantité de ces choses que ce qu’ils en auraient dans tout autre pays avec le même revenu, ils sont portés à chercher quelque autre résidence. Et quand, à raison de la perception de ces impôts, tous ou la plus grande partie des marchands et manufacturiers, c’est-à-dire tous ou la plus gran­de partie de ceux qui font valoir de grands capitaux, viennent à être continuellement exposés aux visites fâcheuses et aux recherches vexatoires des collecteurs de l’impôt, cette disposition à changer de résidence se réalise bientôt par une émigration. L’in­dustrie du pays tombera nécessairement quand on lui aura retiré les capitaux qui la soutenaient, et la ruine du commerce et des manufactures suivra le dépérissement de l’agriculture.
Une opération qui enlève aux possesseurs de ces deux grandes sources de revenu (la terre et les capitaux), aux personnes intéressées immédiatement à ce que chaque portion de terre soit tenue en bon état et à ce que chaque portion du capital soit avan­tageusement dirigée, la plus grande partie des revenus provenant de l’un ou de l’autre de ces sources, pour la transmettre à une autre classe de gens, les créanciers de l’État, qui n’ont nullement cet intérêt, une telle opération doit nécessairement faire, à la longue, que les terres se négligent et que les capitaux se dissipent ou fuient ailleurs. Un créancier de l’État a, sans contredit, un intérêt général à la prospérité de l’agri­culture, des manufactures et du commerce du pays et, par conséquent, à ce que les terres y soient tenues en bon état et les capitaux avantageusement dirigés. Si quelqu’une de ces choses venait à manquer ou à dépérir généralement, le produit des différents impôts ne serait plus suffisant pour lui servir l’annuité ou l’intérêt qui lui est dû. Mais un créancier de l’État, considéré simplement comme tel, n’a aucun intérêt à ce que telle portion de terre soit en bonne valeur, ou telle portion particulière de capi­tal avantageusement dirigée. Comme créancier de l’État, il ne connaît aucune portion particulière de terre ou de capital ; il n’en a aucune sous son inspection. Il n’y en a aucune dont il puisse s’occuper ; il n’y en a pas une en particulier qui ne puisse être totalement anéantie sans que le plus souvent même il s’en doute ou au moins qu’il en soit affecté directement.
La pratique de créer des fonds perpétuels a successivement affaibli tout État qui l’a adoptée. Il semble que ce sont les républiques d’Italie qui ont commencé à en faire usage. Gênes et Venise, les deux seules de ces républiques qui puissent encore pré­ten­dre à une existence indépendante, se sont l’une et l’autre affaiblies par cette prati­que. L’Espagne paraît avoir emprunté cette méthode aux républiques d’Italie ; et comme ses impôts sont vraisemblablement établis moins judicieusement que les leurs, elle a souffert d’une telle pratique encore plus qu’elles, à proportion de ses for­ces naturelles. La dette de l’Espagne est d’une date fort ancienne. Ce royaume était déjà très-obéré avant la fin du xvie siècle, environ cent ans avant que l’Angleterre dût un sou. La France, malgré toutes ses ressources naturelles, languit sous un far­deau accablant du même genre. La république des Provinces-Unies est aussi épuisée par les dettes que l’est Gênes ou Venise. Est-il à présumer qu’une pratique qui a porté avec elle la langueur ou la détresse dans tout autre pays, sera, pour la Grande-Bretagne seule, exempte de suites fâcheuses[17] ?
On dira peut-être que le système d’imposition établi dans ces différents pays est inférieur à celui de l’Angleterre. Je le crois bien aussi. Mais il ne faut pas oublier que le gouvernement le plus sage, quand il a épuisé tous les objets propres à être imposés, se trouve réduit, dans le cas de nécessité urgente, à recourir à ceux qui n’y sont pas propres. La prudente république de Hollande s’est vue obligée, dans certaines occa­sions, d’avoir recours à des espèces d’impôts tout aussi nuisibles que la plupart de ceux de l’Espagne. Une nouvelle guerre commencée avant qu’on soit venu à bout de procurer aucun soulagement considérable au revenu public, et qui peut, dans le cours de ses progrès, devenir aussi dispendieuse que l’a été la dernière, pourrait bien, par l’impulsion d’une irrésistible nécessité, nous entraîner dans un système d’impositions tout aussi oppressif que celui de la Hollande, ou même que celui de l’Espagne[18]. À la vérité, on peut dire, à la gloire de notre système actuel d’imposition, qu’il a jusqu’à ce moment causé si peu de gêne à l’industrie, que, même pendant la durée des guerres les plus ruineuses, l’économie et la bonne conduite des particuliers ont pu suffire, à ce qu’il semble, à force d’épargnes et d’accumulations, à réparer toutes les brèches que les dissipations et les excessives dépenses du gouvernement avaient faites au capital général de la société. À la conclusion de la dernière guerre, la plus coûteuse que la Grande-Bretagne ait jamais eu à soutenir, son agriculture était aussi florissante, ses manufactures aussi nombreuses et aussi pleinement en activité, son commerce aussi étendu qu’ils l’avaient jamais été auparavant. Il faut donc que le capital qui maintenait en activité toutes ces différentes branches d’industrie ait été égal à ce qu’il a jamais pu être auparavant. Depuis la paix, l’agriculture a reçu encore de nouvelles améliorations ; les loyers ont augmenté de prix dans toutes les villes et villages du royaume, preuve d’une augmentation d’opulence et de revenu parmi le peuple, montant annuel de la plupart des anciens impôts, et en particulier des bran­ches principales de l’accise et des douanes, a toujours été en augmentant ; preuve éga­le­ment évidente d’une consommation sans cesse croissante et, par conséquent, d’une augmentation dans le produit, sans quoi cette consommation n’eût pas pu se mainte­nir. La Grande-Bretagne paraît porter avec facilité un fardeau que personne, il y a un demi-siècle, ne l’eût crue capable de soutenir. N’allons pas cependant pour cela en conclure follement qu’elle soit en état d’en porter bien d’autres, ni même nous flatter trop qu’elle puisse, sans une très-grande gêne, recevoir un poids un peu plus lourd que celui qui pèse déjà sur elle[19].
Quand la dette nationale s’est une fois grossie jusqu’à un certain point, il n’y a pas, je crois, un seul exemple qu’elle ait été loyalement et complètement payée. Si jamais la libération du revenu public a été opérée tout à fait, elle l’a toujours été par le moyen d’une banqueroute, quelquefois par une banqueroute ouverte et déclarée, mais tou­jours par une banqueroute réelle, bien que déguisée souvent sous une apparence de payement.
L’expédient le plus ordinaire qu’on ait mis en œuvre pour déguiser une vraie banqueroute nationale sous l’apparence d’un prétendu payement, c’est de hausser la dénomination de la monnaie. Si, par exemple, par un acte du parlement ou par une proclamation royale, une pièce de 6 pence venait à être portée à la dénomination de 1 schelling, et 20 pièces de 6 pence à celle de 1 livre sterling, la personne qui, dans le temps de l’ancienne dénomination, aurait emprunté 20 sous ou à peu près quatre onces d’argent, pourrait, sous le régime de la nouvelle dénomination, payer sa dette avec vingt pièces de 6 pence ou avec quelque chose de moins que deux onces d’ar­gent. De cette manière, une dette nationale d’environ 128 millions (le capital à peu près de la dette fondée et non fondée de la Grande-Bretagne)[20], pourrait se payer avec environ 64 millions de notre monnaie actuelle. Ce ne serait, à la vérité, qu’une apparence de payement, et dans la réalité on aurait fait tort aux créanciers de l’État de 10 sous par livre de ce qui leur était dû. Le dommage s’étendrait aussi beaucoup plus loin qu’aux créanciers de l’État ; ceux de chaque particulier auraient la même perte à essuyer, et cela sans aucun avantage pour les créanciers de l’État, mais même avec un grand surcroît de perte pour ceux-ci. À la vérité, si un créancier de l’État était endetté envers d’autres personnes, il pourrait, jusqu’à un certain point, compenser sa perte en payant ses créanciers de la même monnaie que celle dont il aurait été payé par l’État. Mais dans presque tout pays les créanciers de l’État sont, pour la plupart, des gens opulents, plutôt sur le pied de créanciers, que sur celui de débiteurs avec le reste de leurs concitoyens. Ainsi, un prétendu payement de ce genre aggrave le plus souvent la perte des créanciers de l’État au lieu de le soulager ; et sans aucun avantage pour le public, il étend la plaie sur un grand nombre d’autres personnes qui ne devraient y être pour rien. Il cause dans les fortunes des particuliers une subversion générale et de l’espèce la plus funeste, en enrichissant le plus souvent le débiteur fainéant et dissi­pateur, aux dépens du créancier industrieux et économe, et en ôtant une grande partie du capital national aux mains qui auraient pu l’augmenter et le faire prospérer, pour le faire passer dans celles qui sont les plus propres à le dissiper et à l’anéantir. Quand un État se trouve réduit à la nécessité de faire banqueroute, tout comme quand un particulier s’y trouve réduit, une banqueroute franche, ouverte et déclarée est toujours la mesure qui est la moins déshonorante pour le débiteur, et en même temps la moins nuisible au créancier. À coup sûr, l’honneur de l’État est fort mal mis à couvert quand, pour déguiser la disgrâce d’une véritable banqueroute, il a recours à une misérable jonglerie de cette espèce, qu’il est si aisé à tout le monde d’apercevoir, et qui en même temps a les suites les plus pernicieuses.
Cependant presque tous les États, les anciens comme les modernes, quand ils se sont vus réduits à une telle nécessité, ont fait ressource de ce vrai tour d’escamotage. Les Romains, à la fin de la première guerre punique, réduisirent l’as (qui était la monnaie ou la dénomination par laquelle ils évaluaient toutes les autres monnaies), de douze onces de cuivre qu’il contenait, à deux onces seulement, c’est-à-dire qu’ils éle­vè­rent deux onces de cuivre à une dénomination qui avait toujours exprimé aupara­vant la valeur de douze onces. La république se trouvait, par ce moyen, à même de Payer les dettes énormes qu’elle avait contractées, avec un sixième seulement de ce qu’elle devait réellement. Nous serions aujourd’hui assez disposés à croire qu’une banqueroute aussi forte et aussi subite aurait dû causer les plus violentes clameurs popu­laires. Il ne paraît pas cependant qu’elle en ait occasionné aucune. La loi qui porta cette banqueroute fut, comme toutes les autres lois relatives aux monnaies, proposée et soutenue par un tribun, qui la fit passer dans une assemblée du peuple, et ce fut probablement une loi très-populaire. À Rome, comme dans toutes les autres républiques anciennes, les pauvres étaient perpétuellement endettés envers les riches et les grands, qui pour s’assurer des suffrages aux élections annuelles, avaient coutume de leur prêter de l’argent à un intérêt énorme, lequel, n’étant jamais payé, grossissait bientôt la dette dans une proportion telle, qu’il était impossible au débiteur de la payer ni de trouver personne qui la payât pour lui. Le débiteur, dans la crainte d’une exécution rigoureuse, était obligé, sans recevoir aucune gratification ultérieure, de voter pour le candidat que lui recommandait son créancier. En dépit de toutes les lois portées contre la corruption et la vente des suffrages, les largesses des candidats, jointes aux distributions de blé ordonnées de temps à autre par le sénat, étaient le fonds principal qui fournissait à la subsistance des pauvres citoyens, dans les derniers temps de la république. Pour se délivrer de cet assujettissement envers leurs créan­ciers, les citoyens pauvres étaient continuellement à demander, ou une entière abolition des dettes, ou ce qu’ils appelaient de nouvelles tables, c’est-à-dire une loi qui pût les autoriser à se faire donner une décharge complète en payant seulement une portion déterminée de leurs dettes accumulées. L’équivalent des nouvelles tables les plus avantageuses qu’ils pussent désirer, c’était la loi qui réduisait à un sixième de leur ancienne valeur la monnaie de toute dénomination, puisqu’elle les mettait à même de payer leurs dettes avec un sixième de ce qu’ils devaient réellement. Les grands et les riches avaient déjà été obligés, en plusieurs occasions, pour contenter le peuple, de consentir à des lois, tant pour l’abolition des dettes, que pour l’introduction de nouvelles tables ; et vraisemblablement ce qui les engagea à consentir de même à celle-ci, ce fut en partie le même motif, et en partie l’espoir que la libération du revenu public pourrait redonner de l’énergie à un gouvernement dont ils avaient la principale direction. Une opération de ce genre réduirait tout d’un coup une dette de 128 millions à 21,333,333 liv. 6 sch. 8 den. Dans le cours de la seconde guerre puni­que, l’as fut encore réduit de nouveau, d’abord de deux onces de cuivre à une once, et ensuite d’une once à une demi-once, c’est-à-dire à un vingt-quatrième de sa valeur primitive. En réunissant les trois opérations en une seule, une dette de 128 millions de notre monnaie actuelle pourrait par là se trouver tout d’un coup convertie en une dette de 5,333,333 liv. 6 sch. 8 den. De cette manière la dette de la Grande-Bretagne, tout énorme qu’elle est, se trouverait bientôt éteinte.
Il n’y a, je crois, aucune nation dont la monnaie, à la faveur de ces sortes d’expé­dients, n’ait été successivement réduite de plus en plus au-dessous de sa valeur originaire, de sorte que la même somme nominale en est venue par degrés à contenir une quantité d’argent de plus en plus petite.
Quelquefois les nations ont, par le même motif, altéré le titre de leurs monnaies, c’est-à-dire qu’elles y ont mêlé une plus grande quantité d’alliage. Si, par exemple, dans une livre pesant de notre monnaie d’argent, au lieu de 18 deniers pesant d’allia­ge, conformément au titre actuel, on y en mêlait huit onces, 1 livre sterling ou 20 sch. de cette monnaie ne vaudraient guère plus de 6 sch. 8 deniers de notre mon­naie actuelle. La quantité d’argent que contiennent 6 sch. 8 deniers de notre monnaie actuelle se trouverait portée ainsi à très-peu de chose près à la dénomination de 1 livre sterling. L’altération dans le titre de la monnaie a précisément le même effet que ce que les Français appellent une augmentation des monnaies ou un surhaussement direct de leur dénomination.
Une augmentation des monnaies ou un surhaussement direct de leur dénomination est toujours et ne peut manquer d’être, de sa nature, une opération ouverte et déclarée. Par cette opération, des pièces d’un poids et d’un volume plus petits sont appelées du nom que l’on donnait auparavant à des pièces d’un plus fort poids et d’un plus gros volume. L’altération de titre, au contraire, a été en général une opération cachée. Par cette dernière opération, les hôtels des monnaies mettaient en émission des pièces d’une bien moindre valeur que celles qui avaient eu cours jusqu’alors, mais pourtant de la même dénomination et à peu près semblables, au moins autant qu’on pouvait en venir à bout, quant au poids, au volume et à l’apparence. Quand le roi de France, Jean[21], altéra le titre de ses monnaies pour payer ses dettes, tous les officiers de ses hôtels des monnaies furent obligés, par serment, au secret. Les deux opérations sont injustes ; mais un simple surhaussement est une injustice ouverte et violente, tandis qu’une altération du titre est une fraude et une fourberie. Aussi cette dernière espèce d’opération, du moment qu’elle a été découverte (et elle ne peut pas rester très-longtemps cachée), a toujours excité une indignation beaucoup plus forte que l’autre. Il est très-rare que la monnaie, après avoir subi quelque surhaussement considérable dans sa dénomination, ait jamais été remise sur le pied de son ancien poids ; mais, après les plus fortes altérations dans le titre, elle a été presque toujours rétablie à son ancien degré de fin. C’était le seul moyen qu’on eût d’apaiser la fureur et l’indignation du peuple.
Sur la fin du règne ne de Henri VIII, et dans le commencement de celui Édouard VI, la monnaie d’Angleterre subit non-seulement une hausse dans sa dénomination, mais encore une altération dans son titre. Les mêmes fraudes furent pratiquées en Écosse, sous la minorité de Jacques VI. Elles l’ont été, en certaines circonstances, dans presque tous les autres pays. S’attendre à ce que le revenu public de la Grande-Bretagne puisse jamais être complètement libéré, ou qu’on puisse jamais arriver à faire vers cette libération quelques pas un peu importants, tant que le surplus de ce revenu, ou que l’excédent de ce qui est nécessaire pour couvrir la dépense annuelle de l’établissement de paix sera aussi faible, ce serait, à ce qu’il semble, une espérance tout à fait chimérique. Il est évident qu’on ne saurait se flatter d’atteindre à cette libéra­tion, à moins de quelque augmentation considérable dans le revenu public, ou bien de quelque réduction non moins considérable dans la dépense.
Une taxe foncière répartie avec plus d’égalité, un impôt aussi plus égal sur le loyer des maisons, et des réformes dans le système actuel des douanes et de l’accise, telles que celles proposées dans le chapitre précédent, pourraient peut-être, sans augmenter la charge de la majeure partie du peuple, et seulement en en répartissant le poids d’une manière plus égale sur la totalité, donner lieu à un accroissement considérable du revenu public. Toutefois, il n’y a pas de faiseur de projets, quelque exalté qu’il puisse être dans ses idées, qui ose se flatter qu’avec une augmentation quelconque de ce genre il soit encore possible d’espérer raisonnablement, soit une libération totale du revenu public, soit même un acheminement assez avancé vers cette libération, en temps de paix, pour prévenir ou balancer, dans la guerre suivante, un nouvel accroissement du capital de la dette.
Il y aurait lieu de s’attendre à une plus grande augmentation de revenu si l’on étendait notre système d’imposition à toutes les différentes provinces de l’empire dont les habitants sont d’origine britannique et européenne. C’est pourtant ce qui ne pour­rait peut-être guère se faire d’une manière compatible avec les principes de la consti­tution, sans admettre dans le parlement, ou, si l’on veut, dans les états-généraux de l’empire britannique, une représentation pleine et égale de toutes provinces ; la représentation de chacune d’elles étant, avec le produit de ses impôts, dans la même proportion où serait la représentation de la Grande-Bretagne avec les impôts levés dans la Grande-Bretagne. Il est vrai que l’intérêt privé d’une foule de particuliers puissants, les préjugés enracinés auxquels tiennent les grands corps, paraissent opposer pour le moment, contre une telle innovation, des obstacles extrê­me­ment difficiles, peut-être même tout à fait impossibles à surmonter. Néanmoins, sans prétendre déterminer jusqu’à quel point une telle union serait ou ne serait pas prati­cable, il n’est peut-être pas hors de propos, dans un ouvrage de pure théorie com­me celui-ci, d’examiner à quel degré le système d’imposition de la Grande-Bretagne pourrait s’appliquer à toutes les différentes provinces de l’empire ; quel revenu on pourrait s’en promettre s’il y était appliqué, et de quelle manière il est à présumer qu’une union générale de cette espèce pourrait influer sur le bonheur et la prospérité des différentes provinces qui s’y trouveraient comprises. On pourra, au pis-aller, re­gar­der une pareille spéculation comme une nouvelle utopie moins récréative, à coup sûr, que l’ancienne, mais non pas plus inutile ni plus chimérique[22].
La taxe foncière, les droits de timbre et les différents droits de douane et d’accise constituent les quatre branches principales des contributions de la Grande-Bretagne. 
L’Irlande est assurément aussi en état, et nos colonies d’Amérique et des Indes Occidentales plus en état de payer une taxe foncière que la Grande-Bretagne. Dans des pays où le propriétaire n’est assujetti ni à la dîme ni à la taxe des pauvres, il doit assurément être plus en état de payer cet impôt que dans un pays où il est assujetti à ces deux autres charges. La dîme, dans les endroits où elle n’est pas abonnée et où elle se paie en nature, diminue plus ce qui formerait sans elle le revenu du proprié­taire, que ne le ferait une taxe foncière montant réellement à 5 schellings par livre. On trouvera le plus souvent qu’une telle dîme monte à plus du quart du revenu réel de la terre, ou de ce qui reste après le remplacement entier du capital du fermier, plus un profit raisonnable. Si l’on supprimait tous les abonnements[23] de dîmes et toutes les concessions de dîmes faites à des laïques[24], la dîme ecclésiastique bien complète de la Grande-Bretagne et de l’Irlande ne pourrait guère être évaluée à moins de 6 ou 7 millions. Si donc il n’y avait pas de dîme en Grande-Bretagne et en Irlande, les propriétaires seraient en état de payer 6 ou 7 millions de taxe additionnelle dans la taxe foncière, sans être plus chargés qu’une très-grande partie d’entre eux ne l’est aujourd’hui. L’Amérique ne paie pas de dîme, et serait, par conséquent, très-en état de payer une taxe foncière. Il est vrai qu’en général les terres, en Amérique et dans les Indes Occidentales, ne sont pas amodiées ni données à bail à des fermiers. Elles ne pourraient donc pas être assujetties à l’imposition par des rôles dressés sur le taux de l’amélioration ou du fermage. Cependant, dans la quatrième année de Guillaume et Marie, les terres de la Grande-Bretagne ne furent pas non plus taxées d’après un état des fermages, mais d’après une estimation faite fort au large et sans exactitude. On pourrait taxer les terres en Amérique, ou de la même manière, ou bien d’après une juste évaluation faite en conséquence d’un arpentage exact, tel que celui qui a été dernièrement fait dans le Milanais et dans les États de l’Autriche, de la Prusse et de la Sardaigne.
Quant aux droits de timbre, il est évident que, dans des pays où les formalités de la procédure judiciaire et les actes translatifs d’une propriété soit réelle, soit person­nelle, sont partout les mêmes, ou à peu près les mêmes, ces droits pourraient très-bien être établis, sans la moindre différence, quant à la forme de perception.
Si les lois de la Grande-Bretagne, relatives aux douanes, étaient étendues à l’Irlan­de et aux colonies, pourvu que cette extension fût accompagnée, comme en toute justice elle devrait l’être, d’une extension de la liberté de commerce, elle serait extrê­me­ment avantageuse à ces deux différents pays. On ne verrait plus ces entraves qui accablent aujourd’hui le commerce de l’Irlande, et qui ont été imaginées par une rivalité avide et jalouse ; on ne connaîtrait plus toutes ces distinctions entre les mar­chandises de l’Amérique, énumérées ou non énumérées. Les contrées situées au nord du cap Finistère seraient aussi ouvertes à chaque partie du produit de l’Amé­rique, que le sont aujourd’hui à certaines parties de ce produit les contrées situées au sud de ce cap. Au moyen de cette uniformité dans la législation des douanes, le commerce entre toutes les différentes parties de l’empire britannique se ferait tout aussi bien que celui qui se fait aujourd’hui entre les différentes côtes de la Grande-Bretagne. Cet empire se trouverait ainsi avoir dans son propre sein un immense marché intérieur, pour quelque partie que ce soit du produit de toutes ses diverses provinces. Une si vaste extension de marché indemniserait bientôt et l’Irlande et les colonies de tout ce que pourrait leur coûter l’accroissement des droits de douane.
L’accise est la seule branche de notre système d’imposition qui exigerait certaines modifications selon les diverses provinces de l’empire auxquelles on l’appliquerait. On pourra l’étendre à l’Irlande sans y faire le moindre changement, le produit et la consommation de ce royaume étant précisément de la même nature que ceux de la Grande-Bretagne. À l’égard de son extension à l’Amérique et aux Indes Occidentales, dont le produit et la consommation différent si fort de ceux de la Grande-Bretagne, il y faudrait nécessairement quelques modifications, de la nature de celles qu’on appli­que aux comtés de l’Angleterre qui consomment de la bière et à ceux qui consomment du cidre.
Par exemple, une liqueur fermentée, qui se nomme bière, mais qui se fait avec de la mélasse, et qui a très-peu de rapport avec notre bière, compose en grande partie la boisson commune du peuple en Amérique. Comme cette liqueur ne se garde que quelques jours, on ne peut pas la préparer et l’emmagasiner, pour la vente, dans de vastes brasseries, comme on fait de notre bière ; il faut que chaque ménage la brasse chez soi pour son usage, tout comme il faut qu’il fasse cuire ses aliments. Or, aller assujettir chaque ménage particulier aux visites et aux recherches désagréables des percepteurs de l’impôt, comme on y assujettit nos cabaretiers et nos marchands brasseurs, serait une chose tout à fait incompatible avec la liberté. Si, pour mettre de l’égalité, on jugeait nécessaire d’établir un impôt sur cette boisson, on pourrait l’impo­ser par un droit sur la matière avec laquelle elle se fait, qui serait perçu au lieu où se fabrique cette matière ; ou bien, si la nature du commerce rendait impropre un pareil droit d’accise, on l’imposerait par un droit sur l’importation de cette manière dans la colonie où devrait s’en faire la consommation. Outre le droit de 1 penny par gallon, imposé par le parlement d’Angleterre sur l’importation des mélasses en Amérique, il y a un impôt provincial de cette espèce sur les importations dans la colonie de Massa­chusetts, si elles sont importées dans des vaisseaux appartenant une autre colonie, lequel droit est de 8 deniers par muid ; il y a pareillement un droit de 5 deniers par gallon sur leur importation des colonies du Nord dans la Caroline du Sud. Enfin, si l’on trouvait de l’inconvénient à l’une ou à l’autre de ces méthodes, on pourrait exiger une composition ou abonnement de la part de chaque ménage qui voudrait consom­mer de cette boisson, soit d’après le nombre des personnes qui composeraient le ménage, de la même manière que les ménages particuliers s’abonnent, en Angle­terre, pour la taxe sur la drêche ; ou d’après la différence d’âge et de sexe de ces personnes, comme on le pratique en Hollande, pour la perception de divers impôts, ou bien à peu près comme sir Matthieu Decker propose de lever en Angleterre tous les impôts sur les objets de consommation. Ce mode d’imposition, comme on l’a déjà observé, n’est pas un mode très-convenable lorsqu’on l’applique à des objets d’une prompte consom­mation. On pourrait cependant l’adopter dans les cas où l’on n’en trouverait pas de meilleur.
Le sucre, le rhum et le tabac sont des marchandises qui, n’étant nulle part objets de nécessité, sont néanmoins devenues d’une consommation presque universelle et qui, par conséquent, sont extrêmement propres à être imposées. Si une union avec les colonies avait une fois lieu, alors on pourrait imposer ces denrées avant qu’elles sortissent des mains du manufacturier ou du producteur ; ou bien, si ce mode d’impo­sition ne pouvait s’accommoder avec le commerce de ceux-ci, ces denrées pourraient être déposées dans des magasins publics, tant à l’endroit de la manufacture, qu’à tous les différents ports de l’empire auquel elles seraient transportées par la suite, pour y rester sous la double clef du propriétaire des denrées et de l’officier du fisc, jusqu’au moment où elles sortiraient du magasin et seraient livrées au consommateur ou au marchand en détail pour la consommation du pays, ou enfin à un autre marchand pour l’exportation, l’impôt ne devant être avancé qu’au moment même de cette livraison. Quand elles seraient livrées pour l’exportation, elles sortiraient franches de droits du magasin, moyennant toutefois des assurances suffisantes que réellement les denrées seraient exportées hors de l’empire. Ce sont peut-être là les principales marchandises pour lesquelles il faudrait quelque changement un peu considérable dans le mode actuel d’imposition de la Grande-Bretagne, au cas d’une union avec les colonies.
Quel pourrait être le montant du revenu que produirait ce système d’imposition étendu à toutes les différentes provinces de l’empire ? C’est sans contredit ce dont il est tout à fait impossible de s’assurer avec quelque degré d’exactitude. Au moyen de ce système d’impôts, on perçoit annuellement dans la Grande-Bretagne plus de 10 millions de revenus sur moins de huit millions d’habitants. L’Irlande renferme plus de deux millions d’habitants, et d’après les états mis sous les yeux du congrès, les douze Provinces-Unies de l’Amérique en renferment plus de trois millions. Ces états, cepen­dant, peuvent avoir été exagérés, dans la vue peut-être de donner de la confiance au peuple de ce pays, ou d’intimider celui du nôtre ; et ainsi nous supposerons une population de trois millions seulement dans nos colonies de l’Amérique septentrionale et des Indes occidentales prises ensemble, c’est-à-dire que nous partirons de la supposition que la totalité de l’empire britannique, tant en Europe qu’en Amérique, ne renferme pas plus de treize millions d’habitants. Si sur moins de huit millions d’habi­tants nous levons, avec ce système d’imposition, un revenu de plus de 10 millions sterling, nous devrions sur treize millions d’habitants, avec le même système, lever un revenu de plus de 16,250,000 liv. sterling. De ce revenu, en admettant que ce système pût le produire, il faudrait défalquer le revenu qui se lève habituellement en Irlande et dans les colonies, pour pourvoir respectivement aux dépenses de leurs gouvernements civils. La dépense de l’ établissement civil et militaire de l’Irlande, jointe à l’intérêt de sa dette publique, se monte, d’après un taux moyen pris sur deux années finies en mars 1775, à quelque chose de moins que 750,000 liv. par an[25]. Par un état très-exact du revenu public des principales colonies de l’Amérique et des Indes Occidentales, ce revenu, avant le commencement des troubles actuels, formait un total de 141,800 liv. Cependant, dans cet état, le revenu du Maryland, de la Caroline du Nord et de toutes nos dernières acquisitions, tant sur le continent que dans les îles, se trouve omis ; ce qui fait peut-être une différence de 30 ou 40,000 liv. Ainsi, pour faire un nombre rond, supposons que le revenu nécessaire pour soutenir le gouverne­ment civil de l’Irlande et des colonies se monte à 1 million ; il resterait, par consé­quent, un revenu de 15,250,000 liv. à appliquer à l’acquit de la dépense générale de l’empire et au payement de la dette publique. Or, si l’on peut bien économiser, en temps de paix, sur le revenu actuel de la Grande-Bretagne, 1 million pour le rembour­sement de cette dette, on pourrait très-bien, sur ce revenu ainsi amélioré, économiser 6,250,000 liv. De plus, ce riche fonds d’amortissement pourrait s’augmenter chaque année par l’intérêt de la dette qui aurait été remboursée l’année précédente, et de cette manière il pourrait grossir assez rapidement pour pouvoir suffire à rembourser la totalité de la dette dans un petit nombre d’années, et à rétablir ainsi dans toute leur vigueur les forces affaiblies et languissantes de l’empire. En même temps, le peuple pourrait être soulagé de quelques-uns des impôts les plus onéreux, de ceux qui sont établis sur des objets de nécessité ou sur des matières premières de manufactures. L’ouvrier pauvre pourrait ainsi être mis à même de vivre avec plus d’aisance, de travailler pour un moindre salaire, et d’envoyer ses marchandises au marché à meil­leur compte. Le bon marché de celles-ci en ferait augmenter la demande et, par con­sé­quent, la demande de travail augmenterait pour ceux qui produisent les marchan­dises. Cette augmentation dans la demande de travail accroîtrait à la fois la popula­tion et améliorerait la condition de l’ouvrier pauvre. La consommation de celui-ci augmenterait, et avec elle le revenu provenant de tous ces articles de consommation du pauvre, sur lesquels on aurait laissé subsister les impôts. 
Toutefois, le revenu provenant de ce plan d’imposition n’augmenterait pas tout de suite dans la proportion du nombre des habitants qui y seraient assujettis. Il faudrait, pendant quelque temps, traiter avec une grande indulgence ces provinces de l’empire qui se trouveraient ainsi assujetties à des charges auxquelles elles n’auraient pas été accoutumées auparavant, et même quand on en serait venu à lever partout, aussi exactement que possible, les mêmes impôts, ils ne produiraient pas encore partout un revenu proportionné à la population. Dans un pays pauvre, la consommation des principales marchandises sujettes aux droits de douane et d’accise est fort petite ; et dans un pays faiblement peuplé, il y a bien plus de facilité à frauder les droits. La consommation de boissons faites de drêche est très-faible dans les classes inférieures du peuple en Écosse, et l’accise sur la drêche, la bière et l’ale y rend moins qu’en Angleterre, toute proportion gardée avec la population et avec le taux des droits qui ne sont pas les mêmes sur la drêche, parce qu’on la suppose différente quant à la qualité. Dans ces branches particulières de l’accise, il n’y a pas, à ce que je pense, beaucoup plus de contrebande dans un de ces pays que dans l’autre. Les droits sur les liqueurs distillées et la plus grande partie des droits de douane produisent moins en Écosse qu’en Angleterre, à proportion de la population respective de chacun de ces pays, et cela non-seulement à cause d’une moindre consommation des denrées sujettes à l’impôt, mais encore à cause de la facilité beaucoup plus grande de frauder les droits. En Irlande, les classes inférieures du peuple sont encore plus pauvres qu’en Écosse, et il y a une quantité d’endroits dans le pays qui y sont aussi mal peuplés. Ainsi, en Irlande, la consommation des denrées sujettes à l’impôt pourrait, à propor­tion de la population, être moindre encore qu’en Écosse, et la facilité de frauder à peu près la même. Dans l’Amérique et dans les Indes Occidentales, les Blancs, même de la dernière classe, sont beaucoup plus à leur aise que ceux de la même classe en Angleterre, et ils font probablement une bien plus grosse consommation de toutes les choses de luxe dont ils ont l’habitude de ne point se passer. À la vérité, les Noirs, qui composent la plus grande partie de la population, tant des colonies méridionales du continent que de nos îles des Indes Occidentales, étant dans un état d’esclavage, sont sans contredit dans une condition bien pire que les gens les plus pauvres de l’Écosse et de l’Irlande. Il ne faut pourtant pas nous imaginer pour cela qu’ils soient plus mal nourris, et que la consommation qu’ils font des articles qu’on pourrait assujettir à des impôts modérés soit moindre que celle même des dernières classes du peuple d’Angleterre. C’est l’intérêt de leur maître de les bien nourrir et de les tenir toujours bien portants et bien dispos, afin qu’ils puissent bien travailler, tout comme c’est son intérêt de traiter ainsi le bétail qui travaille pour lui. Aussi ont-ils presque partout leur ration de rhum et de mélasse ou de bière[26], tout comme les domestiques blancs ; et vraisemblablement on ne leur retrancherait pas cette ration, quand même ces articles seraient assujettis à des impôts modérés. Ainsi la consommation des denrées assu­jetties à l’impôt serait Probablement, à proportion de la population, aussi forte en Amérique et dans les Indes Occidentales que dans toute autre partie de l’empire britannique. À la vérité, la facilité de frauder serait beaucoup plus grande, l’Amérique étant beaucoup plus faiblement peuplée, à proportion de l’étendue du territoire, que ne le sont l’Écosse et l’Irlande. Néanmoins, si le revenu qu’on retire actuellement des différents droits sur la drêche et les liqueurs et boissons de drêche venait à être produit par un droit unique sur la drêche, on ôterait absolument tous les moyens qu’il y a de frauder les droits dans la branche la plus importante de l’accise ; et si les droits de douane, au lieu d’être imposés sur presque tous les différents articles d’importation, étaient bornés à un petit nombre d’articles d’un usage et d’une consommation plus générale, que d’ailleurs la perception de ces droits se fit suivant les lois de l’accise, alors, si les moyens de frauder n’étaient pas entièrement ôtés, ils seraient extrême­ment diminués. En conséquence de ces deux réformes, qui paraissent fort simples et très-faciles, les droits de douane et d’accise rendraient vraisemblablement autant de revenu, à proportion de la consommation, dans les provinces les plus mal peuplées, qu’ils en rendent actuellement, à proportion de la consommation, dans les provinces les plus peuplées.
On a objecté, il est vrai, que les Américains n’avaient point de monnaie d’or et d’argent, le commerce intérieur du pays roulant sur un papier qui a cours de monnaie, et tout l’or et l’argent qui peuvent leur survenir étant toujours envoyés dans la Grande-Bretagne en retour des marchandises qu’ils reçoivent de nous. Or, sans or et sans argent, ajoute-t-on, il n’y a pas de possibilité de payer d’impôt. Nous leur avons déjà enlevé tout l’or et l’argent qu’ils avaient, comment est-il possible de tirer d’eux ce qu’ils n’ont pas ?
La disette actuelle de monnaie d’or et d’argent en Amérique ne provient pas de la pauvreté du pays ou du défaut de moyens pour ses habitants de se procurer de ces métaux. Dans un pays où les salaires du travail sont si fort au-dessus du prix de ceux de l’Angleterre, et le prix des vivres si fort au-dessous, assurément la majeure partie des gens y doivent avoir de quoi y acheter une plus grande quantité de ces métaux, s’il leur était nécessaire ou avantageux de le faire. La rareté de ces métaux y est donc une affaire de choix et non de nécessité.
Ce ne peut être que pour des transactions domestiques ou étrangères que la mon­naie d’or et d’argent est nécessaire ou avantageuse.
On a fait voir, dans le IIe livre de ces Recherches[27], que les affaires intérieures d’un pays quelconque, au moins dans les temps de tranquillité, pouvaient marcher à l’aide d’un papier ayant cours de monnaie, avec à peu près autant d’avantage que si l’on employait de la monnaie d’or et d’argent. Pour les Américains, qui sont toujours dans le cas d’employer avec profit à l’amélioration de leurs terres de plus grands capitaux que tous ceux qu’il leur est possible de se procurer aisément, c’est un avantage que d’épargner, autant qu’il se peut, la dépense d’un instrument de commerce aussi dispendieux que l’or et l’argent, et de consacrer cette partie de leur produit superflu qu’absorberait l’achat de ces métaux à acheter bien plutôt les instruments de métier, les matières pour vêtements, les ustensiles de ménage, les ouvrages en fer, et enfin tout ce qui leur est nécessaire pour former leurs établissements et étendre leurs plan­ta­tions, à acquérir un fonds actif et productif, plutôt qu’un fonds mort et stérile. Chaque gouvernement colonial trouve son intérêt à fournir au peuple du papier-monnaie en une quantité largement suffisante, et même en général plus que suffisante pour faire aller toutes les affaires intérieures. Quelques-uns de ces gouvernements, celui de Pennsylvanie en particulier, se font un revenu en prêtant ce papier-monnaie à leurs sujets, à un intérêt de tant pour 100. D’autres, comme celui de Massachusetts, avancent un papier-monnaie de ce genre dans les besoins extraordinaires de l’État, pour subvenir aux dépenses publiques ; et ensuite, quand la colonie se trouve en commodité de le faire, ils le rachètent au bas prix auquel il tombe par degrés. En 1747[28], cette colonie paya ainsi la majeure partie de ses dettes avec le dixième de la valeur pour laquelle elle avait d’abord donné ses billets. Il convient extrêmement aux colons d’épargner les dépenses que leur occasionnerait l’usage de la monnaie d’or et d’argent dans leurs affaires intérieures, et il convient tout autant au gouvernement colonial de leur fournir une valeur intermédiaire qui, bien qu’accompagnée de quel­ques inconvénients assez graves, les met à même d’éviter cette dépense. L’extrê­me abondance de papier-monnaie chasse l’or et l’argent de toutes les transactions intérieures dans les colonies, par la même raison qui lui fait chasser ces métaux de la plus grande partie des transactions intérieures en Écosse ; et ce qui a occasionné, dans un pays comme dans l’autre, cette grande abondance de papier-monnaie, ce n’est pas la pauvreté du pays, mais l’esprit actif et entreprenant du peuple, et le désir qu’il a d’employer, comme capital utile et productif, tous les fonds qu’il peut venir à bout de se procurer.
Dans le commerce extérieur que les différentes colonies font avec la Grande-Bretagne, l’or et l’argent se trouvent plus ou moins employés, précisément à propor­tion qu’ils y sont plus ou moins nécessaires. Quand ces métaux n’y sont pas nécessaires, il est bien rare qu’on les y voie. Quand ils y sont nécessaires, en général, ils ne manquent pas.
Dans le commerce d’entre la Grande-Bretagne et les colonies à tabac, pour l’ordinaire les marchandises de la Grande-Bretagne sont avancées aux colons à un crédit assez long, et elles sont ensuite acquittées en tabac qui se compte à un prix convenu. Il est plus commode pour les colons de payer en tabac que de payer en or et en argent. Un marchand trouvera toujours plus avantageux pour lui de payer les marchandises que lui vendent ses correspondants, en quelque autre espèce de marchandise dont il fait commerce, que de les payer en argent. Alors ce marchand n’aura pas besoin de garder par devers lui une partie de son capital sans emploi et en argent comptant, pour satisfaire aux traites qui lui seraient présentées. Il pourra avoir en tout temps, dans sa boutique ou dans son magasin, une plus grande quantité de marchandises et, en conséquence, donner une plus grande étendue à son commerce. Mais il arrive rarement qu’il soit commode pour tous les correspondants d’un mar­chand de recevoir le payement de tous les objets qu’ils lui vendent, en marchandises de quelque autre espèce que celle dont celui-ci fait commerce. Les marchands anglais qui font des affaires avec le Maryland et la Virginie se trouvent être une classe particulière de correspondants, pour lesquels il est plus commode de recevoir en tabac, qu’en or et argent, le payement des marchandises qu’ils font passer à ces colonies. Ils ont l’expectative d’un profit sur la vente du tabac ; ils n’en auraient aucun à faire sur l’or et l’argent. Ainsi, l’or et l’argent se montrent très-rarement dans le commerce entre la Grande-Bretagne et les colonies à tabac. Le Maryland et la Virginie ont tout aussi peu besoin de ces métaux pour le commerce étranger que pour leur commerce intérieur. Aussi dit-on que de toutes les colonies américaines, ce sont celles qui ont le moins de monnaie d’or et d’argent. Elles n’en passent pas moins cependant pour être tout aussi florissantes et, par conséquent, tout aussi riches qu’au­cun autre des États voisins.
Quant aux colonies du Nord, la Pennsylvanie, New-York, New-jersey, les quatre gouvernements de la Nouvelle-Angleterre, etc., la valeur de ce qu’elles exportent de leur propre produit à la Grande-Bretagne ne fait pas l’équivalent de ce qu’elles en importent en ouvrages de manufacture, tant pour leur propre usage, que pour celui de quelques-unes des autres colonies avec lesquelles elles en font le commerce de transport. Il y a donc nécessairement une balance qu’il faut payer en or et en argent à la mère patrie ; et cette balance, en général, elles savent bien la trouver.
Il en est autrement pour les colonies à sucre. La valeur du produit qu’elles exportent annuellement à la Grande-Bretagne est de beaucoup supérieure à celle de toutes les marchandises qu’elles en importent. Si le sucre et le rhum qui s’envoient annuellement à la métropole étaient acquittés dans les colonies mêmes, la Grande-Bretagne se trouverait obligée d’y faire passer chaque année une très-forte balance en argent, et notre commerce avec les Indes Occidentales serait regardé par une certaine classe de politiques comme un commerce extrêmement désavantageux ; mais il se trouve qu’une quantité des principaux propriétaires des habitations à sucre font leur résidence dans la Grande-Bretagne. La remise de leurs revenus leur est faite en sucre et en rhum, qui sont les productions de leurs biens-fonds. Le sucre et le rhum qu’achètent dans ces colonies, pour leur compte particulier, nos marchands qui font le commerce des Indes Occidentales, n’égalent pas en valeur les marchandises qu’ils y font passer annuellement. Il y a donc une balance à leur payer en or et en argent ; et cette balance aussi, en général, ceux qui la doivent savent bien la trouver.
La difficulté et l’irrégularité que les différentes colonies ont pu faire voir dans leurs payements à l’égard de la Grande-Bretagne, n’ont été nullement dans la propor­tion de la balance plus ou moins forte qu’elles se trouvaient devoir respectivement. Pour l’ordinaire, les payements se sont faits avec plus de régularité par les colonies du Nord que par les colonies à tabac, quoique les premières aient généralement payé une assez forte balance en argent, tandis que les dernières ou n’en ont point eu à payer, ou en ont dû une beaucoup plus faible. La difficulté de se faire payer par nos différentes colonies à sucre a été plus ou moins grande, non pas tant à proportion de la balance plus ou moins forte qu’elles se trouvaient devoir respectivement, qu’à proportion de la quantité de terres incultes qu’elles renfermaient, c’est-à-dire à raison de la tentation plus ou moins vive qu’ont éprouvée les colons d’étendre leurs affaires au-delà de leurs forces, ou d’entreprendre la mise en valeur et la culture d’une plus grande quantité de terres incultes que ne le comportait l’étendue de leurs capitaux. Les retours de la grande île de la Jamaïque, où il y a encore beaucoup de terres incultes, se sont faits par cette raison avec beaucoup moins de régularité, et ont été, en général, moins assu­rés que ceux des petites îles des Barbades, d’Antigoa et de Saint-Christophe, qui sont complètement en culture depuis maintes années, et qui dès lors donnent bien moins matière aux spéculations des planteurs. Les nouvelles acquisitions de la Grenade, de Tabajo, de Saint-Vincent et de la Dominique ont ouvert un nouveau champ à ces spéculations, et les retours de ces îles ont été depuis peu aussi certains et aussi réguliers que ceux de la grande île de la Jamaïque.
Ce n’est donc pas la pauvreté des colonies qui occasionne, dans la plupart d’entre elles, la disette de monnaie d’or et d’argent. La grande demande qui s’y fait de fonds actifs et productifs leur fait trouver de l’avantage à avoir aussi peu que possible de fonds morts et stériles et les porte, en conséquence à se contenter d’un instrument de commerce moins commode, à la vérité, mais aussi bien moins cher que l’or et l’ar­gent. Elles se mettent ainsi en état de convertir la valeur de cet or et de cet argent en instruments de métier, en matières pour vêtements, en meubles et ustensiles de ménage, en ferrures, en tout ce qui leur est nécessaire enfin pour former leurs établis­sements et étendre leurs plantations. Il parait que dans les branches de leurs affaires qui ne peuvent se terminer sans monnaie d’or ou d’argent, elles ont toujours le moyen de trouver la quantité de ces métaux qui leur est nécessaire, et s’il leur arrive souvent de ne la pas trouver, ce n’est pas à l’impuissance forcée de la pauvreté qu’il faut, en général, imputer leur défaut d’exactitude, mais bien à l’impuissance très-volontaire qui résulte de leurs entreprises immodérées. Ce n’est pas parce qu’elles sont pauvres que leurs payements sont incertains et irréguliers, mais c’est parce qu’elles sont trop tourmentées du désir de devenir bien vite extrêmement riches. Quand même toute cette partie du produit des impôts des colonies, qui se trouverait excéder la dépense nécessaire de leurs établissements civils et militaires, devrait être remise en Grande-Bretagne en or et en argent, les colonies ont largement de quoi acheter toute la quantité de ces métaux qu’il leur faudrait pour cela. À la vérité, dans ce cas, elles se verraient obligées d’échanger contre un fonds mort et stérile une partie de leur produit superflu qui maintenant leur sert à acheter des capitaux actifs et productifs. Dans leurs affaires et transactions intérieures, elles seraient obligées de faire usage d’un instrument de commerce dispendieux, au lieu d’un qui ne leur coûtait presque rien, et la dépense d’achat de cet instrument dispendieux pourrait ralentir un peu l’extrême activité de leurs vastes entreprises en défrichements et en améliorations. Il se pourrait bien pourtant qu’il ne fût pas nécessaire de faire en or et argent la remise d’aucune partie du revenu des impôts américains. Cet envoi pourrait se faire en lettres de change tirées sur des négociants particuliers ou des compagnies de commerce de la Grande-Bretagne, et acceptées par eux, auxquels négociants ou compagnies une partie du produit superflu de l’Amérique aurait été envoyée d’avance, et qui verse­raient en argent dans le trésor public le montant du revenu des impôts américains, après qu’ils en auraient eux-mêmes reçu la valeur en marchandises ; le plus souvent, toute l’opération pourrait se consommer sans exporter de l’Amérique une seule once d’or ou d’argent.
Il n’est pas contre la justice que l’Irlande et l’Amérique contribuent à la dette publique de la Grande-Bretagne. Cette dette a été contractée pour soutenir le gouver­ne­ment établi par la révolution, gouvernement auquel les protestants d’Irlande[29] sont redevables non-seulement de toute l’autorité dont ils jouissent actuellement dans leur pays, mais même de tout ce qui leur garantit leur liberté, leur propriété et leur reli­gion ; gouvernement duquel plusieurs des colonies de l’Amérique tiennent leurs char­tes actuelles et, par conséquent, leur présente constitution ; auquel, enfin, toutes ces colonies en général doivent la liberté, la sûreté et la propriété dont elles ont toujours joui jusqu’à présent. Cette dette a été contractée pour la défense, non pas de la seule Grande-Bretagne, mais de toutes les parties de l’empire. La dette immense de la guer­re dernière en particulier, et une grande partie de celle de la guerre qui avait pré­cé­dé ont été, l’une et l’autre, contractées spécialement pour la défense de l’Amérique.
Outre la liberté de commerce, l’Irlande gagnerait à une union avec la Grande-Bretagne d’autres avantages beaucoup plus importants, et qui feraient bien plus que compenser toute augmentation d’impôts que cette union pourrait amener avec elle. Par l’union avec l’Angleterre, les classes moyennes et inférieures du peuple en Écosse ont gagné de se voir totalement délivrées du joug d’une aristocratie qui les avait toujours auparavant tenues dans l’oppression. Par l’union avec la Grande-Bretagne, la majeure partie du peuple de toutes les classes en Irlande aurait également l’avantage de se voir délivrée d’une aristocratie beaucoup plus oppressive ; d’une aristocratie qui n’est pas, comme en Écosse, fondée sur les distinctions naturelles et respectables de la naissance et de la fortune, mais qui porte sur les plus odieuses de toutes les distinc­tions qui, plus que toute autre, excitent à la fois l’insolence des oppresseurs et allu­ment la haine et l’indignation des opprimés ; qui rendent enfin, pour l’ordinaire, les habitants d’un même pays ennemis plus acharnés les uns des autres que ne le furent jamais des hommes de pays différents. À moins d’une union avec la Grande-Bretagne, il n’y a pas à présumer que, de plusieurs siècles encore, les habitants de l’Irlan­de puissent se regarder comme ne formant qu’un peuple[30].
Aucune aristocratie oppressive ne s’est encore fait sentir dans les colonies. Toute­fois, elles n’en auraient pas moins elles-mêmes à gagner considérablement, sous le rapport du bonheur et de la tranquillité, à une union avec la Grande-Bretagne. Au moins cette union les délivrerait-elle de ces factions haineuses et emportées, toujours inséparables des petites démocraties ; factions qui, dans ces États dont la constitution se rapproche tant de la forme démocratique, ont trop souvent fait naître des divisions parmi le peuple et troublé la tranquillité de leurs divers gouvernements. En cas d’une séparation totale d’avec la Grande-Bretagne, événement qui paraît très-probable, si on ne le prévient par une union de ce genre, ces factions vont devenir dix fois plus enve­ni­mées que jamais. Avant le commencement des troubles actuels, le pouvoir coercitif de la métropole a suffi pour contenir ces factions dans certaines bornes et les empê­cher d’aller au-delà de quelques provocations et insultes grossières. Si ce pouvoir répressif était une fois totalement écarté, elles éclateraient bientôt probablement en violences ouvertes et en scènes sanglantes. Dans tous les grands pays qui sont unis sous un gouvernement uniforme, les provinces éloignées sont bien moins exposées à l’influence de l’esprit de parti que ne l’est le centre de l’empire. La distance où ces provinces sont de la capitale, du siège principal où se passent les grandes luttes de l’ambition et des factions, fait qu’elles entrent moins dans les vues d’aucun des partis opposés, et qu’elles demeurent, entre eux tous, spectatrices impartiales et indiffé­ren­tes. L’esprit de parti domine moins en Écosse qu’en Angleterre. Dans le cas d’une union, il dominerait moins probablement encore en Irlande qu’en Écosse, et les colo­nies en viendraient bientôt, selon toute apparence, à jouir d’un degré de concorde et d’unanimité inconnu jusqu’à présent dans toute partie quelconque de l’empire britan­nique. À la vérité, l’Irlande et les colonies se trouveraient assujetties à des impôts plus lourds qu’aucun de ceux qu’elles payent aujourd’hui. Néanmoins, une application soigneuse et fidèle du revenu public à l’acquit de la dette nationale ferait que la ma­jeure partie de ces impôts ne serait pas de longue durée, et que les dépenses de la Grande-Bretagne pourraient être bientôt réduites à la somme simplement nécessaire pour maintenir un établissement de paix modéré.
Une autre source de revenu plus abondante encore que toutes celles dont je viens de parler s’offre peut-être dans les acquisitions territoriales de la compagnie des Indes Orientales, qui forment un droit incontestable de la couronne, c’est-à-dire de l’État et du peuple de la Grande-Bretagne. On représente ces contrées comme plus fertiles, plus étendues que la Grande-Bretagne, et comme beaucoup plus riches et plus peu­plées, à proportion de leur étendue. Pour en tirer un grand revenu, il ne serait vrai­sem­blablement pas nécessaire d’introduire aucun nouveau système d’imposition dans des pays qui sont déjà suffisamment et plus que suffisamment Imposés. Il serait peut-être plus à propos même d’alléger plutôt que d’aggraver le fardeau que portent ces infortunées provinces, et de chercher à en tirer un revenu, non pas en les chargeant de nouveaux impôts, mais en prévenant seulement les désordres et les dilapidations qui absorbent la majeure partie de ceux qui y sont déjà établis.
Enfin, si de tous les moyens que j’ai successivement indiqués pour procurer à la Grande-Bretagne une augmentation un peu considérable de revenu, aucun n’était re­con­nu praticable, alors l’unique ressource qui pourrait lui rester, ce serait une dimi­nution de sa dépense. Quant au mode de perception et à celui de faire la dépense du revenu public, quoiqu’ils puissent se perfectionner l’un et l’autre, cependant sur ce point la Grande-Bretagne paraît apporter au moins autant d’économie que qui que ce soit de ses voisins. L’établissement militaire qu’elle entretient pour sa défense même en temps de paix est plus modéré que celui de tout autre État de l’Europe, qui puisse prétendre à rivaliser avec elle en richesse et en puissance. Ainsi, aucun de ces articles ne paraît être susceptible d’une réduction considérable. La dépense de l’établissement de paix des colonies était très-forte avant le commencement des troubles actuels ; or, c’est une dépense qui peut bien être écono­misée, et qui certainement devrait l’être en entier, si l’on ne peut tirer d’elles aucun revenu. Cette dépense permanente en temps de paix, quoique très-forte, n’est encore rien en comparaison de ce que nous a coûté, en temps de guerre, la défense des colonies. La dernière guerre, qui fut uniquement entreprise à cause d’elles, coûte à la Grande-Bretagne, comme on l’a déjà observé, au-delà de 90 millions. La guerre d’Espagne, de 1739, a été principalement entreprise pour elles ; et dans cette guerre, ainsi que dans la guerre de France qui en a été la suite, la Grande-Bretagne a dépensé plus de 40 millions, dont une grande partie devrait, avec justice, être mise sur le comp­te des colonies. Les colonies ont coûté à la Grande-Bretagne, dans ces deux guerres, bien plus du double de la somme à laquelle se montait la dette nationale avant le commencement de la première. Si nous n’eussions pas eu ces guerres, cette dette eût pu être et aurait été probablement remboursée en entier depuis ce temps ; et si nous n’eussions pas eu les colonies, la première de ces guerres n’eût peut-être pas été entreprise, et à coup sûr la dernière ne l’eût pas été. C’est parce que les colonies étaient censées provinces de l’empire britannique, qu’on a fait pour elles toute cette dépense. Mais des pays qui ne contribuent au sou­tien de l’empire ni par un revenu ni par des forces militaires, peuvent-ils être regardés comme des provinces ? Ce sont tout au plus des dépendances accessoires, une espèce de cortège que l’empire traîne à sa suite pour la magnificence et la parade. Or, si l’empire n’est pas en état de soutenir plus longtemps la dépense de traîner avec lui ce cortège, il doit certainement le réformer ; et s’il ne peut pas élever son revenu à pro­por­tion de sa dépense, il faut au moins qu’il accommode sa dépense à son revenu. Si, malgré leur refus de se soumettre aux impôts de l’empire britannique, il faut tou­jours regarder les colonies comme provinces de cet empire, leur défense peut causer à la Grande-Bretagne, dans quelque guerre future, une aussi forte dépense qu’elle en ait jamais causé dans aucune guerre précédente. Il y a déjà plus d’un siècle révolu que ceux qui dirigent la Grande-Bretagne ont amusé le peuple de l’idée imaginaire qu’il possède un grand empire sur la côte occi­den­tale de la mer Atlantique. Cet empire, cependant, n’a encore existé qu’en imagina­tion seulement. jusqu’à présent, ce n’a pas été un empire ; ce n’a pas été une mine d’or, mais le projet d’une mine d’or ; projet qui a coûté des dépenses énormes, qui continue à en coûter encore, et qui nous menace d’en coûter de semblables à l’avenir, s’il est suivi de la même manière qu’il l’a été jusqu’à présent, et cela sans qu’il promette de nous rapporter aucun profit ; car, ainsi qu’on l’a déjà fait voir, les effets du commerce des colonies sont une véritable perte au lieu d’être un profit pour le corps de la nation. Certes, il est bien temps aujourd’hui qu’enfin ceux qui nous gouvernent ou réalisent ce beau rêve d’or dont ils se sont bercés eux-mêmes peut-être, aussi bien qu’ils en ont bercé le peuple, ou bien qu’ils finissent par faire cesser, et pour eux et pour le peuple, un songe qui n’a que trop duré. Si le projet ne peut pas être mené à sa fin, il faut bien se résoudre à l’abandonner. S’il y a quelques provinces de l’empire britannique qu’on ne puisse faire contribuer au soutien de l’ensemble de l’empire, il est assurément bien temps que la Grande-Bretagne s’affranchisse de la charge de les défendre en temps de guerre et d’entretenir, en temps de paix, une partie quelconque de leur établissement civil et militaire. Il est bien temps qu’enfin elle s’arrange pour accommoder doré­na­vant ses vues et ses desseins à la médiocrité réelle de sa fortune.
 
 
 
↑ Voyez tome 1, page 512.
↑ Si dans l’examen de la question de savoir s’il faut, pour pourvoir aux besoins d’une circonstance, appliquer le système des emprunts, ou chercher dans l’augmentation des taxes les ressources nécessaires, la facilité de se procurer de l’argent était le seul point à examiner, il ne serait pas douteux que la préférence dût être donnée au système des emprunts. La régularité avec laquelle se fait le payement des intérêts stipulés par le gouvernement, la facilité des transactions, l’espoir enfin de profiter des fluctuations de la rente, toutes ces causes réunies déterminent bon nombre de capitalistes à donner leur argent au gouvernement de préférence aux particuliers. Le gouvernement obtient ainsi des ressources considérables, et sans grandes difficultés. D’un autre côté, le public s’accommode fort bien d’un pareil système ; au lieu d’avancer des sommes considérables moyennant de fortes taxes, il ne paye que l’intérêt de ces sommes. Un fardeau pareil, qui n’écrase personne, qui ne rend nécessaire aucune réduction dans les dépenses, est généralement supporté sans murmures. Un pareil système, pour pourvoir aux besoins d’une guerre, fait presque oublier ses charges et ses privations, et nous ne sommes plus étonnés que le gouvernement ait adopte un système qui, tout en lui donnant ce dont il a besoin, ne cesse pas d’être populaire. Mais la valeur du système des emprunts ne doit pas se déterminer par la seule considération de la facilité avec laquelle il s’opère. Cette circonstance est loin d’être indifférente, mais il y en a d’autres qui ont une plus grande importance encore. Ce n’est pas seulement d’après d’amortissement établi en 1716 prétendirent, ainsi qu’avec plus d’assurance encore M. Pitt et M. Price, créateurs du fonds d’amortissement de 1786, que si une certaine portion du revenu était appliquée aux achats de rentes, et que si les dividendes de ces rentes étaient employés de la même manière, le fonds d’amortissement, agissant à intérêt composé, suffirait pour éteindre la dette la plus considérable sans effort. Le docteur Price, pour montrer clairement l’application de son principe, calcule le nombre des globes d’or que formerait maintenant un denier placé à intérêt composé à la naissance de Jésus-Christ. Mais bien qu’un calcul de ce genre soit vrai eu théorie, il n’en est pas moins faux cl absurde en pratique. Le fait est qu’un fonds d’amortissement, se composerait-il même du surplus net des revenus, n’opérera jamais à intérêt composé. Il est vrai qu’en appliquant toujours la même portion du surplus des revenus ainsi que les dividendes provenant des achats à l’acquisition de rentes, la réduction s’effectuera de la même manière que si le surplus des revenus, par sa propre énergie (by an inhérent energy of its own), opérait à intérêt composé ; mais il est important de. savoir que, malgré la ressemblance de leur mode d’action (modus operandi), ces moyens diffèrent radicalement et totalement. La dette est réduite quand une portion du produit des impôts est systématiquement affectée aux payements, et elle ne s’éteindra jamais d’une autre manière. Pour augmenter un capital à intérêt composé, il faut l’employer dans une industrie productive ; les profits, au lieu d’être consommés, doivent être régulièrement ajoutés au principal, pour former ainsi un nouveau capital. Il est inutile de dire qu’un pareil fonds d’amortissement n’a jamais existé. Ceux qui ont été créés en Angleterre ou ailleurs ont été alimentés soit par des emprunts, soit par le produit des impôts, et n’ont jamais, par leur action propre, payé un denier de la dette.
D’après ce que nous venons de dire, il est évident que, là où il n’y a pas de surplus des revenus, un fonds d’amortissement ne saurait exister. M. Price, pourtant, n’hésita pas a développer très-longuement que vouloir supprimer le fonds d’amortissement pendant la guerre, époque où les dépenses excédaient les revenus de beaucoup, serait une folie. Quelque incroyable que cela puisse paraître maintenant, tous les partis du Parlement s’associèrent alors à cette mesure, et on approuva que le fonds d’amortissement fût maintenu pendant toute la durée de la guerre. Les emprunts pour le service de l’armée s’augmentèrent ainsi de tout le montant des sommes mises à la disposition des administrateurs du fonds d’amortissement ; de façon que, pour chaque schilling employé de la sorte, il fallait contracter un emprunt d’un montant égal, sans compter les frais d’administration. Cette jonglerie dura à peu près vingt ans ; le Parlement et la nation étaient convaincus, en dépit des expériences les plus décisives, que la dette publique diminuerait de cette manière. Ce fut le docteur Hamilton d’Aberdeen qui le premier dissipa ces illusions, les plus grossières assurément dont jamais peuple se soit bercé. Il montra dans son ouvrage De la délie nationale, publié en 1813, que le fonds d’amortissement, loin de diminuer la dette, l’avait plutôt accrue ; il prouva jusqu’à la dernière évidence que l’excédant des revenus sur les dépenses était le seul fonds d’amortissement qui pût opérer l’extinction de la dette. « L’augmentation des revenus, dit-il, ou la diminution de la dépense sont les seuls moyens qui puissent former un fonds d’amortissement et rendre ses opérations efficaces, et tous les autres projets pour la réduction de la dette nationale, tels que fonds d’amortissement opérant à intérêt composé et autres, s’ils ne sont pas basés sur ce principe, sont complètement illusoires. » La perte que cette rouerie a causée au pays, pendant la dernière guerre, a été évaluée, d’après des documents exacts, à 6,000,000 liv. sterl. (150,000,000 fr.). À la tin tout le monde comprit la folie d’un procédé qui empruntait pour payer. Le fonds d’amortissement fut diminué après la guerre. En 1810 on essaya de créer un fonds d’amortissement réel s’élevant à 3,000,000 liv. sterl. (125,000,000 fr.), on voulut par conséquent maintenir un excédant des revenus sur les dépenses ; mais on n’y réussit point, et après plusieurs modifications, le système entier fut abandonné en 1829, par l’acte 10 de Georges IV, portant que les sommes applicables à la réduction de la dette nationale seraient à l’avenir prises dans l’excédant, s’il y en a, du total des revenus sur le total des dépenses du royaume. Mac Culloch.
↑ Voyez l’Examen des réflexions politiques sur les finances. (Note de l’auteur.) Voyez tome I, page 385, note 1.
↑ Voyez tome I, page 386, note 1.
↑ Du mot anglais to fund on a fait aussi en français fonder, c’est-à-dire le fonds destiné à servir une dette annuelle.
↑ La taxe annuelle ou ancienne taxe seulement : elle est votée pour 750,000 liv., et ne monte jamais à ce produit.
↑ Quoique Dufresne Saint-lion ne soit pas au nombre des commentateurs d’Adam Smith, il a semblé qu’il pouvait y avoir quelque intérêt à rapprocher des considérations diverses que ce chapitre renferme sur la question de l’amortissement, les idées émises sur le mémo sujet par un écrivain d’un mérite incontestable :
« Entre particuliers, le capital d’une dette, au moment où elle est contractée, est déterminé comme l’intérêt. L’emprunteur reçoit une somme fixe, et il pourra s’acquitter en remboursant cette même somme. Il n’en est pas ainsi dans le système actuel du prédit public. La dénomination de cinq pour cent donnée aux litres des dettes que l’on a contractées ou que l’on contracte est fausse, et seulement une fiction quant au capital et quant à la proportion des intérêts avec le capital ; un exemple rendra cette observation sensible.
L’un en présence de l’autre, le gouvernement dit au préteur : « Prêtez-moi 100 millions, je vous donnerai des rentes à 5 pour cent. » Le prêteur répond : « Je peux acheter à la bourse 5 millions de rentes que vous devez déjà, en ne remboursant que 64 millions, et acheter par conséquent 7,500.000 fr. de rente avec les 100 millions que vous me demandez : donnez-moi cette dernière quantité de rente, et je vous donne mon capital. » Et le gouvernement y consent et se reconnaît débiteur de plus qu’il ne reçoit. (On conçoit que ces rapports de somme entre les capitaux et les intérêts ne sont pas absolus, et qu’au contraire ils sont variables, puisque c’est le cours de la bourse qui les détermine à l’époque de l’emprunt.) Cette fiction, cette espèce de mensonge habituellement adopté dans les emprunts récents du gouvernement, n’est pas une dénomination purement arithmétique, innocente comme celle de la monnaie de compte. Il en résulte une surcharge réelle de dettes en capital ; que par le rachat ou le remboursement réel le gouvernement fendra 100,000 fr. au lieu de 61,000 pour rembourser et éteindre sa dette prétendue à 5 pour 100, quoiqu’elle fût en effet à 7 1/2 ; et qu’ainsi, outre les intérêts excessifs, il lui en coûtera 36 pour 100 de capital de plus. Lors de la discussion d’un emprunt de cette espèce fait en 1817 par le gouvernement français, le premier et le plus instruit de nos négociants français, appelé aux conférences préliminaires chez le ministre, duc de Richelieu, voulait que le taux de 7 1/2 pour 100 fût avoué par la France, et que par conséquent le capital ne subsistât que pour ce qu’il était en effet, 64. Mais le banquier anglais Barring sentit qu’alors la France pourrait un jour se libérer en rendant ce capital ; il exigea la fiction nominale de 100 de capital pour chaque 5 fr. de rente, et la fusion des rentes créées par le nouvel emprunt dans la masse de toute la dette, afin que le gouvernement ne pût pas, dans un moment de prospérité, la distinguer et l’isoler pour en réduire l’intérêt par le choix forcé entre le remboursement et la réduction. Le ministre eut la faiblesse de consentir cette énonciation de 3 pour 100, et, pour un capital vrai de 64 fr., se reconnut débiteur d’un capital nominal de 100 fr., et aujourd’hui la caisse d’amortissement le rachète à ce prix. L’État perd 36 pour 100 sur le capital qu’il débourse, comparativement à celui qu’il a reçu ; et cependant, grâce à cette fausse qualification de 5 pour 100, le prix de cette rente ne paraît qu’au pair idéal, tandis qu’il est de 36 sur chaque 100 au-dessus du pair vrai. Il résulte de là que le capital vrai de chaque portion de la dette publique, qui, par cette fusion, compose aujourd’hui l’ensemble à 3 pour 100 de la dette française composée de plusieurs emprunts successifs, est inconnu en France. Le gouvernement à la rigueur pourrait connaître le montant, au vrai, des sommes qu’il a empruntées, en analysant les divers prix auxquels il a fait ces emprunts successifs ; mais il ne pourrait pas se servir de cette connaissance vis-à-vis de ses créanciers, parce qu’il ne pourrait pas les classer ni même les reconnaître, la circulation ayant fait changer ces renies de main, et parce qu’à chaque nouvel emprunt le gouvernement a bien contracté une dette nouvelle, mais non pas une dette dont les titres aient reçu un caractère nouveau et distinctif. Ce qu’il a donné au prêteur, c’est une inscription sur la liste de ses créanciers (qu’on appelle grand-livre), absolument la même, et qui l’a confondu avec les créanciers antérieurs.
Nous disons qu’un gouvernement doit diminuer le capital de sa dette en restituant à son tour les capitaux. Mais dans cet.état d’une seule masse de dettes contractées à des taux divers d’intérêt, mais dénommées et réputées au seul taux de 5 pour 100, revêtues pour ainsi dire d’une même figure et d’un même uniforme, il ne peut distinguer quel est le rentier qu’il doit et qu’il peut choisir pour le rembourser de préférence, et il est exposé à faire une spoliation manifeste aux propriétaires des rentes qui, sous les régimes antérieurs, ont été, de réduction en réduction, d’amputation eu amputation, réduites à 1 pour 100 d’intérêt, quoique l’on appelle cet intérêt 5 pour 100.
Dans cette position, le gouvernement établit une caisse publique dans laquelle sera versé le produit de tels ou tels impôts ; et les directeurs de cette caisse emploieront ces fonds à acheter à la Bourse des rentes publiques au prix ou cours du jour, sans distinction de personne et sans égard aux prix de ces mêmes rentes à l’époque où celui qui les vend en est devenu le propriétaire, ni au capital réel que le gouvernement a reçu lors de chaque emprunt.
Le gouvernement pourrait éteindre et payer ces rentes à mesure que sa caisse d’amortissement en devient propriétaire ; mais l’effet de cette extinction n’atteindrait pas le but qu’on se propose, lorsqu’une dette publique s’élève, comme en France, à 200 millions ; comme en Angleterre, à 1 milliard de rente. La caisse d’amortissement reste inscrite comme créancière ; perçoit elle-même, comme les autres créanciers de l’État, les renies annuelles qu’elle a achetées ; réunit le montant de ces mêmes arrérages, à mesure qu’elle les perçoit, au montant des impôts qui ont formé sa dot primitive ; emploie le tout à racheter successivement d’autres rentes, et place ainsi, outre son capital, des intérêts à intérêts : c’est ce qu’on appelle l’intérêt composé, et dont la progression cumulative est telle, que, calculée à 5 pour 100 seulement, elle double le capital en quatorze ans deux mois. Considérons, sous ce rapport de l’intérêt des créanciers publics, le système de remboursement par le moyen de rachats au cours de la bourse. Cette opération se présente sous un aspect défavorable au premier abord : en effet, il serait réputé un homme d’improbité, un banqueroutier, le négociant qui, au lieu de payer ses dettes, les ferait racheter à son profit et à perte pour ses créanciers. Mais il est juste d’observer que le créancier de l’État ne peut pas être assimilé au créancier qui a prêté une somme déterminée à un particulier, parce qu’il n’a pas, en prêtant, imposé la condition qu’il serait remboursé, et que s’il garde sa rente et que l’État lui en paye seulement les arrérages avec exactitude, il obtient tout ce qu’il a demandé eu prêtant ; et qu’aucun remboursement ne lui a été promis. Au contraire, ce système est utile aux créanciers, en ce que cette intervention journalière d’un acquéreur riche, en concurrence avec les autres acheteurs à la bourse, élève ou soutient le prix de sa marchandise, de sa propriété.
Considérons-le sous le rapport de l’intérêt de l’État et du gouvernement.
C’est d’abord un avantage pour le gouvernement et pour l’État, dans l’intérêt de la circulation, de pouvoir diminuer chaque jour la dette sans être obligé de rassembler plus lentement des capitaux plus considérables. C’en est un autre de former au profit de la caisse d’amortissement, c’est-à-dire au profit de la nation, des capitaux productifs d’intérêts et sans cesse croissants par la jonction des intérêts successifs ; de faire jouer a l’État débiteur le jeu de créancier, et de créancier qui place successivement ses revenus et multiplie ainsi et son capital et son revenu dans un court espace de temps. En France, la dot de la caisse d’amortissement est de 40 millions, qui lui sont versés annuellement sur les produits des impôts. La première année de son activité elle a pu acheter, au cours que nous supposerons 80 pour 100, 48 millions de capitaux de rentes sur l’État, c’est-à-dire 2,500,000 fr. de rentes ; et la seconde, employer 48 millions, plus ces 2,500,000 l. de rentes qu’elle a perçues du Trésor comme les autres créanciers, et ainsi de suite. Elle possède aujourd’hui, 25 avril 1824, 87 millions de rente au delà de sa dot primitive de 40 millions, et d’une certaine quantité de bois qu’elle peut vendre ; elle est au capital de toute la dette publique comme un à quarante environ.
Je n’ai pas besoin de dire que ces avantages ne peuvent durer qu’autant que la caisse d’amortissement est sacrée, qu’autant que le ministère n’en peut employer lu dot m le revenu à un autre usage. Noli me tangere est sa devise. Base du crédit, sa base à elle-même, sa base indispensable, c’est le système représentatif et de comptabilité publique*. . Cette caisse doit rendre annuellement un compte public de ses opérations et de leurs produits. C’est une action dans le gouvernement qui n’est pas et ne doit pas être du ressort du pouvoir exécutif ; c’est un ministère neutre, c’est la magistrature protectrice de la dette publique. Cette force des choses a voulu qu’en France aujourd’hui les surveillants de la caisse d’amortissement ne fussent pas des fonctionnaires publics choisis par le pouvoir exécutif comme tous les autres, mais seulement choisis par lui sur une liste faite par les deux Chambres, et qu’ensuite ils ne fussent pas révocables et fussent comptables à la nation même. Cette espèce de ministère reçoit de ce caractère exclusif d’indépendance absolue, et en même temps de ce qu’il n’est pas salarié, une dignité qui en fait le poste à la fois le plus moral et le plus honorable.
Le système d’amortissement par la voie de rachat a-t-il d’autres avantages ? n’a-t-il pas aussi des inconvénients relativement à l’État et aux rapports d’une dette nationale avec l’économie publique? Pour résoudre ces questions, il faut examiner ses effets dans deux situations opposées, la hausse et la baisse. Le prix des rentes publiques est avili, le crédit est en baisse ; c’est-à-dire, par exemple, que les rentes sont au cours de 30 pour 100, que 5,000 fr. de rente ne produisent à celui qui les vend qu’un capital de 50,000 fr., et vice versa, que l’acquéreur achète 5,000 fr. de renie moyennant 50,000 fr. Cet état de choses est funeste, d’abord à la chose publique, en ce qu’il élève l’intérêt de l’argent à 10 pour 100 et surfait ainsi à l’agriculture et à l’industrie les avances dont elles peuvent avoir besoin ; ensuite au gouvernement lui-même, qui ne pourra, s’il y est contraint par la nécessité, emprunter qu’à ce taux ruineux. Dans cette circonstance le rachat de rente que fait chaque jour à la bourse la caisse d’amortissement est plus ou moins utile, suivant la quantité des fonds qu’elle emploie, mais l’est cependant toujours en ce que, s’il ne relève pas le prix de la rente, il l’empêche de descendre et de s’avilir davantage ; il le retient dans sa chute, et en arrête la rapidité. Alors la caisse fait de grands profits, place ses fonds à un intérêt excessif au profit de l’État débiteur. Voilà la circonstance où, arithmétiquement, le rachat des rentes au cours est avantageux ; mais c’est comme l’émétique, et quand l’État est malade.
C’est le contraire, c’est l’ivresse si la santé revient, si la prospérité renaît et croît, si la rente est à la hausse ; si, comme aujourd’hui, 5,000 fr. de rentes, que l’État à créées et vendues 64,000 fr. se vendent au cours et au prix de 100,000 fr. Alors la caisse d’amortissement, pour racheter 5,000 fr. de rente, débourse, des deniers de l’État, 36,000 fr. par chaque 100,000 fr. de plus qu’il n’a reçu, et avec une même quantité en capital rachète une moindre quantité de dettes.
Ainsi une caisse d’amortissement, opérant par la voie du rachat, empêche une crise de s’aggraver, et fait même tourner en partie au profit de l’État son propre discrédit. Ces mêmes opérations, dans une situation prospère, alimentent bien cet état de prospérité eu soutenant ou élevant encore le taux du crédit de l’État ; mais elles lui fout payer ce service en lui faisant rendre plus de capitaux qu’il n’en a reçus. Dans le premier cas, elle le défend contre la ruine ; dans le second, elle retarde sa libération et la rend presque douloureuse. Le système d’amortissement qui semble le plus naturel, celui de rembourser seulement et précisément le capital, n’a pas cet inconvénient ; mais il est impraticable là où, comme en France, les dettes contractées à des prix divers sont confondues indistinctement et où le capital réellement reçu n’a pas été avoué lors des emprunts. « (Dufresne Saint-Léon, Études du crédit public, page 60 et suiv.)
↑ Ces observations ne peuvent s’appliquer au fonds d’amortissement créé par M. Pitt en 1786 et 1792, qui a été religieusement respecté et fidèlement suivi, au milieu des besoins si impérieux et si multipliés de la guerre la plus dispendieuse que l’Angleterre ait jamais eu à soutenir. M. C.
↑ Voyez l’Histoire du revenu public, par James Postlethwaite.
(Note de l’auteur.)
↑ Ces mesures consistèrent à offrir aux créanciers leur remboursement, s’ils n’aimaient mieux consentir à la réduction de l’intérêt.
↑ Ces annuités ont été créées pour quatre-vingt-dix-huit et quatre-vingt-dix-neuf ans ; elles doivent cesser en 1860.
↑ Voyez plus haut la note de la page 191.
↑ La guerre d’Amérique. Elle a été beaucoup plus coûteuse encore qu’aucune des guerres précédentes ; elle nous a valu une dette de plus de cent millions. Pendant une paix de onze ans, on a à peine payé dix millions, et pendant une guerre de sept ans, on a contracté plus de cent millions de dettes nouvelles. (Note de l’auteur.)
↑ Pinto, Traité de la Circulation et du Crédit.
↑ Il l’est encore davantage maintenant ; le peuple paye maintenant 60 à 70 millions de taxes, dont la moitié est affectée à payer l’intérêt de la dette existante. Buchanan.
↑ Adam Smith n’a pas donné une réfutation suffisante de cette erreur. En effet, les payements des intérêts de la dette publique ne sont autre chose, ainsi que les apologistes du système des dettes publiques le prétendent, qu’une dette de la main droite à la main gauche ; ce sont autant de richesses transportées d’une classe de la société à une autre. Il est clair cependant que la question de savoir quelle sera l’influence de la dette publique sur la prospérité nationale, dépendra moins du payement de l’intérêt que de la manière dont le principal a été employé. Le principal n’a pas été prêté par une classe de la société à une autre, mais il a été donné au gouvernement, qui l’a dépense en entreprises militaires. Il a été, de fait, et pour parler d’une manière générale, annulé ; et le revenu des possesseurs de rentes ne vient point de ce capital, mais des taxes imposées sur les capitaux et les revenus des autres.

Pour mettre plus en lumière les effets immédiats des emprunts sur les richesses nationales, supposons qu’un pays ayant deux millions d’habitants et 400 millions de capital soit engagé dans une guerre, et que son gouvernement emprunte et dépense 30 millions du capital national. Si le taux des profits était de 10 pour 100, le revenu annuel des capitalistes de cet état avant la guerre serait de 40 millions ; mais après la guerre, et en défalquant les 50 millions empruntés et dépensés, il ne sera que de 33 millions, et les moyens d’employer un travail productif seront par conséquent diminués dans la même proportion. Et, bien qu’il soit vrai que le pays n’est point privé de l’intérêt de la dette, puisqu’il est seulement transporté d’une classe à une autre, il n’en est pas moins évident qu’il reste privé du revenu provenant des 50 millions de capital dépensés, et que le travail productif, qui a servi autrefois à l’entretien de la huitième partie de la population étant perdu pour l’État, il en résultera que cette portion de la population sera, pour un certain temps du moins, entièrement à la charge de ceux qui peut-être étaient déjà embarrassés de se soutenir eux-mêmes.

Cette doctrine est habilement développée par le juge Blackstone. « Par le moyen de notre dette nationale, dit ce grand jurisconsulte, la propriété dans le royaume s’est augmentée relativement à ce qu’elle était auparavant, mais cette augmentation n’est qu’une fiction, car en réalité elle ne s’est pas accrue du tout. Nous nous vantons de nos fonds considérables : mais cet argent, où est-il ? Il n’existe que de nom, en papier, par la foi publique, et par la garantie du Parlement ; et ces circonstances suffisent assurément pour donner de la confiance aux créanciers de l’État. Mais quel est le gage que l’État donne comme sécurité ? Le sol, le commerce et l’industrie des particuliers sont les sources dans lesquelles on puise l’argent pour pourvoir aux différents impôts. C’est en elles, et en elles seules, que consiste le gage des créanciers de l’État.

« Le sol, le commerce et l’industrie des particuliers sont donc diminués, dans leur valeur véritable, de la partie qui sert de gage aux créanciers. Si le revenu de A… s’élève à 100 liv. st. par an, et que les dettes qu’il a contractées envers B… l’obligent à payer à ce dernier 50 liv. par an comme intérêts de ces dettes, il est évident que la moitié de la valeur de la propriété d’A… est transportée à B… le créancier. La propriété du créancier consiste dans ses droits de créance envers le débiteur et nulle part ailleurs, et le débiteur est seulement le dépositaire de la moitié de son revenu par rapport à son créancier. Bref, la propriété du créancier de l’État consiste dans une portion des revenus publics ; il sera, par conséquent, plus riche de tout le montant de cette portion des revenus publics que la nation qui les paye. » (Commentaires, vol. 1.)

Nous n’entendons pas, par ce que nous venons de dire, contester l’utilité des emprunts. Ce point mérite beaucoup d’autres considérations. L’indépendance et l’honneur national doivent être maintenus à tout prix. Quand les revenus ordinaires d’un État ne suffisent pas pour faire face aux dépenses extraordinaires, et qu’on juge plus convenable d’emprunter que d’imposer de nouvelles taxes, il n’y a certes aucune objection à faire. Peut-être serait-ce aller trop loin que de prétendre que depuis la révolution, toutes les guerres dans lesquelles nous étions engagés étaient justes et nécessaires, et que les sommes qui ont servi à les soutenir ont été prélevées de la manière la moins onéreuse. Si cela était, l’augmentation de la dette publique serait complètement justifiée. L’intégrité et l’accroissement de l’empire, la protection de nos droits et de nos libertés, nos triomphes sur terre et sur mer sont des compensations réelles de notre dette, des trésors et du sang que nous avons versés dans ces entreprises. Ce sont des compensations suffisantes, et elles contribuent à notre prospérité comme nation, comme si elles étaient la suite de l’augmentation de notre population et de nos richesses. Il n’y a pas de sacrifice assez grand qu’on ne soit obligé de faire quand il s’agit de la sécurité et de l’indépendance nationale ; et un emprunt, quand il a servi à de pareils projets, est aussi bien employé que s’il avait été appliqué à féconder l’agriculture, l’industrie et le commerce. Il ne faut pas perdre de vue quels sont les effets indirects des emprunts et des taxes prélevées pour en payer les intérêts. Quand ces taxes ne sont pas trop élevées, elles exercent une influence très-salutaire sur l’industrie, et ont souvent pour effet, en stimulant l’activité et l’économie, de remplacer, et quelquefois même en les augmentant, les sommes prêtées au gouvernement. Mac Culloch.
↑ En effet, cet état n’est pas normal ; il a imposé a l’industrie une charge qui l’écrase, et sans le fonds d’amortissement, dont le produit s’élève à 16 millions par an, le crédit public en aurait été mis en danger. Buchanan.
↑ Le système de taxation paraît avoir atteint sa dernière limite, et les exigences du gouvernement sont souvent dures. Chez nous, c’est la taxe même, et non le mode de recouvrement qui constitue l’oppression. En Espagne, dans la plupart des cas, c’était justement le contraire qui avait lieu.
Buchanan.
↑ Depuis l’époque à laquelle l’auteur a écrit, la dette publique de l’Angleterre a augmenté de 400 millions sterling, c’est-à-dire qu’elle a fait plus que quadrupler. Garnier.

— Elle a supporté un fardeau six fois plus considérable, toutefois non sans effort, à la fin de la guerre avec l’Amérique. Buchanan.
↑ Le capital de la dette fondée et non fondée a été évalué, pour 1836, à 19 milliards 739,437,000 francs.
↑ Voyez le Glossaire de Ducange, au mot Moneta, édition des Bénédictins. (Note de l’auteur.)

↑ Le plan proposé par Adam Smith consiste à faire payer aux autres pays la dette de la Grande-Bretagne ; il faut remarquer à ce propos que l’Amérique s’est révoltée contre sa métropole plutôt que do consentir à prendre sa part des impôts, de manière que le projet nous parait complètement chimérique sous tous les rapports. Imposer à l’Irlande et à l’Amérique les impôts anglais eût été une inauguration peu convenable d’une union législative qui devait faire participer aux bénéfices plutôt qu’aux charges. Adam Smith pense que le commerce irlandais et américain, ne devant plus être soumis à des restrictions odieuses, cela aurait donné une compensation suffisante. Mais si ces restrictions étaient odieuses et oppressives , il fallait les supprimer sans autre exigence ; et en effet, pour l’Irlande, elles n’existaient plus. L’Irlande refusa de rester plus longtemps exclue du commerce du monde, au profit des monopoleurs anglais, et les restrictions imposées à son commerce cessèrent d’exister, sans qu’on exigent de nouvelles conditions. À l’époque de l’union de la Grande-Bretagne et de l’Irlande, en 1799, chacun de ces deux pays garda son système de taxation. Avec le système d’impôts de l’Angleterre, l’union aurait été une perte plutôt qu’un avantage pour l’Irlande. Buchanan.
↑ Ces abonnements ou compositions, qui se nomment, dans les lois anglaises, modus decimandi, sont des coutumes locales de certaines paroisses, où la dime ne se perçoit pas selon la coutume générale, mais d’après un mode qui, la plus souvent, est une somme d’argent de tant par acre.
↑ C’est ce qu’on nomme, dans les lois anglaises, impropriation.
↑ Ce revenu, avant l’union, excédait 2 millions et demi sterl. ; mais l’intérêt de la dette en emportait plus de la moitié.
↑ Spruce-beer, sorte de bière colorée avec l’écorce de sapin.
↑ Tome I, page 553 et suiv.
↑ Voyez l’Histoire de Massachussets, par Hutchinson, col. II, page 436 et suiv. (Note de l’auteur).
↑ Et les catholiques ! et les autres dissidents !
↑ Que dirait Adam Smith s’il assistait au spectacle des Meetings présidés aujourd’hui par O’Connell ? A. B.

%%%%%%%%%%%%%%%%%%%%%%%%%%%%%%%%%%%%%%%%%%%%%%%%%%%%%%%%%%%%%%%%%%%%%%%%%%%%%%%%

\end{document}

%%%%%%%%%%%%%%%%%%%%%%%%%%%%%%%%%%%%%%%%%%%%%%%%%%%%%%%%%%%%%%%%%%%%%%%%%%%%%%%%